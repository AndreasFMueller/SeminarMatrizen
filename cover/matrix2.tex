%
% matrix.tex -- Hintergrund für Bucheinband
%
% (c) 2021 Prof Dr Andreas Müller, OST Ostschweizer Fachhochschule
%
\documentclass[tikz]{standalone}
\usepackage{amsmath}
\usepackage{times}
\usepackage{txfonts}
\usepackage{pgfplots}
\usepackage{csvsimple}
\usetikzlibrary{arrows,intersections,math}
\begin{document}
\def\skala{1}
\begin{tikzpicture}[>=latex,thick,scale=\skala]

\def\w{30}
\def\h{10}
\def\l{7}
\def\s{0.22}
\def\vs{0.27}

\fill[color=blue] (0,{-0.3*\h}) rectangle (\w,{2*\h});

% #1 = x coordinate
% #2 = scale factor
\def\verticalline#1#2{
	\pgfmathparse{int(random(0,\h/\vs))*\vs}
	\xdef\initialheight{\pgfmathresult}
	\foreach \y in {0,\vs,...,\l}{
		\pgfmathparse{100*(1-(sqrt(\y/\l)))}
		\xdef\farbe{\pgfmathresult}
		\pgfmathparse{int(random(0,9))}
		\xdef\zeichen{\pgfmathresult}
		\node[color=white!\farbe!blue,opacity={(1-\y/\l)},scale=#2]
		%\node[color=white!\farbe!blue]
			at (#1,{\initialheight+\y*(#2)}) {\tt\zeichen};
	}
}

% #1 = width
% #2 = scalefactor
\def\verticallines#1#2{
	\pgfmathparse{int(#1/(#2*\s))}
	\xdef\xlimit{\pgfmathresult}
	\foreach \x in {0,1,...,\xlimit}{
		\pgfmathparse{\x*\s*(#2)}
		\xdef\X{\pgfmathresult}
		\verticalline{\X}{#2}
	}
}

\begin{scope}
\clip (0,{-0.3*\h}) rectangle (\w,{2*\h});
\def\W{10}

\verticallines{\W}{0.5}
\verticallines{\W}{0.7}
\verticallines{\W}{1.0}
\verticallines{\W}{1.4}
\verticallines{\W}{2}

\end{scope}

\end{tikzpicture}
\end{document}

