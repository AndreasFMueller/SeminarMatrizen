%
% matrix.tex -- Hintergrund für Bucheinband
%
% (c) 2021 Prof Dr Andreas Müller, OST Ostschweizer Fachhochschule
%
\documentclass[tikz]{standalone}
\usepackage{amsmath}
\usepackage{times}
\usepackage{txfonts}
\usepackage{pgfplots}
\usepackage{csvsimple}
\usetikzlibrary{arrows,intersections,math}
\begin{document}
\def\skala{1}
\begin{tikzpicture}[>=latex,thick,scale=\skala]

\def\w{30}
\def\h{10}
\def\l{7}
\def\s{0.155}
\def\vs{0.245}

\fill[color=blue] (0,{-0.3*\h}) rectangle (\w,{2*\h});

\def\verticalline#1{
	\pgfmathparse{int(random(0,\h/\vs))*\vs}
	\xdef\initialheight{\pgfmathresult}
	\foreach \y in {0,\vs,...,\l}{
		\pgfmathparse{100*(1-(sqrt(\y/\l)))}
		\xdef\farbe{\pgfmathresult}
		\pgfmathparse{int(random(0,9))}
		\xdef\zeichen{\pgfmathresult}
		\node[color=white!\farbe!blue,opacity={(1-\y/\l)}]
		%\node[color=white!\farbe!blue]
			at (#1,{\initialheight+\y}) {\tt\zeichen};
	}
}

\begin{scope}
\clip (0,{-0.3*\h}) rectangle (\w,{2*\h});

\foreach \x in {0,\s,...,\w}{
	\verticalline{\x}
}

\end{scope}

\end{tikzpicture}
\end{document}

