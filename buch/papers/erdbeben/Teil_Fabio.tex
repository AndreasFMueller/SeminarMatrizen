\section{Kalman Filter}
\subsection{Was ist ein Erdbeben?}
Für das Verständnis möchten wir zuerst klären, was ein Erdbeben genau ist.
Das soll uns helfen, eine Verknüpfung zwischen dem Naturphänomen und der mathematischen Lösungsfindung herzustellen.

Unter einem Erdbeben verstehen wir eine Erschütterung des Erdkörpers.
Dabei reiben zwei tektonische Platten aneinander, welche aber sich durch die Gesteinsverzahnung gegenseitig blockieren.
Aufgrund dieser Haftreibung entstehen Spannungen, die sich immer mehr bis zum Tipping Point aufbauen.
Irgendwann ist der Punkt erreicht, in dem die Scherfestigkeit der Gesteine überwunden wird.
Wenn dies passiert, entladet sich die aufgebaute Spannung und setzt enorme Energien frei, die wir als Erdbeben wahrnehmen.

Ein Erdbeben pflanzt sich vom Erdbebenherd in allen Richtungen gleich aus.
Vergleichbar ist, wenn man einen Stein in einen Teich wirft und die Wellen beobachten kann, die sich ausbreiten.

Wir möchten nun mittels Kalman-Filter die Erdbebenbeschleunigung herausfinden.
Die Erdbebenbeschleunigung ist in der Praxis zur Entwicklung von Erdbebengefährdungskarten, sowie der Ausarbeitung von Baunormen für erdbebengerechte Bauweise von Bedeutung.


\subsection{Künstliche Erdbebendaten}
Nun möchten wir anhand eines eigenen Beispiels das Kalman-Filter anwenden.
Wir müssen Erdbebendaten künstlich erzeugen, um sie in das Filter zu geben und somit den Prozess zu starten.
Dafür nehmen wir die Formel für harmonische gedämpfte Schwingungen, die

\begin{equation}
	y = A \sin(\omega t e^{-lambda t})
\end{equation} 

lautet.




A ist die Amplitude der Schwingung und beschreibt die Heftigkeit eines Erdbebens, die Magnitude.
Omega repräsentiert die Erdbebenfrequenz, die in der Realität zwischen 1 Hz und 30 Hz betragen kann.
Wir wählen als Erwartungswert 15 Herz und für die Standardabweichung 1 Hz.
Lambda ist die Bodendämpfung, für die wir 0.2 wählen.
Wir haben diese Zahl aus der Literatur entnommen und ist für das Bauwesen bedeutend.
Je grösser Lambda gewählt wird, desto stärker wirkt die Dämpfung der Massenschwingung.
Die Funktion ist zeitabhängig und wir lassen pro Sekunde zehn Messwerte generieren.

Die Frequenz soll im Matlab als Zufallszahl generiert werden.
Mit dem Golay-Filter glätten wir unsere Werte, um unser Output näher an die Realität zu bringen. 
Zusätzlich werden Ausreisser nicht vernachlässigt und wirken geglättet in unsere Datenmenge.

Grafik einfügen

In der Grafik erkennen wir in den Sekunden 0 bis 10, dass die Sinuskurve gezackt ist.
Das deutet darauf hin, dass die Frequenz des Erdbebens einen hohen Einfluss auf die Masse des Seismographen hat.
Ab der 10. Sekunde bis zu tend, pendelt sich die Masse in ihre Eigenfrequenz ein und verhält sich unabhängiger vom Erdbeben.

\subsection{Versuch}
Um den Kalman-Filter auszuprobieren, setzen wir nun Werte ein.
Für die Systemparameter wählen wir m=1.0, D = 0.3 und k = 0.1 und fügen es in die Differentialgleichung

\begin{equation}
	m\ddot x + 2k \dot x + Dx = f	
\end{equation} 

ein und erhalten

\begin{equation}
	1\ddot x + 0.1 \dot x + 0.3x = f	
\end{equation} 


\subsubsection*{Prozessrauschkovarianzmatrix $Q$}





\begin{equation}
	Q = \left(
	\begin{array}{ccc} 	
		(5 \cdot 10^{-5})^2 & 0 & 0 \\ 
		0 & (1 \cdot 10^{-5})^2 & 0\\ 
		0 & 0& ( 1 )^2\\
	\end{array}\right)  
\end{equation} 





\subsection{Resultate}

Vergleichen wir die künstlichen Messdaten mit der geschätzten Schwingung des Kalman-Filters, stellen wir fest, dass wir eine gute Methode gefunden haben, die Erdbebenbeschleunigung zu schätzen.
Obwohl die künstlichen Daten mit einer random-Funktion erzeugt werden, kann das Kalman-Filter präzise Vorhersagungen bilden.

Für die Differentialgleichung zweiter Ordnung brauchen wir im Matlab die Funktion ode45.
Mit dieser Funktion können wir Differentialgleichungen auflösen.













In Matlab fügen wir die Formel und unsere definierten Werte ein.
Die Frequenz generieren wir mit einem Zufallscode, 
Mit einem Zufallscode und einen Zeitraum 

Matlabcode einfügen

