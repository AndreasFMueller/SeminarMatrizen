%
% teil2.tex -- Beispiel-File für teil2 
%
% (c) 2020 Prof Dr Andreas Müller, Hochschule Rapperswil
%
\section{Funktionsweise 
\label{mceliece:section:funktionsweise}}
\rhead{Funktionsweise}
Um den Ablauf des Datenaustausches mittels McEliece-Verschlüsselung zu erläutern,
wird ein Szenario verwendet,
bei dem Bob an Alice eine verschlüsselte Nachricht über ein öffentliches Netzwerk zukommen lässt.

\subsection{Vorbereitung
\label{mceliece:section:vorbereitung}}
Bevor einen Datenaustausch zwischen Sender und Empfänger stattfinden kann,
muss abgemacht werden, welche Länge $n$ das Code-Wort und welche Länge $k$ das Datenwort hat
und wie viele Bitfehler $t$ (angewendet mit Fehlervektor $e_n$)
für das Rauschen des Code-Wortes $c_n$ verwendet werden.
Danach generiert Alice (Empfängerin) ein Schlüsselpaar.
Dazu erstellt sie die einzelnen Matrizen $S_k$, $G_{n,k}$ und $P_n$.
Diese drei Matrizen bilden den privaten Schlüssel von Alice
und sollen geheim bleiben.
Der öffentliche Schlüssel $K_{n,k}$ hingegen berechnet sich
aus der Multiplikation der privaten Matrizen (Abschnitt \ref{mceliece:subsection:k_nk})
und wird anschliessend Bob zugestellt.

\subsection{Verschlüsselung
\label{mceliece:section:verschl}}
Bob berechnet nun die verschlüsselte Nachricht $c_n$, indem er seine Daten $d_k$
mit dem öffentlichen Schlüssel $K_{n,k}$ von Alice multipliziert
und anschliessend durch eine Addition mit einem Fehlervektor $e_n$ einige Bitfehler hinzufügt:
\[
    c_n=K_{n,k}\cdot d_k + e_n.
\]
Dabei wird für jede Nachricht (oder für jedes Nachrichtenfragment) $d_k$
ein neuer, zufälliger Fehlervektor generiert.
Die verschlüsselte Nachricht $c_n$ wird anschliessend Alice zugestellt.

\subsection{Entschlüsselung
\label{mceliece:section:entschl}}
Alice entschlüsselt die erhaltene Nachricht in mehreren einzelnen Schritten.
Um etwas Transparenz in diese Prozedur zu bringen, wird der öffentliche Schlüssel $K_{n,k}$ mit seinen Ursprungsmatrizen dargestellt:
\begin{align*}
    c_n&=K_{n,k}\cdot d_k + e_n \\
    &= P_{n}\cdot G_{n,k}\cdot S_{k}\cdot d_k + e_n.
\end{align*}
Zuerst wird der Effekt der Permutationsmatrix rückgängig gemacht,
indem das Codewort mit der Inversen $P_n^{-1}$ multipliziert wird:
\begin{align*}
    c_{n}''=P_n^{-1}\cdot c_n&= P_n^{-1}\cdot P_{n}\cdot G_{n,k}\cdot S_{k}\cdot d_k + P_n^{-1}\cdot e_n \\
                                         &= G_{n,k}\cdot S_{k}\cdot d_k + P_n^{-1}\cdot e_n.
\end{align*}
Eine weitere Vereinfachung ist nun möglich,
weil $P_n^{-1}$ einerseits auch eine gewöhnliche Permutationsmatrix ist
und andererseits ein zufälliger Fehlervektor $e_n$ multipliziert mit einer Permutationsmatrix
wiederum einen zufälligen Fehlervektor gleicher Länge und mit der gleichen Anzahl Fehlern $e_n'$ ergibt:
\begin{align*}
    c_{n}''&=G_{n,k}\cdot S_{k}\cdot d_k + P_n^{-1}\cdot e_n \\
             &=G_{n,k}\cdot S_{k}\cdot d_k + e'_n \quad \text{mit} \quad
    e'_n=P_n^{-1}\cdot e_n.
\end{align*}
Dank des fehlerkorrigierenden Codes, der durch die implizite Multiplikation mittels $G_{n,k}$ auf die Daten angewendet wurde,
können nun die Bitfehler, verursacht durch den Fehlervektor $e'_n$,
entfernt werden.
Da es sich bei diesem Schritt nicht um eine einfache Matrixmultiplikation handelt,
wird die Operation durch eine Funktion dargestellt.
Wie dieser Decoder genau aufgebaut ist,
hängt vom verwendeten Linearcode ab:
\begin{align*}
    c_{k}'&=\text{Linear-Code-Decoder}(c''_n)\\
            &=\text{Linear-Code-Decoder}(G_{n,k}\cdot S_{k}\cdot d_k + e'_n)\\
            &=S_{k}\cdot d_k.
\end{align*}
Zum Schluss wird das inzwischen fast entschlüsselte Codewort $c'_k$ mit der Inversen der zufälligen Binärmatrix $S^{-1}$ multipliziert,
womit der Inhalt der ursprünglichen Nachricht nun wiederhergestellt wurde:
\begin{equation*}
    d'_{k}=S_{k}^{-1} \cdot c'_k=S_{k}^{-1} \cdot S_{k}\cdot d_k
                                    =d_k.
\end{equation*}
Möchte ein Angreifer die verschlüsselte Nachricht knacken, muss er die drei privaten Matrizen $S_k$, $G_{n,k}$ und $P_n$ kennen.
Aus dem öffentlichen Schlüssel lassen sich diese nicht rekonstruieren
und eine systematische Analyse der Codeworte wird durch das Hinzufügen von zufälligen Bitfehlern zusätzlich erschwert.

\subsection{Beispiel}
Die Verschlüsselung soll mittels eines numerischen Beispiels demonstriert werden. 
Der verwendete Linear-Code wird im Abschnitt \ref{mceliece:subsection:seven_four} beschrieben.
\begin{itemize}
    \item Daten- und Fehlervektor
%    \begin{itemize}
%        \item[]
        \[d_4=
        \begin{pmatrix}
            1\\
            1\\
            1\\
            0 
        \end{pmatrix}
        ,\quad
        e_7=
        \begin{pmatrix}
            0\\
            0\\
            1\\
            0\\
            0\\
            0\\
            0
        \end{pmatrix}.
        \]
%    \end{itemize}
    \item Private Matrizen:
\begin{gather*}
%    \begin{itemize}
%        \item[]
        S_4=
            \begin{pmatrix}
                0 & 0 & 1 & 1\\
                0 & 0 & 0 & 1\\
                0 & 1 & 0 & 1\\
                1 & 0 & 0 & 1
            \end{pmatrix},\quad
            S_4^{-1}=
            \begin{pmatrix}
                0 & 1 & 0 & 1\\
                0 & 1 & 1 & 0\\
                1 & 1 & 0 & 0\\
                0 & 1 & 0 & 0\\
            \end{pmatrix},
	\\
%        \item[]
            G_{7,4}=
            \begin{pmatrix}
                1 & 0 & 0 & 0\\
                1 & 1 & 0 & 0\\
                0 & 1 & 1 & 0\\
                1 & 0 & 1 & 1\\
                0 & 1 & 0 & 1\\
                0 & 0 & 1 & 0\\
                0 & 0 & 0 & 1
            \end{pmatrix},
	\\
%        \item[]
            P_7=
            \begin{pmatrix}
                0 & 1 & 0 & 0 & 0 & 0 & 0\\
                0 & 0 & 0 & 0 & 0 & 0 & 1\\
                1 & 0 & 0 & 0 & 0 & 0 & 0\\
                0 & 0 & 0 & 1 & 0 & 0 & 0\\
                0 & 0 & 0 & 0 & 0 & 1 & 0\\
                0 & 0 & 1 & 0 & 0 & 0 & 0\\
                0 & 0 & 0 & 0 & 1 & 0 & 0
            \end{pmatrix},
        \quad
            P_7^{-1}=P_7^t=
            \begin{pmatrix}
                0 & 0 & 0 & 0 & 0 & 1 & 0\\
                1 & 0 & 0 & 0 & 0 & 0 & 0\\
                0 & 0 & 0 & 1 & 0 & 0 & 0\\
                0 & 0 & 0 & 0 & 1 & 0 & 0\\
                0 & 0 & 0 & 0 & 0 & 0 & 1\\
                0 & 0 & 1 & 0 & 0 & 0 & 0\\
                0 & 1 & 0 & 0 & 0 & 0 & 0
            \end{pmatrix}.
        \\
%    \end{itemize}
\end{gather*}
    \item Öffentlicher Schlüssel:
\index{Schlüssel, öffentlicher}%
\index{offentlicher Schlüssel@öffentlicher Schlüssel}%
%    \begin{itemize}
%        \item[]
        \begin{align*}
            K_{7,4}&=P_{7}\cdot G_{7,4}\cdot S_{4}=\\
            \begin{pmatrix} %k
                0 & 0 & 1 & 0\\
                1 & 0 & 0 & 1\\
                0 & 0 & 1 & 1\\
                1 & 1 & 1 & 1\\
                0 & 1 & 0 & 1\\
                0 & 1 & 0 & 0\\
                1 & 0 & 0 & 0
            \end{pmatrix}
            &=
            \begin{pmatrix} %p
                0 & 1 & 0 & 0 & 0 & 0 & 0\\
                0 & 0 & 0 & 0 & 0 & 0 & 1\\
                1 & 0 & 0 & 0 & 0 & 0 & 0\\
                0 & 0 & 0 & 1 & 0 & 0 & 0\\
                0 & 0 & 0 & 0 & 0 & 1 & 0\\
                0 & 0 & 1 & 0 & 0 & 0 & 0\\
                0 & 0 & 0 & 0 & 1 & 0 & 0
            \end{pmatrix}
            \cdot
            \begin{pmatrix} %g
                1 & 0 & 0 & 0\\
                1 & 1 & 0 & 0\\
                0 & 1 & 1 & 0\\
                1 & 0 & 1 & 1\\
                0 & 1 & 0 & 1\\
                0 & 0 & 1 & 0\\
                0 & 0 & 0 & 1
            \end{pmatrix}
            \cdot
            \begin{pmatrix} %s
                0 & 0 & 1 & 1\\
                0 & 0 & 0 & 1\\
                0 & 1 & 0 & 1\\
                1 & 0 & 0 & 1
            \end{pmatrix}
            .
        \end{align*}
%    \end{itemize}
    \item Verschlüsselung:
%    \begin{itemize}
%        \item[]
        \begin{align*}
            c_7&=K_{7,4}\cdot d_4 + e_7=\\
        \begin{pmatrix} %c
            1\\
            1\\
            0\\
            1\\
            1\\
            1\\
            1
        \end{pmatrix}
        &=
        \begin{pmatrix} %k
            0 & 0 & 1 & 0\\
            1 & 0 & 0 & 1\\
            0 & 0 & 1 & 1\\
            1 & 1 & 1 & 1\\
            0 & 1 & 0 & 1\\
            0 & 1 & 0 & 0\\
            1 & 0 & 0 & 0
        \end{pmatrix}
        \cdot
        \begin{pmatrix} %d
            1\\
            1\\
            1\\
            0 
        \end{pmatrix}
        +
        \begin{pmatrix} %e
            0\\
            0\\
            1\\
            0\\
            0\\
            0\\
            0
        \end{pmatrix} 
        .
        \end{align*}
%    \end{itemize}
    \item Entschlüsselung (Permutation rückgängig machen):
%    \begin{itemize}
%        \item[]
        \begin{align*}
            c_{7}''&=P_7^{-1}\cdot c_7=\\
            \begin{pmatrix} %c''
                0\\
                1\\
                1\\
                1\\
                1\\
                1\\
                1
            \end{pmatrix}
            &=
            \begin{pmatrix} %p^-1
                0 & 0 & 1 & 0 & 0 & 0 & 0\\
                1 & 0 & 0 & 0 & 0 & 0 & 0\\
                0 & 0 & 0 & 0 & 0 & 1 & 0\\
                0 & 0 & 0 & 1 & 0 & 0 & 0\\
                0 & 0 & 0 & 0 & 0 & 0 & 1\\
                0 & 0 & 0 & 0 & 1 & 0 & 0\\
                0 & 1 & 0 & 0 & 0 & 0 & 0
            \end{pmatrix}
            \cdot
            \begin{pmatrix} %c
                1\\
                1\\
                0\\
                1\\
                1\\
                1\\
                1
            \end{pmatrix}
            .
        \end{align*}
%    \end{itemize}
    \item Entschlüsselung (Bitfehlerkorrektur mit Linearcode):
%    \begin{itemize}
%        \item[]
        \begin{align*}
            c_{7}'&=\text{Linear-Code-Decoder($c''_7$)}=\\
            \begin{pmatrix} %c'
                1\\
                0\\
                1\\
                1
            \end{pmatrix}
            &=\text{Linear-Code-Decoder(}
            \begin{pmatrix}
                0\\
                1\\
                1\\
                1\\
                1\\
                1\\
                1
            \end{pmatrix}
            \text{)}
            .
        \end{align*}
%    \end{itemize}
    \item Entschlüsselung (Umkehrung des $S_4$-Matrix-Effekts):
%    \begin{itemize}
%        \item[]
        \begin{align*}
            d'_{4}&=S_{4}^{-1} \cdot c'_4 \,(= d_4)\\
            \begin{pmatrix}
                1\\
                1\\
                1\\
                0
            \end{pmatrix}
            &=
            \begin{pmatrix} %s^-1
                0 & 1 & 0 & 1\\
                0 & 1 & 1 & 0\\
                1 & 1 & 0 & 0\\
                0 & 1 & 0 & 0\\
            \end{pmatrix}
            \cdot
            \begin{pmatrix} %c'
                1\\
                0\\
                1\\
                1
            \end{pmatrix}
            .
        \end{align*}
%    \end{itemize}
\end{itemize}

\subsection{7/4-Code
\label{mceliece:subsection:seven_four}}
Beim 7/4-Code handelt es sich um einen linearen Code,
der einen Bitfehler korrigieren kann.
\index{7/4-Code}%
\index{linearer Code}%
\index{Code, linear}%
Es gibt unterschiedliche Varianten zum Erzeugen eines 7/4-Codes,
wobei der hier verwendete Code mithilfe des irreduziblen Generatorpolynoms $P_g = x^3 +x + 1$ generiert wird.
\index{Generatorpolynom}%
Somit lässt sich das Codepolynom $P_c$ berechnen, indem das Datenpolynom $P_d$ mit dem Generatorpolynom $P_g$ multipliziert wird (Codiervorgang):
\[
    P_c=P_g \cdot P_d.
\]
Damit diese Multiplikation mit Matrizen ausgeführt werden kann, werden die Polynome als Vektoren dargestellt (Kapitel \ref{buch:section:polynome:vektoren}):
\[
    P_g = \textcolor{red}{1}\cdot x^0 + \textcolor{blue}{1}\cdot x^1 + \textcolor{darkgreen}{0}\cdot x^2 + \textcolor{orange}{1}\cdot x^3 \implies
    [\textcolor{red}{1}, \textcolor{blue}{1} ,\textcolor{darkgreen}{0}, \textcolor{orange}{1}] = g_4.
\]
Auch das Datenpolynom wird mit einem Vektor dargestellt: $P_d = d_0 \cdot x^0 + d_1 \cdot x^1 + d_2 \cdot x^2 + d_3 \cdot x^3 \implies [d_0, d_1, d_2, d_3] = d_4$.
Der Vektor $g_4$ wird nun in die sogenannte Generatormatrix $G_{7,4}$ gepackt,
sodass die Polynommultiplikation mit $d_4$ mittels Matrixmultiplikation realisiert werden kann:
\[
    c_7=G_{7,4} \cdot d_4=
    \begin{pmatrix}
              \textcolor{red}{1} &                       0  &                       0  &                       0  \\
             \textcolor{blue}{1} &       \textcolor{red}{1} &                       0  &                       0  \\
        \textcolor{darkgreen}{0} &      \textcolor{blue}{1} &       \textcolor{red}{1} &                       0  \\
           \textcolor{orange}{1} & \textcolor{darkgreen}{0} &      \textcolor{blue}{1} &       \textcolor{red}{1} \\
                              0  &    \textcolor{orange}{1} & \textcolor{darkgreen}{0} &      \textcolor{blue}{1} \\
                              0  &                       0  &    \textcolor{orange}{1} & \textcolor{darkgreen}{0} \\
                              0  &                       0  &                       0  &    \textcolor{orange}{1} 
    \end{pmatrix}
    \begin{pmatrix}
        d_0\\        
        d_1\\
        d_2\\
        d_3
    \end{pmatrix}
    =
    \begin{pmatrix}
        c_0\\        
        c_1\\
        c_2\\
        c_3\\
        c_4\\
        c_5\\
        c_6\\
    \end{pmatrix}.
\]
Beim nun entstandenen Codevektor $c_7=[c_0, ..., c_6]$ entsprechen die Koeffizienten dem dazugehörigen Codepolynom $P_c=c_0\cdot x^0+...+c_6\cdot x^6$.
Aufgrund der Multiplikation mit dem Generatorpolynom $P_g$ lässt sich das Codewort auch wieder restlos durch $P_g$ dividieren.
Wird dem Codewort nun einen Bitfehler hinzugefügt, entsteht bei der Division durch $P_g$ einen Rest.
Beim gewählten Polynom beträgt die sogenannte Hamming-Distanz drei, das bedeutet,
\index{Hamming-Distanz}%
dass vom einen gültigen Codewort zu einem anderen gültigen Codewort drei Bitfehler auftreten müssen.
Somit ist es möglich, auf das ursprüngliche Bitmuster zu schliessen, solange maximal ein Bitfehler vorhanden ist.
Jeder der möglichen acht Bitfehler führt bei der Division zu einem anderen Rest,
womit das dazugehörige Bit identifiziert und korrigiert werden kann,
indem beispielsweise die Bitfehler mit dem dazugehörigen Rest in der sogenannten Syndromtabelle (Tabelle \ref{mceliece:tab:syndrome}) hinterlegt werden.
\index{Syndromtabelle}%
\begin{table}
    \begin{center}
        \begin{tabular}{|l|l|}
            \hline
            Syndrom (Divisionsrest)     &korrespondierender Bitfehler\\
            \hline
            1 ($[1,0,0]$)               &$[1,0,0,0,0,0,0]$\\    
            2 ($[0,1,0]$)               &$[0,1,0,0,0,0,0]$\\    
            3 ($[1,1,0]$)               &$[0,0,0,1,0,0,0]$\\    
            4 ($[0,0,1]$)               &$[0,0,1,0,0,0,0]$\\    
            5 ($[1,0,1]$)               &$[0,0,0,0,0,0,1]$\\    
            6 ($[0,1,1]$)               &$[0,0,0,0,1,0,0]$\\    
            7 ($[1,1,1]$)               &$[0,0,0,0,0,1,0]$\\
            \hline

        \end{tabular}
    \end{center}
    \caption{\label{mceliece:tab:syndrome}Syndromtabelle 7/4-Code}
\end{table}
\index{Syndrom}%
