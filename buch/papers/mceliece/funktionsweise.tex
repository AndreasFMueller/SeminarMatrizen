%
% teil2.tex -- Beispiel-File für teil2 
%
% (c) 2020 Prof Dr Andreas Müller, Hochschule Rapperswil
%
\section{Funktionsweise 
\label{mceliece:section:funktionsweise}}
\rhead{Funktionsweise}
Um den Ablauf des Datenaustausches mittels McEliece-Verschlüsselung zu erläutern,
wird ein Szenario verwendet,
bei dem Bob an Alice eine verschlüsselte Nachticht über ein öffentliches Netzwerk zukommen lässt.

\subsection{Vorbereitung
\label{mceliece:section:vorbereitung}}
Damit der Nachrichtenaustausch stattfinden kann, muss Alice (Empfängerin)
zuerst ein Schlüsselpaar definieren.
Dazu erstellt sie die einzelnen Matrizen $S_k$, $G_{n,k}$ und $P_n$.
Diese drei einzelnen Matrizen bilden den privaten Schlüssel von Alice
und sollen geheim bleiben.
Der öffentliche Schlüssel $K_{n,k}$ hingegen berechnet sich
aus der Multiplikation der privaten Matrizen\ref{mceliece:subsection:k_nk}
und wird anschliessend Bob zugestellt.

\subsection{Verschlüsselung
\label{mceliece:section:verschl}}
Bob berechnet nun die verschlüsselte Nachricht $c_n$, indem er seine Daten $d_k$
mit dem öffentlichen Schlüssel $K_{n,k}$ von Alice multipliziert
und anschliessend durch eine Addition mit einem Fehlervektor $e_n$ einige Bitfehler hinzufügt.
\[
    c_n\,=\,K_{n,k}\cdot d_k + e_n\,.
\]
Dabei wird für jede Nachricht (oder für jedes Nachrichtenfragment)
einen neuen, zufälligen Fehlervektor generiert.
Die verschlüsselte Nachricht $c_n$ wird anschliessend Alice zugestellt.

\subsection{Entschlüsselung
\label{mceliece:section:entschl}}
Alice entschlüsselt die erhaltene Nachricht in mehreren einzelnen Schritten.
Um etwas Transparenz in diese Prozedur zu bringen, wird der öffentliche Schlüssel $K_{n,k}$ mit seinen Ursprungsmatrizen dargestellt.
\begin{align*}
    c_n\,&=\,K_{n,k}\cdot d_k + e_n \\
    &= P_{n}\cdot G_{n,k}\cdot S_{k}\cdot d_k + e_n
\end{align*}
Zuerst wird der Effekt der Permutationsmatrix rückgängig gemacht,
indem das Codewort mit dessen Inversen $P_n^{-1}$ multipliziert wird.
\begin{align*}
    c_{n}''\,=\,P_n^{-1}\cdot c_n\,&= P_n^{-1}\cdot P_{n}\cdot G_{n,k}\cdot S_{k}\cdot d_k + P_n^{-1}\cdot e_n \\
                                         &= G_{n,k}\cdot S_{k}\cdot d_k + P_n^{-1}\cdot e_n \\
\end{align*}
Eine weitere Vereinfachung ist nun möglich,
weil $P_n^{-1}$ einerseits auch eine gewöhnliche Permutationsmatrix ist
und andererseits ein zufälliger Fehlervektor $e_n$ multipliziert mit einer Permutationsmatrix
wiederum einen gleichwertigen, zufälligen Fehlervektor $e_n'$ ergibt.
\begin{align*}
    c_{n}''\,&=\,G_{n,k}\cdot S_{k}\cdot d_k + P_n^{-1}\cdot e_n \\
             &=\,G_{n,k}\cdot S_{k}\cdot d_k + e'_n\quad \quad \quad | \,
    e'_n\,=\,P_n^{-1}\cdot e_n
\end{align*}
Dank des fehlerkorrigierenden Codes, der durch die implizite Multiplikation mittels $G_{n,k}$ auf die Daten angewendet wurde,
können nun die Bitfehler, verursacht durch den Fehlervektor $e'_n$,
entfernt werden.
Da es sich bei diesem Schritt nicht um eine einfache Matrixmultiplikation handelt,
wird die Operation durch eine Funktion dargestellt.
Wie dieser Decoder genau aufgebaut ist ist,
hängt vom verwendeten Linearcode ab.
\begin{align*}
    c_{k}'\,&=\text{Linear-Code-Decoder($c''_n$)}\\
            &=\text{Linear-Code-Decoder($G_{n,k}\cdot S_{k}\cdot d_k + e'_n$)}\\
            &=S_{k}\cdot d_k
\end{align*}
Zum Schluss wird das inzwischen fast entschlüsselte Codewort $c'_k$ mit der inversen der zufälligen Binärmatrix $S^{-1}$ multipliziert,
womit der Inhalt der ursprünglichen Nachricht nun wiederhergestellt wurde.
\begin{align*}
                            c_{k}'\,&=S_{k}\cdot d_k \quad | \cdot S_k^{-1}\\
    d'_{k}\,=\,S_{k}^{-1} \cdot c'_k&=S_{k}^{-1} \cdot S_{k}\cdot d_k\\
                                    &=d_k
\end{align*}

\subsection{Beispiel}

TODO:
-alle Beispielmatrizen- und Vektoren hierhin zügeln, numerisches Beispiel kreieren\\
-erläutern des 7/4-codes (ja/nein)?