%
% einleitung.tex -- Beispiel-File für die Einleitung
%
% (c) 2020 Prof Dr Andreas Müller, Hochschule Rapperswil
%
\section{Aufbau\label{mceliece:section:Aufbau}}
\rhead{Aufbau}
Das McEliece-Kryptosystem besteht aus folgenden Elementen:

\subsection{Datenvektor $d_k$
\label{mceliece:subsection:d_k}}
In diesem Vektor der Länge $k$ sind die zu verschlüsselnden Daten enthalten.
Beispielsweise
\[d_4=
\begin{pmatrix}
    1\\
    1\\
    1\\
    0 
\end{pmatrix}
\]

\subsection{Binäre Zufallsmatrix $S_k$
\label{mceliece:subsection:s_k}}
$S_k$ ist eine Binäre Zufallsmatrix der Grösse $k \times k$.
Auch muss diese Matrix in $\mathbb{F}_2$ invertierbar sein.
Für kleine Matrizen kann durchaus jedes Matrizenelement zufällig generiert werden,
wobei danach mithilfe des Gauss-Algorythmusses deren Inverse bestimmt werden kann.
Da eine solche Matrix möglicherweise singulär ist, muss in diesem Fall eine neue Zufallsmatrix erzeugt werden.
Für grössere Matrizen existieren bessere Methoden, auf welche hier nicht weiter eingegangen wird \cite{mceliece:GenerationRandMatrix}.
Beispielsweise
\[S_4=
    \begin{pmatrix}
        0 & 0 & 1 & 1\\
        0 & 0 & 0 & 1\\
        0 & 1 & 0 & 1\\
        1 & 0 & 0 & 1
    \end{pmatrix}
\]

\[
    S_4^{-1}=
    \begin{pmatrix}
        0 & 1 & 0 & 1\\
        0 & 1 & 1 & 0\\
        1 & 1 & 0 & 0\\
        0 & 1 & 0 & 0\\
    \end{pmatrix}
\]

\subsection{Linear-Code-Generatormatrix $G_{n,k}$
\label{mceliece:subsection:g_m}}
Das wichtigste Element des McEliece-Systems ist ein fehlerkorrigierender Code,
der in der Lage ist, $t$ Fehler zu korrigieren.
Im Zusammenhang mit McEliece werden dabei meist Goppa-Codes verwendet,
es können prinzipiell auch andere Codes wie beispielsweise Reed-Solomin verwendet werden,
jedoch besitzen einige Codes Schwachstellen \cite{mceliece:lorenz}.
Das Codieren mit diesem linearen Code kann mithilfe dessen Generatormatrix $G_{n,k}$ erfolgen.
Da es sich um einen fehlerkorrigierenden Code handelt,
wird das Codewort länger als das Datenwort,
es wird also Redundanz hinzugefügt,
um die Fehlerkorrektur möglich zu machen.

Beispiel
\[
    G_{7,4}=
    \begin{pmatrix}
        1 & 0 & 0 & 0\\
        1 & 1 & 0 & 0\\
        0 & 1 & 1 & 0\\
        1 & 0 & 1 & 1\\
        0 & 1 & 0 & 1\\
        0 & 0 & 1 & 0\\
        0 & 0 & 0 & 1
    \end{pmatrix}
\]

\subsection{Permutations-Matrix $P_n$
\label{mceliece:subsection:p_m}}
Mit der zufällig generierten Permutationsmatrix $P_n$ wird die Reihenfolge der Bits geändert.
Mit der Inversen $P_n^{-1}$ kann die Bitvertauschung rückgängig gemacht werden.
Beispiel
\[
    P_7=
    \begin{pmatrix}
        0 & 1 & 0 & 0 & 0 & 0 & 0\\
        0 & 0 & 0 & 0 & 0 & 0 & 1\\
        0 & 0 & 0 & 0 & 0 & 1 & 0\\
        0 & 0 & 1 & 0 & 0 & 0 & 0\\
        0 & 0 & 0 & 1 & 0 & 0 & 0\\
        1 & 0 & 0 & 0 & 0 & 0 & 0\\
        0 & 0 & 0 & 0 & 1 & 0 & 0
    \end{pmatrix}
\]
,
\[
    P_7^{-1}=P_7^t=
    \begin{pmatrix}
        0 & 0 & 0 & 0 & 0 & 1 & 0\\
        1 & 0 & 0 & 0 & 0 & 0 & 0\\
        0 & 0 & 0 & 1 & 0 & 0 & 0\\
        0 & 0 & 0 & 0 & 1 & 0 & 0\\
        0 & 0 & 0 & 0 & 0 & 0 & 1\\
        0 & 0 & 1 & 0 & 0 & 0 & 0\\
        0 & 1 & 0 & 0 & 0 & 0 & 0
    \end{pmatrix}
\]

\subsection{Fehler-Vektor $e_n$
\label{mceliece:subsection:p_m}}
Dieser Vektor der Länge $n$ besteht aus $t$ Einsen, welche zufällig innerhalb des Vektors angeordnet sind,
alle anderen Einträge sind Null.
Dieser Fehlervektor besitzt also gleich viele Einer,
wie die Anzahl Fehler, die der Linearcode zu korrigieren vermag.

Beispiel
\[
    E_7=
    \begin{pmatrix}
        0\\
        0\\
        1\\
        0\\
        0\\
        0\\
        0
    \end{pmatrix}
\]

\subsection{Public-Key $K_{n,k}$
\label{mceliece:subsection:k_m}}
Der öffentliche Schlüssel, welcher zum Verschlüsseln verwendet wird,
berechnet sich mit
\[
    K_{n,k}=P_{n}\cdot G_{n,k}\cdot S_{k}\,.
\]
Beispiel
\[
    K_{7,4}=
    \begin{pmatrix}
        0 & 0 & 1 & 0\\
        1 & 0 & 0 & 1\\
        0 & 0 & 1 & 1\\
        1 & 1 & 1 & 1\\
        0 & 1 & 0 & 1\\
        0 & 1 & 0 & 0\\
        1 & 0 & 0 & 0
    \end{pmatrix}
\]