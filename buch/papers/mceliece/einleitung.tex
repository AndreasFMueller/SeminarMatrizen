%
% teil1.tex -- Beispiel-File für das Paper
%
% (c) 2020 Prof Dr Andreas Müller, Hochschule Rapperswil
%
\section{Einleitung
\label{mceliece:section:einleitung}}
\rhead{Einleitung}
Beim McEliece-Kryptosystem handelt es sich um ein asymetrisches Verschlüsselungsverfahren, welches erlaubt,
Daten verschlüsselt über ein Netzwerk zu übermitteln, ohne dass vorab ein gemeinsamer,
geheimer Schlüssel unter den Teilnehmern ausgetauscht werden müsste.
Eine andere, bereits erläuterte Variante einer asymetrischen Verschlüsselung ist das Diffie-Hellman-Verfahren \ref{buch:subsection:diffie-hellman}.
Im Gegensatz zu Diffie-Hellman gilt das McEliece-System als Quantencomputerresistent
und das Verschlüsseln/Entschlüsseln von Nachrichten wird hauptsächlich mit Matrizenoperationen durchgeführt.


