\section{Symmetrie}
Das Wort Symmetrie ist sehr alt und hat sich seltsamerweise von seinem
ursprünglichen griechischen Wort
\(\mathrm{\Sigma\nu\mu\mu\varepsilon\tau\rho\iota\alpha}\)
\footnote{\emph{Simmetr\'ia}: ein gemeinsames Mass habend, gleichmässig,
verhältnismässig} fast nicht verändert. In der Alltagssprache mag es ein
locker definierter Begriff sein, aber in der Mathematik hat Symmetrie eine sehr
präzise Bedeutung.
\begin{definition}[Symmetrie]
	Ein mathematisches Objekt wird als symmetrisch bezeichnet, wenn es unter einer
	bestimmten Operation invariant ist.
\end{definition}
Die intuitivsten Beispiele kommen aus der Geometrie, daher werden wir mit
einigen geometrischen Beispielen beginnen. Wie wir jedoch später sehen werden,
ist das Konzept der Symmetrie eigentlich viel allgemeiner.  

\begin{figure}
	\centering
	\includegraphics{papers/punktgruppen/figures/symmetric-shapes}
	\caption{
		Beispiele für geometrisch symmetrische Formen.
		\label{fig:punktgruppen:geometry-example}
	}
\end{figure}

\subsection{Geometrische Symmetrien}

In Abbildung \ref{fig:punktgruppen:geometry-example} haben wir einige Formen,
die offensichtlich symmetrisch sind.  Zum Beispiel hat das Quadrat Gerade, an
deren gespiegelt werden kann, ohne sein Aussehen zu verändern.  Regelmässige
Polygone mit \(n\) Seiten sind auch gute Beispiele, um eine diskrete
Rotationssymmetrie zu veranschaulichen, was bedeutet, dass eine Drehung um
einen Punkt um einen bestimmten Winkel \(360^\circ/n\) die Figur unverändert
lässt.  Das letzte Beispiel auf der rechten Seite ist eine unendliche
Rotationssymmetrie. Sie wird so genannt, weil es unendlich viele Werte für
\(\alpha \in \mathbb{R}\) gibt, die die Form unverändert lassen.  Dies ist
hoffentlich ausreichend, um die Bedeutung hinter der Notation zu verstehen, die
nun eingeführt wird.

\begin{definition}[Symmetriegruppe]
	Sei \(g\) eine Operation, die ein mathematisches Objekt unverändert lässt.
	Bei einer anderen Operation \(h\) definieren wir die Komposition \(h\circ g\)
	als die Anwendung der Operationen nacheinander. Alle Operationen bilden unter
	Komposition eine Gruppe, die Symmetriegruppe genannt wird.
\end{definition}

Mit dem oben Gesagten können wir das \(n\)-Gon Beispiel formalisieren. Wenn wir
\(r\) eine Drehung von \(2\pi/n\) sein lassen, gibt es eine wohlbekannte Symmetriegruppe
\[
	C_n = \langle r \rangle
		= \left\{\mathds{1}, r, r^2, \ldots, r^{n-1}\right\}, \]
die zyklische Gruppe heisst. Hier die Potenzen von \(r\) sind als wiederholte
Komposition gemeint, d.h. \(r^n = r\circ r \circ \cdots r\circ r\).  Die
Schreibweise mit den spitzen Klammern wird als Erzeugendensystem bezeichnet.
Das liegt daran, dass alle Elemente der Symmetriegruppe aus Kombinationen einer
Teilmenge erzeugt werden, die als erzeugende Elemente bezeichnet werden. 

% TODO: more on generators

Die Reflexionssymmetriegruppe ist nicht so interessant, da sie nur
\(\left\{\mathds{1}, \sigma\right\}\) enthält. Kombiniert man sie jedoch mit
der Rotation, erhält man die so genannte Diedergruppe
\[
	D_n = \langle r, \sigma : r^{n-1} = \sigma^2 = (\sigma r)^2 = \mathds{1} \rangle
		= \left\{
				\mathds{1}, r, \ldots, r^{n-1}, \sigma, \sigma r, \ldots, \sigma r^{n-1}
		\right\}.
\]
Diesmal muss die Generator-Notation die Beziehungen zwischen den beiden
Operationen beinhalten. 

% TODO
% Die ersten beiden sind leicht zu erkennen, für die
% letzte empfehlen wir, sie an einem 2D-Quadrat auszuprobieren.

Die Symmetrieoperationen, die wir bis jetzt besprochen haben, haben immer
mindestens einen Punkt gehabt, der wieder auf sich selbst abgebildet wird. Im
Fall der Rotation war es der Drehpunkt, bei der Spiegelung die Punkte der
Spiegelachse. Dies ist jedoch keine Voraussetzung für eine Symmetrie, da es
Symmetrien gibt, die jeden Punkt zu einem anderen Punkt verschieben können.
Diesen Spezialfall, bei dem mindestens ein Punkt unverändert bleibt, nennt man
Punktsymmetrie.
\begin{definition}[Punktgruppe]
	Wenn jede Operation in einer Symmetriegruppe die Eigenschaft hat, mindestens
	einen Punkt unverändert zu lassen, sagt man, dass die Symmetriegruppe eine
	Punktgruppe ist.
\end{definition}

\subsection{Algebraische Symmetrien}
Wir haben nun unseren Operationen Symbole gegeben, mit denen es tatsächlich
möglich ist, Gleichungen zu schreiben. Die naheliegende Frage ist dann, könnte
es sein, dass wir bereits etwas haben, das dasselbe tut?  Natürlich, ja.
Um es formaler zu beschreiben, werden wir ein einige Begriffe einführen.
\begin{definition}[Gruppenhomomorphismus]
	Seien \(G\) und \(H\) Gruppe mit unterschiedlicher Operation \(\diamond\)
	bzw.  \(\star\). Ein Homomorphismus\footnote{ Für eine ausführlichere
	Diskussion siehe \S\ref{buch:grundlagen:subsection:gruppen} im Buch.} ist
	eine Funktion \(f: G \to H\), so dass für jedes \(a, b \in G\) gilt
	\(f(a\diamond b) = f(a) \star f(b)\).  Man sagt, dass der Homomorphismus
	\(f\) \(G\) in \(H\) transformiert.
\end{definition}
\begin{beispiel}
	Die Rotationssymmetrie des Kreises \(C_\infty\), mit einem unendlichen
	Kontinuum von Werten \(\alpha \in \mathbb{R}\), entspricht perfekt dem
	komplexen Einheitskreis. Der Homomorphismus \(\phi: C_\infty \to \mathbb{C}\)
	ist durch die Eulersche Formel \(\phi(r) = e^{i\alpha}\) gegeben.
\end{beispiel}

\begin{definition}[Darstellung einer Gruppe]
	Die Darstellung einer Gruppe ist ein Homomorphismus, der eine Symmetriegruppe
	auf eine Menge von Matrizen abbildet.
	\[
		\Phi: G \to \operatorname{GL}_n(\mathbb{R}).
	\]
	Äquivalent kann man sagen, dass ein Element aus der Symmetriegruppe auf einen
	Vektorraum \(V\) wirkt, indem man definiert \(\Phi : G \times V \to V\).
\end{definition}
\begin{beispiel}
	Die Elemente \(r^k \in C_n\), wobei \(0 < k < n\), stellen abstrakt eine
	Drehung von \(2\pi k/n\) um den Ursprung dar. Die mit der Matrix 
	\[
		\Phi(r^k) = \begin{pmatrix}
			\cos(2\pi k/n) & -\sin(2\pi k/n) \\
			\sin(2\pi k/n) &  \cos(2\pi k/n)
		\end{pmatrix}
	\]
	definierte Funktion von \(C_n\) nach \(O(2)\) ist eine Darstellung von
	\(C_n\). In diesem Fall ist die erste Gruppenoperation die Komposition und
	die zweite die Matrixmultiplikation. Man kann überprüfen, dass \(\Phi(r^2
	\circ r) = \Phi(r^2)\Phi(r)\).
\end{beispiel}

%% TODO: title / fix continuity
Um das Konzept zu illustrieren, werden wir den umgekehrten Fall diskutieren:
eine Symmetrie, die keine Punktsymmetrie ist, die aber in der Physik sehr
nützlich ist, nämlich die Translationssymmetrie.  Von einem mathematischen
Objekt \(U\) wird gesagt, dass es eine Translationssymmetrie \(Q(x) = x + a\)
hat, wenn es die Gleichung 
\[
	U(x) = U(Q(x)) = U(x + a),
\]
für ein gewisses \(a\), erfüllt. Zum Beispiel besagt das erste Newtonsche
Gesetz, dass ein Objekt, auf das keine Kraft einwirkt, eine
zeitranslationsinvariante Geschwindigkeit hat, d.h. wenn \(\vec{F} = \vec{0}\)
dann \(\vec{v}(t) = \vec{v}(t + \tau)\).

% \subsection{Sch\"onflies notation} 

% vim:ts=2 sw=2 spell spelllang=de:
