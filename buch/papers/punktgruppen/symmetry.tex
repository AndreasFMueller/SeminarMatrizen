\section{Symmetrie}
Das Wort Symmetrie ist sehr alt und hat sich seltsamerweise von seinem
ursprünglichen griechischen Wort
\(\mathrm{\sigma\nu\mu\mu\varepsilon\tau\rho\iota\alpha}\)
\footnote{\emph{Simmetr\'ia}: ``ein gemeinsames Mass habend, gleichmässig,
verhältnismässig''} fast nicht verändert. In der Alltagssprache mag es ein
locker definierter Begriff sein, aber in der Mathematik hat Symmetrie eine sehr
präzise Bedeutung.
\begin{definition}[Symmetrie]
	Ein mathematisches Objekt wird als symmetrisch bezeichnet, wenn es unter einer
	bestimmten Operation invariant ist.
\end{definition}

Wenn der Leser noch nicht mit der Gruppentheorie in Berührung gekommen ist, ist
vielleicht nicht ganz klar, was eine Operation ist, aber die Definition sollte
trotzdem Sinn machen. Die Formalisierung dieser Idee wird bald kommen, aber
zunächst wollen wir etwas Intuition aufbauen.

\begin{figure}[h]
	\centering
	\begin{tikzpicture}[
			node distance = 2cm,
			shapetheme/.style = {
				very thick, draw = black, fill = magenta!20!white,
				minimum size = 2cm,
			},
			line/.style = {thick, draw = darkgray},
			axis/.style = {line, dashed},
			dot/.style = {
				circle, draw = darkgray, fill = darkgray,
				minimum size = 1mm, inner sep = 0, outer sep = 0,
			},
		]

		\node[
			shapetheme,
			rectangle
		] (R) {};
		\node[dot] at (R) {};
		\draw[axis] (R) ++(-1.5, 0) to ++(3, 0) node[right] {\(\sigma\)};

		\node[
			shapetheme,
			regular polygon,
			regular polygon sides = 5,
			right = of R,
		] (Ps) {};
		\node[dot] (P) at (Ps) {};
		\draw[line, dotted] (P) to ++(18:1.5);
		\draw[line, dotted] (P) to ++(90:1.5);
		\draw[line, ->] (P) ++(18:1.2) 
			arc (18:90:1.2) node[midway, above right] {\(r, 72^\circ\)};

		\node[
			shapetheme,
			circle, right = of P
		] (Cs) {};
		\node[dot] (C) at (Cs) {};
		\draw[line, dotted] (C) to ++(1.5,0);
		\draw[line, dotted] (C) to ++(60:1.5);
		\draw[line, ->] (C) ++(1.2,0)
			arc (0:60:1.2) node[midway, above right] {\(r, \alpha\)};

	\end{tikzpicture}
	\caption{
		Beispiele für geometrisch symmetrische Formen.
		\label{fig:punktgruppen:geometry-example}
	}
\end{figure}

Die intuitivsten Beispiele kommen aus der Geometrie, daher werden wir mit
einigen geometrischen Beispielen beginnen. Wie wir jedoch später sehen werden,
ist das Konzept der Symmetrie eigentlich viel allgemeiner.  In Abbildung
\ref{fig:punktgruppen:geometry-example} haben wir einige Formen, die
offensichtlich symmetrisch sind.  Zum Beispiel hat ein Quadrat viele Achsen, um
die es gedreht werden kann, ohne sein Aussehen zu verändern.  Regelmässige
Polygone mit \(n\) Seiten sind gute Beispiele, um eine diskrete
Rotationssymmetrie zu veranschaulichen, was bedeutet, dass eine Drehung um
einen Punkt um einen bestimmten Winkel \(360^\circ/n\) sie unverändert lässt.
Das letzte Beispiel auf der rechten Seite ist eine unendliche
Rotationssymmetrie. Sie wird so genannt, weil es unendlich viele Werte für
\(\alpha \in \mathbb{R}\) gibt, die die Form unverändert lassen.  Dies ist
hoffentlich ausreichend, um die Bedeutung hinter der Notation zu verstehen, die
nun eingeführt wird.

\begin{definition}[Symmetriegruppe]
	Sei \(g\) eine Operation, die ein mathematisches Objekt unverändert lässt.
	Bei einer anderen Operation \(r\) definieren wir die Komposition \(r\circ g\)
	als die Anwendung der Operationen nacheinander. Alle Operationen \(g_i\)
	bilden unter Komposition eine Gruppe, die Symmetriegruppe genannt wird.
\end{definition}

Mit dem oben Gesagten können wir das \(n\)-Gon Beispiel formalisieren. Wenn wir
\(r\) eine Drehung von \(2\pi/n\) sein lassen, gibt es eine wohlbekannte Symmetriegruppe
\[
	C_n = \left\{\mathbf{1}, r, r^2, \ldots, r^{n-1}\right\}
\]
die Zyklische Gruppe heisst.

\begin{definition}[Gruppenwirkung]
\end{definition}

% vim:ts=2 sw=2 spell spelllang=de:
