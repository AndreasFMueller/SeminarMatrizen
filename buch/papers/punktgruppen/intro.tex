\section{Einleitung}

Es gibt viele Möglichkeiten sich in Kristallen zu verlieren.
Auch wenn man nur die mathematischen Betrachtungsweisen berücksichtigt, hat man noch viel zu viele Optionen, sich mit Kristallen zu beschäftigen.
In diesem Kapitel wird daher der Fokus ``nur'' auf die Symmetrie gelegt.
Zu Beginn werden wir zeigen, was eine Symmetrie ausmacht und dass sie noch weit mehr in sich verbirgt als nur schön auszusehen.
Die vorgestellten Symmetrien sind äusserst gut geeignet, um die Grundeigenschaften eines Kristalles zu beschreiben.
Mit etwas kniffligen geometrischen Überlegungen kann man zeigen, was in der Welt der Kristallographie alles möglich ist oder nicht.
Diese erlauben alle möglichen Kristalle nach ihren Symmetrien in erstaunlich wenige Klassen zu kategorisieren.
Kategorien sind nicht nur für einen besseren Überblick nützlich, sondern kann man aus ihnen auch auf physikalische Eigenschaften schliessen.
Als spannendes Beispiel: Die Piezoelektrizität.
Piezoelektrizität beschreibt einen Effekt, ohne welchen diverse Altagsgegenständen nicht besonders nützlich wären.
Zum Beispiel sorgt er in den allermeisten Feuerzeugen für die Zündung.
Hiermit ist hoffentlich ein Funken Interesse geweckt um sich mit dem scheinbar trivialen Thema der Symmetrie auseinander zu setzten.

%% vim:linebreak breakindent showbreak=.. spell spelllang=de:
