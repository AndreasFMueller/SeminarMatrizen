\section{Einleitung}
Es gibt viele Möglichkeiten sich in Kristallen zu verlieren.
Auch wen man nur die mathematischen Betrachtungsweisen berücksichtigt, 
hat man noch viel zu viele Optionen sich mit Kristallen zu beschäftigen.
In diesem Kapitel wird daher der Fokus ``nur'' auf die Symmetrie gelegt.
Zu Beginn werden wir zeigen was eine Symmetrie ausmacht und 
dass sie noch weit mehr in sich verbirgt als nur schön auszusehen.
Die vorgestellten Symmetrien sind äusserst gut geeignet, 
um die Grundeigenschaften eines Kristalles zu beschreiben.
Mit etwas kniffligen geometrischen Überlegungen kann man zeigen, 
was in der Welt der Kristallographie alles möglich ist oder nicht.
Einschränkungen in Kristallsymmetrien sind durchaus willkommen, 
da dank ihnen sich die möglichen Kristallgitter in Grenzen halten
und sich kategorisieren lassen. 
Kategorien sind nicht nur für einen besseren Überblick nützlich, 
sondern kann man aus ihnen auch auf Physikalische Eigenschaften schliessen. 
Als spannendes Beispiel: Die Piezoelektrizität.
Die Piezoelektrizität ist vielleicht noch nicht jedem bekannt, 
sie versteckt sich aber in diversen Altagsgegenständen 
zum Beispiel sorgen sie in den meisten Feuerzeugen für die Zündung.
Hiermit ist hoffentlich ein Funken Interesse geweckt 
um sich mit dem scheinbar trivialen Thema der Symmetrie auseinander zu setzten.



%% vim:linebreak breakindent showbreak=.. spell spelllang=de:
