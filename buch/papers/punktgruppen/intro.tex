\section{Einleitung}
Es gibt viele Möglichkeiten sich in Kristallen zu verlieren.
Auch wen man nur die mathematischen Betrachtunngsweisen berüksichtigt, hat man noch viel zu viele Optionen sich mit Kristallen zu beschäftigen.
In diesem Kapitel ist daher der Fokus ``nur'' auf die Symmetrie gelegt.
Zu beginn werden wir zeigen was eine Symmetrie ausmacht und dass sie noch weit mehr in sich verbirgt als nur schön auszusehen.
Die vorgestellten Symmetrien sind äusserst gut geeignet um die Grundeigenschaften eines Kristalles zu Beschreiben.
Mit etwas kiffligen geometrischen Überlegungen kann man zeigen wass in der Welt der Kristallographie alles möglich ist oder nicht.
Die Einschränkungen sind durchaus wilkommen, dank ihnen halten sich die möglichen Kristallgitter in Grenzen und Lassen sich Kategorisieren.
Kategorien sind nicht nur für einen besseren Überblich nützlich, sondern kann man aus ihnen auch auf Physikalische Eigenschaften schliessen, als spannendes Beispiel: Die Piezoelektrizität.
Die Piezoelektrizität ist vielleicht noch nicht jedem bekannt, sie versteckt sich aber in diversen Altagsgegenständen zum Beispiel sorgen sie in den meisten Feuerzeugen für die Zündung.
Ein Funken Interesse ist hoffentlich geweckt um sich mit dem scheinbar trivialen thema der Symmetrie auseinander zu setzten.



