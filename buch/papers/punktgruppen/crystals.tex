\section{Kristalle}
Unter dem Begriff Kristall sollte sich jeder ein Bild machen können. 
Wir werden uns aber nicht auf sein Äusseres fokussieren, sondern was ihn im Inneren ausmacht.
Die Innereien eines Kristalles sind glücklicherweise relativ einfach definiert.
\begin{definition}[Kristall]
    Ein Kristall besteht aus Atomen, welche sich in einem Muster arrangieren, welches sich in drei Dimensionen periodisch wiederholt.
\end{definition}


Ein Zweidimensionales Beispiel eines solchen Muster ist Abbildung \ref{fig:punktgruppen:lattce-grid}.
Für die Überschaubarkeit haben wir ein simples Muster eines einzelnen XgrauenX Punktes gewählt in nur Zwei Dimensionen.
Die eingezeichneten Vektoren a und b sind die kleinstmöglichen Schritte im Raum bis sich das Kristallgitter wiederholt.
Dadurch können von einem einzelnen XGrauenX Gitterpunkt in \ref{fig:punktgruppen:lattce-grid} können mit einer ganzzahligen Linearkombination von a und b alle anderen Gitterpunkte des Kristalles erreicht werden.
Ein Kristallgitter kann eindeutig mit a und b und deren winkeln beschrieben werden weswegen a und b auch Gitterparameter genannt werden.
Im Dreidimensionalen-Raum können alle Gitterpunkte mit derselben Idee und einem zusätzlichen Vektor also FRMEL FÜR TRANSLATIONSVEKTOR  erreicht werden.
Da sich das Ganze Kristallgitter wiederholt, wiederholen sich auch die Eigenschaften eines Gitterpunktes Periodisch mit eiem
