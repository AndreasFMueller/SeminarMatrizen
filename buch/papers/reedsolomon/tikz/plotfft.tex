%
% Plot der Übertrangungsabfolge ins FFT und zurück mit IFFT
%
\documentclass[tikz]{standalone}
\usepackage{amsmath}
\usepackage{times}
\usepackage{pgfplots}
\usepackage{pgfplotstable}
\usepackage{csvsimple}
\usepackage{filecontents}



\begin{document}
\begin{tikzpicture}[]

	%---------------------------------------------------------------
	%Knote
	\matrix(m) [draw = none, column sep=25mm, row sep=2mm]{

		\node(signal)  []    {
		\begin{tikzpicture}
			\begin{axis}
				[title = {\Large {Signal}}, 
				xtick={0,20,40,64,80,98}]
				\addplot[blue] table[col sep=comma] {tikz/signal.txt};
			\end{axis}
		\end{tikzpicture}}; &
		
		\node(codiert) []    {
		\begin{tikzpicture}[]
			% Beschriftung Rechts
			\begin{axis}[axis x line= none, axis y line*=right,ytick={0}]
				\addplot[color=white] {0};
			\end{axis}

			\begin{axis}[ title = {\Large {Codiert}}, axis y line*=left]
				\addplot[color=black!60!green] table[col sep=comma] {tikz/codiert.txt};
			\end{axis}
		\end{tikzpicture}}; \\
		
		\node(decodiert) []    {
		\begin{tikzpicture}
			\begin{axis}[title = {\Large {Decodiert}}]
				\addplot[blue] table[col sep=comma] {tikz/decodiert.txt};
			\end{axis}
		\end{tikzpicture}}; &
	
		\node(empfangen) []    {
		\begin{tikzpicture}
			\begin{axis}[title = {\Large {Empfangen \space + \space Fehler}},
				xtick={0,40,60,100}, axis y line*=left]
				\addplot[color=black!60!green] table[col sep=comma] {tikz/empfangen.txt};
			\end{axis}
			\begin{axis}[xtick={7,21,75}, axis y line*=right]
				\addplot[red] table[col sep=comma] {tikz/fehler.txt};
			\end{axis}
		\end{tikzpicture}};\\
	
		\node(syndrom) []   {
		\begin{tikzpicture}
			\begin{axis}[title = {\Large {Syndrom}}]
				\addplot[black] table[col sep=comma] {tikz/syndrom.txt};
			\end{axis}
		\end{tikzpicture}}; &
	
		\node(locator) []    {
		\begin{tikzpicture}
			% Beschriftung Rechts
			\begin{axis}[axis x line= none, axis y line*=right, ytick={0.3}];
				\addplot[color=black!60] {0.3};
			\end{axis}

			\begin{axis}[title = {\Large {Locator}},axis y line*=left]
				\addplot[gray] table[col sep=comma] {tikz/locator.txt};
			\end{axis}
		\end{tikzpicture}};\\
	};
	%-------------------------------------------------------------
		%FFT & IFFT deskription
	
	\draw[thin,gray,dashed] (0,9) to (0,-9);
	\node(IFFT)  [scale=0.9]  at (0,9.3)   {IFFT};
	\draw[stealth-](IFFT.south west)--(IFFT.south east);
	\node(FFT)  [scale=0.9, above of=IFFT]    {FFT};
	\draw[-stealth](FFT.north west)--(FFT.north east);
	
	%Arrows
	\draw[thick, ->] (signal.east) to (codiert.west);
	\draw[thick, ->] (codiert.south) to (empfangen.north);
	\draw[thick, ->] (empfangen.west) to (decodiert.east);
	\draw[thick, ->] (syndrom.east) to (locator.west);
	\draw[thick](decodiert.south east)++(-1.8,1)  ellipse (1.3cm and 0.8cm) ++(-1.3,0) coordinate(zoom) ;
	\draw[thick, ->] (zoom) to[out=180, in=90] (syndrom.north);
	
	%item
	\node[circle, draw, fill =lightgray] at (signal.north west) {1};
	\node[circle, draw, fill =lightgray] at (codiert.north west) {2};
	\node[circle, draw, fill =lightgray] at (empfangen.north west) {3};
	\node[circle, draw, fill =lightgray] at (decodiert.north west) {4};
	\node[circle, draw, fill =lightgray] at (syndrom.north west) {5};
	\node[circle, draw, fill =lightgray] at (locator.north west) {6};
\end{tikzpicture}	
\end{document}