%
% Plot der Übertrangungsabfolge ins FFT und zurück mit IFFT
%
\documentclass[tikz]{standalone}
\usepackage{amsmath}
\usepackage{times}
\usepackage{pgfplots}
\usepackage{pgfplotstable}
\usepackage{csvsimple}
\usepackage{filecontents}

\def\plotwidth{7.5cm}
\def\plotheight{5.5cm}
\def\xverschiebung{4.5cm}
\def\yverschiebung{-7cm}
\def\yyverschiebung{-14cm}

\def\marke#1{
	\coordinate (M) at (-0.8,4.6);
	\fill[color=lightgray] (M) circle[radius=0.3];
	\draw (M) circle[radius=0.3];
	\node at (M) {#1};
}

\begin{document}
\begin{tikzpicture}[>=latex,thick]

\draw[dashed,line width=2pt,color=lightgray] (2.6,4.4) -- (2.6,-14.3);
\coordinate (B) at (2.6,-1.3);
\node[color=gray] at (B) [rotate=90,above] {Zeitbereich};
\node[color=gray] at (B) [rotate=90,below] {Frequenzbereich};

\begin{scope}[xshift=-\xverschiebung,yshift=0cm]
	\begin{axis}
		[title = {\large Signal\strut}, 
		xtick={0,32,64,96}, width=\plotwidth,height=\plotheight]
		\addplot[blue,line width=1pt] table[col sep=comma]
			{tikz/signal.txt};
	\end{axis}
	\marke{1}
\end{scope}

\begin{scope}[xshift=\xverschiebung,yshift=0cm]
	\begin{axis}[axis x line= none, axis y line*=right,ytick={0},
		width=\plotwidth,height=\plotheight]
		\addplot[color=white] {0};
	\end{axis}

	\begin{axis}[title = {\large Codiert\strut}, axis y line*=left,
		xtick={0,32,64,96},
		width=\plotwidth,height=\plotheight]
		\addplot[color=black!60!green,line width=1pt]
			table[col sep=comma]
			{tikz/codiert.txt};
	\end{axis}
	\marke{2}
	\draw[->,line width=1pt] (3,-0.4) -- node[right] {Übertragung} (3,-2.2);
\end{scope}

\definecolor{pink}{rgb}{0.6,0.2,1}

\begin{scope}[xshift=-\xverschiebung,yshift=\yverschiebung]
	\fill[color=pink!20] (4.65,0.35) ellipse (1.1cm and 0.5cm);
	\begin{axis}[title = {\large Decodiert\strut},
		xtick={0,32,64,96},
		width=\plotwidth,height=\plotheight]
		\addplot[blue,line width=1pt]
			table[col sep=comma] {tikz/decodiert.txt};
	\end{axis}
	\marke{4}
	\draw[color=pink] (4.65,0.35) ellipse (1.1cm and 0.5cm);
	\draw[->,color=pink,line width=1pt]
		(4.65,-0.15) to[out=-90,in=90] (3,-2.2);
\end{scope}
	
\begin{scope}[xshift=\xverschiebung,yshift=\yverschiebung]
	\begin{axis}[title = {\large Empfangen {\color{red} mit Fehler}\strut},
		xtick={0,96},
		axis y line*=left,
		width=\plotwidth,height=\plotheight]
		\addplot[color=black!60!green,line width=1pt]
			table[col sep=comma]
			{tikz/empfangen.txt};
	\end{axis}
	\begin{axis}[xtick={6,20,74}, axis y line*=right,
		width=\plotwidth,height=\plotheight]
		\addplot[red,line width=1pt]
			table[col sep=comma] {tikz/fehler.txt};
	\end{axis}
	\marke{3}
\end{scope}

\begin{scope}[xshift=-\xverschiebung,yshift=\yyverschiebung]
	\begin{axis}[title = {\large \color{pink}Syndrom\strut},
		xtick={0,32,64,96},
		width=\plotwidth,height=\plotheight]
		\addplot[pink,line width=1pt]
			table[col sep=comma] {tikz/syndrom.txt};
	\end{axis}
	\marke{5}
\end{scope}

\begin{scope}[xshift=\xverschiebung,yshift=\yyverschiebung]
	% Beschriftung Rechts
	\begin{axis}[axis x line= none, axis y line*=right, ytick={0.3},
		xtick={0,32,64,96},
		width=\plotwidth,height=\plotheight]
		\addplot[color=black!60,line width=1pt] {0.3};
	\end{axis}
	\begin{axis}[title = {\large Locator\strut},axis y line*=left,
		xtick={0,6,20,74,96},
		width=\plotwidth,height=\plotheight]
		\addplot[gray,line width=1pt]
			table[col sep=comma] {tikz/locator.txt};
	\end{axis}
	\marke{6}
\end{scope}
	
% Fourier-Transformations-Pfeile

\draw[->,line width=1pt] (1.8,2) -- node[above] {DFT\strut}  (3.8,2);

\begin{scope}[yshift=\yverschiebung]
\draw[<-,line width=1pt] (1.8,2) -- node[above] {DFT$\mathstrut^{-1}$}  (3.8,2);
\end{scope}

\begin{scope}[yshift=\yyverschiebung]
\draw[->,line width=1pt] (1.8,2) -- node[above] {DFT\strut}  (3.8,2);
\end{scope}
	
\end{tikzpicture}	
\end{document}
