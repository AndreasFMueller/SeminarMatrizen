%
% Plot der èbertrangungsabfolge ins FFT und zurück mit IFFT
%
\tikzset{
	node/.style={rectangle, draw=black!100, thick, on grid}, % on grid added
	dangling node/.style={node, fill=black!30}
}
\begin{tikzpicture}[]

%---------------------------------------------------------------
	%Knote
\matrix[draw = none, column sep=20mm, row sep=20mm]{
	\node(signal)  []    {
	\begin{tikzpicture}
		\begin{axis}[title = {\Large {Signal}}]
			\addplot[] table[col sep=comma] {signal.txt};
		\end{axis}
	\end{tikzpicture}}; &
	
	\node(codiert) []    {
	\begin{tikzpicture}
		\begin{axis}[title = {\Large {Codiert}}]
			\addplot[] table[col sep=comma] {codiert.txt};
		\end{axis}
	\end{tikzpicture}}; \\
		
	&\node(fehler) []    {
	\begin{tikzpicture}
		\begin{axis}[scale=0.6, title = {\Large {Fehler}}]
			\addplot[] table[col sep=comma] {fehler.txt};
		\end{axis}
	\end{tikzpicture}};\\
	
	\node(decodiert) []    {
	\begin{tikzpicture}
		\begin{axis}[title = {\Large {Decodiert}}]
			\addplot[] table[col sep=comma] {decodiert.txt};
		\end{axis}
	\end{tikzpicture}}; &

	\node(empfangen) []    {
	\begin{tikzpicture}
		\begin{axis}[title = {\Large {Empfangen}}]
			\addplot[] table[col sep=comma] {empfangen.txt};
		\end{axis}
	\end{tikzpicture}};\\

	\node(syndrom) []   {
	\begin{tikzpicture}
		\begin{axis}[title = {\Large {Syndrom}}]
			\addplot[] table[col sep=comma] {syndrom.txt};
		\end{axis}
	\end{tikzpicture}}; &

	\node(locator) []    {
	\begin{tikzpicture}
		\begin{axis}[title = {\Large {Locator}}]
			\addplot[] table[col sep=comma] {locator.txt};
		\end{axis}
	\end{tikzpicture}};\\
};
%-------------------------------------------------------------
	%FFT & IFFT deskription

	\draw[thin,gray,dashed] (0,15) to (0,-15);
	\node(FFT)  [ scale=0.7]  at (0,15.3)   {FFT IFFT};
	
	%Arrows
	\draw[ultra thick, ->] (signal.east) to (codiert.west);
	\draw[ultra thick, ->] (codiert.south) to (fehler.north);
	\draw[ultra thick, ->] (fehler.south) to (empfangen.north);
	\draw[ultra thick, ->] (empfangen.west) to (decodiert.east);
	\draw[ultra thick, ->] (syndrom.east) to (locator.west);
	\draw(decodiert.south east)++(-1.8,1)  ellipse (1.3cm and 0.8cm) ++(-1.3,0) coordinate(zoom) ;
	\draw[ultra thick, ->] (zoom) to[out=180, in=90] (syndrom.north);
	
 \end{tikzpicture}