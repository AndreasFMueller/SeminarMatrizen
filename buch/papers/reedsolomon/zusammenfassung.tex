%
% zusammenfassung.tex -- Zusammenfassung
%
% (c) 2021 Michael Steiner, Hochschule Rapperswil
%
\section{Zusammenfassung
\label{reedsolomon:section:zf}}
\rhead{Zusammenfassung}
Dieser Abschnitt beinhaltet eine Übersicht über die Funktionsweise eines Reed-Solomon-Codes für beliebige endliche Körper.

\subsubsection{Schritt 1: primitives Element}
Zu Beginn soll entschieden werden, in welchem endlichen Körper $\mathbb{F}_{q}$ gerechnet werden soll. 
Ausserdem muss im gewählten Körper eine primitive Einheitswurzel gefunden, bzw. bestimmt werden. 

\subsubsection{Schritt 2: Codierung}
Für die Codierung wird die Nachricht als Koeffizienten des Polynoms $m(X)$ geschrieben, anschliessend wird $a^i$ in $m(X)$ eingesetzt. 
Daraus ergibt sich die Codierungsmatrix
\[
A(a) = 
\begin{pmatrix}
a^0 & a^0 & a^0 & \dots \\
a^0 & a^1 & a^2 & \dots \\
a^0 & a^2 & a^4 & \dots \\
\vdots&\vdots&\vdots&\ddots
\end{pmatrix}
.
\]
Mit dieser Matrix können wir den Nachrichtenblock zum Übertragungsvektor codieren. 

\subsubsection{Schritt 3: Decodierung ohne Fehler}
Im ersten Schritt zur Decodierung muss geprüft werden, ob der Übertragungsvektor Fehler beinhaltet.
Ist dies nicht der Fall, so kann die Matrix $A(a)$ invertiert werden mit
\[
A(a)^{-1} = \frac{1}{q-1} \cdot A(a^{-1}).
\]
Die Codierungsmatrix ändert sich somit zur Decodierungsmatrix
\[
\begin{pmatrix}
	a^0 & a^0 & a^0 & \dots \\
	a^0 & a^1 & a^2 & \dots \\
	a^0 & a^2 & a^4 & \dots \\
	\vdots&\vdots&\vdots &\ddots
\end{pmatrix}
= 
\frac{1}{q-1}
\cdot
\begin{pmatrix}
	a^0 & a^0 & a^0 & \dots \\
	a^0 & a^{-1} & a^{-2} & \dots \\
	a^0 & a^{-2} & a^{-4} & \dots \\
	\vdots&\vdots&\vdots&\ddots
\end{pmatrix}
.
\]
Daraus lässt sich der Nachrichtenblock aus dem Übertragungsvektor rekonstruieren.

\subsubsection{Schritt 4: Decodierung mit Fehler}
Sollte der Übertragungsvektor fehlerhaft empfangen werden, so kann der Nachrichtenblock nicht durch invertieren der Matrix rekonstruiert werden.
Zur Lokalisierung der Fehlerstellen nehmen wir das Polynom $f(X)$ zur Hilfe, welches wir über den Satz von Fermat bestimmt haben.
Berechnen wir daraus das $\operatorname{kgV}$ von $f(X)$ und $d(X)$, so erhalten wir ein Lokatorpolynom.
Durch das bestimmen der Exponenten erhalten wir die Fehlerhaften Stellen im Übertragungsvektor.
Für die Rekonstruktion stellen wir ein Gleichungssystem auf und entfernen daraus die Fehlerhaften Zeilen.
Im Anschluss kann das verkleinerte Gleichungssystem gelöst werden. 
Als Resultat erhalten wir die fehlerfreie Nachricht.
%Aus diesem Grund suchen wir nach einem Lokatorpolynom, welches uns die Fehlerhaften Stellen im Übertragungsvektor anzeigt.
%Dazu nehmen wir das Polynom $f(X)$, welches wir durch den Satz von Fermat erhalten, und berechnen so das $\operatorname{kgV}(f(X),d(X))$ und kommen so auf das Lokatorpolynom $l(X)$. Durch das bestimmen von den Exponenten erhalten wir die Fehlerstellen, welche wir aus dem Gleichungssystem entfernen müssen. Übrig bleibt das berechnen dieses Gleichungssystems. 
