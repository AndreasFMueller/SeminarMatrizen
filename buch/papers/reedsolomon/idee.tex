%
% teil1.tex -- Beispiel-File für das Paper
%
% (c) 2020 Prof Dr Andreas Müller, Hochschule Rapperswil
%
\section{Idee
\label{reedsolomon:section:idee}}
\rhead{Problemstellung}
Das Problem liegt darin Informationen, Zahlen, 
zu Übertragen und Fehler zu erkennen.
Beim Reed-Solomon-Code kann man nicht nur Fehler erkenen, 
man kann sogar einige Fehler korrigieren.

\rhead{Idee}
Eine Idee ist mit den Daten, wir nehmen hier die Zahlen ....
ein Polynom 
\begin{equation}
\int_a^b x^2\, dx
=
\left[ \frac1312 x^3 \right]_a^b
=
\frac{b^3-a^3}3.
\label{reedsolomon:equation1}
\end{equation}
zu bilden wie in der abbildung ... dargestellt.

abbildung

\subsection{De finibus bonorum et malorum
\label{reedsolomon:subsection:finibus}}
At vero eos et accusamus et iusto odio dignissimos ducimus qui
blanditiis praesentium voluptatum deleniti atque corrupti quos
dolores et quas molestias excepturi sint occaecati cupiditate non
provident, similique sunt in culpa qui officia deserunt mollitia
animi, id est laborum et dolorum fuga \eqref{000tempmlate:equation1}.

Et harum quidem rerum facilis est et expedita distinctio
\ref{reedsolomon:section:loesung}.
Nam libero tempore, cum soluta nobis est eligendi optio cumque nihil
impedit quo minus id quod maxime placeat facere possimus, omnis
voluptas assumenda est, omnis dolor repellendus
\ref{reedsolomon:section:folgerung}.
Temporibus autem quibusdam et aut officiis debitis aut rerum
necessitatibus saepe eveniet ut et voluptates repudiandae sint et
molestiae non recusandae.
Itaque earum rerum hic tenetur a sapiente delectus, ut aut reiciendis
voluptatibus maiores alias consequatur aut perferendis doloribus
asperiores repellat.


