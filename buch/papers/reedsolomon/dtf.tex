%
% teil3.tex -- Beispiel-File für Teil 3
%
% (c) 2020 Prof Dr Andreas Müller, Hochschule Rapperswil
%
\section{Diskrete Fourier Transformation
\label{reedsolomon:section:dtf}}
\rhead{Umwandlung mit DTF}
Um die Polynominterpolation zu umgehen, gehen wir nun über in die Fourientransformation.
Dies wird weder eine erklärung der Forientransorfmation noch ein genauer gebrauch
für den Reed-Solomon-Code. Dieser Abschnitt zeigt nur wie die Fourientransformation auf Fehler reagiert.
wobei sie dann bei späteren Berchnungen ganz nützlich ist.

\subsection{Diskrete Fourientransformation Zusamenhang
\label{reedsolomon:subsection:dtfzusamenhang}}
Die Diskrete Fourientransformation ist definiert als
	\[
	\label{ft_discrete}
	\hat{c}_{k} 
	= \frac{1}{N} \sum_{n=0}^{N-1}
	{f}_n \cdot e^{-\frac{2\pi j}{N} \cdot kn}
	\]
, wenn man nun 
	\[
	w = e^{-\frac{2\pi j}{N} k}
	\]
ersetzte, und $N$ konstantbleibt, erhält man
	\[
	\hat{c}_{k}=\frac{1}{N}( {f}_0 w^0 + {f}_1 w^1 + {f}_2 w^2 + \dots + {f}_{N-1} w^N)
	\]
was überaust ähnlich zu unserem Polynomidee ist.
\subsection{Übertragungsabfolge
\label{reedsolomon:subsection:Übertragungsabfolge}}

\begin{enumerate}[1)]
\item Das Signal hat 64 die Daten, Zahlen welche übertragen werden sollen. 
Dabei zusätzlich nach 16 Fehler abgesichert, macht insgesamt 96 Übertragungszahlen.
\item Nun wurde mittels der schnellen diskreten Fourientransformation diese 96 codiert.
Das heisst alle information ist in alle Zahlenvorhanden.
\item Nun kommen drei Fehler dazu an den Übertragungsstellen 7, 21 und 75.
\item Dieses wird nun Empfangen und mittels inversen diskreten Fourientransormation, wieder rücktransformiert.
\item Nun sieht man den Fehler im Decodieren in den Übertragungsstellen 64 bis 96.
\item Nimmt man nun nur diese Stellen 64 bis 96, auch Syndrom genannt, und Transformiert diese.
\item Bekommt man die Fehlerstellen im Locator wieder, zwar nichtso genau, dennoch erkkent man wo die Fehler stattgefunden haben.
\end{enumerate}

\begin{figure}
	\centering
	\resizebox{0.9\textwidth}{!}{
	%\includegraphics[width=0.5\textwidth]{papers/reedsolomon/images/plot.pdf}
    %
% Plot der Übertrangungsabfolge ins FFT und zurück mit IFFT
%
\documentclass[tikz]{standalone}
\usepackage{amsmath}
\usepackage{times}
\usepackage{pgfplots}
\usepackage{pgfplotstable}
\usepackage{csvsimple}
\usepackage{filecontents}


\begin{document}
\begin{tikzpicture}[]

	%---------------------------------------------------------------
	%Knote
	\matrix(m) [draw = none, column sep=25mm, row sep=2mm]{

		\node(signal)  []    {
		\begin{tikzpicture}
			\begin{axis}
				[title = {\Large {Signal}}, 
				xtick={0,20,40,64,80,98}]
				\addplot[black] table[col sep=comma] {tikz/signal.txt};
			\end{axis}
		\end{tikzpicture}}; &
		
		\node(codiert) []    {
		\begin{tikzpicture}[]
			\begin{axis}[ title = {\Large {Codiert \space + \space Fehler}},
				xtick={0,40,60,100}, axis y line*=left]
				\addplot[green] table[col sep=comma] {tikz/codiert.txt};
			\end{axis}
			\begin{axis}[xtick={7,21,75}, axis y line*=right]
					\addplot[red] table[col sep=comma] {tikz/fehler.txt};
			\end{axis}
		\end{tikzpicture}}; \\
		
		\node(decodiert) []    {
		\begin{tikzpicture}
			\begin{axis}[title = {\Large {Decodiert}}]
				\addplot[blue] table[col sep=comma] {tikz/decodiert.txt};
			\end{axis}
		\end{tikzpicture}}; &
	
		\node(empfangen) []    {
		\begin{tikzpicture}
			\begin{axis}[title = {\Large {Empfangen}}]
				\addplot[green] table[col sep=comma] {tikz/empfangen.txt};
			\end{axis}
		\end{tikzpicture}};\\
	
		\node(syndrom) []   {
		\begin{tikzpicture}
			\begin{axis}[title = {\Large {Syndrom}}]
				\addplot[blue] table[col sep=comma] {tikz/syndrom.txt};
			\end{axis}
		\end{tikzpicture}}; &
	
		\node(locator) []    {
		\begin{tikzpicture}
			\begin{axis}[title = {\Large {Locator}}]
				\addplot[] table[col sep=comma] {tikz/locator.txt};
			\end{axis}
		\end{tikzpicture}};\\
	};
	%-------------------------------------------------------------
		%FFT & IFFT deskription
	
	\draw[thin,gray,dashed] (0,9) to (0,-9);
	\node(IFFT)  [scale=0.8]  at (0,9.3)   {IFFT};
	\draw[stealth-](IFFT.south west)--(IFFT.south east);
	\node(FFT)  [scale=0.8, above of=IFFT]    {FFT};
	\draw[-stealth](FFT.north west)--(FFT.north east);
	
	\draw[thick, ->,] (codiert)++(-1,0) +(0.05,0.5) -- +(-0.1,-0.1) -- +(0.1,0.1) -- +(0,-0.5);
	%Arrows
	\draw[thick, ->] (signal.east) to (codiert.west);
	\draw[thick, ->] (codiert.south) to (empfangen.north);
	\draw[thick, ->] (empfangen.west) to (decodiert.east);
	\draw[thick, ->] (syndrom.east) to (locator.west);
	\draw[thick](decodiert.south east)++(-1.8,1)  ellipse (1.3cm and 0.8cm) ++(-1.3,0) coordinate(zoom) ;
	\draw[thick, ->] (zoom) to[out=180, in=90] (syndrom.north);
	
	%item
	\node[circle, draw, fill =lightgray] at (signal.north west) {1};
	\node[circle, draw, fill =lightgray] at (codiert.north west) {2+3};
	\node[circle, draw, fill =lightgray] at (empfangen.north west) {4};
	\node[circle, draw, fill =lightgray] at (decodiert.north west) {5};
	\node[circle, draw, fill =lightgray] at (syndrom.north west) {6};
	\node[circle, draw, fill =lightgray] at (locator.north west) {7};
\end{tikzpicture}	
\end{document}
	}
	\caption{Übertragungsabfolge \ref{reedsolomon:subsection:Übertragungsabfolge}}
	\label{fig:sendorder}
\end{figure}