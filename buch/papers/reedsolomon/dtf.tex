%
% teil3.tex -- Beispiel-File für Teil 3
%
% (c) 2020 Prof Dr Andreas Müller, Hochschule Rapperswil
%
\section{Diskrete Fourier Transformation
\label{reedsolomon:section:dtf}}
\rhead{Umwandlung mit DTF}
Um die Polynominterpolation zu umgehen, gehen wir nun über in die Fourientransformation.
Dies wird weder eine erklärung der Forientransorfmation noch ein genauer gebrauch
für den Reed-Solomon-Code. Dieser Abschnitt zeigt nur wie die Fourientransformation auf Fehler reagiert.
wobei sie dann bei späteren Berchnungen ganz nützlich ist.

\subsection{Diskrete Fourientransformation Zusamenhang
\label{reedsolomon:subsection:dtfzusamenhang}}
Die Diskrete Fourientransformation ist definiert als

\subsection{Übertragungsabfolge
\label{reedsolomon:subsection:Übertragungsabfolge}}
Das Signal.... sind die Daten, Zahlen welche übertragen werden sollen.
Das speziell ist das wir 100 Punkte übertragen und von 64 bis 100,
werden nur Null Punkte übertragen, dies weiss auch unser Empfänger.
Nun wird das Signal in Abbildung... codiert...
Somit wird die Information jedes Punktes auf das ganze spektrum von 0 bis 100 übertragen.
Kommen nuun drei Fehler... hinzu zu diesem codierten Signal sind diese nicht zu erkennen.
Nach dem Empfangen... und decodieren ... erkennt man die fehlerhafte information in den Punkten 64 bis 100.
Filtert man nur diese Punkte heraus und Transformiert sie mit Fourier erhält man die stellen an denen die Fehler sich eingeschlichen haben.





