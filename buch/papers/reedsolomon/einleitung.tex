%
% einleitung.tex -- Beispiel-File für die Einleitung
%
% (c) 2020 Prof Dr Andreas Müller, Hochschule Rapperswil
%
\section{Einleitung
\label{reedsolomon:section:einleitung}}
\rhead{Einleitung}
Der Reed-Solomon-Code ist entstanden um,
das Problem der Fehler, bei der Datenübertragung, zu lösen.
In diesem Abschnitt wird möglichst verständlich die mathematische Abfolge, Funktion oder Algorithmus erklärt.
Es wird jedoch nicht auf die technische Umsetzung oder Implementierung eingegangen.
Um beim Datenübertragen Fehler zu erkennen, könnte man die Daten jeweils doppelt senden,
und so jeweilige Fehler zu erkennen.
Doch nur schon um weinige Fehler zu erkennen werden überproportional viele Daten doppelt und dreifach gesendet.
Der Reed-Solomon-Code macht dies auf eine andere, clevere Weise.



