% polynome1
%-------------------
\documentclass[tikz]{standalone}
\usepackage{amsmath}
\usepackage{times}
\usepackage{txfonts}
\usepackage{pgfplots}
\usepackage{csvsimple}
\usepackage{pgfplotstable}
\usepackage{filecontents}
\usetikzlibrary{arrows,intersections,math}
\newcommand{\x}{10}
\newcommand{\y}{-8}
\begin{document}

\begin{tikzpicture}[]
	
	%---------------------------------------------------------------
	%Knote
	\matrix[draw = none, column sep=20mm, row sep=4mm]{
		\node(signal)  []    {
			\begin{tikzpicture}
				\begin{axis}[ 
					title = {\Large {Signal}}, 
					xlabel={Anzahl Übertragene Zahlen},
					xtick={0,20,40,64,80,98},]
					\addplot[blue] table[col sep=comma] {signal.txt};
				\end{axis}
		\end{tikzpicture}}; &
		
		\node(codiert) []    {
			\begin{tikzpicture}
				\begin{axis}[title = {\Large {Codiert}}]
					\addplot[] table[col sep=comma] {codiert.txt};
				\end{axis}
		\end{tikzpicture}}; \\
		
		&\node(fehler) []    {
			\begin{tikzpicture}
				\begin{axis}[scale=0.6, title = {\Large {Fehler}}]
					\addplot[red] table[col sep=comma] {fehler.txt};
				\end{axis}
		\end{tikzpicture}};\\
		
		\node(decodiert) []    {
			\begin{tikzpicture}
				\begin{axis}[title = {\Large {Decodiert}}]
					\addplot[blue] table[col sep=comma] {decodiert.txt};
				\end{axis}
		\end{tikzpicture}}; &
		
		\node(empfangen) []    {
			\begin{tikzpicture}
				\begin{axis}[title = {\Large {Empfangen}}]
					\addplot[] table[col sep=comma] {empfangen.txt};
				\end{axis}
		\end{tikzpicture}};\\
		
		\node(syndrom) []   {
			\begin{tikzpicture}
				\begin{axis}[title = {\Large {Syndrom}}]
					\addplot[blue] table[col sep=comma] {syndrom.txt};
				\end{axis}
		\end{tikzpicture}}; &
		
		\node(locator) []    {
			\begin{tikzpicture}
				\begin{axis}[title = {\Large {Locator}}]
					\addplot[] table[col sep=comma] {locator.txt};
				\end{axis}
		\end{tikzpicture}};\\
	};
	%-------------------------------------------------------------
	%FFT & IFFT deskription
	
	\draw[thin,gray,dashed] (0,12) to (0,-12);
	\node(IFFT)  [scale=0.7]  at (0,12.3)   {IFFT};
	\draw[<-](IFFT.south west)--(IFFT.south east);
	\node(FFT)  [scale=0.7, above of=IFFT]    {FFT};
	\draw[->](FFT.north west)--(FFT.north east);
	
	\draw[thick, ->,] (fehler.west)++(-1,0) +(0.05,0.5) -- +(-0.1,-0.1) -- +(0.1,0.1) -- +(0,-0.5);
	%Arrows
	\draw[ultra thick, ->] (signal.east) to (codiert.west);
	\draw[ultra thick, ->] (codiert.south) to (fehler.north);
	\draw[ultra thick, ->] (fehler.south) to (empfangen.north);
	\draw[ultra thick, ->] (empfangen.west) to (decodiert.east);
	\draw[ultra thick, ->] (syndrom.east) to (locator.west);
	\draw(decodiert.south east)++(-1.8,1)  ellipse (1.3cm and 0.8cm) ++(-1.3,0) coordinate(zoom) ;
	\draw[ultra thick, ->] (zoom) to[out=180, in=90] (syndrom.north);
	
	%item
	\node[circle, draw, fill =lightgray] at (signal.north west) {1};
	\node[circle, draw, fill =lightgray] at (codiert.north west) {2};
	\node[circle, draw, fill =lightgray] at (fehler.north west) {3};
	\node[circle, draw, fill =lightgray] at (empfangen.north west) {4};
	\node[circle, draw, fill =lightgray] at (decodiert.north west) {5};
	\node[circle, draw, fill =lightgray] at (syndrom.north west) {6};
	\node[circle, draw, fill =lightgray] at (locator.north west) {7};
	
\end{tikzpicture}
\end{document}

