%
% teil1.tex -- Beispiel-File für das Paper
%
% (c) 2020 Prof Dr Andreas Müller, Hochschule Rapperswil
%
\section{Was ist die ungarische Methode?
\label{munkres:section:teil1}}
\rhead{Problemstellung}
Es ist ein kombinatorischer Optimierungsalgorithmus, der das Zuordnungsproblem
in polynomieller Zeit löst.
\begin{itemize}
\item
Polynom = vielgliedrig
\end{itemize}
Der Begriff polynomielle Laufzeit bedeutet, dass die Laufzeit des Programms
wie $n^2$, $n^3$, $n^4$, etc.~wächst und vernünftig skaliert.
Mit der ungarischen Methode können also lineare Optimierungsprobleme gelöst
werden, die bei gewichteten Zuordnungen in bipartiten Graphen entstehen.
Mit ihr kann die eindeutige Zuordnung von Objekten aus zwei Gruppen so
optimiert werden, dass die Gesamtkosten minimiert werden bzw.~der
Gesamtgewinn maximiert werden kann. 
