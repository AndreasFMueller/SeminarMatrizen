%
% einleitung.tex -- Beispiel-File für die Einleitung
%
% (c) 2020 Prof Dr Andreas Müller, Hochschule Rapperswil
%
\section{Geschichte\label{munkres:section:teil0}}
\rhead{Geschichte}
Die Ungarische Methode wurde 1955 von Harold Kuhn entwickelt und veröffentlicht.
Der Name ``Ungarische Methode'' ergab sich, weil der Algorithmus
weitestgehend auf den früheren Arbeiten zweier ungarischer Mathematiker
basierte: Dénes Kőnig und Jenő Egerváry.
James Munkres überprüfte den Algorithmus im Jahr 1957 und stellte fest,
dass der Algorithmus (stark) polynomiell ist.
Seitdem ist der Algorithmus auch als Kuhn-Munkres oder
Munkres-Zuordnungsalgorithmus bekannt.
Die Zeitkomplexität des ursprünglichen Algorithmus war $O(n^4)$,
später wurde zudem festgestellt, dass er modifiziert werden kann,
um eine  $O(n^3)$-Laufzeit zu erreichen.



