%
% teil4.tex -- Beispiel-File für Teil 4
%
% (c) 2020 Prof Dr Andreas Müller, Hochschule Rapperswil
%
\section{Matrix-Interpretation
\label{munkres:section:teil4}}
\rhead{Matrix-Interpretation}
Gegeben ist die quadratische Matrix $C=(c_{ij})$ der Grösse $n\times n$.
Ohne Beschränkung der Allgemeinheit werden eine Zuordnung $j
\rightarrow  s_j$, $j = 1, \dots, n$ mit minimaler Gesamtsumme
$\sum_{j=1}^{n}c_{s_j,j}$ gesucht, wobei die $s_j$ eine Permutation
von $\{1,\ldots ,n\}$ sind.
Soll die Summe maximiert werden, dann kann $C$ durch $-C$ ersetzt werden.
Die Grundlage dieses Verfahrens ist, dass sich die optimale Zuordnung
unter bestimmten Änderungen der Matrix nicht ändert, sondern nur
der Optimalwert.
Diese Änderungen sind durch Knotenpotentiale bzw.~duale Variablen 
\begin{equation}
u_1 u_2,{\dots}, u_n
\end{equation}

für die Zeilen und 

\begin{equation}v_1,v_2,\dots,v_n \end{equation} fuer die Spalten angegeben.
Die modifizierte Matrix hat dann die Komponenten $\tilde{c}_{i,j}
= c_{ij} - u_j - v_j$.

In der Summe über jede kantenmaximale Zuordnung kommt jedes
Knotenpotential genau einmal vor, so dass die Änderung der Zielfunktion
eine Konstante ist.
Sind die Einträge von $C$ nichtnegativ, und sind alle Knotenpotentiale
ebenfalls nichtnegativ, so nennt man die modifizierte Matrix \~{C}
auch eine Reduktion.
Ziel ist, in der reduzierten Matrix möglichst viele Komponenten auf
den Wert Null zu bringen und unter diesen die Zuordnung zu konstruieren.
