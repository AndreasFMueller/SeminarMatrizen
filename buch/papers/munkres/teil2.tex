%
% teil2.tex -- Beispiel-File für teil2 
%
% (c) 2020 Prof Dr Andreas Müller, Hochschule Rapperswil
%
\section{Schwierigkeit der Lösung (Permutationen)
\label{munkres:section:teil2}}
\rhead{Schwierigkeit der Lösung (Permutationen)}

Eine Permutation ist eine Anordnung von Objekten in einer bestimmten Reihenfolge oder eine Umordnung von Objekten aus einer vorgegebenen Reihung. Ist eine maximale Zuordnung (maximales Matching) gefunden, so steht in jeder Zeile und jeder Spalte der Matrix genau ein Element, das zur optimalen Lösung gehört, eine solche Gruppe von Positionen wird auch als Transversale der Matrix bezeichnet. 

Die Problemstellung kann auch so formuliert werden, dass man die Zeilen- oder die Spaltenvektoren so umordnet soll, dass die Summe der Elemente in der Hauptdiagonale maximal wird. Hieraus wird sofort ersichtlich, dass es in einer n×n-Matrix genau so viele Möglichkeiten gibt, die Zeilen- bzw. Spaltenvektoren zu ordnen, wie es Permutationen von n Elementen gibt, also n!. Außer bei kleinen Matrizen ist es nahezu aussichtslos, die optimale Lösung durch Berechnung aller Möglichkeiten zu finden. Schon bei einer 10×10-Matrix gibt es nahezu 3,63 Millionen (3.628.800) zu berücksichtigender Permutationen.

