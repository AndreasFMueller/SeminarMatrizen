%
% teil3.tex -- Beispiel-File für Teil 3
%
% (c) 2020 Prof Dr Andreas Müller, Hochschule Rapperswil
%
\section{Der Algorithmus in Form von bipartiten Graphen
\label{munkres:section:teil3}}
\rhead{Der Algorithmus in Form von bipartiten Graphen}
Mit der ungarischen Methode können also lineare Optimierungsprobleme
gelöst werden, die bei gewichteten Zuordnungen in bipartiten Graphen
entstehen.

Mit ihr kann die eindeutige Zuordnung von Objekten aus zwei Gruppen
so optimiert werden, dass die Gesamtkosten minimiert werden bzw.~der
Gesamtgewinn maximiert werden kann.

Ein bipartiter Graph ist ein mathematisches Modell für Beziehungen
zwischen den Elementen zweier Mengen.
Es eignet sich sehr gut zur Untersuchung von Zuordnungsproblemen» 

\subsection{Beweis, dass der Algorithmus Fortschritte macht
\label{munkres:subsection:malorum}}
Wir müssen zeigen, dass der Algorithmus, solange das Matching nicht
die maximal mögliche Größe hat, immer in der Lage ist, Fortschritte
zu machen --- das heißt, entweder die Anzahl der übereinstimmenden
Kanten zu erhöhen oder mindestens eine Kante zu straffen.
Es genügt zu zeigen, dass bei jedem Schritt mindestens eine der
folgenden Bedingungen erfüllt ist:

\begin{itemize}
\item
$M$ die maximal mögliche Größe.
\item
$Gy$ enthält einen Erweiterungspfad.
\item
$G$ enthält einen losen Pfad: einen Pfad von einem Knoten in $Rs$
zu einem Knoten in $T$ / $Z$ die aus einer beliebigen Anzahl von
festen Kanten, gefolgt von einer einzelnen losen Kante, besteht.
Die freie Kante einer freien Bahn ist also $Z$ (beinhaltet $T$),
so garantiert es, dass Delta gut definiert ist.
\end{itemize}
Wenn $M$ die maximal mögliche Größe hat, sind wir natürlich fertig.
Andernfalls muss es nach Berges Lemma im zugrundeliegenden Graphen
$G$ einen Augmentierungspfad $P$ in Bezug auf $M$ geben.
Dieser Pfad darf jedoch nicht in $G_y$ existieren: Obwohl jede
geradzahlige Kante in $P$ durch die Definition von $M$ fest ist,
können ungeradzahlige Kanten lose sein und in $G_y$ fehlen.
Ein Endpunkt von $P$ liegt in $R_{S}$, der andere in $R_T$; w.l.o.g.,
nehmen Sie an, es beginnt in $R_{S}$.
Wenn jede Kante von $P$ dicht ist, dann bleibt sie ein augmentierender
Pfad in $G_y$ und wir sind fertig.
Andernfalls sei $uv$ die erste lose Kante auf $P$.
Wenn $v$ kein Element von $Z$ ist, dann haben wir einen losen Pfad
gefunden und sind fertig.
Andernfalls ist $v$ von irgendeinem anderen Pfad $Q$ aus festen
Kanten von einem Knoten in $R_{S}$ erreichbar.
Sei $P_{v}$ der Teilpfad von $P$, der bei $v$ beginnt und bis zum
Ende reicht, und sei $P'$ der Pfad, der gebildet wird, indem man
entlang $Q$ gebildet wird, bis ein Scheitelpunkt auf $P_{v}$ erreicht
wird, und dann weiter bis zum Ende von $P_{v}$.
Beachten Sie, dass $P'$ ein erweiternder Pfad in $G$ mit mindestens
einer losen Kante weniger als $P$ ist.
$P$ kann durch $P'$ ersetzt und dieser Argumentationsprozess iteriert
werden (formal, unter Verwendung von Induktion auf die Anzahl der
losen Kanten), bis entweder ein erweiternder Pfad in $G_y$ oder ein
losender Pfad in $G$ gefunden wird.

\subsection{Beweis, dass die Anpassung des Potentials $y$ $M$ unverändert lässt
\label{munkres:subsection:malorum}}
Um zu zeigen, dass jede Kante in $M$ nach der Anpassung von $y$
erhalten bleibt, genügt es zu zeigen, dass für eine beliebige Kante
in $M$ entweder beide Endpunkte oder keiner von ihnen in $Z$ liegen.
Zu diesem Zweck sei $vu$ eine Kante in $M$ von $T$ nach $S$.
Es ist leicht zu sehen, dass wenn $v$ in $Z$ ist, dann muss auch
$u$ in $Z$ sein, da jede Kante in $M$ dicht ist.
Nehmen wir nun an, dass $u$ kein Element von $Z$ und auch $v$ kein
Element von $Z$ ist.
$u$ selbst kann nicht in $R_{S}$ sein, da es der Endpunkt einer
angepassten Kante ist, also muss es einen gerichteten Pfad von engen
Kanten von einem Knoten in $R_{S}$ zu $u$ geben.
Dieser Pfad muss $v$ vermeiden, da es per Annahme nicht in $Z$ ist,
also ist der Knoten, der $u$ in diesem Pfad unmittelbar vorausgeht,
ein anderer Knoten $v$ (ein Element von $T$) und $v$ ein Element
von $u$ ist eine enge Kante von $T$ nach $S$ und ist somit in $M$.
Aber dann enthält $M$ zwei Kanten, die den Knoten $u$ teilen, was
der Tatsache widerspricht, dass $M$ ein Matching ist.
Jede Kante in $M$ hat also entweder beide Endpunkte oder keinen
Endpunkt in $Z$.

\subsection{Beweis, dass $y$ ein Potential bleibt
\label{munkres:subsection:malorum}}
Um zu zeigen, dass y nach der Anpassung ein Potenzial bleibt, genügt
es zu zeigen, dass keine Kante ihr Gesamtpotenzial über ihre Kosten
hinaus erhöht.
Dies ist für Kanten in $M$ bereits durch den vorangegangenen Absatz
bewiesen.
Man betrachtet also eine beliebige Kante $uv$ von $S$ nach $T$.
Wenn $y(u)$ erhöht wird um $\Delta$, dann wird entweder $v\in
\mathbb{Z}_n$ in diesem Fall wird $y(v)$ verringert um $\Delta$,
wobei das Gesamtpotenzial der Kante unverändert bleibt, oder $v\in
T\setminus Z$, wobei die Definition von $\Delta$ garantiert, dass
$y(u)+y(v)+\Delta \le c(u,v)$
Also $y$ bleibt ein Potential.

