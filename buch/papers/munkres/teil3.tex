%
% teil3.tex -- Beispiel-File für Teil 3
%
% (c) 2020 Prof Dr Andreas Müller, Hochschule Rapperswil
%
\section{Der Munkres-Algorithmus (Ungarische Methode)
\label{munkres:section:teil3}}
\rhead{Der Munkres-Algorithmus (Ungarische Methode)}

Mit der ungarischen Methode können also Optimierungsprobleme gelöst
werden, die bei gewichteten Zuordnungen in bipartiten Graphen entstehen.
Mit ihr kann die eindeutige Zuordnung von Objekten aus zwei Gruppen so
optimiert werden, dass die Gesamtkosten minimiert werden bzw.~der
Gesamtgewinn maximiert werden kann. 

\subsection{Geschichte
\label{munkres:subsection:malorum}}
Die Ungarische Methode wurde 1955 von Harold Kuhn entwickelt und veröffentlicht.
Der Name ``Ungarische Methode'' ergab sich, weil der Algorithmus
weitestgehend auf den früheren Arbeiten zweier ungarischer Mathematiker
basierte: Dénes Kőnig und Jenő Egerváry.
James Munkres überprüfte den Algorithmus im Jahr 1957 und stellte fest,
dass der Algorithmus (stark) polynomiell ist.
Seitdem ist der Algorithmus auch als Kuhn-Munkres oder
Munkres-Zuordnungsalgorithmus bekannt.
Die Zeitkomplexität des ursprünglichen Algorithmus war $O(n^4)$,
später wurde zudem festgestellt, dass er modifiziert werden kann,
um eine  $O(n^3)$-Laufzeit zu erreichen.

\subsection{Besondere Leistung der Ungarischen Methode
\label{munkres:subsection:malorum}}
Die Ungarische Methode ist ein kombinatorischer Optimierungsalgorithmus, der das Zuordnungsproblem
in polynomieller Zeit löst.
Der Begriff polynomielle Laufzeit bedeutet, dass die Laufzeit des Programms
wie $n^2$, $n^3$, $n^4$, etc.~wächst und vernünftig skaliert. $n$ ist hierbei die "Grösse" des Problems.

\subsection{Unterschiedliche Anzahl von Quellen und Zielen
\label{munkres:subsection:malorum}}
Es gibt Fälle, in welchen das Ausgangsproblem keine quadratische Form besitzt. Das ist z.B. dann der Fall, wenn drei Mitarbeiter vier verschiedene Eignungstests absolvieren müssen. In diesem Fall wird in der Ungarischen Methode die Matrix künstlich mittels einer Dummy Position quadratisch ergänzt. Dummy-Positionen werden dann mit der größten vorhandenen Zahl aus der Matrix besetzt. Beispielsweise wird eine $3\times 4$ zu einer $4\times 4$-Matrix.

\subsection{Beispiel eines händischen Verfahrens
\label{munkres:subsection:malorum}}

Die ungarische Methode kann in einem einfachen händischen Beispiel erläutert werden. Es gibt eine Ausgangsmatrix. Diese Matrix wird in mehreren Schritten immer weiter reduziert. Anschließend erfolgen mehrere Zuordnungen. Hierbei ist zu beachten, dass jede Zeile und jede Spalte immer genau eine eindeutige Zuordnung ergibt. Die optimale Lösung ist erreicht, wenn genau $n$ Zuordnungen gefunden sind. Das Vorgehen wird in den nachfolgenden Schritten 1-16 beschrieben und auch in der Abbildung 21.5 dargestellt. 

\begin{enumerate}
\item Pro Zeile eruiert man die kleinste Zahl. Diese kleinste Zahl wird bei
allen anderen Ziffern in der jeweiligen Zeile subtrahiert. Mit dieser Subtraktion zieht man die unvermeidbaren Kosten ab, die man hat, um eine Baustelle zu erreichen. 

\item Auch in diesem Schritt werden die unvermeidbaren Weg-Kosten abgezogen. Man zieht die kleinste Zahl in jeder Spalte von allen Zahlen in der Spalte ab.

\item Bei den nachfolgenden Schritten bleiben dann nur noch die Kosten übrig, die man hat, wenn man eine andere Zuordnung wählt. Hierbei sollen möglichst viele Nullen markiert werden, welche freistehend sind.
(Freistehend bedeutet, sowohl in der jeweiligen Zeile und Spalte nur
eine markierte Null zu haben)

\item Weiter werden die jeweiligen Zeilen eruiert, bei welchen keine markierte Null vorhanden sind. Diese kennzeichnet man mit einer blauen Fläche.

\item In der vorherigen, mit blauer Fläche markierten Zeile die 0 eruieren und dann die dazugehörige Spalte ebenfalls
blau markieren.

\item Im der selben Spalte die markierte Null eruieren und die dazugehörige
Zeile ebenfalls blau kennzeichnen.

\item Alle Zeilen mit einem gelben Balken durchstreichen, welche KEINE blauen Markierungen haben.

\item Alle Spalten durchstreichen, welche eine Blaue Markierung besitzt!

\item In den übrigen Zahlen soll nun die kleinste Ziffer ausgewählt werden, welche nicht schon durchgestrichen sind.
(Im Beispiel ist es die Zahl 1 in rot markiert. (Bei diesem Schritt ist es egal, welche 1 man wählt)

\item Die eruierte kleinste Ziffer, wird von den nicht durchgestrichenen Ziffern
subtrahiert. Danach muss die Matrix wieder komplettiert werden. (inkl. Unterstreichen der Nullen)

\item Jeweilige Zahlen eruieren, welche vorgängig doppelt mit einer gelben Fläche durchgestrichen wurden.

\item Kleinste eruierte Ziffer aus Schritt 9, soll nun auf die zwei in rot markierten Ziffern aus Schritt 11 dazu addiert werden. 

\item In diesem Schritt sollen wiederum von neuem möglichst viele Nullen markiert werden,
welche freistehend sind. Es werden nur die markierten Nullen betrachtet.

\item Alle markierten Nullen werden jetzt in eine 1 umgewandelt.

\item Die restlichen Ziffern in der Matrix, exklusiv die einsen, sollen jetzt ignoriert und durch eine Null ersetzt werden.

\item Zu guter Letzt werden überall wo eine 1 steht, die Zahlen aus der Ausgangsmatrix eingefügt. Nach Einsetzen der Zahlen können die in rot markierten Zahlen aufsummiert werden. Es ergibt der minimalste Transportweg. Im erwähnten Beispiel sind es total 13 Kilometer.
\end{enumerate}

\begin{figure}
\centering
\includegraphics[width=14cm]{papers/munkres/figures/Ungarische_Methode_Beispiel.png}
\caption{Händisches Beispiel des Munkres Algorithmus, minimalster Transportweg.}
\label{munkres:Vr2}
\end{figure}

\subsection{Zuordnung der Kräne
\label{munkres:subsection:malorum}}

\begin{itemize}
\item Der Kran von Baustelle A1 soll zur Baustelle B2.
\item Der Kran von Baustelle A2 soll zur Baustelle B3.
\item Der Kran von Baustelle A3 soll zur Baustelle B4.
\item Der Kran von Baustelle A4 soll zur Baustelle B1.
\end{itemize}

\begin{figure}
\centering
\includegraphics[width=3cm]{papers/munkres/figures/Ungarische_Methode_Beispiel_Zuw.png}
\caption{Händisches Beispiel des Munkres Algorithmus, Zuweisung der Kräne }
\label{munkres:Vr2}
\end{figure} Wie in Abbildung 21.6 ersichtlich, kann somit dank der Ungarischen Methode sowohl der minimalste Transportweg als auch die optimalste Zuweisung der Kräne auf die neuen Standorte ermittelt werden.