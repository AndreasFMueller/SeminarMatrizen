%
% teil1.tex -- Beispiel-File für das Paper
%
% (c) 2020 Prof Dr Andreas Müller, Hochschule Rapperswil
%
\section{Fraktale
\label{ifs:section:teil1}}
\rhead{Problemstellung}
Bevor wir die IFS genauer ansehen, schauen wir uns Fraktale genauer an.

\begin{equation}
\int_a^b x^2\, dx
=
\left[ \frac13 x^3 \right]_a^b
=
\frac{b^3-a^3}3.
\label{ifs:equation1}
\end{equation}
Neque porro quisquam est, qui dolorem ipsum quia dolor sit amet,
consectetur, adipisci velit, sed quia non numquam eius modi tempora
incidunt ut labore et dolore magnam aliquam quaerat voluptatem.

Ut enim ad minima veniam, quis nostrum exercitationem ullam corporis
suscipit laboriosam, nisi ut aliquid ex ea commodi consequatur?
Quis autem vel eum iure reprehenderit qui in ea voluptate velit
esse quam nihil molestiae consequatur, vel illum qui dolorem eum
fugiat quo voluptas nulla pariatur?

\subsection{De finibus bonorum et malorum
\label{ifs:subsection:finibus}}
At vero eos et accusamus et iusto odio dignissimos ducimus qui
blanditiis praesentium voluptatum deleniti atque corrupti quos
dolores et quas molestias excepturi sint occaecati cupiditate non
provident, similique sunt in culpa qui officia deserunt mollitia
animi, id est laborum et dolorum fuga \eqref{000tempmlate:equation1}.

Et harum quidem rerum facilis est et expedita distinctio
\ref{ifs:section:loesung}.
Nam libero tempore, cum soluta nobis est eligendi optio cumque nihil
impedit quo minus id quod maxime placeat facere possimus, omnis
voluptas assumenda est, omnis dolor repellendus
\ref{ifs:section:folgerung}.
Temporibus autem quibusdam et aut officiis debitis aut rerum
necessitatibus saepe eveniet ut et voluptates repudiandae sint et
molestiae non recusandae.
Itaque earum rerum hic tenetur a sapiente delectus, ut aut reiciendis
voluptatibus maiores alias consequatur aut perferendis doloribus
asperiores repellat.


