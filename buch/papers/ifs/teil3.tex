%
% teil3.tex -- Beispiel-File für Teil 3
%
% (c) 2020 Prof Dr Andreas Müller, Hochschule Rapperswil
%
\section{Fraktale Bildkomprimierung
\label{ifs:section:teil3}}
\rhead{Fraktale Bildkomprimierung}
Mit dem Prinzip dieser IFS ist es auch möglich Bilder zu Komprimieren.
Diese Idee hatte der Mathematiker Michael Barnsley, welcher mit seinem Buch Fractals Everywhere einen wichtigen beitrag zum verständnis von Fraktalen geiefert hat.
Das Ziel ist es ein IFS zu finden, welches das Bild als Attraktor hat.
In diesem Unterkapitel wollen wir eine Methode dafür anschauen.

\subsection{Titel
\label{ifs:subsection:malorum}}
Bis jetzt wurde in Zusammenhnag mit IFS immer erwähnt, dass die Transformationen auf die ganze Menge angewendet werden.
Dies muss jedoch nicht so sein. 
Es gibt auch einen Attraktor, wenn die Transformationen nur Teile der Menge auf die ganze Menge abbilden.
Diese Eigenschaft wollen wir uns in der Fraktalen Bildkompression zunutze machen.
Sie ermöglicht uns Ähnlichkeiten zwischen kleineren Teilen des Bildes zunutze machen.
Es ist wohl nicht Falsch zu sagen, dass Ähnlichkeiten zur gesamten Menge, wie wir sie zum Beispiel beim Barnsley Fern gesehen haben, bei Bilder aus dem Alltag eher selten anzutreffen sind.
Doch wie Finden wir die richtigen Affinen Transformationen, welche als IFS das Bild als Attraktor haben.




