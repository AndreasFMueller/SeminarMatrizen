%
% uebersicht.tex -- Uebersicht ueber die Seminar-Arbeiten
%
% (c) 2021 Prof Dr Andreas Mueller, OST Ostschweizer Fachhochschule
%
\chapter*{Übersicht}
\lhead{Übersicht}
\rhead{}
\label{buch:uebersicht}
Im zweiten Teil kommen die Teilnehmer des Seminars selbst zu Wort.
Die im ersten Teil dargelegten mathematischen Methoden und
grundlegenden Modelle werden dabei verfeinert, verallgemeinert
und auf vielfältige Weise angewandt.

Den Anfang machen {\em Robine Luchsinger} und {\em Pascal Andreas Schmid},
\index{Luchsinger, Robine}%
\index{Schmid, Pascal Andreas}%
die zeigen, wie man basierend auf der Adjazenzmatrix Suchalgorithmen
für Netzwerke aufbauen kann.
Sie konzentrieren sich dabei auf Verkehrsnetze, die die zusätzliche
Eigenschaft haben, eine geometrische Realsierung zu besitzen.
Diese führt zu zusätzlichen Einschränkungen, die einen positiven
Einfluss auf die Effizienz der Suchalgorithmen haben können.

Die naive Umsetzung der Definition der Matrizenmultiplikation in
ein Coputerprogramm ist nicht unbedingt die effizienteste.
{\em Michael Schmid} stellt die Algorithmen von Strassen und
\index{Schmid, Michael}%
Windograd vor, welche ermöglichen, die Laufzeitkomplexität
von $O(n^3)$ auf $O(n^{2.8074})$ oder noch schneller zu verbessern.
Allerdings nur unter gewissen Voraussetzungen, die im Paper
ebenfalls diskutiert werden.

Eine der schönsten Anwendungen der Gruppentheorie ist die
Kristallographie.
{\em Naoki Pross} und {\em Tim Tönz} zeigen, wie man mit ihrer
\index{Pross, Naoki}%
\index{Tönz, Tim}%
Hilfe Kristalle klassifizieren kann, und sie illustrieren am Beispiel
der Piezoelektrizität, dass man auch physikalische Eigenschaften daraus
ableiten kann.

Der Reed-Solomon-Code ist ein Klassiker unter den fehlerkorrigierenden
Codes.
Berühmt gemacht durch seine Anwendung in den Voyager-Sonden und in CDs
und DVDs, begegnet er uns heute auch in den allgegenwärtigen QR-Codes.
Ein ganzes Arsenal von algebraischen Methoden ist nötig, um seine
Funktionsweise zu verstehen.
{\em Joshua Bär} und {\em Michael Steiner} zeigen in vielen Einzelschritten,
\index{Bär, Joshua}%
\index{Steiner, Michael}%
wie die man die einzelnen Ideen an vertrauteren Beispielen aus der
elementaren Algebra und der Fourier-Theorie verstehen kann.
Die Übertragung in einen Polynomring über einem endlichen Körper 
ist dann nicht mehr schwierig.
Die Analogie wird deutlich, wenn man das Codierungsverfahren und
die diskrete Fourier-Transformation beide als Matrizen schreibt.

Wer glaubt, mit linearen Abbildungen lassen sich nur gradlinige
Objekte beschreiben, liegt völlig falsch.
Die Arbeit von {\em Alain Keller} zeigt, dass die Iteration von
\index{Keller, Alain}%
affinen Abbildungen hochkomplexe Fraktale hervorbringen kann.
Solche iterierten Funktionsschemata erzeugen aber nicht nur schöne
Bilder, man kann daraus auch eine Idee zur Kompression von
Bildern ableiten.

Es gibt zwar noch keine ernstzunehmenden Quantencomputer, aber man weiss
bereits, dass ein leistungsfähriger Quantencomputer viele der heute
im Internet üblichen Verschlüsselungsverfahren, allen voran das RSA-Verfahren,
brechen könnte.
Das McEliece-Kryptosystem kombiniert verschiedene Arten von Matrizen
mit zufälligem Rauschen und einem fehlerkorrigierenden Code.
Wie {\em Reto Fritsche} erklärt, kommt dabei ein Verschlüsselungsverfahren
\index{Fritsche, Reto}%
heraus, welches nach heutigem Wissensstand gegen Angriffe mit
Quantencomputern resistent ist.

Vektoren und Matrizen bilden die Basis vieler geometrischer
Anwendungen.
Doch ist die Beschreibung von Bewegungen im Raum mit Matrizen nicht
immer einfach.
In der Ebene kann man die komplexen Zahlen als Modell verwenden,
wo Drehungen und Translationen durch einfache arithmetische
Operationen mit Zahlen beschrieben werden können.
{\em Marius Baumann} und {\em Thierry Schwaller} tauchen in die
\index{Baumann, Marius}%
\index{Schwaller, Thierry}%
geometrische Algebra ein, welche diese Idee verallgemeinert.
Sie illustrieren, wie sich mit geometrischer Algebra Bewegungen
in $\mathbb{R}^n$ einfach beschreiben lassen.
So gibt es zum Beispiel ein Euler-Formel, für Drehungen und Spiegelungen
kann die selbe Abbildungsformel verwendet werden und die Zusammensetzung
von Transformationen ist eine Multiplikation in einer Algebra, die
aus den Vektoren konstruiert worden ist.
In drei Dimensionen ist diese Algebra der Quaternionen zum Beispiel
in der Computergraphik sehr beliebt.

Man soll sein Haus nicht auf Sand bauen, sagt eine Redensart.
Etwas mathematischer heisst das, dass man den Spannungszustand,
der von einem Gebäude im darunterliegenden Boden aufgebaut wird,
im Detail verstehen und modellieren können sollte.
Dazu muss man erst eine geeignete Darstellung finden.
{\em Thomas Reichlin} und {\em Adrian Schuler} zeigen, wie man 
\index{Reichlin, Thomas}%
\index{Schuler, Adrian}%
dazu eigentlich über die Welt der Matrizen hinaus gehen muss und
sich mit sogenannten Tensoren herumschlagen muss.
Dank sinnvollen Annahmen über die reale Situation im Boden
kann man aber auf eine handlichere Beschreibung zurückkommen,
die wieder nur Matrizen verwendet.
Ausserdem ergeben sich daraus spezielle Blockstrukturen der
Matrizen.

Ein Erdbeben versetzt alles auf der Erdoberfläche in Bewegung und
kann beträchtliche Schäden anrichten.
Daher wird die seismische Aktivität weiltweit überwacht.
Ein Seismograph enthält eine schwingungsfähige Masse, die die Bewegungen
aufzeichen kann.
Doch welcher Teil der aufgezeichneten Bewegung kommt vom Erdbeben
und welcher Teil ist Eigenschwingung der Messmasse?
Dieser Frage gehen {\em Fabio Viecelli} und {\em Lukas Zogg} nach.
\index{Viecelli, Fabio}%
\index{Zogg, Lukas}%
Die Antwort gelingt mit einem Klassiker unter den Ingenieur-Methoden:
dem Kalman-Filter.
Die Autoren stellen die für den Filter nötigen Matrizen zusammen
und illustrieren mit Hilfe von Simulationen, wie er funktioniert.

Eine Matrix kann dazu verwende werden, die Kosten zusammenzustellen,
die die Lösung einzelner Aufgaben durch verschiedene Anbieter
verursachen würden.
Doch wie findet man jetzt diejenige Zuteilung der Aufgaben
zu den Anbietern, die die Gesamtkosten minimiert.
Für dieses klassische Zuordnungsproblem ist die
von {\em Marc Kühne} beschriebene ungarische Methode,
\index{Kühne, Marc}%
auch als Munkres-Algorithmus bekannt, eine besonders effiziente
Lösung.



