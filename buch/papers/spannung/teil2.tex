\section{Dreiachsiger Spannungszustand\label{spannung:section:Dreiachsiger Spannungszustand}}
\rhead{Proportionalität Spannung-Dehnung}
Wie im Kapitel Spannungsausbreitung beschrieben herrscht in jedem Punkt ein anderer Spannungszustand.
Um die Spannung im Boden genauer untersuchen zu können für man einen infinitesimalen Würfel ein.
\begin{figure}
	\centering
	\includegraphics[width=0.5\linewidth,keepaspectratio]{papers/spannung/Grafiken\infinitesimalerWürfel.jpg}
	\caption{infinitesimaler Würfel}
	\label{fig:infintesimaler-wurfel}
\end{figure}

Sobald eine Kraft von oben wirkt hat man auch Kräfte die seitlich wirken.

An diesem infinitesimalen Würfel hat man ein räumliches Koordinatensystem, die Achsen (1,2,3).
Jede dieser 6 Flächen dieses Würfels hat damit 3 Pfeile.
Geschrieben werden diese mit $\sigma$ mit jeweils zwei Indizes gibt.
Die Indizes geben uns an, in welche Richtung der Pfeil zeigt.
Zur Notation wird die Voigt`sche Notation benutzt. Das sieht wie folgt aus:

\[
\overline{\sigma}
=
\left[ \begin{array}{rrr}
	\sigma_{11} & \sigma_{12} & \sigma_{13} \\ 
	\sigma_{21} & \sigma_{22} & \sigma_{23} \\
	\sigma_{31} & \sigma_{32} & \sigma_{33} \\ 
\end{array}\right]
=
\left[ \begin{array}{rrr}
	\sigma_{11} & \sigma_{12} & \sigma_{13} \\ 
	 & \sigma_{22} & \sigma_{23} \\
	sym &  & \sigma_{33} \\ 
\end{array}\right]
\Rightarrow
\overrightarrow{\sigma}
=
\left(\begin{array}{c}\sigma_{11}\\\sigma_{22}\\\sigma_{33}\\\sigma_{23}\\\sigma_{13}\\\sigma_{12}\end{array}\right)
\]

Voigt`sche Notation besagt, dass man diesen Spannungstensor als Vektor aufschreiben darf.
Die Reihenfolge folgt der Regel von Ecke links oben, diagonal zur Ecke rechts unten.
Danach ist noch $\sigma_{23}$, $\sigma_{13}$ und $\sigma_{12}$ aufzuschreiben.

Eine weitere Besonderheit ist die Symmetrie der Matrix.

?????Was könnte man hier noch zu den Pfeilen erklären vom Würfel???????



