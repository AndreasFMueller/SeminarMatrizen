\section{Proportionalität Spannung-Dehnung\label{spannung:section:Proportionalität Spannung-Dehnung}}
\rhead{Proportionalität Spannung-Dehnung}
Das Hook'sche Gesetz beschreibt die elastische Längenänderung von Festkörpern im Zusammenhang mit einer Krafteinwirkung.
Die Längenänderung $\Delta l$ ist proportional zur Krafteinwirkung $F$.
\[
F
\sim
\Delta l
\]
Man kann dies nur im Bereich vom linearen-elastischen Materialverhalten anwenden.
Das heisst, dass alle Verformungen reversibel sind, sobald man die Kraft wegnimmt.
Es findet somit keine dauernde Verformung statt.
Da es sehr praktisch ist die Längenänderung nicht absolut auszudrücken haben wir $\varepsilon$.
Die Dehnung $\varepsilon$ beschreibt die relative Längenänderung.
Die Dehnung $\varepsilon$ ist wiederum proportional zu der aufgebrachten Spannung.
Im Bauingenieurwesen hat man es oft mit grösseren Teilen oder grösseren Betrachtungsräumen zu tun.
Da ist es nun natürlich sehr sinnvoll, wenn wir nicht mit absoluten Zahlen rechnen,
sondern unabhängig von der Länge den Zustand mit Dehnung $\varepsilon$ beschreiben können.
Mithilfe vom E-Modul, (steht für Elastizitätsmodul) einer Proportionalitätskonstante,
kann man das in eine Gleichung bringen, wie man hier sieht. Das E-Modul beschreibt,
das Verhältnis von Kraftaufnahme eines Werkstoffes und dessen zusammenhängender Längenveränderung.
(Quelle Wikipedia)
\[
\sigma
=
E\cdot\varepsilon
\]
\[
E
=
\frac{\Delta\sigma}{\Delta\varepsilon}
=
const.
\]

Aus diesem Verhältnis kann man das E-Modul berechnen.
Je nach Material ist dies verschieden.
Das E-Modul lässt sich nur im linearen-elastischen Materialverhalten anwenden.
Für Bodenmaterial gibt es ein spezielles E-Modul. Dieses wird mit dem Oedometer-Versuch ermittelt.
Es wird mit $E_{OED}$ ausgedrückt. Dieser Versuch wird später noch beschrieben.
Der Oedometer-Versuch ist abhängig von den diesem Kapitel zu untersuchenden Matrizen.