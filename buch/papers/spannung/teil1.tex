\section{Skalare, Vektoren, Matrizen und Tensoren\label{spannung:section:Skalare,_Vektoren,_Matrizen_und_Tensoren}}
\rhead{Skalare, Vektoren, Matrizen und Tensoren}
Tensoren wurden als erstes in der Elastizitätstheorie eingesetzt. (Quelle Herr Müller)
In der Elastizitätstheorie geht es darum viele verschiedene Komponenten zu beschreiben.
Mit einer Matrix oder einem Vektor kann man dies nicht mehr bewerkstelligen.
Wenn man den dreidimensionalen Spannungszustand abbilden möchte, müsste man mehrere Vektoren haben.
Deshalb wurden 1840 von Rowan Hamilton Tensoren in die Mathematik eingeführt.
Woldemar Voigt hat den Begriff in die moderne Bedeutung von Skalar, Matrix und Vektor verallgemeinert.
Albert Einstein hat Tensoren zudem in der allgemeinen Relativitätstheorie benutzt.
Tensor sind eine Stufe höher als Matrizen. Matrizen sind 2. Stufe.
Da Tensoren eine Stufe höher sind, kann man auch Matrizen, Vektoren und Skalare als Tensoren bezeichnen.
Der Nachteil von den Tensoren ist, dass man die gewohnten Rechenregeln, die man bei Vektoren oder Matrizen kennt,
nicht darauf anwenden kann. Man ist deshalb bestrebt die Tensoren als Vektoren und Matrizen darzustellen,
damit man die gewohnten Rechenregeln darauf anwenden kann. (Quelle Wikipedia)
In der vorliegenden Arbeit sind bereits alle Tensoren als Matrizen 2. Stufe abgebildet.
Trotzdem kann man diese Matrizen wie vorher beschrieben als Tensor bezeichnen.
Da diese als Matrizen abgebildet sind, dürfen wir die bekannten Rechenregeln auf unsere Tensoren anwenden.