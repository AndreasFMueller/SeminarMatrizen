\section{Einleitung\label{spannung:section:Einleitung}}
In diesem Kapitel geht es darum die Matrix im dreidimensionalen Spannungszustand genauer zu untersuchen.
In der Geotechnik wendet man solche Matrizen an, um Spannungen im Boden zu berechnen.
Mit diesen Grundlagen dimensioniert man beispielsweise Böschungen, Fundationen, Dämme und Tunnels.
Ebenfalls benötigt man diese Matrix, um aus Versuchen Kennzahlen über den anstehenden Boden zu gewinnen.
Besonderes Augenmerk liegt dabei auf dem Oedometer - Versuch.

Bei dieser Untersuchung der zugehörigen Berechnungen hat man es mit Vektoren, Matrizen und Tensoren zu tun.
Um die mathematische Untersuchung vorzunehmen, beschäftigt man sich zuerst mit den spezifischen Gegebenheiten und Voraussetzungen.
Ebenfalls gilt es ein paar wichtige Begriffe und deren mathematisches Zeichen einzuführen,
damit sich den Berechnungen schlüssig folgen lässt.

In diesem Kapitel hat man es insbesondere mit Spannungen und Dehnungen zu tun.
Mit einer Spannung ist hier jedoch keine elektrische Spannung gemeint,
sondern eine Kraft geteilt durch Fläche.

\section{Einführung wichtige Begriffe\label{spannung:section:Wichtige Begriffe}}
\[
\l
=
Ausgangslänge\enspace[m]
\]
\[
\Delta l
=
Längenänderung\enspacenach\enspaceKraftauftrag\enspace[m]
\]
\[
\varepsilon
=
Dehnung\enspace[-]
\]
\[
\sigma
=
Spannung\enspace[kPa]
\]
\[
E
=
Elastizitätsmodul
\]
\[
F
=
Kraft\enspace[kN]
\]
\[
A
=
Fläche\enspace[m^2]
\]
\[
t
=
Tiefe\enspace[m]
\]
\[
s
=
Setzung,\enspaceAbsenkung\enspace[m]
\]

Beziehungen
\[
\varepsilon
=
\frac{\Delta l}{l_0}
\]
\[
\varepsilon_q
=
\frac{\Delta b}{l_0}
=
\varepsilon_\upsilon
\]
\[
\sigma
=
\frac{N}{A}
\]
\[
N
=
\int_{A} \sigma \dA
\]
\[
\varepsilon^{\prime}
=
\frac{1}{l_0}\]

