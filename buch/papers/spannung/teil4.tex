\section{Oedometer-Versuch\label{spannung:section:Oedometer-Versuch}}
\rhead{Oedometer-Versuch}
Mit dem Oedometer-Versuch kann der Oedometrische Elastizitätsmodul $E_{OED}$ bestimmt werden.
Dieser beschreibt ebenfalls das Verhältnis zwischen Spannung und Dehnung, allerdings unter anderen Bedingungen.
Diese Bedingung ist das Verhindern der seitlichen Verformung, sprich der Dehnung in Richtung $1$ und $2$.
Es wird ein Probeelement mit immer grösseren Gewichten belastet, welche gleichmässig auf das Material drücken.
Die seitliche Verschiebung des Materials wird durch einen Stahlring verhindert.
Die Probe wird sich so steig verdichten.
Das Volumen nimmt ab und die Dehnung nimmt immer mehr zu.
Unter diesen Bedingungen wird das Oedometrische E-Modul mit steigender Dehnung zunehmen.

Da im Boden das umgebende Material ähnliche eine seitliche Verformung verhindert,
gibt dieser Oedometrische E-Modul die Realität besser als der gewöhnliche E-Modul wieder.
Durch dieses Verhindern des seitlichen Ausbrechens ist
\[
\varepsilon_{22}
=
\varepsilon_{33}
=
0
\]
aber auch
\[
\sigma_{22}
=
\sigma_{33}
\neq 0
\]
Die Spannung $\sigma_{11}$ wird durch durch die aufgebrachte Kraft mit
\[
\sigma_{11}
=
\frac{F}{A}
\]
und die Dehnung $\varepsilon_{11}$ jeweils mit den entsprechenden Setzungen berechnet.
Diese Randbedingen können in die vereinfachte Gleichung eingesetzt.
Diese lautet nun:
\[
\begin{pmatrix}
	\sigma_{11}-\sigma_{33} \\
	\sigma_{11}+2\sigma_{33}
\end{pmatrix}
=
\begin{pmatrix}
	\frac{E_{OED}}{(1+\nu)} &                        0 \\
                          0 & \frac{E_{OED}}{(1-2\nu)}
\end{pmatrix}
\begin{pmatrix}
	\varepsilon_{11}\\
	\varepsilon_{11}
\end{pmatrix}
\]
.

Daraus lässt sich bei jedem Setzungsgrad das Oedometrische E-Modul $E_{OED}$ und die seitlichen Spannungen $\sigma_{33}$ mit den 2 Gleichungen

GLEICHUNGEN...

berechnen.
Den Versuch kann man auf einem $\sigma$-$\varepsilon$-Diagramm abtragen (siehe Abbildung 1.7).
Durch die Komprimierung nimmt der Boden mehr Spannung auf, und verformt sich zugleich weniger stark.
Mit diesem ermittelten $E_{OED}$ kann man nun weitere Berechnungen für die Geotechnik durchführen.

\begin{figure}
	\centering
	\includegraphics[width=0.5\linewidth,keepaspectratio]{papers/spannung/Grafiken/DiagrammOedometer-Versuch.png}
	\caption{Diagramm Oedometer-Versuch}
	\label{fig:Diagramm Oedometer-Versuch}
\end{figure}