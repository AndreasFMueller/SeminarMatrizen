\section{Spannungsausbreitung\label{spannung:section:Invarianten}}
\rhead{Invarianten}
Trotz der Vereinfachung lässt sich mit den Invarianten die Realität adäquat abbilden.
Als erste Bedingung stellt man folgendes Verhältnis auf:

\[
\sigma_{22}
=
\sigma_{33}
\]

Dies deshalb, da man von einem isotropen Bodenmaterial ausgeht.
In Achse 22, Richtung 22 hat man den gleichen Boden wie in Achse 33 und Richtung 33.
Das Verhalten bezüglich Kraftaufnahme, Dehnung Spannung ist somit dasselbe.

Man führt die zwei Werte p als hydrostatische Spannung und q als deviatorische Spannung ein.
Die Berechnung von p und q sieht wie folgt aus:

\[
p
=
\frac{\sigma_{11}+\sigma_{22}+\sigma_{33}}{3}
\]

oder durch Vereinfachung, da $\sigma_{22}=\sigma_{33}$ :

\[
p
=
\frac{\sigma_{11}+2\sigma_{33}}{3}
\]

\[
q
=
\sigma_{11}-\sigma_{33}
\]

p ist das arithmetische Mittel von der Spannung im infinitesimalen Würfel.
q ist die Differenz zwischen der Spannung in vertikaler Richtung und der Spannung in Richtung 2 und 3.
Man kann p als Druckspannung und q als Schubspannung anschauen.

Aus der Formel vom vorherigen Kapitel konnten wir die Spannungen berechnen.
Deshalb kann man nun p und q in die Gleichung einsetzen.
Die Dehnungen werden mit neuen Variablen eingeführt.
Die Deviatorische Dehnung kann mit einer Schubdehnung verglichen werden.
Die hydrostatische Dehnung kann mit einer Kompressionsdehnung verglichen werden.

\[
\overbrace{\sigma_{11}-\sigma_{33}}^{q}
=
\frac{3E}{2(1+\nu)} \overbrace{\frac{2}{3}(\varepsilon_{11} - \varepsilon_{33})}^{\varepsilon_{\nu}}
\]

\[
\overbrace{\frac{\sigma_{11}+2\sigma_{33}}{3}}^{p}
=
\frac{E}{3(1-2\nu)} \overbrace{(\varepsilon_{11} - 2\varepsilon_{33})}^{\varepsilon_{s}}
\]

\[
\varepsilon_{s}
=
Hydrostatische Dehnung [-]
\]

\[
\varepsilon_{\nu}
=
Deviatorische Dehnung [-]
\]

Diese Komponenten kann man nun in die Vereinfachte Matrix einsetzen.
Man hat dann eine Matrix multipliziert mit einem Vektor und erhält einen Vektor.

\[
\begin{pmatrix}
	q\\
	p
\end{pmatrix}
=
\begin{pmatrix}
	\frac{3E}{2(1+\nu)} &                   0 \\
	                  0 & \frac{E}{3(1-2\nu)}
\end{pmatrix}
\begin{pmatrix}
	\varepsilon_{s}\\
	\varepsilon_{\nu}
\end{pmatrix}
\]

Mit dieser Formel lassen sich verschieden Parameter von Versuchen analysieren und berechnen.
Ein solcher Versuch, den oft in der Geotechnik durchgeführt wird ist der Oedometer-Versuch.
Im nächsten Kapitel wird die Anwendung der Matrix an diesem Versuch beschrieben.
