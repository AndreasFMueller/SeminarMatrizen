\section{Die geotechnischen Invarianten\label{spannung:section:Die geotechnischen Invarianten}}
\rhead{Die geotechnischen Invarianten}
In vielen Fällen in der Geotechnik und auch in Versuchen hat man gleichmässige Belastungen über eine grössere Fläche.
Durch eine solche Belastung auf den Boden, entstehen gleichermassen Spannungen in Richtung $2$ und $3$,
wenn man von einem isotropen Bodenmaterial ausgeht.
Folglich gilt:

\[
\sigma_{22}
=
\sigma_{33}
.
\]
Dadurch wird der Spannungszustand vereinfacht.
Diesen vereinfachten Spannungszustand kann man mit den zwei geotechnischen Invarianten abbilden.
Die erste Invariante ist die volumetrische Spannung
\begin{equation}
p
=
\frac{\sigma_{11}+\sigma_{22}+\sigma_{33}}{3}
\label{spannung:Invariante_p}
,
\end{equation}
welche als arithmetisches Mittel aller Normalspannungen im infinitesimalen Würfel definiert ist.
Die zweite Invariante ist die deviatorische Spannung
\begin{equation}
q
=
\sqrt{\frac{(\sigma_{11}-\sigma_{22})^{2}+(\sigma_{11}-\sigma_{33})^{2}+(\sigma_{22}-\sigma_{33})^{2}}{2}}
\label{spannung:Invariante_q}
.
\end{equation}
Diese Zusammenhänge werden im Skript \cite{spannung:Stoffgesetze-und-numerische-Modellierung-in-der-Geotechnik} aufgezeigt.
Die hydrostatische Spannung $p$ kann gemäss Gleichung \eqref{spannung:Invariante_p} als
\[
p
=
\frac{\sigma_{11}+2\sigma_{33}}{3}
\]
vereinfacht werden.
Die deviatorische Spannung $q$ wird gemäss Gleichung \eqref{spannung:Invariante_q} als
\[
q
=
\sigma_{11}-\sigma_{33}
\]
vereinfacht. Man kann $p$ als Druck und $q$ als Schub betrachten.

Die Invarianten $p$ und $q$ können mit der Spannungsgleichung \eqref{spannung:Spannungsgleichung} berechnet werden.
Durch geschickte Umformung dieser Gleichung, lassen sich die Module als Faktor separieren.
Dabei entstehen spezielle Faktoren mit den Dehnungskomponenten.
So ergibt sich
\[
\overbrace{\frac{\sigma_{11}+2\sigma_{33}}{3}}^{\displaystyle{p}}
=
\frac{E}{3(1-2\nu)} \overbrace{(\varepsilon_{11} - 2\varepsilon_{33})}^{\displaystyle{{\varepsilon_{v}}}}
\]
und
\[
\overbrace{\sigma_{11}-\sigma_{33}}^{\displaystyle{q}}
=
\frac{3E}{2(1+\nu)} \overbrace{\frac{2}{3}(\varepsilon_{11} - \varepsilon_{33})}^{\displaystyle{\varepsilon_{s}}}
.
\]
Die Faktoren mit den Dehnungskomponenten können so mit
\[
\varepsilon_{v}
=
(\varepsilon_{11} - 2\varepsilon_{33})
\qquad
\text{und}
\qquad
\varepsilon_{s}
=
\frac{2}{3}(\varepsilon_{11} - \varepsilon_{33})
\]
eingeführt werden, mit
\begin{align*}
	\varepsilon_{v} &= \text{Hydrostatische Dehnung [-]} \\
	\varepsilon_{s} &= \text{Deviatorische Dehnung [-].}
\end{align*}
Die hydrostatische Dehnung $\varepsilon_{v}$ kann mit einer Kompression und
die deviatorische Dehnung $\varepsilon_{s}$  mit einer Verzerrung verglichen werden.

Diese zwei Gleichungen kann man durch die Matrixschreibweise
\begin{equation}
\begin{pmatrix}
	q\\
	p
\end{pmatrix}
=
\begin{pmatrix}
	\displaystyle{\frac{3E}{2(1+\nu)}} &                                  0 \\
	                                 0 & \displaystyle{\frac{E}{3(1-2\nu)}}
\end{pmatrix}
\begin{pmatrix}
	\varepsilon_{s}\\
	\varepsilon_{v}
\end{pmatrix}
\label{spannung:Matrixschreibweise}
\end{equation}
vereinfachen.

Änderungen des Spannungszustandes können mit diesen Gleichungen vollumfänglich erfasst werden.
Diese Spannungsgleichung mit den zwei Einträgen ($p$ und $q$) ist gleichwertig
wie die ursprüngliche Spannungsgleichung mit den neun Einträgen
($\sigma_{11}$, $\sigma_{12}$, $\sigma_{13}$, $\sigma_{21}$, $\sigma_{22}$, $\sigma_{23}$, $\sigma_{31}$, $\sigma_{32}$, $\sigma_{33}$).
Mit dieser Formel \eqref{spannung:Matrixschreibweise} lassen sich verschieden Ergebnisse von Versuchen analysieren und berechnen.
Ein solcher Versuch, der oft in der Geotechnik durchgeführt wird, ist der Oedometer-Versuch.
Im nächsten Kapitel wird die Anwendung der Matrix an diesem Versuch beschrieben.
