\section{Vektoroperationen\label{clifford:section:Vektoroperationen}}
\rhead{Vektoroperationen}
\subsection{Vektordarstellung\label{clifford:section:Vektordarstellung}}
Vektoren können neben der üblichen Darstellung, auch als Linearkombination aus Basisvektoren dargestellt werden
\begin{equation}
    \begin{split}
    \textbf{a} 
    &=
    \begin{pmatrix} 
    a_1 \\ a_2 \\ \vdots \\ a_n   
    \end{pmatrix} 
    =
    a_1 \begin{pmatrix}
    1 \\ 0 \\ \vdots \\ 0  
    \end{pmatrix} 
    + 
    a_2\begin{pmatrix} 
    0 \\ 1 \\ \vdots \\ 0  
    \end{pmatrix} + \dots 
    + 
    a_n\begin{pmatrix}
    0 \\ 0 \\ \vdots \\ 1  
    \end{pmatrix} \\\ 
    &= 
    a_1\textbf{e}_1 
    +
    a_2\textbf{e}_2
    + 
    \dots + a_n\textbf{e}_n
    = 
    \sum_{i=1}^{n} a_i \textbf{e}_i
    \qquad
    a_i \in \mathbb{R}
    , \textbf{e}_i \in \mathbb{R}^n.
    \end{split}
\end{equation}
Diese Basisvektoren sollen orthonormal sein und um die Darstellung zu vereinfachen werden sie durch $\textbf{e}_1 , \textbf{e}_2, ...$ ersetzt.
\begin{beispiel}
Linearkombination von Basisvektoren in $\mathbb{R}^4$
    \begin{equation}
        \begin{pmatrix} 
        42 \\ 2 \\ 1291 \\ 4    
        \end{pmatrix} 
        = 
        42 \begin{pmatrix}
        1 \\ 0 \\ 0 \\ 0 
        \end{pmatrix} 
        +
        2 \begin{pmatrix} 
        0 \\ 1 \\ 0 \\ 0 
        \end{pmatrix}
        +
        1291 
        \begin{pmatrix} 
        0 \\ 0 \\ 1 \\ 0 
        \end{pmatrix} 
        +
        4 \begin{pmatrix} 
        0 \\ 0 \\ 0 \\ 1 
        \end{pmatrix} 
        = 
        42\textbf{e}_1
        + 
        2\textbf{e}_2
        + 
        1291\textbf{e}_3
        + 
        4\textbf{e}_4
    \end{equation}
\end{beispiel}
Wobei Beispiel für einen vier dimensionalen Vektor ist, dies kann selbstverständlich für beliebig viele Dimensionen nach demselben Schema erweitert werden.