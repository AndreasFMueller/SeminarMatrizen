%
% einleitung.tex -- Beispiel-File für die Einleitung
%
% (c) 2020 Prof Dr Andreas Müller, Hochschule Rapperswil
%
\section{Pauli-Matrizen}
\rhead{Pauli-Matrizen}

Was ist der beste Weg um einen Computeralgorithmus für die Rechenoperationen in der Clifford-Algebra zu erstellen? Man könnte versuchen ein textueller Rechner zu implementieren der für die Elemente $\mathbf{e}_i$ hartkodierte Vereinfachungen ausführt 
\begin{beispiel}
	der Algorithmus weiss, dass er $a\mathbf{e}_1\cdot b\mathbf{e}_1$ zu $ab\cdot1$ vereinfachen kann
	\begin{align}
		3\mathbf{e}_1 \cdot 2\mathbf{e}_1 + 3\mathbf{e}_2 \Rightarrow 6 + 3\mathbf{e}_2
	\end{align}
\end{beispiel}
Dies ist aber sehr ineffizient. Die Pauli-Matrizen bilden eine elegante und schnellere Alternative, welche für die dreidimensionale Clifford-Algebra verwendet werden können und alle Operationen aus der Clifford-Algebra gleich wie die Matrixoperationen ausführen lassen.
\begin{definition} \label{def:defPauli}
	vier Pauli-Matrizen ($\mathbf{e}_0$ = Skalare)
	\begin{align}
		\mathbf{e}_0 = E = 
		\begin{pmatrix}
			1 & 0 \\
			0 & 1
		\end{pmatrix}\quad
		\mathbf{e}_1 =
		\begin{pmatrix}
			0 & 1 \\
			1 & 0
		\end{pmatrix}\quad
		\mathbf{e}_2 =
		\begin{pmatrix}
			0 & -j \\
			j & 0
		\end{pmatrix}\quad
		\mathbf{e}_3 =
		\begin{pmatrix}
			1 & 0 \\
			0 & -1
		\end{pmatrix}\quad	
	\end{align}
	durch normale Matrizenmultiplikation lassen sich die restlichen Basiselemente der dreidimensionalen Clifford-Algebra herleiten
	\begin{align}
		\mathbf{e}_{12} =  
		\begin{pmatrix}
			j & 0 \\
			0 & -j
		\end{pmatrix}\quad
		\mathbf{e}_{23} =
		\begin{pmatrix}
			0 & j \\
			j & 0
		\end{pmatrix}\quad
		\mathbf{e}_{31} =
		\begin{pmatrix}
			0 & 1 \\
			-1 & 0
		\end{pmatrix}\quad
		\mathbf{e}_{123} =
		\begin{pmatrix}
			j & 0 \\
			0 & j
		\end{pmatrix}\quad	
	\end{align}
\end{definition}
Dabei ist wichtig, dass sich die Matrizen gleich verhalten, wie es die Clifford-Algebra für die Basiselemente definiert hat.
\begin{align}
	\mathbf{e}_1^2 &= \mathbf{e}_0 =
	\begin{pmatrix}
		0 & 1 \\
		1 & 0
	\end{pmatrix}^2 = 
	\begin{pmatrix}
		1 & 0 \\
		0 & 1
	\end{pmatrix}\\
	\mathbf{e}_{12}^2 &= -\mathbf{e}_0 =
	\begin{pmatrix}
		j & 0 \\
		0 & -j
	\end{pmatrix}^2 = 
	\begin{pmatrix}
		-1 & 0 \\
		0 & -1
	\end{pmatrix}
\end{align}
Man kann bei der Definition \ref{def:defPauli} sehen, dass alle Matrizen linear unabhängig voneinander sind. Das bedeutet, dass wenn man die Matrizen der Basiselemente normal addiert und zu einer grossen Matrix zusammenfasst und anschliessend wieder herausgelesen werden können.
\begin{definition}
	Multivektor mit Pauli-Matrizen
	\begin{align}
		M &= a_0\mathbf{e}_0 + a_1\mathbf{e}_1 + a_2\mathbf{e}_3 + a_{12}\mathbf{e}_{12} + a_{23}\mathbf{e}_{23} + a_{31}\mathbf{e}_{31} + a_{123}\mathbf{e}_{123}\\
		M &=
		\begin{pmatrix}
			(a_0+a_3) + (a_{12}+a_{123})j & (a_1+a_{31})+(-a_2+a_{23})j \\
			(a_1-a_{31})+(a_2+a_{23})j & (a_0-a_3)+(-a_{12}+a_{123})j
		\end{pmatrix}  
	\end{align}
\end{definition}
\begin{beispiel}
	\begin{align}
		M &= \begin{pmatrix}
			1 & 0 \\
			0 & 0
		\end{pmatrix}\\
		&\Rightarrow a_0 + a_3 = 1 \land a_0 - a_3 = 0\\
		&\Rightarrow a_0 = 0.5 \land a_3 = 0.5\\
		M &= 0.5 \mathbf{e}_0 + 0.5 \mathbf{e}_3
	\end{align}
\end{beispiel}