\subsection{Multiplikation von Vektoren}
Was geschieht nun, wenn zwei beliebige Vektoren
\begin{equation}
    \textbf{u} = 
    \sum_{i=1}^{n} u_i \textbf{e}_i 
    \quad
    \text{und}
    \quad
    \textbf{v} = \sum_{i=1}^{n} v_i \textbf{e}_i
\end{equation}
 miteinander multipliziert werden? 
 Wieder werden die Vektoren zuerst als Linearkombinationen darstellen und danach in zwei Summen aufgeteilt, eine Summe mit quadrierten Termen und eine Summe mit Mischtermen
\begin{equation}
    \begin{split}
        \textbf{u}\textbf{v} 
        =
        \left ( 
        \sum_{i=1}^{n} u_i \textbf{e}_i
        \right ) 
        \left ( 
        \sum_{i=1}^{n} v_i \textbf{e}_i
        \right) 
        = 
        \sum_{i=1}^n u_iv_i\underbrace{\textbf{e}_i^2}_{1} 
        + \sum_{\begin{subarray}{l}i,j=1\\i \neq j\end{subarray}}^n  u_iv_j\textbf{e}_i\textbf{e}_j,
    \end{split}
\end{equation}
Die Summe der quadrierten Termen ist bereits aus \eqref{eq:quad_a_3}, sie ist nämlich das Skalarprodukt von $u$ und $v$. Die Summe der Mischterme bilden etwas Neues, dass wir das äussere Produkt von $u$ und $v$ nennen.
\begin{beispiel}
    Die Multiplikation von Vektoren in $\mathbb{R}^2$ ergibt
\begin{equation}
    \begin{split}
        \textbf{u}\textbf{v} 
        &= 
        (u_1\textbf{e}_1 + u_2\textbf{e}_2)(v_1\textbf{e}_1 + v_2\textbf{e}_2) 
        = 
        u_1v_1\textbf{e}_1^2
        + 
        u_2v_2\textbf{e}_2^2 
        + 
        u_1v_2\textbf{e}_1\textbf{e}_2 
        +  
        u_2v_1\underbrace{\textbf{e}_2\textbf{e}_1}_{-\textbf{e}_1\textbf{e}_2}
        \\\ 
        &=  
        \underbrace{(u_1v_1 + u_2v_2)}_{\text{Skalarprodukt}} 
        + 
        \underbrace{(u_1v_2 - u_2v_1)\textbf{e}_1\textbf{e}_2}_{\text{Äusseres Produkt}}.
    \end{split}
\end{equation}
\end{beispiel}
\subsubsection{Äusseres Produkt}
Das äussere Produkt von zwei Vektoren wird mit einem $\wedge$ dargestellt:
\begin{equation}
    \textbf{u}\wedge \textbf{v} 
    = 
    \sum_{\begin{subarray}{l}i,j=1\\i \neq j\end{subarray}}^n  u_iv_j\textbf{e}_i\textbf{e}_j .
\end{equation}
\begin{beispiel}
Das äusseres Produkt von zwei Vektoren in $\mathbb{R}^3$ ist
\begin{equation}
	\begin{split}
		u \wedge v 
		&= 
		u_1v_2\textbf{e}_1\textbf{e}_2 
		+ 
		u_1v_3\textbf{e}_1\textbf{e}_3 
		+ 
		u_2v_2\textbf{e}_2\textbf{e}_3 
		+ 
		u_2v_1\textbf{e}_2\textbf{e}_1 
		+ 
		u_3v_1\textbf{e}_3\textbf{e}_1 
		+
		u_3v_2\textbf{e}_3\textbf{e}_2 \\\ 
		&= 
		(u_1v_2 - u_2v_1)\textbf{e}_1\textbf{e}_2 
		+ 
		(u_1v_3 - v_3u_1)\textbf{e}_1\textbf{e}_3 
		+ 
		(u_2v_3 - u_3v_2)\textbf{e}_2\textbf{e}_3.
	\end{split}
\end{equation}
\end{beispiel}

Im letzten Schritt des Beispiels wurden, mit Hilfe der Antikommutativität des Produkts die Vektorprodukte, welche die gleichen Einheitsvektoren beinhalten, zusammengefasst. Dieses Vorgehen kann man auch allgemein anwenden, wie in den Gleichungen \eqref{eq:u_wedge_v}--\eqref{eq:u_wedge_v_5} geteigt werden soll. Die Summe
\begin{align}
        \textbf{u}\wedge \textbf{v}
        &= 
        \sum_{\begin{subarray}{l}i,j=1\\i \neq j\end{subarray}}^n  
        u_iv_j\textbf{e}_i\textbf{e}_j,
        \label{eq:u_wedge_v}
        \intertext{wird in zwei verschiedene Summen}
        %\label{eq:u_wedge_v_1}
        &= 
        \sum_{\begin{subarray}{l}i,j=1\\i < j\end{subarray}}^n u_iv_j\textbf{e}_i\textbf{e}_j 
        + 
        \sum_{\begin{subarray}{l}i,j=1\\j < i\end{subarray}}^n u_iv_j\textbf{e}_i\textbf{e}_j
        \label{eq:u_wedge_v_2}
        \intertext{aufgeteilt. 
        	Die linke Summe beinhaltet den Basisvektor mit dem höheren Index an erster Stelle und die rechte Summe diesen jeweils an zweiter Stelle.Nun werden die Indices der zweiten Summe vertauscht, sie wird}
        &= 
        \sum_{\begin{subarray}{l}i,j=1\\i < j\end{subarray}}^n u_iv_j\textbf{e}_i\textbf{e}_j 
        + 
        \sum_{\begin{subarray}{l}i,j=1\\i < j\end{subarray}}^n u_jv_i\textbf{e}_j\textbf{e}_i.
       	\intertext{Mit Hilfe der Antikommutativität kann dies umgeformt werden zu}
        &= 
        \sum_{\begin{subarray}{l}i,j=1\\i < j\end{subarray}}^n u_iv_j\textbf{e}_i\textbf{e}_j 
        - 
        \sum_{\begin{subarray}{l}i,j=1\\i < j\end{subarray}}^n u_jv_i\textbf{e}_i\textbf{e}_j.
        \intertext{Nun können die zwei Summen wieder}
        \label{eq:u_wedge_v_4}
        &= 
        \sum_{\begin{subarray}{l}i,j=1\\i < j\end{subarray}}^n (u_iv_j -u_jv_i)\textbf{e}_i\textbf{e}_j
        \intertext{zusammengefasst werden. Der Koeffizient $(u_iv_j - u_jv_i)$ in der Summe ist wohlbekannt, es ist nämlich die Determinante einer $2\times2$ Matrix mit $\textbf{u}$ und $\textbf{v}$ als ihre Spalten}
        &= 
        \label{eq:u_wedge_v_5}
        \sum_{\begin{subarray}{l}i,j=1\\i < j\end{subarray}}^n \begin{vmatrix} 
        u_i & v_i \\
        u_j & v_j
    \end{vmatrix}\textbf{e}_i\textbf{e}_j.
\end{align}

Die Determinante einer $2\times2$ Matrix beschreibt die Fläche, welche von den Spaltenvektoren aufgespannt wird, wie in Abbildung \ref{figure:det} dargestellt.
\begin{figure}[htb]
	\centering
	\begin{minipage}[t]{.45\linewidth}
		\centering
		\begin{tikzpicture}
			\draw[thin,gray!40] (0,0) grid (4,4);
			\draw[<->] (0,0)--(4,0) ;
			\draw[<->] (0,0)--(0,4) ;
			\draw[line width=0,fill=gray!40] (0,0)--(3,1)--(4,3)--(1,2);
			\draw[line width=2pt,blue,-stealth](0,0)--(3,1) node[anchor=north
			west]{$\boldsymbol{u}$};
			\draw[line width=2pt,red,-stealth](0,0)--(1,2) node[anchor=south east]{$\boldsymbol{v}$};
			\draw[black] (2,1.5)--(1.8,3.2) node[anchor = south]{$\begin{vmatrix} 
					u_i & v_i \\
					u_j & v_j
				\end{vmatrix} = u_iv_j - v_iu_j$};
		\end{tikzpicture}
		\caption{Geometrische Interpretation der Determinante einer $2 \times 2$ Matrix\label{figure:det}}
	\end{minipage}%
	\hfill%
	\begin{minipage}[t]{.45\linewidth}
		\centering
		\begin{tikzpicture} 
			\draw[thin,gray!40] (0,0) grid (4,4);
			\draw[<->] (0,0)--(4,0) node[right]{$x$};
			\draw[<->] (0,0)--(0,4) node[above]{$y$};
			\draw[line width=0,fill=gray!40] (0,0)--(3,1)--(4,3)--(1,2);
			\draw[line width=2pt,blue,-stealth](0,0)--(3,1) node[anchor=north
			west]{$\boldsymbol{u}$};
			\draw[line width=2pt,red,-stealth](0,0)--(1,2) node[anchor=south east]{$\boldsymbol{v}$};
			\draw[->] (2.15,1.5) arc (0:310:0.3);
			\draw[black] (2,1.5)--(3.3,3.2) node[anchor = south]{$u\wedge v = \begin{vmatrix} 
					u_i & v_i \\
					u_j & v_j
				\end{vmatrix} e_1e_2 = (u_iv_j - v_iu_j)\textbf{e}_1\textbf{e}_2$};
		\end{tikzpicture}
		\caption{Geometrische Interpretation des äusseren Produktes \label{figure:wedge}}
	\end{minipage}
\end{figure}
Das äussere Produkt besteht nun also aus der Summe 
    \(\sum_{\begin{subarray}{l}i,j=1\\i < j\end{subarray}}^n\)
    von Flächen 
    \(\begin{vmatrix} 
    	u_i & v_i \\
    	u_j & v_j
    \end{vmatrix}\)
, welche in $\textbf{e}_i\textbf{e}_j$ aufgespannt sind, wie man in \eqref{eq:u_wedge_v_5} sieht. 
Dieses Produkt $\textbf{e}_i\textbf{e}_j$ der Basisvektoren interpretiert man als Umlaufrichtung.
Die gebildete Fläche wird in Richtung des ersten Vektors umschritten. 
Dies ist in Abbildung \ref{figure:wedge} dargestellt, wobei bei diesem Beispiel die Umlaufrichtung im Gegenuhrzeigersinn ist, da die Fläche in Richtung $\textbf{u}$ umschritten wird. 
Diese Fläche mit einer Richtung nennt man in der geometrischen Algebra einen Bivektor, da er eine Art zwei dimensionaler Vektor ist. 
