\subsection{Geometrisches Produkt}
Die Multiplikation von zwei Vektoren nennt man in der Clifford Algebra das geometrische Produkt, dieses können wir nun als Summe aus dem Skalar- und dem äusseren Produkt darstellen
\begin{equation}
    \textbf{u}\textbf{v} = \textbf{u}\cdot \textbf{v} + \textbf{u} \wedge \textbf{v}.
\end{equation}
Dieses Additionszeichen zwischen diesen zwei Produkten mag vielleicht ein wenig eigenartig wirken, da uns das Skalarprodukt ein Skalar und das äussere Produkt einen Bivektor zurück gibt. Was bedeutet es nun also diese beiden Elemente zu addieren?
Man kann sich die Addition wie bei den komplexen Zahlen vorstellen, wobei die imaginäre Einheit auch nicht explizit zu dem reellen Teil addiert werden kann, sondern die zwei Teile zusammen ein Objekt, eine komplexe Zahl bilden. 
Dieses Objekt, also die Summe von verschiedenen Elemente der Clifford Algebra, wird Multivektor genannt.
\begin{definition}
Neben dem eindimensionalen Vektor, dem zweidimensionalen Bivektor gibt es noch höher dimensionale Vektoren, wie zum Beispiel der dreidimensionale Trivektor.
\end{definition}
\begin{definition}
	Für das Produkt von Basisvektoren wird die Notation
	\begin{equation}
		e_ie_j = e_{ij}
	\end{equation}
	 definiert.
\end{definition}
\begin{definition}
	Die Linearkombination von Vektoren, Bivektoren und höher dimensionalen Vektoren
	\begin{equation}
		M = \sum \left ( \prod a_i\textbf{e}_j \right )
	\end{equation}
  bildet einen Multivektor.
\end{definition}
Besteht eine Clifford Algebra aus $n$ Basisvektoren so hat sie n Dimensionen, dies wird nicht wie in der linearen Algebra mit $\mathbb{R}^n$ sondern mit $G_n(\mathbb{R})$ beschrieben. Dies wird so geschrieben da man eine neue Algebrastruktur um die Vektoren einführt.
\begin{beispiel}
Allgemeiner Multivektor in $G_3(\mathbb{R})$
\begin{equation}
    M = a 
    + 
    \underbrace{b\textbf{e}_1 + c\textbf{e}_2 + d\textbf{e}_3}_{\text{Vektorteil}} 
    +
    \underbrace{f\textbf{e}_1\textbf{e}_2 + g\textbf{e}_1\textbf{e}_3 + h\textbf{e}_2\textbf{e}_3 }_{\text{Bivektorteil}} 
    +   
    \underbrace{k\textbf{e}_1\textbf{e}_2\textbf{e}_3}_{\text{Trivektorteil}}
\end{equation}
\end{beispiel}
\begin{table}
    \label{tab:multip}
    \begin{center}
    \begin{tabular}{ |c|ccc|ccc|c| } 
     \hline
     $1$ & $\textbf{e}_1$ & $\textbf{e}_2$ &$\textbf{e}_3$ & $\textbf{e}_{12}$ & $\textbf{e}_{13}$ & $\textbf{e}_{23}$ & $\textbf{e}_{123}$\\
     \hline
     $\textbf{e}_1$ & 1 & $\textbf{e}_{12}$ & $\textbf{e}_{12}$ & $\textbf{e}_2$ & $\textbf{e}_3$ & $\textbf{e}_{123}$ & $\textbf{e}_{23}$\\
     $\textbf{e}_2$ & $-\textbf{e}_{12}$ & $1$ & $\textbf{e}_{23}$ & $-\textbf{e}_1$ & $-\textbf{e}_{123}$ & $\textbf{e}_3$ & $-\textbf{e}_{13}$\\
     $\textbf{e}_3$ & $-\textbf{e}_{13}$ & $-\textbf{e}_{23}$ & $1$ & $\textbf{e}_{123}$ & $-\textbf{e}_1$ & $-\textbf{e}_2$ & $\textbf{e}_{12}$\\
     \hline
     $\textbf{e}_{12}$ & -$\textbf{e}_2$ & $\textbf{e}_1$& $\textbf{e}_{123}$ & $-1$ & $-\textbf{e}_{23}$ & $\textbf{e}_{13}$ &  $-\textbf{e}_{3}$\\
     $\textbf{e}_{13}$ & $-\textbf{e}_{3}$ & $-\textbf{e}_{123}$ & $\textbf{e}_{1}$ & $\textbf{e}_{23}$ & $-1$ & $-\textbf{e}_{12}$ &  $\textbf{e}_{2}$\\
     $\textbf{e}_{23}$ &  $\textbf{e}_{123}$ & $-\textbf{e}_{3}$ & $\textbf{e}_{2}$ & $-\textbf{e}_{13}$ & $\textbf{e}_{12}$ & $-1$ & $-\textbf{e}_{1}$ \\
     \hline
     $\textbf{e}_{123}$ & $\textbf{e}_{23}$ & $-\textbf{e}_{13}$ & $\textbf{e}_{12}$ & $-\textbf{e}_{3}$& $\textbf{e}_{2}$ & $-\textbf{e}_{1}$ & $-1$ \\
     \hline
    \end{tabular}
    \end{center}
 	\caption{Multiplikationstabelle für $G_3(\mathbb{R})$}
\end{table}
Nun, da das geometrische Produkt vollständig definiert wurde, können Multiplikationstabellen für verschiedene Dimensionen $G_n(\mathbb{R})$ erstellt werden. In Tabelle \ref{tab:multip} ist dies für  $G_3(\mathbb{R})$ gemacht.
