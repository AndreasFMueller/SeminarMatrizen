%
% teil1.tex -- Beispiel-File für das Paper
%
% (c) 2020 Prof Dr Andreas Müller, Hochschule Rapperswil
%
\section{Reflexion/ Spiegelung}
\rhead{Reflexion/ Spiegelung}

Die Spiegelung ist eine grundlegende, geometrische Operation, aus welcher man weitere, wie beispielsweise die später beschriebene Rotation, ableiten kann. Da die Geometrische Algebra für geometrische Anwendungen ausgelegt ist, sollte die Reflexion auch eine einfache, praktische Formulierung besitzen. \\HIER BILD
\subsection{Linearen Algebra}
Aus der linearen Algebra ist bekannt, dass man eine Spiegelung wie folgt beschreiben kann.
\begin{definition}
	Spiegelungsgleichung in der linearen Algebra
	\begin{equation} \label{RefLinAlg}
		\mathbf{v^{'}} = \mathbf{v} - 2 \cdot \mathbf{v_{\perp u}}
	\end{equation}
\end{definition}

$\mathbf{u}$ repräsentiert die Spiegelachse und $\mathbf{v_{\perp u}}$ senkrecht auf dieser Achse steht und den orthogonalen Anteil von $\mathbf{v}$ zu $\mathbf{u}$ bildet. Es scheint für diese Formel aber umständlich zu sein, weitere Spiegelungen, mit weiteren Spiegelachsen, anzufügen. Man kann die Abbildung des Vektors auf den gespiegelten Vektor auch als Matrix schreiben, welche aus den Komponenten des zu der Spiegelachse orthonormalen Vektors $\mathbf{\hat{n}}$ besteht.
\begin{align} 
	\mathbf{\hat{n}}\perp \mathbf{u}\quad \land \quad |\mathbf{\hat{n}}| = 1
\end{align}
\begin{align} \label{Spiegelmatrizen}
	Spiegelmatrizen...
\end{align}
Diese Matrizen Spiegelmatrizen gehören der orthogonalen Matrizengruppe $S\in \text{O}(n)$ an. Diese haben die Eigenschaft $S^t S = E$, was bedeutet, dass zweimal eine Spiegelung an der selben Achse keinen Effekt hat.
\subsection{Geometrische Algebra}
Die Geometrische Algebra leitet aus der obigen Formel \eqref{RefLinAlg} eine einfache und intuitive Form her, welche auch für weitere Operationen einfach erweitert werden kann.
\begin{definition}
	Spiegelungsgleichung in der geometrischen Algebra
	\begin{align}\label{RefGA}
		\mathbf{v}' = \mathbf{uvu}^{-1}
	\end{align}
\end{definition}

Die Inverse eines Vektors ist dabei so definiert, dass multipliziert mit sich dem Vektor selbst das neutrale Element 1 ergibt.
\begin{definition}
	Inverse eines Vektors
	\begin{align}
		\mathbf{u}^{-1} = \dfrac{\mathbf{u}}{|\mathbf{u}|^2} \Rightarrow \mathbf{uu}^{-1} = \dfrac{\mathbf{u}^2}{|\mathbf{u}|^2} = 1
	\end{align}
\end{definition}

verwendet man für $\mathbf{u}$ nur einen Einheitsvektor $\mathbf{\hat{u}}$, welcher die Länge 1 besitzt, wird somit die Formel reduziert zu einer beidseitigen Multiplikation von $\mathbf{\hat{u}}$.
\begin{align}
	\mathbf{v'} = \mathbf{\hat{u}v\hat{u}}
\end{align}
Im Gegensatz zu den Abbildungen in der linearen Algebra, welche in jeder anderen Dimension, wie bei der Definition \eqref{Spiegelmatrizen} ersichtlich, durch andere Matrizen beschrieben werden müssen, ist es in der geometrischen Algebra immer der gleiche Vorgehensweise. Zudem ist diese kompakte Schreibweise in der linearen Algebra nicht möglich, da bis auf das Vektorprodukt in der dritten Dimension keine Multiplikation von Vektoren definiert ist. 
\\BEISPIEL?