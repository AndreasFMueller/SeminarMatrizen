Im Rahmen der Aufgabe, die Zehntauserderstelle der Zahl $5^{5^{5^{5^5}}}$
zu berechnen muss Michael Penn im Video 
\url{https://youtu.be/Xg24FinMiws} bei 12:52 zwei Zahlen $x$ und $y$ finden,
so dass,
\[
5^5x
+
2^5y
=
1
\]
ist.
Verwenden Sie die Matrixform des euklidischen Algorithmus.

\begin{loesung}
Zunächst berechnen wir die beiden Potenzen
\[
5^5 = 3125
\qquad\text{und}\qquad
2^5 = 32.
\]
Damit können wir jetzt den Algorithmus durchführen.
Die Quotienten und Reste sind
\begin{align*}
a_0&=q_0\cdot b_0 + r_0&
3125 &= 97 \cdot 32 + 21& q_0&=97 & r_0&= 21\\
a_1&=q_1\cdot b_1 + r_1&
32 &= 1\cdot 21 + 10    & q_1&= 1 & r_1&= 11\\
a_2&=q_2\cdot b_2 + r_2&
21 &= 1\cdot 11 + 10    & q_2&= 1 & r_2&= 10\\
a_3&=q_3\cdot b_3 + r_3&
11 &= 1\cdot 10 +  1    & q_3&= 1 & r_3&=  1\\
a_4&=q_4\cdot b_4 + r_4&
10 &= 10\cdot 1 +  0    & q_4&=10 & r_4&=  0
\end{align*}
Daraus kann man jetzt auch die Matrizen $Q(q_k)$ bestimmen und
ausmultiplizieren:
\begin{align*}
Q
&=
\begin{pmatrix}
0&1\\1&-10
\end{pmatrix}
\underbrace{
\begin{pmatrix}
0&1\\1&-1
\end{pmatrix}
\begin{pmatrix}
0&1\\1&-1
\end{pmatrix}
}_{}
\underbrace{
\begin{pmatrix}
0&1\\1&-1
\end{pmatrix}
\begin{pmatrix}
0&1\\1&-97
\end{pmatrix}
}_{}
\\
&=
\begin{pmatrix}
0&1\\1&-10
\end{pmatrix}
\underbrace{
\begin{pmatrix}
0&-1\\-1&2
\end{pmatrix}
\begin{pmatrix}
1&-97\\-1&98
\end{pmatrix}
}_{}
\\
&=
\underbrace{
\begin{pmatrix}
0&1\\1&-10
\end{pmatrix}
\begin{pmatrix}
2&-195\\-3&293
\end{pmatrix}
}_{}
\\
&=
\begin{pmatrix}
-3&293\\32&-3125
\end{pmatrix}.
\end{align*}
Daras kann man jetzt ablesen, dass
\[
-3\cdot 3125
+ 
293\cdot 32
=
-9375
+
9376
=
1.
\]
Die gesuchten Zahlen sind also $x=-3$ und $y=293$.
\end{loesung}
