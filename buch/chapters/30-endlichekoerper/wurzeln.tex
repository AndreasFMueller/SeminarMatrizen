%
% wurzeln.tex -- Wurzeln einem endlichen Körper hinzufügen
%
% (c) 2021 Prof Dr Andreas Müller, Hochschule Rapperswil
%
\section{Wurzeln
\label{buch:section:wurzeln}}
\rhead{Wurzeln}
Im Körper $\mathbb{Q}$ kann man zum Beispiel die Wurzel aus $2$ nicht 
ziehen.
Das Problem haben wir in Abschnitt~\ref{buch:section:reelle-zahlen}
dadurch gelöst, dass wir $\mathbb{Q}$ zu den reellen Zahlen
erweitert haben.
Es ist aber auch möglich, nur die Zahl $\sqrt{2}$ hinzuzufügen,
so entsteht der Körper $\mathbb{Q}(\sqrt{2})$.
In diesem Abschnitt zeigen wir, wie man einem Körper beliebige 
Nullstellen $\alpha$ eines Polynoms $f\in\Bbbk[X]$ hinzufügen und
so den Körper $\Bbbk(\alpha)$ konstruieren kann.

\subsection{Irreduzible Polynome
\label{buch:subsection:irreduziblepolynome}}
Die Zahlen, die man dem Körper hinzufügen möchte, müssen Nullstellen
eines Polynoms sein.
Wir gehen daher davon aus, dass $f\in \Bbbk[X]$ ein Polynom mit
Koeffizienten in $\Bbbk$ ist, dessen Nullstelle $\alpha$ hinzugefügt
werden sollen.
Das Ziel ist natürlich, dass diese Erweiterung vollständig beschrieben
werden kann durch das Polynom, ganz ohne Bezug zum Beispiel auf einen
numerischen Wert der Nullstelle, der ohnehin nur in $\mathbb(C)$ sinnvoll
wäre.

Nehmen wir jetzt an, dass sich das Polynom $f$ faktorisieren lässt.
Dann gibt es Polynome $g,h\in\Bbbk[X]$ derart, dass $f=g\cdot h$.
Die Polynome $g$ und $h$ haben geringeren Grad als $f$. 
Setzt man die Nullstelle $\alpha$ ein, erhält man
$0=f(\alpha)=g(\alpha)h(\alpha)$, daher muss einer der Faktoren
verschwinden, also $g(\alpha)=0$ oder $h(\alpha)=0$.
Ohne Beschränkung der Allgemeinheit kann angenommen werden, dass
$g(\alpha)=0$.
Die Operation des Hinzufügens der Nullstelle $\alpha$ von $f$
muss also genauso gut mit $g$ ausgeführt werden.
Indem wir diese Überlegung auf $g$ anwenden können wir schliessen,
dass es ein Polynom $m\in\Bbbk[X]$ kleinstmöglichen Grades geben muss,
welches $\alpha$ als Nullstelle hat.
Zusätzlich kann verlangt werden, dass das Polynom normiert ist.

\begin{definition}
Ein Polynom $f\in \Bbbk[X]$ heisst {\em irreduzibel}, wenn es sich nicht
in zwei Faktoren $g,h\in \Bbbk[X]$ mit $f=gh$ zerlegen lässt.
\index{irreduzibles Polynom}%
\end{definition}

Für die Konstruktion des Körpers $\Bbbk(\alpha)$ muss daher ein irreduzibles
Polynom verwendet werden.

\begin{beispiel}
Das Polynom $f(X)=X^2-2$ ist in $\mathbb{Q}[X]$, es hat die beiden
Nullstellen $\sqrt{2}$ und $-\sqrt{2}$.
Beide Nullstellen haben die exakt gleichen algebraischen Eigenschaften,
sie sind mit algebraischen Mitteln nicht zu unterscheiden.
Nur die Vergleichsrelation ermöglicht, die negative Wurzel von der
positiven zu unterscheiden.
Das Polynom kann in $\mathbb{Q}$ nicht faktorisiert werden, denn die
einzig denkbare Faktorisierung ist $(X-\sqrt{2})(X+\sqrt{2})$, die
Faktoren sind aber keine Polynome in $\mathbb{Q}[X]$.
Also ist ein irreduzibles Polynom über $X^2-2$.

Man kann das Polynom aber auch als Polynom in $\mathbb{F}_{23}[X]$
betrachten.
Im Körper $\mathbb{F}_{23}$ kann man durch probieren zwei Nullstellen
finden:
\begin{align*}
5^2 &= 25\equiv 2\mod 23
\\
\text{und}\quad
18^2 &=324 \equiv 2 \mod 23.
\end{align*}
Und tatsächlich ist in $\mathbb{F}_{23}[X]$
\[
(X-5)(X-18) = X^2 -23X+90
\equiv
X^2 -2 \mod 23,
\]
über $\mathbb{F}_{23}$ ist das Polynom $X^2-2$ also reduzibel.
\end{beispiel}

\begin{beispiel}
Die Zahl 
\[
\alpha = \frac{1+i\sqrt{3}}2
\]
ist eine Nullstelle des Polynoms $f(X)=X^3-1\in\mathbb{Z}[X]$.
$\alpha$ enthält aber nur Quadratwurzeln, man würde also eigentlich
erwarten, dass $\alpha$ Nullstelle eines quadratischen Polynoms ist.
Tatsächlich ist $f(X)$ nicht irreduzibel,  es ist nämlich
\[
X^3-1 = (X-1)(X^2+X+1).
\]
Da $\alpha$ nicht Nullstelle des ersten Faktors ist, muss es Nullstelle
des Polynoms $m(X)=X^2+X+1$ sein.
Der zweite Faktor ist irreduzibel.

Das Polynom $m(X)$ kann man aber auch als Polynom in $\mathbb{F}_7$ 
ansehen.
Dann kann man aber zwei Nullstellen finden,
\[
\begin{aligned}
X&=2&&\Rightarrow& 2^2+2+1=4+2+1&\equiv 0\mod 7
\\
X&=4&&\Rightarrow& 4^2+4+1=16+4+1=21&\equiv 0\mod 7.
\end{aligned}
\]
Dies führt auf die Faktorisierung
\[
(X-2)(X-4)
\equiv
(X+5)(X+3)
=
X^2+8X+15
\equiv
X^2+X+1\mod 7.
\]
Das Polynom $X^2+X+1$ ist daher über $\mathbb{F}_7$ reduzibel und
das Polynom $X^3-1\in\mathbb{F}_7$ zerfällt daher in Linearfaktoren
$X^3-1=(X+6)(X+3)(X+5)$.
\end{beispiel}


\subsection{Körpererweiterungen}
Nach den Vorbereitungen von
Abschnitt~\ref{buch:subsection:irreduziblepolynome}
können wir jetzt definieren, wie die Körpererweiterung
konstruiert werden soll.

\subsubsection{Erweiterung mit einem irreduziblen Polynom}
Sei $m\in\Bbbk[X]$ ein irreduzibles Polynome über $\Bbbk$ mit dem Grad
$\deg m=n$,
wir dürfen es als normiert annehmen und schreiben es in der Form
\[
m(X)
=
m_0+m_1X+m_2X^2 + \dots m_{n-1}X^{n-1}+X^n.
\]
Wir möchten den Körper $\Bbbk$ um eine Nullstelle $\alpha$ von $m$
erweitern.
Da es in $\Bbbk$ keine Nullstelle von $m$ gibt, konstruieren wir
$\Bbbk(\alpha)$ auf abstrakte Weise, ganz so wie das mit der imaginären
Einheit $i$ gemacht wurde.
Die Zahl $\alpha$ ist damit einfach ein neues Symbol, mit dem man
wie in der Algebra üblich rechnen kann.
Die einzige zusätzliche Eigenschaft, die von $\alpha$ verlangt wird,
ist dass $m(\alpha)=0$.
Unter diesen Bedingungen können beliebige Ausdrücke der Form
\begin{equation}
a_0 + a_1\alpha + a_2\alpha^2 + \dots a_k\alpha^k
\label{buch:endlichekoerper:eqn:ausdruecke}
\end{equation}
gebildet werden.
Aus der Bedingung $m(\alpha)=0$ folgt aber, dass
\begin{equation}
\alpha^n = -a_{n-1}\alpha^{n-1} -\dots - a_2\alpha^2  - a_1\alpha-a_0.
\label{buch:endlichekoerper:eqn:reduktion}
\end{equation}
Alle Potenzen mit Exponenten $\ge n$ in
\eqref{buch:endlichekoerper:eqn:ausdruecke}
können daher durch die rechte Seite von
\eqref{buch:endlichekoerper:eqn:reduktion}
ersetzt werden.
Als Menge ist daher
\[
\Bbbk(\alpha)
=
\{
a_0+a_1\alpha+a_2\alpha^2+\dots+a_{n-1}\alpha^{n-1}\;|\; a_i\in\Bbbk\}.
\}
\]
Die Addition von solchen Ausdrücken und die Multiplikation mit Skalaren
aus $\Bbbk$ machen $\Bbbk(\alpha)\cong \Bbbk^n$ zu einem Vektorraum,
die Operationen können auf den Koeffizienten komponentenweise ausgeführt
werden.

\subsubsection{Matrixrealisierung der Multiplikation mit $\alpha$}
Die schwierige Operation ist die Multiplikation mit $\alpha$.
Dazu stellen wir zusammen, wie die Multiplikation mit $\alpha$ auf den
Basisvektoren von $\Bbbk(\alpha)$ wirkt:
\[
\alpha\cdot\colon
\Bbbk^n\to\Bbbk
:
\left\{
\begin{aligned}
     1  &\mapsto \alpha   \\
\alpha  &\mapsto \alpha^2 \\
\alpha^2&\mapsto \alpha^3 \\
        &\phantom{m}\vdots\\
\alpha^{n-2}&\mapsto \alpha^{n-1}\\
\alpha^{n-1}&\mapsto \alpha^n = -m_0-m_1\alpha-m_2\alpha^2-\dots-m_{n-1}\alpha^{n-1}
\end{aligned}
\right.
\]
Diese lineare Abbildung hat die Matrix
\[
M_{\alpha}
=
\begin{pmatrix}
0   &    &    &      &   &-m_0    \\
1   & 0  &    &      &   &-m_1    \\
    & 1  & 0  &      &   &-m_2    \\
    &    & 1  &\ddots&   &-m_3    \\
    &    &    &\ddots& 0 &\vdots  \\
    &    &    &      & 1 &-m_{n-2}\\
    &    &    &      &   &-m_{n-1}
\end{pmatrix}
\]
Aufgrund der Konstruktion die Lineare Abbildung $m(M_\alpha)$,
die man erhält, wenn
man die Matrix $M_\alpha$ in das Polynom $m$ einsetzt, jeden Vektor
in $\Bbbk(\alpha)$ zu Null machen.
Als Matrix muss daher $m(M_\alpha)=0$ sein.
Dies kann man auch mit einem Computeralgebra-System nachprüfen.

\begin{beispiel}
In einem früheren Beispiel haben wir gesehen, dass
$\alpha=\frac12(-1+\sqrt{3})$ 
eine Nullstelle des irreduziblen Polynomes $m(X)=X^2+X+1$ ist.
Die zugehörige Matrix $M_\alpha$ ist
\[
M_{\alpha}
=
\begin{pmatrix}
0&-1\\
1&-1
\end{pmatrix}
\qquad\Rightarrow\qquad
M_{\alpha}^2
=
\begin{pmatrix}
-1& 1\\
-1& 0
\end{pmatrix},\quad
M_{\alpha}^3
=
\begin{pmatrix}
 1& 0\\
 0& 1
\end{pmatrix}.
\]
Wir können auch verifizieren, dass
\[
m(M_\alpha)
=
M_\alpha^2+M_\alpha+I
=
\begin{pmatrix}
-1& 1\\
-1& 0
\end{pmatrix}
+
\begin{pmatrix}
0&-1\\
1&-1
\end{pmatrix}
+
\begin{pmatrix}
1&0\\
0&1
\end{pmatrix}
=
\begin{pmatrix}
0&0\\
0&0
\end{pmatrix}.
\]
Die Matrix ist also eine mögliche Realisierung für das ``mysteriöse''
Element $\alpha$.
Es hat alle algebraischen Eigenschaften von $\alpha$.
\end{beispiel}

Die Menge $\Bbbk(\alpha)$ kann durch die Abbildung $\alpha\mapsto M_\alpha$
mit der Menge aller Matrizen
\[
\Bbbk(M_\alpha)
=
\left\{
\left.
a_0I+a_1M_\alpha+a_2M_\alpha^2+\dots+a_{n-1}M_\alpha^{n-1}\;\right|\; a_i\in\Bbbk
\right\}
\]
in eine Eins-zu-eins-Beziehung gebracht werden.
Diese Abbildung ist ein Algebrahomomorphismus.
Die Menge $\Bbbk(M_\alpha)$ ist also das Bild des
Körpers $\Bbbk(\alpha)$ in der Matrizenalgebra $M_n(\Bbbk)$.

\subsubsection{Inverse}
Im Moment wissen wir noch nicht, wie wir $\alpha^{-1}$ berechnen sollten.
Wir können aber auch die Matrizendarstellung verwenden können.
Für Matrizen wissen wir selbstverständlich, wie Matrizen invertiert
werden können.
Tatsächlich kann man die Matrix $M_\alpha$ direkt invertieren:
\[
M_\alpha^{-1}
=
\frac{1}{m_0}
\begin{pmatrix}
   -m_1 &m_0&   &      &      &   \\
   -m_2 & 0 &m_0&      &      &   \\
   -m_3 &   & 0 &   m_0&      &   \\
 \vdots &   &   &\ddots&\ddots&   \\
-m_{n-1}& 0 & 0 &      &  0   &m_0\\
    -1  & 0 & 0 &      &  0   & 0
\end{pmatrix},
\]
wie man durch Ausmultiplizieren überprüfen kann:
\[
\frac{1}{m_0}
\begin{pmatrix}
   -m_1 &m_0&   &      &      &   \\
   -m_2 & 0 &m_0&      &      &   \\
   -m_3 &   & 0 &   m_0&      &   \\
 \vdots &   &   &\ddots&\ddots&   \\
-m_{n-1}& 0 & 0 &      &  0   &m_0\\
    -1  & 0 & 0 &      &  0   & 0
\end{pmatrix}
\begin{pmatrix}
0   &    &    &      &   &-m_0    \\
1   & 0  &    &      &   &-m_1    \\
    & 1  & 0  &      &   &-m_2    \\
    &    & 1  &\ddots&   &-m_3    \\
    &    &    &\ddots& 0 &\vdots  \\
    &    &    &      & 1 &-m_{n-2}\\
    &    &    &      &   &-m_{n-1}
\end{pmatrix}
=
\begin{pmatrix}
1&0&0&\dots&0&0\\
0&1&0&\dots&0&0\\
0&0&1&\dots&0&0\\
\vdots&\vdots&\vdots&\vdots&\vdots\\
0&0&0&\dots&1&0\\
0&0&0&\dots&0&1
\end{pmatrix}
\]
Die Invertierung in $\Bbbk(M_\alpha)$ ist damit zwar geklärt, aber
es wäre viel einfacher, wenn man die Inverse auch in $\Bbbk(\alpha)$
bestimmen könnte.

Die Potenzen von $M_\alpha^k$ haben in der ersten Spalte genau in
Zeile $k+1$ eine $1$, alle anderen Einträge in der ersten Spalte
sind $0$.
Die erste Spalte eines Elementes
$a(\alpha)=a_0+a_1\alpha+a_2\alpha^2 +a_{n-1}\alpha^{n-1}$
besteht daher genau aus den Elementen $a_i$.
Die Inverse des Elements $a$ kann daher wie folgt gefunden werden.
Zunächst wird die Matrix $a(M_\alpha)$ gebildet und invertiert.
Wir schreiben $B=a(M_\alpha)^{-1}$.
Aus den Einträgen der ersten Spalte kann man jetzt die Koeffizienten
\[
b_0=(B)_{11},
b_1=(B)_{21},
b_2=(B)_{11},\dots,
b_{n-1}=(B)_{n,1}
\]
ablesen und daraus das Element
\[
b(\alpha) = b_0+b_1\alpha+b_2\alpha^2 + \dots + b_{n-1}\alpha^{n-1}
\]
bilden.
Da $b(M_\alpha)=B$ die inverse Matrix von $a(M_\alpha)$ ist, muss $b(\alpha)$
das Inverse von $a(\alpha)$ sein.

\begin{beispiel}
Wir betrachten das Polynom 
\[
m(X) = X^3 + 2X^2 + 2X + 3 \in \mathbb{F}_{7}[X],
\]
es irreduzibel.
Sei $\alpha$ eine Nullstelle von $m$, wir suchen das inverse Element zu
\[
a(\alpha)=1+2\alpha+2\alpha^2\in\mathbb{F}_{7}(\alpha).
\]
Die Matrix $a(M_\alpha)$ bekommt die Form
\[
A=\begin{pmatrix}
 1& 1& 6\\
 2& 4& 5\\
 2& 5& 1
\end{pmatrix}.
\]
Die Inverse kann man bestimmen, indem man den
Gauss-Algorithmus in $\mathbb{F}_{17}$ durchführt.
\begin{align*}
\begin{tabular}{|>{$}c<{$}>{$}c<{$}>{$}c<{$}|>{$}c<{$}>{$}c<{$}>{$}c<{$}|}
\hline
 1& 1& 6& 1& 0& 0\\
 2& 4& 5& 0& 1& 0\\
 2& 5& 1& 0& 0& 1\\
\hline
\end{tabular}
&\rightarrow
\begin{tabular}{|>{$}c<{$}>{$}c<{$}>{$}c<{$}|>{$}c<{$}>{$}c<{$}>{$}c<{$}|}
\hline
 1& 1& 6& 1& 0& 0\\
 0& 2& 0& 5& 1& 0\\
 0& 3& 3& 5& 0& 1\\
\hline
\end{tabular}
\rightarrow
\begin{tabular}{|>{$}c<{$}>{$}c<{$}>{$}c<{$}|>{$}c<{$}>{$}c<{$}>{$}c<{$}|}
\hline
 1& 1& 6& 1& 0& 0\\
 0& 1& 0& 6& 4& 0\\
 0& 0& 3& 1& 2& 1\\
\hline
\end{tabular}
\\
&\rightarrow
\begin{tabular}{|>{$}c<{$}>{$}c<{$}>{$}c<{$}|>{$}c<{$}>{$}c<{$}>{$}c<{$}|}
\hline
 1& 1& 6& 1& 0& 0\\
 0& 1& 0& 6& 4& 0\\
 0& 0& 1& 5& 3& 5\\
\hline
\end{tabular}
\\
&\rightarrow
\begin{tabular}{|>{$}c<{$}>{$}c<{$}>{$}c<{$}|>{$}c<{$}>{$}c<{$}>{$}c<{$}|}
\hline
 1& 1& 0& 6& 3& 5\\
 0& 1& 0& 6& 4& 0\\
 0& 0& 1& 5& 3& 5\\
\hline
\end{tabular}
\rightarrow
\begin{tabular}{|>{$}c<{$}>{$}c<{$}>{$}c<{$}|>{$}c<{$}>{$}c<{$}>{$}c<{$}|}
\hline
 1& 0& 0& 0& 6& 5\\
 0& 1& 0& 6& 4& 0\\
 0& 0& 1& 5& 3& 5\\
\hline
\end{tabular}
\end{align*}
Für die Durchführung braucht man die Inversen in $\mathbb{F}_7$
der Pivot-Elemente, sie sind $2^{-1}=4$ und $3^{-1}=5$.
Im rechten Teil des Tableau steht jetzt die inverse Matrix
\[
A^{-1}
=
B=\begin{pmatrix}
 0& 6& 5\\
 6& 4& 0\\
 5& 3& 5
\end{pmatrix}.
\]
Daraus können wir jetzt das inverse Element
\[
b(\alpha) = 6\alpha+5\alpha^2
\]
ablesen.
Das Produkt $b(X)\cdot a(X)$ ist
\begin{align*}
(1+2X+2X^2)(6X+5X^2)
&=
10X^4 + 22X^3 + 17X^2 + 6X
\\
&=
3X^4+X^3+3X^2+6X
\intertext{
Diese Polynom muss jetzt mit dem Minimalpolynom $m(X)$ reduziert
werden, wir subtrahieren dazu $3Xm(X)$ und erhalten}
&=
-5X^3-3X^2-3X
\\
&=
2X^3+4X^2+4X
\intertext{Die vollständige Reduktion wird erreicht, indem wir nochmals
$2m(X)$ subtrahieren:}
&=
-6 \equiv 1\mod 7,
\end{align*}
das Element $b(\alpha)=6\alpha+5\alpha^2$ ist also das Inverse Element von
$a(\alpha)=1+2\alpha+2\alpha^2$ in $\mathbb{F}_7(\alpha)$.
\label{buch:endlichekoerper:beispiel:inversemitmatrix}
\end{beispiel}

Die Matrixrealisation von $\Bbbk(\alpha)$ führt also auf eine effiziente
Berechnungsmöglichkeit für das Inverse eines Elements von $\Bbbk(\alpha)$.

\subsubsection{Algebraische Konstruktion}
Die Matrixdarstellung von $\alpha$ ermöglicht eine rein algebraische
und für die Rechnung besser geeignete Konstruktion.
Für jedes Polynom $f\in\Bbbk[X]$ ist $f(M_\alpha)\in M_n(\Bbbk)$.
Dies definiert einen Homomorphismus
\[
\varphi\colon \Bbbk[X] \to M_n(\Bbbk) : f \mapsto f(M_\alpha).
\]
Wir haben früher schon gesehen, dass das Bild dieses Homomorphismus
genau die Menge $\Bbbk(M_\alpha)$ ist.
Allerdings ist $\varphi$ nicht injektiv, das Polynom $m$ wird zum
Beispiel auf $\varphi(m) = m(M_\alpha) = 0$ abgebildet.

Der Kern von $\varphi$ besteht aus allen Polynomen $p\in\Bbbk[X]$,
für die $p(M_\alpha)=0$ gilt.
Da aber alle Matrizen $E,M_\alpha,\dots,M_\alpha^{n-1}$ linear
unabhängig sind, muss ein solches Polynom den gleichen Grad haben
we $m$, und damit ein Vielfaches von $m$ sein.
Der Kern besteht daher genau aus den Vielfachen von $m(X)$,
$\ker\varphi = m(X)\Bbbk[X]$.

Es ist nicht a priori klar, dass der Quotient $R/I$ für ein
Ideal $I\subset R$ ein Körper ist.
Hier spielt es eine Rolle, dass das von $m$ erzeugte Ideal
maximal ist im folgenden Sinne.

\begin{definition}
Ein Ideal $I\subset R$ heisst {\em maximal}, wenn für jedes andere Ideal
$J$ mit $I\subset J\subset R$ entweder $I=J$ oder $J=R$ gilt.
\end{definition}

\begin{beispiel}
Die Ideale $p\mathbb{Z}\subset \mathbb{Z}$ sind maximal genau dann, wenn
$p$ eine Primzahl ist.

TODO: XXX Begründung
\end{beispiel}

\begin{satz}
Der Ring $R/I$ ist genau dann ein Körper, wenn $I$ ein maximales Ideal ist.
\end{satz}

\begin{proof}[Beweis]
\end{proof}

Ein irreduzibles Polynom $m\in\Bbbk[X]$ erzeugt ein maximales Ideal,
somit ist $\Bbbk[X]/m\Bbbk[X]\cong \Bbbk(M_\alpha) \cong \Bbbk(\alpha)$.

\subsubsection{Rechnen in $\Bbbk(\alpha)$}
Die algebraische Konstruktion hat gezeigt, dass die arithmetischen
Operationen im Körper $\Bbbk(\alpha)$ genau die Operationen 
in $\Bbbk[X]/m\Bbbk[X]$ sind.
Eine Zahl in $\Bbbk(\alpha)$ wird also durch ein Polynom vom 
$n-1$ dargestellt.
Addieren und Subtrahieren erfolgen Koeffizientenweise in $\Bbbk$.
Bei der Multiplikation entsteht möglicherwise ein Polynom grösseren
Grades, mit dem Polynomdivisionsalgorithmus kann der Rest bei Division
durch $m$ ermittelt werden.

\begin{beispiel}
XXX: Reduktionsbeispiel
\end{beispiel}

Die schwierigste Operation ist die Division.
Wie bei der Berechnung der Inversion in einem Galois-Körper $\mathbb{F}_p$
kann dafür der euklidische Algorithmus verwendet werden.
Sei also $f\in\Bbbk[X]$ ein Polynom vom Grad $\deg f <\deg m$, es soll
das multiplikative Inverse gefunden werden.
Da $m$ ein irreduzibles Polynom ist, müssen $f$ und $m$ teilerfremd sein.
Der euklidische Algorithmus liefert zwei Polynome $a,b\in\Bbbk[X]$ derart,
dass
\[
af+bm=1.
\]
Reduzieren wir modulo $m$, wird daraus $af=1$ in $\Bbbk[X]/m\Bbbk[X]$.
Das Polynom $a$, reduziert module $m$, ist also die multiplikative
Inverse von $f$.

\begin{beispiel}
% XXX verweise auf das frühere Beispiel
Wir berechnen die multiplikative Inverse von
$f=2X^2+2X+1\in\mathbb{F}_7[X]/m\mathbb{F}_7[X]$ 
mit $m = X^3 + 2X^2 + 2X + 3$.

Zunächst müssen wir den euklidischen Algorithmus für die beiden Polynome
$f$ und $m$ durchführen.
Der Quotient $m:f$ ist:
\[
\arraycolsep=1.4pt
\begin{array}{rcrcrcrcrcrcrcrcr}
  X^3&+&2X^2&+&2X&+&3&:&2X^2&+&2X&+&1&=&4X&+&4\rlap{$\mathstrut=q_0$}\\
\llap{$-($}X^3&+& X^2&+&4X\rlap{$)$}& & & &    & &  & & & &  & & \\ \cline{1-5}
     & & X^2&+&5X&+&3& &    & &  & & & &  & & \\
     &&\llap{$-(\phantom{2}$}X^2&+& X&+&4\rlap{$)$}& &    & &  & & & &  & & \\ \cline{3-7}
     & &    & &4X&+&6\rlap{$\mathstrut=r_0$}& &    & &  & & & &  & &
\end{array}
\]
Jetzt muss der Quotient $f:r_0$ berechnet werden:
\[
\arraycolsep=1.4pt
\begin{array}{rcrcrcrcrcrcrcrcr}
  2X^2&+&2X&+&1&:&4X&+&6&=&4X&+&5\rlap{$\mathstrut=q_1$}\\
\llap{$-($}2X^2&+&3X\rlap{$)$}& & & &  & & & &  \\ \cline{1-3}
      & &6X&+&1& &  & & & &  \\
      & &\llap{$-($}6X&+&2\rlap{$)$}& &  & & & &  \\ \cline{3-5}
      & &  & &6\rlap{$\mathstrut=r_1$}& & & &  & & & &
\end{array}
\]
\[
\arraycolsep=1.4pt
\begin{array}{rcrcrcrcr}
4X&+&6&:&6&=&3X&+&1\rlap{$\mathstrut=q_2$} \\
\llap{$-($}4X\rlap{$)$}& & & & & &  & &  \\ \cline{1-1}
 0&+&6& & & &  & & \\
  & &\llap{$-($}6\rlap{$)$}& & & &  & &\\ \cline{3-3}
  & &0\rlap{$\mathstrut=r_2$}& & & &  & &
\end{array}
\]
Die nächste Division ergibt natürlich den Rest $0$ und $6=-1$ ist der
erwartete grösste gemeinsame Teiler.
Durch Ausmultiplizieren der Matrizen können wir jetzt auch die 
Faktoren $a$ und $b$ finden.
\begin{align*}
Q&= Q(q_2)Q(q_1)Q(q_0)
=
\begin{pmatrix}0&1\\1&-q_2\end{pmatrix}
\begin{pmatrix}0&1\\1&-q_1\end{pmatrix}
\begin{pmatrix}0&1\\1&-q_0\end{pmatrix}
\\
&=
\begin{pmatrix}
0&1\\
1&4X+6
\end{pmatrix}
\begin{pmatrix}
0&1\\
1&3X+2
\end{pmatrix}
\begin{pmatrix}
0&1\\
1&3X+3
\end{pmatrix}
\\
&=
\begin{pmatrix}
0&1\\
1&4X+6
\end{pmatrix}
\begin{pmatrix}
   1&3X+3\\
3X+2&2X^2 + X
\end{pmatrix}
\\
&=
\begin{pmatrix}
3X+2           &2X^2+X\\
1+(4X+6)(3X+2) &3X+3 + (4X+6)(2X^2+X)
\end{pmatrix}
\\
&=
\begin{pmatrix}
     3X+2 & 2X^2     +X\\
5X^2+5X+6 &  X^3+2X^2+2X+6
\end{pmatrix}
\end{align*}
Daraus liest man
\[
a
=
3X+2
\qquad\text{und}\qquad
b
=
2X^2+X
\]
ab.
Wir überprüfen, ob die Koeffizienten der ersten Zeile tatsächlich $m$ und $f$
zu $1$ kombinieren.
Es ist
\begin{align*}
(3X+2)\cdot m + (2X^2+X)\cdot f
&= 
(3X+2)
(X^3+3X^2+X+2)
+
(2X^2+X)
(2X^2+2X+1)
\\
&=
6
\end{align*}
Die multiplikative Inverse ist daher
$-(2X^2 + X) = 5X^2+6X$,
was mit dem Beispiel von
Seite~\pageref{buch:endlichekoerper:beispiel:inversemitmatrix}
übereinstimmt.
\end{beispiel}

Besonders einfach ist die Rechung für $\Bbbk=\mathbb{F}_2$.

TODO: XXX Arithmetik in $\mathbb{F}_2$ erklären

\subsection{Zerfällungskörper}





