%
% galois.tex -- Abschnitt über Galois-Körper
%
% (c) 2020 Prof Dr Andreas Müller, Hochschule Rapperswil
%
\section{Galois-Körper
\label{buch:section:galoiskoerper}}
\rhead{Galois-Körper}
Ein Körper $\Bbbk$ enthält mindestens die Zahlen $0$ und $1$.
Die Null ist nötig, damit $\Bbbk$ eine Gruppe bezüglich der
Addition ist, die immer ein neutrales Element, geschrieben $0$
enthält.
Die Eins ist nötig, damit $\Bbbk^*=\Bbbk\setminus\{0\}$ eine
Gruppe bezüglich der Multiplikation ist, die immer eine neutrales
Element, geschrieben $1$ enthält.
Durch wiederholte Addition entstehen auch die Zahlen $2=1+1$, $3=2+1$ 
und so weiter.
Es sieht also so aus, als ob ein Körper immer unendliche viele
Elemente enthalten müsste.
Wie können also endliche Körper entstehen?

In diesem Abschnitt sollen die sogenannten Galois-Körper $\mathbb{F}_p$
mit genau $p$ Elementen konstruiert werden, die es für jede Primzahl $p$ gibt.
Sie sind die Basis für weitere endliche Körper, die eine beliebige
Primzahlpotenz $p^n$ von Elementen haben und die die Basis wichtiger
kryptographischer Algorithmen sind.

%
% Arithmetik module $o$
%
\subsection{Arithmetik modulo $p$
\label{buch:subsection:arithmetik-modulo-p}}
Damit aus den Zahlen $0, 1, 2, \dots$ ein endlicher Körper werden kann,
muss die Folge sich wiederholen.
Schreiben wir $a_0=0,a_1=1,\dots$ für die Folge, dann muss es also
ein Folgenelement $a_k$ geben und ein $n$ derart, dass $a_{k+n}=a_{k}$.
Dies bedeutet, dass $k+n = k$ sein muss.
Subtrahiert man $k$ auf beiden Seiten, dann folgt, dass $n=0$ sein muss.
Damit ein endlicher Körper entsteht, muss also die Menge
\begin{align*}
&\{0,1,2,\dots,n-1\}
\intertext{eine Gruppe bezüglich der Addition sein, und}
&\{1,2,\dots,n-1\}
\end{align*}
eine Gruppe bezüglich der Multiplikation.

\subsubsection{Restklassenring}
Wir definieren die Grundoperationen in einer Menge, die mit den
Zahlen $\{0,1,2,\dots,n-1\}$ identifiziert werden kann.

\begin{definition}
Die Zahlen $a,b\in\mathbb{Z}$ heissen {\em kongruent modulo $n$},
geschrieben
\[
a\equiv b\mod n,
\]
wenn $a-b$ durch $n$ teilbar ist, also $n|(a-b)$.
\end{definition}

Die Zahlen mit gleichem Rest sind Äquivalenzklassen der Kongruenz modulo $n$.
Die Zahlen mit Rest $k$ modulo $n$ bilden die {\em Restklasse}
\[
\llbracket k\rrbracket=\{\dots,k-2n,k-n,k,k+n,k+2n,\dots\} \subset\mathbb{Z}.
\]
Sie bilden eine endliche Menge, die man mit den Resten $0,1,\dots,n-1$
identifizieren kann.

\begin{definition}
Die Menge $\mathbb{Z}/n\mathbb{Z}$ besteht aus den Restklassen
$\llbracket 0\rrbracket,\llbracket 1\rrbracket,\dots,\llbracket n-1\rrbracket$,
die auch einfach $0,1,\dots,n-1$ geschrieben werden.
\end{definition}

Beim Rechnen mit Resten modulo $n$ können Vielfache von $n$ ignoriert werden.
Zum Beispiel gilt 
\[
\begin{aligned}
49&\equiv -1\mod 7& 49&=-1&&\text{in $\mathbb{Z}/7\mathbb{Z}$}
\\
3\cdot 5=15&\equiv 1\mod 7 & 3\cdot 5&=1&&\text{in $\mathbb{Z}/7\mathbb{Z}$.}
\end{aligned}
\]
Das Beispiel zeigt, dass man mindestens in $\mathbb{Z}/7\mathbb{Z}$ mit
Resten ganz ähnlich rechnen kann wie in $\mathbb{Q}$.
In $\mathbb{Z}/7\mathbb{Z}$ scheinen $3$ und $5$ multiplikative inverse
zu sein.

Tatsächlich kann man auf den Restklassen eine Ringstruktur definieren.
Dazu muss man sicherstellen, dass die Auswahl eines Repräsentanten keinen
Einfluss auf den Rest hat.
Der Rest $a$ kann jede Zahl der Form $a+kn$ darstellen.
Ebenso kann der Rest $b$ jede zahl der Form $b+ln$ darstellen.
Deren Summe ist $a+b+(k+l)n\equiv a+b\mod n$.
Der Repräsentant des Restes hat also keinen Einfluss auf die Summe.

Ebenso ist das Produkt der beiden Repräsentaten 
$(a+kn)\cdot(b+ln) = ab + (al+bk)n + kln^2=ab + (al+bk+kln)n\equiv ab\mod n$
für jede Wahl von $k$ und $l$.
Auch die Multiplikation ist also unabhängig vom gewählten Repräsentanten.

\begin{definition}
Die Menge $\mathbb{Z}/n\mathbb{Z}$ ist ein Ring,
heisst der {\em Restklassenring modulo $n$}.
\end{definition}

\subsubsection{Division in $\mathbb{Z}/n\mathbb{Z}$}
Um einen endlichen Körper zu erhalten, muss die Menge
\[
\mathbb{Z}/n\mathbb{Z} \setminus \{\llbracket0\rrbracket\}
=
\{
\llbracket 1\rrbracket,
\llbracket 2\rrbracket,
\dots
\llbracket n-q\rrbracket
\}
\]
eine Gruppe bezüglich der Multiplikation sein.
Insbesondere darf kein Produkt $a\cdot b$ mit Faktoren in 
$\mathbb{Z}/n\mathbb{Z} \setminus \{\llbracket0\rrbracket\}$
zu Null werden.
Für $n=15$ funktioniert dies nicht, das Produkt $3\cdot 5\equiv 0\mod 15$.
Man nennt von Null verschiedene Faktoren, deren Produkt Null ist, einen
{\em Nullteiler}.
Falls sich $n=p_1\cdot p_2$ in zwei Faktoren zerlegen lässt, dann sind
$p_1$ und $p_2$ Nullteiler in $\mathbb{Z}/n\mathbb{Z}$.
Ein Körper kann also nur entstehen, wenn $n$ eine Primzahl ist.

\begin{definition}
Ist $p$ eine Primzahl, dann heisst $\mathbb{F}_p=\mathbb{Z}/p\mathbb{Z}$
der Galois-Körper der Ordnung $p$.
\end{definition}

Diese Definition ist nur gerechtfertigt, wenn $\mathbb{F}_p^*$ tatsächlich
eine Gruppe ist, wenn also jede Zahl zwischen $1$ und $p-1$ ein Inverses
bezüglich der Multiplikation hat.
Zu einem Rest $a\in\mathbb{F}_p^*$ muss also ein Rest $b$ gefunden werden,
so dass $ab\equiv 1\mod p$.
Dies ist gleichbedeutend mit Zahlen $b$ und $n$ derart, dass
\begin{equation}
ab+np=1.
\label{buch:endliche-koerper:teilerfremd}
\end{equation}
In~\eqref{buch:endliche-koerper:teilerfremd} sind $a$ und $p$ gegeben,
gesucht sind $b$ und $n$.

In Abschnitt~\ref{buch:section:euklid} wurde gezeigt, wie der euklidische
Algorithmus eine Gleichung der Form~\eqref{buch:endliche-koerper:teilerfremd}
lösen kann, wenn die beiden gegebenen Zahlen $a$ und $p$ teilerfremd sind.
Dies ist aber dadurch garantiert, dass $p$ eine Primzahl ist und $1\le a <p$.
Die multiplikative Inverse von $a$ in $\mathbb{F}_p^*$ kann also mit
Hilfe des euklidischen Algorithmus effizient gefunden werden.

\begin{beispiel}
Die kleinste Primzahl grösser als $2021$ ist $p=2063$.
Was ist die Inverse von $2021$ in $\mathbb{F}_{2063}$?

Wir führen den euklidischen Algorithmus für das Paar $(2063,2021)$ durch
und erhalten 
\begin{center}
\begin{tabular}{|>{$}c<{$}|>{$}r<{$}|>{$}r<{$}|>{$}r<{$}|>{$}r<{$}|}
\hline
k&  a_k&  b_k& q_k& r_k\\
\hline
0& 2063& 2021&   1&  42\\
1& 2021&   42&  48&   5\\
2&   42&    5&   8&   2\\
3&    5&    2&   2&   1\\
4&    2&    1&   2&   0\\
\hline
\end{tabular}
\end{center}
Die gesuchten Faktoren $b$ und $n$ können aus dem Matrizenprodukt
$Q(q_n)\dots Q(q_0)$ gefunden werden:
\begin{align*}
Q
&=
\begin{pmatrix} 0& 1\\ 1& -2 \end{pmatrix}
\begin{pmatrix} 0& 1\\ 1& -2 \end{pmatrix}
\begin{pmatrix} 0& 1\\ 1& -8 \end{pmatrix}
\begin{pmatrix} 0& 1\\ 1& -48 \end{pmatrix}
\begin{pmatrix} 0& 1\\ 1& -1 \end{pmatrix}
\\
&=
\begin{pmatrix} 0& 1\\ 1& -2 \end{pmatrix}
\begin{pmatrix} 0& 1\\ 1& -2 \end{pmatrix}
\begin{pmatrix} 0& 1\\ 1& -8 \end{pmatrix}
\begin{pmatrix} 1& -1\\ -48& 49\end{pmatrix}
\\
&=
\begin{pmatrix} 0& 1\\ 1& -2 \end{pmatrix}
\begin{pmatrix} 0& 1\\ 1& -2 \end{pmatrix}
\begin{pmatrix} -48& 49\\ 385& -393 \end{pmatrix}
\\
&=
\begin{pmatrix} 0& 1\\ 1& -2 \end{pmatrix}
\begin{pmatrix} 385& -393\\ -818& 835 \end{pmatrix}
\\
&=
\begin{pmatrix} -818&   835\\ 2021& -2063\end{pmatrix}
\end{align*}
Daraus können wir ablesen, dass
\[
-818\cdot 2021 +835 \cdot 2063=1.
\]
Der Rest $ -818\equiv 1245\mod 2063$ ist also die multiplikative
Inverse von $2021$ in $\mathbb{F}_{2063}$.
\end{beispiel}

\subsubsection{Der kleine Satz von Fermat}
In $\mathbb{Z}$ wachsen die Potenzen einer Zahl immer weiter an.
In einem endlichen Körper kann dies nicht gelten, da nur endlich
viele Werte zur Verfügung stehen.
Tatsächlich müssen die Potenzen einer von $0$ verschiedenen Zahl
$a\in\mathbb{F}_p^*$ alle in $\mathbb{F}_p^*$ liegen.
Es gibt aber nur $p-1$ Zahlen in $\mathbb{F}_p^*$, spätestens
die Potenz mit Exponent $p$ muss also mit einer früheren Potenz
übereinstimmen.
Der kleine Satz von Fermat sagt etwas genauer: die $p$-te Potenz
von $a$ ist genau die Zahl $a$:

\begin{satz}[Kleiner Satz von Fermat]
In $\mathbb{F}_p$ gilt $a^p=a$ für alle $a\in\mathbb{F}_p^*$.
\end{satz}

Wir beweisen diesen Satz in der folgenden, traditionelleren 
Formulierung.

\begin{satz}
Für jede ganze Zahl $a>0$ gilt $p|(a^p-a)$ genau dann, wenn
$p$ eine Primzahl ist.
\end{satz}

\begin{proof}[Beweis]
Wir müssen zeigen, dass $p$ ein Teiler ist von $a^p-a$.
Das nachfolgende kombinatorische Argument wird zum Beispiel
von Mathologor auf seinem Youtube-Kanal im Video
\url{https://youtu.be/_9fbBSxhkuA} illustriert.

Zum Beiweis interpretieren wir die vorkommenden Zahlen kombinatorisch.
Die Zahl $a^p$ ist die Anzahl der verschiedenen Perlenketten der Länge
$p$, die sich aus Glasperlen mit $a$ verschiedenen Farben herstellen
lassen.
Davon bestehen $a$ Perlenketten aus nur einer einzigen Farbe.
Die Zahl $a^p-a$ ist also die Anzahl der Perlenketten der Länge $p$
aus Glasperlen mit $a$ verschiedenen Farben, die mindestens zwei
verschiedene Farben verwenden.

Wir stellen jetzt die Frage nach der Anzahl der geschlossenen
Perlenketten der Länge $p$ als Glasperlen in $a$ verschiedenen Farben.
Aus jeder geschlossenen Perlenkette lassen sich $p$ Perlenketten machen,
indem man sie an einer der $p$ Trennstellen zwischen Perlen aufteilt.

Wir müssen uns noch überlegen, unter welchen Voraussetzungen 
alle diese möglichen Auftrennungen zu verschiedenen Perlenketten
führen.
Zwei Trennstellen, die $k$-Perlen auseinander liegen, führen nur dann
zur gleichen Perlenkette, wenn die geschlossenen Ketten durch Drehung 
um $k$ Perlen ineinander umgehen.
Dies bedeutet aber auch, dass sich das Farbmuster alle $k$-Perlen
wiederholen muss.
Folglich ist $k$ ein Teiler von $p$.
$p$ Verschiedene Perlenketten entstehen also immer genau dann, wenn $p$
eine Primzahl ist.

Wir schliessen daraus, dass $a^p-a$ durch $p$ teilbar ist, genau dann,
wenn $p$ eine Primzahl ist.
\end{proof}

Der kleine Satz von Fermat kann auch dazu verwendet werden, Potenzen 
in $\mathbb{F}_p$ zu vereinfachen, wie das folgende Beispiel\footnote{%
Das Beispiel stammt aus dem Video~\url{https://youtu.be/_9fbBSxhkuA},
welches Mathologer zu Halloween 2018 veröffentlich hat}
zeigt.

\begin{beispiel}
Man berechnet in $\mathbb{F}_{13}$ die Potenz $11^{666}$.
Nach dem kleinen Satz von Fermat ist $11^{13} = 11$ oder $11^{12}=1$,
man kann also den Exponenten modulo $12$ reduzieren.
Weil $666=55\cdot 12 + 6$ erhält man $11^{666}= 11^5$.
Da die Potenzen von $11$ etwas mühsam zu berechnen sind,
kann man sie wegen $11=-2$ in $\mathbb{F}_{13}$ auch als Potenzen
von $-2$ bekommen.
Aber $(-2)^6 = 64 = -1 \in\mathbb{F}_{13}$.
\end{beispiel}

In der Form $a^{p-1}=1$ in $\mathbb{F}_p$ liefert der kleine Satz
von Fermat die Inverse von $a$ als $a^{p-2}$.
Dies bedeutet zum Beispiel, dass in $\mathbb{F}_3$ jede von $0$
verschiedene Zahl zu sich selbst invers ist: $1\cdot 1=1$ und $2\cdot 2=1$.
Diese Art, die Inverse zu bestimmen, ist allerdings nicht effizienter
als der euklidische Algorithmus, aber sie ist manchmal für
theoretische Überlegungen nützlich.

\subsubsection{Der Satz von Wilson}
Der Satz von Wilson ermöglicht, die multiplikative Inverse auf eine
andere Art zu berechnen.
Sie ist zwar nicht unbedingt einfacher, aber manchmal nützlich für
theoretische Überlegungen.

\begin{satz}[Wilson]
Die ganze Zahl $p\ge 2$ ist genau dann eine Primzahl, wenn
$(p-1)!\equiv -1\mod p$.
\end{satz}

Mit dem Satz von Wilson kann man die Inverse einer beliebigen Zahl
$a\in\mathbb{F}_p$ finden.
Dazu verwendet man, dass $a$ einer der Faktoren in $(p-1)!$ ist.
Lässt man diesen Faktor weg, erhält man eine Zahl
\[
b = 1\cdot 2 \cdot \ldots\cdot \hat{a}\cdot\ldots\cdot (p-1),
\]
wobei der Hut bedeutet, dass der Faktor $a$ weggelassen werden soll.
Nach dem Satz von Wilson ist $ab=-1$ in $\mathbb{F}_p$, also ist
$-b$ die multiplikative Inverse  von $a$.

\begin{beispiel}
Die Inverse von $2\in\mathbb{F}_7$ ist
\begin{align*}
a^{-1}
&=
-\underbrace{1\cdot 3\cdot 4}_{}\cdot \underbrace{5\cdot 6}_{}
\\
&=
-5\cdot 2
=
-3
=4
\end{align*}
Tatsächlich ist $2\cdot 4=8\equiv 1\mod 7$.
\end{beispiel}

%
% Charakteristik
%
\subsection{Charakteristik
\label{buch:subsection:charakteristik}}

\subsubsection{Frobenius-Homomorphismus}




