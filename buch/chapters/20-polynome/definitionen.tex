%
% definitionen.tex -- Definition für das Kapitel über Polynome
%
% (c) 2021 Prof Dr Andreas Müller, OST Ostschweizer Fachhochschule
%
\section{Definitionen
\label{buch:section:polynome:definitionen}}
\rhead{Definitionen}
In diesem Abschnitt stellen wir einige grundlegende Definitionen für das
Rechnen mit Polynomen zusammen.

%
% Skalare
%
\subsection{Polynome
\label{buch:subsection:polynome:polynome}}
Wie schon in der Einleitung angedeutet sind Polynome nur dann sinnvoll,
wenn man mit den Koeffizienten gewisse Rechenoperationen durchführen kann.
Wir brauchen mindestens die Möglichkeit, Koeffizienten zu addieren.
Wenn wir uns vorstellen, dass wir $X$ durch eine Zahl ersetzen können,
dann brauchen wir zusätzlich die Möglichkeit, einen Koeffizienten mit einer
Zahl zu multiplizieren.

Die Struktur, die wir hier beschrieben haben, hängt davon ab, was wir uns
unter einer ``Zahl'' vorstellen.
Wir bezeichnen die Menge, aus der die ``Zahlen'' kommen können mit $R$ und
nennen sie die Menge der Skalare.
\index{Skalar}%
Wenn wir uns vorstellen, dass man die Elemente von $R$ an Stelle von $X$
in das Polynom einsetzen kann, dann muss es möglich sein, in $R$ zu
Multiplizieren und zu Addieren, und es müssen die üblichen Rechenregeln
der Algebra gelten, $R$ muss also ein Ring sein.
\index{Ring}%
Wir werden im folgenden zusätzlich voraussetzen, dass $R$ sogar kommutativ
ist und eine $1$ hat.

\begin{definition}
Sei $R$ ein Ring.
Die Menge
\[
R[X]
=
\{
p(X) = a_nX^n+a_{n-1}X^{n-1} + \dots a_1X+a_0\;|\; a_k\in R, n\in\mathbb{N}
\}
\]
heisst die Menge der {\em Polynome} mit Koeffizienten in $R$
oder
{\em Polynome über} $R$.
\index{Polynome über $R$}%
Polynome können addiert werden, indem Koeffizienten mit gleichem Index
addiert werden:
\begin{align*}
p(X) &= a_nX^n + a_{n-1}X^{n-1} + \dots + a_1X + a_0\\
q(X) &= b_nX^n + b_{n-1}X^{n-1} + \dots + b_1X + b_0\\
p(X)+q(X)
&=
(a_n+b_n)X^n
+
(a_{n-1}+b_{n-1})X^{n-1}
+
\dots
+
(a_1+b_1)X
+
(a_0+b_0)
\end{align*}
Die Multiplikation ist durch die Formel~\eqref{buch:eqn:polynome:faltung}
definiert.
\end{definition}

Ein Polynom heisst {\em normiert} oder auch {\em monisch}, wenn der
\index{Polynom!normiert}%
\index{normiertes Polynom}%
\index{Polynom!monisch}%
\index{normiertes Polynom}
höchste Koeffizient oder auch {\em Leitkoeffizient} des Polynomus $1$ ist,
also $a_n=1$.
\index{Leitkoeffizient}%
Wenn man in $R$ durch $a_n$ dividieren kann, dann kann man aus dem Polynom
$p(X)=a_nX^n+\dots$ mit Leitkoeffizient $a_n$ das normierte Polynom
\[
\frac{1}{a_n}p(X) = \frac{1}{a_n}(a_nX^n + \dots + a_0)=
X^n + \frac{a_{n-1}}{a_n}X^{n-1} + \dots + \frac{a_0}{a_n}
\]
machen.
Man sagt auch, das Polynom $p(X)$ wurde normiert.
Wenn $R$ ein Körper ist, ist die Normierung immer möglich.

Die Tatsache, dass zwei  Polynome nicht gleich viele von $0$ verschiedene Koeffizienten haben müssen,
verkompliziert die Beschreibung der Rechenoperationen ein wenig.
Wir werden daher im Folgenden oft für ein Polynom
\[
p(X)
=
a_nX^n + a_{n-1}X^{n-1} + \dots a_1X+a_0
\]
annehmen, dass alle Koeffizienten $a_{n+1},a_{n+2},\dots$ implizit mit
Wert $0$ definiert sind.
Wir werden uns also erlauben,
\[
p(X)
=
\sum_{k}a_kX^k
=
\sum_{k=0}^\infty a_kX^k
\]
zu schreiben, wobei in der ersten Form das Summenzeichen bedeuten soll,
dass nur über diejenigen Indizes $k$ summiert wird, für die $a_k$
definiert ist.
\label{summenzeichenkonvention}

Die Menge $R[X]$ aller Polynome über $R$ mit den beschriebenen
Operationen ist ein Ring. 
Das Distributivgesetz
\index{Distributivgesetz}%
\[
p(X)(u(X)+v(X)) = p(X)u(X) + p(X)v(X)
\qquad
(p(X)+q(X)) u(X) = p(X)u(X) + q(X)u(X)
\]
zum Beispiel sagt ja nichts anderes, als dass man ausmultiplizieren
kann.
\index{ausmultiplizieren}%
Oder die Assoziativgesetze
\begin{align*}
p(X)+q(X)+r(X)
&=
(p(X)+q(X))+r(X)
=
p(X)+(q(X)+r(X))
\\
p(X)q(X)r(X)
&=
(p(X)q(X))r(X)
=
p(X)(q(X)r(X))
\end{align*}
für die Multiplikation besagt, das es keine Rolle spielt, in welcher
Reihenfolge man die Additionen oder Multiplikationen ausführt.

%
% Der Grad eines Polynoms
%
\subsection{Grad
\label{buch:subsection:polynome:grad}}

\begin{definition}
Der {\em Grad} eines Polynoms $p(X)$ ist die höchste Potenz von $X$, die im
Polynom vorkommt.
\index{Grad eines Polynoms}%
Das Polynom
\[
p(X) = a_nX^n + a_{n-1}X^{n-1}+\dots a_1X + a_0
\]
hat den Grad $n$, wenn $a_n\ne 0$ ist.
Der Grad von $p$ wird mit $\deg p$ bezeichnet.
Konstante Polynome $p(X)=a_0$ mit $a_0\ne 0$ hat den Grad $0$.
Der Grad des Nullpolynoms $p(X)=0$ ist definiert als
$-\infty$.
\end{definition}

Der Grad eines Polynoms ist sinnvoll in dem Sinn, dass er sich mit
den Rechenoperationen gut verträgt.
Damit lässt sich weiter unten auch die spezielle Wahl des Grades
des Nullpolynoms begründen.
Es gelten nämlich die folgenden Rechenregeln.

\begin{lemma}
\label{lemma:rechenregelnfuerpolynomgrad}
Sind $p$ und $q$ Polynome mit Koeffizienten in $R$ und $0\ne \lambda\in R$,
dann gilt
\begin{align}
\deg(pq) &\le \deg p + \deg q
\label{buch:eqn:polynome:gradsumme}
\\
\deg(p+q) &\le \max(\deg p, \deg q)
\label{buch:eqn:polynome:gradprodukt}
\\
\deg(\lambda p) &\le \deg p
\label{buch:eqn:polynome:gradskalar}
\end{align}
\end{lemma}

Die Formel \eqref{buch:eqn:polynome:gradskalar} ist eigentlich
ein Spezialfall von \eqref{buch:eqn:polynome:gradsumme}.
Die Zahl $\lambda\in R$ kann man als Polynom vom Grad $0$ betrachten,
wofür natürlich \eqref{buch:eqn:polynome:gradsumme} gilt, also
$\deg(\lambda p) \le \deg\lambda + \deg p$.

\begin{proof}[Beweis]
Wir schreiben die Polynome wieder in der Form
\[
\begin{aligned}
p(X) &= a_nX^n + a_{n-1}X^{n-1} + \dots + a_1X + a_0&&\Rightarrow&\deg p&=n\\
q(X) &= b_mX^m + b_{m-1}X^{m-1} + \dots + b_1X + b_0&&\Rightarrow&\deg q&=m.
\end{aligned}
\]
Dann kann der höchste Koeffizient in der Summe $p+q$ nicht weiter oben
sein als die grössere von den beiden Zahlen $n$ und $m$ angibt, dies
beweist \eqref{buch:eqn:polynome:gradsumme}.
Ebenso kann der höchste Koeffizient im Produkt nach der
Formel~\eqref{buch:eqn:polynome:faltung} nicht weiter oben als bei
$n+m$ liegen, dies beweist
beweist \eqref{buch:eqn:polynome:gradprodukt}.
In einem Ring mit Nullteilern
(Siehe Definition~\ref{buch:grundlagen:def:nullteiler})
könnte es passieren, dass $a_nb_m=0$ ist, d.~h.~es ist durchaus möglich,
dass der Grad kleiner ist.
Schliesslich kann der höchsten Koeffizient von $\lambda p(X)$ nicht grösser
als der höchste Koeffizient von $p(X)$ sein, was
\eqref{buch:eqn:polynome:gradskalar} beweist.
\end{proof}

In einem nullteilerfreien Ring gelten die Rechenregeln für den Grad
jetzt exakt:

\begin{lemma}
Sei $R$ ein nullteilerfreier Ring und $p$ und $q$ Polynome über $R$
und $0\ne \lambda\in R$.
Dann gilt
\begin{align}
\deg(pq) &= \deg p + \deg q
\label{buch:eqn:polynome:gradsummeexakt}
\\
\deg(p+q) &\le \max(\deg p, \deg q)
\label{buch:eqn:polynome:gradproduktexakt}
\\
\deg(\lambda p) &= \deg p
\label{buch:eqn:polynome:gradskalarexakt}
\end{align}
\end{lemma}

\begin{proof}[Beweis]
Der Fall, dass der höchste Koeffizient verschwindet, weil $a_n$, $b_m$
oder $\lambda$ Nullteiler sind, kann unter den gegebenen Voraussetzungen
nicht eintreten, daher werden die in
Lemma~\ref{lemma:rechenregelnfuerpolynomgrad} gefunden Ungleichungen
für Produkte exakt.
\end{proof}

Die Gleichung
\eqref{buch:eqn:polynome:gradskalarexakt}
kann im Fall $\lambda=0$ natürlich nicht gelten.
Betrachten wir $\lambda$ wieder als ein Polynom, dann folgt aus
\eqref{buch:eqn:polynome:gradsummeexakt}, dass
\[
\begin{aligned}
\lambda&\ne 0  &&\Rightarrow& \deg (\lambda p) &= \deg\lambda + \deg p = 0+\deg p
\\
\lambda&=0     &&\Rightarrow& \deg (0 p) &= \deg 0 + \deg p = \deg 0
\end{aligned}
\]
Diese Gleichung kann also nur aufrechterhalten werden, wenn die ``Zahl'' $\deg 0$ die Eigenschaft besitzt, dass man immer noch $\deg 0$ bekommt,
wenn man irgend eine Zahl $\deg p$ hinzuaddiert. Wenn also
\[\deg 0 + \deg p = \deg 0 \qquad \forall \deg p \in \mathbb Z\]
gilt.
So eine Zahl gibt es in den ganzen Zahlen nicht.
Wenn man zu einer ganzen Zahl eine andere ganze Zahl hinzuaddiert, ändert sich fast immer etwas.
Man muss daher $\deg 0 = -\infty$ setzen und festlegen, dass
$-\infty + n = -\infty$ für beliebige ganze Zahlen $n$ gilt.

\begin{definition}
\label{buch:def:definitionen:polynomfilterung}
Die Polynome vom Grad $\le n$ mit Koeffizienten in $R$
bilden die Teilmenge
\[
R^{(n)}[X]
=
\{ p\in R[X]\;|\; \deg p \le n\}.
\]
Die Mengen $R^{(n)}[X]$ bilden eine {\em Filtrierung} des Polynomrings
$R[X]$, d.~h.~sie sind ineinander geschachtelt
\[
\arraycolsep=4pt
\begin{array}{ccccccccccccccc}
R^{(-\infty)}[X] & \subset
	& R^{(0)}[X] & \subset
		& R^{(1)}[X] & \subset & \dots & \subset
			& R^{(k)}[X] & \subset
				& R^{(k+1)}[X] & \subset & \dots & \subset
					& R[X]\\[3pt]
\bigg\| &
	&\bigg\| &
		&\bigg\| & & &
			&&
				&& & &
					&
\\[3pt]
\{0\} & \subset
	& R & \subset
		& \{a_1X+a_0\;|a_k\in R\} & \subset & \dots &
\end{array}
\]
und ihre Vereinigung ist $R[X]$.
\end{definition}

Die Formeln für den Grad können wir auch mit den Mengen $R^{(k)}[X]$
ausdrücken:
\begin{align*}
\deg (p+q) &\le \max(\deg p, \deg q)
&&\Rightarrow&
R^{(k)}+R^{(l)}
&\subset R^{(\max(k,l))}
=
R^{(k)}[X] \cup R^{(l)}[X].
\\
\deg (p\cdot q)&=\deg p+\deg q
&&\Rightarrow&
R^{(k)}[X] \cdot R^{(l)}[X]
&=
R^{(k+l)}[X].
\end{align*}


%
% Abschnitt über Teilbarkeit
%
\subsection{Teilbarkeit
\label{buch:subsection:polynome:teilbarkeit}}
Im Ring der ganzen Zahlen sind nicht alle Divisionen ohne Rest
ausführbar, so entsteht das Konzept der Teilbarkeit.
Der Divisionsalgorithmus, den man in der Schule lernt, liefert
\index{Divisionsalgorithmus}%
zu beliebigen ganzen Zahlen $a,b\in\mathbb{Z}$ den Quotienten
$q$ und den Rest $r$ derart, dass $a=qb+r$.
Der Algorithmus basiert auf der Zehnersystemdarstellung 
\begin{align*}
a &= a_n10^{n} + a_{n-1}10^{n-1} + \dots + a_110^{1} + a_0
\\
b &= b_m10^{n} + b_{m-1}10^{n-1} + \dots + b_110^{1} + b_0
\end{align*}
und ermittelt den Quotienten, indem er mit den einzelnen Stellen
$a_k$ und $b_k$ arbeitet.
Er ist also eigentlich ein Algorithmus für die Polynome
\begin{align*}
a &= a_nX^{n} + a_{n-1}X^{n-1} + \dots + a_1X^{1} + a_0
\\
b &= b_mX^{n} + b_{m-1}X^{n-1} + \dots + b_1X^{1} + b_0,
\end{align*}
mit dem einzigen Unterschied, dass statt $X$ mit der festen Zahl $X=10$
gearbeitet wird.
Der Divisionsalgorithmus für Polynome lässt sich aber leicht
rekonstruieren.

\subsubsection{Polynomdivision}
Wir zeigen den Polynomdivisionsalgorithmus an einem konkreten Beispiel.
\index{Polynomdivision}%
Gesucht sind Quotient $q\in \mathbb{Z}[X]$ und Rest $r\in\mathbb{Z}[X]$
der beiden Polynome
\begin{equation}
\begin{aligned}
a(X) &= X^4 - X^3 -7X^2 + X + 6\\
b(X) &= X^2+X+1,
\end{aligned}
\label{buch:polynome:eqn:divisionsaufgabe}
\end{equation}
für die also gilt $a=bq+r$.
Die Division ergibt
\[
\arraycolsep=1.4pt
\begin{array}{rcrcrcrcrcrcrcrcrcrcr}
X^4&-& X^3&-&7X^2&+& X&+&6&:&X^2&+&X&+&1&=&X^2&-&2X&-&6=q\\
\llap{$-($}X^4&+& X^3&+& X^2\rlap{$)$}& &  & & & &   & & & & & &   & &  & & \\ \cline{1-5}
   &-&2X^3&-&8X^2&+& X& & & &   & & & & & &   & &  & & \\
   &\llap{$-($}-&2X^3&-&2X^2&-&2X\rlap{$)$}& & & &   & & & & & &   & &  & & \\ \cline{2-7}
   & &    &-&6X^2&+&3X&+&6& &   & & & & & &   & &  & & \\
   & &    &\llap{$-($}-&6X^2&-&6X&-&6\rlap{$)$}& &   & & & & & &   & &  & & \\ \cline{4-9}
   & &    & &    & &9X&+&12\rlap{$\mathstrut=r$}& &   & & & & & &   & &  & & \\ \cline{7-9}
\end{array}
\]
Durch Nachrechnen kann man überprüfen, dass tatsächlich
\begin{align*}
bq
&=
X^4-X^3-7X^2-8X-6
\\
bq+r&=
X^4-X^3-7X^2+X+6 = a
\end{align*}
gilt.

Das Beispiel~\eqref{buch:polynome:eqn:divisionsaufgabe} war besonders
einfach, weil der führende Koeffizient des Divisorpolynomes $1$ war.
Für $b=2X^2+X+1$ funktioniert der Algorithmus dagegen nicht mehr.
Jedes für $q$ in Frage kommende Polynom vom Grad $2$ muss von der
Form $q=q_2X^2+q_1X+q_0$ sein.
Multipliziert man mit $b$, erhält man $bq=2q_2X^4 + (2q_1+q_2)X^3+\dots$.
Insbesondere ist es nicht möglich mit ganzzahligen Quotienten
$q_k\in\mathbb{Z}$ auch nur den ersten Koeffizienten von $a$ zu
erhalten.
Dazu müsste nämlich $a_n = 1 = 2q_2$ oder $q_2 = \frac12\not\in\mathbb{Z}$
sein.
Der Divisionsalgorithmus funktioniert also nur dann, wenn die 
Division durch den führenden Koeffizienten des Divisorpolynomes $b$ 
immer ausführbar ist.
Im Beispiel~\eqref{buch:polynome:eqn:divisionsaufgabe} war das der
Fall, weil der führende Koeffizient $1$ war.
Für beliebige Polynome $b\in R[X]$ ist dies aber nur dann immer der Fall,
wenn die Koeffizienten in Tat und Wahrheit einem Körper entstammen.

Im Folgenden betrachten wir daher nur noch Polynomringe mit Koeffizienten
in einem Körper $\Bbbk$.
In $\mathbb{Q}[X]$ ist die Division $a:b$ für die Polynome
\begin{equation}
\begin{aligned}
a(X) &= X^4 - X^3 -7X^2 + X + 6\\
b(X) &= X^2+X+1,
\end{aligned}
\label{buch:polynome:eqn:divisionsaufgabe}
\end{equation}
problemlos durchführbar:
\[
\arraycolsep=1.4pt
\renewcommand{\arraystretch}{1.2}
\begin{array}{rcrcrcrcrcrcrcrcrcrcr}
X^4&-&       X^3&-&         7X^2&+&          X&+&           6&:&2X^2&+&X&+&1&=&\frac12X^2&-&\frac34X&-\frac{27}{8} = q\\
\llap{$-($}X^4&+&\frac12X^3&+&   \frac12X^2\rlap{$)$}& &           & &            & &    & & & & & &          & &        &             \\ \cline{1-5}
   &-&\frac32X^3&-&\frac{15}2X^2&+&          X& &            & &    & & & & & &          & &        &             \\
   &\llap{$-($}-&\frac32X^3&-&\frac{ 3}4X^2&-&\frac{ 3}4X\rlap{$)$}& &            & &    & & & & & &          & &        &             \\\cline{2-7}
   & &          &-&\frac{27}4X^2&+&\frac{ 7}4X&+&           6& &    & & & & & &          & &        &             \\
   & &          &\llap{$-($}-&\frac{27}4X^2&-&\frac{27}8X&-&\frac{27}{8}\rlap{$)$}& &    & & & & & &          & &        &             \\\cline{4-9}
   & &          & &             & &\frac{41}8X&+&\frac{75}{8}\rlap{$\mathstrut=r$}& &    & & & & & &          & &        &             \\
\end{array}
\]
Der Algorithmus funktioniert selbstverständlich genauso in $\mathbb{R}[X]$
oder $\mathbb{C}[X]$, und ebenso in den in
Kapitel~\ref{buch:chapter:endliche-koerper} studierten endlichen Körpern.

\subsubsection{Euklidische Ringe und Faktorzerlegung}
Der Polynomring $\Bbbk[X]$ hat noch eine weitere Eigenschaft, die ihn
von einem gewöhnlichen Ring unterschiedet.
Der Polynomdivisionsalgorithmus findet zu zwei Polynomen $f,g\in\Bbbk[X]$
den Quotienten $q\in\Bbbk[X]$ und den Rest $r\in\Bbbk[X]$ mit
$f=qg+r$, wobei ausserdem $\deg r<\deg g$ ist.

\begin{definition}
Ein {\em euklidischer Ring} $R$ ist ein nullteilerfreier Ring mit einer
\index{euklischer Ring}%
Gradfunktion $\deg\colon R\setminus\{0\}\to\mathbb{N}$ mit folgenden
Eigenschaften
\begin{enumerate}
\item Für $x,y\in R$ gilt $\deg(xy) \ge \deg(x)$.
\item Für alle $x,y\in R$ gibt es $q,r\in R$ mit $x=qy+r$ mit
$\deg(y)>\deg(x)$
\label{buch:20-polynome:def:euklidischerring-2}
\end{enumerate}
Bedingung~\ref{buch:20-polynome:def:euklidischerring-2} ist die
{\em Division mit Rest}.
\index{Gradfunktion}%
\index{Division mit Rest}%
\index{euklidischer Ring}%
\end{definition}

Die ganzen Zahlen $\mathbb{Z}$ bilden einen euklidischen Ring mit der 
Gradfunktion $\deg(z)=|z|$ für $z\in \mathbb{Z}$.
Aus dem Divisionsalgorithmus für ganze Zahlen leiten sich alle grundlegenden
Eigenschaften über Teilbarkeit und Primzahlen ab.
Eine Zahl $x$ ist teilbar durch $y$, wenn $x=qy$ mit $q\in \mathbb{Z}$,
es gibt Zahlen $p\in\mathbb{Z}$, die keine Teiler haben und jede Zahl
kann auf eindeutige Art und Weise in ein Produkt von Primfaktoren
zerlegt werden.

\subsubsection{Irreduzible Polynome}
Das Konzept der Primzahl lässt sich wie folgt in den Polynomring übertragen.
\index{Primzahl}%

\begin{definition}
Ein Polynom $f\in R[X]$ heisst irreduzibel, wenn es keine Faktorisierung $f=gh$
in Faktoren $g,h\in R[X]$ mit $\deg(g)>0$ und $\deg(h) >0$ gibt.
\index{irreduzibles Polynom}%
\end{definition}

\begin{beispiel}
Polynome ersten Grades $aX+b$ sind immer irreduzibel, da sie bereits
minimalen Grad haben.

Sei jetzt $f=X^2+bX+c$ ein quadratisches Polynom in $\mathbb{Q}[X]$.
Wenn es faktorisierbar sein soll, dann müssen die Faktoren Polynome
ersten Grades sein, also $f=(X-x_1)(X-x_2)$ mit $x_i\in\mathbb{Q}$.
Die Zahlen $x_i$ die einzigen möglichen Lösungen für $x_i$ können mit
der Lösungsformel für die quadratische Gleichung
\[
x_i = -\frac{b}2\pm\sqrt{\frac{b^2}{4}-c}
\]
gefunden werden.
Die Faktorisierung ist also genau dann möglich, wenn $b^2/4-c$ ein 
Quadrat in $\mathbb{Q}$ ist.
In $\mathbb{R}$ ist das Polynom faktorisierbar, wenn $b^2-4c\ge 0$ ist.
In $\mathbb{C}$ gibt es keine Einschränkung, die Wurzel zu ziehen,
in $\mathbb{C}$ gibt es also keine irreduziblen Polynome im Grad $2$.
\end{beispiel}

\subsubsection{Faktorisierung in einem Polynomring}
Ein Polynomring ist ganz offensichtlich auch ein euklidischer Ring.
Wir erwarten daher die entsprechenden Eigenschaften auch in einem
Polynomring.
Allerdings ist eine Faktorzerlegung nicht ganz eindeutig.
Wenn das Polynom $f\in\mathbb{Z}[X]$ die Faktorisierung
$f=g\cdot h$ mit $g,h\mathbb{Z}[X]$ hat, dann
ist $rg\cdot r^{-1}h$ ebenfalls eine Faktorisierung für jedes $r =\pm1$.
Dasselbe gilt in $\mathbb{Q}$ für jedes $r\in \mathbb{Q}^*$.
Faktorisierung ist also nur eindeutig bis auf Elemente der
Einheitengruppe des Koeffizientenringes.
Diese Mehrdeutigkeit kann in den Polynomringen $\Bbbk[X]$ 
überwunden werden, indem die Polynome normiert werden.

\begin{satz}
Ein normiertes Polynom $f\in \Bbbk[X]$ kann in
normierte Faktoren $g_1,\dots,g_k\in\Bbbk[X]$ zerlegt werden, so dass
$f=g_1\cdot\ldots\cdot g_k$, wobei die Faktoren irreduzibel sind.
Zwei solche Faktorisierungen unterscheiden sich nur durch die Reihenfolge
der Faktoren.
Ein Polynom $f\in \Bbbk[X]$ kann in ein Produkt $a_n g_1\cdot\ldots\cdot g_k$
zerlegt werden, wobei die normierten Faktoren $g_i$ bis auf die Reihenfolge
eindeutig sind.
\end{satz}


