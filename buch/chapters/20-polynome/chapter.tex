%
% chapter.tex -- Kapitel über Polynome
%
% (c) 2021 Prof Dr Andreas Müller, OST Ostschweizer Fachhochschule
%
\chapter{Polynome
\label{buch:chapter:polynome}}
\lhead{Polynome}
Ein {\em Polynom} ist ein Ausdruck der Form
\index{Polynom}%
\begin{equation}
p(X) = a_nX^n+a_{n-1}X^{n-1} + \cdots a_2X^2 + a_1X + a_0.
\label{buch:eqn:polynome:polynom}
\end{equation}
Ursprünglich stand das Symbol $X$ als Platzhalter für eine Zahl.
Die Polynomgleichung $Y=p(X)$ drückt dann einen Zusammenhang zwischen
den Grössen $X$ und $Y$ aus.
Zum Beispiel drückt
\begin{equation}
H = -\frac12gT^2 + v_0T +h_0 = p(T)
\label{buch:eqn:polynome:beispiel}
\end{equation}
im Schwerefeld der Erde nahe der Oberfläche einen Zusammenhang
zwischen der Zeit $T$ und der Höhe $H$ eines frei fallenden Körpers aus.
Setzt man einen Wert für $T$ in \eqref{buch:eqn:polynome:beispiel} ein,
erhält man den zugehörigen Wert für $H$.
Man stellt sich hier also vor, dass $T$ eigentlich eine Zahl ist und dass
\eqref{buch:eqn:polynome:polynom}
nur ein ``unfertiger'' Ausdruck oder ein ``Programm'' für eine Berechnung
ist.
In dieser eher arithmetischen Sichtweise ist es aber eigentlich egal, dass in
\index{arithmetische Sichtweise}%
\eqref{buch:eqn:polynome:polynom} nur einfache Multiplikationen und
Additionen vorkommen.
In einem Programm könnten ja auch beliebig komplizierte Operationen
verwendet werden, warum also diese Beschränkung?

Für die nachfolgenden Betrachtungen stellen wir uns $X$ daher nicht
mehr einfach als einen Platzhalter für eine Zahl vor, sondern als ein neues
algebraisches Objekt, für das man die Rechenregeln erst noch definieren muss.
In diesem Kapitel sollen die Regeln zum Beispiel sicherstellen,
dass man mit Polynomen so rechnen kann, wie wenn $X$ eine Zahl wäre.
Es sollen also zum Beispiel die Regeln
\begin{align}
aX&=Xa
&
(a+b)X&=aX+bX
&
a+X &= X+a
\label{buch:eqn:polynome:basic}
\end{align}
gelten.
In dieser algebraischen Sichtweise können je nach den gewählten algebraischen
Rechenregeln für $X$ interessante rechnerische Strukturen abgebildet werden.
\index{algebraische Sichtweise}%
Ziel dieses Kapitels ist zu zeigen, wie man die Rechenregeln für $X$
mit Hilfe von Matrizen allgemein darstellen kann.
Diese Betrachtungsweise wird später in Anwendungen ermöglichen,
handliche Realisierungen für das Rechnen mit Grössen zu finden,
die polynomielle Gleichungen erfüllen.
Ebenso sollen in späteren Kapiteln die Regeln
\eqref{buch:eqn:polynome:basic}
erweitert oder abgelöst werden um weitere Anwendungen zu erschliessen.

Bei der Auswahl der zusätzlichen algebraischen Regeln muss man sehr
vorsichtig vorgehen.
Nimmt man zum Beispiel an, dass man durch $X$ teilen kann, dann würde
dies in der arithmetischen Sichtweise bereits ausschliessen, dass man
für $X$ die Zahl $0$ einsetzen kann.
Aber auch eine Regel wie $X^2 \ge 0$, die für alle reellen Zahlen gilt,
würde die Anwendungsmöglichkeiten zu stark einschränken.
Es gibt zwar keine reelle Zahl, die man in das Polynom $p(X)=X^2+1$
einsetzen könnte, so dass es den Wert $0$ annimmt.
Man könnte $X$ aber als ein neues Objekt ausserhalb von $\mathbb{R}$
betrachten, welches die Gleichung $X^2+1=0$ erfüllt.
In den komplexen Zahlen $\mathbb{C}$ gibt es mit der imaginären
Einheit $i\in\mathbb{C}$ tatsächlich ein Zahl mit der Eigenschaft
$i^2=-1$ und damit eine Objekt, welches die Ungleichung $X^2\ge 0$
verletzt.

Für das Symbol $X$ sollen also die ``üblichen'' Rechenregeln gelten.
Dies ist natürlich nur sinnvoll, wenn man auch mit den Koeffizienten
$a_0,\dots,a_n$ rechnen kann.
Sie müssen also Elemente einer
algebraischen Struktur sein, in der mindestens die Addition und die
Multiplikation definiert sind.
Die ganzen Zahlen $\mathbb{Z}$ kommen dafür in Frage, aber auch
die rationalen oder reellen Zahlen $\mathbb{Q}$ und $\mathbb{R}$.
Man kann sogar noch weiter gehen: man kann als Koeffizienten auch
Vektoren oder sogar Matrizen zulassen.
Polynome können addiert werden, indem die Koeffizienten addiert werden,
und sie können mit Skalaren aus dem Koeffizentenkörper multipliziert werden.
Polynome können aber auch multipliziert werden, was auf die Faltung
der Koeffizienten hinausläuft:
\begin{align}
p(X) &= a_nX^n + a_{n-1}X^{n-1} + \dots + a_1X + a_0
\notag
\\
q(X) &= b_mX^m + b_{m-1}X^{m-1} + \dots + b_1X + b_0
\notag
\\
p(X) q(X) &=
a_{n}b_{m}X^{n+m}
+
(a_{n}b_{m-1}+a_{n-1}b_{m})X^{n+m-1}
+
\ldots
+
(a_1b_0+a_0b_1)X
+
a_0b_0
\\
&=
\sum_{i + j = k}a_ib_j X^k.
\label{buch:eqn:polynome:faltung}
\end{align}
Dies ist aber nur möglich, wenn die Koeffizienten selbst miteinander
multipliziert werden können, wenn also die Koeffizienten mindestens
Elemente einer Algebra sind.

%
% definitionen.tex -- Definition für das Kapitel über Polynome
%
% (c) 2021 Prof Dr Andreas Müller, OST Ostschweizer Fachhochschule
%
\section{Definitionen
\label{buch:section:polynome:definitionen}}
\rhead{Definitionen}
In diesem Abschnitt stellen wir einige grundlegende Definitionen für das
Rechnen mit Polynomen zusammen.

%
% Skalare
%
\subsection{Skalare
\label{buch:subsection:polynome:skalare}}
Wie schon in der Einleitung angedeutet sind Polynome nur dann sinnvoll,
wenn man mit den Koeffizienten gewisse Rechenoperationen durchführen kann.
Wir brauchen mindestens die Möglichkeit, Koeffizienten zu addieren.
Wenn wir uns vorstellen, dass wir $X$ durch eine Zahl ersetzen können,
dann brauchen wir zusätzlich die Möglichkeit, einen Koeffizienten mit einer
Zahl zu multiplizieren.

Die Struktur, die wir hier beschrieben haben, hängt davon ab, was wir uns
unter einer ``Zahl'' vorstellen.
Wir bezeichnen die Menge, aus der die ``Zahlen'' kommen können mit $R$ und
nennen sie die Menge der Skalare.
\index{Skalar}%
Wenn wir uns vorstellen, dass man die Elemente von $R$ an Stelle von $X$
in das Polynom einsetzen kann, dann muss es möglich sein, in $R$ zu
Multiplizieren und zu Addieren, und es müssen die üblichen Rechenregeln
der Algebra gelten, $R$ muss also ein Ring sein.
\index{Ring}%
Wir werden im folgenden meistens voraussetzen, dass $R$ sogar kommutativ
ist und eine $1$ hat.

\begin{definition}
Sei $R$ ein Ring.
Die Menge
\[
R[X]
=
\{
p(X) = a_nX^n+a_{n-1}X^{n-1} + \dots a_1X+a_0\;|\; a_k\in R, n\in\mathbb{N}
\}
\]
heisst die Menge der {\em Polynome} mit Koeffizienten in $R$
oder
{\em Polynome über} $R$.
\index{Polynome über $R$}%
Polynome können addiert werden, indem Koeffizienten mit gleichem Index
addiert werden:
\begin{align*}
p(X) &= a_nX^n + a_{n-1}X^{n-1} + \dots + a_1X + a_0\\
q(X) &= b_nX^n + b_{n-1}X^{n-1} + \dots + b_1X + b_0\\
p(X)+q(X)
&=
(a_n+b_n)X^n
+
(a_{n-1}+b_{n-1})X^{n-1}
+
\dots
+
(a_1+b_1)X
+
(a_0+b_0)
\end{align*}
Die Multiplikation ist durch die Formel~\eqref{buch:eqn:polynome:faltung}
definiert.
\end{definition}

Ein Polynom heisst {\em normiert} oder auch {\em monisch}, wenn der
\index{Polynom!normiert}%
\index{normiertes Polynom}%
\index{Polynom!monisch}%
\index{normiertes Polynom}
höchste Koeffizient oder auch {\em Leitkoeffizient} des Polynomus $1$ ist,
also $a_n=1$.
\index{Leitkoeffizient}%
Wenn man in $R$ durch $a_n$ dividieren kann, dann kann man aus dem Polynom
$p(X)=a_nX^n+\dots$ mit Leitkoeffizient $a_n$ das normierte Polynom
\[
\frac{1}{a_n}p(X) = \frac{1}{a_n}(a_nX^n + \dots + a_0)=
X^n + \frac{a_{n-1}}{a_n}X^{n-1} + \dots + \frac{a_0}{a_n}
\]
machen.
Man sagt auch, das Polynom $p(X)$ wurde normiert.

Die Tatsache, dass zwei  Polynome nicht gleich viele von $0$ verschiedene Koeffizienten haben müssen,
verkompliziert die Beschreibung der Rechenoperationen ein wenig.
Wir werden daher im Folgenden oft für ein Polynom
\[
p(X)
=
a_nX^n + a_{n-1}X^{n-1} + \dots a_1X+a_0
\]
annehmen, dass alle Koeffizienten $a_{n+1},a_{n+2},\dots$ implizit mit
Wert $0$ definiert sind.
Wir werden uns also erlauben,
\[
p(X)
=
\sum_{k}a_kX^k
=
\sum_{k=0}^\infty a_kX^k
\]
zu schreiben, wobei in der ersten Form das Summenzeichen bedeuten soll,
dass nur über diejenigen Indizes $k$ summiert wird, für die $a_k$
definiert ist.
\label{summenzeichenkonvention}

%
% Abschnitt über Polynomring Definition
%
\subsection{Der Polynomring
\label{buch:subsection:polynome:ring}}
Die Menge $R[X]$ aller Polynome über $R$ wird zu einem Ring, wenn man die
Rechenoperationen Addition und Multiplikation so definiert, wie man das
in der Schule gelernt hat.
Die Summe von zwei Polynomen
\begin{align*}
p(X) &= a_nX^n + a_{n-1}X^{n-1} + \dots + a_1X + a_0\\
q(X) &= b_mX^m + b_{m-1}X^{m-1} + \dots + b_1X + b_0
\end{align*}
ist
\[
p(X)+q(X)
=
\sum_{k} (a_k+b_k)X^k,
\]
wobei die Summe wieder so zu interpretieren ist, über alle Terme
summiert wird, für die mindestens einer der Summanden von $0$
verschieden ist.

Für das Produkt verwenden wir die Definition
\[
p(X)q(X)
=
\sum_{k}\sum_{l} a_kb_l X^{k+l},
\]
die natürlich mit Formel~\eqref{buch:eqn:polynome:faltung}
gleichbedeutend ist.
Die Polynom-Multiplikation und Addition sind nur eine natürliche
Erweiterung der Rechenregeln, die man schon in der Schule lernt,
es ist daher nicht überraschend, dass die bekannten Rechenregeln
auch für Polynome gelten.
Das Distributivgesetz
\[
p(X)(u(X)+v(X)) = p(X)u(X) + p(X)v(X)
\qquad
(p(X)+q(X)) u(X) = p(X)u(X) + q(X)u(X)
\]
zum Beispiel sagt ja nichts anderes, als dass man ausmultiplizieren
kann.
Oder die Assoziativgesetze
\begin{align*}
p(X)+q(X)+r(X)
&=
(p(X)+q(X))+r(X)
=
p(X)+(q(X)+r(X))
\\
p(X)q(X)r(X)
&=
(p(X)q(X))r(X)
=
p(X)(q(X)r(X))
\end{align*}
für die Multiplikation besagt, das es keine Rolle spielt, in welcher
Reihenfolge man die Additionen oder Multiplikationen ausführt.

%
% Der Grad eines Polynoms
%
\subsection{Grad
\label{buch:subsection:polynome:grad}}

\begin{definition}
Der {\em Grad} eines Polynoms $p(X)$ ist die höchste Potenz von $X$, die im
Polynom vorkommt.
Das Polynom
\[
p(X) = a_nX^n + a_{n-1}X^{n-1}+\dots a_1X + a_0
\]
hat den Grad $n$, wenn $a_n\ne 0$ ist.
Der Grad von $p$ wird mit $\deg p$ bezeichnet.
Konstante Polynome $p(X)=a_0$ mit $a_0\ne 0$ hat den Grad $0$.
Der Grad des Nullpolynoms $p(X)=0$ ist definiert als
$-\infty$.
\end{definition}

Der Grad eines Polynoms ist sinnvoll in dem Sinn, dass er sich mit
den Rechenoperationen gut verträgt.
Damit lässt sich weiter unten auch die spezielle Wahl des Grades
des Nullpolynoms begründen.
Es gelten nämlich die folgenden Rechenregeln.

\begin{lemma}
\label{lemma:rechenregelnfuerpolynomgrad}
Sind $p$ und $q$ Polynome mit Koeffizienten in $R$ und $0\ne \lambda\in R$,
dann gilt
\begin{align}
\deg(pq) &\le \deg p + \deg q
\label{buch:eqn:polynome:gradsumme}
\\
\deg(p+q) &\le \max(\deg p, \deg q)
\label{buch:eqn:polynome:gradprodukt}
\\
\deg(\lambda p) &\le \deg p
\label{buch:eqn:polynome:gradskalar}
\end{align}
\end{lemma}

Die Formel \eqref{buch:eqn:polynome:gradskalar} ist eigentlich
ein Spezialfall von \eqref{buch:eqn:polynome:gradsumme}.
Die Zahl $\lambda\in R$ kann man als Polynom vom Grad $0$ betrachten,
wofür natürlich \eqref{buch:eqn:polynome:gradsumme} gilt, also
$\deg(\lambda p) \le \deg\lambda + \deg p$.

\begin{proof}[Beweis]
Wir schreiben die Polynome wieder in der Form
\begin{align*}
p(X) &= a_nX^n + a_{n-1}X^{n-1} + \dots + a_1X + a_0&&\Rightarrow&\deg p&=n\\
q(X) &= b_mX^m + b_{m-1}X^{m-1} + \dots + b_1X + b_0&&\Rightarrow&\deg q&=m.
\end{align*}
Dann kann der höchste Koeffizient in der Summe $p+q$ nicht weiter oben
sein als die grössere von den beiden Zahlen $n$ und $m$ angibt, dies
beweist \eqref{buch:eqn:polynome:gradsumme}.
Ebenso kann der höchste Koeffizient im Produkt nach der
Formel~\eqref{buch:eqn:polynome:faltung} nicht weiter oben als bei
$n+m$ liegen, dies beweist
beweist \eqref{buch:eqn:polynome:gradprodukt}.
Es könnte aber passieren, dass $a_nb_m=0$ ist, d.~h.~es ist durchaus möglich,
dass der Grad kleiner ist.
Schliesslich kann der höchsten Koeffizient von $\lambda p(X)$ nicht grösser
als der höchste Koeffizient von $p(X)$ sein, was
\eqref{buch:eqn:polynome:gradskalar} beweist.
\end{proof}

Etwas enttäuschend an diesen Rechenregeln ist, dass der Grad eines
Produktes nicht exakt die Summe der Grade hat.
Der Grund ist natürlich, dass es in gewissen Ringen $R$ passieren kann,
dass das Produkt $a_n\cdot b_m=0$ ist.
Zum Beispiel ist im Ring der $2\times 2$ Matrizen das Produkt der Elemente
\begin{equation}
a_n = \begin{pmatrix}1&0\\0&0\end{pmatrix}
\quad\text{und}\quad
b_m = \begin{pmatrix}0&0\\0&1\end{pmatrix}
\qquad\Rightarrow\qquad
a_nb_m = \begin{pmatrix}0&0\\0&0\end{pmatrix}.
\label{buch:eqn:definitionen:nullteilerbeispiel}
\end{equation}
Diese unangehme Situation tritt immer ein, wenn es von Null verschiedene
Elemente gibt, deren Produkt $0$ ist.
In Matrizenringen ist das der Normalfall, man kann diesen Fall also nicht
einfach ausschliessen.
In den Zahlenmengen wie $\mathbb{Z}$, $\mathbb{Q}$ und $\mathbb{R}$ passiert
das natürlich nie.

\begin{definition}
Ein Ring $R$ heisst {\em nullteilerfrei}, wenn für zwei Elemente
$a,b\in R$ aus $ab=0$ immer geschlossen werden kann, dass
$a=0$ oder $b=0$.
Ein von $0$ verschiedenes Element $a\in R$ heisst Nullteiler,
wenn es eine $b\in R$ mit $b\ne 0$ gibt derart dass $ab=0$.
\index{Nullteiler}
\index{nullteilerfrei}
\end{definition}

Die beiden Matrizen in
\eqref{buch:eqn:definitionen:nullteilerbeispiel}
sind Nullteiler im Ring $M_2(\mathbb{Z})$ der $2\times 2$-Matrizen.
Der Matrizenring $M_2(\mathbb{Z})$ ist also nicht nullteilerfrei.

In einem nullteilerfreien Ring gelten die Rechenregeln für den Grad
jetzt exakt:

\begin{lemma}
Sei $R$ ein nullteilerfreier Ring und $p$ und $q$ Polynome über $R$
und $0\ne \lambda\in R$.
Dann gilt
\begin{align}
\deg(pq) &= \deg p + \deg q
\label{buch:eqn:polynome:gradsummeexakt}
\\
\deg(p+q) &\le \max(\deg p, \deg q)
\label{buch:eqn:polynome:gradproduktexakt}
\\
\deg(\lambda p) &= \deg p
\label{buch:eqn:polynome:gradskalarexakt}
\end{align}
\end{lemma}

\begin{proof}[Beweis]
Der Fall, dass der höchste Koeffizient verschwindet, weil $a_n$, $b_m$
oder $\lambda$ Nullteiler sind, kann unter den gegebenen Voraussetzungen
nicht eintreten, daher werden die in
Lemma~\ref{lemma:rechenregelnfuerpolynomgrad} gefunden Ungleichungen
für Produkte exakt.
\end{proof}

Die Gleichung
\eqref{buch:eqn:polynome:gradskalarexakt}
kann im Fall $\lambda=0$ natürlich nicht gelten.
Betrachten wir $\lambda$ wieder als ein Polynom, dann folgt aus
\eqref{buch:eqn:polynome:gradsummeexakt}, dass
\[
\begin{aligned}
\lambda&\ne 0  &&\Rightarrow& \deg (\lambda p) &= \deg\lambda + \deg p = 0+\deg p
\\
\lambda&=0     &&\Rightarrow& \deg (0 p) &= \deg 0 + \deg p = \deg 0
\end{aligned}
\]
Diese Gleichung kann also nur aufrechterhalten werden, wenn die ``Zahl'' $\deg 0$ die Eigenschaft besitzt, dass man immer noch $\deg 0$ bekommt,
wenn man irgend eine Zahl $\deg p$ hinzuaddiert. Wenn also
\[\deg 0 + \deg p = \deg 0 \qquad \forall \deg p \in \mathbb Z\]
gilt.
So eine Zahl gibt es in den ganzen Zahlen nicht.
Wenn man zu einer ganzen Zahl eine andere ganze Zahl hinzuaddiert, ändert sich fast immer etwas.
Man muss daher $\deg 0 = -\infty$ setzen und festlegen, dass
$-\infty + n = -\infty$ für beliebige ganze Zahlen $n$ gilt.

\begin{definition}
\label{buch:def:definitionen:polynomfilterung}
Die Polynome vom Grad $\le n$ mit Koeffizienten in $R$
bilden die Teilmenge
\[
R^{(n)}[X]
=
\{ p\in R[X]\;|\; \deg p \le n\}.
\]
Die Mengen $R^{(n)}[X]$ bilden eine {\em Filtrierung} des Polynomrings
$R[X]$, d.~h.~sie sind ineinander geschachtelt
\[
\arraycolsep=4pt
\begin{array}{ccccccccccccccc}
R^{(-\infty)}[X] & \subset
	& R^{(0)}[X] & \subset
		& R^{(1)}[X] & \subset & \dots & \subset
			& R^{(k)}[X] & \subset
				& R^{(k+1)}[X] & \subset & \dots & \subset
					& R[X]\\[3pt]
\bigg\| &
	&\bigg\| &
		&\bigg\| & & &
			&&
				&& & &
					&
\\[3pt]
\{0\} & \subset
	& R & \subset
		& \{a_1X+a_0\;|a_k\in R\} & \subset & \dots &
\end{array}
\]
und ihre Vereinigung ist $R[X]$.
\end{definition}

Die Formeln für den Grad können wir auch mit den Mengen $R^{(k)}[X]$
ausdrücken:
\begin{align*}
\deg (p+q) &\le \max(\deg p, \deg q)
&&\Rightarrow&
R^{(k)}+R^{(l)}
&\subset R^{(\max(k,l))}
=
R^{(k)}[X] \cup R^{(l)}[X].
\\
\deg (p\cdot q)&=\deg p+\deg q
&&\Rightarrow&
R^{(k)}[X] \cdot R^{(l)}[X]
&=
R^{(k+l)}[X].
\end{align*}


%
% Abschnitt über Teilbarkeit
%
\subsection{Teilbarkeit
\label{buch:subsection:polynome:teilbarkeit}}
XXX TODO

%
% Abschnitt über formale Potenzreihen
%
\subsection{Formale Potenzreihen
\label{buch:subsection:polynome:potenzreihen}}
XXX TODO




%
% vektoren.tex -- Darstellung von Polynomen als Vektoren
%
% (c) 2021 Prof Dr Andreas Müller, OST Ostschweizer Fachhochschule Rapperswil
%
\section{Polynome als Vektoren
\label{buch:section:polynome:vektoren}}
\rhead{Polynome als Vektoren}
Ein Polynom
\[
p(X) = a_nX^n + a_{n-1}X^{n-1} + \dots a_1X+a_0
\]
mit Koeffizienten in einem Ring $R$
ist spezifiziert, wenn die Koeffizienten $a_k$ bekannt sind.
Die Potenzen von $X$ dienen hier nur dazu, die verschiedenen
Koeffizienten zu unterscheiden.
Das Polynom $p(X)$ vom Grad $n$ ist also auch gegeben durch den
$n+1$-dimensionalen Vektor
\[
\begin{pmatrix}
a_0\\
a_1\\
\vdots\\
a_{n-1}\\
a_{n}
\end{pmatrix}
\in
R^n.
\]
Diese Darstellung eines Polynoms gibt auch die Addition von Polynomen
und die Multiplikation von Polynomen mit Skalaren aus $R$ korrekt wieder.
Die Abbildung von Vektoren auf Polynome
\[
\varphi
\colon  R^n \to R[X]
:
\begin{pmatrix}a_0\\\vdots\\a_n\end{pmatrix}
\mapsto
a_nX^n + a_{n-1}X^{n-1}+\dots+a_1X+a_0
\]
erfüllt also
\[
\varphi( \lambda a) = \lambda \varphi(a)
\qquad\text{und}\qquad
\varphi(a+b) = \varphi(a) + \varphi(b)
\]
und ist damit eine lineare Abbildung.
Umgekehrt kann man auch zu jedem Polynom $p(X)$ vom Grad~$\le n$ einen
Vektor finden, der von $\varphi$ auf das Polynom $p(X)$ abgebildet wird.
Die Abbildung $\varphi$ ist also ein Isomorphismus
\[
\varphi
\colon
\{p\in R[X]\;|\; \deg(p) \le n\}
\overset{\equiv}{\to}
R^{n+1}
\]
zwischen der Menge
der Polynome vom Grad $\le n$ auf $R^{n+1}$.
Für alle Rechnungen, bei denen es nur um Addition von Polynomen oder
um Multiplikation mit Skalaren geht, ist also diese vektorielle Darstellung
mit Hilfe von $\varphi$ eine zweckmässige Darstellung.

In zwei Bereichen ist die Beschreibung von Polynomen mit Vektoren allerdings
ungenügend: einerseits können Polynome können beliebig hohen Grad haben,
während Vektoren in $R^{n+1}$ höchstens $n+1$ Komponenten haben können.
Andererseits geht bei der vektoriellen Beschreibung die multiplikative
Struktur vollständig verloren.

\subsection{Polynome beliebigen Grades
\label{buch:subsection:polynome:beliebigergrad}}
Ein Polynom
\[
q(X)
=
b_mX^m + b_{m-1}X^{m-1} + \dots + b_1X + b_0
\]
vom Grad $m<n$ kann dargestellt werden als ein Vektor
\[
\begin{pmatrix}
b_0\\
b_1\\
\vdots\\
b_{m-1}\\
b_{m}\\
0\\
\vdots
\end{pmatrix}
\in
R^{n+1}
\]
mit der Eigenschaft, dass die Komponenten mit Indizes
$m+1,\dots n$ verschwinden.
Polynome vom Grad $m<n$ bilden einen Unterraum der Polynome vom Grad $n$.
Wir können auch die $m+1$-dimensionalen Vektoren in den $n+1$-dimensionalen
Vektoren einbetten, indem wir die Vektoren durch ``auffüllen'' mit Nullen
auf die richtige Länge bringen.
Es gibt also eine lineare Abbildung
\[
R^{m+1} \to R^{n+1}
\colon
\begin{pmatrix}
b_0\\b_1\\\vdots\\b_m
\end{pmatrix}
\mapsto
\begin{pmatrix}
b_0\\b_1\\\vdots\\b_m\\0\\\vdots
\end{pmatrix}
.
\]
Die Moduln $R^{k}$ sind also alle ineinandergeschachtelt, können aber
alle auf konsistente Weise mit der Abbildung $\varphi$ in den Polynomring
$R[X]$ abgebildet werden.
\begin{center}
\begin{tikzcd}
\{0\}\ar[r] %\arrow[d,"\varphi"]
	&R \ar[r] %\arrow[d, "\varphi"]
		&R^2 \ar[r] %\arrow[d, "\varphi"]
			&\dots \ar[r]
				&R^k \ar[r] %\arrow[d, "\varphi"]
					&R^{k+1} \ar[r] %\arrow[d, "\varphi"]
						&\dots
\\
R^{(0)}[X]\arrow[r,hook] \arrow[drrr,hook]
	&R^{(1)}[X]\arrow[r,hook] \arrow[drr,hook]
		&R^{(2)}[X]\arrow[r,hook] \arrow[dr,hook]
			&\dots\arrow[r,hook]
				&R^{(k)}[X]\arrow[r,hook] \arrow[dl,hook]
					&R^{(k+1)}[X]\arrow[r,hook] \arrow[dll,hook]
						&\dots
\\
	&
		&
			&R[X]
				&
					&
						&
\end{tikzcd}
\end{center}
\subsection{Multiplikative Struktur
\label{buch:subsection:polynome:multiplikativestruktur}}






%%
% permutationsmatrizen.tex -- Permutationsmatrizen
%
% (c) 2020 Prof Dr Andreas Müller, Hochschule Rapperswil
%
\section{Permutationsmatrizen
\label{buch:section:permutationsmatrizen}}
\rhead{Permutationsmatrizen}
Die Eigenschaft, dass eine Vertauschung das Vorzeichen kehrt, ist
eine wohlebekannte Eigenschaft der Determinanten.
In diesem Abschnitt soll daher eine Darstellung von Permutationen
als Matrizen gezeigt werden und die Verbindung zwischen dem
Vorzeichen einer Permutation und der Determinanten hergestellt
werden.

\subsection{Matrizen}
Gegeben sei jetzt eine Permutation $\sigma\in S_n$. 
Aus $\sigma$ lässt sich eine lineare Abbildung $\Bbbk^n\to\Bbbk^n$
konstruieren, die die Standardbasisvektoren permutiert, also
\[
f_{\sigma}\colon
\Bbbk^n \to \Bbbk^n
:
\left\{
\begin{aligned}
e_1&\mapsto e_{\sigma(1)} \\
e_2&\mapsto e_{\sigma(2)} \\
\vdots&\\
e_n&\mapsto e_{\sigma(n)}
\end{aligned}
\right.
\]
Die Matrix $P_\sigma$ der linearen Abbildung $f_{\sigma}$ hat in Spalte $i$
genau eine $1$ in der Zeile $\sigma(i)$, also
\[
(P_\sigma)_{ij} = \delta_{j\sigma(i)}.
\]

\begin{beispiel}
Die zur Permutation
\[
\begin{pmatrix}
1&2&3&4&5&6\\
2&1&3&5&6&4
\end{pmatrix}
\]
gehörige lineare Abbildung $f_\sigma$ hat die Matrix
\[
A_\sigma
=
\begin{pmatrix}
0&1&0&0&0&0\\
1&0&0&0&0&0\\
0&0&1&0&0&0\\
0&0&0&0&0&1\\
0&0&0&1&0&0\\
0&0&0&0&1&0
\end{pmatrix}
\qedhere
\]
\end{beispiel}

\begin{definition}
Eine Permutationsmatrix ist eine Matrix $P\in M_n(\Bbbk)$ 
derart, die in jeder Zeile und Spalte genau eine $1$ enhalten,
während alle anderen Matrixelemente $0$ sind.
\end{definition}

Es ist klar, dass aus einer Permutationsmatrix auch die Permutation
der Standardbasisvektoren abgelesen werden kann.
Die Verknüpfung von Permutationen wird zur Matrixmultiplikation
von Permutationsmatrizen, die Zuordnung $\sigma\mapsto P_\sigma$
ist also ein Homomorphismus
$
S_n \to M_n(\Bbbk^n),
$
es ist $P_{\sigma_1\sigma_2}=P_{\sigma_1}P_{\sigma_2}$.

\subsection{Transpositionen}
Transpositionen sind Permutationen, die genau zwei Elemente von $[n]$
vertauschen.
Wir ermitteln jetzt die Permutationsmatrix der Transposition $\tau=\tau_{ij}$
\[
P_{\tau_{ij}}
=
\begin{pmatrix}
1&      & &      &     &      & &      & \\
 &\ddots& &      &     &      & &      & \\
 &      &1&      &     &      & &      & \\
 &      & &0     &\dots&1     & &      & \\
 &      & &\vdots&     &\vdots& &      & \\
 &      & &1     &\dots&0     & &      & \\
 &      & &      &     &      &1&      & \\
 &      & &      &     &      & &\ddots& \\
 &      & &      &     &      & &      &1
\end{pmatrix}
\qedhere
\]

Die Permutation $\sigma$ mit dem Zyklus $1\to 2\to\dots\to l-1\to l\to 1$
der Länge $l$ kann aus aufeinanderfolgenden Transpositionen zusammengesetzt
werden, die zugehörigen Permutationsmatrizen sind
\begin{align*}
P_\sigma
&=
P_{\tau_{12}}
P_{\tau_{23}}
P_{\tau_{34}}\dots
P_{\tau_{l-2,l-1}}
P_{\tau_{l-1,l}}
\\
&=
\begin{pmatrix}
0&1&0&0&\dots\\
1&0&0&0&\dots\\
0&0&1&0&\dots\\
0&0&0&1&\dots\\
\vdots&\vdots&\vdots&\vdots&\ddots
\end{pmatrix}
\begin{pmatrix}
1&0&0&0&\dots\\
0&0&1&0&\dots\\
0&1&0&0&\dots\\
0&0&0&1&\dots\\
\vdots&\vdots&\vdots&\vdots&\ddots
\end{pmatrix}
\begin{pmatrix}
1&0&0&0&\dots\\
0&1&0&0&\dots\\
0&0&0&1&\dots\\
0&0&1&0&\dots\\
\vdots&\vdots&\vdots&\vdots&\ddots
\end{pmatrix}
\dots
\\
&=
\begin{pmatrix}
0&0&1&0&\dots\\
1&0&0&0&\dots\\
0&1&0&0&\dots\\
0&0&0&1&\dots\\
\vdots&\vdots&\vdots&\vdots&\ddots
\end{pmatrix}
\begin{pmatrix}
1&0&0&0&\dots\\
0&1&0&0&\dots\\
0&0&0&1&\dots\\
0&0&1&0&\dots\\
\vdots&\vdots&\vdots&\vdots&\ddots
\end{pmatrix}
\dots
\\
&=
\begin{pmatrix}
0&0&0&1&\dots\\
1&0&0&0&\dots\\
0&1&0&0&\dots\\
0&0&1&0&\dots\\
\vdots&\vdots&\vdots&\vdots&\ddots
\end{pmatrix}
\\
&\vdots\\
&=
\begin{pmatrix}
0&0&0&0&\dots&0&1\\
1&0&0&0&\dots&0&0\\
0&1&0&0&\dots&0&0\\
0&0&1&0&\dots&0&0\\
\vdots&\vdots&\vdots&\vdots&\ddots&\vdots&\vdots\\
0&0&0&0&\dots&1&0
\end{pmatrix}
\end{align*}

\subsection{Determinante und Vorzeichen}
Die Transpositionen haben Permutationsmatrizen, die aus der Einheitsmatrix
entstehen, indem genau zwei Zeilen vertauscht werden.
Die Determinante einer solchen Permutationsmatrix ist
\[
\det P_{\tau} = - \det E = -1 = \operatorname{sgn}(\tau).
\]
Nach der Produktregel für die Determinante folgt für eine Darstellung
der Permutation $\sigma=\tau_1\dots\tau_l$ als Produkt von Transpositionen,
dass
\[
\det P_{\sigma}
=
\det P_{\tau_1} \dots \det P_{\tau_l}
=
(-1)^l
=
\operatorname{sgn}(\sigma).
\]
Das Vorzeichen einer Permutation ist also identisch mit der Determinante
der zugehörigen Permutationsmatrix.



%%
% minimalpolynom.tex
%
% (c) 2021 Prof Dr Andreas Müller, OST Ostschweizer Fachhochschule
%
\section{Minimalpolynom
\label{buch:polynome:section:minimalpolynom}}
\rhead{Minimalpolynom}



