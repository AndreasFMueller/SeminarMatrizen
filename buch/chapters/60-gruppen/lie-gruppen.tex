%
% lie-gruppen.tex -- Lie-Gruppebn
%
% (c) 2020 Prof Dr Andreas Müller, Hochschule Rapperswil
%
\section{Lie-Gruppen
\label{buch:section:lie-gruppen}}
\rhead{Lie-Gruppen}

\subsection{Drehungen in der Ebene
\label{buch:gruppen:drehungen2d}}
Drehungen der Ebene können in einer orthonormierten Basis durch
Matrizen der Form
\[
D_{\alpha}
=
\begin{pmatrix}
\cos\alpha&-\sin\alpha\\
\sin\alpha& \cos\alpha
\end{pmatrix}
\]
dargestellt werden.
Wir bezeichnen die Menge der Drehmatrizen in der Ebene mit
$\operatorname{SO}(2)\subset\operatorname{GL}_2(\mathbb{R})$.
Die Abbildung
\[
D_{\bullet}
\colon
\mathbb{R}\to \operatorname{SO}(2)
:
\alpha \mapsto D_{\alpha}
\]
hat die Eigenschaften
\begin{align*}
D_{\alpha+\beta}&= D_{\alpha}D_{\beta}
\\
D_0&=I
\\
D_{2k\pi}&=I\qquad \forall k\in\mathbb{Z}.
\end{align*}
Daraus folgt zum Beispiel, dass $D_{\bullet}$ eine $2\pi$-periodische
Funktion ist.
$D_{\bullet}$ bildet die Menge der Winkel $[0,2\pi)$ bijektiv auf
die Menge der Drehmatrizen in der Ebene ab.

Ein alternatives Bild für die Drehungen der Ebene kann man in der komplexen
Ebene $\mathbb{C}$ erhalten.
Die Multiplikation mit der komplexen Zahl $e^{i\alpha}$ beschreibt eine
Drehung der komplexen Ebene um den Winkel $\alpha$.
Die Zahlen der Form $e^{i\alpha}$ haben den Betrag $1$ und die Abbildung
\[
f\colon \mathbb{R}\to \mathbb{C}:\alpha \mapsto e^{i\alpha}
\]
hat die Eigenschaften
\begin{align*}
f(\alpha+\beta) &= f(\alpha)f(\beta)
\\
f(0)&=1
\\
f(2\pi k)&=1\qquad\forall k\in\mathbb{Z},
\end{align*}
die zu den Eigenschaften der Abbildung $\alpha\mapsto D_{\alpha}$ 
analog sind.

Jede komplexe Zahl $z$ vom Betrag $1$ kann geschrieben werden in der Form
$z=e^{i\alpha}$, die Abbildung $f$ ist also eine Parametrisierung des
Einheitskreises in der Ebene.
Wir bezeichen $S^1=\{z\in\mathbb{C}\;|\; |z|=1\}$ die komplexen Zahlen vom
Betrag $1$.
$S^1$ ist eine Gruppe bezüglich der Multiplikation, da für jede Zahl
$z,w\in S^1$ gilt
$|z^{-1}|=1$ und $|zw|=1$ und damit $z^{-1}\in S^1$ und $zw\in S^1$.

Zu einer komplexen Zahl $z\in S^1$ gibt es einen bis auf Vielfache
von $2\pi$ eindeutigen Winkel $\alpha(z)$ derart, dass $e^{i\alpha(z)}=z$.
Damit kann man jetzt die Abbildung
\[
\varphi
\colon
S^1\to \operatorname{SO}(2)
:
z\mapsto  D_{\alpha(z)}
\]
konstruieren.
Da $D_{\alpha}$ $2\pi$-periodisch ist, geben um Vielfache
von $2\pi$ verschiedene Wahlen von $\alpha(z)$ die gleiche
Matrix $D_{\alpha(z)}$, die Abbildung $\varphi$ ist daher
wohldefiniert.
$\varphi$ erfüllt ausserdem die Bedingungen
\begin{align*}
\varphi(z_1z_2)
&=
D_{\alpha(z_1z_2)}
=
D_{\alpha(z_1)+\alpha(z_2)}
=
D_{\alpha(z_1)}D_{\alpha(z_2)}
=
\varphi(z_1)\varphi(z_2)
\\
\varphi(1)
&=
D_{\alpha(1)}
=
D_0
=
I
\end{align*}
Die Abbildung $\varphi$ ist ein Homomorphismus der Gruppe $S^1$
in die Gruppe $\operatorname{SO}(2)$.
Die Menge der Drehmatrizen in der Ebene kann also mit dem Einheitskreis
in der komplexen Ebene identifiziert werden.

%
% Isometrien von R^n
%
\subsection{Isometrien von $\mathbb{R}^n$
\label{buch:gruppen:isometrien}}
Lineare Abbildungen der Ebene $\mathbb{R}^n$ mit dem üblichen Skalarprodukt
können durch $n\times n$-Matrizen beschrieben werden.
Die Matrizen, die das Skalarprodukt erhalten, bilden eine Gruppe,
die in diesem Abschnitt genauer untersucht werden soll.
Eine Matrix $A\in M_{2}(\mathbb{R})$ ändert das Skalarprodukt nicht, wenn
für jedes beliebige Paar $x,y$ von Vektoren gilt
$\langle Ax,Ay\rangle = \langle x,y\rangle$.
Das Standardskalarprodukt kann mit dem Matrixprodukt ausgedrückt werden:
\[
\langle Ax,Ay\rangle
=
(Ax)^tAy
=
x^tA^tAy
=
x^ty
=
\langle x,y\rangle
\]
für jedes Paar von Vektoren $x,y\in\mathbb{R}$.

Mit dem Skalarprodukt kann man auch die Matrixelemente einer Matrix
einer Abbildung $f$ in der Standardbasis bestimmen.
Das Skalarprodukt $\langle e_i, v\rangle$ ist die Länge der Projektion
des Vektors $v$ auf die Richtung $e_i$.
Die Komponenten von $Ae_j$ sind daher $a_{ij}=\langle e_i,f(e_j)\rangle$.
Die Matrix $A$ der Abbildung $f$ hat also die Matrixelemente
$a_{ij}=e_i^tAe_j$.

\subsubsection{Die orthogonale Gruppe $\operatorname{O}(n)$}
Die Matrixelemente von $A^tA$ sind
$\langle A^tAe_i, e_j\rangle =\langle e_i,e_j\rangle = \delta_{ij}$
sind diejenigen der Einheitsmatrix,
die Matrix $A$ erfüllt $AA^t=I$ oder $A^{-1}=A^t$.
Dies sind die {\em orthogonalen} Matrizen.
Die Menge $\operatorname{O}(n)$ der isometrischen Abbildungen besteht
daher aus den Matrizen
\[
\operatorname{O}(n)
=
\{ A\in M_n(\mathbb{R})\;|\; AA^t=I\}.
\]
Die Matrixgleichung $AA^t=I$ liefert $n(n+1)/2$ unabhängige Bedingungen,
die die orthogonalen Matrizen innerhalb der $n^2$-dimensionalen
Menge $M_n(\mathbb{R})$ auszeichnen.
Die Menge $\operatorname{O}(n)$ der orthogonalen Matrizen hat daher
die Dimension
\[
n^2 - \frac{n(n+1)}{2}
=
\frac{2n^2-n^2-n}{2}
=
\frac{n(n-1)}2.
\]
Im Spezialfall $n=2$ ist die Gruppe $O(2)$ eindimensional.

\subsubsection{Die Gruppe $\operatorname{SO}(n)$}
Die Gruppe $\operatorname{O}(n)$ enhält auch Isometrien, die
die Orientierung des Raumes umkehren, wie zum Beispiel Spiegelungen.
Wegen $\det (AA^t)=\det A\det A^t = (\det A)^2=1$ kann die Determinante
einer orthogonalen Matrix nur $\pm 1$ sein.
Orientierungserhaltende Isometrien haben Determinante $1$.

Die Gruppe
\[
\operatorname{SO}(n)
=
\{A\in\operatorname{O}(n)\;|\; \det A=1\}
\]
heisst die {\em spezielle orthogonale Gruppe}.
Die Dimension der Gruppe $\operatorname{O}(n)$ ist $n(n-1)/2$.

\subsubsection{Die Gruppe $\operatorname{SO}(3)$}
Die Gruppe $\operatorname{SO}(3)$ der Drehungen des dreidimensionalen
Raumes hat die Dimension $3(3-1)/2=3$.
Eine Drehung wird festgelegt durch die Richtung der Drehachse und den
Drehwinkel.
Die Richtung der Drehachse ist ein Einheitsvektor, also ein Punkt
auf der zweidimensionalen Kugel.
Der Drehwinkel ist der dritte Parameter.

Drehungen mit kleinen Drehwinkeln können zusammengesetzt werden
aus den Matrizen
\[
D_{x,\alpha}
=
\begin{pmatrix}
1&0&0\\
0&\cos\alpha&-\sin\alpha\\
0&\sin\alpha& \cos\alpha
\end{pmatrix},
\qquad
D_{y,\beta}
=
\begin{pmatrix}
 \cos\beta&0&\sin\beta\\
      0    &1&     0    \\
-\sin\beta&0&\cos\beta
\end{pmatrix},
\qquad
D_{z,\gamma}
=
\begin{pmatrix}
\cos\gamma&-\sin\gamma&0\\
\sin\gamma& \cos\gamma&0\\
    0     &     0     &1
\end{pmatrix},
\]
die Drehungen um die Koordinatenachsen um den Winkel $\alpha$
beschreiben.
Auch die Winkel $\alpha$, $\beta$ und $\gamma$ können als die
drei Koordinaten der Mannigkfaltigkeit $\operatorname{SO}(3)$
angesehen werden.

%
% Die Gruppe SU(2)
%
\subsection{Die Gruppe $\operatorname{SU}(2)$
\label{buch:gruppen:su2}}
Die Menge der Matrizen
\[
\operatorname{SU}(2)
=
\left\{
\left.
A=\begin{pmatrix} a&b\\c&d\end{pmatrix}
\;\right|\;
a,b,c,d\in\mathbb{C},\det(A)=1, AA^*=I
\right\}
\]
heisst die {\em spezielle unitäre Gruppe}.
Wegen $\det(AB)=\det(A)\det(B)=1$ und $(AB)^*AB=B^*A^*AB=B^*B=I$ ist 
$\operatorname{SU}(2)$ eine Untergruppe von $\operatorname{GL}_2(\mathbb{C})$.
Die Bedingungen $\det A=1$ und $AA^*=I$ schränken die möglichen Werte
von $a$ und $b$ weiter ein.
Aus 
\[
A^*
=
\begin{pmatrix}
\overline{a}&\overline{c}\\
\overline{b}&\overline{d}
\end{pmatrix}
\]
und den Bedingungen führen die Gleichungen
\[
\begin{aligned}
a\overline{a}+b\overline{b}&=1
&&\Rightarrow&|a|^2+|b|^2&=1
\\
a\overline{c}+b\overline{d}&=0
&&\Rightarrow&
\frac{a}{b}&=-\frac{\overline{d}}{\overline{c}}
\\
c\overline{a}+d\overline{b}&=0
&&\Rightarrow&
\frac{c}{d}&=-\frac{\overline{b}}{\overline{a}}
\\
c\overline{c}+d\overline{d}&=1&&\Rightarrow&|c|^2+|d|^2&=1
\\
ad-bc&=1
\end{aligned}
\]
Aus der zweiten Gleichung kann man ableiten, dass es eine Zahl $t\in\mathbb{C}$
gibt derart, dass $c=-t\overline{b}$ und $d=t\overline{a}$.
Damit wird die Bedingung an die Determinante zu
\[
1
=
ad-bc = at\overline{a} - b(-t\overline{b})
=
t(|a|^2+|b|^2)
=
t,
\]
also muss die Matrix $A$ die Form haben
\[
A
=
\begin{pmatrix}
a&b\\
-\overline{b}&\overline{a}
\end{pmatrix}
\qquad\text{mit}\quad |a|^2+|b|^2=1.
\]
Schreibt man $a=a_1+ia_2$ und $b=b_1+ib_2$ mit rellen $a_i$ und $b_i$,
dann besteht $SU(2)$  aus den Matrizen der Form
\[
A=
\begin{pmatrix}
 a_1+ia_2&b_1+ib_2\\
-b_1+ib_2&a_1-ia_2
\end{pmatrix}
\]
mit der zusätzlichen Bedingung
\[
|a|^2+|b|^2
=
a_1^2 + a_2^2 + b_1^2 + b_2^2 = 1.
\]
Die Matrizen von $\operatorname{SU}(2)$ stehen daher in einer
eins-zu-eins-Beziehung zu den Vektoren $(a_1,a_2,b_1,b_2)\in\mathbb{R}^4$
eines vierdimensionalen reellen Vektorraums mit Länge $1$.
Geometrisch betrachtet ist also $\operatorname{SU}(2)$ eine dreidmensionalen
Kugel, die in einem vierdimensionalen Raum eingebettet ist.



