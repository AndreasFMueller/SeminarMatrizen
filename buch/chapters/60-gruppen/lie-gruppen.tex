%
% lie-gruppen.tex -- Lie-Gruppebn
%
% (c) 2020 Prof Dr Andreas Müller, Hochschule Rapperswil
%
\section{Lie-Gruppen
\label{buch:section:lie-gruppen}}
\rhead{Lie-Gruppen}

\subsection{Drehungen in der Ebene
\label{buch:gruppen:drehungen2d}}
Drehungen der Ebene können in einer orthonormierten Basis durch
Matrizen der Form
\[
D_{\alpha}
=
\begin{pmatrix}
\cos\alpha&-\sin\alpha\\
\sin\alpha& \cos\alpha
\end{pmatrix}
\]
dargestellt werden.
Wir bezeichnen die Menge der Drehmatrizen in der Ebene mit
$\operatorname{SO}(2)\subset\operatorname{GL}_2(\mathbb{R})$.
Die Abbildung
\[
D_{\bullet}
\colon
\mathbb{R}\to \operatorname{SO}(2)
:
\alpha \mapsto D_{\alpha}
\]
hat die Eigenschaften
\begin{align*}
D_{\alpha+\beta}&= D_{\alpha}D_{\beta}
\\
D_0&=I
\\
D_{2k\pi}&=I\qquad \forall k\in\mathbb{Z}.
\end{align*}
Daraus folgt zum Beispiel, dass $D_{\bullet}$ eine $2\pi$-periodische
Funktion ist.
$D_{\bullet}$ bildet die Menge der Winkel $[0,2\pi)$ bijektiv auf
die Menge der Drehmatrizen in der Ebene ab.

Ein alternatives Bild für die Drehungen der Ebene kann man in der komplexen
Ebene $\mathbb{C}$ erhalten.
Die Multiplikation mit der komplexen Zahl $e^{i\alpha}$ beschreibt eine
Drehung der komplexen Ebene um den Winkel $\alpha$.
Die Zahlen der Form $e^{i\alpha}$ haben den Betrag $1$ und die Abbildung
\[
f\colon \mathbb{R}\to \mathbb{C}:\alpha \mapsto e^{i\alpha}
\]
hat die Eigenschaften
\begin{align*}
f(\alpha+\beta) &= f(\alpha)f(\beta)
\\
f(0)&=1
\\
f(2\pi k)&=1\qquad\forall k\in\mathbb{Z},
\end{align*}
die zu den Eigenschaften der Abbildung $\alpha\mapsto D_{\alpha}$ 
analog sind.

Jede komplexe Zahl $z$ vom Betrag $1$ kann geschrieben werden in der Form
$z=e^{i\alpha}$, die Abbildung $f$ ist also eine Parametrisierung des
Einheitskreises in der Ebene.
Wir bezeichen $S^1=\{z\in\mathbb{C}\;|\; |z|=1\}$ die komplexen Zahlen vom
Betrag $1$.
$S^1$ ist eine Gruppe bezüglich der Multiplikation, da für jede Zahl
$z,w\in S^1$ gilt
$|z^{-1}|=1$ und $|zw|=1$ und damit $z^{-1}\in S^1$ und $zw\in S^1$.

Zu einer komplexen Zahl $z\in S^1$ gibt es einen bis auf Vielfache
von $2\pi$ eindeutigen Winkel $\alpha(z)$ derart, dass $e^{i\alpha(z)}=z$.
Damit kann man jetzt die Abbildung
\[
\varphi
\colon
S^1\to \operatorname{SO}(2)
:
z\mapsto  D_{\alpha(z)}
\]
konstruieren.
Da $D_{\alpha}$ $2\pi$-periodisch ist, geben um Vielfache
von $2\pi$ verschiedene Wahlen von $\alpha(z)$ die gleiche
Matrix $D_{\alpha(z)}$, die Abbildung $\varphi$ ist daher
wohldefiniert.
$\varphi$ erfüllt ausserdem die Bedingungen
\begin{align*}
\varphi(z_1z_2)
&=
D_{\alpha(z_1z_2)}
=
D_{\alpha(z_1)+\alpha(z_2)}
=
D_{\alpha(z_1)}D_{\alpha(z_2)}
=
\varphi(z_1)\varphi(z_2)
\\
\varphi(1)
&=
D_{\alpha(1)}
=
D_0
=
I
\end{align*}
Die Abbildung $\varphi$ ist ein Homomorphismus der Gruppe $S^1$
in die Gruppe $\operatorname{SO}(2)$.
Die Menge der Drehmatrizen in der Ebene kann also mit dem Einheitskreis
in der komplexen Ebene identifiziert werden.

\subsection{Isometrien von $\mathbb{R}^n$
\label{buch:gruppen:isometrien}}
Lineare Abbildungen der Ebene $\mathbb{R}^n$ mit dem üblichen Skalarprodukt
können durch $n\times n$-Matrizen beschrieben werden.
Die Matrizen, die das Skalarprodukt erhalten, bilden eine Gruppe,
die in diesem Abschnitt genauer untersucht werden soll.
Eine Matrix $A\in M_{2}(\mathbb{R})$ ändert das Skalarprodukt nicht, wenn
für jedes beliebige Paar $x,y$ von Vektoren gilt
$\langle Ax,Ay\rangle = \langle x,y\rangle$.
Das Standardskalarprodukt kann mit dem Matrixprodukt ausgedrückt werden:
\[
\langle Ax,Ay\rangle
=
(Ax)^tAy
=
x^tA^tAy
=
x^ty
=
\langle x,y\rangle
\]
für jedes Paar von Vektoren $x,y\in\mathbb{R}$.

Mit dem Skalarprodukt kann man auch die Matrixelemente einer Matrix
einer Abbildung $f$ in der Standardbasis bestimmen.
Das Skalarprodukt $\langle e_i, v\rangle$ ist die Länge der Projektion
des Vektors $v$ auf die Richtung $e_i$.
Die Komponenten von $Ae_j$ sind daher $a_{ij}=\langle e_i,f(e_j)\rangle$.
Die Matrix $A$ der Abbildung $f$ hat also die Matrixelemente
$a_{ij}=e_i^tAe_j$.


\subsection{Die Gruppe $\operatorname{SU}(2)$
\label{buch:gruppen:su2}}
