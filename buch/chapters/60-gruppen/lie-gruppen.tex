%
% lie-gruppen.tex -- Lie-Gruppebn
%
% (c) 2020 Prof Dr Andreas Müller, Hochschule Rapperswil
%
\section{Lie-Gruppen
\label{buch:section:lie-gruppen}}
\rhead{Lie-Gruppen}
Die in bisherigen Beispielen untersuchten Matrizengruppen zeichnen sich
durch zusätzliche Eigenschaften aus.
Die Gruppe
\[
\operatorname{GL}_n(\mathbb{R}) 
=
\{ A \in M_n(\mathbb{R}) \mid \det A \ne 0\}
\]
besteht aus den Matrizen, deren Determinante nicht $0$ ist.
Da die Menge der Matrizen mit $\det A=0$ eine abgeschlossene Menge
in $M_n(\mathbb{R}) \cong \mathbb{R}^{n^2}$ ist, ist
$\operatorname{GL}_n(\mathbb{R})$ eine offene Teilmenge in $\mathbb{R}^{n^2}$,
sie besitzt also automatisch die Struktur einer $n^2$-Mannigfaltigkeit.
Doch auch alle anderen Matrizengruppen,
die in diesem Abschnitt genauer untersucht werden sollen,
stellen sich als Untermannigfaltigkeiten von
$\operatorname{GL}_n(\mathbb{R})$ heraus.

\subsection{Mannigfaltigkeitsstruktur der Matrizengruppen
\label{buch:subsection:mannigfaltigkeitsstruktur-der-matrizengruppen}}
Eine Matrizengruppe wird automatisch zu einer Mannigfaltigkeit,
wenn es gelingt, eine Karte für eine Umgebung des neutralen Elements
zu finden.
Dazu muss gezeigt werden, dass sich aus einer solchen Karte für jedes
andere Gruppenelement eine Karte für eine Umgebung ableiten lässt.
Sei also $\varphi_e\colon U_e \to \mathbb{R}^N$ eine Karte für die Umgebung
$U_e\subset G$ von $e\in G$.
Für $g\in G$ ist dann die Abbildung
\[
\varphi_g
\colon
U_g
=
gU_e
\to
\mathbb{R}
:
h\mapsto \varphi_e(g^{-1}h)
\]
eine Karte für die Umgebung $U_g$ des Gruppenelementes $g$.
Schreibt man $l_{g}$ für  die Abbildung $h\mapsto gh$, dann
kann man die Kartenabbildung auch $\varphi_g = \varphi_e\circ l_{g^{-1}}$
schreiben.

\subsubsection{Kartenwechsel}
Die Kartenwechsel-Abbildungen für zwei Karten $\varphi_{g_1}$
und $\varphi_{g_2}$ ist die Abbildung
\[
\varphi_{g_1,g_2}
=
\varphi_{g_1}\circ \varphi_{g_2}^{-1}
=
\varphi_e\circ l_{g_1^{-1}} \circ (\varphi_e\circ l_{g_2^{-1}})^{-1}
=
\varphi_e\circ l_{g_1^{-1}} \circ l_{g_2^{-1}}^{-1} \varphi_e^{-1}
=
\varphi_e\circ l_{g_1^{-1}} \circ l_{g_2}\varphi_e^{-1}
=
\varphi_e\circ l_{g_1^{-1}g_2}\varphi_e^{-1}
\]
mit der Ableitung
\[
D\varphi_e\circ Dl_{g_1^{-1}g_2} D\varphi_e^{-1}
=
D\varphi_e\circ Dl_{g_1^{-1}g_2} (D\varphi_e)^{-1}.
\]
Die Abbildung $l_{g_1^{-1}g_2}$ ist aber nur die Multiplikation mit
einer Matrix, also eine lineare Abbildung, so dass der Kartenwechsel
nichts anderes ist als die Darstellung der Matrix der Linksmultiplikation
$l_{g_1^{-1}g_2}$ im Koordinatensystem der Karte $U_e$ ist.
Differenzierbarkeit der Kartenwechsel ist damit sichergestellt.
Somit sind
die Matrizengruppen automatisch differenzierbare Mannigfaltigkeiten.

Die Konstruktion aller Karten aus einer einzigen Karte für eine
Umgebung des neutralen Elements zeigt auch, dass es für die Matrizengruppen
reicht, wenn man die Elemente in einer Umgebung des neutralen
Elementes parametrisieren kann.
Dies ist jedoch nicht nur für die Matrizengruppen möglich.
Wenn eine Gruppe gleichzeitig eine differenzierbare Mannigfaltigkeit
ist, dann können Karten über die ganze Gruppe transportiert werden,
wenn die Multiplikation mit Gruppenelementen eine differenzierbare
Abbildung ist.
Solche Gruppen heissen auch Lie-Gruppen gemäss der folgenden Definition.

\begin{definition}
\index{Lie-Gruppe}%
Eine {\em Lie-Gruppe} ist eine Gruppe, die gleichzeitig eine differenzierbare
Mannigfaltigkeit ist derart, dass die Abbildungen
\begin{align*}
G\times G \to G &: (g_1,g_2)\mapsto g_1g_2
\\
G\to G &: g \mapsto g^{-1},
\end{align*}
die zu den Gruppenoperationen gehören,
differenzierbare Abbildungen zwischen Mannigfaltigkeiten sind.
\end{definition}

Die Abstraktheit dieser Definition täuscht etwas über die 
Tatsache hinweg, dass sich mit Hilfe der Darstellungstheorie
jede beliebige Lie-Gruppe als Untermannigfaltigkeit einer 
Matrizengruppe verstehen lässt.
Das Studium der Matrizengruppen erlaubt uns daher ohne grosse
Einschränkungen ein Verständnis für die Theorie der Lie-Gruppen
zu entwickeln.

\subsubsection{Tangentialvektoren und die Exponentialabbildung}
Die Matrizengruppen sind alle in der
$n^2$-dimensionalen Mannigfaltigkeit $\operatorname{GL}_n(\mathbb{R})$
enthalten.
Diffferenzierbare Kurven $\gamma(t)$ in $\operatorname{GL}_n(\mathbb{R})$
haben daher in jedem Punkt Tangentialvektoren, die als Matrizen in
$M_n(\mathbb{R})$ betrachtet werden können.
Wenn $\gamma(t)$ die Matrixelemente $\gamma_{i\!j}(t)$ hat, dann ist der
Tangentialvektor im Punkt $\gamma(t)$ durch
\[
\frac{d}{dt}
\gamma(t)
=
\begin{pmatrix}
\dot{\gamma}_{11}(t)&\dots &\dot{\gamma}_{1n}(t)\\
\vdots              &\ddots&\vdots              \\
\dot{\gamma}_{n1}(t)&\dots &\dot{\gamma}_{nn}(t)
\end{pmatrix}
\]
gegeben.

Im Allgemeinen kann man Tangentialvektoren in verschiedenen Punkten
einer Mannigfaltigkeit nicht miteinander vergleichen.
Die Multiplikation $l_g$, die den Punkt $e$ in den Punkt $g$ verschiebt,
transportiert auch die Tangentialvektoren im Punkt $e$ in 
Tangentialvektoren im Punkt $g$.

\begin{aufgabe}
Gibt es eine Kurve $\gamma(t)\in\mathbb{GL}_n(\mathbb{R})$ mit
$\gamma(0)=e$ derart, dass der Tangentialvektor im Punkt $\gamma(t)$
für $t>0$ derselbe ist wie der Tangentialvektor im Punkt $e$, transportiert
durch Matrixmultiplikation mit $\gamma(t)$?
\end{aufgabe}

Eine solche Kurve muss die Differentialgleichung
\begin{equation}
\frac{d}{dt}\gamma(t)
=
\gamma(t)\cdot A
\label{buch:gruppen:eqn:expdgl}
\end{equation}
erfüllen, wobei $A\in M_n(\mathbb{R})$ der gegebene Tangentialvektor
in $e=I$ ist.

Die {\em Matrixexponentialfunktion}
\index{Matrixexponentialfunktion}%
\[
e^{At}
=
1+At+\frac{A^2t^2}{2!}+\frac{A^3t^3}{3!}+\frac{A^4t^4}{4!}+\dots
\]
liefert eine Einparametergruppe
$\mathbb{R}\to \operatorname{GL}_n(\mathbb{R})$ mit der Ableitung
\[
\frac{d}{dt} e^{At}
=
\lim_{h\to 0} \frac{e^{A(t+h)}-e^{At}}{h}
=
\lim_{h\to 0} e^{At}\frac{e^{Ah}-I}{h}
=
e^{At} A.
\]
Sie ist also Lösung der Differentialgleichung~\eqref{buch:gruppen:eqn:expdgl}.

\subsection{Drehungen in der Ebene
\label{buch:gruppen:drehungen2d}}
Die Drehungen der Ebene sind die orientierungserhaltenden Symmetrien
des Einheitskreises, der in Abbildung~\ref{buch:gruppen:fig:kartenkreis}
als Mannigfaltigkeit erkannt wurde.
Sie bilden eine Lie-Gruppe, die auf verschiedene Arten als Matrizen
beschrieben werden können.

\subsubsection{Die Untergruppe
$\operatorname{SO}(2)\subset \operatorname{GL}_2(\mathbb{R})$}
Drehungen der Ebene können in einer orthonormierten Basis durch
Matrizen der Form
\begin{equation}
R_{\alpha}
=
\begin{pmatrix}
\cos\alpha&-\sin\alpha\\
\sin\alpha& \cos\alpha
\end{pmatrix}
\label{buch:lie:eqn:ralphadefinition}
\end{equation}
dargestellt werden.
Wir bezeichnen die Menge der Drehmatrizen in der Ebene mit
$\operatorname{SO}(2)\subset\operatorname{GL}_2(\mathbb{R})$.
Die Abbildung
\[
R_{\bullet}
\colon
\mathbb{R}\to \operatorname{SO}(2)
:
\alpha \mapsto R_{\alpha}
\]
hat die Eigenschaften
\begin{equation}
\begin{aligned}
R_{\alpha+\beta}&= R_{\alpha}R_{\beta}
\\
R_0&=I
\\
R_{2k\pi}&=I\qquad \forall k\in\mathbb{Z}.
\end{aligned}
\label{buch:lie:so2matrizen}
\end{equation}
Daraus folgt zum Beispiel, dass $R_{\bullet}$ eine $2\pi$-periodische
Funktion ist.
$R_{\bullet}$ bildet die Menge der Winkel $[0,2\pi)$ bijektiv auf
die Menge der Drehmatrizen in der Ebene ab.

Für jedes Intervall $(a,b)\subset\mathbb{R}$ mit Länge
$b-a < 2\pi$ ist die Abbildung $\alpha\mapsto R_{\alpha}$ umkehrbar,
die Umkehrung kann als Karte verwendet werden.
Zwei verschiedene Karten $\alpha_1\colon U_1\to\mathbb{R}$ und
$\alpha_2\colon U_2\to\mathbb{R}$ bilden die Elemente $g\in U_1\cap U_2$
in Winkel $\alpha_1(g)$ und $\alpha_2(g)$ ab, für die 
$R_{\alpha_1(g)}=R_{\alpha_2(g)}$ gilt.
Dies ist gleichbedeutend damit, dass $\alpha_1(g)=\alpha_2(g)+2\pi k$
mit $k\in \mathbb{Z}$.
In einem Intervall in $U_1\cap U_2$ muss $k$ konstant sein.
Die Kartenwechselabbildung ist also nur die Addition eines Vielfachen
von $2\pi$, mit der identischen Abbildung als Ableitung.
Diese Karten führen also auf besonders einfache Kartenwechselabbildungen.

\subsubsection{Die Untergruppe $S^1\subset\mathbb{C}$}
Ein alternatives Bild für die Drehungen der Ebene kann man in der komplexen
Ebene $\mathbb{C}$ erhalten.
Die Multiplikation mit der komplexen Zahl $e^{i\alpha}$ beschreibt eine
Drehung der komplexen Ebene um den Winkel $\alpha$.
Die Zahlen der Form $e^{i\alpha}$ haben den Betrag $1$ und die Abbildung
\[
f\colon \mathbb{R}\to \mathbb{C}:\alpha \mapsto e^{i\alpha}
\]
hat die Eigenschaften
\begin{equation}
\begin{aligned}
f(\alpha+\beta) &= f(\alpha)f(\beta)
\\
f(0)&=1
\\
f(2\pi k)&=1\qquad\forall k\in\mathbb{Z},
\end{aligned}
\label{buch:lie:so2komplex}
\end{equation}
die zu den Eigenschaften
\eqref{buch:lie:so2matrizen} der Abbildung $\alpha\mapsto R_{\alpha}$ 
analog sind.

Jede komplexe Zahl $z$ vom Betrag $1$ kann geschrieben werden in der Form
$z=e^{i\alpha}$.
Die Abbildung $f$ ist also eine Parametrisierung des
Einheitskreises in der Ebene.
Wir bezeichen $S^1=\{z\in\mathbb{C} \mid |z|=1\}$ die komplexen Zahlen vom
Betrag $1$.
$S^1$ ist eine Gruppe bezüglich der Multiplikation, da für alle Zahlen
$z,w\in S^1$ gilt
$|z^{-1}|=1$ und $|zw|=1$ und damit $z^{-1}\in S^1$ und $zw\in S^1$.

Zu einer komplexen Zahl $z\in S^1$ gibt es einen bis auf Vielfache
von $2\pi$ eindeutigen Winkel $\alpha(z)$ derart, dass $e^{i\alpha(z)}=z$.
Damit kann man jetzt die Abbildung
\[
\varphi
\colon
S^1\to \operatorname{SO}(2)
:
z\mapsto  R_{\alpha(z)}
\]
konstruieren.
Da $R_{\alpha}$ $2\pi$-periodisch ist, geben um Vielfache
von $2\pi$ verschiedene Wahlen von $\alpha(z)$ die gleiche
Matrix $R_{\alpha(z)}$, die Abbildung $\varphi$ ist daher
wohldefiniert.
$\varphi$ erfüllt ausserdem die Bedingungen
\begin{align*}
\varphi(z_1z_2)
&=
R_{\alpha(z_1z_2)}
=
R_{\alpha(z_1)+\alpha(z_2)}
=
R_{\alpha(z_1)}R_{\alpha(z_2)}
=
\varphi(z_1)\varphi(z_2),
\\
\varphi(1)
&=
R_{\alpha(1)}
=
R_0
=
I.
\end{align*}
Die Abbildung $\varphi$ ist ein Homomorphismus der Gruppe $S^1$
in die Gruppe $\operatorname{SO}(2)$.
Die Menge der Drehmatrizen in der Ebene kann also mit dem Einheitskreis
in der komplexen Ebene identifiziert werden.

\subsubsection{Tangentialvektoren von $\operatorname{SO}(2)$}
Da die Gruppe $\operatorname{SO}(2)$ eine eindimensionale Gruppe
ist, kann jede Kurve $\gamma(t)$ durch den Drehwinkel $\alpha(t)$
mit $\gamma(t) = R_{\alpha(t)}$ beschrieben werden.
Die Ableitung in $M_2(\mathbb{R})$ ist
\begin{align*}
\frac{d}{dt} \gamma(t)
&=
\frac{d}{d\alpha}
\begin{pmatrix}
\cos\alpha(t) & - \sin\alpha(t)\\
\sin\alpha(t) &   \cos\alpha(t)
\end{pmatrix}
\cdot
\frac{d\alpha}{dt}
\\
&=
\begin{pmatrix}
-\sin\alpha(t)&-\cos\alpha(t)\\
 \cos\alpha(t)&-\sin\alpha(t)
\end{pmatrix}
\cdot
\dot{\alpha}(t)
\\
&=
\begin{pmatrix}
\cos\alpha(t) & - \sin\alpha(t)\\
\sin\alpha(t) &   \cos\alpha(t)
\end{pmatrix}
\begin{pmatrix}
0&-1\\
1&0
\end{pmatrix}
\cdot
\dot{\alpha}(t)
=
R_{\alpha(t)}J\cdot\dot{\alpha}(t).
\end{align*}
Alle Tangentialvektoren von $\operatorname{SO}(2)$ im Punkt $R_\alpha$
entstehen aus $J$ durch Drehung mit der Matrix $R_\alpha$ und Skalierung
mit der Winkelgeschwindigkeit $\dot{\alpha}(t)$.
\index{Winkelgeschwindigkeit}%

\subsection{Symmetrien des harmonischen Oszillators
\label{buch:gruppen:symmetrien-harm-osz}}
Im Abschnitt über den harmonischen Oszillator
auf Seite~\pageref{buch:gruppen:harmonischer-oszillator}
wurde für die Einparameteruntergruppe
$\Phi_t\in\operatorname{GL}_2(\mathbb{R})$ der
Ausdruck~\eqref{buch:gruppen:eqn:phi} gefunden.
Die Ableitung von $\Phi_t$ an der Stelle $t=0$ ist
\begin{align*}
\frac{d}{dt}\Phi_t\bigg|_{t=0}
&=
\frac{d}{dt}
\begin{pmatrix}
\cos\omega t&-\frac{1}{\omega}\sin\omega t\\
\omega\sin\omega t&\cos\omega t
\end{pmatrix}
\bigg|_{t=0}
=
\begin{pmatrix}
-\omega\sin\omega t&-\cos\omega t\\
\omega^2\cos\omega t&-\omega\sin\omega t
\end{pmatrix}
\bigg|_{t=0}
=
\begin{pmatrix}
0&-1\\\omega^2&0
\end{pmatrix}
=
A.
\end{align*}
Die Potenzen von $A$ sind
\[
A^2
=
\begin{pmatrix} -\omega^2&0\\0&-\omega^2\end{pmatrix}
=
-\omega^2 I,
\quad
A^3
=
-\omega^2 A,
\quad
A^4
=
\omega^4 I.
\]
Die Potenzen wiederholen sich bis auf den Faktor $\omega^4$ mit Periode 4.
Damit kann man jetzt die Exponentialabbildung für $At$ berechnen:
\begin{align*}
e^{At}
&=
I+At+\frac{A^2t^2}{2!}+\frac{A^3t^3}{3!}+\frac{A^4t^4}{4!}+\frac{A^5t^5}5!+\dots
\\
&=
I+\frac{1}{\omega}A\omega t-I\frac{\omega^2t^2}{2!}
-\frac1{\omega}A\frac{\omega^3t^3}{3!}
+\frac{\omega^4t^4}{4!}
+\frac{1}{\omega}\frac{\omega^5t^5}{5!}+\dots
\\
&= I\cos\omega t + \frac1{\omega}A\sin\omega t
=
\begin{pmatrix}
\cos\omega t       &-\frac{1}{\omega}\sin\omega t\\
\omega\sin\omega t & \cos\omega t
\end{pmatrix} = \Phi_t.
\end{align*}
Der Fluss der Differentialgleichung des harmonischen Oszillators ist
also nichts anderes als die Exponentialabbildung der Ableitung $A$ zur
Zeit $t=0$.

%
% Isometrien von R^n
%
\subsection{Isometrien von $\mathbb{R}^n$
\label{buch:gruppen:isometrien}}
Isometrien von $\mathbb{R}^n$ führen automatisch auf eine interessante
Lie-Gruppe, die in diesem Abschnitt untersucht werden soll.

\subsubsection{Skalarprodukt}
Lineare Abbildungen des Raumes $\mathbb{R}^n$ können durch
$n\times n$-Matrizen beschrieben werden.
Die Matrizen, die das Standardskalarprodukt $\mathbb{R}^n$ erhalten,
bilden eine Gruppe, die in diesem Abschnitt genauer untersucht werden soll.
Eine Matrix $A\in M_{n}(\mathbb{R})$ ändert das Skalarprodukt nicht, wenn
für jedes beliebige Paar $x,y$ von Vektoren 
$\langle Ax,Ay\rangle = \langle x,y\rangle$ gilt.
Das Standardskalarprodukt kann mit dem Matrixprodukt ausgedrückt werden:
\begin{equation}
\langle Ax,Ay\rangle
=
(Ax)^tAy
=
x^tA^tAy
\overset{!}{=}
x^ty
=
\langle x,y\rangle
\label{buch:gruppen:eqn:orthogonalbed}
\end{equation}
für jedes Paar von Vektoren $x,y\in\mathbb{R}$.
%
Mit dem Skalarprodukt kann man auch die Matrixelemente einer Matrix
einer Abbildung $f$ in der Standardbasis bestimmen.
Das Skalarprodukt $\langle e_i, v\rangle$ ist die Länge der Projektion
des Vektors $v$ auf die Richtung $e_i$.
Die Komponenten von $Ae_j$ sind daher $a_{i\!j}=\langle e_i,f(e_j)\rangle$.
Die Matrix $A$ der Abbildung $f$ hat folglich die Matrixelemente
$a_{i\!j}=e_i^tAe_j$.

\subsubsection{Die orthogonale Gruppe $\operatorname{O}(n)$}
Die Matrixelemente von $A^tA$ können
mit der Bedingung \eqref{buch:gruppen:eqn:orthogonalbed}
berechnet werden als
$\langle A^tAe_i, e_j\rangle =\langle e_i,e_j\rangle = \delta_{i\!j}$.
Die Matrix $A$ erfüllt also $AA^t=I$ oder $A^{-1}=A^t$.
Solche Matrizen heissen {\em orthogonale} Matrizen.
\index{orthogonale Matrix}%
Die Menge $\operatorname{O}(n)$ der isometrischen Abbildungen
\index{O(n)@$\operatorname{O}(n)$}%
von $\mathbb{R}^n$ besteht
daher aus den Matrizen
\[
\operatorname{O}(n)
=
\{ A\in M_n(\mathbb{R}) \mid AA^t=I\}.
\]
Die Matrixgleichung $AA^t=I$ liefert $n(n+1)/2$ unabhängige Bedingungen,
die die orthogonalen Matrizen innerhalb der $n^2$-dimensionalen
Menge $M_n(\mathbb{R})$ auszeichnen.
Die Menge $\operatorname{O}(n)$ der orthogonalen Matrizen hat daher
die Dimension
\[
n^2 - \frac{n(n+1)}{2}
=
\frac{2n^2-n^2-n}{2}
=
\frac{n(n-1)}2.
\]
Im Spezialfall $n=2$ ist die Gruppe $\operatorname{O}(2)$ eindimensional.

\subsubsection{Tangentialvektoren}
Die orthogonalen Matrizen bilden eine abgeschlossene Untermannigfaltigkeit
von $\operatorname{GL}_n(\mathbb{R})$, nicht jede Matrix $M_n(\mathbb{R})$ 
kann also ein Tangentialvektor von $\operatorname{O}(n)$ sein.
Um herauszufinden, welche Matrizen als Tangentialvektoren in Frage
kommen, betrachten wir eine Kurve
$\gamma\colon\mathbb{R}\to \operatorname{O}(n)$
von orthogonalen Matrizen mit $\gamma(0)=I$.
Orthogonal bedeutet 
\[
\begin{aligned}
&&
0
&=
\frac{d}{dt}I
=
\frac{d}{dt}
(\gamma(t)^t\gamma(t))
=
\dot{\gamma}(t)^t\gamma(t))
+
\gamma(t)^t\dot{\gamma}(t))
\\
&\Rightarrow&
0
&=
\dot{\gamma}(0)^t \cdot I + I\cdot \dot{\gamma(0)}
=
\dot{\gamma}(0)^t + \dot{\gamma}(0)
=
A^t+A=0
\\
&\Rightarrow&
A^t&=-A
\end{aligned}
\]
Die Tangentialvektoren von $\operatorname{O}(n)$ sind also genau
die antisymmetrischen Matrizen.
\index{antisymmetrisch}%

Für $n=2$ sind alle antisymmetrischen Matrizen Vielfache der Matrix
$J$, wie in Abschnitt~\ref{buch:gruppen:drehungen2d}
gezeigt wurde.

Für jedes Paar $i<j$ ist die Matrix
\begin{equation}
\begin{tikzpicture}[>=latex,thick,baseline=(O)]
\coordinate (O) at (0,0);
\draw[dotted,color=gray] (-1.2,0.42) -- (2.5,0.42);
\draw[dotted,color=gray] (-1.2,-0.4) -- (2.5,-0.4);
\draw[dotted,color=gray] (-0.14,-1.4) -- (-0.14,1.4);
\draw[dotted,color=gray] (0.96,-1.4) -- (0.96,1.4);
\node at (2.5,0.42) [right] {$i\mathstrut$};
\node at (2.5,-0.40) [right] {$j\mathstrut$};
\node at (-0.14,1.4) [above] {$i\mathstrut$};
\node at (0.96,1.4) [above] {$j\mathstrut$};
\node at (0,0) {$\displaystyle
\Omega_{i\!j}
=
\begin{pmatrix*}[r]
& & & & &  & & & \\
& & & & &  & & & \\
& & &0& &-1& & & \\
& & & & &  & & & \\
& & &1& & 0& & & \\
& & & & &  & & & \\
& & & & &  & & &
\end{pmatrix*}
$};
\end{tikzpicture}
\label{buch:gruppen:eqn:Omega}
\end{equation}
mit den Matrixelementen
$(\Omega_{i\!j})_{i\!j}=-1$ und $(\Omega_{i\!j})_{ji}=1$
antisymmetrisch.
Für $n=2$ ist $\Omega_{12}=J$.
Die $n(n-1)/2$ Matrizen $\Omega_{i\!j}$ bilden eine Basis des
$n(n-1)/2$-dimensionale Tangentialraumes von $\operatorname{O}(n)$.

Tangentialvektoren in einem anderen Punkt $g\in\operatorname{O}(n)$
haben die Form $gA$, wobei $A$ eine antisymmetrische Matrix ist.
Diese Matrizen sind nur noch in speziellen Fällen antisymmetrisch,
zum Beispiel im Punkt $-I\in\operatorname{O}(n)$.

\subsubsection{Die Gruppe $\operatorname{SO}(n)$}
Die Gruppe $\operatorname{O}(n)$ enhält auch Isometrien, die
die Orientierung des Raumes umkehren, wie zum Beispiel Spiegelungen.
Wegen $\det (AA^t)=\det A\det A^t = (\det A)^2=1$ kann die Determinante
einer orthogonalen Matrix nur $\pm 1$ sein.
Orientierungserhaltende Isometrien haben Determinante $1$.

\begin{definition}
Die Gruppe
\[
\operatorname{SO}(n)
=
\{A\in\operatorname{O}(n)\mid\det A=1\}
\]
der orientierungserhaltenden Isometrien von $\mathbb{R}^n$
heisst die {\em spezielle orthogonale Gruppe}.
\index{spezielle orthogonale Gruppe}%
\index{orthogonale Gruppe, speziell}%
\index{Gruppe, spezielle orthogonale}%
\index{SO(n)@$\operatorname{SO}(n)$}%
\end{definition}

%Die Dimension der Gruppe $\operatorname{SO}(n)$ ist $n(n-1)/2$.

\subsubsection{Die Gruppe $\operatorname{SO}(3)$}
Die Gruppe $\operatorname{SO}(3)$ der Drehungen des dreidimensionalen
Raumes hat die Dimension $3(3-1)/2=3$.
Eine Drehung wird festgelegt durch die Richtung der Drehachse und den
Drehwinkel.
Die Richtung der Drehachse ist ein Einheitsvektor, also ein Punkt
auf der zweidimensionalen Kugel.
Der Drehwinkel ist der dritte Parameter.

Drehungen mit kleinen Drehwinkeln können zusammengesetzt werden
aus den Matrizen
\begin{align*}
R_{x,\alpha}
&=
\begin{pmatrix}
1&0&0\\
0&\cos\alpha&-\sin\alpha\\
0&\sin\alpha& \cos\alpha
\end{pmatrix},
&
R_{y,\beta}
&=
\begin{pmatrix}
 \cos\beta&0&\sin\beta\\
      0    &1&     0    \\
-\sin\beta&0&\cos\beta
\end{pmatrix},
&
R_{z,\gamma}
&=
\begin{pmatrix}
\cos\gamma&-\sin\gamma&0\\
\sin\gamma& \cos\gamma&0\\
    0     &     0     &1
\end{pmatrix}
\\
&=
e^{\Omega_{23}t}
&
&=
e^{-\Omega_{13}t}
&
&=
e^{\Omega_{21}t}
\end{align*}
die Drehungen um die Koordinatenachsen um den Winkel $\alpha$
beschreiben.
Auch die Winkel $\alpha$, $\beta$ und $\gamma$ können als die
drei Koordinaten der Mannigkfaltigkeit $\operatorname{SO}(3)$
angesehen werden.

\begin{figure}
\centering
\includegraphics{chapters/60-gruppen/images/rodriguez.pdf}
\caption{Herleitung der Rodrigues-Formel~\eqref{buch:lie:eqn:rodrigues}
für die Beschreibung einer
Drehung mit Drehachse $\vec{k}$.
\label{buch:lie:fig:rodrigues}}
\end{figure}
Die Drehung des Vektors $\vec{x}$ um die Achse mit Richtung $\vec{k}$,
$|\vec{k}|=1$, kann man mit dem Vektorprodukt und dem Skalarprodukt
beschreiben.
Die Vektoren $\vec{x}-(\vec{x}\cdot\vec{k})\vec{k}$, $-\vec{x}\times\vec{k}$
und $\vec{k}$ bilden ein Rechtssystem im Punkt $\vec{x}$, dessen zweite
Achse tangential an die Bahn von $\vec{x}$ unter der Drehung ist
(siehe Abbildung~\ref{buch:lie:fig:rodrigues}).
%
Die Komponente $(\vec{k}\cdot\vec{x})\vec{k}$ parallel zu $\vec{k}$
ändert sich bei der Drehung nicht.
In der Ebene mit der orthogonalen Basis aus den Vektoren
$\vec{x}-(\vec{x}\cdot\vec{k})\vec{k}$ und $-\vec{x}\times\vec{k}$
kann man die Drehung $R_\alpha$ um den Winkel $\alpha$ mit den
trigonometrischen Funktionen beschreiben:
\begin{align}
\vec{x}
\mapsto
R_\alpha\vec{x}
&=
(\vec{x}-(\vec{x}\cdot\vec{k})\vec{k})
\cos\alpha
-
\vec{x}\times\vec{k}
\sin\alpha
+
(\vec{k}\cdot\vec{x})\vec{k}
\notag
\\
&=
\vec{x}\cos\alpha
+
(1-\cos\alpha)(\vec{x}\cdot\vec{k})\vec{k}
+
\vec{k}\times\vec{x}\sin\alpha.
\label{buch:lie:eqn:rodrigues}
\end{align}
\eqref{buch:lie:eqn:rodrigues} 
ist bekannt als die {\em Formel von Rodrigues}.
\index{Formel von Rodrigues}%
\index{Rodrigues-Formel}%
Wir halten noch fest, dass die Ableitung
von \eqref{buch:lie:eqn:rodrigues} 
an der Stelle $\alpha=0$
der Tangentialvektor
\begin{equation}
\frac{d}{d\alpha}R_\alpha\vec{x}\,\bigg|_{\alpha=0}
=
\vec{k}\times\vec{x}
\label{buch:lie:eqn:so3tangentialvektor}
\end{equation}
ist.

%
% Spezielle lineare Gruppe
%
\subsection{Volumenerhaltende Abbildungen und
die Gruppe $\operatorname{SL}_n(\mathbb{R})$
\label{buch:gruppen:sl}}
Die Elemente der Gruppe $\operatorname{SO}(n)$ erhalten Längen, Winkel und die
Orientierung, also auch das Volumen.
Es gibt aber volumenerhaltende Abbildungen, die Längen oder Winkel
nicht notwendigerweise erhalten, zum Beispiel Scherungen.
Matrizen $A\in M_n(\mathbb{R})$, die das Volumen erhalten,
haben die Determinante $\det A=1$.
Wegen $\det(AB)=\det A\det B$ ist das Produkt zweier Matrizen mit
Determinante $1$ wieder eine solche, sie bilden daher eine Gruppe.

\begin{definition}
Die volumenerhaltenden Abbildungen bilden die Gruppe
\index{volumenerhaltend}%
\[
\operatorname{SL}_n(\mathbb{R})
=
\{
A\in M_n(\mathbb{R})
\mid
\det (A) = 1
\},
\]
sie heisst die {\em spezielle lineare Gruppe}.
\index{spezielle lineare Gruppe}%
\index{Gruppe, spezielle lineare}%
\index{SLn(R)@$\operatorname{SL}_n(\mathbb{R})$}%
\end{definition}

Wir wollen jetzt die Tangentialvektoren von $\operatorname{SL}_n(\mathbb{R})$
bestimmen.
Dazu sei $A(t)$ eine Kurve in $\operatorname{SL}_n(\mathbb{R})$
mit $A(0)=I$.
Für alle $t\in\mathbb{R}$ ist $\det A(t)=1$, daher ist die Ableitung
\[
\frac{d}{dt} \det A(t) = 0
\quad\text{an der Stelle $t=0$.}
\]
Für $n=2$ ist
\begin{align}
A(t)
&=
\begin{pmatrix}
a(t)&b(t)\\
c(t)&d(t)
\end{pmatrix}
\in
\operatorname{SL}_2(\mathbb{R})
&&\Rightarrow&
\frac{d}{dt}
\det A(t)\bigg|_{t=0}
&=
\frac{d}{dt}\bigl(a(t)d(t)-b(t)c(t)\bigr)\bigg|_{t=0}
\notag
\\
&&&&
&=
\dot{a}(0) d(0)+a(0)\dot{d}(0)
-
\dot{b}(0) c(0)-b(0)\dot{c}(0)
\notag
\\
&&&&
&=
\dot{a}(0) + \dot{d}(0)
\notag
\\
&&&&
\frac{d}{dt}
\det A(t)\bigg|_{t=0}
&=
\operatorname{Spur}\frac{dA}{dt}.
\label{buch:gruppen:eqn:spurformel}
\end{align}
Die Spurformel~\eqref{buch:gruppen:eqn:spurformel}
gilt nicht nur im Falle $n=2$, sondern ganz allgemein für beliebige
$n\times n$-Matrizen.

\begin{satz}
Ist $A(t)$ eine differenzierbare Kurve in $\operatorname{SL}_n(\mathbb{R})$
mit $A(0)=I$, dann ist $\operatorname{Spur}\dot{A}(0)=0$.
\end{satz}

\begin{proof}[Beweis]
Die Entwicklung der Determinante von $A$ nach der ersten Spalte ist
\[
\det A(t) = \sum_{i=1}^n (-1)^{i+1} a_{i1}(t) \det A_{i1}(t),
\]
Wobei $A_{i\!j}(t)$ der $i$-$k$-Minor von $A(t)$ ist
(Seite~\pageref{buch:linear:def:minor}).
Die Ableitung nach $t$ ist
\[
\frac{d}{dt} \det A(t)
=
\sum_{i=1}^n (-1)^{i+1} \dot{a}_{i1}(t) \det A_{i1}(t).
+
\sum_{i=1}^n (-1)^{i+1} a_{i1}(t) \frac{d}{dt}\det A_{i1}(t).
\]
An der Stelle $t=0$ enthält $\det A_{i1}(0)$ für $i\ne 1$
eine Nullzeile, der einzige nichtverschwindende Term in der ersten
Summe ist daher der erste.
In der zweiten Summe ist das einzige nicht verschwindende $a_{i1}(0)$
jenes für $i=1$, somit ist die Ableitung von $\det A(t)$
\begin{equation}
\frac{d}{dt} \det A(t)
=
\dot{a}_{11}(t) \det A_{11}(t).
+
\frac{d}{dt}\det A_{11}(t)
=
\dot{a}_{11}(0) 
+
\frac{d}{dt}\det A_{11}(t).
\label{buch:gruppen:eqn:detspur}
\end{equation}
Die Beziehung \eqref{buch:gruppen:eqn:detspur} kann wie folgt
für einen Beweis mit vollständiger Induktion verwendet werden.

Die Induktionsverankerung für $n=1$ besagt, dass $\det A(t)=a_{11}(t)$
genau dann konstant $=1$ ist, wenn $\dot{a}_{11}(0)=\operatorname{Spur}A(0)$
ist.
Unter der Induktionsannahme, dass für eine $(n-1)\times(n-1)$-Matrix
$\tilde{A}(t)$ mit $\tilde{A}(0)=I$ die Ableitung der Determinante
\[
\frac{d}{dt}\tilde{A}(0)
=
\operatorname{Spur}\dot{\tilde{A}}(0)
\]
ist, folgt jetzt mit
\eqref{buch:gruppen:eqn:detspur}, dass
\[
\frac{d}{dt}A(0)
=
\dot{a}_{11}(0)
+
\frac{d}{dt} \det A_{11}(t)\bigg|_{t=0}
=
\dot{a}_{11}(0)
+
\operatorname{Spur}\dot{A}_{11}(0)
=
\operatorname{Spur}\dot{A}(0).
\]
Damit folgt jetzt die Behauptung für alle $n$.
\end{proof}

\begin{beispiel}
Die Tangentialvektoren von $\operatorname{SL}_2(\mathbb{R})$ sind 
die spurlosen Matrizen
\[
A=\begin{pmatrix}a&b\\c&d\end{pmatrix}
\quad\Rightarrow\quad
\operatorname{Spur}A=a+d=0
\quad\Rightarrow\quad
A=\begin{pmatrix}a&b\\c&-a\end{pmatrix}.
\]
Der Tangentialraum ist also dreidimensional.
\begin{figure}
\centering
\includegraphics{chapters/60-gruppen/images/sl2.pdf}
\caption{Tangentialvektoren und die davon erzeugen Einparameteruntergruppen
für die Lie-Gruppe $\operatorname{SL}_2(\mathbb{R})$ der flächenerhaltenden
linearen Abbildungen von $\mathbb{R}^2$.
In allen drei Fällen wird das blaue Quadrat mit den Ecken in den
Standardbasisvektoren von einer Matrix der Einparameteruntergruppe 
zum roten Viereck verzerrt, der Flächeninhalt bleibt aber erhalten.
In den beiden Fällen $B$ und $C$ stellen die grünen Kurven die Bahnen
der Bilder der Standardbasisvektoren dar.
\label{buch:gruppen:fig:sl2}}
\end{figure}%
Als Basis könnte man die folgenden Vektoren verwenden:
\begin{align*}
A
&=
\begin{pmatrix}1&0\\0&-1\end{pmatrix}
&&\Rightarrow&
e^{At}
&=
\begin{pmatrix} e^t & 0 \\ 0 & e^{-t} \end{pmatrix}
\\
B
&=
\begin{pmatrix}0&-1\\1&0\end{pmatrix}
&&\Rightarrow&
e^{Bt}
&=
\begin{pmatrix}
\cos t & -\sin t\\
\sin t &  \cos t
\end{pmatrix}
\\
C
&=
\begin{pmatrix}0&1\\1&0\end{pmatrix}
&&\Rightarrow&
e^{Ct}
&=
I + Ct + \frac{C^2t^2}{2!} + \frac{C^3t^3}{3!} + \frac{C^4t^4}{4!}+\dots
\\
&&&&
&=
I\biggl(1 + \frac{t^2}{2!} + \frac{t^4}{4!}+\dots \biggr)
+
C\biggl(t + \frac{t^3}{3!} + \frac{t^5}{5!}+\dots \biggr)
\\
&&&&
&=
I\cosh t + C \sinh t
=
\begin{pmatrix}
\cosh t & \sinh t\\
\sinh t & \cosh t
\end{pmatrix},
\end{align*}
wobei in der Auswertung der Potenzreihe für $e^{Ct}$ verwendet wurde,
dass $C^2=I$.
Die von $A$, $B$ und $C$ erzeugten Einparameteruntergruppen sind in
Abbildung~\ref{buch:gruppen:fig:sl2} visualisiert.

Die Matrizen $e^{At}$ sind Streckungen der einen Koordinatenachse und
Stauchungen der anderen derart, dass das Volumen erhalten bleibt.
Die Bahn eines Punktes unter Wirkung von $e^{At}$ ist eine Hyperbel
mit den Koordinatenachsen als Asymptoten.

Die Matrizen $e^{Bt}$ sind Drehmatrizen, die Längen und Winkel und
damit erst recht den Flächeninhalt erhalten.
Die Bahn eines Punktes ist ein Kreis um den Nullpunkt.

Die Matrizen der Form $e^{Ct}$ haben die Vektoren $(1,\pm1)$ als
Eigenvektoren:
\begin{align*}
\begin{pmatrix}1\\1\end{pmatrix}
&\mapsto
e^{Ct}
\begin{pmatrix}1\\1\end{pmatrix}
=
(\cosh t +\sinh t)
\begin{pmatrix}1\\1\end{pmatrix}
=
\biggl(
\frac{e^t+e^{-t}}2
+
\frac{e^t-e^{-t}}2
\biggr)
\begin{pmatrix}1\\1\end{pmatrix}
=
e^t
\begin{pmatrix}1\\1\end{pmatrix}
\\
\begin{pmatrix}1\\-1\end{pmatrix}
&\mapsto
e^{Ct}
\begin{pmatrix}1\\-1\end{pmatrix}
=
(\cosh t -\sinh t)
\begin{pmatrix}1\\-1\end{pmatrix}
=
\biggl(
\frac{e^t+e^{-t}}2
-
\frac{e^t-e^{-t}}2
\biggr)
\begin{pmatrix}1\\-1\end{pmatrix}
=
e^{-t}
\begin{pmatrix}1\\-1\end{pmatrix}
\end{align*}
Die Matrizen $e^{Ct}$ strecken die Richtung $(1,1)$ um $e^t$ und
die dazu orthogonale Richtung $(1,-1)$ um den Faktor $e^{-t}$.
Dies ist die gegenüber $e^{At}$ um $45^\circ$ verdrehte Situation,
auch diese Matrizen sind flächenerhaltend.
Die Bahnen einzelner Punkte unter $e^{Ct}$ sind Hyperbeln mit
den Winkelhalbierenden als Asymptoten.
\begin{figure}
\centering
\includegraphics{chapters/60-gruppen/images/scherungen.pdf}
\caption{Weitere Matrizen mit Spur $0$ und ihre Wirkung.
Die linken beiden Beispiele $M$ und $N$ sind nilpotente Matrizen,
die zugehörigen Einparameteruntergruppen beschreiben Scherungen.
\label{buch:gruppen:fig:scherungen}}
\end{figure}

Die Gruppe $\operatorname{SL}_2(\mathbb{R})$ hat aber auch die
Tangentialvektoren
\begin{align*}
M&=\begin{pmatrix}0&1\\0&0\end{pmatrix}=\frac12(B+C)
&&\text{und}&
N&=\begin{pmatrix}0&0\\1&0\end{pmatrix}=\frac12(-B+C),
\intertext{die die Scherungen}
e^{Mt}&= \begin{pmatrix}1&0\\t&0\end{pmatrix}
&&
e^{NT}&=\begin{pmatrix}1&t\\0&1\end{pmatrix}
\end{align*}
als Einparameteruntergruppen haben.
Diese sind in Abbildung~\ref{buch:gruppen:fig:scherungen} dargestellt.
\end{beispiel}

%
% Die Gruppe SU(2)
%
\subsection{Die Gruppe $\operatorname{SU}(2)$
\label{buch:gruppen:su2}}
Die Menge der Matrizen
\[
\operatorname{SU}(2)
=
\left\{
\left.
A=\begin{pmatrix} a&b\\c&d\end{pmatrix}
\;\right|\;
a,b,c,d\in\mathbb{C},\det(A)=1, AA^*=I
\right\}
\]
heisst die {\em spezielle unitäre Gruppe}.
\index{spezielle unitäre Gruppe}%
\index{unitäre Gruppe, speziell}%
\index{Gruppe, speziell unitäre}%
\index{SU(n)@$\operatorname{SU}(n)$}%
Wegen $\det(AB)=\det(A)\det(B)=1$ und $(AB)^*AB=B^*A^*AB=B^*B=I$ ist 
$\operatorname{SU}(2)$ eine Untergruppe von $\operatorname{GL}_2(\mathbb{C})$.
Die Bedingungen
\begin{equation}
\det A=1
\qquad\text{und}\qquad
AA^*=I
\label{buch:lie:eqn:su2bed}
\end{equation}
schränken die möglichen Werte
von $a$ und $b$ weiter ein.
Aus 
\[
A^*
=
\begin{pmatrix}
\overline{a}&\overline{c}\\
\overline{b}&\overline{d}
\end{pmatrix}
\]
und den Bedingungen~\eqref{buch:lie:eqn:su2bed} folgen die Gleichungen
\[
\begin{aligned}
a\overline{a}+b\overline{b}&=1
&&\Rightarrow&|a|^2+|b|^2&=1
\\
a\overline{c}+b\overline{d}&=0
&&\Rightarrow&
\frac{a}{b}&=-\frac{\overline{d}}{\overline{c}}
\\
c\overline{a}+d\overline{b}&=0
&&\Rightarrow&
\frac{c}{d}&=-\frac{\overline{b}}{\overline{a}}
\\
c\overline{c}+d\overline{d}&=1&&\Rightarrow&|c|^2+|d|^2&=1
\\
ad-bc&=1.
\end{aligned}
\]
Aus der zweiten Gleichung kann man ableiten, dass es eine Zahl $t\in\mathbb{C}$
gibt derart, dass $c=-t\overline{b}$ und $d=t\overline{a}$.
Damit wird die Bedingung an die Determinante zu
\[
1
=
ad-bc = at\overline{a} - b(-t\overline{b})
=
t(|a|^2+|b|^2)
=
t,
\]
also muss die Matrix $A$ die Form 
\[
A
=
\begin{pmatrix}
a&b\\
-\overline{b}&\overline{a}
\end{pmatrix}
\qquad\text{mit}\quad |a|^2+|b|^2=1
\]
haben.
Schreibt man $a=a_1+ia_2$ und $b=b_1+ib_2$ mit rellen $a_i$ und $b_i$,
dann besteht $\operatorname{SU}(2)$  aus den Matrizen der Form
\[
A=
\begin{pmatrix}
 a_1+ia_2&b_1+ib_2\\
-b_1+ib_2&a_1-ia_2
\end{pmatrix}
\]
mit der zusätzlichen Bedingung
\[
|a|^2+|b|^2
=
a_1^2 + a_2^2 + b_1^2 + b_2^2 = 1.
\]
Die Matrizen von $\operatorname{SU}(2)$ stehen daher in einer
eins-zu-eins-Beziehung zu den Vektoren $(a_1,a_2,b_1,b_2)\in\mathbb{R}^4$
eines vierdimensionalen reellen Vektorraums mit Länge $1$.
Geometrisch betrachtet ist also $\operatorname{SU}(2)$ eine dreidmensionalen
Kugel, die in einem vierdimensionalen Raum eingebettet ist.



