%
% phasenraum.tex -- 
%
% (c) 2021 Prof Dr Andreas Müller, OST Ostschweizer Fachhochschule
%
\documentclass[tikz]{standalone}
\usepackage{amsmath}
\usepackage{times}
\usepackage{txfonts}
\usepackage{pgfplots}
\usepackage{csvsimple}
\usetikzlibrary{arrows,intersections,math}
\begin{document}
\def\skala{1}
\begin{tikzpicture}[>=latex,thick,scale=\skala]

\pgfmathparse{1/sqrt(2)}
\xdef\o{\pgfmathresult}

\def\punkt#1#2{ ({#2*cos(#1)},{\o*#2*sin(#1)}) }

\foreach \r in {1,2,...,6}{
	\draw[line width=0.5pt]
		plot[domain=0:359,samples=360]
			({\r*cos(\x)},{\o*\r*sin(\x)}) -- cycle;
}
\draw[color=red,line width=1.4pt]
	plot[domain=0:359,samples=360]
		({4*cos(\x)},{\o*4*sin(\x)}) -- cycle;

\draw[->] (-6.1,0) -- (6.3,0) coordinate[label={$x$}];
\draw[->] (0,-4.4) -- (0,4.7) coordinate[label={right:$p$}];

\node at \punkt{0}{4} [below right] {$x_0$};
\node at \punkt{90}{4} [above left] {$\omega x_0$};

\fill[color=white] \punkt{60}{4} rectangle \punkt{58}{5.9};

\fill[color=red] \punkt{60}{4} circle[radius=0.08];
\node[color=red] at \punkt{60}{4} [above right]
	{$\begin{pmatrix}x(t)\\p(t)\end{pmatrix}$};

\end{tikzpicture}
\end{document}

