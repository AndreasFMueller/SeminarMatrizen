%
% scherungen.tex -- template for standalon tikz images
%
% (c) 2021 Prof Dr Andreas Müller, OST Ostschweizer Fachhochschule
%
\documentclass[tikz]{standalone}
\usepackage{amsmath}
\usepackage{times}
\usepackage{txfonts}
\usepackage{pgfplots}
\usepackage{csvsimple}
\usetikzlibrary{arrows,intersections,math}
\begin{document}
\def\skala{1}
\begin{tikzpicture}[>=latex,thick,scale=\skala]

\definecolor{blau}{rgb}{0,0.8,1}
\definecolor{blau}{rgb}{0,0.6,0}
\def\s{1.1}

\begin{scope}[xshift=-4.6cm]

	\fill[color=blue!20] (0,0) rectangle (2,2);
	\fill[color=red!40,opacity=0.5] (0,0) -- (2,\s) -- (2,{2+\s}) -- (0,2)
		-- cycle;

	\foreach \x in {-1,...,3}{
		\draw[color=blau]  (\x,-1) -- (\x,3);
		\draw[color=blau]  (-1,\x) -- (3,\x);
	}

	\begin{scope}
		\clip (-1,-1) rectangle (3,3);
		\foreach \x in {-1,...,3}{
			\draw[color=orange] (\x,-1) -- (\x,3);
			\draw[color=orange] (-1,{\x-0.5*\s}) -- (3,{\x+1.5*\s});
		}
	\end{scope}

	\draw[->] (-1.1,0) -- (3.3,0) coordinate[label={$x$}];
	\draw[->] (0,-1.1) -- (0,3.5) coordinate[label={right:$y$}];

	\node[color=blue] at (0,2) [above left] {$1$};
	\node[color=blue] at (2,0) [below right] {$1$};
	\draw[->,color=blue] (0,0) -- (2,0);
	\draw[->,color=blue] (0,0) -- (0,2);

	\draw[->,color=red] (0,0) -- (2,\s);
	\draw[->,color=red] (0,0) -- (0,2);

	\node[color=red] at (2,\s) [below right] {$(1,t)$};

	\node at (0,0) [below right] {$O$};
	\node at (1,-1.1) [below] {$\displaystyle
		\begin{aligned}
		M &= \begin{pmatrix}0&0\\1&0 \end{pmatrix}
		\\
		e^{Mt}
		&=
		\begin{pmatrix}1&0\\t&1 \end{pmatrix}
		\end{aligned}
	$};
\end{scope}

\begin{scope}
	\fill[color=blue!20] (0,0) rectangle (2,2);
	\fill[color=red!40,opacity=0.5] (0,0) -- (2,0) -- ({2+\s},2) -- (\s,2)
		-- cycle;

	\foreach \x in {-1,...,3}{
		\draw[color=blau]  (\x,-1) -- (\x,3);
		\draw[color=blau]  (-1,\x) -- (3,\x);
	}

	\begin{scope}
		\clip (-1,-1) rectangle (3,3);
		\foreach \x in {-1,...,3}{
			\draw[color=orange] (-1,\x) -- (3,\x);
			\draw[color=orange] ({\x-0.5*\s},-1) -- ({\x+1.5*\s},3);
		}
	\end{scope}

	\draw[->] (-1.1,0) -- (3.3,0) coordinate[label={$x$}];
	\draw[->] (0,-1.1) -- (0,3.5) coordinate[label={right:$y$}];

	\node[color=blue] at (0,2) [above left] {$1$};
	\node[color=blue] at (2,0) [below right] {$1$};
	\draw[->,color=blue] (0,0) -- (2,0);
	\draw[->,color=blue] (0,0) -- (0,2);

	\draw[->,color=red] (0,0) -- (2,0);
	\draw[->,color=red] (0,0) -- (\s,2);

	\node[color=red] at (\s,2) [above left] {$(t,1)$};

	\node at (0,0) [below right] {$O$};
	
	\node at (1,-1.1) [below] {$\displaystyle
		\begin{aligned} N &= \begin{pmatrix}0&1\\0&0 \end{pmatrix}
		\\
		e^{Nt}
		&=
		\begin{pmatrix}1&t\\0&1 \end{pmatrix}
		\end{aligned}
	$};
\end{scope}

\begin{scope}[xshift=3.6cm,yshift=0cm]
	\def\punkt#1#2{({1.6005*(#1)+0.4114*(#2)},{-0.2057*(#1)+0.5719*(#2)})}
	\fill[color=blue!20] (0,0) rectangle (2,2);
	\fill[color=red!40,opacity=0.5]
		(0,0) -- \punkt{2}{0} -- \punkt{2}{2} -- \punkt{0}{2} -- cycle;

	\foreach \x in {0,...,4}{
		\draw[color=blau]  (\x,-1) -- (\x,3);
	}
	\foreach \y in {-1,...,3}{
		\draw[color=blau]  (0,\y) -- (4,\y);
	}

	\begin{scope}
		\clip (-0,-1) rectangle (4,3);
		\foreach \x in {-1,...,6}{
			\draw[color=orange] \punkt{\x}{-3} -- \punkt{\x}{6};
			\draw[color=orange] \punkt{-3}{\x} -- \punkt{6}{\x};
		}
	\end{scope}

	\draw[->] (-0.1,0) -- (4.3,0) coordinate[label={$x$}];
	\draw[->] (0,-1.1) -- (0,3.5) coordinate[label={right:$y$}];

	\node[color=blue] at (0,2) [above left] {$1$};
	\node[color=blue] at (2,0) [below right] {$1$};
	\draw[->,color=blue] (0,0) -- (2,0);
	\draw[->,color=blue] (0,0) -- (0,2);

	\draw[->,color=red] (0,0) -- \punkt{2}{0};
	\draw[->,color=red] (0,0) -- \punkt{0}{2};

	\node at (0,0) [below right] {$O$};
	
	\node at (2,-1.1) [below] {$\displaystyle
		\begin{aligned} D &= \begin{pmatrix}0.5&0.4\\-0.2&-0.5 \end{pmatrix}
		\\
		e^{D\cdot 1}
		&=
		\begin{pmatrix}
			   1.6005 & 0.4114\\
			  -0.2057 & 0.5719
		\end{pmatrix}
		\end{aligned}
	$};
\end{scope}

\end{tikzpicture}
\end{document}

