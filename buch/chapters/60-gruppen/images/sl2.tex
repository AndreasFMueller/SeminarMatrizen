%
% sl2.tex -- template for standalon tikz images
%
% (c) 2021 Prof Dr Andreas Müller, OST Ostschweizer Fachhochschule
%
\documentclass[tikz]{standalone}
\usepackage{amsmath}
\usepackage{times}
\usepackage{txfonts}
\usepackage{pgfplots}
\usepackage{csvsimple}
\usetikzlibrary{arrows,intersections,math}
\begin{document}
\def\skala{1}
\begin{tikzpicture}[>=latex,thick,scale=\skala]

\definecolor{darkgreen}{rgb}{0,0.6,0}

\begin{scope}[xshift=-4.5cm]
	\fill[color=blue!20]
		(1.4,0) -- (0,1.4) -- (-1.4,0) -- (0,-1.4) -- cycle;
	\fill[color=red!40,opacity=0.5]
		(1.96,0) -- (0,1) -- (-1.96,0) -- (0,-1) -- cycle;

	\begin{scope}
		\clip (-2.1,-2.1) rectangle (2.3,2.3);
		\draw[color=darkgreen]
			plot[domain=-1:1,samples=100]
				({(1/1.4)*exp(\x)},{(1/1.4)*exp(-\x)});
		\draw[color=darkgreen]
			plot[domain=-1:1,samples=100]
				({(1/1.4)*exp(\x)},{-(1/1.4)*exp(-\x)});
		\draw[color=darkgreen]
			plot[domain=-1:1,samples=100]
				({-(1/1.4)*exp(\x)},{(1/1.4)*exp(-\x)});
		\draw[color=darkgreen]
			plot[domain=-1:1,samples=100]
				({-(1/1.4)*exp(\x)},{-(1/1.4)*exp(-\x)});
	\end{scope}

	\draw[->] (-2.1,0) -- (2.3,0) coordinate[label={$x$}];
	\draw[->] (0,-2.1) -- (0,2.3) coordinate[label={right:$y$}];

	\draw[->,color=blue] (0,0) -- (1.4,0);
	\draw[->,color=blue] (0,0) -- (0,1.4);

	\draw[->,color=red] (0,0) -- (1.96,0);
	\draw[->,color=red] (0,0) -- (0,1);
	\node at (0,-3.2) 
		{$\displaystyle
		\begin{aligned}
		A&=\begin{pmatrix}1&0\\0&-1\end{pmatrix}
		\\
		e^{At}
		&=\begin{pmatrix}e^t&0\\0&e^{-t}\end{pmatrix}
		\end{aligned}
		$};
	
\end{scope}


\begin{scope}
	\fill[color=blue!20]
		(0:1.4) -- (90:1.4) -- (180:1.4) -- (270:1.4) -- cycle;
	\fill[color=red!40,opacity=0.5]
		(33:1.4) -- (123:1.4) -- (213:1.4) -- (303:1.4) -- cycle;

	\draw[color=darkgreen] (0,0) circle[radius=1.4];

	\draw[->] (-2.1,0) -- (2.3,0) coordinate[label={$x$}];
	\draw[->] (0,-2.1) -- (0,2.3) coordinate[label={right:$y$}];

	\draw[->,color=blue] (0,0) -- (1.4,0);
	\draw[->,color=blue] (0,0) -- (0,1.4);

	\draw[->,color=red] (0,0) -- (33:1.4);
	\draw[->,color=red] (0,0) -- (123:1.4);

	\node at (0,-3.2)
		{$\displaystyle
		\begin{aligned}
		B
		&=\begin{pmatrix}0&-1\\1&0 \end{pmatrix}
		\\
		e^{Bt}
		&=
		\begin{pmatrix}
		\cos t&-\sin t\\
		\sin t& \cos t
		\end{pmatrix}
		\end{aligned}$};
\end{scope}


\begin{scope}[xshift=4.5cm]
	\fill[color=blue!20]
		(0:1.4) -- (90:1.4) -- (180:1.4) -- (270:1.4) -- cycle;
	\def\x{0.5}
	\fill[color=red!40,opacity=0.5]
		({1.4*cosh(\x)},{1.4*sinh(\x})
		--
		({1.4*sinh(\x},{1.4*cosh(\x)})
		--
		({-1.4*cosh(\x)},{-1.4*sinh(\x})
		--
		({-1.4*sinh(\x},{-1.4*cosh(\x)})
		-- cycle;

	\begin{scope}
	\clip (-2.1,-2.1) rectangle (2.2,2.2);
	\draw[color=darkgreen]
		plot[domain=-1:1,samples=100] ({1.4*cosh(\x)},{1.4*sinh(\x)});
	\draw[color=darkgreen]
		plot[domain=-1:1,samples=100] ({1.4*sinh(\x)},{1.4*cosh(\x)});
	\draw[color=darkgreen]
		plot[domain=-1:1,samples=100] ({-1.4*cosh(\x)},{1.4*sinh(\x)});
	\draw[color=darkgreen]
		plot[domain=-1:1,samples=100] ({1.4*sinh(\x)},{-1.4*cosh(\x)});
	\end{scope}

	\draw[->] (-2.1,0) -- (2.3,0) coordinate[label={$x$}];
	\draw[->] (0,-2.1) -- (0,2.3) coordinate[label={right:$y$}];

	\draw[->,color=blue] (0,0) -- (1.4,0);
	\draw[->,color=blue] (0,0) -- (0,1.4);

	\draw[->,color=red] (0,0) -- ({1.4*cosh(\x)},{1.4*sinh(\x)});
	\draw[->,color=red] (0,0) -- ({1.4*sinh(\x)},{1.4*cosh(\x)});

	\node at (0,-3.2) {$\displaystyle
		\begin{aligned}
		C&=\begin{pmatrix}0&1\\1&0\end{pmatrix}
		\\
		e^{Ct}
		&=
		\begin{pmatrix}
		\cosh t&\sinh t\\
		\sinh t&\cosh t
		\end{pmatrix}
		\end{aligned}
		$};
\end{scope}

\end{tikzpicture}
\end{document}

