Für die Lie-Algebra $\operatorname{sl}_2(\mathbb{R})$ wurde die Basis
\[
A=\begin{pmatrix} 1&0\\0&-1 \end{pmatrix},
\qquad
N=\begin{pmatrix} 0&1\\0&0\end{pmatrix},
\qquad
N=\begin{pmatrix} 0&0\\1&0\end{pmatrix}
\]
gefunden.
Dies bedeutet, dass die Elemente
der Gruppe $\operatorname{SL}_2(\mathbb{R})$ nahe der Einheitsmatrix
als ein Produkt von Matrizen der Form
\[
D=e^{At}=\begin{pmatrix}e^t&0\\0&e^{-1}\end{pmatrix},
\quad
S=e^{Ns} = \begin{pmatrix}1&s\\0&1\end{pmatrix},
\quad
T=e^{Mt} = \begin{pmatrix}1&0\\t&1\end{pmatrix}
\]
geschrieben werden können.
\begin{teilaufgaben}
\item
Finden Sie zur Drehung $R_\alpha\in\operatorname{SO}(2)$
aus \eqref{buch:lie:eqn:ralphadefinition} eine solche Zerlegung
$R_\alpha=DST$.
\item
Schreiben Sie die Matrix
\[
A=\begin{pmatrix}
\frac12&-\frac{\sqrt{3}}2\\
\frac{\sqrt{3}}2&\frac12
\end{pmatrix}
\]
als Produkt $A=DST$.
\end{teilaufgaben}

\begin{loesung}
\begin{teilaufgaben}
\item
Zunächst schreiben wir etwas einfacher 
\[
D=\begin{pmatrix}c&0\\0&c^{-1}\end{pmatrix}.
\]
Dann multiplizeren wir 
\begin{align*}
DST
&=
\begin{pmatrix}c&0\\0&c^{-1}\end{pmatrix}
\begin{pmatrix}1&s\\0&1\end{pmatrix}
\begin{pmatrix}1&0\\t&1\end{pmatrix}
\\
&=
\begin{pmatrix}c&0\\0&c^{-1}\end{pmatrix}
\begin{pmatrix}1+st&s\\t&1\end{pmatrix}
\\
&=
\begin{pmatrix}
(1+st)c&sc\\
c^{-1}t&c^{-1}
\end{pmatrix}.
\end{align*}
Der Vergleich mit 
\[
R_\alpha
=
\begin{pmatrix}
\cos\alpha&-\sin\alpha\\
\sin\alpha& \cos\alpha
\end{pmatrix}
=
\begin{pmatrix}
(1+st)c&sc\\
c^{-1}t&c^{-1}
\end{pmatrix}
\]
erlaubt jetzt, die Parameter, $c$, $s$ und $t$ abzulesen.
Zunächst folgt aus dem Eintrag rechts unten, dass
\[
c=\frac{1}{\cos\alpha}
\]
sein muss.
Aus dem Eintrag links unten in der Matrix folgt dann
\[
c^{-1}t = t\cos\alpha = \sin\alpha
\quad\Rightarrow\quad
t=\frac{\sin\alpha}{\cos\alpha}=\tan\alpha.
\]
Der Eintrag rechts oben führt schliesslich auf die Gleichung
\[
sc=\frac{s}{\cos\alpha}=-\sin\alpha
\quad\Rightarrow\quad
s=-\sin\alpha\cos\alpha
\]
für $s$.
Damit sind zwar die Parameter bestimmt, es ist aber noch nachzuprüfen,
dass sich damit auch der korrekte Eintrag oben links in der Matrix
ergibt.
Es ist
\[
(1+st)c
=
\frac{1-\sin\alpha\cos\alpha\tan\alpha}{\cos\alpha}
=
\frac{1-\sin^2\alpha}{\cos\alpha}
=
\frac{\cos^2\alpha}{\cos\alpha}=\cos\alpha,
\]
somit ist
\[
c=\frac{1}{\cos\alpha},\; t=\tan\alpha,\; s=-\sin\alpha\cos\alpha=-\frac12\sin2\alpha
\]
tatsächlich die gesuchte Lösung.
\item
Die Matrix $A$ ist die Drehung $A=R_{60^\circ}$, daher können wir nach
a) folgern:
\begin{align*}
c&=\frac{1}{\cos 60^\circ}= 2\\
s&=-\frac12\sin120^\circ =-\frac{\sqrt{3}}4\\
t&=\tan 60^\circ = \sqrt{3}.
\end{align*}
Daher gilt
\[
DST
=
\begin{pmatrix}2&0\\0&\frac12\end{pmatrix}
\begin{pmatrix}1&-\frac{\sqrt{3}}4\\0&1\end{pmatrix}
\begin{pmatrix}1&0\\ \sqrt{3}&1\end{pmatrix}
=
A,
\]
wie man mit einem Computeralgebraprogramm leicht nachprüfen kann.
\qedhere
\end{teilaufgaben}
\end{loesung}
