Die Elemente der Gruppe $G$ der Translationen und Streckungen von
$\mathbb{R}$ kann durch Paare $(\lambda,t)\in\mathbb{R}^+\times\mathbb{R}$
beschrieben werden,
wobei $\lambda$ durch Streckung und $t$ durch Translation wirkt:
\[
(\lambda,t)\colon \mathbb{R}\to\mathbb{R}: x\mapsto \lambda x+t.
\]
Dies ist allerdings noch keine Untergruppe einer Matrizengruppe.
Dazu bettet man $\mathbb{R}$ mit Hilfe der Abbildung
\[
\mathbb{R}\to\mathbb{R}^2 : x\mapsto \begin{pmatrix}x\\1\end{pmatrix}
\]
in $\mathbb{R}^2$ ein.
Die Wirkung von $(\lambda,t)$ ist dann
\[
\begin{pmatrix}(\lambda,t)\cdot x\\1\end{pmatrix}
=
\begin{pmatrix} \lambda x + t\\1\end{pmatrix}
=
\begin{pmatrix}\lambda&1\\0&1\end{pmatrix}\begin{pmatrix}x\\1\end{pmatrix}.
\]
Die Wirkung des Paares $(\lambda,t)$ kann also mit Hilfe einer 
$2\times 2$-Matrix beschrieben werden.
Die Abbildung
\[
G\to \operatorname{GL}_2(\mathbb{R})
:
(\lambda,t)
\mapsto
\begin{pmatrix}\lambda&t\\0&1\end{pmatrix}
\]
bettet die Gruppe $G$ in $\operatorname{GL}_2(\mathbb{R})$ ein.
\begin{teilaufgaben}
\item
Berechnen Sie das Produkt $g_1g_2$ zweier Elemente
$g_j=(\lambda_j,t_j)$.
\item 
Bestimmen Sie das inverse Elemente von $(\lambda,t)$ in $G$.
\item
Der sogenannte Kommutator zweier Elemente ist $g_1g_2g_1^{-1}g_2^{-1}$,
berechnen Sie den Kommutator für die Gruppenelemente von a).
\item
Rechnen Sie nach, dass
\[
s\mapsto \begin{pmatrix}e^s&0\\0&1\end{pmatrix}
,\qquad
t\mapsto \begin{pmatrix}1&t\\0&1\end{pmatrix}
\]
Einparameteruntergruppen von $\operatorname{GL}_2(\mathbb{R})$ sind.
\item
Berechnen Sie die Tangentialvektoren $S$ und $T$ dieser beiden
Einparameteruntergruppen.
\item
Berechnen Sie den Kommutator $[S,T]$
\end{teilaufgaben}

\begin{loesung}
\begin{teilaufgaben}
\item
Die beiden Gruppenelemente wirken auf $x$ nach
\[
(\lambda_1,t_1)
(\lambda_2,t_2)
\cdot
x
=
(\lambda_1,t_1)(\lambda_2x+t_2)
=
\lambda_1(\lambda_2x+t_2)+t_1)
=
\lambda_1\lambda_2 x + (\lambda_1t_2+t_1),
\]
also ist $g_1g_2=(\lambda_1\lambda_2,\lambda_1t_2+t_1)$.
\item
Die Inverse von $(\lambda,t)$ kann erhalten werden, indem man die
Abbildung $x\mapsto y=\lambda x +t$ nach $x$ auflöst:
\[
y=\lambda x+t
\qquad\Rightarrow\qquad
\lambda^{-1}(y-t)
=
\lambda^{-1}y - \lambda^{-1}t.
\]
Daraus liest man ab, dass $(\lambda,t)^{-1}=(\lambda^{-1},-\lambda^{-1}t)$
ist.
\item
Mit Hilfe der Identität $g_1g_2g_1^{-1}g_2^{-1}=g_1g_2(g_2g_1)^{-1}$
kann man den Kommutator leichter berechnen
\begin{align*}
g_1g_2&=(\lambda_1\lambda_2,t_1+\lambda_1t_2)
\\
g_2g_1&= (\lambda_2\lambda_1,t_2+\lambda_2t_1)
\\
(g_2g_1)^{-1}
&=
(\lambda_1^{-1}\lambda_2^{-1},
	-\lambda_2^{-1}\lambda_1^{-1}(t_2+\lambda_2t_1))
\\
g_1g_2g_1^{-1}g_2^{-1}
&=
(\lambda_1\lambda_2,t_1+\lambda_1t_2)
(\lambda_1^{-1}\lambda_2^{-1},
	-\lambda_2^{-1}\lambda_1^{-1}(t_2+\lambda_2t_1))
\\
&=(1,t_1+\lambda_1t_2 + \lambda_1\lambda_2(
	-\lambda_2^{-1}\lambda_1^{-1}(t_2+\lambda_2t_1))
)
\\
&=(1, t_1+\lambda_1t_2 - t_2 -\lambda_2t_1)
=
(1,(1-\lambda_2)(t_1-t_2))
\end{align*}
Der Kommutator ist also das neutrale Element, wenn $\lambda_2=1$ ist.
\item
Dies ist am einfachsten in der Matrixform nachzurechnen:
\begin{align*}
\begin{pmatrix} e^{s_1}&0\\0&1\end{pmatrix}
\begin{pmatrix} e^{s_2}&0\\0&1\end{pmatrix}
&=
\begin{pmatrix}e^{s_1+s_2}&0\\0&1\end{pmatrix}
&
\begin{pmatrix} 1&t_1\\0&1\end{pmatrix}
\begin{pmatrix} 1&t_2\\0&1\end{pmatrix}
&=
\begin{pmatrix} 1&t_1+t_2\\0&1\end{pmatrix}
\end{align*}
\item
Die Tangentialvektoren werden erhalten durch ableiten der
Matrixdarstellung nach dem Parameter
\begin{align*}
S
&=
\frac{d}{ds} \begin{pmatrix}e^s&0\\0&1\end{pmatrix}\bigg|_{s=0}
=
\begin{pmatrix}1&0\\0&0\end{pmatrix}
\\
T
&=
\frac{d}{dt} \begin{pmatrix}1&t\\0&1\end{pmatrix}\bigg|_{t=0}
=
\begin{pmatrix}0&1\\0&0\end{pmatrix}
\end{align*}
\item Der Kommutator ist
\[
[S,T]
=
\begin{pmatrix}1&0\\0&0\end{pmatrix}
\begin{pmatrix}0&1\\0&0\end{pmatrix}
-
\begin{pmatrix}0&1\\0&0\end{pmatrix}
\begin{pmatrix}1&0\\0&0\end{pmatrix}
=
\begin{pmatrix}0&1\\0&0\end{pmatrix}
-
\begin{pmatrix}0&0\\0&0\end{pmatrix}
=
T.
\qedhere
\]
\end{teilaufgaben}
\end{loesung}

