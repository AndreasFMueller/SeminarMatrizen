Eine Drehung eines Vektors $\vec{x}$ der Ebene $\mathbb{R}^2$
um den Winkel $\alpha$ gefolgt von einer Translation um $\vec{t}$
ist gegeben durch $D_\alpha\vec{x}+\vec{t}$.
Darauf lässt sich jedoch die Theorie der Matrizengruppen nicht
darauf anwenden, weil die Operation nicht die Form einer Matrixmultiplikation
schreiben.
Die Drehung und Translation kann in eine Matrix zusammengefasst werden,
indem zunächst die Ebene mit
\[
\mathbb{R}^2\to\mathbb{R}^3
:
\begin{pmatrix}x\\y\end{pmatrix}
\mapsto
\begin{pmatrix}x\\y\\1\end{pmatrix}
\qquad\text{oder in Vektorschreibweise }\qquad
\vec{x}\mapsto\begin{pmatrix}\vec{x}\\1\end{pmatrix}
\]
in den dreidimensionalen Raum eingebettet wird.
Die Drehung und Verschiebung kann damit in der Form
\[
\begin{pmatrix}D_\alpha\vec{x}+\vec{t}\\1
\end{pmatrix}
=
\begin{pmatrix}D_\alpha&\vec{t}\\0&1\end{pmatrix}
\begin{pmatrix}\vec{x}\\1\end{pmatrix}
\]
als Matrizenoperation geschrieben werden.
Die Gruppe der Drehungen und Verschiebungen der Ebene ist daher
die Gruppe
\[
G
=
\left\{
\left.
A
=
\begin{pmatrix}
D_\alpha&\vec{t}\\
0&1
\end{pmatrix}
=
\begin{pmatrix}
\cos\alpha & -\sin\alpha & t_x \\
\sin\alpha &  \cos\alpha & t_y \\
     0     &       0     &  1
\end{pmatrix}
\;
\right|
\;
\alpha\in\mathbb{R},\vec{t}\in\mathbb{R}^2
\right\}.
\]
Wir kürzen die Elemente von $G$ auch als $(\alpha,\vec{t})$ ab.
\begin{teilaufgaben}
\item
Verifizieren Sie, dass das Produkt zweier solcher Matrizen 
$(\alpha_1,\vec{t}_1)$ und $(\alpha_2,\vec{t}_2)$
wieder die selbe Form $(\alpha,\vec{t})$ hat und berechnen Sie
$\alpha$ und $\vec{t}_j$.
\item
Bestimmen Sie das inverse Element zu $(\alpha,\vec{t}) \in G$.
\item
Die Elemente der Gruppe $G$ sind parametrisiert durch den Winkel $\alpha$
und die Translationskomponenten $t_x$ und $t_y$.
Rechnen Sie nach, dass
\[
\alpha\mapsto \begin{pmatrix} D_{\alpha}&0\\0&1\end{pmatrix},
\quad
t_x\mapsto
\begin{pmatrix} I&\begin{pmatrix}t_x\\0\end{pmatrix}\\0&1\end{pmatrix},
\qquad
t_y\mapsto
\begin{pmatrix} I&\begin{pmatrix}0\\t_y\end{pmatrix}\\0&1\end{pmatrix}
\]
Einparameteruntergruppen von $G$ sind.
\item
Berechnen Sie die Tangentialvektoren $D$, $X$ und $Y$,
die zu den Einparameteruntergruppen von c) gehören.
\item
Berechnen Sie die Lie-Klammer für alle Paare von Tangentialvektoren.
\end{teilaufgaben}

\begin{loesung}
\begin{teilaufgaben}
\item
Die Wirkung beider Gruppenelemente auf dem Vektor $\vec{x}$ ist
\begin{align*}
\begin{pmatrix}D_{\alpha_1}&\vec{t}_1\\0&1\end{pmatrix}
\begin{pmatrix}D_{\alpha_2}&\vec{t}_2\\0&1\end{pmatrix}
\begin{pmatrix}\vec{x}\\1\end{pmatrix}
&=
\begin{pmatrix}D_{\alpha_1}&\vec{t}_1\\0&1\end{pmatrix}
\begin{pmatrix}D_{\alpha_2}\vec{x}+\vec{t}_2\\1\end{pmatrix}
=
\begin{pmatrix}
D_{\alpha_1}(D_{\alpha_2}\vec{x}+\vec{t}_2)+\vec{t}_1\\1
\end{pmatrix}
\\
&=
\begin{pmatrix}
D_{\alpha_1}D_{\alpha_2}\vec{x} + D_{\alpha_1}\vec{t}_2+\vec{t}_1\\1
\end{pmatrix}
=
\begin{pmatrix}
D_{\alpha_1+\alpha_2}&D_{\alpha_1}\vec{t}_2+\vec{t}_1\\
0&1
\end{pmatrix}
\begin{pmatrix}\vec{x}\\1\end{pmatrix}.
\end{align*}
Das Produkt in der Gruppe $G$ kann daher
\[
(\alpha_1,\vec{t}_1) (\alpha_2,\vec{t}_2)
=
(\alpha_1+\alpha_2,\vec{t}_1+D_{\alpha_1}\vec{t}_2)
\]
geschrieben werden.
\item
Die Inverse der Abbildung $\vec{x}\mapsto \vec{y}=D_\alpha\vec{x}+\vec{t}$
kann gefunden werden, indem man auf der rechten Seite nach $\vec{x}$
auflöst:
\begin{align*}
\vec{y}&=D_\alpha\vec{x}+\vec{t}
&&\Rightarrow&
D_{\alpha}^{-1}( \vec{y}-\vec{t}) &= \vec{x}
\\
&&&& \vec{x} &= D_{-\alpha}\vec{y} + (-D_{-\alpha}\vec{t})
\end{align*}
Die Inverse von $(\alpha,\vec{t})$ ist also $(-\alpha,-D_{-\alpha}\vec{t})$.
\item
Da $D_\alpha$ eine Einparameteruntergruppe von $\operatorname{SO}(2)$ ist,
ist $\alpha\mapsto (D_\alpha,0)$ ebenfalls eine Einparameteruntergruppe.
Für die beiden anderen gilt
\[
\biggl(I,\begin{pmatrix}t_{x1}\\0\end{pmatrix}\biggr)
\biggl(I,\begin{pmatrix}t_{x2}\\0\end{pmatrix}\biggr)
=
\biggl(I,\begin{pmatrix}t_{x1}+t_{x2}\\0\end{pmatrix}\biggr)
\quad\text{und}\quad
\biggl(I,\begin{pmatrix}0\\t_{y1}\end{pmatrix}\biggr)
\biggl(I,\begin{pmatrix}0\\t_{y2}\end{pmatrix}\biggr)
=
\biggl(I,\begin{pmatrix}0\\t_{y1}+t_{y2}\end{pmatrix}\biggr),
\]
also sind dies auch Einparameteruntergruppen.
\item
Die Ableitungen sind
\begin{align*}
D
&=
\frac{d}{d\alpha}\begin{pmatrix}D_\alpha&0\\0&1\end{pmatrix}\bigg|_{\alpha=0}
=
\begin{pmatrix}J&0\\0&0\end{pmatrix}
=
\begin{pmatrix}
0&-1&0\\
1& 0&0\\
0& 0&0
\end{pmatrix}
\\
X
&=
\frac{d}{dt_x}
\left.
\begin{pmatrix}I&\begin{pmatrix}t_x\\0\end{pmatrix}\\0&1\end{pmatrix}
\right|_{t_x=0}
=
\begin{pmatrix}
0&0&1\\
0&0&0\\
0&0&0
\end{pmatrix}
&
Y
&=
\frac{d}{dt_y}
\left.
\begin{pmatrix}I&\begin{pmatrix}0\\t_y\end{pmatrix}\\0&1\end{pmatrix}
\right|_{t_y=0}
=
\begin{pmatrix}
0&0&0\\
0&0&1\\
0&0&0
\end{pmatrix}.
\end{align*}
\item
Die Vertauschungsrelationen sind
\begin{align*}
[D,X]
&=
DX-XD
=
\begin{pmatrix}
0&0&0\\
0&0&1\\
0&0&0
\end{pmatrix}
-
\begin{pmatrix}
0&0&0\\
0&0&0\\
0&0&0
\end{pmatrix}
=
Y
\\
[D,Y]
&=
DY-YD
=
\begin{pmatrix}
0&0&-1\\
0&0&0\\
0&0&0
\end{pmatrix}
-
\begin{pmatrix}
0&0&0\\
0&0&0\\
0&0&0
\end{pmatrix}
=
-X
\\
[X,Y]
&=
XY-YX
=
0-0=0.
\qedhere
\end{align*}
\end{teilaufgaben}
\end{loesung}
