%
% lie-algebren.tex -- Lie-Algebren
%
% (c) 2020 Prof Dr Andreas Müller, Hochschule Rapperswil
%
\section{Lie-Algebren
\label{buch:section:lie-algebren}}
\rhead{Lie-Algebren}
Im vorangegangenen Abschnitt wurde gezeigt, dass alle beschriebenen
Matrizengruppen als Untermannigfaltigkeiten im $n^2$-dimensionalen
Vektorraum $M_n(\mathbb{R}9$ betrachtet werden können.
Die Gruppen haben damit nicht nur die algebraische Struktur einer
Matrixgruppe, sie haben auch die geometrische Struktur einer 
Mannigfaltigkeit.
Insbesondere ist es sinnvoll, von Ableitungen zu sprechen.

Eindimensionale Untergruppen einer Gruppe können auch als Kurven
innerhalb der Gruppe angesehen werden.
In diesem Abschnitt soll gezeigt werden, wie man zu jeder eindimensionalen
Untergruppe einen Vektor in $M_n(\mathbb{R})$ finden kann derart, dass
der Vektor als Tangentialvektor an diese Kurve gelten kann.
Aus einer Abbildung zwischen der Gruppe und diesen Tagentialvektoren
erhält man dann auch eine algebraische Struktur auf diesen Tangentialvektoren,
die sogenannte Lie-Algebra.
Sie ist charakteristisch für die Gruppe.
Insbesondere werden wir sehen, wie die Gruppen $\operatorname{SO}(3)$ 
und $\operatorname{SU}(2)$ die gleich Lie-Algebra haben und dass die
Lie-Algebra von $\operatorname{SO}(3)$ mit dem Vektorprodukt in $\mathbb{R}^3$
übereinstimmt.

%
% Tangentialvektoren und SO(2)
%
\subsection{Tangentialvektoren und $\operatorname{SO}(2)$}
Die Drehungen in der Ebene können reell als Matrizen der Form
\[
D_{\alpha}
=
\begin{pmatrix}
\cos\alpha&-\sin\alpha\\
\sin\alpha& \cos\alpha
\end{pmatrix}
\]
als eidimensionale Kurve innerhalb von $M_2(\mathbb{R})$ beschrieben
werden.
Alternativ können Drehungen um den Winkel $\alpha$ als mit Hilfe von
der Abbildung
$
\alpha\mapsto e^{i\alpha}
$
als komplexe Zahlen vom Betrag $1$ beschrieben werden.
Dies sind zwei verschiedene Parametrisierungen der gleichen
geometrischen Transformation.

Die Ableitung nach $\alpha$ ist $ie^{i\alpha}$, der Tangentialvektor
im Punkt $e^{i\alpha}$ ist also $ie^{i\alpha}$.
Die Multiplikation mit $i$ ist die Drehung um $90^\circ$, der Tangentialvektor
ist also der um $90^\circ$ gedrehte Ortsvektor zum Punkt auf der Kurve.

In der Darstelllung als $2\times 2$-Matrix ist die Ableitung
\[
\frac{d}{d\alpha}D_\alpha
=
\frac{d}{d\alpha}
\begin{pmatrix}
\cos\alpha& -\sin\alpha\\
\sin\alpha&  \cos\alpha
\end{pmatrix}
=
\begin{pmatrix}
-\sin\alpha & -\cos\alpha \\
 \cos\alpha & -\sin\alpha
\end{pmatrix}.
\]
Die rechte Seite kann wieder mit der Drehmatrix $D_\alpha$ geschrieben
werden, es ist nämlich
\[
\frac{d}{d\alpha}D_\alpha
=
\begin{pmatrix}
-\sin\alpha & -\cos\alpha \\
 \cos\alpha & -\sin\alpha
\end{pmatrix}
=
\begin{pmatrix}
\cos\alpha & -\sin\alpha\\
\sin\alpha &  \cos\alpha
\end{pmatrix}
\begin{pmatrix}
0&-1\\
1& 0
\end{pmatrix}
=
D_\alpha J.
\]
Der Tangentialvektor an die Kurve $\alpha\mapsto D_\alpha$ innerhalb
$M_2(\mathbb{R})$ im Punkt $D_\alpha$ ist also die Matrix 
$JD_\alpha$.
Die Matrix $J$ ist die Drehung um $90^\circ$, denn $J=D_{\frac{\pi}2}$.
Der Zusammenhang zwischen dem Punkt $D_\alpha$ und dem Tangentialvektor
ist also analog zur Beschreibug mit komplexen Zahlen.

Im Komplexen vermittelt die Exponentialfunktion den Zusammenhang zwischen
dem Winkel $\alpha$ und dre Drehung $e^{i\alpha}$.
Der Grund dafür ist natürlich die Differentialgleichung
\[
\frac{d}{d\alpha} z(\alpha) =  iz(\alpha).
\]
Die analoge Differentialgleichung 
\[
\frac{d}{d\alpha} D_\alpha = J D_\alpha
\]
führt auf die Matrix-Exponentialreihe
\begin{align*}
D_\alpha
=
\exp (J\alpha)
&=
\sum_{k=0}^\infty \frac{(J\alpha)^k}{k!}
=
I\biggl(
1-\frac{\alpha^2}{2!} + \frac{\alpha^4}{4!} -\frac{\alpha^6}{6!}+\dots
\biggr)
+
J\biggl(
\alpha - \frac{\alpha^3}{3!}
+ \frac{\alpha^5}{5!}
- \frac{\alpha^7}{7!}+\dots
\biggr)
\\
&=
I\cos\alpha
+
J\sin\alpha,
\end{align*}
welche der Eulerschen Formel $e^{i\alpha} = \cos\alpha + i \sin\alpha$
analog ist.

In diesem Beispiel gibt es nur eine Tangentialrichtung und alle in Frage
kommenden Matrizen vertauschen miteinander.
Es ist daher nicht damit zu rechnen, dass sich eine interessante 
Algebrastruktur für die Ableitungen konstruieren lässt.

%
% Die Lie-Algebra einer Matrizengruppe
%
\subsection{Lie-Algebra einer Matrizengruppe}
Das eindimensionale Beispiel $\operatorname{SO}(2)$ hat gezeigt, dass
die Tangentialvektoren in einem beliebigen Punkt $D_\alpha$ aus dem
Tangentialvektor im Punkt $I$ durch Anwendung der Drehung hervorgehen,
die $I$ in $D_\alpha$ abbildet.
Die Drehungen einer eindimensionalen Untergruppe transportieren daher
den Tangentialvektor in $I$ entlang der Kurve auf jeden beliebigen
anderen Punkt.
Zu jedem Tangentialvektor im Punkt $I$ dürfte es daher genau eine
eindimensionale Untergruppe geben.

Sei die Abbildung $\varrho\colon\mathbb{R}\to G$ eine Einparameter-Untergruppe
von $G\subset M_n(\mathbb{R})$.
Durch Ableitung der Gleichung $\varrho(t+x) = \varrho(t)\varrho(x)$ nach
$x$ folgt die Differentialgleichung
\[
\varrho'(t)
=
\frac{d}{dx}\varrho(t+x)\bigg|_{x=0}
=
\varrho(t) \frac{d}{dx}\varrho(0)\bigg|_{x=0}
=
\varrho(t) \varrho'(0).
\]
Der Tangentialvektor in $\varrho'(t)$ in $\varrho(t)$ ist daher
der Tangentialvektor $\varrho'(0)$ in $I$ transportiert in den Punkt
$\varrho(t)$ mit Hilfe der Matrix $\varrho(t)$.

Aus der Differentialgleichung folgt auch, dass
\[
\varrho(t) = \exp (t\varrho'(0)).
\]
Zu einem Tangentialvektor in $I$ kann man also immer die
Einparameter-Untergruppe mit Hilfe der Differentialgleichung 
oder der expliziten Exponentialreihe rekonstruieren.

Die eindimensionale Gruppe $\operatorname{SO}(2)$ ist abelsch und
hat einen eindimensionalen Tangentialraum, man kann also nicht mit
einer interessanten Algebrastruktur rechnen.
Für eine höherdimensionale, nichtabelsche Gruppe sollte sich aus
der Tatsache, dass es verschiedene eindimensionale Untergruppen gibt,
deren Elemente nicht mit den Elemente einer anderen solchen Gruppe
vertauschen, eine interessante Algebra konstruieren lassen, deren
Struktur die Nichtvertauschbarkeit wiederspiegelt.

Seien also $A$ und $B$ Tangentialvektoren einer Matrizengruppe $G$,
die zu den Einparameter-Untergruppen $\varphi(t)=\exp At$ und
$\varrho(t)=\exp Bt$ gehören.
Insbesondere gilt $\varphi'(0)=A$ und $\varrho'(0)=B$.
Das Produkt $\pi(t)=\varphi(t)\varrho(t)$ ist allerdings nicht notwendigerweise
eine Einparametergruppe, denn dazu müsste gelten
\begin{align*}
\pi(t+s)
&=
\varphi(t+s)\varrho(t+s)
=
\varphi(t)\varphi(s)\varrho(t)\varrho(s)
\\
=
\pi(t)\pi(s)
&=
\varphi(t)\varrho(t)\varphi(s)\varrho(s)
\end{align*}
Durch Multiplikation von links mit $\varphi(t)^{-1}$ und
mit $\varrho(s)^{-1}$ von rechts folgt, dass dies genau dann gilt,
wenn
\[
\varphi(s)\varrho(t)=\varrho(t)\varphi(s).
\]
Die beiden Seiten dieser Gleichung sind erneut verschiedene Punkte
in $G$.
Durch Multiplikation mit $\varrho(t)^{-1}$ von links und mit
$\varphi(s)^{-1}$ von rechts erhält man die äquivaliente
Bedingung
\begin{equation}
\varrho(-t)\varphi(s)\varrho(t)\varphi(-s)=I.
\label{buch:lie:konjugation}
\end{equation}
Ist die Gruppe $G$ nicht kommutativ, kann man nicht
annehmen, dass diese Bedingung erfüllt ist.

Aus \eqref{buch:lie:konjugation} erhält man jetzt eine Kurve
\[
t \mapsto \gamma(t,s) = \varrho(-t)\varphi(s)\varrho(t)\varphi(-s) \in G
\]
in der Gruppe, die für $t=0$ durch $I$ geht.
Ihren Tangentialvektor kann man durch Ableitung bekommen:
\begin{align*}
\frac{d}{dt}\gamma(t,s)
&=
-\varrho'(-t)\varphi(s)\varrho(t)\varphi(-s)
+\varrho(-t)\varphi(s)\varrho'(t)\varphi(-t)
\\
\frac{d}{dt}\gamma(t)\bigg|_{t=0}
&=
-B\varphi(s) + \varphi(-s)B
\end{align*}
Durch erneute Ableitung nach $s$  erhält man dann 
\begin{align*}
\frac{d}{ds} \frac{d}{dt}\gamma(t,s)\bigg|_{t=0}
&=
-B\varphi'(s) - \varphi(-s)B
\end{align*}

%
% Die Lie-Algebra von SO(3)
%
\subsection{Die Lie-Algebra von $\operatorname{SO}(3)$}

%
% Die Lie-Algebra von SU(2)
%
\subsection{Die Lie-Algebra von $\operatorname{SU}(2)$}




