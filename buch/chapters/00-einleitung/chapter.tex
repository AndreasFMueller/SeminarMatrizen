%
% einleitung.tex
%
% (c) 2020 Prof Dr Andreas Müller
%
\chapter*{Einleitung\label{chapter:einleitung}}
\lhead{Einleitung}
\rhead{}
\addcontentsline{toc}{chapter}{Einleitung}
Die Mathematik befasst sich neben dem Rechnen mit Zahlen, der Arithmetik,
mit einer Vielzahl von Abstraktionen, die oft überhaupt nichts mit 
Zahlen zu tun haben.
Die Geometrie studiert zum Beispiel Objekte wie Punkte, Geraden, Kreise
und deren Beziehungen untereinander, die man definieren kann ganz ohne
das Wissen, was eine Zahl ist.
Apollonius von Perga (262--190 BCE) hat in seinem Buch über Kegelschnitte
als erster einen algebraischen Zusammenhang zwischen Zahlen festgestellt,
die man also die Vorläufer heutiger Koordinaten eines Punktes ansehen könnte.
Erst im 16.~Jahrhundert entwickelte sich die Algebra allerdings weit genug,
dass eine Algebraisierung der Geometrie möglich wurde.
Pierre de Fermat
\index{Fermat, Pierre de}%
und René Descartes
\index{Descartes, René}%
schufen die sogenannte {\em analytische Geometrie}. 
Das rechtwinklige Koordinatensystem, nach Descartes auch karteisches
Koordinatensystem genannt, beschreibt Punkte als Zahlenpaare $(x,y)$
und Kurven in der Ebene durch ihre Gleichungen.
Geraden können als Graphen der Funktion $f(x) = ax+b$ oder als Lösungsmenge
linearer Gleichungen wie $ax+by=c$ verstanden werden.
Eine Parabel kann als Graph einer quadratischen Funktion $f(x)=ax^2+bx+c$
dargestellt werden.
Die Punkte $(x,y)$ eines Kreises lösen eine Gleichung der Form
\[
(x-x_M)^2 + (y-y_M)^2 = r^2.
\]
Mit dieser einfachen Idee konnte jedes geometrische Problem in der Ebene
in ein algebraisches Problem übersetzt werden und umgekehrt.

Die Algebraisierung macht allerdings auch klar, dass dem Aufbau des
Zahlensystems mehr Beachtung geschenkt werden muss.
Zum Beispiel beschreibt die Gleichung
\[
x^2+(y-1)^2=4
\]
einen Kreis mit Radius $2$ um den Punkt $(0,1)$.
Der Kreis hat natürlich zwei Schnittpunkte mit der $x$-Achse, wie mit jeder
Gerade, deren Abstand vom Mittelpunkt des Kreises kleiner ist als der Radius.
Schnittpunkte haben die Koordinaten $(x_S,0)$ und $x_S$ muss die
Gleichung
\[
x_S^2 + (0-1)^2 = x_S^2+1=4
\qquad\Rightarrow\qquad
x_S^2=3
\]
erfüllen.
Eine solche Lösung ist nicht möglich, wenn man sich auf rationale
Koordinaten $x_S\in\mathbb{Q}$ beschränkt, die Erweiterung auf
reelle Zahlen ist notwendig.

Kapitel~\ref{buch:chapter:zahlen} übernimmt die Aufgabe, die Zahlensysteme
klar zu definieren und ihre wichtigsten Eigenschaften zusammenzutragen.
Sie bilden das Fundament aller folgenden Konstruktionen.

Die reellen Zahlen erweitern die rationalen Zahlen derart, dass damit
zum Beispiel quaddratische Gleichungen gelöst werden können.
Dies ist aber nicht die einzige mögliche Vorgehensweise.
Die Zahl $\alpha=\sqrt{2}$ ist ja nur ein Objekt, mit dem gerechnet werden
kann wie mit jeder anderen Zahl, welche aber die zusätzliche Rechenregel
$\alpha^2=2$ erfüllt.
Die Erweiterung von $\mathbb{R}$ zu den komplexen Zahl verlangt nur,
dass man der Menge $\mathbb{R}$ ein neues algebraisches Objekt $i$
hinzufügt, welches als spezielle Eigenschaft die Gleichung $i^2=-1$ hat.
Bei $\sqrt{2}$ hat die geometrische Anschauung suggeriert, dass es eine
solche Zahl ``zwischen'' den rationalen Zahlen gibt, aber für $i$
gibt es keine solche Anschauung.
Die imaginäre Einheit $i$ erhielt daher auch diesen durchaus
abwertend gemeinten Namen.

Die Zahlensysteme lassen sich also verstehen als einfachere Zahlensysteme,
denen man zusätzliche Objekte mit besonderen algebraischen Eigenschaften
hinzufügt.
Doch was sind das für Objekte?
Gibt es die überhaupt?
Kann man deren Existenz einfach so postulieren, so wie man das mit $i$
gemacht hat?
Und was macht man, wenn man sich den nächsten ``algebraischen Wunsch''
erfüllen will, auch einfach wieder die Existenz des neuen Objektes
postulieren?

Komplexen Zahlen und Matrizen zeigen, wie das gehen könnte.
Indem man vier rationale Zahlen als $2\times 2$-Matrix in der Form
\[
A=
\begin{pmatrix}
a_{11}&a_{12}\\
a_{21}&a_{22}
\end{pmatrix}
\]
gruppiert und die Rechenoperationen
\begin{align*}
A+B
&=
\begin{pmatrix}
a_{11}&a_{12}\\
a_{21}&a_{22}
\end{pmatrix}
+
\begin{pmatrix}
b_{11}&b_{12}\\
b_{21}&b_{22}
\end{pmatrix}
=
\begin{pmatrix}
a_{11}+b_{11}&a_{12}+b_{12}\\
a_{21}+b_{21}&a_{22}+b_{22}
\end{pmatrix}
\\
AB
&=
\begin{pmatrix}
a_{11}&a_{12}\\
a_{21}&a_{22}
\end{pmatrix}
\begin{pmatrix}
b_{11}&b_{12}\\
b_{21}&b_{22}
\end{pmatrix}
=
\begin{pmatrix}
a_{11}b_{11} + a_{12}b_{21} & a_{11}b_{12} + a_{12}b_{22} \\
a_{21}b_{11} + a_{22}b_{21} & a_{21}b_{12} + a_{22}b_{22}
\end{pmatrix}
\end{align*}
definiert, kann man neue Objekte mit zum Teil bekannten, zum Teil
aber auch ungewohnten algebraischen Eigenschaften bekommen.
Die Matrizen der Form
\[
aI
=
\begin{pmatrix} a&0\\0&a \end{pmatrix},
\quad
a\in\mathbb{Q}
\]
zum Beispiel erfüllen alle Regeln für das Rechnen mit rationalen Zahlen.
$\mathbb{Q}$ kann man also als Teilmenge des neuen ``Zahlensystems'' ansehen.
Aber die Matrix
\[
J
=
\begin{pmatrix} 0&-1\\1&0 \end{pmatrix}
\]
hat die Eigenschaft
\[
J^2 = 
\begin{pmatrix} 0&-1\\1&0 \end{pmatrix}
\begin{pmatrix} 0&-1\\1&0 \end{pmatrix}
=
\begin{pmatrix} -1&0\\0&-1\end{pmatrix}
=
-I.
\]
Das neue Objekt $J$ ist ein explizit konstruiertes Objekt, welches
genau die rechnerischen Eigenschaften der imaginären Einheit $i$ hat.

Die imaginäre Einheit ist nicht die einzige Grösse, die sich auf diese
Weise konstruieren lässt.
Zum Beispiel erfüllt die Matrix
\[
W=\begin{pmatrix} 0&2\\1&0 \end{pmatrix}
\qquad\text{die Gleichung}\qquad
W^2 = \begin{pmatrix} 2&0\\0&2\end{pmatrix} = 2I,
\]
die Menge der Matrizen
\[
\mathbb{Q}(\!\sqrt{2})
=
\left\{\left.
\begin{pmatrix} a&2b\\ b&a\end{pmatrix}
\;\right|\;
a,b\in\mathbb{Q}
\right\}
\]
verhält sich daher genau so wie die Menge der rationalen Zahlen, denen
man ein ``imaginäres'' neues Objekt $\!\sqrt{2}$ hinzugefügt hat.

Matrizen sind also ein Werkzeug, mit dem sich ein algebraisches Systeme
mit fast beliebigen Eigenschaften konstruieren lässt.
Dies führt zu einer Explosion der denkbaren algebraischen Strukturen.
Kapitel~\ref{buch:chapter:vektoren-und-matrizen} bringt etwas Ordnung
in diese Vielfalt, indem die grundlegenden Strukturen charakterisiert
und benannt werden.

In den folgenden Kapiteln sollen dann weitere algebraische Konstrukte
studiert und mit Matrizen realisiert werden.
Den Anfang machen in Kapitel~\ref{buch:chapter:polynome} die Polynome.
Polynome beschreiben grundlegende algebraische Eigenschaften eines
einzelnen Objektes, sowohl $\sqrt{2}$ wie auch $i$ sind Lösungen einer
Polynomgleichung.

Eine besondere Rolle spielen in der Mathematik die Symmetrien.
Eine der frühesten Anwendungen dieses Gedankens in der Algebra war
die Überlegung, dass sich die Nullstellen einer Polynomgleichung
permutieren lassen.
Die Idee der Permutationsgruppe taucht auch in algebraischen Konstruktionen
wie der Determinanten auf.
Tatsächlich lassen sich Permutationen auch als Matrizen schreiben
und die Rechenregeln für Determinanten sind ein direktes Abbild
gewisser Eigenschaften von Transpositionen.
Einmal mehr haben Matrizen ermöglicht, ein neues Konzept in einer
bekannten Sprache auszudrücken.

Die Darstellungstheorie ist das Bestreben, nicht nur Permutationen,
sondern beliebige Gruppen von Symmetrien als Mengen von Matrizen
darzustellen.
Die abstrakten Symmetriegruppen erhalten damit immer konkrete 
Realisierungen als Matrizenmengen.
Auch kompliziertere Strukturen wie Ringe, Körper oder Algebren
lassen sich mit Matrizen realisieren.
Aber die Idee ist nicht auf die Geometrie beschränkt, auch analytische
oder kombinatorische Eigenschaften lassen sich in Matrizenstrukturen
abbilden und damit neuen rechnerischen Behandlungen zugänglich
machen.

Das Kapitel~\ref{buch:chapter:homologie} illustriert, wie weit dieser
Plan führen kann.
Die Konstruktion der Homologiegruppen zeigt, wie sich die Eigenschaften
der Gestalt gewisser geometrischer Strukturen zunächst mit Matrizen,
die kombinatorische Eigenschaften beschreiben, ausdrücken lassen.
Anschliessend können daraus wieder algebraische Strukturen gewonnen
werden.
Gestalteigenschaften werden damit der rechnerischen Untersuchung zugänglich.

Die folgenden Kapitel sollen zeigen, wie Matrizen der Schlüssel dafür
sein können, fast jede denkbare rechnerische Struktur zu verstehen und
auch zum Beispiel für die Berechnung mit dem Computer zu realisieren.




