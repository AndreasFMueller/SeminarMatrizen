%
% spielBtilde.tex -- Zustandsdiagramm des modifzierten Spiels
%
% (c) 2021 Prof Dr Andreas Müller, OST Ostschweizer Fachhochschule
%
\documentclass[tikz]{standalone}
\usepackage{amsmath}
\usepackage{times}
\usepackage{txfonts}
\usepackage{pgfplots}
\usepackage{csvsimple}
\usetikzlibrary{arrows,intersections,math}
\begin{document}
\def\skala{1}
\begin{tikzpicture}[>=latex,thick,scale=\skala]

\def\R{2.5}
\def\r{0.5}
\coordinate (A) at (0,\R);
\coordinate (B) at ({\R*sqrt(3)/2},{-0.5*\R});
\coordinate (C) at ({-\R*sqrt(3)/2},{-0.5*\R});

\draw[->,shorten >= 0.5cm,shorten <= 0.5cm] (A) -- (B);
\draw[->,shorten >= 0.5cm,shorten <= 0.5cm] (A) -- (C);
\draw[->,shorten >= 0.5cm,shorten <= 0.5cm] (C) -- (B);

\draw[->,shorten >= 0.5cm,shorten <= 0.5cm] (B) to[out=90,in=-30] (A);
\draw[->,shorten >= 0.5cm,shorten <= 0.5cm] (C) to[out=90,in=-150] (A);
\draw[->,shorten >= 0.5cm,shorten <= 0.5cm] (B) to[out=-150,in=-30] (C);

\pgfmathparse{0.93*\R}
\xdef\Rgross{\pgfmathresult}

\node at (30:\Rgross) {$\frac34-\varepsilon$};
\node at (150:\Rgross) {$\frac14+\varepsilon$};
\node at (-90:\Rgross) {$\frac14+\varepsilon$};

\pgfmathparse{0.32*\R}
\xdef\Rklein{\pgfmathresult}

\node at (-90:\Rklein) {$\frac34-\varepsilon$};
\node at (30:\Rklein) {$\frac9{10}+\varepsilon$};
\node at (150:\Rklein) {$\frac1{10}-\varepsilon$};

\fill[color=white] (A) circle[radius=\r];
\draw (A) circle[radius=\r];
\node at (A) {$0$};

\fill[color=white] (B) circle[radius=\r];
\draw (B) circle[radius=\r];
\node at (B) {$2$};

\fill[color=white] (C) circle[radius=\r];
\draw (C) circle[radius=\r];
\node at (C) {$1$};

\end{tikzpicture}
\end{document}

