%
% markov.tex -- diskrete Markov-Ketten und Übergangsmatrizen
%
% (c) 2020 Prof Dr Andreas Müller, Hochschule Rapperswil
%
\section{Diskrete Markov-Ketten und Wahrscheinlichkeitsmatrizen
\label{buch:section:diskrete-markov-ketten}}
\rhead{Diskrete Markov-Ketten}
Die einführend im Abschnitt~\ref{buch:section:google-matrix}
vorgestellte Google-Matrix ist nur ein Beispiel für ein
Modell eines stochastischen Prozesses, der mit Hilfe von Matrizen
modelliert werden kann.
In diesem Abschnitt soll diese Art von Prozessen etwas formalisiert
werden.

%
% Beschreibung der Markov-Eigenschaft
% 
\subsection{Markov-Eigenschaft}
% XXX Notation, Zustände, Übergangswahrscheinlichkeit
Ein stochastischer Prozess ist eine Familie von Zustandsvariablen
$X_t$ mit Werten in einer Menge $\mathcal{S}$ von Zuständen.
Der Parameter $t$ wird üblicherweise als die Zeit interpretiert,
er kann beliebige reelle Werte oder diskrete Werte annahmen, im letzten
Fall spricht man von einem Prozess mit diskreter Zeit.

Das Ereignis $\{X_t=x\}$ wird gelesen als ``zur Zeit $t$ befindet sich
der Prozess im Zustand $x$''.
Mit $P(X_t = x)$ wir die Wahrscheinlichkeit bezeichnet, dass sich
der Prozess zur Zeit $t$ im Zustand $x$ befindet.
Die Funktion $t\mapsto X_t$ beschreiben also den zeitlichen Ablauf
der vom Prozess durchlaufenen Zustände.
Dies ermöglicht, Fragen nach dem Einfluss früherer Zustände,
also des Eintretens eines Ereignisses $\{X_{t_0}=x\}$ auf das Eintreten eines
Zustands $s\in\mathcal{S}$ zu einem späteren Zeitpunkt $t_1>t_0$
zu studieren.
Das Ereignis $\{X_t = x\}$ kann man sich als abhängig von der Vorgeschichte
vorstellen.
Die Vorgeschichte besteht dabei aus dem Eintreten gewisser Ereignisse
\[
\{X_0=x_0\},
\{X_1=x_1\},
\{X_2=x_2\},
\dots,
\{X_n=x_n\}
\]
zu früheren Zeiten $t_0<t_1<\dots<t_n<t$.
Die bedingte Wahrscheinlichkeit
\begin{equation}
P(X_t = x|
X_{t_n}=x_n\wedge X_{t_{n-1}}=x_{n-1}\wedge\dots\wedge X_{t_1}=x_1\wedge
X_{t_0}=x_0)
\label{buch:wahrscheinlichkeit:eqn:historybedingt}
\end{equation}
ist die Wahrscheinlichkeit dafür, dass der Prozess zur Zeit $t$ den
Zustand $x$ erreicht, wenn er zu den Zeitpunkten $t_0,t_1,\dots,t_n$
die Zustände $x_0,x_1,\dots,x_n$ durchlaufen hat.

\subsubsection{Gedächtnislosigkeit}
% XXX Gedächtnislösigkeit, Markov-Eigenschaft
In vielen Fällen ist nur der letzte durchlaufene Zustand wichtig.
Die Zustände in den Zeitpunkten $t_0<\dots<t_{n-1}$ haben dann keinen
Einfluss auf die Wahrscheinlichkeit.
Auf die bedingte
Wahrscheinlichkeit~\eqref{buch:wahrscheinlichkeit:eqn:historybedingt}
sollten also die Ereignisse $\{X_{t_0}=x_0\}$ bis $\{X_{t_{n-1}}=x_{n-1}\}$
keinen Einfluss haben.

\begin{definition}
Ein stochastischer Prozess erfüllt die Markov-Eigenschaft, wenn 
für jede Folge von früheren Zeitpunkten $t_0<t_1<\dots <t_n<t$ und Zuständen
$x_0,\dots,x_n,x\in \mathcal{S}$ die 
Wahrscheinlichkeit~\eqref{buch:wahrscheinlichkeit:eqn:historybedingt}
nicht von der Vorgeschichte abhängt, also
\[
P(X_t = x|
X_{t_n}=x_n\wedge X_{t_{n-1}}=x_{n-1}\wedge\dots\wedge X_{t_1}=x_1\wedge
X_{t_0}=x_0)
=
P(X_t = x|
X_{t_n}=x_n).
\]
\index{Markov-Eigenschaft}
\end{definition}

Die Wahrscheinlichkeiten $P(X_t=x|X_s=y)$ mit $t>s$ bestimmen das
zeitliche Verhalten der Wahrscheinlichkeiten vollständig.
Wir schreiben daher auch
\[
p_{xy}(t, s)
=
P(X_t = x|X_s=y)
\]
für die sogenannte {\em transiente Übergangswahrscheinlichkeit}.
Für eine endliche Menge von Zuständen, können die transienten
Übergangswahrscheinlichkeiten auch als zeitabhängige 
quadratische Matrix $P(s,t)$ geschrieben werden, deren
Einträge
\[
(P(s,t))_{xy}
=
p_{xy}(t,s)
\]
mit den Zuständen $x,y\in\mathcal{S}$ indiziert sind.

\subsubsection{Die Chapman-Kolmogorov-Gleichung}
% XXX Chapman-Kolmogorov-Gleichung
Man beachte, dass in der Definition der Markov-Eigenschaft
keine Voraussetzungen darüber gemacht werden, wie nahe
am Zeitpunkt $t$ der letzte Zeitpunkt $t_n$ der Vorgeschichte liegt.
Die transienten Übergangswahrscheinlichkeiten $p_{xy}(s,t)$ werden
aber im allgemeinen davon abhängen, wie weit in der Vergangenheit
der Zeitpunkt $s<t$ liegt.
Für eine näheren Zeitpunkt $\tau$ mit $s<\tau <t$ muss es daher
einen Zusammenhang zwischen den transienten Übergangswahrscheinlichkeiten
$p_{xy}(s,\tau)$, $p_{xy}(\tau,t)$ und $p_{xy}(s,t)$ geben.

\begin{satz}[Chapman-Kolmogorov]
Hat der Prozess die Markov-Eigenschaft und ist $s<\tau <t$, dann gilt
\[
p_{xy}(t,s) = \sum_{z\in\mathcal{S}} p_{xz}(t,\tau) p_{zy}(\tau,s),
\]
was in Matrixform auch als
\[
P(t,s) = P(t,\tau)P(\tau,s)
\]
geschrieben werden kann.
\end{satz}

Auch hier spielt es keine Rolle, wie nahe an $t$ der Zwischenzeitpunkt
$\tau$ liegt.
Die Formel von Chapman-Kolmogoroff kann natürlich für zusätzliche
Zwischenpunkte $s<t_1<t_2<\dots< t_n< t$ formuliert werden.
In Matrix-Notation gilt
\[
P(t,s) = P(t,t_n)P(t_n,t_{n-1})\dots P(t_2,t_1)P(t_1,s),
\]
was ausgeschrieben zu
\[
p_{xy}(t,s)
=
\sum_{x_1,\dots,x_n\in\mathcal{S}}
p_{xx_n}(t,t_n)
p_{x_nx_{n-1}}(t_n,t_{n-1})
\dots
p_{x_2x_1}(t_2,t_1)
p_{x_1y}(t_1,s)
\]
wird.
Jeder Summand auf der rechten Seite beschreibt einen Weg des Prozesses
derart, dass er zu den Zwischenzeitpunkten bestimmte 
Zwischenzustände durchläuft.

% XXX Pfadwahrscheinlichkeit
\begin{definition}
Die Wahrscheinlichkeit, dass der stochastische Prozess zwischen Zeitpunkten
$t_0$ und $t_n$ die Zwischenzustände $x_i$ zu Zeiten $t_i$ durchläuft ist
das Produkt
\[
\sum_{x_1,\dots,x_n\in\mathcal{S}}
p_{x_{n+1}x_n}(t_{n+1},t_n)
p_{x_nx_{n-1}}(t_n,t_{n-1})
\dots
p_{x_2x_1}(t_2,t_1)
p_{x_1x_0}(t_1,s)
=
\prod_{i=0}^{n}
p_{x_{i+1}x_i}(t_{i+1}t_i)
\]
heisst die {\em Pfadwahrscheinlichkeit} für genannten Pfad.
\index{Pfadwahrscheinlichkeit}%
\end{definition}

%
% Diskrete Markov-Kette
%
\subsection{Diskrete Markov-Kette}
% XXX Diskrete Zeit, Endliche Zustandsmenge
Die Markov-Eigenschaft besagt, dass man keine Information verliert,
wenn man die Vorgeschichte eines Zeitpunktes ignoriert.
Insbesondere kann man eine Menge von geeigneten diskreten
Zeitpunkten wählen, ohne viel Information über den Prozess zu
verlieren.
Eine {\em diskrete Markov-Kette} ist eine stochastischer Prozess,
für den die Menge der Zeitpunkte $t$ diskret ist.
Es ist üblich, für die Zeitpunkte ganze oder natürliche Zahlen zu
verwenden.

\begin{definition}
Eine diskrete Markov-Kette ist ein stochastischer Prozess
$(X_t)_{t\in\mathbb{N}}$ mit Werten in $\mathcal{S}$, der die
Markov-Eigenschaft
\[
P(X_{n+1}=x_{n+1}|X_n=x_n\wedge\dots X_0=x_0)
=
P(X_{n+1}=x_{n+1}|X_n=x_n)
\]
hat.
\end{definition}

\begin{figure}
\centering
\includegraphics{chapters/80-wahrscheinlichkeit/images/markov.pdf}
\caption{Diskrete Markovkette mit Zuständen $\mathcal{S}=\{1,2,3,\dots,s\}$
und Übergangsmatrizen $T(n+1,n)$.
\label{buch:wahrscheinlichkeit:fig:diskretemarkovkette}}
\end{figure}

Die transienten Übergangswahrscheinlichkeiten zwischen aufeinanderfolgenden
Zeitpunkten stellen jetzt die vollständige Information über die
zeitliche Entwicklung dar
(Abbildung~\ref{buch:wahrscheinlichkeit:fig:diskretemarkovkette}).
Aus der Matrix
\[
T(n+1,n)
=
\begin{pmatrix}
p_{11}(n+1,n) & \dots  & p_{1s}(n+1,n)\\
\vdots        & \ddots & \vdots       \\
p_{11}(n+1,n) & \dots  & p_{1s}(n+1,n)
\end{pmatrix},
\]
auch die $1$-Schritt Übergangswahrscheinlichkeit genannt, kann man jetzt
auch die Matrix der Überganswahrscheinlichkeiten für mehrere Schritte
\[
T(n+m,n)
=
T(n+m,n+m-1)
T(n+m-1,n+m-2)
\dots
T(n+1,n)
\]
mit der Chapman-Komogorov-Formel bestimmen.
Die Markov-Eigenschaft stellt also sicher, dass man nur die 
$1$-Schritt-Übergangswahrscheinlichkeiten kennen muss.

Eine Matrix $T$ kann als Matrix der Übergangswahrscheinlichkeiten
verwendet werden, wenn sie zwei Bedingungen erfüllt:
\begin{enumerate}
\item Die Einträge von $T$ müssen als Wahrscheinlichkeiten interpretiert
werden können, sie müssen also alle zwischen $0$ und $1$ sein:
$0\le t_{ij}\le 1$ für $i,j\in\mathcal{S}$
\item Die Matrix muss alle möglichen Fälle erfassen.
Dazu ist notwendig, dass sich die Wahrscheinlichkeiten aller Übergänge
aus einem Zustand $j$ zu $1$ summieren, also
\[
\sum_{i\in\mathcal{S}} p_{ij} = 1.
\]
Die Summe der Elemente einer Spalte 
\end{enumerate}

\begin{beispiel}
Die Permutationsmatrix einer Permutation $\sigma\in S_n$ 
(Abschnitt~\label{buch:section:permutationsmatrizen})
ist eine Matrix mit Einträgen $0$ und $1$, so dass die erste Bedingung
erfüllt ist.
In jeder Zeile oder Spalte kommt genau eine $1$ vor, so dass auch die
zweite Bedingung erfüllt ist.
Eine Permutationsmatrix beschreibt einen stochastischen Prozess, dessen
Übergänge deterministisch sind.
\end{beispiel}

\subsubsection{Zustandswahrscheinlichkeiten}
% XXX Zustandswahrscheinlichkeit
Die Wahrscheinlichkeit, mit der sich der Prozess zum Zeitpunkt $n$
im Zustand $i\in\mathcal{S}$ befindet, wird
\[
p_i(n)
=
P(X_i=n)
\]
geschrieben, die auch in einem Vektor $p(n)$ zusammengefasst
werden können.
Die Matrix der Übergangswahrscheinlichkeiten erlaubt, die Verteilung
$p(n+1)$ aus der Verteilung $p(n)$ zu berechnen.
Nach dem Satz von der totalen Wahrscheinlichkeit ist nämlich
\[
P(X_{n+1}=x)
=
\sum_{y\in\mathcal{S}} 
P(X_{n+1}=x|X_n=y) P(X_n=y)
\qquad\text{oder}\qquad
p^{(n+1)} = T(n+1,n) p^{(n)}
\]
in Matrixform.
Die Zeitentwicklung kann also durch Multiplikation mit der Übergangsmatrix
berechnet werden.

\subsubsection{Zeitunabhängige Übergangswahrscheinlichkeiten}
% XXX Übergangswahrscheinlichkeit
Besonderes einfach wird die Situation, wenn die Übergangsmatrix $T(n+1,n)$
nicht von der Zeit abhängt.
In diesem Fall ist $T(n+1,n) = T$ für alle $n$.
Eine solche Markov-Kette heisst {\em homogen}.
\index{homogene Markov-Kette}%
Die Mehrschritt-Übergangswahrscheinlichkeiten sind dann gegeben
durch die Matrix-Potenzen $T(n+m,n)=T^m$.
Im Folgenden gehen wir immer von einer homogenen Markov-Kette aus.

\subsubsection{Stationäre Verteilung}
% XXX stationäre Verteilung
Im Beispiel der Google-Matrix erwarten wir intuitiv, dass sich mit
der Zeit eine Verteilung einstellt,  die sich über die Zeit nicht
mehr ändert.
Ein solche Verteilung heisst stationär.

\begin{definition}
Eine Verteilungsvektor $p$ heisst {\em stationär} für die
homogene Markov-Kette mit Übergangsmatrix $T$, wenn $Tp=p$.
\index{stationäre Verteilung}%
\end{definition}

Eine stationäre Verteilung ist offenbar ein Eigenvektor der Matrix
$T$  zum Eigenwert $1$.
Gefunden werden kann er als Lösung des Gleichungssystems $Tp=p$.
Dazu muss die Matrix $T-E$ singulär sein.
Die Summe einer Spalte von $T$ ist aber immer ein, da $E$ in jeder Spalte
genau eine $1$ enthält, ist die Summe der Einträge einer Spalte von
$T-E$ folglich $0$.
Die Summe aller Zeilen von $T-E$ ist also $0$, die Matrix $T-E$ 
ist singulär.
Dies garantiert aber noch nicht, dass alle Einträge in diesem
Eigenvektor auch tatsächlich nichtnegativ sind.
Die Perron-Frobienus-Theorie von
Abschnitt~\ref{buch:section:positive-vektoren-und-matrizen}
beweist, dass sich immer ein Eigenvektor mit nichtnegativen
Einträgen finden lässt.

Es ist aber nicht garantiert, dass eine stationäre Verteilung
auch eindeutig bestimmt ist.
Dieser Fall tritt immer ein, wenn die geometrische Vielfachheit
des Eigenwerts $1$ grösser ist als $1$.
In Abschnitt~\ref{buch:subsection:elementare-eigenschaften}
werden Bedingungen an eine Matrix $T$ untersucht, die garantieren,
dass der Eigenraum zum Eigenvektor $1$ einedeutig bestimmt ist.

\begin{beispiel}
Als Beispiel dafür betrachten wir eine Permutation $\sigma\in S_n$
und die zugehörige Permutationsmatrix $P$,
wie sie in Abschnitt~\label{buch:section:permutationsmatrizen}
beschrieben worden ist.
Wir verwenden die 
Zyklenzerlegung (Abschnitt~\ref{buch:subsection:zyklenzerlegung})
\(
[n] = \{ Z_1, Z_2,\dots \}
\)
der Permutation $\sigma$, ist ist also $\sigma(Z_i) = Z_i$ für alle
Zyklen.

Jede Verteilung $p$, die auf jedem Zyklus konstant ist, ist eine
stationäre Verteilung.
Ist nämlich $i\in Z_k$, dann ist natürlich auch $\sigma(i)\in Z_k$,
und damit ist $p_{\sigma(i)}=p_i$.

Für jede Wahl von nichtnegativen Zahlen $z_i$ für $i=1,\dots,k$
mit der Eigenschaft $z_1+\dots+z_k=1$ kann man eine stationäre
Verteilung $p(z)$ konstruieren, indem man
\[
p_i(z)
=
\frac{z_i}{|Z_r|}
\qquad\text{wenn}\quad i\in Z_r
\]
setzt.
Die Konstruktion stellt sicher, dass sich die Komponenten zu $1$
summieren.
Wir können aus dem Beispiel auch ableiten, dass die geometrische
Vielfachheit des Eigenvektors $1$ mindestens so gross ist wie die
Anzahl der Zyklen der Permutation $\sigma$.
\end{beispiel}

\subsubsection{Irreduzible Markov-Ketten}
Die Zyklen-Zerlegung einer Permutation bilden voneinander isolierte
Mengen von Zuständen, es gibt keine Möglichkeit eines Übergangs zu
einem anderen Zyklus.
Die Zyklen können daher unabhängig voneinander studiert werden.
Diese Idee kann auf allgemeine Markov-Ketten verallgemeinert werden.

\begin{definition}
Zwei Zustände $i,j\in\mathcal{S}$ kommunizieren, wenn die
Übergangswahrscheinlichkeiten $T_{ij}(n) \ne 0$ und $T_{ij}(n)\ne 0$ sind
für $n$ gross genug.
\end{definition}

Die Zustände, die zu verschiedenen Zyklen einer Permutation gehören,
kommunizieren nicht.
Gerade deshalb waren auch die verschiedenen stationären Verteilungen
möglich.
Eine eindeutige stationäre Verteilung können wir also nur erwarten,
wenn alle Zustände miteinander kommunizieren.

% XXX irreduzible Markov-Ketten
\begin{definition}
Eine homogene Markov-Kette heisst {\em irreduzibel}, alle Zustände miteinander
kommunizieren.
\index{irreduzible Markov-Kette}
\end{definition}

\begin{figure}
\centering
\includegraphics{chapters/80-wahrscheinlichkeit/images/markov2.pdf}
\caption{Diese Markov-Kette zerfällt in verschiedene irreduzible
Markov-Ketten, dere Zustandsmengen nicht miteinander kommunizieren.
Solche Markov-Ketten können unabhängig voneinander studiert werden.
\label{buch:wahrscheinlichkeit:fig:markovzerfall}}
\end{figure}

Die Bedingung der Irreduzibilität ist gleichbedeutend damit,
dass für genügend grosses $n$ alle Matrixelemente von $T^n$ positiv sind.
Solche Matrizen nennt man positiv, 
in Abschnitt~\ref{buch:section:positive-vektoren-und-matrizen}
wird gezeigt, dass positive Matrizen immer eine eindeutige
stationäre Verteilung haben.
In Abbildung~\ref{buch:wahrscheinlichkeit:fig:markovzerfall}
ist eine reduzible Markov-Kette dargestellt, die Zustandsmenge
zerfällt in zwei Teilmengen von Zuständen, die nicht miteinander
kommunizieren.
Ein irreduzible Markov-Kette liegt vor, wenn sich ähnlich wie
in Abbildung~\ref{buch:wahrscheinlichkeit:fig:diskretemarkovkette}
jeder Zustand von jedem anderen aus erreichen lässt.

Wenn sich der Vektorraum $\mathbb{R}^n$ in zwei unter $T$ invariante
Unterräme zerlegen lässt, dann hat nach Wahl von Basen in den Unterräumen
die Matrix $T$ die Form
\[
\left(
\begin{array}{c|c}
T_1&0\\
\hline
0&T_2
\end{array}
\right).
\]
Insbesondere kann man stationäre Verteilungen von $T_1$ und $T_2$ 
unabhängig voneinander suchen.
Ist $p_i$ eine stationäre Verteilung von $T_i$, dann ist
\[
T
\left(
\begin{array}{c}
g_1p_1\\
\hline g_2p_2
\end{array}
\right)
=
\left(
\begin{array}{c}
g_1T_1p_1\\
\hline
g_2T_2p_2
\end{array}
\right)
=
\left(
\begin{array}{c}
g_1p_1\\
\hline
g_2p_2
\end{array}
\right),\qquad
\text{ für $g_i\in\mathbb{R}$.}
\]
Durch Wahl der Gewichte $g_i\ge 0$ mit $g_1+g_2=1$ lassen sich so
die stationären Verteilungen für $T$ aus den stationären Verteilungen
der $T_i$ ermitteln.
Das Problem, die stationären Verteilungen von $T$ zu finden, ist
auf die Untermatrizen $T_i$ reduziert worden.

\subsubsection{Die konvexe Menge der stationären Verteilungen}
\begin{figure}
\centering
\includegraphics{chapters/80-wahrscheinlichkeit/images/konvex.pdf}
\caption{Die Konvexe Kombination von Vektoren $\vec{p}_1,\dots,\vec{p}_n$ ist
eine Summe der Form $\sum_{i=1}^n t_i\vec{p}_i$ wobei die $t_i\ge 0$
sind mit $\sum_{i=1}^nt_i=1$.
Für zwei Punkte bilden die konvexen Kombinationen die Verbindungsstrecke
zwischen den Punkten, für drei Punkte in drei Dimensionen spannen die
konvexen Kombinationen ein Dreieck auf.
\label{buch:wahrscheinlichkeit:fig:konvex}}
\end{figure}
Die stationären Verteilungen
\[
\operatorname{Stat}(T)
=
\{
p\in\mathbb R_+^n\;|\; \text{$Tp=p $ und $\|p\|_1=1$}
\}
\]
bilden was man eine konvexe Menge nennt.
Sind nämlich $p$ und $q$ stationäre Verteilungen, dann gilt zunächst
$Tp=p$ und $Tq=q$.
Wegen der Linearität gilt aber auch $T(tp+(1-t)q)=tTp + (1-t)Tq
=tp+(1-t)q$.
Jede Verteilung auf der ``Verbindungsstrecke'' zwischen den beiden
Verteilungen ist auch wieder stationär.

\begin{definition}
Eine {\em konvexe Kombination} von Vektoren $v_1,\dots,v_k\in\mathbb{R^n}$
ist ein Vektor der Form
\[
v=t_1v_1+\dots + t_kv_k
\qquad\text{mit}\quad
t_i\ge 0\;\text{und}\;
t_1+\dots+t_n = 1.
\]
\index{konvexe Kombination}%
Eine Teilmenge $M\subset \mathbb{R}^n$ heisst konvex, wenn zu
zwei Vektoren $x,y\in M$ auch jede konvexe Kombination von $x$ und $y$
wieder in $M$ ist.
\index{konvex}%
\end{definition}

Die konvexen Kombinationen der Vektoren sind Linearkombination
mit nichtnegativen Koeffizienten. Sie bilden im Allgemeinen
einen $(k-1)$-Simplex in $\mathbb{R}^n$.
Für zwei Punkte $x$ und $y$ bilden die konvexen Kombination
$tx+(1-t)y$ für $t\in[0,1]$ die Verbindungsstrecke der beiden
Vektoren.
Eine Menge ist also konvex, wenn sie mit zwei Punkten immer auch
ihre Verbindungsstrecke enthält
% XXX Bild für Konvexe Menge



% XXX Grenzverteilung
\subsubsection{Grenzverteilung}
Im Beispiel der Google-Matrix wurde ein iterativer Algorithmus
zur Berechnung des Pagerank verwendet.
Es stellt sich daher die Frage, ob diese Methode für andere homogene
Markov-Ketten auch funkioniert.
Man beginnt also mit einer beliebigen Verteilung $p(0)$ und wendet
die Übergangsmatrix $T$ wiederholt an.
Es entsteht somit eine Folge $p(n) = T^np(0)$.

\begin{definition}
Falls die Folge $p(n) = T^np(0)$ konvergiert, heisst der Grenzwert
\[
p(\infty) = \lim_{n\to\infty} p(n)
\]
eine {\em Grenzverteilung} von $T$.
\index{Grenzverteilung}%
\end{definition}

Falls eine Grenzverteilung existiert, dann ist sie eine stationäre
Verteilung.
Für eine stationäre Verteilung $p(0)$ ist die Folge $p(n)$ eine
konstante Folge, sie konvergiert also gegen $p(0)$.
Stationäre Verteilungen sind also automatisch Grenzverteilungen.
Falls der Raum der stationären Verteilungen mehrdimensional sind,
dann ist auch die Grenzverteilung nicht eindeutig bestimmt, selbst
wenn sie existiert.
Aber nicht einmal die Existenz einer Grenzverteilung ist garantiert,
wie das folgende Beispiel zeigt.

\begin{beispiel}
Sei $T$ die Permutationsmatrix der zyklischen Verteilung von drei
Elementen in $S_3$, also die Matrix
\[
T=\begin{pmatrix}
0&0&1\\
1&0&0\\
0&1&0
\end{pmatrix}.
\]
Die konstante Verteilung $\frac13U$ ist offensichtlich eine
stationäre Verteilung.
In Abschnitt~\ref{buch:section:positive-vektoren-und-matrizen}
wird gezeigt, dass es die einzige ist.
Sei jetzt $p(0)$ eine beliebiger Vektor in $\mathbb{R}^3$ mit
nichtnegativen Einträgen, die sich zu $1$ summieren.
Dann bilden die Vektoren $p(n)=T^np(0)$ einen Dreierzyklus
\begin{align*}
p(0)&=p(3)=p(6)=\dots =\begin{pmatrix}p_1(0)\\p_2(0)\\p_3(0)\end{pmatrix},
\\
p(1)&=p(4)=p(7)=\dots =\begin{pmatrix}p_2(0)\\p_3(0)\\p_1(0)\end{pmatrix},
\\
p(2)&=p(5)=p(8)=\dots =\begin{pmatrix}p_3(0)\\p_1(0)\\p_2(0)\end{pmatrix}.
\end{align*}
Die Folge $p(n)$ kann also nur dann konvergieren, wenn die drei
Komponenten gleich sind.
\end{beispiel}

\subsubsection{Erwartungswert und Varianz}
% XXX Erwartungswert und Varianz für eine Grenzverteilung
Wenn sich im Laufe der Zeit eine Grenzverteilung einstellen soll, dann
muss es auch möglich sein, Erwartungswert und Varianz dieser Verteilung
zu berechnen.
Dazu muss jedem Zustand ein Zahlenwert zugeordnet werden.
Sei also
\(
g: \mathcal{S}\to R
\)
eine Funktion, die einem Zustand eine reelle Zahl zuordnet.
Aus der Zufallsvariable $X_n$ des Zustands zur Zeit $n$ wird daraus
die Zufallsvariable $Y_n=g(X_n)$ des Wertes zur Zeit $n$.
Die Abbildung $g$ kann auch als Vektor mit der Komponenten $g_i$ 
für $i\in\mathcal{S}$ betrachtet werden, wir verwenden für diesen
Vektor wieder die Schreibweise $g$.

Für die Verteilung $p(n)$ kann man jetzt auch Erwartungswert und
Varianz berechnen.
Der Erwartungswert ist
\[
E(Y)
=
\sum_{i\in\mathcal{S}} g_i p_i(n)
=
g^t p(n).
\]
Für die Varianz muss $g_i$ durch $g_i^2$ ersetzt werden.
Dies kann am einfachsten mit dem Hadamard-Produkt geschrieben werden:
\begin{align*}
E(Y^2)
&=
\sum_{i\in\mathcal{S}} g_i p_i(n)
=
(g\odot g)^t p(n)
\\
E(Y^k)
&=
(g^{\odot k})^t p(n),
\end{align*}
wobei wir die Hadamard-Potenz $A^{\odot k}$ einer Matrix $A$ rekursiv
durch
\[
A^{\odot 0}=E
\qquad\text{und}\qquad
A^{\odot k} = A\odot A^{\odot (k-1)}
\]
definieren.

\subsubsection{Erwartungswert von Werten auf Übergängen}
% XXX Erwartungswert für Zufallsvariablen, die von den Übergängen abhängen
In Abschnitt~\ref{buch:section:paradoxon-von-parrondo} wird ein Spiel
vorgestellt, in dem der Gewinn davon abhängt, welcher Übergang stattfindet,
nicht welcher Zustand erreicht wird.
Es git daher eine Matrix $G$ von Gewinnen, der Eintrag $g_{ij}$ ist
der Gewinn, der bei einem Übergang von Zustand $j$ in den Zustand $i$
ausgezahlt wird.
Mit dieser Matrix lassen sich jetzt viele verschiedene Fragen beantworten:

\begin{frage}
\label{buch:wahrscheinlichkeit:frage1}
Mit welchem Gewinn kann man in Runde $n$ des Spiels rechnen,
wenn $p(n-1)$ die Verteilung zur Zeit $n-1$ ist?
\end{frage}

Der Erwartungswert ist
\begin{align*}
E(Y)
&=
\sum_{i,j\in\mathcal{S}}
g_{ji} t_{ji} p_i(n-1)
\intertext{oder in Matrixform}
&=
U^t
(G\odot T)
p(n-1).
\end{align*}

\begin{frage}
Mit welchen Gewinnen kann man rechnen, wenn der Prozess sich zu Beginn 
einer Spielrunde im Zustand $i$ befindet?
\end{frage}

Dies ist der Spezialfall der Frage~\ref{buch:wahrscheinlichkeit:frage1}
für die Verteilung $p_j(n-1) = \delta_{ij}$.
Der Erwartungswert ist die Summe der Spalte $j$ der Matrix $G\odot T$.
Man kann das Produkt $U^t(G\odot T)$ also auch als eine Zeilenvektor
von Gewinnerwartungen unter der Vorbedingung $X_{n-1}=j$ betrachten.
\[
\begin{pmatrix}
E(Y|X_{n-1}=1)
&\dots&
E(Y|X_{n-1}=n)
\end{pmatrix}
=
U^t (G\odot T).
\]
Indem man $G$ durch $G^{\odot k}$ ersetzt, kann man beliebige höhere
Momente berechnen.

\subsection{Absorbierende Zustände}
% XXX Definition
Eine Grenzverteilung beschreibt die relative Häufigkeit, mit der
der Prozess in den verschiedenen Zuständen vorbeikommt.
In einem Spiel, in dem der Spieler ruiniert werden kann, gibt es
aus dem Ruin-Zustand keinen Weg zurück.
Der Spieler bleibt in diesem Zustand.

\begin{definition}
Ein Zustand $i$ einer homogenen Markov-Kette mit Übergangsmatrix $T$
heisst {\em absorbierend}, wenn $T_{ii}=1$ ist.
\index{absorbierender Zustand}%
Eine Markov-Kette mit mindestens einem absorbierenden Zustand heisst
{\em absorbierende Markov-Kette}.
\index{absorbierende Markov-Kette}%
Nicht absorbierende Zustände heissen {\em transient}
\index{transienter Zustand}%
\end{definition}

\begin{figure}
\centering
\includegraphics{chapters/80-wahrscheinlichkeit/images/markov3.pdf}
\caption{Markov-Kette mit absorbierenden Zuständen (blau hinterlegt).
Erreicht die Markov-Kette einen absorbierenden Zustand, dann verbleibt
sie für alle zukünftigen Zustände in diesem Zustand.
\label{buch:wahrscheinlichkeit:fig:abs}}
\end{figure}

Eine Markov-Kette kann mehrere absorbierende Zustände haben, wie in
Abbildung~\ref{buch:wahrscheinlichkeit:fig:abs} dargestellt.
Indem man die absorbierenden Zustände zuerst auflistet, bekommt die 
Übergangsmatrix die Form
\[
T=
\left(
\begin{array}{c|c}
E&R\\
\hline
0&Q
\end{array}
\right).
\]
Die Matrix $R$ beschreibt die Wahrscheinlichkeiten, mit denen man
ausgehend von einem transienten Zustand
in einem bestimmten absorbierenden Zustand landet.
Die Matrix $Q$ beschreibt die Übergänge, bevor dies passiert.
Die Potenzen von $T$ sind
\[
T^2
=
\left(
\begin{array}{c|c}
E&R+RQ \\
\hline
0&Q^2
\end{array}
\right),
\quad
T^3
=
\left(
\begin{array}{c|c}
E&R+RQ+RQ^2 \\
\hline
0&Q^3
\end{array}
\right),
\;
\dots,
\;
T^k
=
\left(
\begin{array}{c|c}
E&\displaystyle R\sum_{l=0}^{k-1} Q^l \\
\hline
0&Q^k
\end{array}
\right).
\]
Da man früher oder später in einem absorbierenden Zustand landet,
muss $\lim_{k\to\infty} Q^k=0$ sein.
Die Summe in der rechten oberen Teilmatrix kann man als geometrische
Reihe summieren, man erhält die Matrix
\[
\sum_{l=0}^{k-1} Q^l = (E-Q)^{-1}(E-Q^k),
\]
die für $k\to\infty$ gegen
\[
N
=
\lim_{k\to\infty} \sum_{l=0}^{k-1} Q^l
=
(E-Q)^{-1}
\]
konvergiert.
Die Matrix $N$ heisst die {\em Fundamentalmatrix} der absorbierenden
Markov-Kette.
\index{Fundamental-Matrix}%

\subsubsection{Absorbtionszeit}
% XXX Absorptionszeit
Wie lange dauert es im Mittel, bis der Prozess in einem
Absorptionszustand $i$ stecken bleibt?
Die Fundamentalmatrix $N$ der Markov-Kette beantwortet diese
Frage.
Wenn der Prozess genau im Schritt $k$ zum ersten Mal Zustand $i$
ankommt, dann ist $E(k)$ die mittlere Wartezeit.
Der Prozess verbringt also zunächst $k-1$ Schritte in transienten
Zuständen, bevor er in einen absorbierenden Zustand wechselt.

Wir brauchen die Wahrscheinlichkeit für einen Entwicklung des Zustandes
ausgehend vom Zustand $j$, die nach $k-1$ Schritten im Zustand $l$
landet, von wo er in den absorbierenden Zustand wechselt.
Diese Wahrscheinlichkeit ist
\[
P(X_k = i\wedge X_{k-1} = l \wedge X_0=j)
=
\sum_{i_1,\dots,i_{k-2}}
r_{il} q_{li_{k-2}} q_{i_{k-2}i_{k-3}}\dots q_{i_2i_1} q_{i_1j}
\]
Von den Pfaden, die zur Zeit $k-1$ im Zustand $l$ ankommen gibt es
aber auch einige, die nicht absorbiert werden.
Für die Berechnung der Wartezeit möchten wir nur die Wahrscheinlichkeit
innerhalb der Menge der Pfade, die auch tatsächlich absorbiert werden,
das ist die bedingte Wahrscheinlichkeit
\begin{equation}
\begin{aligned}
P(X_k = i\wedge X_{k-1} = l \wedge X_0=j|X_k=i)
&=
\frac{
P(X_k = i\wedge X_{k-1} = l \wedge X_0=j)
}{
P(X_k=i)
}
\\
&=
\sum_{i_1,\dots,i_{k-2}}
q_{li_{k-2}} q_{i_{k-2}i_{k-3}}\dots q_{i_2i_1} q_{i_1j}.
\end{aligned}
\label{buch:wahrscheinlichkeit:eqn:ankunftswahrscheinlichkeit}
\end{equation}
Auf der rechten Seite steht das Matrixelement $(l,j)$ von $Q^{k-1}$.

% XXX Differenz 

Für die Berechnung der erwarteten Zeit ist müssen wir die
Wahrscheinlichkeit mit $k$ multiplizieren und summieren:
\begin{align}
E(k)
&=
\sum_{k=0}^\infty
k(
q^{(k)}_{lj} 
-
q^{(k-1)}_{lj} 
)
\notag
\\
&=
\dots
+
(k+1)(
q^{(k)}_{lj} 
-
q^{(k+1)}_{lj} 
)
+
k(
q^{(k-1)}_{lj} 
-
q^{(k)}_{lj} 
)
+
\dots
\label{buch:wahrscheinlichkeit:eqn:telescope}
\\
&=
\dots
+
q^{(k-1)}_{lj}
+
\dots
=
\sum_{k} q^{(k)}_{lj}.
\notag
\end{align}
In zwei benachbarten Termen in 
\eqref{buch:wahrscheinlichkeit:eqn:telescope}
heben sich die Summanden $kq^{(k)}_{lj}$ weg, man spricht von
einer teleskopischen Reihe.
Die verbleibenden Terme sind genau die Matrixelemente der Fundamentalmatrix $N$.
Die Fundamentalmatrix enthält also im Eintrag $(l,j)$ die Wartezeit
bis zur Absorption über den Zustand $l$.

\subsubsection{Wartezeit}
% XXX Mittlere Zeit bis zu einem bestimmten Zustand
Die mittlere Wartezeit bis zum Erreichen eines Zustands kann mit der
Theorie zur Berechnung der Absorptionszeit berechnet werden.
Dazu modifiziert man den Prozess dahingehend, dass der Zielzustand
ein absorbierender Zustand wird.
Der Einfachheit halber gehen wir davon aus, dass der Zustand $1$ 
der Zielzustand ist.
Wir ersetzen die Übergangsmatrix $T$ durch die Matrix
\[
\tilde{T}
=
\left(
\begin{array}{c|ccc}
1     &t_{12}&\dots &t_{1n}\\
\hline
0     &t_{22}&\dots &t_{2n}\\
\vdots&\dots &\ddots&\vdots\\
0     &t_{n2}&\dots &t_{nn}
\end{array}\right).
\]
$\tilde{T}$ hat den Zustand $1$ als absorbierenden Zustand.
Die $Q$ und $R$ sind
\[
\tilde{R}
=
\begin{pmatrix}t_{12}&\dots&t_{1n}\end{pmatrix},
\quad
\tilde{Q}
=
\begin{pmatrix}
t_{22}&\dots &t_{2n}\\
\vdots&\ddots&\vdots\\
t_{n2}&\dots &t_{nn}
\end{pmatrix}.
\]
Die Wartezeit bis zum Erreichen des Zustands $i$ ausgehend von einem
Zustand $n$ kann jetzt aus der Absorbtionszeit der Markov-Kette
im Zustand $1$ mit Hilfe der Fundamentalmatrix
\[
\tilde{N} 
=
(E-\tilde{Q})^{-1}
\]
berechnet werden.


