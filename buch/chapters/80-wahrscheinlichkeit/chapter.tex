%
% chapter.tex -- Anwendungen von wahrscheinlichkeitsmatrizen
%
% (c) 2020 Prof Dr Andreas Müller, Hochschule Rapperswil
%
\chapter{Wahrscheinlichkeitsmatrizen
\label{buch:chapter:wahrscheinlichkeit}}
\lhead{Wahrscheinlichkeitsmatrizen}
\rhead{}
Matrizen beschreiben lineare Abbildungen, also einen Prozess, der
jedem Vektor einen neuen Vektor zuordnet.
Es ist daher nicht abwegig zu erwarten, dass sich 
die Zeitentwicklung eines vom Zufall beeinflussten Systems, welches sich
in mehreren verschiedenen Zuständen befinden kann, ebenfalls mit Hilfe
von Matrizen beschreiben lässt.
Eine solche Beschreiben ermöglicht leicht Verteilungen,
Erwartungswerte und stationäre Zustände zu ermitteln.

Im Abschnitt~\ref{buch:section:google-matrix} wird an Hand der Google
Matrix bezeigt, wie ein anschauliches Beispiel in natürlicher Weise
auf eine Matrix führt.
Abschnitt~\ref{buch:section:diskrete-markov-ketten} stellt dann die abstrakte
mathematische Theorie der Markov-Ketten dar und behandelt einige wichtige
Eigenschaften von Wahrscheinlichkeitsmatrizen.
Es stellt sich heraus, dass thermodynamische Quantensysteme sehr gut
mit solchen Matrizen beschrieben werden können, zum Beispiel kann man
einfache Formen von Laser auf diese Art behandeln.
Aus einem solchen System hat Parrondo ein System abgeleitet, welches 
ziemlich unerwartetes Verhalten an den Tag gelegt hat, welches mit
Hilfe von Matrizen leicht zu analysieren ist. 
Dies wird in Abschnitt~\ref{buch:section:paradoxon-von-parrondo}
dargestellt.

%
% google.tex
%
% (c) 2019 Prof Dr Andreas Müller, Hochschule Rapperswil
%
\begin{frame}
\begin{center}
\includegraphics[width=\hsize]{../slides/8/tokyo/transportnetworkgraph.png}
\end{center}
\end{frame}


%
% markov.tex
%
% (c) 2021 Prof Dr Andreas Müller, OST Ostschweizer Fachhochschule
%
\bgroup
\setlength{\abovedisplayskip}{5pt}
\setlength{\belowdisplayskip}{5pt}
\begin{frame}[t]
\frametitle{Markovketten}
\vspace{-20pt}
\begin{columns}[t,onlytextwidth]
\begin{column}{0.48\textwidth}
\begin{center}
\begin{tikzpicture}[>=latex,thick]

\def\r{2.2}

\coordinate (A) at ({\r*cos(0*72)},{\r*sin(0*72)});
\coordinate (B) at ({\r*cos(1*72)},{\r*sin(1*72)});
\coordinate (C) at ({\r*cos(2*72)},{\r*sin(2*72)});
\coordinate (D) at ({\r*cos(3*72)},{\r*sin(3*72)});
\coordinate (E) at ({\r*cos(4*72)},{\r*sin(4*72)});

\draw[->,shorten >= 0.1cm,shorten <= 0.1cm,line width=4pt,color=black!40]
	(A) -- (C);
\draw[color=white,line width=8pt] (B) -- (D);
\draw[->,shorten >= 0.1cm,shorten <= 0.1cm,line width=4pt,color=black!80]
	(B) -- (D);

\draw[->,shorten >= 0.1cm,shorten <= 0.1cm,line width=4pt,color=black!60]
	(A) -- (B);
\draw[->,shorten >= 0.1cm,shorten <= 0.1cm,line width=4pt,color=black!20]
	(B) -- (C);
\draw[->,shorten >= 0.1cm,shorten <= 0.1cm,line width=4pt,color=black]
	(C) -- (D);
\draw[->,shorten >= 0.1cm,shorten <= 0.1cm,line width=4pt,color=black]
	(D) -- (E);
\draw[->,shorten >= 0.1cm,shorten <= 0.1cm,line width=4pt,color=black]
	(E) -- (A);

\fill[color=white] (A) circle[radius=0.2];
\fill[color=white] (B) circle[radius=0.2];
\fill[color=white] (C) circle[radius=0.2];
\fill[color=white] (D) circle[radius=0.2];
\fill[color=white] (E) circle[radius=0.2];

\draw (A) circle[radius=0.2];
\draw (B) circle[radius=0.2];
\draw (C) circle[radius=0.2];
\draw (D) circle[radius=0.2];
\draw (E) circle[radius=0.2];

\node at (A) {$1$};
\node at (B) {$2$};
\node at (C) {$3$};
\node at (D) {$4$};
\node at (E) {$5$};

\node at ($0.5*(A)+0.5*(B)-(0.1,0.1)$) [above right] {$\scriptstyle 0.6$};
\node at ($0.5*(B)+0.5*(C)+(0.05,-0.07)$) [above left] {$\scriptstyle 0.2$};
\node at ($0.5*(C)+0.5*(D)+(0.05,0)$) [left] {$\scriptstyle 1$};
\node at ($0.5*(D)+0.5*(E)$) [below] {$\scriptstyle 1$};
\node at ($0.5*(E)+0.5*(A)+(-0.1,0.1)$) [below right] {$\scriptstyle 1$};
\node at ($0.6*(A)+0.4*(C)$) [above] {$\scriptstyle 0.4$};
\node at ($0.4*(B)+0.6*(D)$) [left] {$\scriptstyle 0.8$};

\end{tikzpicture}
\end{center}
\vspace{-10pt}
\uncover<7->{%
\begin{block}{Verteilung}
\begin{itemize}
\item<8->
Welche stationäre Verteilung auf den Knoten stellt sich ein?
\item<9->
$P(i)=?$
\end{itemize}
\end{block}}
\end{column}
\begin{column}{0.48\textwidth}
\uncover<2->{%
\begin{block}{\strut\mbox{Übergang\only<3->{s-/Wahrscheinlichkeit}smatrix}}
$P_{ij} = P(i | j)$, Wahrscheinlichkeit, in den Zustand $i$ überzugehen,
\begin{align*}
P
&=
\begin{pmatrix}
   &   & & &1\phantom{.0}\\
0.6&   & & & \\
0.4&0.2& & & \\
   &0.8&1\phantom{.0}& & \\
   &   & &1\phantom{.0}& 
\end{pmatrix}
\end{align*}
\end{block}}
\vspace{-10pt}
\uncover<4->{%
\begin{block}{Eigenschaften}
\begin{itemize}
\item<5-> $P_{ij}\ge 0\;\forall i,j$
\item<6-> Spaltensumme:
\(
\displaystyle
\sum_{i=1}^n P_{ij} = 1\;\forall j
\)
\end{itemize}
\end{block}}
\end{column}
\end{columns}
\end{frame}

%
% positiv.tex
%
% (c) 2021 Prof Dr Andreas Müller, OST Ostschweizer Fachhochschule
%
\section{Positive Vektoren und Matrizen
\label{buch:section:positive-vektoren-und-matrizen}}
\rhead{Positive Vektoren und Matrizen}
Die Google-Matrix und die Matrizen, die wir in Markov-Ketten angetroffen
haben, zeichnen sich dadurch aus, dass alle ihre Einträge positiv oder
mindestens nicht negativ sind.
Die Perron-Frobenius-Theorie, die in diesem Abschnitt entwickelt
werden soll, zeigt, dass Positivität einer Matrix nützliche
Konsequenzen für Eigenwerte und Eigenvektoren hat.
Das wichtigste Resultat ist die Tatsache, dass postive Matrizen immer
einen einzigen einfachen Eigenwert mit Betrag $\varrho(A)$ haben,
was zum Beispiel die Konvergenz des Pagerank-Algorithmus garantiert.
Dies wird im Satz von Perron-Frobenius in
Abschnitt~\ref{buch:subsection:der-satz-von-perron-frobenius}
erklärt.

%
% Elementare Definitionen und Eigenschaften 
%
\subsection{Elementare Eigenschaften
\label{buch:subsection:elementare-eigenschaften}}
In diesem Abschnitt betrachten wir ausschliesslich reelle Vektoren
und Matrizen.
Die Komponenten sind somit immer mit miteinander vergleichbar, daraus
lässt sich auch eine Vergleichsrelation zwischen Vektoren
ableiten.

\begin{definition}
Ein Vektor $v\in\mathbb{R}^n$ heisst {\em positiv}, geschrieben
$v>0$, wenn alle seine Komponenten positiv sind: $v_i>0\forall i$.
Ein Vektor $v\in\mathbb{R}^n$ heisst {\em nichtnegativ}, in Formeln
$v\ge 0$, wenn alle
seine Komponenten nicht negativ sind: $v_i\ge 0\forall i$.
\index{positiver Vektor}%
\index{nichtnegativer Vektor}%
\end{definition}

Geometrisch kann man sich die Menge der positven Vektoren in zwei Dimensionen
als die Punkte des ersten Quadranten oder in drei Dimensionen als die
Vektoren im ersten Oktanten vorstellen.

Aus der Positivität eines Vektors lässt sich jetzt eine Vergleichsrelation
für beliebige Vektoren ableiten.
Mit der folgenden Definition wird erreicht, das mit Ungleichungen für Vektoren
auf die gleiche Art und Weise gerechnet werden kann, wie man sich
dies von Ungleichungen zwischen Zahlen gewohnt ist.

\begin{definition}
Für zwei Vektoren $u,v\in\mathbb{R}^n$ ist genau dann $u>v$, wenn
$u-v > 0$ ist.
Ebenso ist $u\ge v$ genau dann, wenn $u-v\ge 0$.
\end{definition}

Ungleichungen zwischen Vektoren kann man daher auch so interpretieren,
dass sie für jede Komponente einzeln gelten müssen.
Die Definition funktionieren analog auch für Matrizen:

\begin{definition}
Eine Matrix $A\in M_{m\times n}(\mathbb{R})$  heisst {\em positiv},
wenn alle ihre Einträge $a_{ij}$ positiv sind: $a_{ij}>0\forall i,j$.
Eine Matrix $A\in M_{m\times n}(\mathbb{R})$  heisst {\em nichtnegativ},
wenn alle ihre Einträge $a_{ij}$ nichtnegativ sind: $a_{ij}\ge 0\forall i,j$.
\index{positive Matrix}%
\index{nichtnegative Matrix}%
Man schreibt $A>B$ bzw.~$A\ge B$ wenn $A-B>0$ bzw.~$A-B\ge 0$.
\end{definition}

Die Permutationsmatrizen sind Beispiele von nichtnegativen Matrizen,
deren Produkte wieder nichtnegativ sind.
Dies ist aber ein sehr spezieller Fall, wie das folgende Beispiel
zeigt.

\begin{beispiel}
Wir betrachten die Matrix
\begin{equation}
A=
\begin{pmatrix}
0.9&0.1&   &   &   &   \\
0.1&0.8&0.1&   &   &   \\
   &0.1&0.8&0.1&   &   \\
   &   &0.1&0.8&0.1&   \\
   &   &   &0.1&0.8&0.1\\
   &   &   &   &0.1&0.9
\end{pmatrix}
\label{buch:wahrscheinlichkeit:eqn:diffusion}
\end{equation}
Die Multiplikation eines Vektors mit dieser Matrix bewirkt, dass die
Komponenten des Vektors auf benachbarte Komponenten ``verschmiert'' werden.
Wendet man $A$ wiederholt auf den ersten Standardbasisvektor $v_1=e_1$ an,
erhält man nacheinander die Vektoren $v_2=Av_1$, $v_n = Av_{n-1}$.
\begin{figure}
\centering
\includegraphics{chapters/80-wahrscheinlichkeit/images/diffusion.pdf}
\caption{Die sechs Komponenten für $k=1$ bis $k=6$ der Vektoren $A^{n-1}e_1$
für die Matrix $A$ in \eqref{buch:wahrscheinlichkeit:eqn:diffusion}
sind als Säulen dargestellt.
Sie zeigen, dass für genügend grosses $n$, alle Komponenten
des Vektors $A^{n-1}e_1$ positiv werden.
\label{buch:wahrscheinlichkeit:fig:diffusion}}
\end{figure}
In Abbildung~\ref{buch:wahrscheinlichkeit:fig:diffusion} sind die Komponenten
als Säulen dargestellt.
Man kann erkennen, dass für genügend grosse $n$ alle Komponenten
der Vektoren positiv werden.

Man kann auch direkt die Potenzen $A^n$ ausrechen und sehen, dass
\[
A^5
=
\begin{pmatrix}
   0.65658&  0.27690&  0.05925&  0.00685&  0.00041&  0.00001\\
   0.27690&  0.43893&  0.22450&  0.05281&  0.00645&  0.00041\\
   0.05925&  0.22450&  0.43249&  0.22410&  0.05281&  0.00685\\
   0.00685&  0.05281&  0.22410&  0.43249&  0.22450&  0.05925\\
   0.00041&  0.00645&  0.05281&  0.22450&  0.43893&  0.27690\\
   0.00001&  0.00041&  0.00685&  0.05925&  0.27690&  0.65658
\end{pmatrix}
>0
\]
und dass daher für alle $n\ge 5$ die Matrix $A^n$ positiv ist.
\end{beispiel}

Die Eigenschaft der Matrix $A$ von
\eqref{buch:wahrscheinlichkeit:eqn:diffusion}, dass $A^n>0$
für genügend grosses $n$ ist bei Permutationsmatrizen nicht
vorhanden.
Die Zyklen-Zerlegung einer Permutationsmatrix zeigt, welche
Unterräume von $\mathbb{R}^n$ die iterierten Bilder eines
Standardbasisvektors aufspannen.
Diese sind invariante Unterräume der Matrix.
Das im Beispiel illustrierte Phänomen findet dann nur in invarianten
Unterräumen statt.

\begin{beispiel}
Die Matrix
\begin{equation}
A=\begin{pmatrix}
0.9&0.1&   &   &   &   \\
0.1&0.8&0.1&   &   &   \\
   &0.1&0.9&   &   &   \\
   &   &   &0.9&0.1&   \\
   &   &   &0.1&0.8&0.1\\
   &   &   &   &0.1&0.9
\end{pmatrix}
\label{buch:wahrscheinlichkeit:eqn:diffusionbloecke}
\end{equation}
besteht aus zwei $3\times 3$-Blöcken.
Die beiden Unterräume $V_1=\langle e_1,e_2,e_3\rangle$
und $V_2=\langle e_4,e_5,e_6\rangle$ sind daher invariante
Unterräume von $A$ und damit auch von $A^n$.
Die Potenzen haben daher auch die gleich Blockstruktur.
Insbesondere sind zwar die Blöcke von $A^n$ für $n>1$ positive
Teilmatrizen, aber die Matrix $A^n$ ist für alle $n$ nicht positiv.
\end{beispiel}

\begin{definition}
Eine nichtnegative Matrix mit der Eigenschaft, dass $A^n>0$ für
ein genügend grosses $n$, heisst {\em primitiv}.
\end{definition}

Die Matrix $A$ von \eqref{buch:wahrscheinlichkeit:eqn:diffusion}
ist also primitiv, Permutationsmatrizen sind niemals primitiv.
Die Matrix $A$ von \eqref{buch:wahrscheinlichkeit:eqn:diffusionbloecke}
ist nicht primitiv, aber die einzelnen Blöcke sind primitiv.
Viele der Ausssagen über positive Matrizen lassen sich auf primitive
nichtnegative Matrizen verallgemeinern.

Das Beispiel zeigt auch, dass der Begriff der primitiven Matrix 
eng mit der Idee verknüpft ist, die Problemstellung in invariante
Unterräume aufzuteilen, in denen eine primitive Matrix vorliegt.
Primitive Matrizen werden damit zu naheliegenden Bausteinen für
die Problemlösung für nicht primitive Matrizen.

Eine interessante Eigenschaft positiver Vektoren oder Matrizen
ist, dass die Positivität sich manchmal ``upgraden'' lässt, 
wie im folgenden Satz.
Er zeigt, dass ein Vektor, der grösser ist als ein anderer, auch
um einen definierten Faktor $>1$ grösser ist.
Dies wird geometrisch in 
Abbildung~\ref{buch:wahrscheinlichkeit:figure:trenn} illustriert.

\begin{figure}
\centering
\includegraphics{chapters/80-wahrscheinlichkeit/images/trenn.pdf}
\caption{Die Vektoren $w\le u$ liegen im grauen Rechteck.
Zwei nichtnegative Vektoren $u$ und $v$ mit $u>v$
haben keine gleichen Komponenten.
Daher kann man $v$ mit einer Zahl $\vartheta=1+\varepsilon > 1$
strecken, so dass der gestreckte Vektor $(1+\varepsilon)v$ gerade noch
im grauen Rechteck liegt: $u\ge (1+\varepsilon)v$.
Streckung mit einem grösseren Faktor führt dagegen aus dem Rechteck
hinaus.
\label{buch:wahrscheinlichkeit:figure:trenn}}
\end{figure}

\begin{satz}[Trenntrick]
\label{buch:wahrscheinlichkeit:satz:trenntrick}
Sind $u$ und $v$ nichtnegative Vektoren und $u>v$, dann gibt es eine
positive Zahl $\varepsilon>0$ derart, dass
$u\ge (1+\varepsilon)v$.
Ausserdem kann $\varepsilon$ so gewählt werden, dass $u\not\ge(1+\mu)v$
für $\mu>\varepsilon$.
\end{satz}

\begin{proof}[Beweis]
Wir betrachten die Zahl
\[
\vartheta
=
\max_{v_i\ne 0} \frac{u_i}{v_i}.
\]
Wegen $u>v$ sind die Quotienten auf der rechten Seite alle $>0$.
Da nur endlich viele Quotienten miteinander verglichen werden, ist
daher auch $\vartheta >1$.
Es folgt $u\ge \vartheta v$. 
Wegen $\vartheta >1$ ist $\varepsilon = \vartheta -1 >0$ und
$u\ge (1+\varepsilon)v$.
\end{proof}

Der Satz besagt also, dass es eine Komponente $v_i\ne 0$ gibt
derart, dass $u_i = (1+\varepsilon)v_i$.
Diese Komponenten limitiert also, wie stark man $v$ strecken kann,
so dass er immer noch $\le u$ ist.
Natürlich folgt aus den der Voraussetzung $u>v$ auch, dass $u$ ein 
positiver Vektor ist (Abbildung~\ref{buch:wahrscheinlichkeit:figure:trenn}).

\begin{figure}
\centering
\includegraphics{chapters/80-wahrscheinlichkeit/images/vergleich.pdf}
\caption{Eine positive Matrix $A$ bildet nichtnegative Vektoren in
positive Vektoren ab
(Korollar~\ref{buch:wahrscheinlichkeit:satz:Au>0korollar}).
Zwei verschiedene Vektoren auf einer Seitenfläche erfüllen $u\ge v$,
aber nicht $u>v$, da sie sich in der Koordinaten $x_2$ nicht unterscheiden.
Die Bilder unter $A$ unterscheiden sich dann auch in $x_2$, es gilt
$Au>Av$ (siehe auch Satz~\ref{buch:wahrscheinlichkeit:satz:vergleichstrick})
\label{buch:wahrscheinlichkeit:fig:vergleich}}
\end{figure}

\begin{satz}[Vergleichstrick]
\label{buch:wahrscheinlichkeit:satz:vergleichstrick}
Sei $A$ eine positive Matrix und seinen $u$ und $v$ Vektoren
mit $u\ge v$ und $u\ne v$, dann ist $Au > Av$
(siehe auch Abbildung~\ref{buch:wahrscheinlichkeit:fig:vergleich}).
\end{satz}

\begin{proof}[Beweis]
Wir schreiben $d=u-v$, nach Voraussetzung ist $d\ne 0$.
Der Satz besagt dann, dass aus $d\ge 0$ folgt, dass $Ad>0$, dies
müssen wir beweisen.

Die Ungleichung $Ad>0$ besagt, dass alle Komponenten von $Ad$ 
positiv sind.
Um dies nachzuweisen, berechnen wir
\begin{equation}
(Ad)_i
=
\sum_{j=1}^n
a_{ij}
d_j.
\label{buch:wahrscheinlichkeit:eqn:Adpositiv}
\end{equation}
Alle Terme $a_{ij}>0$, weil $A$ positiv ist, und mindestens eine
der Komponenten $d_j>0$, weil $d\ne 0$.
Insbesondere sind alle Terme der Summe $\ge 0$, woraus wir
bereits schliessen können, dass $(Ad)_i\ge 0$ sein muss.
Die Komponente $d_j>0$ liefert einen positiven Beitrag
$a_{ij}d_j>0$
zur Summe~\eqref{buch:wahrscheinlichkeit:eqn:Adpositiv},
also ist $(Ad)_i>0$.
\end{proof}

Der folgende Spezialfall folgt unmittelbar aus dem
Satz~\ref{buch:wahrscheinlichkeit:satz:vergleichstrick}.

\begin{korollar}
\label{buch:wahrscheinlichkeit:satz:Au>0korollar}
Ist $A$ eine positive Matrix und $u\ge 0$ mit $u\ne 0$, dann
ist $Au>0$.
\end{korollar}

Eine positive Matrix macht also aus nicht verschwindenden
und nicht negativen Vektoren positive Vektoren.

%
% Die verallgemeinerte Dreiecksungleichung
%
\subsection{Die verallgemeinerte Dreiecksungleichung
\label{buch:subsection:verallgemeinerte-dreiecksungleichung}}
Die Dreiecksungleichung besagt, dass für beliebige Vektoren
$u,v\in\mathbb{R}^n$ gilt
\[
|u+v|\le |u|+|v|
\]
mit Gleichheit genau dann, wenn $u$ und $v$ linear abhängig sind.
Wenn beide von $0$ verschieden sind, dann gibt es eine positive Zahl
$t$ mit $u=tv$.
Wir brauchen eine Verallgemeinerung für eine grössere Zahl von
Summanden.

\begin{figure}
\centering
\includegraphics{chapters/80-wahrscheinlichkeit/images/dreieck.pdf}
\caption{Die verallgemeinerte Dreiecksungleichung von
Satz~\ref{buch:wahrscheinlichkeit:satz:verallgemeinerte-dreiecksungleichung}
besagt, dass
die Länge einer Summe von Vektoren (blau) höchstens so gross ist wie die
Summe der Längen, mit Gleichheit genau dann, wenn alle Vektoren die
gleiche Richtung haben (rot).
Hier dargestellt am Beispiel von Zahlen in der komplexen Zahlenebene.
In dieser Form wird die verallgemeinerte Dreiecksungleichung in
Satz~\ref{buch:wahrscheinlichkeit:satz:verallgdreieckC}
\label{buch:wahrscheinlichkeit:fig:dreieck}}
\end{figure}

\begin{satz}[Verallgemeinerte Dreiecksungleichung]
\label{buch:wahrscheinlichkeit:satz:verallgemeinerte-dreiecksungleichung}
Für $n$ Vektoren $v_i\ne 0$ gilt
\[
|u_1+\dots+u_n| \le |u_1|+\dots+|u_n|
\]
mit Gleichheit genau dann, wenn alle Vektoren nichtnegative Vielfache
eines gemeinsamen Einheitsvektors $c$ sind: $u_i=|u_i|c$
(siehe auch Abbildung~\ref{buch:wahrscheinlichkeit:fig:dreieck}).
\end{satz}

\begin{proof}[Beweis]
Die Aussage kann mit vollständiger Induktion bewiesen werden.
Die Induktionsverankerung ist der Fall $n=2$ gegeben durch die
gewöhnliche Dreiecksungleichung.

Wir nehmen daher jetzt an, die Aussage sei für $n$ bereits bewiesen,
wir müssen sie dann für $n+1$ beweisen.
Die Summe von $n+1$ Vektoren kann man $u=u_1+\dots+u_n$ und $v=u_{n+1}$
aufteilen.
Es gilt dann
\[
|u+v|
=
|u_1+\dots+u_n+u_{n+1}|
\]
und
\[
|u_1+\dots+u_n| = |u_1|+\dots+|u_n|.
\]
Aus der Induktionsannahme folgt dann, dass die Vektoren $u_1,\dots,u_n$
positive Vielfache eines Einheitsvektors $u$ sind, $u_i=|u_i|c$.
Es ist dann
\[
u=u_1+\dots+u_n = \biggl(\sum_{i=1}^n |u_i|\biggr).
\]
Aus der gewöhnlichen Dreiecksungleichung, angewendet auf $u$ und $v$
folgt jetzt, dass $v$ ebenfalls ein nichtnegatives Vielfaches von $c$ ist.
Damit ist der Induktionsschritt vollzogen.
\end{proof}

\begin{satz}
\label{buch:wahrscheinlichkeit:satz:verallgdreieckC}
Seien $a_1,\dots,a_n$ positive Zahlen und $u_i\in\mathbb C$ derart,
dass 
\[
\biggl|
\sum_{i=1}^n a_i u_i
\biggr|
=
\sum_{i=1}^n a_i |u_i|,
\]
dann gibt es eine komplexe Zahl $c$ und einen nichtnegativen Vektor $v$
derart, dass $u=cv$.
\end{satz}

Der Satz besagt, dass die komplexen Komponenten $u_i$ alle das gleiche
Argument haben.
Die motiviert den nachstehenden geometrischen Beweis des Satzes.

\begin{proof}[Beweis]
Wer stellen uns die komplexen Zahlen $u_i$ als Vektoren in der
zweidimensionalen Gaussschen Ebene vor.
Dann ist die Aussage nichts anderes als ein Spezialfall von
Satz~\ref{buch:wahrscheinlichkeit:satz:verallgemeinerte-dreiecksungleichung}
für den zweidimensionalen reellen Vektorraum $\mathbb{C}$.
\end{proof}


%
% Der Satz von Perron-Frobenius
%
\subsection{Der Satz von Perron-Frobenius
\label{buch:subsection:der-satz-von-perron-frobenius}}
Wir sind an den Eigenwerten und Eigenvektoren einer positiven
oder primitiven Matrix interessiert.
Nach Definition des Spektralradius $\varrho(A)$ muss es einen Eigenvektor 
zu einem Eigenwert $\lambda$ mit Betrag $|\lambda|=\varrho(A)$ geben,
aber a priori wissen wir nicht, ob es einen reellen Eigenwert vom
Betrag $\varrho(A)$ gibt, und ob der Eigenvektor dazu reell ist.

\begin{figure}
\centering
\includegraphics{chapters/80-wahrscheinlichkeit/images/positiv.pdf}
\caption{Die Iteration einer positiven Matrix bildet den positiven Oktanten
in immer enger werdende Kegel ab, die die Richtung des gesuchten Eigenvektors
gemeinsam haben.
\label{buch:wahrscheinlichkeit:figure:positiv}}
\end{figure}

In Abbildung~\ref{buch:wahrscheinlichkeit:fig:vergleich} kann man sehen,
dass eine positive Abbildung den positiven Oktanten in einen etwas engeren
Kegel hinein abbildet.
Iteriert man dies (Abbildung~\ref{buch:wahrscheinlichkeit:figure:positiv}),
wird die Bildmenge immer enger, bis sie nur ein
sehr enger Kegel um die Richtung des Eigenvektors ist.
Tatsächlich kann man aus dieser Idee auch einen topologischen
Beweis des untenstehenden Satzes von Perron-Frobenius konstruieren.
Er beruht darauf, dass eine Abbildung, die Distanzen verkleinert,
einen Fixpunkt hat.
Die Konstruktion einer geeigneten Metrik ist allerdings eher 
kompliziert, weshalb wir im Beweise der nachstehenden Aussagen
den konventionellen Weg wählen.

Wir beginnen damit zu zeigen, dass für positive Matrizen $A$, 
nichtnegative Eigenvektoren zu Eigenwerten $\lambda\ne 0$
automatisch positiv sind.
Ausserdem müssen die zugehörigen Eigenwerte sogar positiv sein.

\begin{satz}
Sei $A$ eine positive Matrix und $u$ ein nichtnegativer Eigenvektor zum
Eigenwert $\lambda\ne 0$.
Dann ist $u$ ein positiver Vektor und $\lambda > 0$.
\end{satz}

\begin{proof}[Beweis]
Nach dem Korollar~\ref{buch:wahrscheinlichkeit:satz:Au>0korollar}
folgt, dass $Au>0$ ein positiver Vektor ist, es sind
also alle Komponenten positiv.
Der Vektor $u$ ist aber auch ein Eigenvektor, es gilt also
$\lambda u = Au$.
Da alle Komponenten von $Au$ positiv sind, müssen auch
alle Komponenten von $\lambda u$ positiv sein.
Das ist nur möglich, wenn $\lambda > 0$.
\end{proof}

\begin{satz}
\label{buch:wahrscheinlichkeit:satz:positivereigenvektor}
Sei $A$ eine positive Matrix und $v$ ein Eigenvektor von $A$ zu einem
Eigenwert $\lambda$ mit Betrag $|\lambda|=\varrho(A)$,
dann ist der Vektor $u$  mit den Komponenten $u_i=|v_i|$ ein
positiver Eigenvektor zu Eigenwert $\varrho(A)$.
\end{satz}

\begin{proof}[Beweis]
Es gilt natürlich auch, dass
\[
(Au)_i
=
\sum_{j=1}^n a_{ij}u_j
=
\sum_{j=1}^n |a_{ij}v_j|
\ge
\biggl|
\sum_{j=1}^n a_{ij}v_j
\biggr|
=
|(Av)_i|
=
|\lambda v_i|
=
\varrho(A) |v_i|
=
\varrho(A) u_i,
\]
oder $Au \ge \varrho(A)u$.
Wir müssen zeigen, dass sogar $Au=\varrho(A)u$ gilt.
Wir nehmen daher an, dass $Au\ne \varrho(A)u$ ist, und führen dies zu
einem Widerspruch.

Da $\varrho(A)u$ ein nichtnegativer Vektor ist, können wir den Vergleichstrick
Satz~\ref{buch:wahrscheinlichkeit:satz:vergleichstrick}, auf die beiden
Vektoren $Au$ und $\varrho(A)u$ anwenden.
Wir schliessen $A^2u > \varrho(A)Au$.

Mit dem Trenntrick
Satz~\ref{buch:wahrscheinlichkeit:satz:trenntrick}
können wir jetzt eine Zahl $\vartheta>1$ finden derart, dass
\[
A^2 u \ge \vartheta \varrho(A) Au
\]
ist.
Durch weitere Anwendung von $A$ findet man
\begin{align*}
A^3 u & \ge (\vartheta \varrho(A))^2 Au
\\
&\phantom{0}\vdots
\\
A^{k+1} u & \ge (\vartheta \varrho(A))^{k} Au
\end{align*}
Daraus kann man jetzt die Norm abschätzen:
\[
\begin{aligned}
\| A^{k}\|\, |Au|
&\ge 
\| A^{k+1}u\|
\ge
(\vartheta\varrho(A))^{k} |Au|
&&
\Rightarrow
&
\|A^k\| &\ge  (\vartheta\varrho(A))^k 
\\
&&&\Rightarrow&
\|A^k\|^{\frac{1}{k}} &\ge \vartheta\varrho(A)
\\
&&&\Rightarrow&
\lim_{k\to\infty}
\|A^k\|^{\frac{1}{k}} &\ge \vartheta\varrho(A)
\\
&&&\Rightarrow&
\varrho(A)&\ge \vartheta\varrho(A)
\end{aligned}
\]
Wegen $\vartheta>1$ ist dies aber gar nicht möglich.
Dieser Widerspruch zeigt, dass $u=v$ sein muss, insbesondere ist
$v$ ein nichtnegativer Eigenvektor.
\end{proof}

\begin{satz}
Sei $A$ eine positive Matrix und $v$ ein Eigenvektor zu einem 
Eigenwert $\lambda$ mit Betrag $|\lambda|=\varrho(A)$.
Dann ist $\lambda=\varrho(A)$.
\end{satz}

\begin{proof}[Beweis]
Nach Satz~\ref{buch:wahrscheinlichkeit:satz:positivereigenvektor}
ist der Vektor $u$ mit den Komponenten $u_i=|v_i|$ ein positiver
Eigenvektor zum Eigenwert $\varrho(A)$.
Aus der Eigenvektorgleichung für $u$ folgt
\begin{equation}
Au = \varrho(A) u
\quad\Rightarrow\quad
\sum_{j=1}^n a_{ij}|v_j| = \varrho(A) |v_i|.
\label{buch:wahrscheinlichkeit:eqn:pev1}
\end{equation}
Anderseits ist $v$ ein Eigenvektor zum Eigenwert $\lambda$, also gilt
\[
\sum_{j=1}^n a_{ij}v_j = \lambda v_i.
\]
Der Betrag davon ist
\begin{equation}
\biggl|
\sum_{j=1}^n a_{ij}v_j
\biggr|
=
|\lambda v_i|
=
\varrho(A) |v_i|
=
\varrho |v_i|.
\label{buch:wahrscheinlichkeit:eqn:pev2}
\end{equation}
Die beiden Gleichungen
\eqref{buch:wahrscheinlichkeit:eqn:pev1}
und
\eqref{buch:wahrscheinlichkeit:eqn:pev2}
zusammen ergeben die Gleichung
\[
\biggl|
\sum_{j=1}^n a_{ij}v_j
\biggr|
=
\sum_{j=1}^n a_{ij}|v_j|.
\]
Nach der verallgemeinerten Dreiecksungleichung
Satz~\ref{buch:subsection:verallgemeinerte-dreiecksungleichung}
folgt jetzt, dass es eine komplexe Zahl $c$ vom Betrag $1$ gibt derart,
dass $v_j = |v_j|c=u_jc$.
Insbesondere ist $v=cu$ und damit ist 
\[
\lambda v = Av = Acu = c Au = c\varrho(A) u = \varrho(A) v,
\]
woraus $\lambda=\varrho(A)$ folgt.
\end{proof}

\begin{satz}
\label{buch:wahrscheinlichkeit:satz:geometrischeinfach}
Der Eigenraum einer positiven Matrix $A$ zum Eigenwert $\varrho(A)$ ist
eindimensional.
\end{satz}

\begin{proof}[Beweis]
Sei $u$ der bereits gefundene Eigenvektor von $A$ zum Eigenwert $\varrho(A)$
und sei $v$ ein weiterer, linear unabhängiger Eigenvektor zum
gleichen Eigenwert.
Da $A$ und $\varrho(A)$ reell sind, sind auch die Vektoren $\Re v$ und $\Im v$
aus den Realteilen $\Re v_i$ oder den Imaginärteilen $\Im v_i$ Eigenvektoren.
Beide Vektoren sind reelle Vektoren und mindestens einer davon ist mit
$u$ linear unabhängig.
Wir dürfen daher annehmen, dass $v$ ein linear unabhängiger Eigenvektor
zum Eigenwert $\varrho(A)$ ist.

Weil wir wissen, dass $u$ ein positiver Vektor ist, gibt es einen
grösstmöglichen Faktor $c>0$ derart, dass $u\ge cv$ oder $u\ge cv$.
Insbesondere verschwindet mindestens eine Komponente von $u-cv$.
Da $u$ und $v$ Eigenvektoren zum Eigenwert $\varrho(A)$ sind,
ist 
\[
A(u-cv)
=
\varrho(A)(u-cv).
\]
Der Vektor auf der rechten Seite hat mindestens eine verschwindende 
Komponente.
Der Vektor auf der linken Seite ist nach Vergleichstrick
Satz~\ref{buch:wahrscheinlichkeit:satz:vergleichstrick}
\[
A(u-cv) > 0,
\]
alle seine Komponenten sind $>0$.
Dieser Widerspruch zeigt, dass die Annahme, es gäbe einen von $u$ linear 
unabhängigen Eigenvektor zum Eigenwert $\varrho(A)$ nicht haltbar ist.
\end{proof}

\begin{satz}
\label{buch:wahrscheinlichkeit:satz:algebraischeinfach}
Der verallgemeinerte Eigenraum zum Eigenwert $\varrho(A)$ einer 
positiven Matrix $A$ ist eindimensional.
Ist $u$ der Eigenvektor von $A$ zum Eigenwert $\varrho(A)$ nach
Satz~\ref{buch:wahrscheinlichkeit:satz:geometrischeinfach}
und $p^t$ der entsprechende Eigenvektor $A^t$, dann
ist
\[
\mathbb{R}^n
=
\langle u\rangle
\oplus
\{ x\in\mathbb{R}^n\;|\; px=0\}
=
\langle u\rangle
\oplus
\ker p
\]
eine Zerlegung in invariante Unterräume von $A$.
\end{satz}

\begin{proof}[Beweis]
Die beiden Vektoren $x$ und $p$ sind beide positiv, daher ist das 
Produkt $pu\ne 0$.
Insbesondere ist $u\not\in\ker p$

Es ist klar, dass $A\langle u\rangle = \langle Au\rangle = \langle u\rangle$
ein invarianter Unterraum ist.
Für einen Vektor $x\in\mathbb{R}^n$ mit $px=0$ erfüllt das Bild $Ax$
\[
p(Ax)=(pA)x=(A^tp^t)^tx=
\varrho(A)(p^t)^tx
=
\varrho(A)px = 0,
\]
somit ist $A\ker p \subset \ker p$.
Beide Räume sind also invariante Unteräume.

$\ker p$ ist $(n-1)$-dimensional, $\langle u\rangle$ ist eindimensional
und $u$ ist nicht in $\ker p$ enthalten.
Folglich spannen $\langle u\rangle$ und $\ker p$ den ganzen Raum auf.

Gäbe es einen weitern linear unabhängigen Vektor im verallgemeinerten
Eigenraum von $\mathcal{E}_{\varrho(A)}$, dann müsste es auch einen
solchen Vektor in $\ker p$ geben.
Da $\ker p$ invariant ist, müsste es also auch einen weiteren Eigenvektor
$u_2$ zum Eigenwert $\varrho(A)$ in $\ker p$ geben.
Die beiden Vektoren $u$ und $u_1$ sind dann beide Eigenvektoren, was
nach Satz~\ref{buch:wahrscheinlichkeit:satz:geometrischeinfach}
nicht möglich ist.
\end{proof}

Die in den Sätzen
\ref{buch:wahrscheinlichkeit:satz:positivereigenvektor}
bis
\ref{buch:wahrscheinlichkeit:satz:algebraischeinfach}
gefundenen Resultate können wir folgt zusammengefasst werden:

\begin{satz}[Perron-Frobenius]
\label{buch:wahrscheinlichkeit:satz:perron-frobenius}
Sei $A$ eine positive Matrix mit Spektralradius $\varrho(A)$.
Dann gibt es einen positiven Eigenvektor zum Eigenwert $\varrho(A)$,
mit geometrischer und algebraischer Vielfachheit $1$.
\end{satz}

Der Satz~\ref{buch:wahrscheinlichkeit:satz:perron-frobenius}
von Perron-Frobenius kann auf primitive Matrizen verallgemeinert
werden.

\begin{satz}
Sei $A$ ein primitive, nichtnegative Matrix.
Dann ist $\varrho(A)$ der einzige Eigenwert vom Betrag $\varrho(A)$
und er hat geometrische und algebraische Vielfachheit $1$.
\end{satz}

\begin{proof}[Beweis]
Nach Voraussetzung gibt es ein $n$ derart, dass $A^n>0$.
Für $A^n$ gelten die Resultate von 
Satz~\ref{buch:wahrscheinlichkeit:satz:perron-frobenius}.
\end{proof}

%
% parrondo.tex -- Anwendung: analyse von Parrondos Paradoxon
%
% (c) 2020 Prof Dr Andreas Müller, Hochschule Rapperswil
%
\section{Das Paradoxon von Parrondo
\label{buch:section:paradoxon-von-parrondo}}
\rhead{Das Paradoxon von Parrondo}
Das Paradoxon von Parrondo ist ein der Intuition widersprechendes
Beispiel für eine Kombination von Spielen mit negativer Gewinnerwartung,
deren Kombination zu einem Spiel mit positiver Gewinnerwartung führt.
Die Theorie der Markov-Ketten und der zugehörigen Matrizen ermöglicht
eine sehr einfache Analyse.

%
% Parrondo Teilspiele
%
\subsection{Die beiden Teilspiele
\label{buch:subsection:teilspiele}}

\subsubsection{Das Spiel $A$}
Das Spiel $A$ besteht darin, eine Münze zu werfen.
Je nach Ausgang gewinnt oder verliert der Spieler eine Einheit.
Sei $X$ die Zufallsvariable, die den gewonnen Betrag beschreibt.
Für eine faire Münze ist die Gewinnerwartung in diesem Spiel natürlich
$E(X)=0$.
Wenn die Wahrscheinlichkeit für einen Gewinn $\frac12+e$ ist, dann muss
die Wahrscheinlichkeit für einen Verlust $\frac12-e$ sein, und die 
Gewinnerwartung ist
\(
E(X)
=
1\cdot P(X=1) + (-1)\cdot P(X=-1)
=
\frac12+e + (-1)\biggl(\frac12-e\biggr)
=
2e.
\)
Die Gewinnerwartung ist also genau dann negativ, wenn $e<0$ ist.

\subsubsection{Das Spiel $B$}
Das zweite Spiel $B$ ist etwas komplizierter, da der Spielablauf vom 
aktuellen Kapital $K$ des Spielers abhängt.
Wieder gewinnt oder verliert der Spieler eine Einheit,
die Gewinnwahrscheinlichkeit hängt aber vom Dreierrest des Kapitals ab.
Sei $Y$ die Zufallsvariable, die den Gewinn beschreibt.
Ist $K$ durch drei teilbar, ist die Gewinnwahrscheinlichkeit $\frac1{10}$,
andernfalls ist sie $\frac34$.
Formell ist
\begin{equation}
\begin{aligned}
P(Y=1|\text{$K$ durch $3$ teilbar}) &=  \frac{1}{10}
\\
P(Y=1|\text{$K$ nicht durch $3$ teilbar}) &= \frac{3}{4}
\end{aligned}
\label{buch:wahrscheinlichkeit:eqn:Bwahrscheinlichkeiten}
\end{equation}
Insbesondere ist die Wahrscheinlichkeit für einen Gewinn in zwei der
Fälle recht gross, in einem Fall aber sehr klein.

\subsubsection{Übergangsmatrix im Spiel $B$}
\begin{figure}
\centering
\includegraphics{chapters/80-wahrscheinlichkeit/images/spielB.pdf}
\caption{Zustandsdiagramm für das Spiel $B$, Zustände sind die
Dreierreste des Kapitals.
\label{buch:wahrscheinlichkeit:fig:spielB}}
\end{figure}%
Für den Verlauf des Spiels spielt nur der Dreierrest des Kapitals
eine Rolle.
Es gibt daher drei mögliche Zustände $0$, $1$ und $2$.
In einem Spielzug finde ein Übergang in einen anderen Zustand
statt, der Eintrag $b_{ij}$ ist die Wahrscheinlichkeit
\[
b_{ij}
=
P(K\equiv i|K\equiv j),
\]
dass ein Übergang vom Zustand $j$ in den Zustand $i$ stattfindet.
Die Matrix ist
\[
B=
\begin{pmatrix}
0          &\frac14 &\frac34\\
\frac1{10} &0       &\frac14\\
\frac9{10} &\frac34 &0
\end{pmatrix}.
\]

\subsubsection{Gewinnerwartung in einem Einzelspiel $B$}
Die Gewinnerwartung einer einzelnen Runde des Spiels $B$ hängt natürlich
ebenfalls vom Ausgangskapital ab.
Mit den Wahrscheinlichkeiten von 
\eqref{buch:wahrscheinlichkeit:eqn:Bwahrscheinlichkeiten}
findet man die Gewinnerwartung
\begin{equation}
\begin{aligned}
E(Y| \text{$K$ durch $3$ teilbar})
&=
1\cdot P(Y=1|K\equiv 0\mod 3)
+
(-1)\cdot P(Y=-1|K\equiv 0\mod 3)
\\
&=
\frac1{10}
-
\frac{9}{10}
=
-\frac{8}{10}
\\
E(Y| \text{$K$ nicht durch $3$ teilbar})
&=
1\cdot P(Y=1|K\not\equiv 0\mod 3)
+
(-1)\cdot P(Y=-1|K\not\equiv 0\mod 3)
\\
&=
\frac34-\frac14
=
\frac12.
\end{aligned}
\label{buch:wahrscheinlichkeit:eqn:Berwartungen}
\end{equation}
Falls $K$ durch drei teilbar ist, muss der Spieler
also mit einem grossen Verlust rechnen, andernfalls mit einem
moderaten Gewinn.

Ohne weiteres Wissen über das Anfangskapital ist es zulässig anzunehmen,
dass die drei möglichen Reste die gleiche Wahrscheinlichkeit haben.
Die Gewinnerwartung in diesem Fall ist dann
\begin{align}
E(Y)
&=
E(Y|\text{$K$ durch $3$ teilbar}) \cdot \frac13
+
E(Y|\text{$K$ nicht durch $3$ teilbar}) \cdot \frac23
\notag
\\
&=
-\frac{8}{10}\cdot\frac{1}{3}
+
\frac{1}{2}\cdot\frac{2}{3}
=
-\frac{8}{30}+\frac{10}{30}
=
\frac{2}{30}
=
\frac{1}{15}.
\label{buch:wahrscheinlichkeit:eqn:Beinzelerwartung}
\end{align}
Unter der Annahme, dass alle Reste die gleiche Wahrscheinlichkeit haben,
ist das Spiel also ein Gewinnspiel.

Die Berechnung der Gewinnerwartung in einem Einzelspiel kann man 
wie folgt formalisieren.
Die Matrix $B$ gibt die Übergangswahrscheinlichkeiten zwischen
verschiedenen Zuständen.
Die Matrix 
\[
G=\begin{pmatrix}
 0&-1& 1\\
 1& 0&-1\\
-1& 1& 0
\end{pmatrix}
\]
gibt die Gewinne an, die bei einem Übergang anfallen.
Die Matrixelemente $g_{ij}b_{ij}$ des Hadamard-Produktes 
$G\odot B$
von $G$ mit $B$ enthält in den Spalten die Gewinnerwartungen
für die einzelnen Übergänge aus einem Zustand.
Die Summe der Elemente der Spalte $j$ enthält die Gewinnerwartung
\[
E(Y|K\equiv j)
=
\sum_{i=0}^2 g_{ij}b_{ij}
\]
für einen Übergang aus dem Zustand $j$.
Man kann dies auch als einen Zeilenvektor schreiben, der durch Multiplikation
der Matrix $G\odot B$ mit dem Zeilenvektor
$U^t=\begin{pmatrix}1&1&1\end{pmatrix}$
entsteht:
\[
\begin{pmatrix}
E(Y|K\equiv 0)&
E(Y|K\equiv 1)&
E(Y|K\equiv 2)
\end{pmatrix}
=
U^t
G\odot B.
\]
Die Gewinnerwartung ist dann das Produkt
\[
E(Y)
=
\sum_{i=0}^2
E(Y|K\equiv i) p_i
=
U^t
(G\odot B)p.
\]
Tatsächlich ist
\[
G\odot B
=
\begin{pmatrix}
 0          &-\frac14 & \frac34\\
 \frac1{10} & 0       &-\frac14\\
-\frac9{10} & \frac34 & 0
\end{pmatrix}
\quad\text{und}\quad
U^t G\odot B
=
\begin{pmatrix}-\frac{8}{10}&\frac12&\frac12\end{pmatrix}.
\]
Dies stimmt mit den Erwartungswerten in 
\eqref{buch:wahrscheinlichkeit:eqn:Berwartungen}
überein.
Die gesamte Geinnerwartung ist dann
\begin{equation}
(G\odot B)
\begin{pmatrix}\frac13\\\frac13\\\frac13\end{pmatrix}
=
\begin{pmatrix}-\frac{8}{10}&\frac12&\frac12\end{pmatrix}
\frac13U
=
\frac13\biggl(-\frac{8}{10}+\frac12+\frac12\biggr)
=
\frac13\cdot\frac{2}{10}
=
\frac{1}{15},
\label{buch:wahrscheinlichkeit:eqn:BodotEinzelerwartung}
\end{equation}
dies stimmt mit \eqref{buch:wahrscheinlichkeit:eqn:Beinzelerwartung}
überrein.

\subsubsection{Das wiederholte Spiel $B$}
Natürlich spielt man das Spiel nicht nur einmal, sondern man wiederholt es.
Es ist verlockend anzunehmen, dass die Dreierreste $0$, $1$ und $2$ des
Kapitals immer noch gleich wahrscheinlich sind.
Dies braucht jedoch nicht so zu sein.
Wir prüfen die Hypothese daher, indem wir die Wahrscheinlichkeit
für die verschiedenen Dreierreste des Kapitals in einem interierten
Spiels ausrechnen.

Das Spiel kennt die Dreierreste als die drei für das Spiel ausschlaggebenden
Zuständen.
Das Zustandsdiagramm~\ref{buch:wahrscheinlichkeit:fig:spielB} zeigt
die möglichen Übergänge und ihre Wahrscheinlichkeiten, die zugehörige
Matrix ist
\[
B
=
\begin{pmatrix}
0          &\frac14 &\frac34\\
\frac1{10} &0       &\frac14\\
\frac9{10} &\frac34 &0
\end{pmatrix}
\]
Die Matrix $B$ ist nicht negativ und man kann nachrechnen, dass $B^2>0$ ist.
Damit ist die Perron-Frobenius-Theorie von
Abschnitt~\ref{buch:section:positive-vektoren-und-matrizen}
anwendbar.

Ein Eigenvektor zum Eigenwert $1$ kann mit Hilfe des Gauss-Algorithmus
gefunden werden:
\begin{align*}
\begin{tabular}{|>{$}c<{$}>{$}c<{$}>{$}c<{$}|}
\hline
-1         &\frac14 &\frac34 \\
\frac1{10} &-1      &\frac14 \\
\frac9{10} &\frac34 &-1      \\
\hline
\end{tabular}
&\rightarrow
\begin{tabular}{|>{$}c<{$}>{$}c<{$}>{$}c<{$}|}
\hline
1          &-\frac14       &-\frac34       \\
0          &-\frac{39}{40} & \frac{13}{40} \\
0          & \frac{39}{40} &-\frac{13}{40} \\
\hline
\end{tabular}
\rightarrow
\begin{tabular}{|>{$}c<{$}>{$}c<{$}>{$}c<{$}|}
\hline
1 &-\frac14 &-\frac34 \\
0 & 1       &-\frac13 \\
0 & 0       & 0       \\
\hline
\end{tabular}
\rightarrow
\begin{tabular}{|>{$}c<{$}>{$}c<{$}>{$}c<{$}|}
\hline
1 & 0 &-\frac56 \\
0 & 1 &-\frac13 \\
0 & 0 & 0       \\
\hline
\end{tabular}
\end{align*}
Daraus liest man einen möglichen Lösungsvektor mit den Komponenten
$5$, $2$ und $6$ ab.
Wir suchen aber einen Eigenvektor, der als Wahrscheinlichkeitsverteilung
dienen kann.
Dazu müssen sich die Komponente zu $1$ summieren, was man durch normieren
in der $l^1$-Norm erreichen kann:
\begin{equation}
p
=
\begin{pmatrix}
P(K\equiv 0)\\
P(K\equiv 1)\\
P(K\equiv 2)
\end{pmatrix}
=
\frac{1}{5+2+6}
\begin{pmatrix}
5\\2\\6
\end{pmatrix}
=
\frac{1}{13}
\begin{pmatrix}
5\\2\\6
\end{pmatrix}
\approx
\begin{pmatrix}
   0.3846 \\
   0.1538 \\
   0.4615
\end{pmatrix}.
\label{buch:wahrscheinlichkeit:spielBP}
\end{equation}
Die Hypothese, dass die drei Reste gleich wahrscheinlich sind, ist
also nicht zutreffend.

Die Perron-Frobenius-Theorie sagt, dass sich die
Verteilung~\ref{buch:wahrscheinlichkeit:spielBP} nach einiger Zeit
einstellt.
Wir können jetzt auch die Gewinnerwartung in einer einzelnen 
Runde des Spiels ausgehend von dieser Verteilung der Reste des Kapitals
berechnen.
Dazu brauchen wir zunächst die Wahrscheinlichkeiten für Gewinn oder
Verlust, die wir mit dem Satz über die totale Wahrscheinlichkeit 
nach
\begin{align*}
P(Y=+1)
&=
P(Y=+1|K\equiv 0) \cdot P(K\equiv 0)
+
P(Y=+1|K\equiv 1) \cdot P(K\equiv 1)
+
P(Y=+1|K\equiv 2) \cdot P(K\equiv 2)
\\
&=
\frac{1}{10}\cdot\frac{5}{13}
+
\frac{3}{4} \cdot\frac{2}{13}
+
\frac{3}{4} \cdot\frac{6}{13}
\\
&=
\frac1{13}\biggl(
\frac{1}{2}+\frac{3}{2}+\frac{9}{2}
\biggr)
=
\frac{13}{26}
=
\frac12
\\
P(Y=-1)
&=
P(Y=-1|K\equiv 0) \cdot P(K\equiv 0)
+
P(Y=-1|K\equiv 1) \cdot P(K\equiv 1)
+
P(Y=-1|K\equiv 2) \cdot P(K\equiv 2)
\\
&=
\frac{9}{10}\cdot\frac{5}{13}
+
\frac{1}{4} \cdot\frac{2}{13}
+
\frac{1}{4} \cdot\frac{6}{13}
\\
&=
\frac{1}{13}\biggl(
\frac{9}{2} + \frac{1}{2} + \frac{3}{2}
\biggr)
=
\frac{1}{2}
\end{align*}
berechnen können.
Gewinn und Verlust sind also gleich wahrscheinlich, das Spiel $B$ ist also
ebenfalls fair.

Auch diese Gewinnwahrscheinlichkeit kann etwas formeller mit dem
Hadamard-Produkt berechnet werden:
\[
U^t (G\odot B) p
=
\begin{pmatrix}-\frac{8}{10}&\frac12&\frac12\end{pmatrix}
\frac{1}{13}
\begin{pmatrix}
5\\2\\6
\end{pmatrix}
=
-\frac{8}{10}\cdot\frac{5}{13}
+\frac{1}{2} \cdot\frac{2}{13}
+\frac{1}{2} \cdot\frac{6}{13}
=
\frac{1}{26}(-8 + 2+ 6)
=
0,
\]
wie erwartet.

\subsubsection{Das modifizierte Spiel $\tilde{B}$}
\begin{figure}
\centering
\includegraphics{chapters/80-wahrscheinlichkeit/images/spielBtilde.pdf}
\caption{Zustandsdiagramm für das modifizerte Spiel $\tilde{B}$,
Zustände sind die Dreierreste des Kapitals.
Gegenüber dem Spiel $B$
(Abbildung~\ref{buch:wahrscheinlichkeit:fig:spielB})
sind die Wahrscheinlichkeiten für Verlust 
um $\varepsilon$ vergrössert und die Wahrscheinlichkeiten für Gewinn um
$\varepsilon$ verkleinert worden.
\label{buch:wahrscheinlichkeit:fig:spielBtile}}
\end{figure}
%
Wir modifizieren jetzt das Spiel $B$ derart, dass die Wahrscheinlichkeiten
für Gewinn um $\varepsilon$ verringert werden und die Wahrscheinlichkeiten
für Verlust um $\varepsilon$ vergrössert werden.
Die Übergangsmatrix des modifzierten Spiels $\tilde{B}$ ist
\[
\tilde{B}
=
\begin{pmatrix}
 0                       & \frac{1}{4}+\varepsilon & \frac{3}{4}-\varepsilon \\
\frac{1}{10}-\varepsilon & 0                       & \frac{1}{4}+\varepsilon \\
\frac{9}{10}+\varepsilon & \frac{3}{4}-\varepsilon & 0
\end{pmatrix}
=
B
+
\varepsilon
\underbrace{
\begin{pmatrix}
 0& 1&-1\\
-1& 0& 1\\
 1&-1& 0
\end{pmatrix}
}_{\displaystyle F}
\]
Wir wissen bereits, dass der Vektor $p$
von \eqref{buch:wahrscheinlichkeit:spielBP}
als stationäre Verteilung
Eigenvektor zum Eigenwert
$B$ ist, wir versuchen jetzt in erster Näherung die modifizierte
stationäre Verteilung $p_{\varepsilon}=p+\varepsilon p_1$ des modifizierten
Spiels zu bestimmen.

\subsubsection{Gewinnerwartung im modifizierten Einzelspiel}
Die Gewinnerwartung aus den verschiedenen Ausgangszuständen kann mit Hilfe
des Hadamard-Produktes berechnet werden.
Wir berechnen dazu zunächst
\[
G\odot \tilde{B}
=
G\odot (B+\varepsilon F)
=
G\odot B + \varepsilon G\odot F
\quad\text{mit}\quad
G\odot F = \begin{pmatrix}
0&1&1\\
1&0&1\\
1&1&0
\end{pmatrix}.
\]
Nach der früher dafür gefundenen Formel ist
\begin{align*}
\begin{pmatrix}
E(Y|K\equiv 0)&
E(Y|K\equiv 1)&
E(Y|K\equiv 2)
\end{pmatrix}
&=
U^t (G\odot \tilde{B})
\\
&=
U^t (G\odot B)
+
\varepsilon
U^t (G\odot F)
\\
&=
\begin{pmatrix} -\frac{8}{10}&\frac12&\frac12 \end{pmatrix}
+
2\varepsilon U^t
\\
&=
\begin{pmatrix} -\frac{8}{10}+2\varepsilon&\frac12+2\varepsilon&\frac12+2\varepsilon \end{pmatrix}.
\end{align*}
Unter der Annahme gleicher Wahrscheinlichkeiten für die Ausgangszustände,
erhält man die Gewinnerwartung
\begin{align*}
E(Y)
&=
U^t(G\odot \tilde{B})
\begin{pmatrix}
\frac13\\
\frac13\\
\frac13
\end{pmatrix}
\\
&=
U^t
(G\odot B)
\frac13 U
+
\varepsilon
U^t
(G\odot F)
\frac13 U
\\
&=
\frac1{15}
+
2\varepsilon
\end{align*}
unter Verwendung der in
\eqref{buch:wahrscheinlichkeit:eqn:BodotEinzelerwartung}
berechneten Gewinnerwartung für das Spiel $B$.

\subsubsection{Iteration des modifizierten Spiels}
Der Gaussalgorithmus liefert nach einiger Rechnung, die man am besten
mit einem Computeralgebrasystem durchführt,
\[
\begin{tabular}{|>{$}c<{$}>{$}c<{$}>{$}c<{$}|}
\hline
-1                       & \frac{1}{4}+\varepsilon & \frac{3}{4}-\varepsilon \\
\frac{1}{10}-\varepsilon & -1                      & \frac{1}{4}+\varepsilon \\
\frac{9}{10}+\varepsilon & \frac{3}{4}-\varepsilon & -1                      \\
\hline
\end{tabular}
\rightarrow
%                [           2                   ]
%                [ 80 epsilon  + 12 epsilon + 78 ]
%(%o15)  Col 1 = [                               ]
%                [               0               ]
%                [                               ]
%                [               0               ]
%         [               0               ]
%         [                               ]
% Col 2 = [           2                   ]
%         [ 80 epsilon  + 12 epsilon + 78 ]
%         [                               ]
%         [               0               ]
%         [              2                    ]
%         [ (- 80 epsilon ) + 40 epsilon - 65 ]
%         [                                   ]
% Col 3 = [              2                    ]
%         [ (- 80 epsilon ) - 12 epsilon - 26 ]
%         [                                   ]
%         [                 0                 ]
\begin{tabular}{|>{$}c<{$}>{$}c<{$}>{$}c<{$}|}
\hline
1&0&-\frac{65-40\varepsilon+80\varepsilon^2}{78+12\varepsilon+80\varepsilon^2}\\
0&0&-\frac{26+12\varepsilon+80\varepsilon^2}{78+12\varepsilon+80\varepsilon^2}\\
0&0&0\\
\hline
\end{tabular},
\]
woraus man die Lösung
\[
p
=
\begin{pmatrix}
65-40\varepsilon+80\varepsilon^2\\
26+12\varepsilon+80\varepsilon^2\\
78+12\varepsilon+80\varepsilon^2\\
\end{pmatrix}
\]
ablesen kann.
Allerdings ist dies keine Wahrscheinlichkeitsverteilung,
wir müssen dazu wieder normieren.
Die Summe der Komponenten ist
\[
\|p\|_1
=
169 - 16 \varepsilon + 240 \varepsilon^2.
\]
Damit bekommen wir für die Lösung bis zur ersten Ordnung
\[
p_\varepsilon
=
\frac{1}{ 169 - 16 \varepsilon + 240 \varepsilon^2}
\begin{pmatrix}
65-40\varepsilon+80\varepsilon^2\\
26+12\varepsilon+80\varepsilon^2\\
78+12\varepsilon+80\varepsilon^2\\
\end{pmatrix}
=
%          [                                 2                   3         ]
%          [ 5    440 epsilon   34080 epsilon    17301120 epsilon          ]
%          [ -- - ----------- - -------------- + ----------------- + . . . ]
%          [ 13      2197           371293           62748517              ]
%          [                                                               ]
%          [                                 2                  3          ]
%(%o19)/T/ [ 2    188 epsilon   97648 epsilon    6062912 epsilon           ]
%          [ -- + ----------- + -------------- - ---------------- + . . .  ]
%          [ 13      2197           371293           62748517              ]
%          [                                                               ]
%          [                                 2                   3         ]
%          [ 6    252 epsilon   63568 epsilon    11238208 epsilon          ]
%          [ -- + ----------- - -------------- - ----------------- + . . . ]
%          [ 13      2197           371293           62748517              ]
\frac{1}{13}
\begin{pmatrix} 5\\2\\6 \end{pmatrix}
+
\frac{\varepsilon}{2197}
\begin{pmatrix}
-440\\188\\252
\end{pmatrix}
+
O(\varepsilon^2).
\]
Man beachte, dass der konstante Vektor der ursprüngliche Vektor $p$
für das Spiel $B$ ist.
Der lineare Term ist ein Vektor, dessen Komponenten sich zu $1$ summieren,
in erster Ordnung ist also die $l^1$-Norm des Vektors wieder 
$\|p_\varepsilon\|_1=0+O(\varepsilon^2)$.

Mit den bekannten Wahrscheinlichkeiten kann man jetzt die
Gewinnerwartung in einem einzeln Spiel ausgehend von der Verteilung
$p_{\varepsilon}$ berechnen.
Dazu braucht man das Hadamard-Produkt
\[
G\odot \tilde{B}
=
G=\begin{pmatrix}
 0&-1& 1\\
 1& 0&-1\\
-1& 1& 0
\end{pmatrix}
\odot
\begin{pmatrix}
0                        &\frac14+\varepsilon & \frac34-\varepsilon \\
\frac{1}{10}-\varepsilon & 0                  & \frac14+\varepsilon \\
\frac{9}{10}+\varepsilon &\frac34-\varepsilon & 0
\end{pmatrix}
=
\begin{pmatrix}
 0                        &-\frac14-\varepsilon & \frac34-\varepsilon \\
 \frac{1}{10}-\varepsilon & 0                   &-\frac14-\varepsilon \\
-\frac{9}{10}-\varepsilon & \frac34-\varepsilon & 0
\end{pmatrix}
\]
Wie früher ist der erwartete Gewinn
\begin{align*}
E(Y)
&=
U^t (G\odot \tilde{B}) p_{\varepsilon}
\\
&=
\begin{pmatrix}
-\frac{3}{10}-2\varepsilon & \frac12-2\varepsilon & \frac12-2\varepsilon
\end{pmatrix}
p_{\varepsilon}
\\
%                               3             2
%                    480 epsilon  - 48 epsilon  + 294 epsilon
%(%o50)            - ----------------------------------------
%                                   2
%                        240 epsilon  - 16 epsilon + 169
&=
-
\varepsilon\cdot
\frac{
294-48\varepsilon+480\varepsilon^2
}{
169-16\varepsilon+240\varepsilon^2
}
=
-\frac{294}{169}\varepsilon + O(\varepsilon^2).
\end{align*}
Insbesondere ist also die Gewinnerwartung negativ für nicht zu grosse 
$\varepsilon>0$.
Das Spiel ist also ein Verlustspiel.

%
% Die Kombination
%
\subsection{Kombination der Spiele
\label{buch:subsection:kombination}}
Jetzt werden die beiden Spiele $A$ und $B$ zu einem neuen
Spiel kombiniert.
Für das Spiel $A$ haben wir bis jetzt keine Übergansmatrix aufgestellt,
da das Kapital darin keine Rolle spielt.
Um die beiden Spiele kombinieren zu können brauchen wir aber die Übergansmatrix
für die drei Zustände $K\equiv 0,1,2$.
Sie ist
\[
A=\begin{pmatrix}
0&\frac12&\frac12\\
\frac12&0&\frac12\\
\frac12&\frac12&0
\end{pmatrix}.
\]

\subsubsection{Das Spiel $C$}
In jeder Durchführung des Spiels wird mit einem Münzwurf entschieden,
ob Spiel $A$ oder Spiel $B$ gespielt werden soll.
Mit je Wahrscheinlichkeit $\frac12$ werden also die Übergansmatrizen
$A$ oder $B$ verwendet:
\[
P(K\equiv i|K\equiv j)
=
A\cdot P(\text{Münzwurf Kopf})
+
B\cdot P(\text{Münzwurf Kopf})
=
\frac12(A+B)
=
\begin{pmatrix}
0            & \frac{3}{8} & \frac{5}{8} \\
\frac{3}{10} & 0           & \frac{3}{8} \\
\frac{7}{10} & \frac{5}{8} & 0
\end{pmatrix}
\]
Die Gewinnerwartung in einem Einzelspiel ist
\begin{align*}
E(Y)
&=
U^t
(G\odot C)
\frac13U
\\
&=
U^t
\begin{pmatrix}
 0            &-\frac{3}{8} & \frac{5}{8} \\
 \frac{3}{10} & 0           &-\frac{3}{8} \\
-\frac{7}{10} & \frac{5}{8} & 0
\end{pmatrix}
\frac13U
\\
&=
\begin{pmatrix}
-\frac{2}{5} & \frac{1}{4} & \frac{1}{4}
\end{pmatrix}
\frac13U
=
\frac13\biggl(-\frac{2}{5}+\frac{1}{4}+\frac{1}{4}\biggr)
=
-\frac{1}{30}
\end{align*}
Das Einzelspiel ist also ein Verlustspiel.

\subsubsection{Das iterierte Spiel $C$}
Für das iterierte Spiel muss man wieder den Eigenvektor von $C$ zum
Eigenwert $1$ finden, die Rechnung mit dem Gauss-Algorithmus liefert
\[
p=
\frac{1}{709}
\begin{pmatrix}
245\\180\\84
\end{pmatrix}.
\]
Damit kann man jetzt die Gewinnwahrscheinlichkeit im iterierten Spiel
berechnen, es ist
\begin{align*}
E(Y)
&=
U^t
(G\odot C) p
\\
&=
\begin{pmatrix}
-\frac{2}{5} & \frac{1}{4} & \frac{1}{4}
\end{pmatrix}
\frac{1}{709}
\begin{pmatrix}
245\\180\\84
\end{pmatrix}
\\
&=
\frac{
-2\cdot 49 + 45 + 71
}{709}
=
\frac{18}{709},
\end{align*}
Das iteriert Spiel $B$ ist also ein Gewinnspiel!
Obwohl die Spiele $A$ und $B$ für sich alleine in der iterierten Form
keine Gewinnspiele sind, ist das kombinierte Spiel, wo man zufällig
die beiden Spiel verbindet immer ein Gewinnspiel.

Man kann statt des Spiels $B$ auch das modifizierte Spiel $\tilde{B}$ 
verwenden, welches für kleine $\varepsilon>0$ ein Verlustspiel ist.
Die Analyse lässt sich in der gleichen Weise durchführen und liefert
wieder, dass für nicht zu grosses $\varepsilon$ das kombinierte Spiel
ein Gewinnspiel ist.





