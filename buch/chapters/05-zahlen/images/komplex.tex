%
% komplex.tex -- Betrag und Argument einer komplexen Zahl
%
% (c) 2021 Prof Dr Andreas Müller, OST Ostschweizer Fachhochschule
%
\documentclass[tikz]{standalone}
\usepackage{amsmath}
\usepackage{times}
\usepackage{txfonts}
\usepackage{pgfplots}
\usepackage{csvsimple}
\usetikzlibrary{arrows,intersections,math}
\begin{document}
\def\skala{1.5}
\begin{tikzpicture}[>=latex,thick,scale=\skala]

\pgfmathparse{atan(2/3)}
\xdef\winkel{\pgfmathresult}
\fill[color=blue!20] (0,0) -- (1.5,0) arc (0:\winkel:1.5) -- cycle;
\draw[->] (-1,0) -- (4,0) coordinate[label={$\Re z$}];
\draw[->] (0,-1) -- (0,3) coordinate[label={right:$\Im z$}];
\draw[line width=0.5pt] (3,0) -- (3,2);
\node at (3,1) [right] {$\Im z=b$};
\node at (1.5,0) [below] {$\Re z=a$};
\draw[->,color=red,line width=1.4pt] (0,0) -- (3,2);
\node at (3,2) [above right] {$z=a+bi$};
\def\punkt#1{
	\fill[color=white] #1 circle[radius=0.04];
	\draw #1 circle[radius=0.04];
}
\punkt{(0,0)}
\punkt{(3,2)}
\node[color=red] at (1.5,1) [rotate=\winkel,above] {$r=|z|$};
\node[color=blue] at ({\winkel/2}:1.0)
	[rotate={\winkel/2}] {$\varphi=\operatorname{arg}z$};

\end{tikzpicture}
\end{document}

