%
% rational.tex -- rationale Zahlen
%
% (c) 2021 Prof Dr Andreas Müller, OST Ostschweizer Fachhochschule
%
\section{Rationale Zahlen
\label{buch:section:rationale-zahlen}}
\rhead{Rationale Zahlen}
In den ganzen Zahlen sind immer noch nicht alle linearen Gleichungen
lösbar, es gibt keine ganze Zahl $x$ mit $3x=1$.
Die nötige Erweiterung der ganzen Zahlen lernen Kinder noch bevor sie
die negativen Zahlen kennenlernen.

\subsubsection{Brüche}
Rationale Zahlen sind Paare von ganzen Zahlen $a$ und $b\ne 0$,
die in der speziellen Schreibweise $\frac{a}{b}$ dargestellt werden.
Die Rechenregeln für Addition und Multiplikation sind
\begin{align*}
\frac{a}{b}+\frac{c}{d}
&=
\frac{ad+bc}{bd},
\\
\frac{a}{b}\cdot\frac{c}{d}
&=
\frac{ac}{bd}.
\end{align*}
Die speziellen Brüche $\frac{0}{b}$ und $\frac{1}{1}$ erfüllen die
Regeln
\begin{align*}
\frac{a}{b}+\frac{0}{d} &= \frac{ad}{bd}
\\
\frac{a}{b}\cdot \frac{0}{c} &= \frac{0}{bc}
\\
\frac{a}{b}\cdot \frac{1}{1} &= \frac{a}{b}.
\end{align*}
Wir sind uns gewohnt, die Brüche $\frac{0}{b}$ mit der Zahl $0$ und
$\frac{1}{1}$ mit der Zahl $1$ zu identifizieren.

\subsubsection{Kürzen}
Wie bei den ganzen Zahlen entstehen durch die Rechenregeln viele Brüche,
denen wir den gleichen Wert zuordnen möchten
Zum Beispiel folgt
\[
\frac{ac}{bc} - \frac{a}{b} 
=
\frac{abc-abc}{b^2c}
=
\frac{0}{b^2c},
\]
wir müssen also die beiden Brüche als gleichwertig betrachten.
Allgemein gelten die zwei Brüche $\frac{a}{b}$ und $\frac{c}{d}$
als äquivalent, wenn $ad-bc= 0$ gilt.

Die Definition bestätigt, dass die beiden Brüche
\[
\frac{ac}{bc} 
\qquad\text{und}\qquad
\frac{a}{b}
\]
als gleichwertig zu betrachten sind.
Der Übergang von links nach rechts heisst {\em Kürzen},
\index{Kürzen}%
der Übergang von rechts nach links heisst {\em Erweitern}.
\index{Erweitern}%
Eine rationale Zahl ist also eine Menge von Brüchen, die durch
Kürzen und Erweitern ineinander übergeführt werden können.

Die Menge der Äquivalenzklassen von Brüchen ist die Menge $\mathbb{Q}$
der rationalen Zahlen.
In $\mathbb{Q}$ sind Addition, Subtraktion und Multiplikation mit den
gewohnten Rechenregeln, die bereits in $\mathbb{Z}$ gegolten haben,
uneingeschränkt möglich.

\subsubsection{Kehrwert}
Zu jedem Bruch $\frac{a}{b}$ lässt sich der Bruch $\frac{b}{a}$,
der sogenannte {\em Kehrwert}
\index{Kehrwert}
konstruieren.
Er hat die Eigenschaft, dass
\[
\frac{a}{b}\cdot\frac{b}{a}
=
\frac{ab}{ba}
=
1
\]
gilt.
Der Kehrwert ist also das multiplikative Inverse, jede von $0$ verschiedene
rationale Zahl hat eine Inverse.

\subsubsection{Lösung von linearen Gleichungen}
Mit dem Kehrwert lässt sich jetzt jede lineare Gleichung lösen.
Die Gleichung $ax=b$ hat die Lösung
\[
ax = \frac{a}{1} \frac{u}{v} = \frac{b}{1}
\qquad\Rightarrow\qquad
\frac{1}{a}
 \frac{a}{1} \frac{u}{v} = \frac{1}{a}\frac{b}{1} 
\qquad\Rightarrow\qquad
\frac{u}{v} = \frac{b}{a}.
\]
Dasselbe gilt auch für rationale Koeffizienten $a$ und $b$.
In der Menge $\mathbb{Q}$ kann man also beliebige lineare Gleichungen
lösen.

\subsubsection{Körper}
$\mathbb{Q}$ ist ein Beispiel für einen sogenannten {\em Körper}, 
in dem die arithmetischen Operationen Addition, Subtraktion, Multiplikation
und Division möglich sind mit der einzigen Einschränkung, dass nicht durch
$0$ dividiert werden kann.
Körper sind die natürliche Bühne für die lineare Algebra, da sich lineare
Gleichungssysteme ausschliesslich mit den Grundoperation lösen lassen.

