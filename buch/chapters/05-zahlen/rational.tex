%
% rational.tex -- rationale Zahlen
%
% (c) 2021 Prof Dr Andreas Müller, OST Ostschweizer Fachhochschule
%
% !TeX spellcheck = de_CH
\section{Rationale Zahlen
\label{buch:section:rationale-zahlen}}
\rhead{Rationale Zahlen}
In den ganzen Zahlen sind immer noch nicht alle linearen Gleichungen
lösbar, es gibt keine ganze Zahl $x$ mit $3x=1$.
Die nötige Erweiterung der ganzen Zahlen lernen Kinder noch bevor sie
die negativen Zahlen kennenlernen.

Wir können hierbei denselben Trick anwenden,
wie schon beim Übergang von den natürlichen zu den ganzen Zahlen.
Wir kreieren wieder Paare $(z, n)$, deren Elemente wir \emph{Zähler} und
\emph{Nenner} nennen, wobei $z, n \in \mathbb Z$ und zudem $n \ne 0$.
Die Rechenregeln für Addition und Multiplikation lauten
\[
(a, b) + (c, d)
=
(ad + bc, bd)
\qquad \text{und} \qquad
(a, b) \cdot (c, d)
=
(ac, bd)
.
\]
Die ganzen Zahlen $z\in\mathbb{Z}$ lassen sich in dieser Darstellung als
$z \mapsto (z, 1)$ in diese Menge von Paaren einbetten.

Ähnlich wie schon bei den ganzen Zahlen ist diese Darstellung
aber nicht eindeutig.
Zwei Paare sind äquivalent, wenn sich deren beide Elemente um denselben Faktor
unterscheiden,
\[
(a, b)
\sim
(c, d)
\quad \Leftrightarrow \quad
\exists \lambda \in \mathbb Z \colon
\lambda a = c
\wedge
\lambda b = d
.
\]
Dass es sich hierbei wieder um eine Äquivalenzrelation handelt, lässt sich
einfach nachprüfen.

Durch die neuen Regen gibt es nun zu jedem Paar $(a, b)$ mit $a \ne 0$
ein Inverses $(b, a)$ bezüglich der Multiplikation,
wie man anhand der folgenden Rechnung sieht,
\[
(a, b) \cdot (b, a)
=
(a \cdot b, b \cdot a)
=
(a \cdot b, a \cdot b)
\sim
(1, 1)
.
\]

\subsubsection{Brüche}
Rationale Zahlen sind genau die Äquivalenzklassen dieser Paare $(a, b)$ von
ganzen Zahlen $a$ und $b\ne 0$.
Da diese Schreibweise recht unhandlich ist, wird normalerweise die Notation
als Bruch $\frac{a}{b}$ verwendet.
\index{Bruch}%
Die Rechenregeln werden dadurch zu den wohlvertrauten
\[
\frac{a}{b}+\frac{c}{d}
=
\frac{ad+bc}{bd},
\qquad\text{und}\qquad
\frac{a}{b}\cdot\frac{c}{d}
=
\frac{ac}{bd}
\]
und die speziellen Brüche $\frac{0}{b}$ und $\frac{1}{1}$ erfüllen die
Regeln
\[
\frac{a}{b}+\frac{0}{d} = \frac{ad}{bd} \sim \frac{a}{b},
\qquad
\frac{a}{b}\cdot \frac{0}{c} = \frac{0}{bc}
\qquad\text{und}\qquad
\frac{a}{b}\cdot \frac{1}{1} = \frac{a}{b}.
\]
Wir sind uns gewohnt, die Brüche $\frac{0}{b}$ mit der Zahl $0$ und
$\frac{1}{1}$ mit der Zahl $1$ zu identifizieren.

\subsubsection{Kürzen}
Wie bei den ganzen Zahlen entstehen durch die Rechenregeln viele Brüche,
denen wir den gleichen Wert zuordnen möchten.
Zum Beispiel folgt
\[
\frac{ac}{bc} - \frac{a}{b} 
=
\frac{abc-abc}{b^2c}
=
\frac{0}{b^2c},
\]
wir müssen also die beiden Brüche als gleichwertig betrachten.
Allgemein gelten die zwei Brüche $\frac{a}{b}$ und $\frac{c}{d}$
als äquivalent, wenn $ad-bc= 0$ gilt.
Dies ist gleichbedeutend mit der früher definierten Äquivalenzrelation
und bestätigt, dass die beiden Brüche
\[
\frac{ac}{bc} 
\qquad\text{und}\qquad
\frac{a}{b}
\]
als gleichwertig zu betrachten sind.
Der Übergang von links nach rechts heisst {\em Kürzen},
\index{Kürzen}%
der Übergang von rechts nach links heisst {\em Erweitern}.
\index{Erweitern}%
Eine rationale Zahl ist also eine Menge von Brüchen, die durch
Kürzen und Erweitern ineinander übergeführt werden können.

Die Menge der Äquivalenzklassen von Brüchen ist die Menge $\mathbb{Q}$
der rationalen Zahlen.
\index{Q@$\mathbb{Q}$}%
In $\mathbb{Q}$ sind Addition, Subtraktion und Multiplikation mit den
gewohnten Rechenregeln, die bereits in $\mathbb{Z}$ gegolten haben,
uneingeschränkt möglich.

\subsubsection{Kehrwert}
Zu jedem Bruch $\frac{a}{b}$ lässt sich der Bruch $\frac{b}{a}$,
der sogenannte {\em Kehrwert}
\index{Kehrwert}%
konstruieren.
Er hat die Eigenschaft, dass
\[
\frac{a}{b}\cdot\frac{b}{a}
=
\frac{ab}{ba}
=
1
\]
gilt.
Der Kehrwert ist also das multiplikative Inverse, jede von $0$ verschiedene
rationale Zahl hat eine solche Inverse.

\subsubsection{Lösung von linearen Gleichungen}
Mit dem Kehrwert lässt sich jetzt jede lineare Gleichung lösen.
\index{lineares Gleichungssystem}%
Die Gleichung $ax=b$ hat die Lösung
\[
ax = \frac{a}{1} \frac{u}{v} = \frac{b}{1}
\qquad\Rightarrow\qquad
\frac{1}{a}
 \frac{a}{1} \frac{u}{v} = \frac{1}{a}\frac{b}{1} 
\qquad\Rightarrow\qquad
\frac{u}{v} = \frac{b}{a}.
\]
Dasselbe gilt auch für rationale Koeffizienten $a$ und $b$.
In der Menge $\mathbb{Q}$ kann man also beliebige lineare Gleichungen
lösen.

\subsubsection{Körper}
$\mathbb{Q}$ ist ein Beispiel für einen sogenannten {\em Körper}, 
\index{Körper}%
in dem die arithmetischen Operationen Addition, Subtraktion, Multiplikation
und Division möglich sind mit der einzigen Einschränkung, dass nicht durch
$0$ dividiert werden kann.
Körper sind die natürliche Bühne für die lineare Algebra, da sich lineare
Gleichungssysteme ausschliesslich mit den Grundoperation lösen lassen.
Eine formelle Definition eines Körpers werden wir in 
Abschnitt~\ref{buch:subsection:koerper} geben.

Wir werden im Folgenden für verschiedene Anwendungszwecke weitere Körper
konstruieren, zum Beispiel die reellen Zahlen $\mathbb{R}$ und die
rationalen Zahlen $\mathbb{C}$.
Wann immer die Wahl des Körpers keine Rolle spielt, werden wir den
Körper mit $\Bbbk$ bezeichnen.
\index{k@$\Bbbk$}%

Ein Körper $\Bbbk$ zeichnet sich dadurch aus, dass alle ELemente ausser $0$
invertierbar sind. 
Diese wichtige Teilmenge wird mit $\Bbbk^* = \Bbbk \setminus\{0\}$ mit
bezeichnet.
\label{buch:zahlen:def:bbbk*}
In dieser Relation sind beliebige Multiplikationen ausführbar, das Element
$1\in\Bbbk^*$ ist neutrales Element bezüglich der Multiplikation.
Die Menge $\Bbbk^*$ trägt die Struktur einer Gruppe, siehe dazu auch
den Abschnitt~\ref{buch:grundlagen:subsection:gruppen}.


