%
% ganz.tex -- Ganze Zahlen
%
% (c) 2021 Prof Dr Andreas Müller, OST Ostschweizer Fachhochschule
%
% !TeX spellcheck = de_CH
\section{Ganze Zahlen
\label{buch:section:ganze-zahlen}}
Die Menge der ganzen Zahlen löst das Problem, dass nicht jede
Gleichung der Form $x+a=b$ mit $a, b \in \mathbb N$
eine Lösung $x \in \mathbb N$ hat.
Dazu ist erforderlich, den natürlichen Zahlen die negativen Zahlen
hinzuzufügen, also wieder die Existenz neuer Objekte zu postulieren,
die die Rechenregeln weiterhin erfüllen.

\rhead{Ganze Zahlen}
\subsubsection{Paare von natürlichen Zahlen}
Die ganzen Zahlen können konstruiert werden als Paare $(u,v)$ von 
natürlichen Zahlen $u,v\in\mathbb{N}$.
Die Paare der Form $(u,0)$ entsprechen den natürlichen Zahlen, die
Paare $(0,v)$ sind die negativen Zahlen.
Die Rechenoperationen sind wie folgt definiert:
\begin{equation}
\begin{aligned}
(a,b)+(u,v) &= (a+u,b+v)
\\
(a,b)\cdot (u,v) &= (au+bv,av+bu)
\end{aligned}
\label{buch:zahlen:ganze-rechenregeln}
\end{equation}
Die Darstellung ganzer Zahlen als Paare von natürlichen Zahlen
findet man auch in der Buchhaltung, wo man statt eines Vorzeichen
{\em Soll} und {\em Haben} verwendet.
\index{Soll und Haben}%
Dabei kommt es nur auf die Differenz der beiden Positionen an.
Fügt man beiden Positionen den gleichen Betrag hinzu, ändert sich
nichts.
Viele der Paare $(a,b)$ müssen also als äquivalent angesehen
werden.

\subsubsection{Äquivalenzrelation}
Die Definition~\eqref{buch:zahlen:ganze-rechenregeln}
erzeugt neue Paare, die wir noch nicht interpretieren können.
Zum Beispiel ist $0=1+(-1) = (1,0) + (0,1) = (1,1)$.
Die Paare $(u,u)$ müssen daher alle mit $0$ identifiziert werden.
Es folgt dann auch, dass alle Paare von natürlichen Zahlen mit 
``gleicher Differenz'' den gleichen ganzzahligen Wert darstellen,
allerdings können wir das nicht so formulieren, da ja der Begriff
der Differenz noch gar nicht definiert ist.
Stattdessen gelten zwei Paare als äquivalent, wenn
\begin{equation}
(a,b) \sim (c,d)
\qquad\Leftrightarrow\qquad
a+d = c+b
\label{buch:zahlen:ganz-aquivalenz}
\end{equation}
gilt.
Diese Bedingung erhält man, indem man zu $a-b=c-d$ die Summe $b+d$ 
hinzuaddiert.
Ein ganzen Zahl $z$ ist daher eine Menge von Paaren von natürlichen
Zahlen mit der Eigenschaft
\[
(a,b)\in z\;\wedge (a',b')\in z
\qquad\Leftrightarrow\qquad
(a,b)\sim(a',b')
\qquad\Leftrightarrow\qquad
a+b' = a'+b.
\]
Man nennt eine solche Menge eine {\em Äquivalenzklasse} der Relation $\sim$.
\index{Aquivalenzklasse@Äquivalenzklasse}
Die Menge $\mathbb{Z}$ der {\em ganzen Zahlen} ist die Menge aller solchen
\index{ganze Zahlen}%
Äquivalenzklassen.
Die Menge der natürlichen Zahlen $\mathbb{N}$ ist in evidenter Weise
darin eingebettet als die Menge der Äquivalenzklassen von Paaren der
Form $(n,0)$.

\subsubsection{Entgegengesetzter Wert}
Zu jeder ganzen Zahl $z$ dargestellt durch das Paar $(a,b)$ 
stellt das Paar $(b,a)$ eine ganze Zahl dar mit der Eigenschaft
\begin{equation}
z+(b,a)
=
(a,b) + (b+a) = (a+b,a+b) \sim (0,0) = 0
\label{buch:zahlen:eqn:entgegengesetzt}
\end{equation}
dar.
Die von $(b,a)$ dargestellte ganze Zahl wird mit $-z$ bezeichnet,
die Rechnung~\eqref{buch:zahlen:eqn:entgegengesetzt} lässt sich damit
abgekürzt als $z+(-z)=0$ schreiben.
$-z$ heisst der $z$ {\em entgegengesetzte Wert} oder die
\index{entgegengesetzte Zahl}%
{\em entgegengesetzte Zahl} zu $z$.

\subsubsection{Lösung von Gleichungen}
Gleichungen der Form $a=x+b$ können jetzt für beliebige natürliche Zahlen
immer gelöst werden.
Dazu schreibt man $a,b\in\mathbb{N}$ als Paare und sucht die
Lösung in der Form $x=(u,v)$.
Man erhält
\begin{align*}
(a,0) &= (u,v) + (b,0)
\\
(a+b,b) &= (u+b,v)
\end{align*}
Das Paar $(u,v) = (a,b)$ ist eine Lösung, die man normalerweise als
$a-b = (a,0) + (-(b,0)) = (a,0) + (0,b) = (a,b)$ schreibt.

Für ganze Zahlen $a=(a_+,a_-)$ und $b=(b_+,b_-)$ kann man die Gleichung
mit der gleichen Methode lösen, man addiert $-b=(b_-,b_+)$ und bekommt
die Lösung
\[
\begin{aligned}
(a_+,a_-) &= (u,v) + (b_+,b_-)
&
\quad &\Rightarrow \quad
&
(u,v)+(b_+,b_-) + (b_-,b_+)
&=
(a_+,a_-) + (b_-,b_+)
\\
&&
\quad &\Rightarrow \quad
&
(u,v) &= (a_++b_-,a_-+b_+).
\end{aligned}
\]

\subsubsection{Ring}
\index{Ring}%
Die ganzen Zahlen sind ein Beispiel für einen sogenannten {\em Ring},
\index{Ring}%
eine algebraische Struktur, in der Addition, Subtraktion und
Multiplikation definiert sind.
Weitere Beispiele von Ringen werden später vorgestellt,
darunter
der Ring der Polynome $\mathbb{Z}[X]$ in Kapitel~\ref{buch:chapter:polynome}
\index{Polynomring}%
\index{ZX@$\mathbb{Z}[X]$}
und
der Ring der $n\times n$-Matrizen in
\index{Matrizenring}%
Kapitel~\ref{buch:chapter:vektoren-und-matrizen}.
In einem Ring wird nicht verlangt, dass die Multiplikation kommutativ
ist, Matrizenringe zum Beispiel sind meistens nicht kommutativ, selbst
wenn die Matrixelemente Elemente eines kommutativen Rings sind.
$\mathbb{Z}$ ist ein kommutativer Ring, ebenso sind die Polynomringe 
kommutativ.
Die Theorie der nicht kommutativen Ringe ist sehr viel reichhaltiger
und leider auch komplizierter als die kommutative Theorie.
\index{Ring!kommutativ}%





