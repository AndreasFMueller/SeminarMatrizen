%
% reell.tex -- reelle Zahlen
%
% (c) 2021 Prof Dr Andreas Müller, OST Ostschweizer Fachhochschule
%
\section{Reelle Zahlen
\label{buch:section:reelle-zahlen}}
\rhead{Reelle Zahlen}
In den rationalen Zahlen lassen sich algebraische Gleichungen höheren
Grades immer noch nicht lösen.
Dass die Gleichung $x^2=2$ keine rationale Lösung hat, ist schon den
Pythagoräern aufgefallen.
\index{Pythagoräer}
Ziel dieses Abschnitts ist, den Körper $\mathbb{Q}$ zu einem
Körper $\mathbb{R}$ zu erweitern, in dem die Gleichung
gelöst werden kann, ohne dabei die Ordnungsrelation zu zerstören, die
die hilfreiche und anschauliche Vorstellung der Zahlengeraden
liefert.
\index{Zahlengerade}%

\subsubsection{Intervallschachtelung}
Die geometrische Intuition der Zahlengeraden führt uns dazu, nach
Zahlen zu suchen, die gute Approximationen für $\sqrt{2}$ sind.
Wir können zwar keinen Bruch angeben, dessen Quadrat $2$ ist, aber
wenn es eine Zahl $\sqrt{2}$ mit dieser Eigenschaft gibt, dann können
wir dank der Ordnungsrelation feststellen, dass sie in all den folgenden,
kleiner werdenden Intervallen
\[
\biggl[1,\frac32\biggr],\;
\biggl[\frac75,\frac{17}{12}\biggr],\;
\biggl[\frac{41}{29},\frac{99}{70}\biggr],\;
\biggl[\frac{239}{169},\frac{577}{408}\biggr],\;
\dots
\]
enthalten sein muss\footnote{Die Näherungsbrüche konvergieren sehr
schnell, sie sind mit der sogenannten Kettenbruchentwicklung der
Zahl $\sqrt{2}$ gewonnen worden.}.
Jedes der Intervalle enthält auch das nachfolgende Intervall, und
die intervalllänge konvergiert gegen 0.
Eine solche \emph{Intervallschachtelung} beschreibt also genau eine ``Zahl'',
\index{Intervallschachtelung}%
aber möglicherweise keine, die sich als Bruch schreiben lässt.

\subsubsection{Reelle Zahlen als Folgengrenzwerte}
Mit einer Intervallschachtelung lässt sich $\sqrt{2}$ zwar festlegen,
noch einfacher wäre aber eine Folge von rationalen Zahlen $a_n\in\mathbb{Q}$
derart, die $\sqrt{2}$ beliebig genau approximiert.
In der Analysis definiert man zu diesem Zweck, dass $a$ der Grenzwert
einer Folge $(a_n)_{n\in\mathbb{N}}$ ist, wenn es zu jedem $\varepsilon > 0$
ein $N$ gibt derart, dass $|a_n-a|<\varepsilon$ für $n>N$ ist.
Das Problem dieser wohlbekannten Definition für die Konstruktion
reeller Zahle ist, dass im Falle der Folge
\[
(a_n)_{n\in\mathbb{N}}=
\biggl(1,
\frac75,
\frac{41}{29},
\frac{239}{169},\dots\biggr) \to a=\sqrt{2}
\]
das Objekt $a$ noch gar nicht existiert.
Es gibt keine rationale Zahl, die als Grenzwert dieser Folge dienen
könnte.

Folgen, die gegen Werte in $\mathbb{Q}$ konvergieren sind dagegen
nicht in der Lage, neue Zahlen zu approximieren.
Wir müssen also auszudrücken versuchen, dass eine Folge konvergiert,
ohne den zugehörigen Grenzwert zu kennen.

\subsubsection{Cauchy-Folgen}
Die Menge $\mathbb{R}$ der reellen Zahlen kann man auch als Menge
aller Cauchy-Folgen $(a_n)_{n\in\mathbb{N}}$, $a_n\in\mathbb{Q}$,
betrachten.
\index{Cauchy-Folge}%
Eine Folge ist eine {\em Cauchy-Folge}, wenn es für jedes $\varepsilon>0$
eine Zahl $N(\varepsilon)$ gibt derart, dass $|a_n-a_m|<\varepsilon$
für $n,m>N(\varepsilon)$.
Ab einer geeigneten Stelle $N(\varepsilon)$ sind die Folgenglieder also
mit Genauigkeit $\varepsilon$ nicht mehr unterscheidbar.


\subsubsection{Relle Zahlen als Äquivalenzklassen von Cauchy-Folgen}
Nicht jede Cauchy-Folge hat eine rationale Zahl als Grenzwert.
Da wir für solche Folgen noch keine Zahlen als Grenzwerte haben,
nehmen wir die Folge als eine mögliche Darstellung der Zahl.
Die Folge kann man ja auch verstehen als eine Vorschrift, wie man
Approximationen der Zahl berechnen kann.

Zwei verschiedene Cauchy-Folgen $(a_n)_{n\in\mathbb{N}}$ und
$(b_n)_{n\in\mathbb{N}}$ 
können den gleichen Grenzwert haben.
So sind 
\[
\begin{aligned}
a_n&\colon&&
1,\frac32,\frac75,\frac{17}{12},\frac{41}{29},\frac{99}{70},\frac{239}{169},
\frac{577}{408},\dots
\\
b_n&\colon&&
1,1.4,1.41,1.412,1.4142,1.41421,1.414213,1.4142135,\dots
\end{aligned}
\]
beide Folgen, die die Zahl $\sqrt{2}$ approximieren.
Im Allgemeinen tritt dieser Fall ein, wenn $|a_n-b_n|$ eine
Folge mit Grenzwert $0$ oder {\em Nullfolge} ist.
\index{Nullfolge}%
Eine reelle Zahl ist also die Menge aller rationalen Cauchy-Folgen,
deren Differenzen Nullfolgen sind.

Die Menge $\mathbb{R}$ der reellen Zahlen kann man also ansehen
als bestehend aus Äquivalenzklassen von Folgen, die alle den gleichen
Grenzwert haben.
Die Rechenregeln der Analysis 
\[
\lim_{n\to\infty} (a_n + b_n)
=
\lim_{n\to\infty} a_n +
\lim_{n\to\infty} b_n
\qquad\text{und}\qquad
\lim_{n\to\infty} a_n \cdot b_n
=
\lim_{n\to\infty} a_n \cdot
\lim_{n\to\infty} b_n 
\]
stellen sicher, dass sich die Rechenoperationen von den rationalen
Zahlen auf die reellen Zahlen übertragen lassen.




