%
% natuerlich.tex
%
% (c) 2021 Prof Dr Andreas Müller, OST Ostschweizer Fachhochschule
%
% !TeX spellcheck = de_CH
\section{Natürlich Zahlen
\label{buch:section:natuerliche-zahlen}}
\rhead{Natürliche Zahlen}
Die natürlichen Zahlen sind die Zahlen, mit denen wir zählen.
\index{natürliche Zahlen}%
\index{$\mathbb{N}$}%
Sie abstrahieren das Konzept der Anzahl der Elemente einer endlichen
Menge.
Da die leere Menge keine Elemente hat, muss die Menge der natürlichen
Zahlen auch die Zahl $0$ enthalten.
Wir schreiben
\[
\mathbb{N}
=
\{
0,1,2,3,\dots
\}.
\]

\subsubsection{Peano-Axiome}
Man kann den Zählprozess durch die folgenden Axiome von Peano beschreiben:
\index{Peano-Axiome}%
\begin{enumerate}
\item $0\in\mathbb N$.
\item Jede Zahl $n\in \mathbb{N}$ hat einen {\em Nachfolger}
$n'\in \mathbb{N}$.
\index{Nachfolger}%
\item $0$ ist nicht Nachfolger einer Zahl.
\item Wenn zwei Zahlen $n,m\in\mathbb{N}$ den gleichen Nachfolger haben,
$n'=m'$, dann sind sie gleich $n=m$.
\item Enthält eine Menge $X$ die Zahl $0$ und mit jeder Zahl auch ihren
Nachfolger, dann ist $\mathbb{N}\subset X$.
\end{enumerate}

\subsubsection{Vollständige Induktion}
Es letzte Axiom formuliert das Prinzip der vollständigen Induktion.
Um eine Aussage $P(n)$ für alle natürlichen Zahlen $n$
mit vollständiger Induktion zu beweisen, bezeichnet man mit
$X$ die Menge aller Zahlen, für die $P(n)$ wahr ist.
Die Induktionsverankerung beweist, dass $P(0)$ wahr ist, dass also $0\in X$.
Der Induktionsschritt beweist, dass mit einer Zahl $n\in X$ auch der
Nachfolger $n'\in X$ ist.
Nach dem letzten Axiom ist $\mathbb{N}\subset X$, oder anders ausgedrückt,
die Aussage $P(n)$ ist wahr für jede natürliche Zahl.

\subsubsection{Addition}
Aus der Nachfolgereigenschaft lässt sich durch wiederholte Anwendung
die vertrautere Addition konstruieren.
\index{Addition!in $\mathbb{N}$}%
Um die Zahl $n\in\mathbb{N}$ um $m\in\mathbb{N}$ zu vermehren, also
$n+m$ auszurechnen, kann man rekursive Regeln
\begin{align*}
n+0&=n\\
n+m'&=(n+m)'
\end{align*}
festlegen.
Nach diesen Regeln ist
\[
5+3
=
5+2'
=
(5+2)'
=
(5+1')'
=
((5+1)')'
=
((5+0')')'
=
(((5)')')'.
\]
Dies ist genau die Art und Weise, wie kleine Kinder Rechnen lernen.
Sie Zählen von $5$ ausgehend um $3$ weiter.
Der dritte Nachfolger von $5$ heisst üblicherweise $8$.

Die algebraische Struktur, die hier konstruiert worden ist, heisst
eine Halbgruppe.
Allerdings kann man darin zum Beispiel nur selten Gleichungen
lösen, zum Beispiel hat $3+x=1$ keine Lösung.
Die Addition ist nicht immer umkehrbar.

\subsubsection{Multiplikation}
Es ist klar, dass auch die Multiplikation definiert werden kann, 
sobald die Addition definiert ist.
Die Rekursionsformeln
\begin{align}
n\cdot 0 &= 0 \notag \\
n\cdot m' &= n\cdot m + n
\label{buch:zahlen:multiplikation-rekursion}
\end{align}
legen jedes Produkt von natürlichen Zahlen fest, zum Beispiel
\[
5\cdot 3
=
5\cdot 2'
=
5\cdot 2 + 5
=
5\cdot 1' + 5
=
5\cdot 1 + 5 + 5
=
5\cdot 0' + 5 + 5
=
5\cdot 0 + 5 + 5 + 5
=
5 + 5 + 5.
\]
Doch auch bezüglich der Multiplikation ist $\mathbb{N}$ unvollständig,
die Beispielgleichung $3x=1$ hat keine Lösung in $\mathbb{N}$.

\subsubsection{Rechenregeln}
Aus den Definitionen lassen sich auch die Rechenregeln ableiten,
die man für die alltägliche Rechnung braucht.
Zum Beispiel kommt es nicht auf die Reihenfolge der Summanden
oder Faktoren an. 
Das {\em Kommutativgesetz} besagt
\[
a+b=b+a
\qquad\text{und}\qquad
a\cdot b = b\cdot a.
\]
\index{Kommutativgesetz}%
Die Kommutativität der Addition werden wir auch in allen weiteren
Konstruktionen voraussetzen.
Die Kommutativität des Produktes ist allerdings weniger selbstverständlich
und wird beim Matrizenprodukt nur noch für spezielle Faktoren zutreffen.

Eine Summe oder ein Produkt mit mehr als zwei Summanden bzw.~Faktoren
kann in jeder beliebigen Reihenfolge ausgewertet werden,
\[
(a+b)+c
=
a+(b+c)
\qquad\text{und}\qquad
(a\cdot b)\cdot c
=
a\cdot (b\cdot c)
\]
dies ist das Assoziativgesetz.
Es gestattet auch eine solche Summe oder ein solches Produkt einfach
als $a+b+c$ bzw.~$a\cdot b\cdot c$ zu schreiben, da es ja keine Rolle
spielt, in welcher Reihenfolge man die Teilprodukte berechnet.

Die Konstruktion der Multiplikation als iterierte Addition mit Hilfe
der Rekursionsformel \eqref{buch:zahlen:multiplikation-rekursion}
hat auch zur Folge, dass die {\em Distributivgesetze}
\[
a\cdot(b+c) = ab+ac
\qquad\text{und}\qquad
(a+b)c = ac+bc
\]
gelten.
Bei einem nicht-kommutativen Produkt ist es hierbei notwendig,
zwischen Links- und Rechts-Distributivgesetz zu unterscheiden.

Die Distributivgesetze drücken die wohlbekannte Regel des
Ausmultiplizierens aus.
Ein Distributivgesetz ist also grundlegend dafür, dass man mit den
Objekten so rechnen kann, wie man das in der elementaren Algebra 
gelernt hat.
Auch die Distributivgesetze sind daher Rechenregeln, die wir in
Zukunft immer dann fordern werden, wenn Addition und Multiplikation
definiert sind.
Sie gelten immer für Matrizen.

\subsubsection{Teilbarkeit}
Die Lösbarkeit von Gleichungen der Form $ax=b$ mit $a,b\in\mathbb{N}$
gibt Anlass zum sehr nützlichen Konzept der Teilbarkeit.
\index{Teilbarkeit}%
Die Zahl $b$ heisst teilbar durch $a$, wenn die Gleichung $ax=b$ eine
Lösung in $\mathbb{N}$ hat.
\index{teilbar}%
Jede natürlich Zahl $n$ ist durch $1$ und durch sich selbst teilbar,
denn $n\cdot 1 = n$.
Andere Teiler sind dagegen nicht selbstverständlich.
Die Zahlen
\[
\mathbb{P}
=
\{2,3,5,7,11,13,17,19,23,29,\dots\}
\]
haben keine weiteren Teiler. Sie heissen {\em Primzahlen}.
\index{Primzahl}%
Die Menge der natürlichen Zahlen ist die naheliegende Arena
für die Zahlentheorie.
\index{Zahlentheorie}%

\subsubsection{Konstruktion der natürlichen Zahlen aus der Mengenlehre}
Die Peano-Axiome postulieren, dass es natürliche Zahlen gibt.
Es werden keine Anstrengungen unternommen, die natürlichen Zahlen
aus noch grundlegenderen mathematischen Objekten zu konstruieren.
Die Mengenlehre bietet eine solche Möglichkeit.

Da die natürlichen Zahlen das Konzept der Anzahl der Elemente einer
Menge abstrahieren, gehört die leere Menge zur Zahl $0$.
Die Zahl $0$ kann also durch die leere Menge $\emptyset = \{\}$
wiedergegeben werden.

Der Nachfolger muss jetzt als eine Menge mit einem Element konstruiert
werden.
Das einzige mit Sicherheit existierende Objekt, das für diese Menge
zur Verfügung steht, ist $\emptyset$.
Zur Zahl $1$ gehört daher die Menge $\{\emptyset\}$, eine Menge mit
genau einem Element.
Stellt die Menge $N$ die Zahl $n$ dar, dann können wir die zu $n+1$
gehörige Menge $N'$ dadurch konstruieren, dass wir zu den Elemente
von $N$ ein zusätzliches Element hinzufügen, das noch nicht in $N$ ist,
zum Beispiel $\{N\}$:
\[
N' = N \cup \{ N \}.
\]

Die natürlichen Zahlen existieren also, wenn wir akzeptieren, dass es
Mengen gibt.
Die natürlichen Zahlen sind dann nacheinander die Mengen
\begin{align*}
0 &= \emptyset 
\\
1 &= 0 \cup \{0\} = \emptyset \cup \{0\} = \{0\}
\\
2 &= 1 \cup \{1\} = \{0\}\cup\{1\} = \{0,1\}
\\
3 &= 2 \cup \{2\} = \{0,1\}\cup \{2\} = \{0,1,2\}
\\
&\phantom{n}\vdots
\\
n+1&= n \cup \{n\} = \{0,\dots,n-1\} \cup \{n\} = \{0,1,\dots,n\}
\\
&\phantom{n}\vdots
\end{align*}





