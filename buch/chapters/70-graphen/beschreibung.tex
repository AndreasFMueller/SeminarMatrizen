%
% beschreibung.tex -- Beschreibung von Graphen mit Matrizen
%
% (c) 2020 Prof Dr Andreas Müller, Hochschule Rapperswil
%
\section{Beschreibung von Graphen mit Matrizen
\label{buch:section:beschreibung-von-graphen-mit-matrizen}}
\rhead{Beschreibung mit Matrizen}
Ein Graph ist eine Menge von Knoten, die untereinander mit Kanten
verbunden sind.
Graphen können zum Beispiel verwendet werden um Netzwerke zu beschreiben,
aber auch viele andere Datenstrukturen.
Die Knoten können einzelne Objekte beschreiben, die Kanten beschreiben
dann Beziehungen zwischen diesen Objekten.

\subsection{Definition von Graphen
\label{subsection:definition-von-graphen}}
In der Einleitung zu diesem Abschnitt wurde bereits eine informelle
Beschreibung des Konzeptes eines Graphen gegeben.
Um zu einer Beschreibung mit Hilfe von Matrizen kommen können,
wird eine exakte Definition benötigt.
Dabei werden sich einige Feinheiten zeigen, die für Anwendungen wichtig
sind und sich in Unterschieden in der Definition der zugehörigen Matrix 
äussern.

\subsubsection{Ungerichtete Graphen}
Die Grundlage für alle Arten von Graphen ist eine Menge $V$ von {\em Knoten},
auch {\em Vertices} genannt.
\index{Knoten}%
\index{Vertex}%
Die Unterschiede zeigen sich in der Art und Weise, wie die Knoten
mit sogenannten die Kanten 
\index{Kante}%
verbunden werden.
Bei einen ungerichteten Graphen sind die beiden Endpunkte einer Kante
gleichwertig, es gibt keine bevorzugte Reihenfolge oder Richtung der
Kante.
Eine Kante wird daher vollständig spezifiziert, wenn wir die
Menge der Endpunkte kennen.
Dies führt auf die folgende Definition eines ungerichteten Graphen.

\begin{definition}
\label{buch:def:ungerichteter-graph}
\index{Graph!ungerichteter}%
\index{ungerichteter Graph}%
Ein {\em ungerichteter Graph} ist eine endliche Menge $V$ von {\em Knoten}
und eine Menge $E$ von zweielementigen Teilmengen 
\[
E \subset \{\, \{a,b\}\subset V\,|\, a\ne b\}.
\]
Die Elemente von $E$ heissen {\em Kanten} ({\em edges}).
\end{definition}

Man beachte, dass es keine Kante gibt, die einen Knoten $a\in V$
mit sich selbst verbindet, da die zugehörige Menge $\{a,a\}=\{a\}$
nicht aus zwei verschiedenen Elementen besteht, wie die
Definition~\ref{buch:def:ungerichteter-graph} dies verlangt.

Ein elektrisches Netzwerk von ohmschen Widerständen kann mit Hilfe
eines ungerichteten Graphen beschrieben werden.
Ohmsche Widerstände hängen nicht von der Richtung des Stromflusses
durch die Widerstände ab.
Will man Spannungen und Ströme in einem solchen Netzwerk berechnen,
ist auch das Fehlen von Schleifen, die von $a$ zu $a$ führen, kein
Verlust.
Die Endpunkte solcher Widerstände wären immer auf gleichem Potential,
es würde daher kein Strom fliessen, sie haben daher keinen Einfluss auf
das Verhalten des Netzwerkes und können weggelassen werden.

\subsubsection{Gerichtete Graphen}
In vielen Anwendungen sind die Endpunkte einer Kante nicht austauschbar.
In einem Strassennetz sind Einbahnstrassen nicht in beiden Richtungen
befahrbar.
Anfangs- und Endpunkt einer Kante müssen in einem solche Graphen
unterschieden werden.
Eine zweielementige Menge ist daher nicht mehr eine geeignete Abstraktion
für die Kante, ein (geordnetes) Paar von Vertizes passt besser.

\begin{definition}
\label{buch:def:gerichteter-graph}
\index{Graph!gerichteter}%
\index{gerichteter Graph}%
Ein {\em gerichteter Graph} ist eine endliche Menge $V$ von Knoten
und eine Menge $E \subset V\times V$ von gerichteten Kanten.
Ausserdem gibt es zwei Abbildungen
\[
\begin{aligned}
a&\colon E\to V: (p,q) \mapsto a((p,q)) = p
\\
e&\colon E\to V: (p,q) \mapsto e((p,q)) = q.
\end{aligned}
\]
Der Knoten $a(k)$ heisst der {\em Anfangspunkt} der Kante $k\in E$,
$e(k)$ heisst der {\em Endpunkt}.
\end{definition}

In einem gerichteten Graphen gehört also zu jeder Kante auch eine Richtung
und die Unterscheidung von Anfangs- und Endpunkt einer Kante ist sinnvoll
geworden.
Ausderdem ist eine Kante $(a,a)$ wohldefiniert, also eine Kante, die vom
Knoten $a$ wieder zu $a$ zurückführen.

Man kann einen ungerichteten Graphen in einen gerichteten Graphen
verwandeln, indem wir jede Kante $\{a,b\}$ durch zwei Kanten 
$(a,b)$ und $(b,a)$ ersetzen.
Aus dem ungerichteten Graphen $(V,E)$ mit Knotenmenge $V$ und Kantenmenge
$E$ wird so der gerichtete Graph
$(V,E')$ mit der Kantenmenge
\[
E' 
=
\{
(a,e)
\,|\,
\{a,e\}\in E
\}.
\]
Eine umgekehrte Zuordnung eines ungerichteten Graphen zu einem gerichteten
Graphen ist nicht möglich, da eine ``Schleife'' $(a,a)$ nicht in Kante
des ungerichteten Graphen abgebildet werden kann.

\subsubsection{Beschriftete Graphen}
Bei der Beschreibung eines elektrischen Netzwerkes mit Hilfe eines
ungerichteten Graphen muss jeder Kante zusätzlich ein Widerstandswert
zugeordnet werden.
Dies ist, was eine Beschriftung einer Kante bewerkstelligt.

\begin{definition}
Eine Beschriftung mit Elementen der Menge $L$
eines gerichteten oder ungerichteten Graphen $G=(V,E)$ 
ist eine Abbildung $l\colon E\to L$.
\end{definition}

\subsection{Die Adjazenz-Matrix
\label{subsection:adjazenz-matrix}}

\subsection{Die Laplace-Matrix
\label{subsection:laplace-matrix}}



