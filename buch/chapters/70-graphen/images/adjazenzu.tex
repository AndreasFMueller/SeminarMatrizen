%
% adjazenzu.tex -- Adjazenz-Matrix für einen ungerichten Graphen
%
% (c) 2021 Prof Dr Andreas Müller, OST Ostschweizer Fachhochschule
%
\documentclass[tikz]{standalone}
\usepackage{amsmath}
\usepackage{times}
\usepackage{txfonts}
\usepackage{pgfplots}
\usepackage{csvsimple}
\usetikzlibrary{arrows,intersections,math,calc}
\begin{document}
\def\skala{1}
\begin{tikzpicture}[>=latex,thick,scale=\skala]

\def\r{1.8}

\begin{scope}
\coordinate (A) at ({\r*cos(0*72)},{\r*sin(0*72)});
\coordinate (B) at ({\r*cos(1*72)},{\r*sin(1*72)});
\coordinate (C) at ({\r*cos(2*72)},{\r*sin(2*72)});
\coordinate (D) at ({\r*cos(3*72)},{\r*sin(3*72)});
\coordinate (E) at ({\r*cos(4*72)},{\r*sin(4*72)});

\draw[shorten >= 0.2cm,shorten <= 0.2cm] (A) -- (C);
\draw[color=white,line width=5pt] (B) -- (D);
\draw[shorten >= 0.2cm,shorten <= 0.2cm] (B) -- (D);

\draw[shorten >= 0.2cm,shorten <= 0.2cm] (A) -- (B);
\draw[shorten >= 0.2cm,shorten <= 0.2cm] (B) -- (C);
\draw[shorten >= 0.2cm,shorten <= 0.2cm] (C) -- (D);
\draw[shorten >= 0.2cm,shorten <= 0.2cm] (D) -- (E);
\draw[shorten >= 0.2cm,shorten <= 0.2cm] (E) -- (A);

\draw (A) circle[radius=0.2];
\draw (B) circle[radius=0.2];
\draw (C) circle[radius=0.2];
\draw (D) circle[radius=0.2];
\draw (E) circle[radius=0.2];

\node at (A) {$1$};
\node at (B) {$2$};
\node at (C) {$3$};
\node at (D) {$4$};
\node at (E) {$5$};
\node at (0,0) {$G$};

\node at ($0.5*(A)+0.5*(B)-(0.1,0.1)$) [above right] {$\scriptstyle 1$};
\node at ($0.5*(B)+0.5*(C)+(0.05,-0.07)$) [above left] {$\scriptstyle 2$};
\node at ($0.5*(C)+0.5*(D)+(0.05,0)$) [left] {$\scriptstyle 3$};
\node at ($0.5*(D)+0.5*(E)$) [below] {$\scriptstyle 4$};
\node at ($0.5*(E)+0.5*(A)+(-0.1,0.1)$) [below right] {$\scriptstyle 5$};
\node at ($0.6*(A)+0.4*(C)$) [above] {$\scriptstyle 6$};
\node at ($0.4*(B)+0.6*(D)$) [left] {$\scriptstyle 7$};

\end{scope}

\begin{scope}[xshift=3cm,yshift=-1.1cm]
\node at (0,0) [right] {$\displaystyle
B(G)
=
\begin{pmatrix}
1&0&0&0&1&0&0\\
1&1&0&0&0&1&0\\
0&1&1&0&0&0&1\\
0&0&1&1&0&1&0\\
0&0&0&1&1&0&1
\end{pmatrix}$};
\end{scope}

\begin{scope}[xshift=3cm,yshift=1.1cm]
\node at (0,0) [right] {$\displaystyle
A(G)
=
\begin{pmatrix}
0&1&1&0&1\\
1&0&1&1&0\\
1&1&0&1&0\\
0&1&1&0&1\\
1&0&0&1&0
\end{pmatrix},
\quad
D(G)
=
\begin{pmatrix}
3&0&0&0&0\\
0&3&0&0&0\\
0&0&3&0&0\\
0&0&0&3&0\\
0&0&0&0&2
\end{pmatrix}
$};
\end{scope}

\end{tikzpicture}
\end{document}

