%
% tikztemplate.tex -- template for standalon tikz images
%
% (c) 2021 Prof Dr Andreas Müller, OST Ostschweizer Fachhochschule
%
\documentclass[tikz]{standalone}
\usepackage{amsmath}
\usepackage{mathtools}
\usepackage{times}
\usepackage{txfonts}
\usepackage{pgfplots}
\usepackage{csvsimple}
\usetikzlibrary{arrows,intersections,math}
\begin{document}
\def\skala{1}
\begin{tikzpicture}[>=latex,thick,scale=\skala]

\def\r{3}

\foreach \w in {0,20,...,340}{
	\draw (\w:\r) circle[radius=0.2];
	\draw[->,shorten >= 0.2cm,shorten <= 0.2cm] (\w:\r) -- ({\w+20}:\r);
}

\foreach \x in {1,...,15}{
	\node at ({20*(\x-1)}:\r) {$\scriptstyle \x$};
}
\node at (340:\r) {$\scriptstyle n$};
\node at (320:\r) {$\scriptstyle \dots$};
\node at (300:\r) {$\scriptstyle \dots$};

\begin{scope}[xshift=4cm]
\node at (0,0) [right] {$\displaystyle
L=\begin{pmatrix*}[r]
 2&-1& 0& 0&\dots& 0&-1\\
-1& 2&-1& 0&\dots& 0& 0\\
 0&-1& 2&-1&\dots& 0& 0\\
 0& 0&-1& 2&\dots& 0& 0\\
\vdots&\vdots&\vdots&\vdots&\ddots&\vdots&\vdots\\
 0& 0& 0& 0&\dots& 2&-1\\
-1& 0& 0& 0&\dots&-1& 2
\end{pmatrix*}$};
\end{scope}

\end{tikzpicture}
\end{document}

