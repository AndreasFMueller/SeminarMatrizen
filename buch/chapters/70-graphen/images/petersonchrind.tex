%
% tikztemplate.tex -- template for standalon tikz images
%
% (c) 2021 Prof Dr Andreas Müller, OST Ostschweizer Fachhochschule
%
\documentclass[tikz]{standalone}
\usepackage{amsmath}
\usepackage{times}
\usepackage{txfonts}
\usepackage{pgfplots}
\usepackage{csvsimple}
\usetikzlibrary{arrows,intersections,math}
\begin{document}
\def\skala{1}
\begin{tikzpicture}[>=latex,thick,scale=\skala]

\def\Ra{2}
\def\Ri{1}
\def\e{1.0}
\def\r{0.2}

\begin{scope}[xshift=-3.5cm]

\definecolor{rot}{rgb}{0.8,0,0.8}
\definecolor{gruen}{rgb}{0.2,0.6,0.2}
\definecolor{blau}{rgb}{1,0.6,0.2}

\coordinate (PA) at ({\Ri*sin(0*72)},{\e*\Ri*cos(0*72)});
\coordinate (PB) at ({\Ri*sin(1*72)},{\e*\Ri*cos(1*72)});
\coordinate (PC) at ({\Ri*sin(2*72)},{\e*\Ri*cos(2*72)});
\coordinate (PD) at ({\Ri*sin(3*72)},{\e*\Ri*cos(3*72)});
\coordinate (PE) at ({\Ri*sin(4*72)},{\e*\Ri*cos(4*72)});

\coordinate (QA) at ({\Ra*sin(0*72)},{\e*\Ra*cos(0*72)});
\coordinate (QB) at ({\Ra*sin(1*72)},{\e*\Ra*cos(1*72)});
\coordinate (QC) at ({\Ra*sin(2*72)},{\e*\Ra*cos(2*72)});
\coordinate (QD) at ({\Ra*sin(3*72)},{\e*\Ra*cos(3*72)});
\coordinate (QE) at ({\Ra*sin(4*72)},{\e*\Ra*cos(4*72)});

\draw (PA)--(PC)--(PE)--(PB)--(PD)--cycle;
\draw (QA)--(QB)--(QC)--(QD)--(QE)--cycle;
\draw (PA)--(QA);
\draw (PB)--(QB);
\draw (PC)--(QC);
\draw (PD)--(QD);
\draw (PE)--(QE);

\fill[color=blau] (PA) circle[radius=\r];
\fill[color=rot] (PB) circle[radius=\r];
\fill[color=rot] (PC) circle[radius=\r];
\fill[color=gruen] (PD) circle[radius=\r];
\fill[color=gruen] (PE) circle[radius=\r];

\fill[color=rot] (QA) circle[radius=\r];
\fill[color=blau] (QB) circle[radius=\r];
\fill[color=gruen] (QC) circle[radius=\r];
\fill[color=rot] (QD) circle[radius=\r];
\fill[color=blau] (QE) circle[radius=\r];

\draw (PA) circle[radius=\r];
\draw (PB) circle[radius=\r];
\draw (PC) circle[radius=\r];
\draw (PD) circle[radius=\r];
\draw (PE) circle[radius=\r];

\draw (QA) circle[radius=\r];
\draw (QB) circle[radius=\r];
\draw (QC) circle[radius=\r];
\draw (QD) circle[radius=\r];
\draw (QE) circle[radius=\r];

\node at (0,{-\Ra}) [below] {$\operatorname{chr}P=3\mathstrut$};

\end{scope}

\begin{scope}[xshift=3.5cm]
\definecolor{rot}{rgb}{0.8,0,0.8}
\definecolor{gruen}{rgb}{0.2,0.6,0.2}
\definecolor{blau}{rgb}{1,0.6,0.2}
\definecolor{gelb}{rgb}{0,0,1}

\coordinate (PA) at ({\Ri*sin(0*72)},{\e*\Ri*cos(0*72)});
\coordinate (PB) at ({\Ri*sin(1*72)},{\e*\Ri*cos(1*72)});
\coordinate (PC) at ({\Ri*sin(2*72)},{\e*\Ri*cos(2*72)});
\coordinate (PD) at ({\Ri*sin(3*72)},{\e*\Ri*cos(3*72)});
\coordinate (PE) at ({\Ri*sin(4*72)},{\e*\Ri*cos(4*72)});

\coordinate (QA) at ({\Ra*sin(0*72)},{\e*\Ra*cos(0*72)});
\coordinate (QB) at ({\Ra*sin(1*72)},{\e*\Ra*cos(1*72)});
\coordinate (QC) at ({\Ra*sin(2*72)},{\e*\Ra*cos(2*72)});
\coordinate (QD) at ({\Ra*sin(3*72)},{\e*\Ra*cos(3*72)});
\coordinate (QE) at ({\Ra*sin(4*72)},{\e*\Ra*cos(4*72)});

\draw (PA)--(PC)--(PE)--(PB)--(PD)--cycle;
\draw (QA)--(QB)--(QC)--(QD)--(QE)--cycle;
\draw (PA)--(QA);
\draw (PB)--(QB);
\draw (PC)--(QC);
\draw (PD)--(QD);
\draw (PE)--(QE);

\fill[color=rot] (QA) circle[radius={1.5*\r}];
\fill[color=rot!40] (QB) circle[radius=\r];
\fill[color=rot!40] (QE) circle[radius=\r];
\fill[color=rot!40] (PA) circle[radius=\r];

\fill[color=blau] (PB) circle[radius={1.5*\r}];
\fill[color=blau!40] (PD) circle[radius=\r];
\fill[color=blau!40] (PE) circle[radius=\r];
\fill[color=blau!80,opacity=0.5] (QB) circle[radius=\r];

\fill[color=gruen] (PC) circle[radius={1.5*\r}];
\fill[color=gruen!40] (QC) circle[radius=\r];
\fill[color=gruen!80,opacity=0.5] (PA) circle[radius=\r];
\fill[color=gruen!80,opacity=0.5] (PE) circle[radius=\r];

\fill[color=gelb] (QD) circle[radius={1.5*\r}];
\fill[color=gelb!80,opacity=0.5] (QC) circle[radius=\r];
\fill[color=gelb!80,opacity=0.5] (QE) circle[radius=\r];
\fill[color=gelb!80,opacity=0.5] (PD) circle[radius=\r];

\draw (PA) circle[radius=\r];
\draw (PB) circle[radius={1.5*\r}];
\draw (PC) circle[radius={1.5*\r}];
\draw (PD) circle[radius=\r];
\draw (PE) circle[radius=\r];

\draw (QA) circle[radius={1.5*\r}];
\draw (QB) circle[radius=\r];
\draw (QC) circle[radius=\r];
\draw (QD) circle[radius={1.5*\r}];
\draw (QE) circle[radius=\r];

\node at (0,{-\Ra}) [below] {$\operatorname{ind}P=4\mathstrut$};

\end{scope}



\end{tikzpicture}
\end{document}

