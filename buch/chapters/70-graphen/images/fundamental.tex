%
% fundamental.tex -- template for standalon tikz images
%
% (c) 2021 Prof Dr Andreas Müller, OST Ostschweizer Fachhochschule
%
\documentclass[tikz]{standalone}
\usepackage{amsmath}
\usepackage{times}
\usepackage{txfonts}
\usepackage{pgfplots}
\usepackage{csvsimple}
\usetikzlibrary{arrows,intersections,math}
\begin{document}
\def\skala{1}
\begin{tikzpicture}[>=latex,thick,scale=\skala]

\begin{scope}[xshift=-4.6cm]
	\draw[color=red,line width=2pt] (1.8,0) -- (1.8,2);
	\draw[color=red,line width=2pt] (0,0) -- (4,0);
	\node at (1.8,0) [below] {$i$};
	\draw[->] (-0.1,0) -- (4.3,0) coordinate[label={$x$}];
	\draw[->] (0,-2.1) -- (0,2.3) coordinate[label={right:$y$}];

	\node at (2,-2.3) [below] {Standarbasis};
\end{scope}

\begin{scope}
	\draw[color=red,line width=1.4pt] 
		plot[domain=0:360,samples=100] ({\x/90},{2*sin(\x)});
	\draw[color=blue,line width=1.4pt] 
		plot[domain=0:360,samples=100] ({\x/90},{2*cos(\x)});
	\node[color=blue] at (1,-1) {$\Re f_i$};
	\node[color=red] at (2,1) {$\Im f_i$};
	\draw[->] (-0.1,0) -- (4.3,0) coordinate[label={$x$}];
	\draw[->] (0,-2.1) -- (0,2.3) coordinate[label={right:$y$}];
	\node at (2,-2.3) [below] {Eigenbasis};
\end{scope}

\begin{scope}[xshift=4.6cm]
	\foreach \t in {0.02,0.05,0.1,0.2,0.5}{
		\draw[color=red,line width=1.0pt]
			plot[domain=-1.8:2.2,samples=100]
				({\x+1.8},{exp(-\x*\x/(4*\t))/(sqrt(4*3.1415*\t))});
	}
	\fill[color=red] (1.8,0) circle[radius=0.08];
	\node at (1.8,0) [below] {$\xi$};
	\draw[->] (-0.1,0) -- (4.3,0) coordinate[label={$x$}];
	\draw[->] (0,-2.1) -- (0,2.3) coordinate[label={right:$y$}];
	\node at (2,-2.3) [below] {Fundamentallösung};
\end{scope}

\end{tikzpicture}
\end{document}

