%
% tikztemplate.tex -- template for standalon tikz images
%
% (c) 2021 Prof Dr Andreas Müller, OST Ostschweizer Fachhochschule
%
\documentclass[tikz]{standalone}
\usepackage{amsmath}
\usepackage{times}
\usepackage{txfonts}
\usepackage{pgfplots}
\usepackage{csvsimple}
\usetikzlibrary{arrows,intersections,math}
\begin{document}
\def\skala{1}
\begin{tikzpicture}[>=latex,thick,scale=\skala]

\def\l{0.25}
\def\r{1}
\def\punkt#1{({\r*sin(((#1)-1)*72)},{\r*cos(((#1)-1)*72)})}
\def\R{2}
\def\Punkt#1{({\R*sin(((#1)-6)*72)},{\R*cos(((#1)-6)*72)})}
\draw \Punkt{6} -- \Punkt{7} -- \Punkt{8} -- \Punkt{9} -- \Punkt{10} -- cycle;
\draw \punkt{1} -- \punkt{3} -- \punkt{5} -- \punkt{2} -- \punkt{4} -- cycle;
\foreach \k in {1,...,5}{
	\draw \punkt{\k} -- \Punkt{(\k+5)};
	\fill[color=white] \punkt{\k} circle[radius=\l];
	\node at \punkt{\k} {$\k$};
	\draw \punkt{\k} circle[radius=\l];
}
\foreach \k in {6,...,10}{
	\fill[color=white] \Punkt{\k} circle[radius=\l];
	\node at \Punkt{\k} {$\k$};
	\draw \Punkt{\k} circle[radius=\l];
}

\end{tikzpicture}
\end{document}

