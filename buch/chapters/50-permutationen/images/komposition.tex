%
% komposition.tex -- Komposition zweier Permutationen
%
% (c) 2021 Prof Dr Andreas Müller, OST Ostschweizer Fachhochschule
%
\documentclass[tikz]{standalone}
\usepackage{amsmath}
\usepackage{times}
\usepackage{txfonts}
\usepackage{pgfplots}
\usepackage{csvsimple}
\usetikzlibrary{arrows,intersections,math}
\begin{document}
\def\skala{1}
\begin{tikzpicture}[>=latex,thick,scale=\skala]

\begin{scope}[xshift=-4.5cm]
\node at (0,0) {$\displaystyle
\sigma_1=\begin{pmatrix}
1&2&3&4&5&6\\
2&1&3&5&6&4
\end{pmatrix}$};
\node at (0,-1) {$\displaystyle
\sigma_2=\begin{pmatrix}
1&2&3&4&5&6\\
3&4&5&6&1&2
\end{pmatrix}
$};
\end{scope}
\begin{scope}
\node at (0,0) {$\displaystyle
\begin{pmatrix}
1&2&3&4&5&6\\
2&1&3&5&6&4
\end{pmatrix}$};
\node at (0,-1) {$\displaystyle
\begin{pmatrix}
2&1&3&5&6&4\\
4&3&5&1&2&6
\end{pmatrix}
$};

\end{scope}
\begin{scope}[xshift=4.5cm]
\node at (0,-0.5) {$\displaystyle
\sigma_2\sigma_1=\begin{pmatrix}
1&2&3&4&5&6\\
4&3&5&1&2&6
\end{pmatrix}
$};
\end{scope}

\end{tikzpicture}
\end{document}

