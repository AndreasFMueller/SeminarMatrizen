%
% permutation.tex -- Definition einer Permutation
%
% (c) 2021 Prof Dr Andreas Müller, OST Ostschweizer Fachhochschule
%
\documentclass[tikz]{standalone}
\usepackage{amsmath}
\usepackage{times}
\usepackage{txfonts}
\usepackage{pgfplots}
\usepackage{csvsimple}
\usetikzlibrary{arrows,intersections,math}
\begin{document}
\def\skala{1}
\begin{tikzpicture}[>=latex,thick,scale=\skala]

\def\sx{0.8}
\def\sy{1}
\begin{scope}[xshift=-3cm]
\foreach \x in {1,...,6}{
	\node at ({(\x-1)*\sx},\sy) [above] {$\tiny\x$};
	\fill ({(\x-1)*\sx},\sy) circle[radius=0.05];
	\fill ({(\x-1)*\sx},0) circle[radius=0.05];
}
\draw[->] (0,\sy) to[out=-70,in=110] (\sx,0);
\draw[<-] (0,0) to[out=70,in=-110] (\sx,\sy);
\draw[->] ({2*\sx},\sy) -- ({2*\sx},0);
\draw[->] ({3*\sx},\sy) to[out=-70,in=110] ({4*\sx},0);
\draw[->] ({4*\sx},\sy) to[out=-70,in=110] ({5*\sx},0);
\draw[->] ({5*\sx},\sy) to[out=-110,in=70] ({3*\sx},0);
\end{scope}
\node at (2.4,{\sy/2}) {$\mathstrut=\mathstrut$};
\node at (5,{\sy/2}) {$\displaystyle
\renewcommand{\arraystretch}{1.4}
\begin{pmatrix}
1&2&3&4&5&6\\
2&1&3&5&6&4
\end{pmatrix}.
$};

\end{tikzpicture}
\end{document}

