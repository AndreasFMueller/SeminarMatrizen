%
% determinante.tex -- Formel für die Determinante mit Vorzeichen der
%                     Permutation
%
% (c) 2020 Prof Dr Andreas Müller, Hochschule Rapperswil
%
\section{Determinante
\label{buch:section:determinante}}
\rhead{Determinante}
Das Signum einer Permutationsmatrizen lässt sich
gemäss~\eqref{buch:permutationen:determinante}
mit der Determinanten berechnen.
Umgekehrt sollte es auch möglich sein, eine Formel
für die Determinante zu finden.
Die Basis dafür ist der
Entwicklungssatz 
\begin{equation}
\det(A)
=
\sum_{i=1}^n (-1)^{i+j} a_{i\!j} \cdot \det(A_{i\!j})
\label{buch:permutationen:entwicklungssatz}
\end{equation}
von Laplace für die Determinante.
\index{Entwicklungssatz}%
\index{Laplace, Entwicklungssatz von}%
Die Matrizen $A_{i\!j}$ sind die Minoren der Matrix $A$
(siehe auch Seite~\pageref{buch:linear:def:minor}).
In den Produkten $a_{i\!j}\cdot\det(A_{i\!j})$ enthält 
die Untermatrix $A_{i\!j}$ weder Elemente der Zeile $i$ noch der 
Zeile $j$.
Die Summanden auf der rechten Seite von
\eqref{buch:permutationen:entwicklungssatz}
sind daher Produkte der Form
\[
a_{1i_1}
a_{2i_2}
a_{3i_3}
\cdots
a_{ni_n},
\]
in denen nur Faktoren aus verschiedenen Spalten der Matrix $A$
vorkommen.
Das ist gleichbedeutend damit, dass unter den Spaltenindizes
$i_1,i_2,i_3,\dots,i_n$ keine zwei gleich sind, dass also
\[
\sigma
=
\begin{pmatrix}
1&2&3&\dots&n\\
i_1&i_2&i_3&\dots&i_n
\end{pmatrix}
\]
eine Permutation ist.

Die Determinante muss sich daher als Summe über alle Permutationen
in der Form
\begin{equation}
\det(A)
=
\sum_{\sigma\in S_n} 
c(\sigma)
\,
a_{1\sigma(1)}
a_{2\sigma(2)}
\cdots
a_{n\sigma(n)}
\label{buch:permutationen:cformel}
\end{equation}
schreiben lassen, wobei die Koeffizienten $c(\sigma)$ noch zu bestimmen
sind.
Setzt man in
\eqref{buch:permutationen:cformel}
eine Permutationsmatrix $P_\tau$ ein, dann verschwinden alle
Terme auf der rechten Seite ausser dem zur Permutation $\tau$,
also
\[
\det(P_\tau)
=
\sum_{\sigma \in S_n}
c(\sigma)
\,
(P_\tau)_{1\sigma(1)}
(P_\tau)_{2\sigma(2)}
\cdots
(P_\tau)_{n\sigma(n)}
=
c(\tau)
\,
1\cdot 1\cdots 1
=
c(\tau).
\]
Der Koeffizientn $c(\tau)$ ist also genau das Vorzeichen
der Permutation $\tau$.
Damit erhalten wir den folgenden Satz:

\begin{satz}
Die Determinante einer $n\times n$-Matrix $A$ kann berechnet werden als
\[
\det(A)
=
\sum_{\sigma\in S_n}
\operatorname{sgn}(\sigma)
a_{1\sigma(1)}
a_{2\sigma(2)}
\cdots
a_{n\sigma(n)}
=
\sum_{\tau\in S_n}
\operatorname{sgn}(\tau)
a_{\tau(1)1}
a_{\tau(2)2}
\cdots
a_{\tau(n)n}.
\]
Insbesondere folgt auch $\det(A)=\det(A^t)$.
\end{satz}

