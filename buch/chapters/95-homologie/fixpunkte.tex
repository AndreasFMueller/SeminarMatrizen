%
% fixpunkte.tex
%
% (c) 2021 Prof Dr Andreas Müller, OST Ostschweizer Fachhochschule
%
\section{Fixpunkte
\label{buch:section:fixpunkte}}
\rhead{Fixpunkte}
Zu jeder Abbildung $f\colon X\to X$ eines topologischen Raumes in sich
selbst gehört die zugehörige lineare Abbildung $f_*\colon H_*(X)\to H_*(X)$
der Homologiegruppen.
Diese linearen Abbildungen sind im Allgemeinen viel einfacher zu
analysieren.
Zum Beispiel soll in Abschnitt~\ref{buch:subsection:lefshetz}
die Lefshetz-Spurformel abgeleitet werden, die eine Aussagen darüber
ermöglicht, ob eine Abbildung einen Fixpunkt haben kann.
In Abschnitt~\ref{buch:subsection:brower} wird gezeigt wie man damit 
den Browerschen Fixpunktsatz beweisen kann, der besagt, dass jede
Abbildung eines Einheitsballs in sich selbst immer einen Fixpunkt hat.

\subsection{Lefshetz-Spurformel
\label{buch:subsection:lefshetz}}

\subsection{Brower-Fixpunktsatz
\label{buch:subsection:brower}}
