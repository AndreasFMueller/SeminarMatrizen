%
% gausshomoex.tex -- Beispiel für die Bestimmung einer Basis von H_1
%
% (c) 2021 Prof Dr Andreas Müller, OST Ostschweizer Fachhochschule
%
\documentclass[tikz]{standalone}
\usepackage{amsmath}
\usepackage{times}
\usepackage{txfonts}
\usepackage{pgfplots}
\usepackage{csvsimple}
\usetikzlibrary{arrows,intersections,math}
\begin{document}
\def\skala{1}
\begin{tikzpicture}[>=latex,thick,scale=\skala]

\def\s{2.0}

\def\punkt#1#2{({((#1)+0.5*(#2))*\s},{(#2)*\s*sqrt(3)/2})}

\def\knoten#1#2#3{
	\fill[color=white] \punkt{#1}{#2} circle[radius=0.3];
	\node at \punkt{#1}{#2} {$#3$\strut};
	\draw \punkt{#1}{#2} circle[radius=0.3];
}
\def\dreieck#1#2#3{
	\fill[color=gray] \punkt{#1}{#2} -- \punkt{#1+1}{#2}
		-- \punkt{#1}{(#2)+1} -- cycle;
	\node at \punkt{#1+0.3333}{#2+0.3333} {$#3$\strut};
	\draw[->,line width=1pt,shorten >= 0.3cm,shorten <= 0.3cm]
		\punkt{#1}{#2} -- \punkt{#1+1}{#2};
	\draw[->,line width=1pt,shorten >= 0.3cm,shorten <= 0.3cm]
		\punkt{#1+1}{#2} -- \punkt{#1}{#2+1};
	\draw[->,line width=1pt,shorten >= 0.3cm,shorten <= 0.3cm]
		\punkt{#1}{#2+1} -- \punkt{#1}{#2};
}

\def\Dreieck#1#2#3{
	\fill[color=gray!50] \punkt{#1}{#2} -- \punkt{#1+1}{#2}
		-- \punkt{#1+1}{(#2)-1} -- cycle;
	\node at \punkt{#1+0.3333}{#2+0.3333} {$#3$\strut};
}

\def\kante#1#2#3{
	\fill[color=white,opacity=0.8] \punkt{#1}{#2} circle[radius=0.15];
	\node at \punkt{#1}{#2} {$\scriptstyle #3$};
}

\dreieck{0}{0}{1}
\dreieck{1}{0}{2}
\dreieck{2}{0}{3}
\dreieck{3}{0}{4}

\dreieck{0}{1}{5}
\dreieck{2}{1}{6}

\dreieck{0}{2}{7}
\dreieck{1}{2}{8}

\dreieck{0}{3}{9}


\knoten{0}{0}{1}
\knoten{1}{0}{2}
\knoten{2}{0}{3}
\knoten{3}{0}{4}
\knoten{4}{0}{5}

\knoten{0}{1}{6}
\knoten{1}{1}{7}
\knoten{2}{1}{8}
\knoten{3}{1}{9}

\knoten{0}{2}{10}
\knoten{1}{2}{11}
\knoten{2}{2}{12}

\knoten{0}{3}{13}
\knoten{1}{3}{14}

\knoten{0}{4}{15}

\kante{0.5}{0}{1}
\kante{1.5}{0}{2}
\kante{2.5}{0}{3}
\kante{3.5}{0}{4}

\kante{0}{0.5}{5}
\kante{0.5}{0.5}{6}
\kante{1}{0.5}{7}
\kante{1.5}{0.5}{8}
\kante{2}{0.5}{9}
\kante{2.5}{0.5}{10}
\kante{3}{0.5}{11}
\kante{3.5}{0.5}{12}

\kante{0.5}{1}{13}
\kante{2.5}{1}{14}

\kante{0}{1.5}{15}
\kante{0.5}{1.5}{16}
\kante{2}{1.5}{17}
\kante{2.5}{1.5}{18}

\kante{0.5}{2}{19}
\kante{1.5}{2}{20}

\kante{0}{2.5}{21}
\kante{0.5}{2.5}{22}
\kante{1}{2.5}{23}
\kante{1.5}{2.5}{24}

\kante{0.5}{3}{25}

\kante{0}{3.5}{26}
\kante{0.5}{3.5}{27}

\end{tikzpicture}
\end{document}

