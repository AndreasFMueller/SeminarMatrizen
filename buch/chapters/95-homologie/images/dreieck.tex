%
% dreieck.tex -- Dreieck und Simplex
%
% (c) 2021 Prof Dr Andreas Müller, OST Ostschweizer Fachhochschule
%
\documentclass[tikz]{standalone}
\usepackage{amsmath}
\usepackage{times}
\usepackage{txfonts}
\usepackage{pgfplots}
\usepackage{csvsimple}
\usetikzlibrary{arrows,intersections,math}
\begin{document}
\def\skala{1}
\begin{tikzpicture}[>=latex,thick,scale=\skala]

\def\punkt#1{
	\fill[color=white] #1 circle[radius=0.07];
	\draw #1 circle[radius=0.07];
}
\begin{scope}[xshift=3cm]
\draw[->] (0,0) -- (3,3);
\draw[->] (0,0) -- (4,1);
\draw[->] (4,1) -- (3,3);
\node at (0,0) [below left] {$P_0$};
\node at (4,1) [below right] {$P_1$};
\node at (3,3) [above] {$P_2$};
\punkt{(0,0)}
\punkt{(4,1)}
\punkt{(3,3)}
\node at (2,0.5) [below] {$k_{01}$};
\node at (1.5,1.5) [above left] {$k_{02}$};
\node at (3.5,2) [right] {$k_{12}$};
\end{scope}
\begin{scope}[xshift=-3cm]
\fill[color=gray!40] (0,0) -- (4,1) -- (3,3) -- cycle;
\draw[->] (0,0) -- (3,3);
\draw[->] (0,0) -- (4,1);
\draw[->] (4,1) -- (3,3);
\node at (0,0) [below left] {$P_0$};
\node at (4,1) [below right] {$P_1$};
\node at (3,3) [above] {$P_2$};
\node at (2,0.5) [below] {$k_{01}$};
\node at (1.5,1.5) [above left] {$k_{02}$};
\node at (3.5,2) [right] {$k_{12}$};
\node at (2.333,1.333) {$\triangle$};
\punkt{(0,0)}
\punkt{(4,1)}
\punkt{(3,3)}
\end{scope}

\end{tikzpicture}
\end{document}

