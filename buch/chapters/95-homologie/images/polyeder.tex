%
% tikztemplate.tex -- template for standalon tikz images
%
% (c) 2021 Prof Dr Andreas Müller, OST Ostschweizer Fachhochschule
%
\documentclass[tikz]{standalone}
\usepackage{amsmath}
\usepackage{times}
\usepackage{txfonts}
\usepackage{pgfplots}
\usepackage{csvsimple}
\usetikzlibrary{arrows,intersections,math,calc}
\begin{document}
\def\skala{1}
\begin{tikzpicture}[>=latex,thick,scale=\skala]

% add image content here
\begin{scope}[xshift=-3.5cm,scale=0.5]
\coordinate (A) at (0,0);
\coordinate (B) at (4,0);
\coordinate (C) at (5,-2);
\coordinate (D) at (8,-1);
\coordinate (E) at (7,1);
\coordinate (F) at (7,3);
\coordinate (G) at (1,3);
\coordinate (H) at (5,4);
\coordinate (I) at (9,5);
\coordinate (J) at (4,7);
\coordinate (K) at (-1,9);
\coordinate (L) at (7,11);
\coordinate (M) at (6,-0.5);

\fill[color=gray,opacity=0.5] (A)--(B)--(H)--(G)--cycle;
\fill[color=gray,opacity=0.5] (G)--(I)--(K)--cycle;
\fill[color=gray,opacity=0.5] (G)--(L)--(K)--cycle;

\draw (K)--(G)--(A)--(B)--(D);
\draw (C)--(E);
\draw (G)--(I)--(K);
\draw (G)--(L)--(K);
\draw (B)--(H);
\draw (B)--(F);

\fill (A) circle[radius=0.1];
\fill (B) circle[radius=0.1];
\fill (C) circle[radius=0.1];
\fill (D) circle[radius=0.1];
\fill (E) circle[radius=0.1];
\fill (F) circle[radius=0.1];
\fill (G) circle[radius=0.1];
\fill (H) circle[radius=0.1];
\fill (I) circle[radius=0.1];
%\fill (J) circle[radius=0.1];
\fill (K) circle[radius=0.1];
\fill (L) circle[radius=0.1];
%\fill (M) circle[radius=0.1];

\draw[color=red] (H) circle[radius=0.5];
\draw[color=red] (J) circle[radius=0.5];
\draw[color=red] (M) circle[radius=0.5];
\draw[color=red] ($0.25*(A)+0.25*(B)+0.25*(G)+0.25*(H)$) circle[radius=0.5];

\end{scope}

\begin{scope}[xshift=3.5cm,scale=0.5]
\coordinate (A) at (0,0);
\coordinate (B) at (4,0);
\coordinate (C) at (5,-2);
\coordinate (D) at (8,-1);
\coordinate (E) at (7,1);
\coordinate (F) at (7,3);
\coordinate (G) at (1,3);
\coordinate (H) at (5,4);
\coordinate (I) at (9,5);
\coordinate (J) at (4,7);
\coordinate (K) at (-1,9);
\coordinate (L) at (7,11);
\coordinate (M) at (6,-0.5);

\fill[color=gray!50] (A)--(B)--(H)--(I)--(J)--(L)--(K)--(G)--cycle;

\draw (K)--(G)--(A)--(B)--(D);
\draw (C)--(E);
\draw (G)--(I)--(K);
\draw (G)--(L)--(K);
\draw (B)--(H);
\draw (B)--(F);
\draw (H)--(J);
\draw (A)--(H);

\fill (A) circle[radius=0.1];
\fill (B) circle[radius=0.1];
\fill (C) circle[radius=0.1];
\fill (D) circle[radius=0.1];
\fill (E) circle[radius=0.1];
\fill (F) circle[radius=0.1];
\fill (G) circle[radius=0.1];
\fill (H) circle[radius=0.1];
\fill (I) circle[radius=0.1];
\fill (J) circle[radius=0.1];
\fill (K) circle[radius=0.1];
\fill (L) circle[radius=0.1];
\fill (M) circle[radius=0.1];

\end{scope}

\end{tikzpicture}
\end{document}

