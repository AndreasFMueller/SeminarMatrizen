%
% homoclasses.tex -- template for standalon tikz images
%
% (c) 2021 Prof Dr Andreas Müller, OST Ostschweizer Fachhochschule
%
\documentclass[tikz]{standalone}
\usepackage{amsmath}
\usepackage{times}
\usepackage{txfonts}
\usepackage{pgfplots}
\usepackage{csvsimple}
\usetikzlibrary{arrows,intersections,math}
\begin{document}
\def\skala{1.4}
\begin{tikzpicture}[>=latex,thick,scale=\skala]

\definecolor{darkgreen}{rgb}{0,0.6,0}
\def\s{0.4}
\def\h{-0.3}

\def\punkt#1#2{({((#1)+0.5*(#2))*\s},{(#2)*\s*sqrt(3)/2})}
\def\A{\punkt{0}{0}}
\def\B{\punkt{1}{0}}
\def\C{\punkt{2}{0}}
\def\D{\punkt{3}{0}}
\def\E{\punkt{4}{0}}
\def\F{\punkt{0}{1}}
\def\G{\punkt{1}{1}}
\def\H{\punkt{2}{1}}
\def\I{\punkt{3}{1}}
\def\J{\punkt{0}{2}}
\def\K{\punkt{1}{2}}
\def\L{\punkt{2}{2}}
\def\M{\punkt{0}{3}}
\def\N{\punkt{1}{3}}
\def\O{\punkt{0}{4}}

%\def\knoten#1#2#3{
%	\fill[color=white] \punkt{#1}{#2} circle[radius=0.3];
%	\node at \punkt{#1}{#2} {$#3$\strut};
%	\draw \punkt{#1}{#2} circle[radius=0.3];
%}
\def\dreieck#1#2#3{
	\fill[color=gray] \punkt{#1}{#2} -- \punkt{#1+1}{#2}
		-- \punkt{#1}{(#2)+1} -- cycle;
%	\node at \punkt{#1+0.3333}{#2+0.3333} {$#3$\strut};
%	\draw[->,line width=1pt,shorten >= 0.3cm,shorten <= 0.3cm]
%		\punkt{#1}{#2} -- \punkt{#1+1}{#2};
%	\draw[->,line width=1pt,shorten >= 0.3cm,shorten <= 0.3cm]
%		\punkt{#1+1}{#2} -- \punkt{#1}{#2+1};
%	\draw[->,line width=1pt,shorten >= 0.3cm,shorten <= 0.3cm]
%		\punkt{#1}{#2+1} -- \punkt{#1}{#2};
}

%\def\Dreieck#1#2#3{
%	\fill[color=gray!50] \punkt{#1}{#2} -- \punkt{#1+1}{#2}
%		-- \punkt{#1+1}{(#2)-1} -- cycle;
%	\node at \punkt{#1+0.3333}{#2+0.3333} {$#3$\strut};
%}

%\def\kante#1#2#3{
%	\fill[color=white,opacity=0.8] \punkt{#1}{#2} circle[radius=0.15];
%	\node at \punkt{#1}{#2} {$\scriptstyle #3$};
%}

\def\gebiet{
	\dreieck{0}{0}{1}
	\dreieck{1}{0}{2}
	\dreieck{2}{0}{3}
	\dreieck{3}{0}{4}
	\dreieck{0}{1}{5}
	\dreieck{2}{1}{6}
	\dreieck{0}{2}{7}
	\dreieck{1}{2}{8}
	\dreieck{0}{3}{9}
}

\begin{scope}
\gebiet
\draw[color=darkgreen] \B -- \G -- \J -- \F -- cycle;
\draw[->,color=darkgreen] \B -- \G;
\node[color=darkgreen] at ({2*\s},{\h}) {$z_5'$};
\end{scope}

\begin{scope}[xshift=2cm]
\gebiet
\draw[color=darkgreen] \D -- \I -- \L -- \H -- cycle;
\draw[->,color=darkgreen] \D -- \I;
\node[color=darkgreen] at ({2*\s},{\h}) {$z_6'$};
\end{scope}

\begin{scope}[xshift=4cm]
\gebiet
\draw[color=darkgreen] \C -- \L -- \N -- \K -- \M -- \J -- cycle;
\draw[->,color=darkgreen] \C -- \L;
\node[color=darkgreen] at ({2*\s},{\h}) {$z_9'$};
\end{scope}

\begin{scope}[xshift=6cm]
\gebiet
\draw[color=darkgreen] \K -- \N -- \O -- \M -- cycle;
\draw[->,color=darkgreen] \K -- \N;
\node[color=darkgreen] at ({2*\s},{\h}) {$z_{12}'$};
\end{scope}


\end{tikzpicture}
\end{document}

