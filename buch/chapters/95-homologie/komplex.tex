%
% komplex.tex -- simpliziale Komplexe und Kettenkomplexe
%
% (c) 2021 Prof Dr Andreas Müller, OST Ostschweizer Fachhochschule
%
\section{Kettenkomplexe
\label{buch:section:komplex}}
\rhead{Kettenkomplexe}
Die algebraische Struktur, die in Abschnitt~\ref{buch:subsection:triangulation}
konstruiert wurde, kann noch etwas abstrakter konstruiert werden.
Es ergibt sich das Konzept eines Kettenkomplexes.
Die Triangulation gibt also Anlass zu einem Kettenkomplex.
So lässt sich zu einem geometrischen Objekt ein algebraisches 
Vergleichsobjekt konstruieren.
Im Idealfall lassens ich anschliessend geometrische Eigenschaften mit
algebraischen Rechnungen zum Beispiel in Vektorräumen mit Matrizen
beantworten.

\subsection{Definition
\label{buch:subsection:kettenkomplex-definition}}
Die Operation $\partial$, die für Simplizes konstruiert worden ist,
war linear und hat die Eigenschaft $\partial^2$ gehabt.
Diese Eigenschaften reichen bereits für Definition eines Kettenkomplexes.

\begin{definition}
Eine Folge $C_0,C_1,C_2,\dots$ von Vektorräumen über dem Körper $\Bbbk$
mit einer Folge von linearen Abbildungen
$\partial_k\colon C_k \to C_{k-1}$, dem {\em Randoperator},
heisst ein Kettenkomplex, wenn $\partial_{k-1}\partial_k=0$ gilt
für alle $k>0$.
\end{definition}

Die aus den Triangulationen konstruieren Vektorräme von
Abschnitt~\ref{buch:subsection:triangulation} bilden einen
Kettenkomplex.

XXX nachrechnen: $\partial^2 = 0$ ?

\subsection{Abbildungen
\label{buch:subsection:abbildungen}}
Wenn man verschiedene geometrische Objekte mit Hilfe von Triangulationen
vergleichen will, dann muss man auch das Konzept der Abbildungen zwischen
den geometrischen Objekten in die Kettenkomplexe transportieren.

Eine Abbildung zwischen Kettenkomplexen muss einerseits eine lineare
Abbildung der Vektorräume $C_k$ sein, andererseits muss sich eine
solche Abbildung mit dem Randoperator vertragen.
Wir definieren daher

\begin{definition}
Eine Abbildung $f_*$ zwischen zwei Kettenkomplexe $(C_*,\partial^C_*)$ und 
$(D_*,\partial^D_*)$ heisst eine Abbildung von Kettenkomplexen, wenn
für jedes $k$ 
\begin{equation}
\partial^D_k
\circ
f_{k}
=
f_{k+1}
\circ
\partial^C_k
\label{buch:komplex:abbildung}
\end{equation}
gilt.
\end{definition}

Die Beziehung~\eqref{buch:komplex:abbildung} kann übersichtlich als
kommutatives Diagramm dargestellt werden.
\begin{equation}
\begin{tikzcd}
0 
	& C_0 \arrow[l, "\partial_0^C"]
		\arrow[d, "f_0"]
		& C_1 \arrow[l,"\partial_1^C"]
			\arrow[d, "f_1"]
			& C_2 \arrow[l,"\partial_2^C"]
				\arrow[d, "f_2"]
				& \dots \arrow[l]
					\arrow[l, "\partial_{k-1}^C"]
					& C_k
						\arrow[l, "\partial_k^C"]
						\arrow[d, "f_k"]
						& C_{k+1}\arrow[l, "\partial_{k+1}^C"]
							\arrow[d, "f_{k+1}"]
							& \dots
\\
0 
	& D_0 \arrow[l, "\partial_0^D"]
		& D_1 \arrow[l,"\partial_1^D"]
			& D_2 \arrow[l,"\partial_2^D"]
				& \dots \arrow[l]
					\arrow[l, "\partial_{k-1}^D"]
					& D_k
						\arrow[l, "\partial_k^D"]
						& D_{k+1}\arrow[l, "\partial_{k+1}^D"]
							& \dots
\end{tikzcd}
\label{buch:komplex:abbcd}
\end{equation}
Die Relation~\eqref{buch:komplex:abbildung} drückt aus, dass man jeden
den Pfeilen im Diagram~\eqref{buch:komplex:abbcd} folgen kann und
dabei zwischen zwei Vektorräumen unabhängig vom Weg die gleiche Abbildung
resultiert.

Die Verfeinerung einer Triangulation erzeugt eine solche Abbildung von
Komplexen.


% XXX simpliziale Approximation

