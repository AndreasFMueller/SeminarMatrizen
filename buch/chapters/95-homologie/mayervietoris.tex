%
% mayervietoris.tex
%
% (c) 2021 Prof Dr Andreas Müller, OST Ostschweizer Fachhochschule
%
\section{Exaktheit und die Mayer-Vietoris-Folge
\label{buch:section:mayervietoris}}
\rhead{Exaktheit und die Mayer-Vietoris-Folge}
Die Berechnung der Homologie-Gruppen ist zwar im Wesentlichen ein 
kombinatorisches Problem, trotzdem ist eher aufwändig.
Oft weiss man, wie sich toplogische Räume aus einfacheren Räumen
zusammensetzen lassen.
Eine Mannigkfaltigkeit zum Beispiel wird durch die Karten
definiert, also zusammenziehbare Teilmengen von $\mathbb{R}^n$,
die die Mannigkfaltigkeit überdecken.
Das Ziel dieses Abschnittes ist, Regeln zusammenzustellen, mit denen
man die Homologie eines solchen zusammengesetzten Raumes aus der
Homologie der einzelnen Teile und aus den ``Verklebungsabbildungen'',
die die Teile verbinden, zu berechnen.

\subsection{Kurze exakte Folgen von Kettenkomplexen
\label{buch:subsection:exaktefolgen}}

\subsection{Schlangenlemma und lange exakte Folgen
\label{buch:subsection:schlangenlemma}}

\subsection{Mayer-Vietoris-Folge
\label{buch:subsection:mayervietoris}}
