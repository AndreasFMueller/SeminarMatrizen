\subsection{Euler-Charakteristik}
Die Homologiegruppen fassen die Idee, die ``Löcher'' in 
Dimension $k$ eines Polyeders zu zählen, algebraisch exakt.
Dazu ist aber die algebraische Struktur von $H_k(C)$  gar 
nicht nötig, nur schon die Dimension des Vektorraumes $H_k(C)$
liefert bereits die verlange Information.

Dies ist auch der Ansatz, den der eulersche Polyedersatz verfolgt.
Euler hat für dreidimensionale Polyeder eine Invariante gefunden, 
die unabhängig ist von der Triangulation.

\begin{definition}
\label{buch:homologie:def:eulerchar0}
Ist $E$ die Anzahl der Ecken, $K$ die Anzahl der Kanten und $F$
die Anzahl der Flächen eines dreidimensionalen Polyeders $P$, dann
heisst
\[
\chi(P) = E-K+F
\]
die {\em Euler-Charakteristik} des Polyeders $P$.
\end{definition}

Der Eulersche Polyedersatz, den wir nicht gesondert beweisen
wollen, besagt, dass $\chi(P)$ unabhängig ist von der 
Triangulation.
Alle regelmässigen Polyeder sind verschiedene Triangulationen
einer Kugel, sie haben alle den gleichen Wert $2$
der Euler-Charakteristik.

Ändert man die Triangulation, dann wird die Dimension der
Vektorräume $B_k(C)$ und $Z_k(C)$ grösser werden.
Kann man eine Grösse analog zu $\chi(P)$ finden, die sich nicht ändert?

\begin{definition}
\label{buch:homologie:def:eulerchar}
Sei $C$ ein Kettenkomplex, dann heisst
\[
\chi(C) = \sum_{k=0}^n (-1)^k\dim H_k(C)
\]
die Euler-Charakteristik von $C$.
\end{definition}

Die Summe in Definition~\ref{buch:homologie:def:eulerchar} erstreckt
sich bis zum Index $n$, der Dimension des Simplexes höchster Dimension
in einem Polyeder.
Für $k>n$ ist $H_k(C)=0$, es ändert sich also nichts, wenn wir
die Summe bis $\infty$ erstrecken, da die zusätzlichen Terme alle
$0$ sind.
Wir werden dies im folgenden zur Vereinfachung der Notation tun.

Die Definition verlangt, dass man erst die Homologiegruppen
berechnen muss, bevor man die Euler-Charakteristik bestimmen
kann.
Dies ist aber in vielen Fällen gar nicht nötig, da dies nur
eine Frage der Dimensionen ist, die man direkt aus den
$C_k$ ablesen kann, wie wir nun zeigen wollen.

Die Dimension der Homologiegruppen ist
\begin{equation}
\dim H_k(C)
=
\dim \bigl(Z_k(C) / B_k(C)\bigr)
=
\dim Z_k(C) - \dim B_k(C).
\label{buch:homologie:eqn:dimHk}
\end{equation}
Die Bestimmung der Dimensionen der Zyklen und Ränder erfordert
aber immer noch, dass wir dafür Basen bestimmen müssen, es ist
also noch nichts eingespart.
Die Zyklen bilden den Kern von $\partial$, also 
\[
\dim Z_k(C) = \dim\ker \partial_k.
\]
Die Ränder $B_k(C)$ sind die Bilder von $\partial_{k+1}$, also
\[
\dim B_k(C)
=
\dim C_{k+1} - \ker\partial_{k+1}
=
\dim C_{k+1} - \dim Z_{k+1}(C).
\]
Daraus kann man jetzt eine Formel für die Euler-Charakteristik
gewinnen.
Sie ist
\begin{align*}
\chi(C)
&=
\sum_{k=0}^\infty (-1)^k \dim H_k(C)
\\
&=
\sum_{k=0}^\infty (-1)^k \bigl(\dim Z_k(C) - \dim B_k(C)\bigr)
\\
&=
\sum_{k=0}^\infty (-1)^k \dim Z_k(C) 
-
\sum_{k=0}^\infty (-1)^k \bigl(\dim C_{k+1} - \dim_{k+1}(C)\bigr)
\\
&=
-\sum_{k=0}^\infty (-1)^k \dim C_{k+1} 
+
\sum_{k=0}^\infty (-1)^k \dim Z_k(C) 
+
\sum_{k=0}^\infty (-1)^k \dim Z_{k+1}(C).
\intertext{Indem wir in der letzten Summe den Summationsindex $k$ durch
$k-1$ ersetzen, können wir bis auf den ersten Term die Summen
der $\dim Z_k(C)$ zum Verschwinden bringen:}
&=
-\sum_{k=0}^\infty (-1)^k \dim C_{k+1} 
+
\sum_{k=0}^\infty (-1)^k \dim Z_k(C) 
-
\sum_{k=1}^\infty (-1)^k \dim Z_k(C)
\\
&=
\sum_{k=1}^\infty (-1)^k \dim C_{k}
+
\dim \underbrace{Z_0(C)}_{\displaystyle =C_0}.
\intertext{In der letzten Umformung haben wir auch in der ersten
Summe den Summationsindex $k$ durch $k-1$ ersetzt.
Damit beginnt die Summation bei $k=1$.
Der fehlende Term ist genau der Term, der von den Summen der
$\dim Z_k(C)$ übrig bleibt.
Damit erhalten wir}
&=
\sum_{k=0}^\infty (-1)^k \dim C_{k}.
\end{align*}

\begin{satz}
Für die Euler-Charakteristik eines endlichdimensionalen Kettenkomplexes $C$ gilt
\[
\chi(C)
=
\sum_{k=0}^\infty (-1)^k \dim H_k(C)
=
\sum_{k=0}^\infty (-1)^k \dim C_k.
\]
\end{satz}
Im nächsten Abschnitt wird gezeigt, dass die Euler-Charakteristik
als Spezialfall der sogenannten Lefshetz-Zahl verstanden werden kann.
