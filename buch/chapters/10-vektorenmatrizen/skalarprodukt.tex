%
% skalarprodukt.tex
%
% (c) 2021 Prof Dr Andreas Müller, OST Ostschweizer Fachhochschulen
%
\section{Skalarprodukt
\label{buch:section:skalarprodukt}}
\rhead{Skalarprodukt}
In der bisher dargestellten Form ist die lineare Algebra nicht
in der Lage, unsere vom Abstandsbegriff dominierte Geometrie adäquat
darzustellen.
Als zusätzliches Hilfsmittel wird eine Methode benötigt, Längen
und Winkel auszudrücken.
Das Skalarprodukt passt in den algebraischen Rahmen der
linearen Algebra, bringt aber auch einen Abstandsbegriff hervor,
der genau der geometrischen Intuition entspricht.

\subsection{Bilinearformen und Skalarprodukte
\label{buch:subsection:bilinearformen}}
Damit man mit einem Skalarprodukt rechnen kann wie mit jedem anderen
Produkt, müssen man auf beiden Seiten des Zeichesn ausmultiplizieren können:
\begin{align*}
(\lambda x_1 + \mu x_2)\cdot y &= \lambda x_1\cdot y + \mu x_2\cdot y\\
x\cdot (\lambda y_1 + \mu x_2) &= \lambda x\cdot y_1 + \mu x\cdot y_2.
\end{align*}
Man kann dies interpretieren als Linearität der Abbildungen 
$x\mapsto x\cdot y$ und $y\mapsto x\cdot y$.
Dies wird Bilinearität genannt und wie folgt definiert.

% XXX Bilinearität
\begin{definition}
Seien $U,V,W$ $\Bbbk$-Vektorräume.
Eine Abbildung $f\colon U\times V\to W$  heisst {\em bilinear},
\index{bilinear}%
wenn die partiellen Abbildungen $U\to W:x\mapsto f(x,y_0)$ und
$V\to W:y\mapsto f(x_0,y)$
linear sind für alle $x_0\in U$ und $y_0\in V$, d.~h.
\begin{align*}
f(\lambda x_1 + \mu x_2,y) &= \lambda f(x_1,y) + \mu f(x_2,y)
\\
f(x,\lambda y_1 + \mu y_2) &= \lambda f(x,y_1) + \mu f(x,y_2)
\end{align*}
Eine bilineare Funktion mit Werten in $\Bbbk$ heisst auch {\em Bilinearform}.
\index{Bilinearform}%
\end{definition}

\subsubsection{Symmetrische bilineare Funktionen}
Das Skalarprodukt hängt nicht von der Reihenfolge der Faktoren ab.
In Frage dafür kommen daher nur Bilnearformen $f\colon V\times V\to\Bbbk$,
die zusätzlich $f(x,y)=f(y,x)$ erfüllen.
Solche Bilinearformen heissen symmetrisch.
Für eine symmetrische Bilinearform gilt die binomische Formel
\begin{align*}
f(x+y,x+y)
&=
f(x,x+y)+f(y,x+y)
=
f(x,x)+f(x,y)+f(y,x)+f(y,y)
\\
&=
f(x,x)+2f(x,y)+f(y,y)
\end{align*}
wegen $f(x,y)=f(y,x)$.

\subsubsection{Positiv definite Bilinearformen und Skalarprodukt}
Bilinearität alleine genügt nicht, um einen Vektorraum mit einem
nützlichen Abstandsbegriff auszustatten.
Dazu müssen die berechneten Abstände vergleichbar sein, es muss also
eine Ordnungsrelation definiert sein, wie wir sie nur in $\mathbb{R}$
kennen.
Wir sind daher gezwungen uns auf $\mathbb{R}$- oder
$\mathbb{Q}$-Vektorräume zu beschränken.

Man lernt in der Vektorgeometrie, dass sich mit einer Bilinearform
$f\colon V\times V\to\mathbb{R}$ 
die Länge eines definieren lässt, indem man $\|x\|^2 = f(x,x)$
setzt.
Ausserdem muss $f(x,x)\ge 0$ sein für alle $x$, was die Bilinearität
allein nicht garantieren kann.
Verschiedene Punkte in einem Vektorraum sollen in dem aus der Bilinearform
abgeleiteten Abstandsbegriff immer unterscheidbar sein.
Dazu muss jeder von $0$ verschiedene Vektor positive Länge haben.

% XXX Positiv definite Form
\begin{definition}
Eine Bilinearform $f\colon V\times V\to\mathbb{R}$
heisst {\em positiv definit}, wenn
\index{positiv definit}%
\[
f(x,x) > 0\qquad\forall x\in V\setminus\{0\}.
\]
Das zugehörige {\em Skalarprodukt} wird $f(x,y)=\langle x,y\rangle$
geschrieben.
\index{Skalarprodukt}%
Die {\em $l^2$-Norm} $\|x\|_2$ eines Vektors ist definiert durch
$\|x\|_2^2 = \langle x,x\rangle$.
\end{definition}

\subsubsection{Dreiecksungleichung}
% XXX Dreiecksungleichung
Damit man sinnvoll über Abstände sprechen kann, muss die Norm
$\|\;\cdot\;\|_2$ der geometrischen Intuition folgen, die durch
die Dreiecksungleichung ausgedrückt wird.
In diesem Abschnitt soll gezeigt werden, dass die $l^2$-Norm
diese immer erfüllt.
Dazu sei $V$ ein $\mathbb{R}$-Vektorraum mit Skalarprodukt
$\langle\;,\;\rangle$.

\begin{satz}[Cauchy-Schwarz-Ungleichung]
Für $x,y\in V$ gilt
\[
|\langle x,y\rangle |
\le
\| x\|_2\cdot \|y\|_2
\]
mit Gleichheit genau dann, wenn $x$ und $y$ linear abhängig sind.
\end{satz}

\begin{proof}[Beweis]
Wir die Norm von $z=x-ty$:
\begin{align}
\|x-ty\|_2^2
&=
\|x\|_2^2 -2t\langle x,y\rangle +t^2\|y\|_2^2 \ge 0.
\notag
\end{align}
Sie nimmt den kleinsten Wert genau dann an, wenn es ein $t$ gibt derart,
dass $x=ty$.
Die rechte Seite ist ein quadratischer Ausdruck in $t$,
er hat sein Minimum bei
\begin{align*}
t&=-\frac{-2\langle x,y\rangle}{2\|y\|_2^2}
&&\Rightarrow&
\biggl\|
x  - \frac{\langle x,y\rangle}{\|y\|_2^2}y
\biggr\|_2^2
&=
\|x\|_2^2 
-
2\frac{(\langle x,y\rangle)^2}{\|y\|_2^2}
+
\frac{(\langle x,y\rangle)^2}{\|y\|_2^4} \|y\|_2^2
\\
&&&&
&=
\|x\|_2^2 
-
\frac{(\langle x,y\rangle)^2}{\|y\|_2^2}
=
\frac{
\|x\|_2^2\cdot\|y\|_2^2 - (\langle x,y\rangle)^2
}{
\|y\|_2^2
}
\ge 0
\intertext{Es folgt}
&&&\Rightarrow&
\|x\|_2^2\cdot\|y\|_2^2 - (\langle x,y\rangle)^2 &\ge 0
\\
&&&\Rightarrow&
\|x\|_2\cdot\|y\|_2 &\ge |\langle x,y\rangle |
\end{align*}
mit Gleichheit genau dann, wenn es ein $t$ gibt mit $x=ty$.
\end{proof}

\begin{satz}[Dreiecksungleichung]
Für $x,y\in V$ ist
\[
\| x + y \|_2 \le \|x\|_2 + \|y\|_2
\]
mit Gleichheit genau dann, wenn $x=ty$ ist für ein $t\ge 0$.
\end{satz}

\begin{proof}[Beweis]
\begin{align*}
\|x+y\|_2^2
&=
\langle x+y,x+y\rangle
=
\langle x,x\rangle
+
2\langle x,y\rangle
+
\langle y,y\rangle
\\
&=
\|x\|_2^2
+
2\langle x,y\rangle
+
\|y\|_2^2
=
\|x\|_2^2 + 2\langle x,y\rangle + \|y\|_2^2
\le
\|x\|_2^2 + 2\|x\|_2\cdot\|y\|_2 + \|y\|_2^2
\\
&=
(\|x\|_2 + \|y\|_2)^2
\\
\|x\|_2 + \|y\|_2
&\le \|x\|_2 + \|y\|_2,
\end{align*}
Gleichheit tritt genau dann ein, wenn 
$\langle x,y\rangle=\|x\|_2\cdot \|y\|_2$.
Dies tritt genau dann ein, wenn die beiden Vektoren linear abhängig sind.
\end{proof}

\subsubsection{Polarformel}
% XXX Polarformel
Auf den ersten Blick scheint die Norm $\|x\|_2$ weniger Information 
zu beinhalten, als die symmetrische Bilinearform, aus der sie
hervorgegangen ist.
Dem ist aber nicht so, denn die Bilinearform lässt sich aus der
Norm zurückgewinnen.
Dies ist der Inhalt der sogenannte Polarformel.

\begin{satz}[Polarformel]
Ist $\|\;\cdot\;\|_2$ eine Norm, die aus einer symmetrischen Bilinearform
$\langle\;,\;\rangle$ hervorgegangen ist, dann kann die Bilinearform
mit Hilfe der Formel
\begin{equation}
\langle x,y\rangle
=
\frac12(
\|x+y\|_2^2
-
\|x\|_2^2
-
\|y\|_2^2
)
\label{buch:grundlagen:eqn:polarformel}
\end{equation}
für $x,y\in V$ wiedergewonnen werden.
\end{satz}

\begin{proof}[Beweis]
Die binomischen Formel
\begin{align*}
\|x+y\|_2^2
&=
\|x\|_2^2 + 2\langle x,y\rangle + \|y\|_2^2
\intertext{kann nach $\langle x,y\rangle$ aufgelöst werden, was}
\langle x,y\rangle &= \frac12 (
\|x+y\|_2^2 - \|x\|_2^2 - \|y\|_2^2
)
\end{align*}
ergibt.
Damit ist die
Polarformel~\eqref{buch:grundlagen:eqn:polarformel}
bewiesen.
\end{proof}

\subsubsection{Komplexe Vektorräume uns Sesquilinearformen}
% XXX Sesquilinearform
Eine Bilinearform auf einem komplexen Vektorraum führt nicht
auf eine Grösse, die sich als Norm eignet.
Selbst wenn $\langle x,x\rangle >0$ ist,
\[
\langle ix,iy\rangle = i^2 \langle x,y\rangle
=
-\langle x,y\rangle < 0.
\]
Dies kann verhindert werden, wenn verlangt wird, dass der Faktor
$i$ im ersten Faktor der Bilinearform als $-i$ aus der Bilinearform
herausgenommen werden muss.

\begin{definition}
Seien $U,V,W$ komplexe Vektorräume.
Eine Abbildung $f\colon U\times V\to W$ heisst
{\em sesquilinear}\footnote{Das lateinische Wort {\em sesqui} bedeutet
eineinhalb, eine Sesquilinearform ist also eine Form, die in einem 
Faktor (dem zweiten) linear ist, und im anderen nur halb linear.}
\index{sesquilinear}
wenn gilt
\begin{align*}
f(\lambda x_1+\mu x_2,y) &= \overline{\lambda}f(x_1,y) + \overline{\mu}f(x_2,y)
\\
f(x,\lambda y_1+\mu y_2) &= \lambda f(x,y_1) + \mu f(x,y_2)
\end{align*}
\end{definition}

Für die Norm $\|x\|_2^2=\langle x,x\rangle$ bedeutet dies jetzt
\[
\|\lambda x\|_2^2
=
\langle \lambda x,\lambda x\rangle
=
\overline{\lambda}\lambda \langle x,x\rangle
=
|\lambda|^2 \|x\|_2^2
\qquad\Rightarrow\qquad
\|\lambda x\|_2 = |\lambda|\, \|x\|_2.
\]

\subsection{Orthognormalbasis
\label{buch:subsection:orthonormalbasis}}
\index{orthonormierte Basis}%

\subsubsection{Gram-Schmidt-Orthonormalisierung}
Mit Hilfe des Gram-Schmidtschen Orthonormalisierungsprozesses kann aus
einer beliebige Basis $\{a_1,a_2,\dots,a_n\}\subset V$ eines Vektorraums
mit einem SKalarprodukt eine orthonormierte Basis 
$\{b_1,b_2,\dots,b_n\}$ gefunden werden derart, dass für alle $k$
$\langle b_1,\dots,b_k\rangle = \langle a_1,\dots ,a_k\rangle$.
\index{Gram-Schmidt-Orthonormalisierung}%
Der Zusammenhang zwischen den Basisvektoren $b_i$ und $a_i$ ist
gegeben durch
\begin{align*}
b_1&=\frac{a_1}{\|a_1\|_2}
\\
b_2&=\frac{a_2-b_1\langle b_1,a_2\rangle}{\|a_2-b_1\langle b_1,a_2\rangle\|_2}
\\
b_3&=\frac{a_3-b_1\langle b_1,a_3\rangle-b_2\langle b_2,a_3\rangle}{\|a_3-b_1\langle b_1,a_3\rangle-b_2\langle b_2,a_3\rangle\|_2}
\\
&\phantom{n}\vdots\\
b_n
&=
\frac{
a_n-b_1\langle b_1,a_n\rangle-b_2\langle b_2,a_n\rangle
-\dots-b_{n-1}\langle b_{n-1},a_n\rangle
}{
\|
a_n-b_1\langle b_1,a_n\rangle-b_2\langle b_2,a_n\rangle
-\dots-b_{n-1}\langle b_{n-1},a_n\rangle
\|_2
}.
\end{align*}

\subsection{Symmetrische und selbstadjungierte Matrizen
\label{buch:subsection:symmetrisch-und-selbstadjungiert}}
%

\subsection{Orthogonale und unitäre Matrizen
\label{buch:subsection:orthogonale-und-unitaere-matrizen}}
% XXX Skalarprodukt und Lineare Abbildungen
% XXX Symmetrische Matrizen
% XXX Selbstadjungierte Matrizen

\subsection{Orthogonale Unterräume
\label{buch:subsection:orthogonale-unterraeume}}
% XXX Invariante Unterräume 
% XXX Kern und Bild orthogonaler Abbildungen

\subsection{Andere Normen auf Vektorräumen
\label{buch:subsection:andere-normen}}
% XXX l1 Norm
% XXX linfty Norm
% XXX Normen auf Funktionenräumen
% XXX Operatornorm
