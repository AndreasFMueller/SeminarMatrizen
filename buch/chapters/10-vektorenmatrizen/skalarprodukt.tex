%
% skalarprodukt.tex
%
% (c) 2021 Prof Dr Andreas Müller, OST Ostschweizer Fachhochschulen
%
\section{Skalarprodukt
\label{buch:section:skalarprodukt}}
\rhead{Skalarprodukt}
In der bisher dargestellten Form ist die lineare Algebra nicht
in der Lage, unsere vom Abstandsbegriff dominierte Geometrie adäquat
darzustellen.
Als zusätzliches Hilfsmittel wird eine Methode benötigt, Längen
und Winkel auszudrücken.
Das Skalarprodukt passt in den algebraischen Rahmen der
linearen Algebra, bringt aber auch einen Abstandsbegriff hervor,
der genau der geometrischen Intuition entspricht.

\subsection{Bilinearformen
\label{buch:subsection:bilinearformen}}
% XXX Bilinearität
% XXX Polarformel
% XXX Positiv definite Form
% XXX Sesquilinearform

\subsection{Orthogonale und unitäre Matrizen
\label{buch:subsection:orthogonale-und-unitaere-matrizen}}
% XXX Skalarprodukt und Lineare Abbildungen
% XXX Symmetrische Matrizen
% XXX Selbstadjungierte Matrizen

\subsection{Orthogonale Unterräume
\label{buch:subsection:orthogonale-unterraeume}}
% XXX Invariante Unterräume 
% XXX Kern und Bild orthogonaler Abbildungen

\subsection{Andere Normen auf Vektorräumen
\label{buch:subsection:andere-normen}}
% XXX l1 Norm
% XXX linfty Norm
% XXX Normen auf Funktionenräumen
% XXX Operatornorm
