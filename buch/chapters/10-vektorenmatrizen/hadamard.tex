%
% hadamard.tex
%
% (c) 2021 Prof Dr Andreas Müller, OST Ostschweizer Fachhochschule
%
\section{Hadamard-Algebra
\label{buch:section:hadamard-algebra}}
\rhead{Hadamard-Algebra}
Das Matrizenprodukt ist nicht die einzige Möglichkeit, ein Produkt auf
Vektoren oder Matrizen zu definieren.
In diesem Abschnitt soll das Hadamard-Produkt beschrieben werden,
welches zu einer kommutativen-Algebra-Struktur führt.

%
% Definition des Hadamard-Produktes
%
\subsection{Hadamard-Produkt
\label{buch:vektorenmatrizen:subsection:hadamard-produkt}}
Im Folgenden werden wir $\Bbbk^n =M_{n\times 1}(\Bbbk)$ setzen
und den Fall der Vektoren nicht mehr separat diskutieren.
Die Addition und Multiplikation mit Skalaren ist in 
$M_{m\times n}(\Bbbk)$ komponentenweise definiert.
Wir können natürlich auch ein Produkt komponentenweise definieren,
dies ist das Hadamard-Produkt.

\begin{definition}
Das {\em Hadamard-Produkt} zweier Matrizen
\index{Hadamard-Produkt}%
$A,B\in M_{m\times n}(\Bbbk)$ ist definiert als die Matrix
$A\odot B$ 
mit den Komponenten
\[
(A\odot B)_{i\!j} = (A)_{i\!j} (B)_{i\!j}.
\]
Wir nennen $M_{m\times n}(\Bbbk)$ mit der Multiplikation $\odot$ 
auch die {\em Hadamard-Algebra} $H_{m\times n}(\Bbbk)$.
\index{Hadamard-Algebra}%
\end{definition}

Dies ist jedoch nur interessant, wenn $M_{m\times n}(\Bbbk)$ mit diesem
Produkt eine interessante algebraische Struktur erhält.
Dazu müssen die üblichen Verträglichkeitsgesetze zwischen den
Vektorraumoperationen von $M_{m\times n}(\Bbbk)$ und dem neuen Produkt
gelten, wir erhalten dann eine Algebra.
Da alle Operationen elementweise definiert sind, muss man auch alle
Rechengesetze nur elementweise prüfen.
Es gilt
\begin{align*}
A\odot(B\odot C) &= (A\odot B)\odot C
&&\Leftrightarrow&
a_{i\!j}(b_{i\!j}c_{i\!j}) &= (a_{i\!j}b_{i\!j})c_{i\!j}
\\
A\odot(B+C) &= A\odot B + A\odot C
&&\Leftrightarrow&
a_{i\!j}(b_{i\!j}+c_{i\!j}) &= a_{i\!j}b_{i\!j} + a_{i\!j}c_{i\!j}
\\
(A+B)\odot C&=A\odot C+B\odot C
&&\Leftrightarrow&
(a_{i\!j}+b_{i\!j})c_{i\!j}&=a_{i\!j}c_{i\!j} + b_{i\!j}c_{i\!j}
\\
(\lambda A)\odot B &= \lambda (A\odot B)
&&\Leftrightarrow&
(\lambda a_{i\!j})b_{i\!j}&=\lambda(a_{i\!j}b_{i\!j})
\\
A\odot(\lambda B)&=\lambda(A\odot B)
&&\Leftrightarrow&
a_{i\!j}(\lambda b_{i\!j})&=\lambda(a_{i\!j}b_{i\!j})
\end{align*}
für alle $i,j$.

Das Hadamard-Produkt ist kommutativ, da die Multiplikation in $\Bbbk$
kommuativ ist.
Das Hadamard-Produkt kann auch für Matrizen mit Einträgen in einem
Ring definiert werden, in diesem Fall ist es möglich, dass die entstehende
Algebra nicht kommutativ ist.

Die Hadamard-Algebra hat auch ein Eins-Elemente, nämlich die Matrix,
die aus lauter Einsen besteht.

\begin{definition}
Die sogenannte {\em Einsmatrix} $U$ ist die Matrix
\index{Einsmatrix}
\[
U=\begin{pmatrix}
1&1&\dots&1\\
\vdots&\vdots&\ddots&\vdots\\
1&1&\dots&1
\end{pmatrix}
\in
M_{m\times n}(\Bbbk)
\]
mit lauter Einträgen $1\in\Bbbk$.
\end{definition}

Die Hadamard-Algebra ist ein Spezialfall der Algebra der Funktionen
$\Bbbk^X$.
Ordnet man dem Vektor $v\in \Bbbk^n$ mit den Komponenten $v_i$ 
die Abbildung
\[
v\colon [n] \to \Bbbk: i \mapsto v_i
\]
zu, dann geht die Addition von Vektoren in die Addition von
Funktionen über, die Multiplikation von Skalaren mit Vektoren
geht in die Multiplikation von Funktionen mit Skalaren über 
und die Hadamard-Multiplikation geht über in das Produkt von
Funktionen.

Auch die Hadamard-Algebra $H_{m\times n}(\Bbbk)$ kann als Funktionenalgebra
betrachtet werden.
Einer Matrix $A\in H_{m\times n}(\Bbbk)$ ordnet man die Funktion
\[
a\colon [m]\times [n] : (i,j) \mapsto a_{i\!j}
\]
zu.
Dabei gehen die Algebraoperationen von $H_{m\times n}(\Bbbk)$ über
in die Algebraoperationen der Funktionenalgebra $\Bbbk^{[m]\times [n]}$.
Aus der Einsmatrix der Hadamard-Algebra wird dabei zur konstanten
Funktion $1$ auf $[m]\times[n]$.

\subsection{Hadamard-Produkt und Matrizenalgebra
\label{buch:vektorenmatrizen:subsection:vertraeglichkeit}}
Es ist nur in Ausnahmefällen, Hadamard-Produkt und Matrizen-Produkt
gleichzeitig zu verwenden.
Das liegt daran, dass die beiden Produkte sich überhaupt nicht 
vertragen.

\subsubsection{Unverträglichkeit von Hadamard- und Matrizen-Produkt}
Das Hadamard-Produkt und das gewöhnliche Matrizenprodukt sind
in keiner Weise kompatibel.
Die beiden Matrizen 
\[
A=\begin{pmatrix}3&4\\4&5\end{pmatrix}
\qquad\text{und}\qquad
B=\begin{pmatrix}-5&4\\4&-3\end{pmatrix}
\]
sind inverse Matrizen bezüglich des Matrizenproduktes, also
$AB=I$.
Für das Hadamard-Produkt gilt dagegen
\[
A\odot B
=
\begin{pmatrix}
-15& 16\\
 16&-15
\end{pmatrix}.
\]
Die Inverse einer Matrix $A$ bezüglich des Hadamard-Produktes hat
die Einträge $a_{i\!j}^{-1}$.
Die Matrix $I$ ist bezüglich des gewöhnlichen Matrizenproduktes
invertierbar, aber sie ist bezüglich des Hadamard-Produktes nicht
invertierbar.
Umgekehrt ist die Einsmatrix $U$ invertierbar bezüglich des
Hadamard-Produktes, aber für $n>1$ nicht für das Matrizenprodukt.

\subsubsection{Einbettung der Hadamard-Algebra in eine Matrizenalgebra}
Hadamard-Algebren können als Unteralgebren einer Matrizenalgebra
betrachtet werden.
Der Operator $\operatorname{diag}$ bildet Vektoren ab in Diagonalmatrizen
nach der Regel
\[
\operatorname{diag}
\colon
\Bbbk^n \to M_n(\Bbbk)
:
\begin{pmatrix}
v_1\\
\vdots\\
v_n
\end{pmatrix}
\mapsto
\begin{pmatrix}
v_1&\dots&0\\
\vdots&\ddots&\vdots\\
0&\dots&v_n
\end{pmatrix}
\]
Das Produkt von Diagonalmatrizen ist besonders einfach.
Für zwei Vektoren $a,b\in\Bbbk^n$ 
\[
a\odot b
=
\begin{pmatrix}
a_1b_1\\
\vdots\\
a_nb_n
\end{pmatrix}
\mapsto
\begin{pmatrix}
a_1b_1&\dots&0\\
\vdots&\ddots&\vdots\\
0&\dots&a_nb_n
\end{pmatrix}
=
\begin{pmatrix}
a_1&\dots&0\\
\vdots&\ddots&\vdots\\
0&\dots&a_n
\end{pmatrix}
\begin{pmatrix}
b_1&\dots&0\\
\vdots&\ddots&\vdots\\
0&\dots&b_n
\end{pmatrix}.
\]
Das Hadamard-Produkt der Vektoren geht also über in das gewöhnliche
Matrizenprodukt der Diagonalmatrizen.

Für die Hadamard-Matrix ist die Einbettung etwas komplizierter.
Wir machen aus einer Matrix erst einen Vektor, den wir dann mit
dem $\operatorname{diag}$ in eine Diagonalmatrix umwandeln:
\[
\begin{pmatrix}
a_{11}&\dots&a_{1n}\\
\vdots&\ddots&\vdots\\
a_{m1}&\dots
\end{pmatrix}
\mapsto
\begin{pmatrix}
a_{11}\\
\vdots\\
a_{1n}\\
a_{21}\\
\vdots\\
a_{2n}\\
\vdots\\
a_{nn}
\end{pmatrix}
\]
Bei dieser Abbildung geht die Hadamard-Multiplikation wieder in
das gewöhnliche Matrizenprodukt über.

\subsubsection{Transposition}
Das Hadamard-Produkt verträgt sich mit der Transposition:
\index{Transposition}%
\[
(A\odot B)^t = A^t \odot B^t.
\]
Insbesondere ist das Hadamard-Produkt zweier symmetrischer Matrizen auch 
wieder symmetrisch.

\subsubsection{Frobenius-Norm}
Das Hadamard-Produkt in der Hadamard-Algebra $H_{m\times n}(\mathbb{R})$
nimmt keine Rücksicht auf die Dimensionen einer Matrix und ist nicht
unterscheidbar von $\mathbb{R}^{m\times n}$ mit dem Hadamard-Produkt.
Daher darf auch der Begriff einer mit den algebraischen Operationen
verträglichen Norm nicht von den spezifischen Dimensionen $m$ und $n$ abhängen.
Dies führt auf die folgende Definition einer Norm.

\begin{definition}
Die {\em Frobenius-Norm} einer Matrix $A\in H_{m\times n}(\mathbb{R})$
\index{Frobenius-Norm}%
mit den Einträgen $(a_{i\!j})=A$ ist
\[
\| A\|_F
=
\sqrt{
\sum_{i,j} a_{i\!j}^2
}.
\]
Das {\em Frobenius-Skalarprodukt} zweier Matrizen
$A,B\in H_{m\times n}(\mathbb{R})$
ist
\[
\langle A,B\rangle_F
=
\sum_{i,j} a_{i\!j} b_{i\!j}
=
\operatorname{Spur} A^t B
\]
und es gilt $\|A\|_F = \sqrt{\langle A,A\rangle}$.
\end{definition}

Für komplexe Matrizen muss die Definition angepasst werden, damit 
das Skalarprodukt sesquilinear und positiv definit wird.

\begin{definition}
Die {\em komplexe Frobenius-Norm} einer Matrix $A\in H_{m\times n}(\mathbb{C})$
ist
\[
\| A\|
=
\sqrt{
\sum_{i,j} |a_{i\!j}|^2
}
=
\sqrt{
\sum_{i,u} \overline{a}_{i\!j} a_{i\!j}
}
\]
das {\em komplexe Frobenius-Skalarprodukt} zweier Matrizen
$A,B\in H_{m\times n}(\mathbb{C})$ ist das Produkt
\[
\langle A,B\rangle_F
=
\sum_{i,j}\overline{a}_{i\!j} b_{i\!j}
=
\operatorname{Spur} (A^* B)
\]
und es gilt $\|A\|_F = \sqrt{\langle A,A\rangle}$.
\end{definition}

