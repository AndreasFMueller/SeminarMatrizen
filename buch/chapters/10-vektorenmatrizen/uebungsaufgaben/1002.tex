Nach Aufgabe \ref{1001} hat die Matrix 
\[
A
=
\begin{pmatrix}
0&0&1\\
1&0&a\\
0&1&b
\end{pmatrix}
\in
M_2(\mathbb{Z})
\quad\text{die Inverse}\quad
A^{-1}
=
\begin{pmatrix}
-b&1&0\\
-a&0&1\\
1&0&0
\end{pmatrix}
\in
M_2(\mathbb{Z}).
\]
Kann man $A^{-1}$ als Linearkombination der Matrizen $E$, $A$ und $A^2$
schreiben?

\begin{loesung}
Wir berechnen zunächst $A^2$:
\[
A^2
=
\begin{pmatrix}
0&0&1\\
1&0&a\\
0&1&b
\end{pmatrix}
\begin{pmatrix}
0&0&1\\
1&0&a\\
0&1&b
\end{pmatrix}
=
\begin{pmatrix}
0&1&b\\
0&a&1+ab\\
1&b&a+b^2
\end{pmatrix}
\]
Gesucht sind jetzt die Koeffizienten
$\lambda_i$ einer Linearkombination
\[
A^{-1} = \lambda_0 E + \lambda_1 A + \lambda_2 A^2.
\]
Die drei Matrizen auf der rechten Seite haben in der ersten
Spalte nur Nullen und Einsen, so dass wir an der ersten Spalten von
$A^{+}$ unmittelbar ablesen können, welche Werte wir für $\lambda_i$
verwenden müssen.
Wir finden $\lambda_0=-b$, $\lambda_1=-a$ und $\lambda_2=1$.
Wir setzen dies ein:
\begin{align*}
-bE-aA+A^2
&=
{\color{red}-b}
\begin{pmatrix}
1&0&0\\
0&1&0\\
0&0&1
\end{pmatrix}
{\color{blue}-a}
\begin{pmatrix}
0&0&1\\
1&0&a\\
0&1&b
\end{pmatrix}
+
\begin{pmatrix}
0&1&b\\
0&a&1+ab\\
1&b&a+b^2
\end{pmatrix}
\\
&=
\begin{pmatrix}
{\color{red}-b}      &{\color{darkgreen}1}                 &{\color{darkgreen}b}{\color{blue}-a}        \\
{\color{blue}-a}     &{\color{red}-b}+{\color{darkgreen}a} &{\color{blue}-a^2}+{\color{darkgreen}1+ab}  \\
{\color{darkgreen}1} &{\color{blue}-a}+{\color{darkgreen}b}&{\color{red}-b}{\color{blue}-ab}+{\color{darkgreen}a+b^2}\\
\end{pmatrix}
=
\begin{pmatrix}
-b       & 1         &b-a        \\
-a       &a-b       &1+a(b-a)  \\
 1       &b-a       &(1-b)(a-b)\\
\end{pmatrix}
\end{align*}
Diese Matrix kann nur dann mit $A^{-1}$ übereinstimmen, wenn $a-b=0$ ist, 
als $a=b$.
\end{loesung}

