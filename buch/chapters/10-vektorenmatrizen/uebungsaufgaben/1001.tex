Gegeben ist die Matrix
\[
A
=
\begin{pmatrix}
0&0&0&\dots&0&a_{1n}\\
1&0&0&\dots&0&a_{2n}\\
0&1&0&\dots&0&a_{3n}\\
0&0&1&\dots&0&a_{4n}\\
\vdots&\vdots&\vdots&\ddots&\vdots&\vdots\\
0&0&0&\dots&1&a_{nn}
\end{pmatrix}
\]
\begin{teilaufgaben}
\item Berechnen Sie $\det A$
\item Finden Sie die inverse Matrix $A^{-1}$
\item Nehmen Sie an, dass $a_{in}\in\mathbb{Z}$.
Formulieren Sie eine Bedingung an die Koeffizienten $a_{in}$, die garantiert,
dass $A^{-1}$ eine Matrix mit ganzzahligen Koeffizienten ist.
\end{teilaufgaben}

\begin{loesung}
\begin{teilaufgaben}
\item
Die Determinante ist am einfachsten mit Hilfe des Entwicklungssatzes durch
Entwicklung nach der ersten Zeile zu bestimmen:
\[
\det A
=
\left|
\begin{matrix}
0&0&0&\dots&0&a_{1n}\\
1&0&0&\dots&0&a_{2n}\\
0&1&0&\dots&0&a_{3n}\\
0&0&1&\dots&0&a_{4n}\\
\vdots&\vdots&\vdots&\ddots&\vdots&\vdots\\
0&0&0&\dots&1&a_{nn}
\end{matrix}
\right|
=
(-1)^{n+1}
a_{1n} \det E_n
=
-1^{n+1}
a_{1n}.
\]
\item
Die inverse Matrix kann am einfachsten mit Hilfe des Gauss-Algorithmus
gefunden werden.
Dazu schreiben wir die Matrix $A$ in die linke Hälfte eines Tableaus
und die Einheitsmatrix in die rechte Hälfte und führen den Gauss-Algorithmus
durch.
\[
\begin{tabular}{|>{$}c<{$}>{$}c<{$}>{$}c<{$}>{$}c<{$}>{$}c<{$}|>{$}c<{$}>{$}c<{$}>{$}c<{$}>{$}c<{$}>{$}c<{$}|}
\hline
0&0&0&\dots&a_{1n}&1&0&0&\dots&0\\
1&0&0&\dots&a_{2n}&0&1&0&\dots&0\\
0&1&0&\dots&a_{3n}&0&0&1&\dots&0\\
\vdots&\vdots&\vdots&\ddots&\vdots&\vdots&\vdots&\vdots&\ddots&\vdots\\
0&0&0&\dots&a_{nn}&0&0&0&\dots&1\\
\hline
\end{tabular}
\]
Die Arbeit wird wesentlich vereinfacht, wenn wir zunächst die erste Zeile 
ganz nach unten schieben:
\[
\rightarrow
\begin{tabular}{|>{$}c<{$}>{$}c<{$}>{$}c<{$}>{$}c<{$}>{$}c<{$}|>{$}c<{$}>{$}c<{$}>{$}c<{$}>{$}c<{$}>{$}c<{$}|}
\hline
1&0&\dots&0&a_{2n}&0&1&0&\dots&0\\
0&1&\dots&0&a_{3n}&0&0&1&\dots&0\\
\vdots&\vdots&\ddots&\vdots&\vdots&\vdots&\vdots&\vdots&\ddots&\vdots\\
0&0&\dots&1&a_{nn}&0&0&0&\dots&1\\
0&0&\dots&0&a_{1n}&1&0&0&\dots&0\\
\hline
\end{tabular}
\]
Mit einer einzigen Gauss-Operationen kann man jetzt die inverse Matrix
finden.
Dazu muss man zunächst durch das Pivot-Elemente $a_{1n}$ dividieren,
und dann in der Zeile $k$ das $a_{k+1,n}$-fache der letzten Zeile
subtrahieren.
Dies hat nur eine Auswirkung auf die erste Spalte in der rechten Hälfte:
\[
\renewcommand\arraystretch{1.2}
\rightarrow
\begin{tabular}{|>{$}c<{$}>{$}c<{$}>{$}c<{$}>{$}c<{$}>{$}c<{$}|>{$}c<{$}>{$}c<{$}>{$}c<{$}>{$}c<{$}>{$}c<{$}|}
\hline
1&0&\dots&0&0&-\frac{a_{2n}}{a_{1n}}&1&0&\dots&0\\
0&1&\dots&0&0&-\frac{a_{3n}}{a_{1n}}&0&1&\dots&0\\
\vdots&\vdots&\ddots&\vdots&\vdots&\vdots&\vdots&\vdots&\ddots&\vdots\\
0&0&\dots&1&0&-\frac{a_{nn}}{a_{1n}}&0&0&\dots&1\\
0&0&\dots&0&1&\frac{1}{a_{1n}}&0&0&\dots&0\\
\hline
\end{tabular}
\]
Die inverse Matrix von $A$ ist also
\begin{align}
A^{-1}
=
\renewcommand\arraystretch{1.2}
\begin{pmatrix}
-\frac{a_{2n}}{a_{1n}}&1&0&\dots&0\\
-\frac{a_{3n}}{a_{1n}}&0&1&\dots&0\\
\vdots&\vdots&\vdots&\ddots&\vdots\\
-\frac{a_{nn}}{a_{1n}}&0&0&\dots&1\\
\frac{1}{a_{1n}}&0&0&\dots&0\\
\end{pmatrix}
\label{buch:1001:inverse}
\end{align}
\item
Aus der Darstellung \eqref{buch:1001:inverse} der Inversen $A^{-1}$
können wir ablesen, dass $A^{-1}$ nur dann eine ganzzahlige Matrix sein
kann, wenn $a_{1n}$ invertierbar ist, also $a_{1n}=\pm1$.
\qedhere
\end{teilaufgaben}
\end{loesung}
