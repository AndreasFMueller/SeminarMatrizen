%
% gausszahlen.tex -- Ganze Gausssche Zahlen
%
% (c) 2021 Prof Dr Andreas Müller, OST Ostschweizer Fachhochschule
%
\documentclass[tikz]{standalone}
\usepackage{amsmath}
\usepackage{times}
\usepackage{txfonts}
\usepackage{pgfplots}
\usepackage{csvsimple}
\usepackage{color}
\usetikzlibrary{arrows,intersections,math}
\begin{document}
\definecolor{darkgreen}{rgb}{0,0.6,0}
\begin{tikzpicture}[>=latex,thick,scale=0.8]
\draw[->] (-8.5,0) -- (8.5,0) coordinate[label={$\Re z$}];
\draw[->] (0,-4.5) -- (0,4.5) coordinate[label={right:$\Im z$}];
\foreach \x in {-8,...,8}{
	\foreach \y in {-4,...,4}{
		\fill (\x,\y) circle[radius=0.05];
	}
}


\coordinate (O) at (0,0);
\coordinate (A) at (2,2);
\coordinate (B) at (-3,1);
\coordinate (C) at (-8,-4);
\coordinate (D) at (-1,3);
\draw[line width=0.5pt] (A)--(D)--(B);
\draw[->,color=red] (O) -- (A);
\draw[->,color=red] (O) -- (B);
\draw[->,color=blue] (O) -- (C);
\draw[->,color=darkgreen] (O) -- (D);
\fill[color=red] (A) circle[radius=0.08];
\fill[color=red] (B) circle[radius=0.08];
\fill[color=blue] (C) circle[radius=0.08];
\fill[color=darkgreen] (D) circle[radius=0.08];
\fill[color=black] (O) circle[radius=0.08];
\node[color=red] at (A) [above right] {$z$};
\node[color=red] at (B) [above left] {$w$};
\node[color=darkgreen] at (D) [above] {$z+w$};
\node[color=blue] at (C) [below right] {$z\cdot w$};

\end{tikzpicture}
\end{document}

