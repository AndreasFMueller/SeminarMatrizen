%
% algebren.tex -- Grundlegende Konstruktionen für Algebren
%
% (c) 2021 Prof Dr Andreas Müller, OST Ostschweizer Fachhochschule
%
\subsection{Algebren
\label{buch:grundlagen:subsection:algebren}}
Die Skalar-Multiplikation eines Vektorraums ist in einem Ring nicht
vorhanden.
Die Menge der Matrizen $M_n(\Bbbk)$ ist sowohl ein Ring als auch
ein Vektorraum.
Man nennt eine {\em $\Bbbk$-Algebra} oder {\em Algebra über $\Bbbk$}
ein Ring $A$, der auch eine $\Bbbk$-Vektorraum ist.
Die Multiplikation des Ringes muss dazu mit der Skalarmultiplikation
verträglich sein.
Dazu müssen Assoziativgesetze
\[
\lambda(\mu a) = (\lambda \mu) a
\qquad\text{und}\qquad
\lambda(ab) = (\lambda a) b
\]
für $a,b\in A$ und $\lambda,\mu\in\Bbbk$
und eine Regel der Form
\begin{equation}
a(\lambda b) = \lambda (ab)
\label{buch:vektorenmatrizen:eqn:algebrakommutativ}
\end{equation}
gelten.
Die Bedingung \eqref{buch:vektorenmatrizen:eqn:algebrakommutativ} ist
eine Folge der Forderung, dass die Multiplikation 
eine lineare Abbildung sein soll.
Dies bedeutet, dass
\begin{equation}
a(\lambda b+\mu c) = \lambda (ab) + \mu (ac),
\label{buch:vektorenmatrizen:eqn:algebralinear}
\end{equation}
woraus 
\eqref{buch:vektorenmatrizen:eqn:algebrakommutativ}
für $\mu=0$ folgt.
Die Regel \eqref{buch:vektorenmatrizen:eqn:algebralinear}
beinhaltet aber auch das Distributivgesetz.
$M_n(\Bbbk)$ ist eine Algebra.

\subsubsection{Die Algebra der Funktionen $\Bbbk^X$}
Sie $X$ eine Menge und $\Bbbk^X$ die Menge aller Funktionen $X\to \Bbbk$.
Auf $\Bbbk^X$ kann man Addition, Multiplikation mit Skalaren und
Multiplikation von Funktionen punktweise definieren.
Für zwei Funktion $f,g\in\Bbbk^X$ und $\lambda\in\Bbbk$ definiert man
\[
\begin{aligned}
&\text{Summe $f+g$:}
&
(f+g)(x) &= f(x)+g(x)
\\
&\text{Skalare $\lambda f$:}
&
(\lambda f)(x) &= \lambda f(x)
\\
&\text{Produkt $f\cdot g$:}
&
(f\cdot g)(x) &= f(x) g(x)
\end{aligned}
\]
Man kann leicht nachprüfen, dass die Menge der Funktionen $\Bbbk^X$
mit diesen Verknüfungen die Struktur einer $\Bbbk$-Algebra erhält.

Die Algebra der Funktionen $\Bbbk^X$ hat auch ein Einselement:
die konstante Funktion
\[
1\colon [a,b] \to \Bbbk : x \mapsto 1
\]
mit Wert $1$ erfüllt
\[
(1\cdot f)(x) = 1(x) f(x) = f(x)
\qquad\Rightarrow\qquad 1\cdot f = f,
\]
die Eigenschaft einer Eins in der Algebra.

\subsubsection{Die Algebra der stetigen Funktionen $C([a,b])$}
Die Menge der stetigen Funktionen $C([a,b])$ ist natürlich eine Teilmenge
aller Funktionen: $C([a,b])\subset \mathbb{R}^{[a,b]}$ und erbt damit
auch die Algebraoperationen.
Man muss nur noch sicherstellen, dass die Summe von stetigen Funktionen,
das Produkt einer stetigen Funktion mit einem Skalar und das Produkt von
stetigen Funktionen wieder eine stetige Funktion ist.
Eine Funktion ist genau dann stetig, wenn an jeder Stelle der Grenzwert
mit dem Funktionswert übereinstimmt.
Genau dies garantieren die bekannten Rechenregeln für stetige Funktionen.
Für zwei stetige Funktionen $f,g\in C([a,b])$ und einen Skalar
$\lambda\in\mathbb{R}$ gilt
\[
\begin{aligned}
&\text{Summe:}
&
\lim_{x\to x_0} (f+g)(x)
&=
\lim_{x\to x_0} (f(x)+g(x))
=
\lim_{x\to x_0} f(x) + \lim_{x\to x_0}g(x)
=
f(x_0)+g(x_0) = (f+g)(x_0)
\\
&\text{Skalare:}
&
\lim_{x\to x_0} (\lambda f)(x)
&=
\lim_{x\to x_0} (\lambda f(x)) = \lambda \lim_{x\to x_0} f(x)
=
\lambda f(x_0) = (\lambda f)(x_0)
\\
&\text{Produkt:}
&
\lim_{x\to x_0}(f\cdot g)(x)
&=
\lim_{x\to x_0} f(x)\cdot g(x)
=
\lim_{x\to x_0} f(x)\cdot
\lim_{x\to x_0} g(x)
=
f(x_0)g(x_0)
=
(f\cdot g)(x_0).
\end{aligned}
\]
für jeden Punkt $x_0\in[a,b]$.
Damit ist $C([a,b])$ eine $\mathbb{R}$-Algebra.
Die Algebra hat auch eine Eins, da die konstante Funktion $1(x)=1$ 
stetig ist.





