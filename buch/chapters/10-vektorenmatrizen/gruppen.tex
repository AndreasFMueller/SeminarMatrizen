%
% gruppen.tex
%
% (c) 2021 Prof Dr Andreas Müller, Hochschule Rapeprswil
%
\subsection{Gruppen
\label{buch:grundlagen:subsection:gruppen}}
Die kleinste sinnvolle Struktur ist die einer Gruppe.
Eine solche besteht aus einer Menge $G$ mit einer Verknüpfung,
die additiv
\index{additive Verknüpfung}%
\begin{align*}
G\times G \to G&: (g,h) = g+h
\intertext{oder multiplikativ }
G\times G \to G&: (g,h) = gh
\end{align*}
\index{multiplikative Verknüpfung}%
geschrieben werden kann.
Ein Element $0\in G$ heisst {\em neutrales Element} bezüglich der additiv
\index{neutrales Element}%
geschriebenen Verknüpfung falls $0+x=x$ für alle $x\in G$.
\index{neutrales Element}%
Ein Element $e\in G$ heisst neutrales Element bezüglich der multiplikativ 
geschriebneen Verknüpfung, wenn $ex=x$ für alle $x\in G$.
In den folgenden Definitionen werden wir immer die multiplikative
Schreibweise verwenden, für Fälle additiv geschriebener Verknüpfungen
siehe auch die Beispiele weiter unten.

\begin{definition}
\index{Gruppe}%
Ein {\em Gruppe}
\index{Gruppe}%
ist eine Menge $G$ mit einer Verknüfung mit folgenden
Eigenschaften:
\begin{enumerate}
\item
Die Verknüpfung ist assoziativ: $(ab)c=a(bc)$ für alle $a,b,c\in G$.
\index{assoziativ}%
\item
Es gibt ein neutrales Element $e\in G$
\item
Für jedes Element $g\in G$ gibt es ein Element $h\in G$ mit 
$hg=e$.
\end{enumerate}
Das Element $h$ heisst auch das inverse Element zu $g$.
\index{inverses Element}%
\end{definition}

Falls nicht jedes Element invertierbar ist, aber wenigstens ein neutrales
Element vorhanden ist, spricht man von einem {\em Monoid}.
\index{Monoid}%
Hat man nur eine Verknüpfung, aber kein neutrales Element,
spricht man oft von einer {\em Halbruppe}.
\index{Halbgruppe}%

\begin{definition}
Eine Gruppe $G$ heisst abelsch, wenn $ab=ba$ für alle $a,b\in G$.
\end{definition}
\index{abelsch}%

Additiv geschrieben Gruppen werden immer als abelsch angenommen,
multiplikativ geschrieben Gruppen können abelsch oder nichtabelsch sein.

\subsubsection{Beispiele von Gruppen}

\begin{beispiel}
Die Menge $\mathbb{Z}$ mit der Addition ist eine additive Gruppe mit
dem neutralen Element $0$.
Das additive Inverse eines Elementes $a$ ist $-a$.
\end{beispiel}

\begin{beispiel}
Die von Null verschiedenen Elemente $\Bbbk^*=\Bbbk\setminus\{0\}$ (definiert
auf Seite~\pageref{buch:zahlen:def:bbbk*})
eines Zahlekörpers bilden
bezüglich der Multiplikation eine Gruppe mit neutralem Element $1$.
Das multiplikative Inverse eines Elementes $a\in \Bbbk$ mit $a\ne 0$
ist $a^{-1}=\frac1{a}$.
\end{beispiel}

\begin{beispiel}
Die Vektoren $\Bbbk^n$ bilden bezüglich der Addition eine Gruppe mit
dem Nullvektor als neutralem Element.
Betrachtet man $\Bbbk^n$ als Gruppe, verliert man die Multiplikation
mit Skalaren aus den Augen.
$\Bbbk^n$ als Gruppe zu bezeichnen ist also nicht falsch, man
verliert dadurch aber den Blick auf die Multiplikation mit Skalaren.
\end{beispiel}

\begin{beispiel}
Die Menge aller quadratischen $n\times n$-Matrizen $M_n(\Bbbk)$ ist
eine Gruppe bezüglich der Addition mit der Nullmatrix als neutralem
Element.
Bezügich der Matrizenmultiplikation ist $M_n(\Bbbk)$ aber keine
Gruppe, da sich die singulären Matrizen nicht inverieren lassen.
Die Menge der invertierbaren Matrizen
\[
\operatorname{GL}_n(\Bbbk)
=
\{
A\in M_n(\Bbbk)\;|\; \text{$A$ invertierbar}
\}
\]
ist bezüglich der Multiplikation eine Gruppe.
Die Gruppe $\operatorname{GL}_n(\Bbbk)$ ist eine echte Teilmenge 
von $M_n(\Bbbk)$, die Addition und Multiplikation führen im Allgemeinen
aus der Gruppe heraus, es gibt also keine Mögichkeit, in der Gruppe
$\operatorname{GL}_n(\Bbbk)$ diese Operationen zu verwenden.
\end{beispiel}

\subsubsection{Einige einfache Rechenregeln in Gruppen}
Die Struktur einer Gruppe hat bereits eine Reihe von
Einschränkungen zur Folge.
Zum Beispiel sprach die Definition des neutralen Elements $e$ nur von
Produkten der Form $ex=x$, nicht von Produkten $xe$.
Und die Definition des inversen Elements $h$ von $g$ hat nur
verlangt, dass $gh=e$, es wurde nichts gesagt über das Produkt $hg$.

\begin{satz}
\label{buch:vektorenmatrizen:satz:gruppenregeln}
Ist $G$ eine Gruppe mit neutralem Element $e$, dann gilt
\begin{enumerate}
\item
$xe=x$ für alle $x\in G$
\item
Es gibt nur ein neutrales Element.
\index{neutrales Element}%
Wenn also $f\in G$ mit $fx=x$ für alle $x\in G$, ist dann folgt $f=e$.
\item 
Wenn $hg=e$ gilt, dann auch $gh=e$ und $h$ ist durch $g$ eindeutig bestimmt.
\end{enumerate}
\end{satz}

\begin{proof}[Beweis]
Wir beweisen als Erstes den ersten Teil der Eigenschaft~3.
Sei $h$ die Inverse von $g$, also $hg=e$.
Sei weiter $i$ die Inverse von $h$, also $ih=e$.
Damit folgt jetzt
\[
g
=
eg
=
(ih)g
=
i(hg)
=
ie.
\]
Wende man dies auf das Produkt $gh$ an, folgt
\[
gh
=
(ie)h
=
i(eh)
=
ih
=
e
\]
Es ist also nicht nur $hg=e$ sondern immer auch $gh=e$.

Für eine Inverse $h$ von $g$ folgt
\[
ge
=
g(hg)
=
(gh)g
=
eg
=
g,
\]
dies ist die Eigenschaft~1.

Sind $f$ und $e$ neutrale Elemente, dann folgt
\[
f = fe = e
\]
aus der Eigenschaft~1.

Schliesslich sei $x$ ein beliebiges Inverses von $g$.
Dann ist $xg=e$ und es folgt
$x=xe=x(gh)=(xg)h = eh = h$, es gibt also nur ein Inverses von $g$.
\end{proof}

Der Frage, ob Linksinverse und Rechtsinverse übereinstimmen,
sind wir zum Beispiel bereits in
Abschnitt~\ref{buch:grundlagen:subsection:gleichungssyteme}
begegnet.
Dort haben wir bereits gezeigt, dass nicht nur $AA^{-1}=I$,
sondern auch $A^{-1}A=I$.
Die dabei verwendete Methode war identisch mit dem hier gezeigten
Beweis.
Da die invertierbaren Matrizen eine Gruppe bilden, stellt sich
dieses Resultat jetzt als Spezialfall des
Satzes~\ref{buch:vektorenmatrizen:satz:gruppenregeln} dar.

\subsubsection{Homomorphismen} \label{buch:gruppen:subsection:homomorphismen}
Lineare Abbildung zwischen Vektorräumen zeichnen sich dadurch aus,
dass sie die algebraische Struktur des Vektorraumes respektieren.
Für eine Abbildung zwischen Gruppen heisst dies, dass die Verknüpfung,
das neutrale Element und die Inverse respektiert werden müssen.

\begin{definition}
Ein Abbildung $\varphi\colon G\to H$ zwischen Gruppen heisst ein
{\em Homomorphismus}, wenn 
$\varphi(g_1g_2)=\varphi(g_1)\varphi(g_2)$ für alle $g_1,g_2\in G$ gilt.
\index{Homomorphismus}%
\end{definition}

Der Begriff des Kerns einer linearen Abbildung lässt sich ebenfalls auf
die Gruppensituation erweitern.
Auch hier ist der Kern der Teil der Gruppe, er unter dem 
Homomorphismus ``unsichtbar'' wird.

\begin{definition}
Ist $\varphi\colon G\to H$ ein Homomorphisus, dann ist
\[
\ker\varphi
=
\{g\in G\;|\; \varphi(g)=e\}
\]
eine Untergruppe.
\index{Kern}%
\end{definition}

\subsubsection{Normalteiler}
Der Kern eines Homomorphismus ist nicht nur eine Untergruppe, er erfüllt
noch eine zusätzliche Bedingung. 
Für jedes $g\in G$ und $h\in\ker\varphi$ gilt 
\[
\varphi(ghg^{-1})
=
\varphi(g)\varphi(h)\varphi(g^{-1})
=
\varphi(g)\varphi(g^{-1})
=
\varphi(gg^{-1})
=
\varphi(e)
=
e
\qquad\Rightarrow\qquad
ghg^{-1}\in\ker\varphi.
\]
Der Kern wird also von der Abbildung $h\mapsto ghg^{-1}$,
der {\em Konjugation}, in sich abgebildet.
\index{Konjugation in einer Gruppe}

\begin{definition}
Eine Untergruppe $H \subset G$ heisst ein {\em Normalteiler},
geschrieben $H \triangleleft G$
wenn $gHg^{-1}\subset H$ für jedes $g\in G$.
\index{Normalteiler}%
\end{definition}

Die Konjugation selbst ist ebenfalls keine Unbekannte, sie ist uns
bei der Basistransformationsformel
\eqref{buch:vektoren-und-matrizen:eqn:basiswechselabb}
schon begegnet.
Die Tatsache, dass $\ker\varphi$ unter Konjugation erhalten bleibt,
kann man also interpretieren als eine Eigenschaft, die unter
Basistransformation erhalten bleibt.

\subsubsection{Faktorgruppen}
Ein Unterraum $U\subset V$ eines Vektorraumes gibt Anlass zum
Quotientenraum, der dadurch entsteht, dass man die Vektoren in $U$
zu $0$ kollabieren lässt.
Eine ähnliche Konstruktion könnte man für eine Untergruppe $H \subset G$
versuchen.
Man bildet also wieder die Mengen von Gruppenelementen, die sich um
ein Elemente in $H$ unterscheiden.
Man kann diese Mengen in der Form $gH$ mit $g\in G$ schreiben.

Man möchte jetzt aber auch die Verknüpfung für solche Mengen 
definieren, natürlich so, dass $g_1H\cdot g_2H = (g_1g_2)H$ ist.
Da die Verknüpfung nicht abelsch sein muss, entsteht hier
ein Problem.
Für $g_1=e$ folgt, dass $Hg_2H=g_2H$ sein muss.
Das geht nur, wenn $Hg_2=g_2H$ oder $g_2Hg_2^{-1}=H$ ist, wenn
also $H$ ein Normalteiler ist.

\begin{definition}
Für eine Gruppe $G$ mit Normalteiler  $H\triangleleft G$ ist die
Menge
\[
G/H = \{ gH \;|\; g\in G\}
\]
eine Gruppe mit der Verknüpfung $g_1H\cdot g_2H=(g_1g_2)H$.
$G/H$ heisst {\em Faktorgruppe} oder {\em Quotientengruppe}.
\index{Faktorgruppe}%
\index{Quotientengruppe}%
\end{definition}

Für abelsche Gruppen ist die Normalteilerbedingung keine zusätzliche
Einschränkung, jeder Untergruppe ist auch ein Normalteiler.

\begin{beispiel}
Die ganzen Zahlen $\mathbb{Z}$ bilden eine abelsche Gruppe und
die Menge der Vielfachen von $n$
$n\mathbb{Z}\subset\mathbb{Z}$ ist eine Untergruppe.
Da $\mathbb{Z}$ abelsch ist, ist $n\mathbb{Z}$ ein Normalteiler
und die Faktorgruppe $\mathbb{Z}/n\mathbb{Z}$ ist wohldefiniert.
Nur die Elemente
\[
0+n\mathbb{Z},
1+n\mathbb{Z},
2+n\mathbb{Z},
\dots
(n-1)+n\mathbb{Z}
\]
sind in der Faktorgruppe verschieden.
Die Gruppe $\mathbb{Z}/n\mathbb{Z}$ besteht also aus den Resten
bei Teilung durch $n$.
Diese Gruppe wird in Kapitel~\ref{buch:chapter:endliche-koerper}
genauer untersucht.
\end{beispiel}

Das Beispiel suggeriert, dass man sich die Elemente von $G/H$
als Reste vorstellen kann.

\subsubsection{Darstellungen}
Abstrakt definierte Gruppen können schwierig zu verstehen sein.
Oft hilft es, wenn man eine geometrische Darstellung der Gruppenoperation
finden kann.
Die Gruppenelemente werden dann zu umkehrbaren linearen Operationen
auf einem geeigneten Vektorraum.

\begin{definition}
\label{buch:vektorenmatrizen:def:darstellung}
Eine {\em Darstellung} einer Gruppe $G$ ist ein Homomorphismus 
$G\to\operatorname{GL}_n(\mathbb{R})$.
\index{Darstellung}
\end{definition}

\begin{beispiel}
Die Gruppen $\operatorname{GL}_n(\mathbb{Z})$,
$\operatorname{SL}_n(\mathbb{Z})$ oder $\operatorname{SO}(n)$ 
sind alle Teilmengen von $\operatorname{GL}_n(\mathbb{R})$.
Die Einbettungsabbildung $G\hookrightarrow \operatorname{GL}_n(\mathbb{R})$
ist damit automatisch eine Darstellung, sie heisst auch die
{\em reguläre Darstellung} der Gruppe $G$.
\index{reguläre Darstellung}%
\index{Darstellung, reguläre}%
\end{beispiel}

In Kapitel~\ref{buch:chapter:permutationen} wird gezeigt, 
dass Permutationen einer endlichen Menge eine Gruppe bilden und wie
sie durch Matrizen dargestellt werden können.






