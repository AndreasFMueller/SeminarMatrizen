%
% chapter.tex -- Kapitel mit den Grunddefinition und Notationen
%
% (c) 2020 Prof Dr Andreas Müller, OST Ostschweizer Fachhochschule
%
\chapter{Vektoren und Matrizen
\label{buch:chapter:vektoren-und-matrizen}}
\lhead{Vektoren und Matrizen }
\rhead{}

%
% linear.tex
%
% (c) 2021 Prof Dr Andreas Müller, OST Ostschweizer Fachhochschule
%
\section{Lineare Algebra
\label{buch:grundlagen:section:linearealgebra}}
\rhead{Lineare Algebra}
In diesem Abschnitt tragen wir die bekannten Resultate der linearen
Algebra zusammen.
Meistens lernt man diese zuerst für Vektoren und Gleichungssyteme mit
reellen Variablen.
In der linearen Algebra werden aber nur die arithmetischen
Grundoperationen verwendet, es gibt also keinen Grund, warum sich
die Theorie nicht über einem beliebigen Zahlenkörper entwickeln
lassen sollte.
Die in Kapitel~\ref{buch:chapter:endliche-koerper} untersuchten
endlichen Körper sind zum Beispiel besser geeignet für Anwendungen in
der Kryptographie oder für die diskrete schnelle Fourier-Transformation.
Daher geht es in diesem Abschnitt weniger darum alles herzuleiten,
sondern vor allem darum, die Konzepte in Erinnerung zu rufen und
so zu formulieren, dass offensichtlich wird, dass alles mit einem
beliebigen Zahlkörper $\Bbbk$ funktioniert.

%
% Vektoren
%
\subsection{Vektoren
\label{buch:grundlagen:subsection:vektoren}}
Koordinatensysteme haben ermöglicht, Punkte als Zahlenpaare zu beschreiben.
Dies ermöglicht, geometrische Eigenschaften als Gleichungen auszudrücken,
aber mit Punkten kann man trotzdem noch nicht rechnen.
Ein Vektor fasst die Koordinaten eines Punktes in einem Objekt zusammen,
mit dem man auch rechnen und zum Beispiel Parallelverschiebungen
algebraisieren kann.
Um auch Streckungen ausdrücken zu können, wird auch eine Menge von 
Streckungsfaktoren benötigt, mit denen alle Komponenten eines Vektors
multipliziert werden können.
Sie heissen auch {\em Skalare} und liegen in $\Bbbk$.

\subsubsection{Zeilen- und Spaltenvektoren}
Vektoren sind Tupel von Elementen aus $\Bbbk$.

\begin{definition}
Ein $n$-dimensionaler {\em Spaltenvektor} ist ein $n$-Tupel von Zahlen aus
$\Bbbk$ geschrieben als
\[
v = \begin{pmatrix} v_1\\v_2\\\vdots\\v_n\end{pmatrix}
\in \Bbbk^n.
\]
Ein $m$-dimensionaler {\em Zeilenvektor} wird geschrieben als
\[
u = \begin{pmatrix}u_1&u_2&\dots&u_m\end{pmatrix} \in \Bbbk^m.
\]
\end{definition}

Für Vektoren gleicher Dimension sind zwei Rechenoperationen definiert.
Die {\em Addition von Vektoren} $a,a\in\Bbbk^n$  und die Multiplikation
eines Vektors mit einem Skalar $\lambda\in\Bbbk$ erfolgt elementweise:
\[
a+b
=
\begin{pmatrix}a_1\\\vdots\\a_n\end{pmatrix}
+
\begin{pmatrix}b_1\\\vdots\\b_n\end{pmatrix}
=
\begin{pmatrix}a_1+b_1\\\vdots\\a_n+b_n\end{pmatrix},
\qquad
\lambda a
=
\lambda
\begin{pmatrix}a_1\\\vdots\\a_n\end{pmatrix}
=
\begin{pmatrix}\lambda a_1\\\vdots\\\lambda a_n\end{pmatrix}.
\]
Die üblichen Rechenregeln sind erfüllt, nämlich 
\begin{equation}
\begin{aligned}
&\text{Kommutativität:}
&
a+b&=b+a
&&
&&\forall a,b\in V
\\
&\text{Assoziativgesetze:}
&
(a+b)+c&=a+(b+c)
&
(\lambda\mu)a&=\lambda(\mu a)
&&\forall a,b,c\in V,\; \lambda,\mu\in\Bbbk
\\
&\text{Distributivgesetze:}
&
\lambda(a+b)&=\lambda a + \lambda b
&
(\lambda+\mu)a&=\lambda a + \mu a
&&\forall a,b\in V,\; \lambda,\mu\in\Bbbk.
\\
\end{aligned}
\label{buch:vektoren-und-matrizen:eqn:vrgesetze}
\end{equation}
Diese Gesetze drücken aus, dass man mit Vektoren so rechnen kann, wie man
das in der Algebra gelernt hat, mit der einzigen Einschränkung, dass
man Skalare immer links von Vektoren schreiben muss.
Die Distributivgesetze zum Beispiel sagen, dass man Ausmultipilizieren
oder Ausklammern kann genauso wie in Ausdrücken, die nur Zahlen enthalten.

Man beachte, dass es im allgemeinen kein Produkt von Vektoren gibt.
Das aus der Vektorgeometrie bekannte Vektorprodukt ist eine Spezialität
des dreidimensionalen Raumes, es gibt keine Entsprechung dafür in anderen
Dimensionen.

\subsubsection{Standardbasisvektoren}
In $\Bbbk^n$ findet man eine Menge von speziellen Vektoren, durch die
man alle anderen Vektoren ausdrücken kann.
Mit den sogenannten {\em Standardbasisvektoren}
\[
e_1=\begin{pmatrix}1\\0\\\vdots\\0\end{pmatrix},
e_2=\begin{pmatrix}0\\1\\\vdots\\0\end{pmatrix},
\dots,
e_n=\begin{pmatrix}0\\0\\\vdots\\1\end{pmatrix}
\]
kann der Vektor $a\in\Bbbk^n$ als
\[
a
=
\begin{pmatrix}a_1\\a_2\\\vdots\\a_n\end{pmatrix}
=
a_1 \begin{pmatrix}1\\0\\\vdots\\0\end{pmatrix}
+
a_2 \begin{pmatrix}0\\1\\\vdots\\0\end{pmatrix}
+
\dots
+
a_n \begin{pmatrix}0\\0\\\vdots\\1\end{pmatrix}
=
a_1e_1+a_2e_2+\dots+a_ne_n
\]
ausgedrückt werden.

\subsubsection{Vektorraum}
Die Rechnungen, die man gemäss der Rechengesetze
\eqref{buch:vektoren-und-matrizen:eqn:vrgesetze}
anstellen kann, verlangen nicht, dass Elemente $a$ und $b$, mit denen man
da rechnet, Zeilen- oder Spaltenvektoren sind.
Jede Art von mathematischem Objekt, mit dem man so rechen kann,
kann als (abstrakter) Vektor betrachtet werden.

\begin{definition}
Eine Menge $V$ von Objekten, auf der zwei Operationen definiert,
nämlich die Addition, geschrieben $a+b$ für $a,b\in V$ und die
Multiplikation mit Skalaren, geschrieben $\lambda a$ für $a\in V$ und 
$\lambda\in \Bbbk$, heisst ein {\em $\Bbbk$-Vektorraum} oder {\em Vektorraum
über $\Bbbk$} (oder
einfach nur {\em Vektorraum}, wenn $\Bbbk$ aus dem Kontext klar sind),
wenn die Rechenregeln~\eqref{buch:vektoren-und-matrizen:eqn:vrgesetze}
gelten
\end{definition}

Die Mengen von Spaltenvektoren $\Bbbk^n$ sind ganz offensichtlich
Vektorräume.
Die in Kapitel~\ref{buch:chapter:polynome} studierten Mengen von
Polynomen mit Koeffizienten in $\Bbbk$ sind ebenfalls Vektorräume.

\begin{beispiel}
Die Zahlenmenge $\mathbb{C}$ ist ein $\mathbb{R}$-Vektorraum.
Elemente von $\mathbb{C}$ können addiert und mit reellen Zahlen
multipliziert werden.
Die Rechenregeln für die komplexen Zahlen umfassen auch alle Regeln
\eqref{buch:vektoren-und-matrizen:eqn:vrgesetze}, also ist
$\mathbb{C}$ ein Vektorraum über $\mathbb{R}$.
\end{beispiel}

\begin{beispiel}
Die Menge $C([a,b])$ der stetigen Funktionen $[a,b]\to\mathbb{Re}$ 
bildet ein Vektorraum.
Funktionen können addiert und mit reellen Zahlen multipliziert werden:
\[
(f+g)(x) = f(x) + g(x)
\qquad\text{und}\qquad
(\lambda f)(x) = \lambda f(x).
\]
Dies reicht aber noch nicht ganz, denn $f+g$ und $\lambda f$ müssen
ausserdem auch {\em stetige} Funktionen sein.
Das dem so ist, lernt man in der Analysis.
Die Vektorraum-Rechenregeln
\eqref{buch:vektoren-und-matrizen:eqn:vrgesetze} sind ebenfalls erfüllt.
\end{beispiel}

Die Beispiele zeigen, dass der Begriff des Vektorraums die algebraischen
Eigenschaften eine grosse Zahl sehr verschiedenartiger mathematischer 
Objekte beschreiben kann.
Alle Erkenntnisse, die man ausschliesslich aus Vekotorraumeigenschaften
gewonnen hat, sind auf alle diese Objekte übertragbar.
Im folgenden werden wir alle Aussagen für einen Vektorraum $V$ formulieren,
wenn wir die Darstellung als Tupel $\Bbbk^n$ nicht brauchen.

\subsubsection{Gleichungssysteme in Vektorform}
Die Vektorraum-Operationen erlauben nun auch, lineare Gleichungssysteme
in {\em Vektorform} zu schreiben:
\index{Vektorform eines Gleichungssystems}%
\begin{equation}
\left.
\begin{linsys}{4}
a_{11} x_1 &+& \dots &+& a_{1n}x_n &=& b_1\\
\vdots     & & \ddots& & \vdots    & & \vdots \\
a_{m1} x_1 &+& \dots &+& a_{1n}x_n &=& b_m\\
\end{linsys}
\quad
\right\}
\qquad
\Rightarrow
\qquad
x_1
\begin{pmatrix}a_{11}\\\vdots\\a_{m1} \end{pmatrix}
+
\dots
+
x_n
\begin{pmatrix}a_{1n}\\\vdots\\a_{mn} \end{pmatrix}
=
\begin{pmatrix}b_1\\\vdots\\b_m\end{pmatrix}
\label{buch:vektoren-und-matrizen:eqn:vektorform}
\end{equation}
Die rechte Seite von~\eqref{buch:vektoren-und-matrizen:eqn:vektorform}
ist eine Linearkombination der Spaltenvektoren.

\begin{definition}
Eine Linearkombination der Vektoren $v_1,\dots,v_n\in V$ ist ein Ausdruck
der Form
\[
v
=
\lambda_1v_1+\dots + \lambda_n v_n
\]
mit $\lambda_1,\dots,\lambda_n\in \Bbbk$.
\end{definition}

Die Menge aller Vektoren, die sich als Linearkombinationen einer gegebenen
Menge ausdrücken lässt, heisst der aufgespannte Raum.

\begin{definition}
\index{aufgespannter Raum}%
Sind $a_1,\dots,a_n\in V$ Vektoren, dann heisst die Menge
\[
\langle a_1,\dots,a_n\rangle
=
\{x_1a_1+\dots+x_na_n\;|\; x_1,\dots,x_n\in\Bbbk\}
\]
aller Vektoren, die sich durch Linearkombination aus den Vektoren
$a_1,\dots,a_n$ gewinnen lassen, der von $a_1,\dots,a_n$
aufgespannte Raum.
\end{definition}

\subsubsection{Lineare Abhängigkeit}
Die Gleichung~\eqref{buch:vektoren-und-matrizen:eqn:vektorform}
drückt aus, dass sich der Vektor $b$ auf der rechten Seite als
Linearkombination der Spaltenvektoren ausdrücken lässt.
Oft ist eine solche Darstellung auf nur eine Art und Weise möglich.
Betrachten wir daher jetzt den Fall, dass es zwei verschiedene
Linearkombinationen der Vektoren $a_1,\dots,a_n$ gibt, die beide den
Vektor $b$ ergeben.
Deren Differenz ist
\begin{equation}
\left.
\begin{linsys}{4}
x_1 a_1 &+& \dots &+& x_n a_n &=& b \\
x_1'a_1 &+& \dots &+& x_n'a_n &=& b \\
\end{linsys}
\quad\right\}
\qquad
\Rightarrow
\qquad
(\underbrace{x_1-x_1'}_{\lambda_1}) a_1
+
\dots
+
(\underbrace{x_n-x_n'}_{\lambda_n}) a_n
=
0.
\label{buch:vektoren-und-matrizen:eqn:linabhkomb}
\end{equation}
Die Frage, ob ein Gleichungssystem genau eine Lösung hat, hängt also
damit zusammen, ob es Zahlen $\lambda_1,\dots,\lambda_n$ gibt, für
die die Gleichung~\label{buch:vektoren-und-matrizen:eqn:linabhkomb}
erfüllt ist.

\begin{definition}
Die Vektoren $a_1,\dots,a_n$ heissen linear abhängig, wenn es Zahlen
$\lambda_1,\dots,\lambda_n\in\Bbbk$ gibt, die nicht alle $0$ sind, so dass
\begin{equation}
\lambda_1a_1+\dots+\lambda_na_n = 0.
\label{buch:vektoren-und-matrizen:eqn:linabhdef}
\end{equation}
Die Vektoren heissen linear abhängig, wenn aus
\eqref{buch:vektoren-und-matrizen:eqn:linabhdef}
folgt, dass alle $\lambda_1,\dots,\lambda_n=0$ sind.
\end{definition}

Lineare Abhängigkeit der Vektoren $a_1,\dots,a_n$ bedeutet auch, dass
man einzelne der Vektoren durch andere ausdrücken kann.
Hat man nämlich eine
Linearkombination~\eqref{buch:vektoren-und-matrizen:eqn:linabhdef} und 
ist der Koeffizient $\lambda_k\ne 0$, dann kann man nach $a_k$ auflösen:
\[
a_k = -\frac{1}{\lambda_k}(\lambda_1a_1+\dots+\widehat{\lambda_ka_k}+\dots+\lambda_na_n).
\]
Darin bedeutet der Hut, dass der entsprechende Term weggelassen werden
muss.
Da dies für jeden von $0$ verschiedenen Koeffizienten möglich ist,
sagt man eben nicht, $a_k$ ist linear abhängig von den anderen, sondern
man sagt $a_1,\dots,a_n$ sind (untereinander) linear abhängig.

\subsubsection{Basis}
Ein lineares Gleichungssystem fragt danach, ob und wie ein Vektor $b$ als
Linearkombination der Vektoren $a_1,\dots,a_n$ ausgedrückt werden kann.
Wenn dies eindeutig möglich ist, dann haben die Vektoren $a_1,\dots,a_n$
offenbar eine besondere Bedeutung.

\begin{definition}
\index{Basis}%
\index{Dimension}%
Eine linear unabhängig Menge von Vektoren
$\mathcal{B}=\{a_1,\dots,a_n\}\subset V$
heisst {\em Basis} von $V$.
Die maximale Anzahl linear unabhängiger Vektoren in $V$ heisst 
{\em Dimension} von $V$.
\end{definition}

Die Standardbasisvektoren bilden eine Basis von $V=\Bbbk^n$.

\subsubsection{Unterräume}
Die Mengen $\langle a_1,\dots,a_n\rangle$ sind Teilmengen
von $V$, in denen die Addition von Vektoren und die Multiplikation mit 
Skalaren immer noch möglich ist.

\begin{definition}
Eine Teilmenge $U\subset V$ heisst ein {\em Unterraum} von $V$, wenn
$U$ selbst ein $\Bbbk$-Vektorraum ist, also
\[
\begin{aligned}
a,b&\in U &&\Rightarrow &a+b&\in U
\\
a&\in U, \lambda\in\Bbbk &&\Rightarrow & \lambda a&\in U
\end{aligned}
\]
gilt.
\end{definition}

%
% Matrizen
%
\subsection{Matrizen
\label{buch:grundlagen:subsection:matrizen}}
Die Koeffizienten eines linearen Gleichungssystems finden in einem 
Zeilen- oder Spaltenvektor nicht Platz.
Wir erweitern das Konzept daher in einer Art, dass Zeilen- und
Spaltenvektoren Spezialfälle sind.

\subsubsection{Definition einer Matrix}
\begin{definition}
Eine $m\times n$-Matrix $A$ (über $\Bbbk$) ist rechteckiges Schema
\index{Matrix}%
\[
A
=
\begin{pmatrix}
a_{11}&a_{12}&\dots &a_{1n}\\
a_{21}&a_{22}&\dots &a_{2n}\\
\vdots&\vdots&\ddots&\vdots\\
a_{m1}&a_{m2}&\dots &a_{mn}\\
\end{pmatrix}
\]
mit $a_{ij}\in\Bbbk$.
Die Menge aller $m\times n$-Matrizen wird mit
\[
M_{m\times n}(\Bbbk) = \{ A\;|\; \text{$A$ ist eine $m\times n$-Matrix}\}.
\]
Falls $m=n$ gilt, heisst die Matrix $A$ auch {\em quadratisch}
\index{quadratische Matrix}%
Man kürzt die Menge der quadratischen Matrizen als 
$M_n(\Bbbk) = M_{n\times n}(\Bbbk)$ ab.
\end{definition}

Die $m$-dimensionalen Spaltenvektoren $v\in \Bbbk^m$ sind $m\times 1$-Matrizen 
$v\in M_{n\times 1}(\Bbbk)$, die $n$-dimensionalen Zeilenvetoren $u\in\Bbbk^n$
sind $1\times n$-Matrizen $v\in M_{1\times n}(\Bbbk)$.
Eine $m\times n$-Matrix $A$ mit den Koeffizienten $a_{ij}$ besteht aus 
den $n$ Spaltenvektoren
\[
a_1 = \begin{pmatrix} a_{11} \\ a_{21} \\ \vdots \\ a_{m1} \end{pmatrix},\quad
a_2 = \begin{pmatrix} a_{12} \\ a_{22} \\ \vdots \\ a_{m2} \end{pmatrix},\dots,
a_n = \begin{pmatrix} a_{1n} \\ a_{2n} \\ \vdots \\ a_{mn} \end{pmatrix}.
\]
Sie besteht auch aus den $m$ Zeilenvektoren
\[
\begin{pmatrix} a_{k1} & a_{k2} & \dots & a_{kn} \end{pmatrix}
\]
mit $k=1,\dots,m$.

\subsubsection{Addition und Multiplikation mit Skalaren}
Die $m\times n$-Matrizen $M_{m\times n}(\Bbbk)$ bilden eine Vektorraum,
die Addition von Matrizen und die Multiplikation wird wie folgt definiert.

\begin{definition}
Sind $A,B\in M_{m\times n}(\Bbbk)$ und $\lambda\in\Bbbk$, dann setzt man
\[
A+B
=
\begin{pmatrix}
a_{11}+b_{11}&a_{12}+b_{12}&\dots &a_{1n}+b_{1n}\\
a_{21}+b_{21}&a_{22}+b_{22}&\dots &a_{2n}+b_{2n}\\
\vdots       &\vdots       &\ddots&\vdots       \\
a_{m1}+b_{m1}&a_{m2}+b_{m2}&\dots &a_{mn}+b_{mn}
\end{pmatrix}
\qquad\text{und}\qquad
\lambda A
=
\begin{pmatrix}
\lambda a_{11}&\lambda a_{12}&\dots &\lambda a_{1n}\\
\lambda a_{21}&\lambda a_{22}&\dots &\lambda a_{2n}\\
\vdots        &\vdots        &\ddots&\vdots        \\
\lambda a_{m1}&\lambda a_{m2}&\dots &\lambda a_{mn}
\end{pmatrix}.
\]
\end{definition}

\subsubsection{Multiplikation}
Will man ein lineares Gleichungssystem mit Hilfe der Matrix $A$ der
Koeffizienten schreiben, bekommt es die Form $Ax=b$, wobei der Vektor
der rechten Seiten ist, und $x$ ein Vektor von unbekannten Zahlen.
Dies ist jedoch nur sinnvoll, wenn das Produkt $Ax$ sinnvoll definiert
werden kann.

\begin{definition}
Eine $m\times n$-Matrix $A\in M_{m\times n}(\Bbbk)$ und eine
$n\times l$-Matrix $B\in M_{n\times l}(\Bbbk)$ haben als Produkt
eine $n\times l$-Matrix $C=AB\in M_{n\times l}(\Bbbk)$ mit den
Koeffizienten
\begin{equation}
c_{ij} = \sum_{k=1}^n a_{ik} b_{kj}.
\label{buch:vektoren-unbd-matrizen:eqn:matrixmultiplikation}
\end{equation}
\end{definition}

Die Koeffizienten $a_{ik}$ kommen aus der Zeile $i$ von $A$, die Koeffizienten
$b_{kj}$ stehen in der Spalte $j$ von $B$, die Multiplikationsregel
\eqref{buch:vektoren-unbd-matrizen:eqn:matrixmultiplikation}
besagt also, dass das Element $c_{ij}$ entsteht als das Produkt
der Zeile $i$ von $A$ mit der Spalte $j$ von $C$.

\subsubsection{Einheitsmatrix}
Welche $m\times m$-Matrix $I\in M_{m}(\Bbbk)$ hat die Eigenschaft, dass
$IA=A$ für jede beliebige Matrix $A\in M_{m\times n}(\Bbbk)$.
Wir bezeichnen die Einträge von $I$ mit $\delta_{ij}$.
Die Bedingung $IA=A$ bedeutet
\[
a_{ij} = \delta_{i1}a_{1j} + \dots + \delta_{im}a_{mj},
\]
Da auf der linken Seite nur $a_{ij}$ vorkommt, müssen alle Terme auf der
rechten Seite verschwinden ausser dem Term mit $a_{ij}$, dessen
Koeffizient $\delta_{ii}=1$ sein muss.
Die Koeffizienten sind daher
\[
\delta_{ij}
=
\begin{cases}
1&\qquad i=j\\
0&\qquad\text{sonst}
\end{cases}
\]
Die Zahlen $\delta_{ij}$ heissen auch das {\em Kronecker-Symbol} oder
{\em Kronecker-Delta}.
\index{Kronecker-$\delta$}%
\index{Kronecker-Symbol}%
Die Matrix $I$ hat die Einträge $\delta_{ij}$ und heisst die
{\em Einheitsmatrix}
\index{Einheitsmatrix}%
\[
I
=
\begin{pmatrix}
1     &0     &\dots &0     \\
0     &1     &\dots &0     \\[-2pt]
\vdots&\vdots&\ddots&\vdots\\
0     &0     &\dots &1  
\end{pmatrix}.
\]


%
% Gleichungssysteme
%
\subsection{Gleichungssysteme
\label{buch:grundlagen:subsection:gleichungssyteme}}
Lineare Gleichungssysteme haben wir bereits in Vektorform geschrieben.
Matrizen wurden eingeführt, um sie noch kompakter in der Matrixform
$Ax=b$ zu schreiben.
In diesem Abschnitt sollen die bekannten Resultate über die Lösung
von linearen Gleichungssytemen zusammengetragen werden.

\subsubsection{Eindeutige Lösung}
Mit Hilfe der Vektorform eines linearen Gleichungssystems wurde
gezeigt, dass die Lösung genau dann eindeutig ist, wenn die Spaltenvektoren
der Koeffizientenmatrix linear unabhängig sind.
Dies bedeutet, dass das Gleichungssystem
\begin{equation}
\begin{linsys}{3}
a_{11}x_1 &+& \dots &+& a_{1n}x_n &=& 0      \\
\vdots    & & \ddots& & \vdots    & & \vdots \\
a_{m1}x_1 &+& \dots &+& a_{mn}x_n &=& 0
\end{linsys}
\label{buch:grundlagen:eqn:homogenessystem}
\end{equation}
eine nichttriviale Lösung haben muss.
Das Gleichungssystem $Ax=b$ ist also genau dann eindeutig lösbar, wenn
das homogene Gleichungssystem $Ax=0$ nur die Nulllösung hat.

\subsubsection{Inhomogene und homogene Gleichungssysteme}
Ein Gleichungssystem mit $0$ auf der rechten Seite ist also bereits
ausreichend um zu entscheiden, ob die Lösung eindeutig ist.
Ein Gleichungssystem mit rechter Seite $0$ heisst {\em homogen}.
\index{homogenes Gleichungssystem}%
Zu jedem {\em inhomogenen} Gleichungssystem $Ax=b$ mit $b\ne 0$ 
ist $Ax=0$ das zugehörige homogene Gleichungssystem.

Ein homogenes Gleichungssytem $Ax=0$ hat immer mindestens die 
Lösung $x=0$, man nennt sie auch die {\em triviale} Lösung.
Eine Lösung $x\ne 0$ heisst auch eine nichttriviale Lösung.
Die Lösungen eines inhomgenen Gleichungssystem $Ax=b$ ist also nur dann
eindeutig, wenn das zugehörige homogene Gleichungssystem eine nichttriviale
Lösung hat.

\subsubsection{Gauss-Algorithmus}
Der Gauss-Algorithmus oder genauer Gausssche Eliminations-Algorithmus
löst ein lineare Gleichungssystem der
Form~\eqref{buch:vektoren-und-matrizen:eqn:vektorform}.
Die Koeffizienten werden dazu in das Tableau 
\[
\begin{tabular}{|>{$}c<{$}>{$}c<{$}>{$}c<{$}|>{$}c<{$}|}
\hline
a_{11}&\dots &a_{1n}&b_1   \\[-2pt]
\vdots&\ddots&\vdots&\vdots\\
a_{m1}&\dots &a_{mn}&b_m   \\
\hline
\end{tabular}
\]
geschrieben.
Die vertikale Linie erinnert an die Position des Gleichheitszeichens.
Es beinhaltet alle Informationen zur Durchführung des Algorithmus.
Der Algorithmus is so gestaltet, dass er nicht mehr Speicher als
das Tableau benötigt, alle Schritte operieren direkt auf den Daten
des Tableaus.

In jedem Schritt des Algorithmus wird zunächst eine Zeile $i$ und 
Spalte $j$ ausgewählt, das Elemente $a_{ij}$ heisst das Pivotelement.
\index{Pivotelement}%
Die {\em Pivotdivision}
\[
\begin{tabular}{|>{$}c<{$}>{$}c<{$}>{$}c<{$}>{$}c<{$}>{$}c<{$}|>{$}c<{$}|}
\hline
a_{11}&\dots &a_{1j}&\dots &a_{1n}&b_1   \\[-2pt]
\vdots&      &\vdots&\ddots&\vdots&\vdots\\
a_{i1}&\dots &{\color{red}a_{ij}}&\dots &a_{in}&b_i   \\[-2pt]
\vdots&      &\vdots&\ddots&\vdots&\vdots\\
a_{m1}&\dots &a_{mj}&\dots &a_{mn}&b_m   \\
\hline
\end{tabular}
\rightarrow
\begin{tabular}{|>{$}c<{$}>{$}c<{$}>{$}c<{$}>{$}c<{$}>{$}c<{$}|>{$}c<{$}|}
\hline
a_{11}&\dots &a_{1j}&\dots &a_{1n}&b_1   \\[-2pt]
\vdots&      &\vdots&\ddots&\vdots&\vdots\\
{\color{red}\frac{a_{i1}}{a_{ij}}}&\dots &{\color{red}1}&\dots &{\color{red}\frac{a_{in}}{a_{ij}}}&{\color{red}\frac{b_i}{a_{ij}}}\\[-2pt]
\vdots&      &\vdots&\ddots&\vdots&\vdots\\
a_{m1}&\dots &a_{mj}&\dots &a_{mn}&b_m   \\
\hline
\end{tabular}
\]
stellt sicher, dass das Pivot-Element zu $1$ wird.
\index{Pivotdivision}
Dies ist gleichbedeutend mit der Auflösung der Gleichung $i$ noch der
Variablen $x_j$.
Mit der {\em Zeilensubtraktion} auf Zeile $k\ne i$ können die Einträge in der
Spalte $j$ zu Null gemacht werden.
Dazu wird das $a_{kj}$-fache der Zeile $i$ von Zeile $k$ subtrahiert:
\[
\begin{tabular}{|>{$}c<{$}>{$}c<{$}>{$}c<{$}>{$}c<{$}>{$}c<{$}|>{$}c<{$}|}
\hline
\vdots&      &\vdots&\ddots&\vdots&\vdots\\
a_{i1}&\dots &{\color{red}1}&\dots &a_{in}&b_i   \\[-2pt]
\vdots&      &\vdots&\ddots&\vdots&\vdots\\
a_{k1}&\dots &a_{kj}&\dots &a_{kn}&b_m   \\[-2pt]
\vdots&      &\vdots&\ddots&\vdots&\vdots\\
\hline
\end{tabular}
\rightarrow
\begin{tabular}{|>{$}c<{$}>{$}c<{$}>{$}c<{$}>{$}c<{$}>{$}c<{$}|>{$}c<{$}|}
\hline
\vdots&      &\vdots&\ddots&\vdots&\vdots\\
a_{i1}&\dots &{\color{red}1}&\dots &a_{in}&b_i   \\[-2pt]
\vdots&      &\vdots&\ddots&\vdots&\vdots\\
{\color{blue}a_{k1}-a_{kj}a_{i1}}&\dots &{\color{blue}0}&\dots &{\color{blue}a_{kn}-a_{kj}a_{in}}&{\color{blue}b_m-a_{kj}b_{n}}\\[-2pt]
\vdots&      &\vdots&\ddots&\vdots&\vdots\\
\hline
\end{tabular}
\]
Typischerweise werden nach jeder Pivotdivision mehrer Zeilensubtraktionen
durchgeführt um alle anderen Elemente der Pivotspalte ausser dem
Pivotelement zu $0$ zu machen.
Beide Operationen können in einem Durchgang durchgeführt werden.

Die beiden Operationen Pivotdivision und Zeilensubtraktion werden jetzt
kombiniert um im linken Teil des Tableaus möglichst viele Nullen und
Einsen zu erzeugen.
Im Idealfall wird ein Tableau der Form
\[
\begin{tabular}{|>{$}c<{$}>{$}c<{$}>{$}c<{$}>{$}c<{$}|>{$}c<{$}|}
\hline
     1&     0&\dots &     0&u_1   \\
     0&     1&\dots &     0&u_2   \\[-2pt]
\vdots&\vdots&\ddots&\vdots&\vdots\\
     0&     0&\dots &     1&u_m   \\
\hline
\end{tabular}
\]
erreicht, was natürlich nur $m=n$ möglich ist.
Interpretiert man die Zeilen dieses Tableaus wieder als Gleichungen,
dann liefert die Zeile $i$ den Wert $x_i=u_i$ für die Variable $i$.
Die Lösung kann also in der Spalte rechts abgelesen werden.

\begin{figure}
\centering
\includegraphics[width=\textwidth]{chapters/10-vektorenmatrizen/images/rref.pdf}
\caption{Zweckmässiger Ablauf der Berechnung des Gauss-Algorithmus.
Falls in einer Spalte kein weiteres von $0$ verschiedenes Pivotelement
zur Verfügung steht, wird die Zeile übersprungen.
Weisse Felder enthalten $0$, dunkelgraue $1$.
Die roten Kreise bezeichnen Pivot-Elemente, die blauen Felder
die mit einer Zeilensubtraktion zu $0$ gemacht werden sollen.
\label{buch:grundlagen:fig:gaussalgorithmus}}
\end{figure}
Die effizienteste Strategie für die Verwendung der beiden Operationen
ist in Abbildung~\ref{buch:grundlagen:fig:gaussalgorithmus} dargestellt.
In der Phase der {\em Vorwärtsreduktion} werden Pivotelemente von links
nach rechts möglichst auf der Diagonale gewählt und mit Zeilensubtraktionen
die darunterliegenden Spalten freigeräumt.
\index{Vorwärtsreduktion}%
Während des Rückwärtseinsetzens werden die gleichen Pivotelemente von 
rechts nach links genutzt, um mit Zeilensubtraktionen auch die
Spalten über den Pivotelemnten frei zu räumen.
\index{Rückwärtseinsetzen}%
Wenn in einer Spalte kein von $0$ verschiedenes Element als Pivotelement
zur Verfügung steht, wird diese Spalte übersprungen.
Die so erzeuge Tableau-Form heisst auch die {\em reduzierte Zeilenstufenform}
({\em reduced row echelon form}, RREF).
\index{reduzierte Zeilenstufenform}%
\index{reduced row echelon form}%

Da der Ablauf des Gauss-Algorithmus vollständig von den Koeffizienten der
Matrix $A$ bestimmt ist, kann er gleichzeitig für mehrere Spalten auf der
rechten Seite oder ganz ohne rechte Seite durchgeführt werden.

\subsubsection{Lösungsmenge}
\index{Lösungsmenge}%
Die Spalten, in denen im Laufe des Gauss-Algorithmus kein Pivotelement
gefunden werden kann, gehören zu Variablen, nach denen sich das
Gleichungssystem nicht auflösen lässt.
Diese Variablen sind daher nicht bestimmt, sie können beliebig gewählt
werden.
Alle anderen Variablen sind durch diese frei wählbaren Variablen
bestimmt.

Für ein Gleichungssystem $Ax=b$ mit Schlusstableau
\index{Schlusstableau}%
\begin{equation}
\begin{tabular}{|>{$}c<{$}>{$}c<{$}>{$}c<{$}>{$}c<{$}>{$}c<{$}>{$}c<{$}>{$}c<{$}>{$}c<{$}>{$}c<{$}>{$}c<{$}>{$}c<{$}|>{$}c<{$}|}
\hline
   x_1&   x_2&\dots &x_{j_i-1}&{\color{darkgreen}x_{j_1}}&x_{j_1+1}&\dots &x_{j_2-1}&{\color{darkgreen}x_{j_2}}&\dots&{\color{darkgreen}x_{j_k}}& \\
\hline
     1&     0&\dots &        0&c_{1j_1}   &     0&\dots &     0&c_{1j_2}     &\dots &c_{1j_k}     &d_1      \\
     0&     1&\dots &        0&c_{2j_1}   &     0&\dots &     0&c_{2j_2}     &\dots &c_{1j_k}     &d_2      \\[-2pt]
\vdots&\vdots&\ddots&\vdots   &\vdots     &\vdots&\ddots&\vdots&\vdots       &\ddots&\vdots       &\vdots   \\
     0&     0&\dots &        1&c_{i_1,j_1}&     0&\dots &     0&c_{i_1,j_2}  &\dots &c_{i_1j_k}   &d_{i_1}  \\
     0&     0&\dots &        0&          0&     1&\dots &     0&c_{i_1+1,j_2}&\dots &c_{i_1+1,j_k}&d_{i_1+1}\\[-2pt]
\vdots&\vdots&\ddots&\vdots   &\vdots     &\vdots&\vdots&\vdots&\vdots       &\ddots&\vdots       &\vdots   \\
     0&     0&\dots &        0&          0&     0&\dots &     1&c_{i_2,j_2}  &\dots &c_{i_2j_k}   &d_{i_2}  \\
     0&     0&\dots &        0&          0&     0&\dots &     0&            0&\dots &c_{i_2+1,j_k}&d_{i_2+1}\\[-2pt]
\vdots&\vdots&\ddots&\vdots   &\vdots     &\vdots&\ddots&\vdots&\vdots       &\ddots&\vdots       &\vdots   \\
     0&     0&\dots &        0&          0&     0&\dots &     0&            0&\dots &            0&d_{m}    \\
\hline
\end{tabular}
\end{equation}
mit den $k$ frei wählbaren Variablen
$x_{j_1}, x_{j_2},\dots, x_{j_k}$ kann die Lösungsmenge als
\[
\mathbb{L}
=
\left\{
\left.
\begin{pmatrix}
d_1\\
d_2\\
\vdots\\
d_{i_1}\\
d_{i_1+1}\\
\vdots\\
d_{i_2}\\
d_{i_2+1}\\
\vdots\\
d_{m}
\end{pmatrix}
+
{\color{darkgreen}x_{j_1}}
\begin{pmatrix}
-c_{1j_1}\\
-c_{2j_1}\\
\vdots\\
-c_{i_1,j_1}\\
{\color{darkgreen}1}\\
\vdots\\
0\\
0\\
\vdots\\
0\\
\end{pmatrix}
+
{\color{darkgreen}x_{j_1}}
\begin{pmatrix}
-c_{1j_2}\\
-c_{2j_2}\\
\vdots\\
-c_{j_1,j_2}\\
-c_{j_1+1,j_2}\\
\vdots\\
-c_{i_2,j_2}\\
{\color{darkgreen}1}\\
\vdots\\
0\\
\end{pmatrix}
+
\dots
+
{\color{darkgreen}x_{j_k}}
\begin{pmatrix}
-c_{1j_k}\\
-c_{2j_k}\\
\vdots\\
-c_{j_1,j_k}\\
-c_{j_1+1,j_k}\\
\vdots\\
-c_{i_2,j_k}\\
-c_{i_2+1,j_k}\\
\vdots\\
0\\
\end{pmatrix}
\;
\right|
{\color{darkgreen}x_{i_1}},{\color{darkgreen}x_{i_2}},\dots,{\color{darkgreen}x_{i_k}}\in\Bbbk
\right\}
\]
geschrieben werden.
Insbesondere ist die Lösungsmenge $k$-dimensional.

\subsubsection{Inverse Matrix}
Zu jeder quadratischen Matrix $A\in M_n(\Bbbk)$ kann man versuchen, die
Gleichungen
\[
Ac_1 = e_1,\quad Ac_2 = e_2, \dots, Ac_n = e_n
\]
mit den Standardbasisvektoren $e_i$ als rechten Seiten zu lösen, wobei
die $c_i$ Vektoren in $\Bbbk^n$ sind.
Diese Vektoren kann man mit Hilfe des Gauss-Algorithmus finden:
\[
\begin{tabular}{|>{$}c<{$}>{$}c<{$}>{$}c<{$}>{$}c<{$}|>{$}c<{$}>{$}c<{$}>{$}c<{$}>{$}c<{$}|}
\hline
a_{11}&a_{12}&\dots &a_{1n}&1     &0     &\dots &0     \\
a_{21}&a_{22}&\dots &a_{2n}&0     &1     &\dots &0     \\
\vdots&\vdots&\ddots&\vdots&\vdots&\vdots&\ddots&\vdots\\
a_{n1}&a_{n2}&\dots &a_{nn}&0     &0     &\dots &1     \\
\hline
\end{tabular}
\rightarrow
\begin{tabular}{|>{$}c<{$}>{$}c<{$}>{$}c<{$}>{$}c<{$}|>{$}c<{$}>{$}c<{$}>{$}c<{$}>{$}c<{$}|}
\hline
1     &0     &\dots &0     &c_{11}&c_{12}&\dots &c_{1n}\\
0     &1     &\dots &0     &c_{21}&c_{22}&\dots &c_{2n}\\
\vdots&\vdots&\ddots&\vdots&\vdots&\vdots&\ddots&\vdots\\
0     &0     &\dots &1     &c_{n1}&c_{n2}&\dots &c_{nn}\\
\hline
\end{tabular}
\]
Die Vektoren $c_k$ sind die Spaltenvektoren der Matrix $C$ mit den
Einträgen $c_{ij}$.

Mit den Vektoren $c_k$ können jetzt beliebige inhomogene Gleichungssysteme
$Ax=b$ gelöst werden.
Da $b = b_1e_1 + b_2e_2 + \dots + b_ne_n$, kann man die Lösung $x$ als
$x = b_1c_1+b_2c_2+\dots+b_nc_n$ konstruieren.
Tatsächlich gilt
\begin{align*}
Ax
&= 
A( b_1c_1+b_2c_2+\dots+b_nc_n)
\\
&=
b_1Ac_1 + b_2Cc_2 + \dots + b_nAc_n
\\
&=
b_1e_1 + b_2e_2 + \dots + b_ne_n
=
b.
\end{align*}
Die Linearkombination $x=b_1c_1+\dots+b_nc_n$ kann in Vektorform als $x=Cb$
geschrieben werden.

Die Konstruktion von $C$ bedeutet auch, dass $AC=E$, daher heisst $C$ auch
die zu $A$ {\em inverse Matrix}.
\index{inverse Matrix}
Sie wird auch $C=A^{-1}$ geschrieben.

Die Definition der inversen Matrix stellt sicher, dass $AA^{-1}=I$ gilt,
daraus folgt aber noch nicht, dass auch $A^{-1}A=I$ ist.
Diese Eigenschaft kann man jedoch wie folgt erhalten.
Sei $C$ die inverse Matrix von $A$, also $AC=I$.
Sei weiter $D$ die inverse Matrix von $C$, also $CD=I$.
Dann ist zunächst $A=AE=A(CD)=(AC)D=ID=D$ und weiter
$CA=CD=I$.
Mit der Bezeichnung $C=A^{-1}$ erhalten wir also auch $A^{-1}A=I$.

Die Eigenschaften der Matrizenmultiplikation stellen sicher,
dass die Menge der invertierbaren Matrizen eine Struktur bilden,
die man Gruppe nennt, die in Abschnitt~\ref{buch:grundlagen:subsection:gruppen}
genauer untersucht wird.
In diesem Zusammenhang wird dann auf
Seite~\pageref{buch:vektorenmatrizen:satz:gruppenregeln}
die Eigenschaft $A^{-1}A=I$ ganz allgemein gezeigt.

\subsubsection{Determinante}
XXX TODO

%
% Lineare Abbildungen
%
\subsection{Lineare Abbildungen
\label{buch:grundlagen:subsection:lineare-abbildungen}}
Der besondere Nutzen der Matrizen ist, dass sie auch lineare Abbildungen
zwischen Vektorräumen beschreiben können.
In diesem Abschnitt werden lineare Abbildungen abstrakt definiert
und die Darstellung als Matrix mit Hilfe einer Basis eingeführt.


\subsubsection{Definition}
Eine lineare Abbildung zwischen Vektorräumen muss so gestaltet sein,
dass die Operationen des Vektorraums erhalten bleiben.
Dies wird von der folgenden Definition erreicht.

\begin{definition}
Eine Abbildung $f\colon V\to U$ zwischen Vektorräumen $V$ und $U$
heisst linear, wenn
\[
\begin{aligned}
f(v+w) &= f(v) + f(w)&&\forall v,w\in V 
\\
f(\lambda v) &= \lambda f(v) &&\forall v\in V,\lambda \in \Bbbk
\end{aligned}
\]
gilt.
\end{definition}

Lineare Abbildungen sind in der Mathematik sehr verbreitet.

\begin{beispiel}
Sie $V=C^1([a,b])$ die Menge der stetig differenzierbaren Funktionen 
auf dem Intervall $[a,b]$ und $U=C([a,b])$ die Menge der
stetigen Funktion aif $[a,b]$. 
Die Ableitung $\frac{d}{dx}$ macht aus einer Funktion $f(x)$ die
Ableitung $f'(x)$.
Die Rechenregeln für die Ableitung stellen sicher, dass 
\[
\frac{d}{dx}
\colon
C^1([a,b]) \to  C([a,b]) 
:
f \mapsto f'
\]
eine lineare Abbildung ist.
\end{beispiel}

\begin{beispiel}
Sei $V$ die Menge der Riemann-integrierbaren Funktionen auf dem
Intervall $[a,b]$ und $U=\mathbb{R}$.
Das bestimmte Integral
\[
\int_a^b \;\colon V \to U : f \mapsto \int_a^b f(x)\,dx
\]
ist nach den bekannten Rechenregeln für bestimmte Integrale
eine lineare Abbildung.
\end{beispiel}

\subsubsection{Matrix}
Um mit linearen Abbildungen rechnen zu können, ist eine Darstellung 
mit Hilfe von Matrizen nötig.
Sei also $\mathcal{B}=\{b_1,\dots,b_n\}$ eine Basis von $V$ und
$\mathcal{C} = \{ c_1,\dots,c_m\}$ eine Basis von $U$.
Das Bild des Basisvektors $b_i$ kann als Linearkombination der
Vektoren $c_1,\dots,c_m$ dargestellt werden.
Wir verwenden die Bezeichnung
\[
f(b_i) 
=
a_{1i} c_1 + \dots + a_{mi} c_m.
\]
Die lineare Abbildung $f$ bildet den Vektor $x$ mit Koordinaten
$x_1,\dots,x_n$ ab auf 
\begin{align*}
f(x)
&=
f(x_1b_1  + \dots x_nb_n)
\\
&=
x_1 f(b_1) + \dots x_nf(b_n)
\\
&=
x_1(a_{11} c_1 + \dots + a_{m1} c_m)
+
\dots
+
x_n(a_{1n} c_1 + \dots + a_{mn} c_m)
\\
&=
( a_{11} x_1 + \dots + a_{1n} x_n ) c_1
+
\dots
+
( a_{m1} x_1 + \dots + a_{mn} x_n ) c_m
\end{align*}
Die Koordinaten von $f(x)$ in der Basis $\mathcal{C}$ in $U$ sind 
also gegeben durch das Matrizenprodukt $Ax$, wenn $x$ der Spaltenvektor
aus den Koordinaten in der Basis $\mathcal{B}$ in $V$ ist.

Die Matrix einer linearen Abbildung macht Aussagen über eine lineare
Abbilung der Rechnung zugänglich.
Allerdings hängt die Matrix einer linearen Abbildung von der Wahl der
Basis ab.
Gleichzeitig ist dies eine Chance, durch Wahl einer geeigneten Basis
kann man eine Matrix in eine Form bringen, die zur Lösung eines
Problems optimal geeignet ist.

\subsubsection{Basiswechsel}
In einem Vektorraum $V$ seien zwei Basen $\mathcal{B}=\{b_1,\dots,b_n\}$
und $\mathcal{B}'=\{b_1',\dots,b_n'\}$ gegeben.
Ein Vektor $v\in V$ kann in beiden beiden Basen dargestellt werden.
Wir bezeichnen mit dem Spaltenvektor $x$ die Koordinaten von $v$ in der
Basis $\mathcal{B}$ und mit dem Spaltenvektor $x'$ die Koordinaten
in der Basisi $\mathcal{B}'$.
Um die Koordinaten umzurechnen, muss man die Gleichung
\begin{equation}
x_1b_1 + \dots + x_nb_n = x_1'b_1' + \dots + x_n'b_n'
\label{buch:vektoren-und-matrizen:eqn:basiswechselgleichung}
\end{equation}
lösen.

Stellt man sich die Vektoren $b_i$ und $b_j'$ als $m$-dimensionale
Spaltenvektoren vor mit $m\ge n$, dann bekommt
\eqref{buch:vektoren-und-matrizen:eqn:basiswechselgleichung}
die Form eines Gleichungssystems
\[
\begin{linsys}{6}
b_{11}x_1&+& \dots &+&b_{1n}x_n&=&b_{11}'x_1'&+& \dots &+&b_{1n}'x_n'\\
\vdots   & & \ddots& &\vdots   & &\vdots     & & \ddots& &\vdots     \\
b_{m1}x_1&+& \dots &+&b_{mn}x_n&=&b_{m1}'x_1'&+& \dots &+&b_{mn}'x_n'
\end{linsys}
\]
Dieses Gleichungssystem kann man mit Hilfe eines Gauss-Tableaus lösen.
Wir schreiben die zugehörigen Variablen 
\[
\renewcommand{\arraystretch}{1.1}
\begin{tabular}{|>{$}c<{$} >{$}c<{$} >{$}c<{$}|>{$}c<{$}>{$}c<{$}>{$}c<{$}|}
\hline
x_1&\dots&x_n&x_1'&\dots&x_n'\\
\hline
b_{11}&\dots &b_{1n}&b_{11}'&\dots &v_{1n}'\\
\vdots&\ddots&\vdots&\vdots &\ddots&\vdots \\
b_{n1}&\dots &b_{nn}&b_{n1}'&\dots &v_{nn}'\\
\hline
b_{n+1,1}&\dots &b_{n+1,n}&b_{n+1,1}'&\dots &v_{n+1,n}'\\
\vdots&\ddots&\vdots&\vdots &\ddots&\vdots \\
b_{m1}&\dots &b_{mn}&b_{m1}'&\dots &v_{mn}'\\
\hline
\end{tabular}
\rightarrow
\begin{tabular}{|>{$}c<{$} >{$}c<{$} >{$}c<{$}|>{$}c<{$}>{$}c<{$}>{$}c<{$}|}
\hline
x_1&\dots&x_n&x_1'&\dots&x_n'\\
\hline
1     &\dots &0     &t_{11}             &\dots              &t_{1n}        \\
\vdots&\ddots&\vdots&\vdots             &\ddots             &\vdots        \\
0     &\dots &1     &t_{n1}             &\dots              &t_{nn}        \\
\hline
0     &\dots &0     &{\color{red}0}     &{\color{red}\dots} &{\color{red}0}\\
\vdots&\ddots&\vdots&{\color{red}\vdots}&{\color{red}\ddots}&{\color{red}\vdots}\\
0     &\dots &0     &{\color{red}0}     &{\color{red}\dots} &{\color{red}0}\\
\hline
\end{tabular}
\]
Das rechte untere Teiltableau enthält lauter Nullen genau dann, wenn jeder
Vektor in $V$ sich in beiden Mengen $\mathcal{B}$ und $\mathcal{B}'$
ausdrücken lässt.
Dies folgt aber aus der Tatsache, dass $\mathcal{B}$ und $\mathcal{B}'$
beide Basen sind, also insbesondere den gleichen Raum aufspannen.
Die $n\times n$-Matrix $T$ mit Komponenten $t_{ij}$ rechnet Koordinaten
in der Basis $\mathcal{B}'$ um in Koordinaten in der Basis $\mathcal{B}$.

\subsubsection{Umkehrabbbildung}
Sei $f$ eine umkehrbare lineare Abbildung $U\to V$ und $g\colon V\to U$.
die zugehörige Umkehrabbildung.
Für zwei Vektoren $u$ und $w$ in $U$ gibt es daher Vektoren $a=g(u)$
und $b=g(w)$ in $V$ derart, dass $f(a)=u$ und $f(b)=w$.
Weil $f$ linear ist, folgt daraus $f(a+b)=u+w$ und $f(\lambda a)=\lambda a$
für jedes $\lambda\in\Bbbk$.
Damit kann man jetzt 
\begin{align*}
g(u+w)&=g(f(a)+f(b)) = g(f(a+b)) = a+b = g(u)+g(w)
\\
g(\lambda u) &= g(\lambda f(a))=g(f(\lambda a)) = \lambda a = \lambda g(u)
\end{align*}
berechnen, was zeigt, dass auch $g$ eine lineare Abbildung ist.
Hat $f$ in geeignet gewählten Basen die Matrix $F$, dann hat die
Umkehrabbildung $g=f^{-1}$ die Matrix $G=F^{-1}$.
Da auch $f(g(y))=y$ gilt für jeden Vektor $y\in V$ folgt, dass $FF^{-1}=E$
und $F^{-1}F=E$.

\subsubsection{Kern und Bild}
Für die Eindeutigkeit der Lösung eines linearen Gleichungssytems
ist entscheidend, ob das zugehörige homogene Gleichungssystem $Ax=0$
eine nichttriviale Lösung hat.
Seine Lösungmenge spielt also eine besondere Rolle, was rechtfertigt,
ihr einen Namen zu geben.

\begin{definition}
\index{Kern}%
Ist $f$ eine lineare Abbildung $U\to V$, dann heisst die Menge
\[
\ker f
=
\{x\in U\;|\; f(x)=0\}
\]
der {\em Kern} oder {\em Nullraum} der linearen Abbildung $f$.
Ist $A \in M_{m\times n}(\Bbbk)$ Matrix, dann gehört dazu eine lineare
Abbildung $f\colon\Bbbk^n\to\Bbbk^m$.
Der Kern oder Nullraum der Matrix $A$ ist die Menge
\[
\ker A
=
\{ x\in\Bbbk^m \;|\; Ax=0\}.
\]
\end{definition}

Der Kern ist ein Unterraum, denn für zwei Vektoren $u,w\in \ker f$ 
\[
\begin{aligned}
f(u+v)&=f(u) + f(v) = 0+0 = 0 &&\Rightarrow& u+v&\in\ker f\\
f(\lambda u)&=\lambda f(u) = \lambda\cdot 0=0&&\Rightarrow& \lambda u&\in\ker f
\end{aligned}
\]
gilt.

Ob ein Gleichungssystem $Ax=b$ überhaupt eine Lösung hat, hängt davon,
ob der Vektor $b$ als Bild der durch $A$ beschriebenen linearen Abbildung
$\Bbbk^n \to \Bbbk^m$ enthalten ist.
Wir definieren daher das Bild einer linearen Abbildung oder Matrix.

\begin{definition}
Ist $f\colon V\to U$ eine lineare Abbildung dann ist das Bild von $f$
der Unterraum 
\[
\operatorname{im}f = \{ f(v)\;|\;v\in V\} \subset U
\]
von $U$.
Das Bild einer $m\times n$-Matrix $A$ ist die Menge
\[
\operatorname{im}A = \{ Av \;|\; v\in\Bbbk^n\} \subset \Bbbk^m.
\]
\end{definition}

Zwei Vektoren $a,b\in\operatorname{im} f$ haben Urbilder $u,w\in V$ mit
$f(u)=a$ und $f(w)=b$.
Für Summe und Multiplikation mit Skalaren folgt
\[
\begin{aligned}
a+b&= f(u)+f(v)=f(u+v) &&\Rightarrow a+b\in\operatorname{im}f\\
\lambda a&=\lambda f(u) = f(\lambda u) &&\Rightarrow \lambda a&\in\operatorname{im}f,
\end{aligned}
\]
also ist auch das Bild $\operatorname{im}f$ ein Unterraum von $U$.
Das Bild der Matrix $A$ ist der Unterraum
\[
\{ x_1f(b_1) + \dots x_n f(b_n) | x_i\in\Bbbk\}
=
\langle f(b_1),\dots,f(b_n)\rangle
=
\langle a_1,\dots,a_n\rangle
\]
von $\Bbbk^m$, aufgespannt von den Spaltenvektoren $a_i$ von $A$.

\subsubsection{Rang und Defekt}
Die Dimensionen von Bild und Kern sind wichtige Kennzahlen einer Matrix.
\begin{definition}
Sei $A$ eine Matrix $A\in M_{m\times n}(\Bbbk)$.
Der {\em Rang} der Matrix $A$ ist die Dimension des Bildraumes von $A$:
$\operatorname{rank}A=\dim\operatorname{im} A$.
\index{Rang einer Matrix}%
Der {\em Defekt} der Matrix $A$ ist die Dimension des Kernes von $A$:
$\operatorname{def}A=\dim\ker A$.
\index{Defekt einer Matrix}%
\end{definition}

Da der Kern mit Hilfe des Gauss-Algorithmus bestimmt werden kann,
können Rang und Defekt aus dem Schlusstableau 
eines homogenen Gleichungssystems mit $A$ als Koeffizientenmatrix
abgelesen werden.

\begin{satz}
Ist $A\in M_{m\times n}(\Bbbk)$ eine $m\times n$-Matrix,
dann gilt
\[
\operatorname{rank}A
=
n-\operatorname{def}A.
\]
\end{satz}

\subsubsection{Quotient}
TODO: $\operatorname{im} A \simeq \Bbbk^m/\ker A$

%
% gruppen.tex
%
% (c) 2021 Prof Dr Andreas Müller, Hochschule Rapeprswil
%
\subsection{Gruppen
\label{buch:grundlagen:subsection:gruppen}}
Die kleinste sinnvolle Struktur ist die einer Gruppe.
Eine solche besteht aus einer Menge $G$ mit einer Verknüpfung,
die additiv
\index{additive Verknüpfung}%
\begin{align*}
G\times G \to G&: (g,h) = g+h
\intertext{oder multiplikativ }
G\times G \to G&: (g,h) = gh
\end{align*}
\index{multiplikative Verknüpfung}%
geschrieben werden kann.
Ein Element $0\in G$ heisst {\em neutrales Element} bezüglich der additiv
\index{neutrales Element}%
geschriebenen Verknüpfung falls $0+x=x$ für alle $x\in G$.
\index{neutrales Element}%
Ein Element $e\in G$ heisst neutrales Element bezüglich der multiplikativ 
geschriebneen Verknüpfung, wenn $ex=x$ für alle $x\in G$.
In den folgenden Definitionen werden wir immer die multiplikative
Schreibweise verwenden, für Fälle additiv geschriebener Verknüpfungen
siehe auch die Beispiele weiter unten.

\begin{definition}
\index{Gruppe}%
Ein {\em Gruppe}
\index{Gruppe}%
ist eine Menge $G$ mit einer Verknüfung mit folgenden
Eigenschaften:
\begin{enumerate}
\item
Die Verknüpfung ist assoziativ: $(ab)c=a(bc)$ für alle $a,b,c\in G$.
\index{assoziativ}%
\item
Es gibt ein neutrales Element $e\in G$.
\item
Für jedes Element $g\in G$ gibt es ein Element $h\in G$ mit 
$hg=e$.
\end{enumerate}
Das Element $h$ heisst auch das inverse Element zu $g$.
\index{inverses Element}%
\end{definition}

Falls nicht jedes Element invertierbar ist, aber wenigstens ein neutrales
Element vorhanden ist, spricht man von einem {\em Monoid}.
\index{Monoid}%
Hat man nur eine Verknüpfung, aber kein neutrales Element,
spricht man oft von einer {\em Halbruppe}.
\index{Halbgruppe}%

\begin{definition}
Eine Gruppe $G$ heisst {\em abelsch}, wenn $ab=ba$ für alle $a,b\in G$.
\end{definition}
\index{abelsch}%

Additiv geschrieben Gruppen werden immer als abelsch angenommen,
multiplikativ geschriebene Gruppen können abelsch oder nichtabelsch sein.

\subsubsection{Beispiele von Gruppen}

\begin{beispiel}
Die Menge $\mathbb{Z}$ mit der Addition ist eine additive Gruppe mit
dem neutralen Element $0$.
Das additive Inverse eines Elementes $a$ ist $-a$.
\end{beispiel}

\begin{beispiel}
Die von Null verschiedenen Elemente $\Bbbk^*=\Bbbk\setminus\{0\}$ (definiert
auf Seite~\pageref{buch:zahlen:def:bbbk*})
eines Zahlekörpers bilden
bezüglich der Multiplikation eine Gruppe mit neutralem Element $1$.
Das multiplikative Inverse eines Elementes $a\in \Bbbk$ mit $a\ne 0$
ist $a^{-1}=\frac1{a}$.
\end{beispiel}

\begin{beispiel}
Die Vektoren $\Bbbk^n$ bilden bezüglich der Addition eine Gruppe mit
dem Nullvektor als neutralem Element.
Betrachtet man $\Bbbk^n$ als Gruppe, verliert man die Multiplikation
mit Skalaren aus den Augen.
$\Bbbk^n$ als Gruppe zu bezeichnen ist also nicht falsch, man
verliert dadurch aber den Blick auf die Multiplikation mit Skalaren.
\end{beispiel}

\begin{beispiel}
Die Menge aller quadratischen $n\times n$-Matrizen $M_n(\Bbbk)$ ist
eine Gruppe bezüglich der Addition mit der Nullmatrix als neutralem
Element.
Bezügich der Matrizenmultiplikation ist $M_n(\Bbbk)$ aber keine
Gruppe, da sich die singulären Matrizen nicht inverieren lassen.
Die Menge der invertierbaren Matrizen
\[
\operatorname{GL}_n(\Bbbk)
=
\{
A\in M_n(\Bbbk) \mid \text{$A$ invertierbar}
\}
\]
ist bezüglich der Multiplikation eine Gruppe.
Die Gruppe $\operatorname{GL}_n(\Bbbk)$ ist eine echte Teilmenge 
von $M_n(\Bbbk)$, die Addition und Multiplikation führen im Allgemeinen
aus der Gruppe heraus, es gibt also keine Mögichkeit, in der Gruppe
$\operatorname{GL}_n(\Bbbk)$ diese Operationen zu verwenden.
\end{beispiel}

\subsubsection{Einige einfache Rechenregeln in Gruppen}
Die Struktur einer Gruppe hat bereits eine Reihe von
Einschränkungen zur Folge.
Zum Beispiel sprach die Definition des neutralen Elements $e$ nur von
Produkten der Form $ex=x$, nicht von Produkten $xe$.
Und die Definition des inversen Elements $h$ von $g$ hat nur
verlangt, dass $gh=e$, es wurde nichts gesagt über das Produkt $hg$.

\begin{satz}
\label{buch:vektorenmatrizen:satz:gruppenregeln}
Ist $G$ eine Gruppe mit neutralem Element $e$, dann gilt
\begin{enumerate}
\item
$xe=x$ für alle $x\in G$
\item
Es gibt nur ein neutrales Element.
\index{neutrales Element}%
Wenn also $f\in G$ mit $fx=x$ für alle $x\in G$, ist dann folgt $f=e$.
\item 
Wenn $hg=e$ gilt, dann auch $gh=e$ und $h$ ist durch $g$ eindeutig bestimmt.
\end{enumerate}
\end{satz}

\begin{proof}[Beweis]
Wir beweisen als Erstes den ersten Teil der Eigenschaft~3.
Sei $h$ die Inverse von $g$, also $hg=e$.
Sei weiter $i$ die Inverse von $h$, also $ih=e$.
Damit folgt jetzt
\[
g
=
eg
=
(ih)g
=
i(hg)
=
ie.
\]
Wendet man dies auf das Produkt $gh$ an, folgt
\begin{equation}
gh
=
(ie)h
=
i(eh)
=
ih
=
e.
\label{buch:gruppen:eqn:gh=e}
\end{equation}
Es ist also nicht nur $hg=e$ sondern immer auch $gh=e$.

Für eine Inverse $h$ von $g$ folgt
aus \eqref{buch:gruppen:eqn:gh=e}
\[
ge
=
g(hg)
=
(gh)g
=
eg
=
g,
\]
dies ist die Eigenschaft~1.

Sind $f$ und $e$ neutrale Elemente, dann folgt
\[
f = fe = e
\]
aus der Eigenschaft~1.

Schliesslich sei $x$ ein beliebiges Inverses von $g$.
Dann ist $xg=e$ und es folgt
$x=xe=x(gh)=(xg)h = eh = h$, es gibt also nur ein Inverses von $g$.
\end{proof}

Der Frage, ob Linksinverse und Rechtsinverse übereinstimmen,
sind wir zum Beispiel bereits in
Abschnitt~\ref{buch:grundlagen:subsection:gleichungssyteme}
begegnet.
Dort haben wir bereits gezeigt, dass nicht nur $AA^{-1}=I$,
sondern auch $A^{-1}A=I$.
Die dabei verwendete Methode war identisch mit dem hier gezeigten
Beweis.
Da die invertierbaren Matrizen eine Gruppe bilden, stellt sich
dieses Resultat jetzt als Spezialfall des
Satzes~\ref{buch:vektorenmatrizen:satz:gruppenregeln} dar.

\subsubsection{Homomorphismen} \label{buch:gruppen:subsection:homomorphismen}
Lineare Abbildung zwischen Vektorräumen zeichnen sich dadurch aus,
dass sie die algebraische Struktur des Vektorraumes respektieren.
Für eine Abbildung zwischen Gruppen heisst dies, dass die Verknüpfung,
das neutrale Element und die Inverse respektiert werden müssen.

\begin{definition}
\label{buch:gruppen:def:homomorphismus}
Ein Abbildung $\varphi\colon G\to H$ zwischen Gruppen heisst ein
{\em Homomorphismus}, wenn 
$\varphi(g_1g_2)=\varphi(g_1)\varphi(g_2)$ für alle $g_1,g_2\in G$ gilt.
\index{Homomorphismus}%
\end{definition}

Der Begriff des Kerns einer linearen Abbildung lässt sich ebenfalls auf
die Gruppensituation erweitern.
Auch hier ist der Kern der Teil der Gruppe, er unter dem 
Homomorphismus ``unsichtbar'' wird.

\begin{definition}
Ist $\varphi\colon G\to H$ ein Homomorphisus, dann ist
\[
\ker\varphi
=
\{g\in G \mid \varphi(g)=e\}
\]
eine Untergruppe.
\index{Kern}%
\end{definition}

\subsubsection{Normalteiler}
Der Kern eines Homomorphismus ist nicht nur eine Untergruppe, er erfüllt
noch eine zusätzliche Bedingung. 
Für jedes $g\in G$ und $h\in\ker\varphi$ gilt 
\[
\varphi(ghg^{-1})
=
\varphi(g)\varphi(h)\varphi(g^{-1})
=
\varphi(g)\varphi(g^{-1})
=
\varphi(gg^{-1})
=
\varphi(e)
=
e
\qquad\Rightarrow\qquad
ghg^{-1}\in\ker\varphi.
\]
Der Kern wird also von der Abbildung $h\mapsto ghg^{-1}$,
der {\em Konjugation}, in sich abgebildet.
\index{Konjugation in einer Gruppe}

\begin{definition}
Eine Untergruppe $H \subset G$ heisst ein {\em Normalteiler},
geschrieben $H \triangleleft G$
wenn $gHg^{-1}\subset H$ für jedes $g\in G$.
\index{Normalteiler}%
\end{definition}

Die Konjugation selbst ist ebenfalls keine Unbekannte, sie ist uns
bei der Basistransformationsformel
\eqref{buch:vektoren-und-matrizen:eqn:basiswechselabb}
schon begegnet.
Die Tatsache, dass $\ker\varphi$ unter Konjugation erhalten bleibt,
kann man also interpretieren als eine Eigenschaft, die unter
Basistransformation erhalten bleibt.

\subsubsection{Faktorgruppen}
Ein Unterraum $U\subset V$ eines Vektorraumes gibt Anlass zum
Quotientenraum, der dadurch entsteht, dass man die Vektoren in $U$
zu $0$ kollabieren lässt.
Eine ähnliche Konstruktion könnte man für eine Untergruppe $H \subset G$
versuchen.
Man bildet also wieder die Mengen von Gruppenelementen, die sich um
ein Elemente in $H$ unterscheiden.
Man kann diese Mengen in der Form $gH$ mit $g\in G$ schreiben.

Man möchte jetzt aber auch die Verknüpfung für solche Mengen 
definieren, natürlich so, dass $g_1H\cdot g_2H = (g_1g_2)H$ ist.
Da die Verknüpfung nicht abelsch sein muss, entsteht hier
ein Problem.
Für $g_1=e$ folgt, dass $Hg_2H=g_2H$ sein muss.
Das geht nur, wenn $Hg_2=g_2H$ oder $g_2Hg_2^{-1}=H$ ist, wenn
also $H$ ein Normalteiler ist.

\begin{definition}
Für eine Gruppe $G$ mit Normalteiler  $H\triangleleft G$ ist die
Menge
\[
G/H = \{ gH \mid g\in G\}
\]
eine Gruppe mit der Verknüpfung $g_1H\cdot g_2H=(g_1g_2)H$.
$G/H$ heisst {\em Faktorgruppe} oder {\em Quotientengruppe}.
\index{Faktorgruppe}%
\index{Quotientengruppe}%
\end{definition}

Für abelsche Gruppen ist die Normalteilerbedingung keine zusätzliche
Einschränkung, jeder Untergruppe ist auch ein Normalteiler.

\begin{beispiel}
Die ganzen Zahlen $\mathbb{Z}$ bilden eine abelsche Gruppe und
die Menge der Vielfachen von $n$
$n\mathbb{Z}\subset\mathbb{Z}$ ist eine Untergruppe.
Da $\mathbb{Z}$ abelsch ist, ist $n\mathbb{Z}$ ein Normalteiler
und die Faktorgruppe $\mathbb{Z}/n\mathbb{Z}$ ist wohldefiniert.
Nur die Elemente
\[
0+n\mathbb{Z},
1+n\mathbb{Z},
2+n\mathbb{Z},
\dots
(n-1)+n\mathbb{Z}
\]
sind in der Faktorgruppe verschieden.
Die Gruppe $\mathbb{Z}/n\mathbb{Z}$ besteht also aus den Resten
bei Teilung durch $n$.
Diese Gruppe wird in Kapitel~\ref{buch:chapter:endliche-koerper}
genauer untersucht.
\end{beispiel}

Das Beispiel suggeriert, dass man sich die Elemente von $G/H$
als Reste vorstellen kann.

\subsubsection{Darstellungen}
Abstrakt definierte Gruppen können schwierig zu verstehen sein.
Oft hilft es, wenn man eine geometrische Darstellung der Gruppenoperation
finden kann.
Die Gruppenelemente werden dann zu umkehrbaren linearen Operationen
auf einem geeigneten Vektorraum.

\begin{definition}
\label{buch:vektorenmatrizen:def:darstellung}
Eine {\em Darstellung} einer Gruppe $G$ ist ein Homomorphismus 
$G\to\operatorname{GL}_n(\mathbb{R})$.
\index{Darstellung}
\end{definition}

\begin{beispiel}
Die Gruppen $\operatorname{GL}_n(\mathbb{Z})$,
$\operatorname{SL}_n(\mathbb{Z})$ oder $\operatorname{SO}(n)$ 
sind alle Teilmengen von $\operatorname{GL}_n(\mathbb{R})$.
Die Einbettungsabbildung $G\hookrightarrow \operatorname{GL}_n(\mathbb{R})$
ist damit automatisch eine Darstellung, sie heisst auch die
{\em reguläre Darstellung} der Gruppe $G$.
\index{reguläre Darstellung}%
\index{Darstellung, reguläre}%
\end{beispiel}

In Kapitel~\ref{buch:chapter:permutationen} wird gezeigt, 
dass Permutationen einer endlichen Menge eine Gruppe bilden und wie
sie durch Matrizen dargestellt werden können.







%
% ringe.tex -- Grundlegende Konstruktionen für Ringe
%
% (c) 2021 Prof Dr Andreas Müller, OST Ostschweizer Fachhochschule
%
\subsection{Ringe und Moduln
\label{buch:grundlagen:subsection:ringe}}
Die ganzen Zahlen haben ausser der Addition mit neutralem Element $0$
auch noch eine Multiplikation mit dem neutralen Element $1$.
Die Multiplikation ist aber nicht immer invertierbar und zwar
nicht nur für $0$.
Eine ähnliche Situation haben wir bei $M_n(\Bbbk)$ angetroffen.
$M_n(\Bbbk)$ ist eine zunächst eine Gruppe bezüglich der Addition,
hat aber auch noch eine Multiplikation, die nicht immer umkehrbar ist.
Diese Art von Struktur nennt man einen Ring.
\index{Ring}

\subsubsection{Definition eines Rings}

\begin{definition}
\index{Ring}%
Eine Menge $R$ mit einer additiven Operation $+$ mit neutralem Element
$0$ und einer multiplikativ geschriebenen Operation $\cdot$ heisst ein
{\em Ring}, wenn folgendes gilt.
\index{Ring}%
\begin{enumerate}
\item
$R$ ist eine Gruppe bezüglich der Addition.
\item
$R\setminus\{0\}$ ist eine Halbgruppe.
\item
Es gelten die {\em Distributivgesetze}
\[
a(b+c)=ab+ac
\qquad\text{und}\qquad
(a+b)c=ac+bc
\]
für beliebige Elemente $a,b,c\in R$.
\index{Distributivgesetz}%
\end{enumerate}
\end{definition}

Die Distributivgesetze stellen sicher, dass man in $R$ beliebig
ausmultiplizieren kann.
Man kann also so rechnen kann, wie man sich das gewohnt ist.
Es stellt auch sicher, dass die Multiplikation mit $0$ immer $0$
ergibt, denn es ist
\[
r0 = r(a-a) = ra-ra=0.
\]

Man beachte, dass weder verlangt wurde, dass die Multiplikation
ein neutrales Element hat oder kommutativ ist.
Der Ring $\mathbb{Z}$ erfüllt beide Bedingungen.
Die Beispiele weiter unten werden zeigen, dass es auch Ringe gibt,
in denen die Multiplikation nicht kommutativ ist, die Multiplikation
kein neutrales Element hat oder beides.

\begin{definition}
\index{Ring!mit Eins}%
Ein Ring $R$ heisst ein {\em Ring mit Eins}, wenn die Multiplikation ein
neutrales Element hat.
\index{Ring mit Eins}%
\end{definition}

\begin{definition}
\index{Ring!kommutativ}%
\index{kommutativer Ring}%
Ein Ring $R$ heisst {\em kommutativ}, wenn die Multiplikation kommutativ
ist.
\end{definition}

\subsubsection{Beispiele von Ringen}

\begin{beispiel}
Alle Zahlenkörper aus Kapitel~\ref{buch:chapter:zahlen} sind kommutative
Ringe mit Eins.
\end{beispiel}

\begin{beispiel}
Die Menge $c(\mathbb{Z})$ der Folgen $(a_n)_{n\in\mathbb{N}}$ mit
Folgengliedern in $\mathbb{Z}$ wird eine Ring, wenn man die Addition
und Multiplikation elementweise definiert, also
\begin{align*}
&\text{Addition:}
&
a+b&\text{\;ist die Folge mit Folgengliedern}&
(a+b)_n &= a_nb_n \quad\text{für alle $n\in\mathbb{N}$}
\\
&\text{Multiplikation:}
&
a\cdot b&\text{\;ist die Folge mit Folgengliedern}&
(a\cdot b)_n &=  a_nb_n \quad\text{für alle $n\in\mathbb{N}$}
\end{align*}
für $a,b\in c(\mathbb{Z})$.
Die Algebra ist kommutativ und hat die konstante Folge 
$u_n = 1\;\forall n$ als Eins.

Wir betrachten jetzt den Unterring $c_0(\mathbb{Z})\subset c(\mathbb{Z})$
bestehend aus den Folgen, die nur für endlich viele Folgenglieder von
$0$ verschieden sind.
Für eine Folge $a\in c_0(\mathbb{Z})$ gibt es eine Zahl $N$ derart, dass
$a_n=0$ für $n\ge N$.
Die konstante Folge $u_n=1$, die in $c(\mathbb{Z})$ erfüllt diese
Bedingung nicht, die Eins des Ringes $c(\mathbb{Z})$ ist also nicht in
$c_0(\mathbb{Z})$.
$c_0(\mathbb{Z})$ ist immer noch ein Ring, aber er hat kein Eins.
\end{beispiel}

\begin{beispiel}
\begin{figure}
\centering
\includegraphics{chapters/10-vektorenmatrizen/images/gausszahlen.pdf}
\caption{Der Ring der ganzen Gausschen Zahlen besteht aus den ganzahligen
Gitterpunkten in der Gausschen Zahlenebene
\label{buch:vektorenmatrizen:fig:ganzgauss}}
\end{figure}
Die Menge
\[
\mathbb{Z} + i\mathbb{Z}
=
\{a+bi\;|\; a,b\in\mathbb{Z}\}
=
\mathbb{Z}[i]
\subset
\mathbb{C}
\]
ist eine Teilmenge von $\mathbb{C}$ und erbt natürlich die 
arithmetischen Operationen.
Die Summe zweier solcher Zahlen $a+bi\in\mathbb{Z}[i]$ und
$c+di\in\mathbb{Z}[i]$ ist
$(a+bi)+(c+di)=(a+c) + (b+d)i\in \mathbb{Z}[i]$, weil $a+c\in\mathbb{Z}$
und $b+d\in\mathbb{Z}$ ganze Zahlen sind.
Ebenso ist das Produkt dieser Zahlen
\(
(a+bi)(c+di)
=
(ac-bd) + (ad+bc)i
\in \mathbb{Z}[i]
\)
weil Realteil $ac-bd\in\mathbb{Z}$ und der Imaginärteil $ad+bc\in\mathbb{Z}$
ganze Zahlen sind.
Die Menge $\mathbb{Z}[i]$ ist also ein kommutative Ring mit Eins, er
heisst der Ring der {\em ganzen Gaussschen Zahlen}.
\index{ganze Gausssche Zahlen}%
\end{beispiel}

\begin{beispiel}
Die Menge der Matrizen $M_n(\mathbb{Z})$ ist ein Ring mit Eins.
Für $n>1$ ist er nicht kommutativ.
Der Ring $M_2(\mathbb{Z})$ enthält den Teilring
\[
G
=
\biggl\{
\begin{pmatrix}
a&-b\\b&a
\end{pmatrix}
\;\bigg|\;
a,b\in\mathbb{Z}
\biggr\}
=
\mathbb{Z}+ \mathbb{Z}J
\subset
M_2(\mathbb{Z}).
\]
Da die Matrix $J$ die Relation $J^2=-E$ erfüllt, ist der Ring $G$
nichts anderes als der Ring der ganzen Gaussschen Zahlen.
Der Ring $\mathbb{Z}[i]$ ist also ein Unterring des Matrizenrings
$M_2(\mathbb{Z})$.
\end{beispiel}

\subsubsection{Einheiten}
In einem Ring mit Eins sind normalerweise nicht alle von $0$ verschiedenen
Elemente intertierbar.
Die Menge der von $0$ verschiedenen Elemente in $R$ wir mit $R^*=R\setminus\{0\}$
bezeichnet.
\index{R*@$R^*$}%
Die Menge der invertierbaren Elemente verdient einen besonderen Namen.

\begin{definition}
Ist $R$ ein Ring mit Eins, dann heissen die Elemente von
\[
U(R) = \{ r\in R \;|\; \text{$r$ in $R$ invertierbar}\}.
\]
die {\em Einheiten} von $R$.
\index{Einheit}%
\end{definition}

\begin{satz}
$U(R)$ ist eine Gruppe, die sogenannte {\em Einheitengruppe}.
\index{Einheitengruppe}%
\end{satz}

\begin{beispiel}
Die Menge $M_2(\mathbb{Z})$ ist ein Ring mit Eins, die Einheitengruppe
besteht aus den invertierbaren $2\times 2$-Matrizen. 
Aus der Formel für 
\[
\begin{pmatrix}
a&b\\
c&d
\end{pmatrix}^{-1}
=
\frac{1}{ad-bc}\begin{pmatrix}
d&-b\\
-c&a
\end{pmatrix}
\]
zeigt, dass $U(M_2(\mathbb{Z})) = \operatorname{SL}_2(\mathbb{Z})$.
\end{beispiel}

\begin{beispiel}
Die Einheitengruppe von $M_n(\Bbbk)$ ist die allgemeine lineare Gruppe 
$U(M_n(\Bbbk))=\operatorname{GL}_n(\Bbbk)$.
\end{beispiel}

\subsubsection{Nullteiler}
Ein möglicher Grund, warum ein Element $r\in R$ nicht invertierbar
ist, kann sein, dass es ein Element $s\in R$ mit $rs=0$ gibt.
Wäre nämlich $t$ ein inverses Element, dann wäre $0=t0 = t(rs) = (tr)s=s$.

\begin{definition}
\label{buch:grundlagen:def:nullteiler}
Ein Element $r\in R^*$ heisst ein {\em Nullteiler} in $R$,
wenn es ein $s\in R^*$ gibt mit $rs=0$
Ein Ring ohne Nullteiler heisst {\em nullteilerfrei}.
\end{definition}
\index{Nullteiler}%
\index{nullteilerfrei}%

In $\mathbb{R}$ ist man sich gewohnt zu argumentieren, dass wenn ein
Produkt $ab=0$ ist, dann muss einer der Faktoren $a=0$ oder $b=0$ sein.
Dieses Argument funktioniert nur, weil $\mathbb{R}$ ein nullteilerfreier
Ring ist.
In $M_2(\mathbb{R})$ ist dies nicht mehr möglich.
Die beiden Matrizen
\[
A=\begin{pmatrix}
1&0\\0&0
\end{pmatrix}
,\qquad
B=\begin{pmatrix}
0&0\\0&1
\end{pmatrix}
\qquad\Rightarrow\qquad
AB=0
\]
sind Nullteiler in $M_2(\mathbb{Z})$.

\subsubsection{Homomorphismus}
Eine Abbildung zwischen Ringen muss die algebraische Struktur respektieren,
wenn sich damit Eigenschaften vom einen Ring auf den anderen transportieren
lassen sollen.

\begin{definition}
Eine Abbildung $\varphi:R \to S$ zwischen Ringen heisst ein
{\em Homomorphismus}
\index{Homomorphismus}%
oder {\em Ringhomomorphismus},
\index{Ringhomomorphismus}%
wenn $\varphi$ ein Gruppenhomomorphismus der additiven Gruppen der Ringe
ist und ausserdem gilt
\[
\varphi(r_1r_2) = \varphi(r_1)\varphi(r_2).
\]
Der Kern ist die Menge
\[
\ker\varphi = \{ r\in R\;|\; \varphi(r)=0\}
\]
\index{Kern}%
\end{definition}

Wieder hat der Kern zusätzliche Eigenschaften.
Er ist natürlich bezüglich der additiven Struktur des Ringes ein
Normalteiler, aber weil die additive Gruppe ja abelsch ist, ist das
keine wirkliche Einschränkung.
Für ein beliebiges Element $r\in R$ und $k\in \ker\varphi$ gilt
\begin{align*}
\varphi(kr) &= \varphi(k)\varphi(r) = 0\cdot\varphi(r) = 0
\\
\varphi(rk) &= \varphi(r)\varphi(k) = \varphi(r)\cdot 0 = 0.
\end{align*}
Für den Kern gilt also, dass $\ker\varphi\cdot R\subset \ker\varphi$
und $R\cdot\ker\varphi\subset\ker\varphi$.

\subsubsection{Ideale}
\begin{figure}
\centering
\includegraphics{chapters/10-vektorenmatrizen/images/ideale.pdf}
\caption{Ideale im Ring der ganzen Gaussschen Zahlen $\mathbb{Z}[i]$.
Für jedes Element $r\in \mathbb{Z}[i]$ ist die Menge  $r\mathbb{Z}[i]$
ein ein Ideal in $\mathbb{Z}[i]$.
Links das Ideal $(1+2i)\mathbb{Z}[i]$ (blau), rechts das Ideal
$(1+i)\mathbb{Z}[i]$ (rot).
\label{buch:vektorenmatrizen:fig:ideale}}
\end{figure}
Bei der Betrachtung der additiven Gruppe des Ringes $\mathbb{Z}$ der
ganzen Zahlen wurde bereits die Untergruppe $n\mathbb{Z}$ diskutiert
und die Faktorgruppe $\mathbb{Z}/n\mathbb{Z}$ der Reste konstruiert.
Reste können aber auch multipliziert werden, es muss also auch möglich
sein, der Faktorgruppe eine multiplikative Struktur zu verpassen.

Sei jetzt also $I\subset R$ ein Unterring.
Die Faktorgruppe $R/I$ hat bereits die additive Struktur, es muss
aber auch die Multiplikation definiert werden.
Die Elemente $r_1+I$ und $r_2+I$ der Faktorgruppe $R/I$ haben das
Produkt
\[
(r_1+I)(r_2+I)
=
r_1r_2 + r_1I + Ir_2 + II.
\]
Dies stimmt nur dann mit $r_1r_2+I$ überein, wenn $r_1I\subset I$ und
$r_2I\subset I$ ist.

\begin{definition}
Ein Unterring $I\subset R$ heisst ein {\em Ideal}, wenn für jedes $r\in R$ gilt
$rI\subset I$ und $Ir\subset I$ gilt.
\index{Ideal}%
Die Faktorgruppe $R/I$ erhält eine natürliche Ringstruktur, $R/I$ 
heisst der {\em Quotientenring}.
\index{Quotientenring}%
\end{definition}

\begin{beispiel}
Die Menge $n\mathbb{Z}\subset\mathbb{Z}$ besteht aus den durch $n$ teilbaren
Zahlen.
Multipliziert man durch $n$ teilbare Zahlen mit einer ganzen Zahl,
bleiben sie durch $n$ teilbar, $n\mathbb{Z}$ ist also ein Ideal in
$\mathbb{Z}$.
Der Quotientenring ist der Ring der Reste bei Teilung durch $n$,
er wird in 
Kapitel~\ref{buch:chapter:endliche-koerper}
im Detail untersucht.
\end{beispiel}

Ein Ideal $I\subset R$ drückt als die Idee ``gemeinsamer Faktoren''
auf algebraische Weise aus und der Quotientenring $R/I$ beschreibt
das, was übrig bleibt, wenn man diese Faktoren ignoriert.

\begin{beispiel}
In Abbildung~\ref{buch:vektorenmatrizen:fig:ideale} sind zwei
Ideale im Ring der ganzen Gaussschen Zahlen dargestellt.
Die blauen Punkte sind $I_1=(1+2i)\mathbb{Z}$ und die roten Punkte sind
$I_2=(1+i)\mathbb{Z}$.
Die Faktorgruppen $R/I_1$ und $R/I_2$ fassen jeweils Punkte, die sich
um ein Element von $I_1$ bzw.~$I_2$ unterscheiden, zusammen.

Im Falle von $I_2$ gibt es nur zwei Arten von Punkten, nämlich
die roten und die schwarzen, der Quotientenring hat
daher nur zwei Elemente, $R/I_2 = \{0+I_2,1+I_2\}$.
Wegen $1+1=0$ in diesem Quotientenring, ist $R/I_2=\mathbb{Z}/2\mathbb{Z}$.

Im Falle von $I_1$ gibt es fünf verschiedene Punkte, als Menge ist
\[
R/I_1 
=
\{
0+I_1,
1+I_1,
2+I_1,
3+I_1,
4+I_1
\}.
\]
Die Rechenregeln sind also dieselben wie im Ring $\mathbb{Z}/5\mathbb{Z}$.
In gewisser Weise verhält sich die Zahl $1+2i$ in den ganzen 
Gaussschen Zahlen bezüglich Teilbarkeit ähnlich wie die Zahl $5$ in den
ganzen Zahlen.
\end{beispiel}


%
% algebren.tex -- Grundlegende Konstruktionen für Algebren
%
% (c) 2021 Prof Dr Andreas Müller, OST Ostschweizer Fachhochschule
%
\subsection{Algebren
\label{buch:grundlagen:subsection:algebren}}

\subsubsection{Die Algebra der Funktionen $\Bbbk^X$}
Sie $X$ eine Menge und $\Bbbk^X$ die Menge aller Funktionen $X\to \Bbbk$.
Auf $\Bbbk^X$ kann man Addition, Multiplikation mit Skalaren und
Multiplikation von Funktionen punktweise definieren.
Für zwei Funktion $f,g\in\Bbbk^X$ und $\lambda\in\Bbbk$ definiert man
\[
\begin{aligned}
&\text{Summe $f+g$:}
&
(f+g)(x) &= f(x)+g(x)
\\
&\text{Skalare $\lambda f$:}
&
(\lambda f)(x) &= \lambda f(x)
\\
&\text{Produkt $f\cdot g$:}
&
(f\cdot g)(x) &= f(x) g(x)
\end{aligned}
\]
Man kann leicht nachprüfen, dass die Menge der Funktionen $\Bbbk^X$
mit diesen Verknüfungen die Struktur einer $\Bbbk$-Algebra erhält.

Die Algebra der Funktionen $\Bbbk^X$ hat auch ein Einselement:
die konstante Funktion
\[
1\colon [a,b] \to \Bbbk : x \mapsto 1
\]
mit Wert $1$ erfüllt
\[
(1\cdot f)(x) = 1(x) f(x) = f(x)
\qquad\Rightarrow\qquad 1\cdot f = f,
\]
die Eigenschaft einer Eins in der Algebra.

\subsubsection{Die Algebra der stetigen Funktionen $C([a,b])$}
Die Menge der stetigen Funktionen $C([a,b])$ ist natürlich eine Teilmenge
aller Funktionen: $C([a,b])\subset \mathbb{R}^{[a,b]}$ und erbt damit
auch die Algebraoperationen.
Man muss nur noch sicherstellen, dass die Summe von stetigen Funktionen,
das Produkt einer stetigen Funktion mit einem Skalar und das Produkt von
stetigen Funktionen wieder eine stetige Funktion ist.
Eine Funktion ist genau dann stetig, wenn an jeder Stelle der Grenzwert
mit dem Funktionswert übereinstimmt.
Genau dies garantieren die bekannten Rechenregeln für stetige Funktionen.
Für zwei stetige Funktionen $f,g\in C([a,b])$ und einen Skalar
$\lambda\in\mathbb{R}$ gilt
\[
\begin{aligned}
&\text{Summe:}
&
\lim_{x\to x_0} (f+g)(x)
&=
\lim_{x\to x_0} (f(x)+g(x))
=
\lim_{x\to x_0} f(x) + \lim_{x\to x_0}g(x)
=
f(x_0)+g(x_0) = (f+g)(x_0)
\\
&\text{Skalare:}
&
\lim_{x\to x_0} (\lambda f)(x)
&=
\lim_{x\to x_0} (\lambda f(x)) = \lambda \lim_{x\to x_0} f(x)
=
\lambda f(x_0) = (\lambda f)(x_0)
\\
&\text{Produkt:}
&
\lim_{x\to x_0}(f\cdot g)(x)
&=
\lim_{x\to x_0} f(x)\cdot g(x)
=
\lim_{x\to x_0} f(x)\cdot
\lim_{x\to x_0} g(x)
=
f(x_0)g(x_0)
=
(f\cdot g)(x_0).
\end{aligned}
\]
für jeden Punkt $x_0\in[a,b]$.
Damit ist $C([a,b])$ eine $\mathbb{R}$-Algebra.
Die Algebra hat auch eine Eins, da die konstante Funktion $1(x)=1$ 
stetig ist.






%
% koerper.tex -- Definition eines Körpers
%
% (c) 2021 Prof Dr Andreas Müller, OST Ostschwêizer Fachhochschule
%
\subsection{Körper
\label{buch:subsection:koerper}}
Die Multiplikation ist in einer Algebra nicht immer umkehrbar.
Die Zahlenkörper von Kapitel~\ref{buch:chapter:zahlen} sind also
sehr spezielle Algebren, man nennt sie Körper.
In diesem Abschnitt sollen die wichtigsten Eigenschaften von Körpern
zusammengetragen werden.

\begin{definition}
Ein Körper $K$ ist ein additive Gruppe mit einer multiplikativen
Verknüpfung derart, dass $K^* = K \setminus \{0\}$ eine Gruppe bezüglich
der Multiplikation ist.
Ausserdem gelten die Distributivgesetze 
\[
(a+b)c = ac+bc
\qquad a,b,c\in K.
\]
\end{definition}

Ein Körper ist also ein Ring derart, dass die Einheitengruppe $K^*$ ist.

\begin{beispiel}
Die Menge $\mathbb{F}_2=\{0,1\}$ mit der Additions- und
Mutliplikationstabelle
\begin{center}
\begin{tabular}{|>{$}c<{$}|>{$}c<{$}>{$}c<{$}|}
\hline
+&0&1\\
\hline
0&0&1\\
1&1&0\\
\hline
\end{tabular}
\qquad
\qquad
\qquad
\begin{tabular}{|>{$}c<{$}|>{$}c<{$}>{$}c<{$}|}
\hline
\cdot&0&1\\
\hline
0&0&0\\
1&0&1\\
\hline
\end{tabular}
\end{center}
ist der kleinste mögliche Körper.
\end{beispiel}

\begin{beispiel}
Die Menge der rationalen Funktionen
\[
\mathbb{Q}(z)
=
\biggl\{
f(z)
=
\frac{p(z)}{q(z)}
\,
\bigg|
\,
\begin{minipage}{5.5cm}
\raggedright
$p(z), q(z)$ sind Polynome mit rationalen Koeffizienten, $q(z)\ne 0$
\end{minipage}
\,
\biggr\}
\]
ist ein Körper.
\end{beispiel}





\section*{Übungsaufgaben}
\aufgabetoplevel{chapters/10-vektorenmatrizen/uebungsaufgaben}
\begin{uebungsaufgaben}
\uebungsaufgabe{1001}
\uebungsaufgabe{1002}
\end{uebungsaufgaben}

