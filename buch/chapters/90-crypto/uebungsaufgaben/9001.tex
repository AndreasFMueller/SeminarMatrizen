$A$ und $B$ einigen sich darauf, das Diffie-Hellman-Verfahren für
$p=2027$ durchzuführen und mit $g=3$ zu arbeiten.
$A$ verwenden $a=49$ als privaten Schlüssel und erhält von $B$
den öffentlichen Schlüssel $y=1772$.
Welchen gemeinsamen Schlüssel verwenden $A$ und $B$?

\begin{loesung}
Der zu verwendende gemeinsame Schlüssel ist
$g^{ab}=(g^b)^a = y^a\in\mathbb{F}_{2027}$.
Diese Potenz kann man mit dem Divide-and-Conquer-Algorithmus effizient
berechnen.
Die Binärdarstellung des privaten Schlüssels von $A$ ist
$a=49_{10}=\texttt{110001}_2$.
Der Algorithmus verläuft wie folgt:
\begin{center}
\begin{tabular}{|>{$}r<{$}|>{$}r<{$}|>{$}r<{$}|>{$}r<{$}|}
\hline
i&g^{2^i}&a_i&     x\\
\hline
0&      3&  1&     3\\
1&      9&  0&     3\\
2&     81&  0&     3\\
3&    480&  0&     3\\
4&   1349&  1&  2020\\
5&   1582&  1&  1088\\
\hline
\end{tabular}
\end{center}
Der gemeinsame Schlüssel ist daher $s=1088$.
\end{loesung}

