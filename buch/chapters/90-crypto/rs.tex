%
% rs.tex -- Reed-Solomon-Code
%
% (c) 2020 Prof Dr Andreas Müller, Hochschule Rapperswil
%
\section{Fehlerkorrigierende Codes nach Reed-Solomon
\label{buch:section:reed-solomon}}
\rhead{Fehlerkorrigierende Codes}
Jede Art von Datenübertragung muss sich mit dem Problem der Fehler befassen,
die auf dem Übertragungskanal entstehen können.
Die einfachste Lösung dieses Problem versucht, Fehler zu erkennen und 
dann eine erneute Übermittelung zu veranlassen.
Dies ist zum Beispiel bei der Datenübertragung von einer Raumsonde
wie Voyager~1 nicht möglich, die Signallaufzeit von der Sonde und wieder
zurück ist über 40 Stunden.
Es ist auch nicht sinnvoll beim Lesen eines optischen Mediums wie einer
CD oder DVD, wenn ein Fehler durch eine Beschädigung der Oberfläche
des Mediums verursacht wird.
Erneutes Lesen würde das Resultat auch nicht ändern.
Es wird also eine Möglichkeit gesucht, die Daten so zu codieren, dass
ein Fehler nicht nur erkannt sondern auch korrigiert werden kann.

In diesem Abschnitt werden die algebraisch besonders interessanten
Reed-Solmon-Codes beschrieben.
Ihren ersten Einsatz hatten Sie bei den Voyager-Raumsonden, die 1977
gestartet wurden.
Sie befinden sich im Moment in einer Entfernung von 
Zum ersten mal kommerziell verwendet wurden sie für die optischen
Medien CD und DVD.

% https://www.youtube.com/watch?v=uOLW43OIZJ0
% https://www.youtube.com/watch?v=4BfCmZgOKP8

\subsection{Was ist ein Code?
\label{buch:subsection:was-ist-ein-code}}

\subsection{Reed-Solomon-Code
\label{buch:subsection:reed-solomon-code}}

\subsection{Decodierung
\label{buch:subsection:decodierung}}
