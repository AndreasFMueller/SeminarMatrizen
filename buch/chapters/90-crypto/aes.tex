%
% aes.tex -- Beschreibung des AES Algorithmus
%
% (c) 2020 Prof Dr Andreas Müller, Hochschule Rapperswil
%
\section{Advanced Encryption Standard -- AES
\label{buch:section:aes}}
\rhead{Advanced Encryption Standard}
Eine wichtige Forderung bei der Konzeption des damals neuen
Advanced Encryption Standard war, dass darin keine ``willkürlich''
erscheinenden Operationen geben darf, bei denen der Verdacht
entstehen könnte, dass sich dahinter noch offengelegtes Wissen
über einen möglichen Angriff auf den Verschlüsselungsalgorithmus
verbergen könnte.
Dies war eine Schwäche des vor AES üblichen DES Verschlüsselungsalgorithmus.
In seiner Definition kommt eine Reihe von Konstanten vor, über deren
Herkunft nichts bekannt war.
Die Gerüchteküche wollte wissen, dass die NSA die Konstanten aus dem
ursprünglichen Vorschlag abgeändert habe, und dass dies geschehen sei,
um den Algorithmus durch die NSA angreifbar zu machen.

Eine weiter Forderung war, dass die Sicherheit des neuen
Verschlüsselungsstandards ``skalierbar'' sein soll, dass man also
die Schlüssellänge mit der Zeit von 128~Bit auf 196 oder sogar 256~Bit
steigern kann.
Der Standard wird dadurch langlebiger und gleichzeitig entsteht die
Möglichkeit, Sicherheit gegen Rechenleistung einzutauschen.
Weniger leistungsfähige Systeme können den Algorithmus immer noch
nutzen, entweder mit geringerer Verschlüsselungsrate oder geringerer
Sicherheit.

In diesem Abschnitt soll gezeigt werde, wie sich die AES
spezifizierten Operationen als mit der Arithmetik der
endlichen Körper beschreiben lassen.

