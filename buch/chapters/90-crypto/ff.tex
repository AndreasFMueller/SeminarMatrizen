%
% ff.tex -- Kryptographie und endliche Körper
%
% (c) 2020 Prof Dr Andreas Müller, Hochschule Rapperswil
%

\section{Kryptographie und endliche Körper
\label{buch:section:kryptographie-und-endliche-koerper}}
\rhead{Kryptographie und endliche Körper}

\subsection{Potenzen in $\mathbb{F}_p$ und diskreter Logarithmus
\label{buch:subsection:potenzen-diskreter-logarithmus}}
Für kryptographische Anwendungen wird eine einfach zu berechnende
Funktion benötigt,
die ohne zusätzliches Wissen, üblicherweise der Schlüssel genannt,
nicht ohne weiteres umkehrbar ist.
Die arithmetischen Operationen in einem endlichen Körper sind
mit geringem Aufwand durchführbar.
Für die ``schwierigste'' Operation, die Division, steht der
euklidische Algorithmus zur Verfügung.

Die nächstschwierigere Operation ist die Potenzfunktion.
Für $g\in \Bbbk$ und $a\in\mathbb{N}$ ist die Potenz $g^a\in\Bbbk$
natürlich durch die wiederholte Multiplikation definiert.
In der Praxis werden aber $g$ und $a$ Zahlen mit vielen Binärstellen
sein, die die wiederholte Multiplikation ist daher sicher nicht
effizient, das Kriterium der einfachen Berechenbarkeit scheint
also nicht erfüllt.
Der folgende Algorithmus berechnet die Potenz in $O(\log_2 a$
Multiplikationen.

\begin{algorithmus}[Divide-and-conquer]
\label{buch:crypto:algo:divide-and-conquer}
Sei $a=a_0 + a_12^1 + a_22^2 + \dots + a_k2^k$ die Binärdarstellung
der Zahl $a$.
\begin{enumerate}
\item setze $f=g$, $x=1$, $i=0$
\label{divide-and-conquer-1}
\item solange $i\ge k$ ist, führe aus
\label{divide-and-conquer-2}
\begin{enumerate}
\item
\label{divide-and-conquer-3}
falls $a_i=1$ setze $x \coloneqq x \cdot f$
\item
\label{divide-and-conquer-4}
$i \coloneqq i+1$ und $f\coloneqq f\cdot f$
\end{enumerate}
\end{enumerate}
Die Potenz $x=g^a$ kann so in $O(\log_2a)$ Multiplikationen
berechnet werden.
\end{algorithmus}

\begin{proof}[Beweis]
Die Initalisierung in Schritt~\ref{divide-and-conquer-1} stellt sicher,
dass $x$ den Wert $g^0$ hat. 
Schritt~\ref{divide-and-conquer-4} stellt sicher,
dass die Variable $f$ immer den Wert $g^{2^i}$ hat.
Im Schritt~\ref{divide-and-conquer-3} wird zu $x$ die Potenz
$g^{a_i2^i}$ hinzumultipliziert.
Am Ende des Algorithmus hat daher $x$ den Wert
\[
x = g^{a_02^0} \cdot g^{a_12^1} \cdot g^{a_22^2} \cdot\ldots\cdot 2^{a_k2^k}
=
g^{a_0+a_12+a_22^2+\dots+a_k2^k}
=
g^a.
\]
Die Schleife wird $\lfloor1+\log_2ab\rfloor$ mal durchlaufen.
In jedem Fall wird auf jeden Fall die Multiplikation in 
Schritt~\ref{divide-and-conquer-4} durchgeführt
und im schlimmsten Fall auch noch die Multiplikation in
Schritt~\ref{divide-and-conquer-3}.
Es werden also nicht mehr als $2\lfloor 1+\log_2a\rfloor=O(\log_2a)$
Multiplikationen durchgeführt.
\end{proof}

\begin{beispiel}
Man berechne die Potenz $7^{2021}$ in $\mathbb{F}_p$.
Die Binärdarstellung von 2021 ist $2021_{10}=\texttt{11111100101}_2$.
Wir stellen die nötigen Operationen des
Algorithmus~\ref{buch:crypto:algo:divide-and-conquer} in der folgenden
Tabelle
\begin{center}
\begin{tabular}{|>{$}r<{$}|>{$}r<{$}|>{$}r<{$}|>{$}r<{$}|}
\hline
 i&   f& a_i&    x\\
\hline
 0&   7&   1&    7\\
 1&  49&   0&    7\\
 2&1110&   1&   24\\
 3& 486&   0&   24\\
 4&1234&   0&   24\\
 5& 667&   1&  516\\
 6& 785&   1&  977\\
 7& 418&   1&  430\\
 8& 439&   1&  284\\
 9& 362&   1&  819\\
10& 653&   1&  333\\
\hline
\end{tabular}
\end{center}
Daraus liest man ab, dass $7^{2021}=333\in\mathbb{F}_{1291}$.
\end{beispiel}

Die Tabelle suggeriert, dass die Potenzen von $g$ ``wild'', also
scheinbar ohne System in $\mathbb{F}_p$ herumspringen.
Dies deutet an, dass die Umkehrung der Exponentialfunktion in $\mathbb{F}_p$
schwierig ist.
Die Umkehrfunktion der Exponentialfunktion, die Umkehrfunktion von 
$x\mapsto g^x$ in $\mathbb{F}_p$ heisst der {\em diskrete Logarithmus}.
\index{diskreter Logarithmus}%
Tatsächlich ist der diskrete Logarithmus ähnlich schwierig zu bestimmen
wie das Faktorisieren von Zahlen, die das Produkt grosser
Primafaktoren ähnlicher Grössenordnung wie $p$ sind.
Die Funktion $x\mapsto g^x$ ist die gesuchte, schwierig zu invertierende
Funktion.

Auf dern ersten Blick scheint der
Algorithmus~\ref{buch:crypto:algo:divide-and-conquer}
den Nachteil zu haben, dass erst die Binärdarstellung der Zahl $a$ 
ermittelt werden muss.
In einem Computer ist dies aber normalerweise kein Problem, da $a$
im Computer ohnehin binär dargestellt ist.
Die Binärziffern werden in der Reihenfolge vom niederwertigsten zum
höchstwertigen Bit benötigt.
Die folgende Modifikation des Algorithmus ermittelt laufend
auch die Binärstellen von $a$.
Die dazu notwendigen Operationen sind im Binärsystem besonders
effizient implementierbar, die Division durch 2 ist ein Bitshift, der
Rest ist einfach das niederwertigste Bit der Zahl.

\begin{algorithmus}
\label{buch:crypto:algo:divide-and-conquer2}
\begin{enumerate}
\item
Setze $f=g$, $x=1$, $i=0$
\item
Solange $a>0$ ist, führe aus
\begin{enumerate}
\item
Verwende den euklidischen Algorithmus um $r$ und $b$ zu bestimmen mit $a=2b+r$
\item
Falls $r=1$ setze $x \coloneqq x \cdot f$
\item
$i \coloneqq i+1$, $a = b$ und $f\coloneqq f\cdot f$
\end{enumerate}
\end{enumerate}
Die Potenz $x=g^a$ kann so in $O(\log_2a)$ Multiplikationen
berechnet werden.
\end{algorithmus}


%
% Diffie-Hellman Schlüsseltausch
%
\subsection{Diffie-Hellman-Schlüsseltausch
\label{buch:subsection:diffie-hellman}}
Eine Grundaufgabe der Verschlüsselung im Internet ist, dass zwei
Kommunikationspartner einen gemeinsamen Schlüssel für die Verschlüsselung
der Daten aushandeln können müssen.
Es muss davon ausgegangen werden, dass die Kommunikation abgehört wird.
Trotzdem soll es für einen Lauscher nicht möglich sein, den 
ausgehandelten Schlüssel zu ermitteln.

% XXX Historisches zu Diffie und Hellman

Die beiden Partner $A$ und $B$ einigen sich zunächst auf eine Zahl $g$,
die öffentlich bekannt sein darf.
Weiter erzeugen sie eine zufällige Zahl $a$ und $b$, die sie geheim
halten.
Das Verfahren soll aus diesen beiden Zahlen einen Schlüssel erzeugen,
den beide Partner berechnen können, ohne dass sie $a$ oder $b$ 
übermitteln müssen.
Die beiden Zahlen werden daher auch die privaten Schlüssel genannt.

Die Idee von Diffie und Hellman ist jetzt, die Werte $x=g^a$ und $y=g^b$
zu übertragen.
In $\mathbb{R}$ würden dadurch natürlich dem Lauscher auch $a$ offenbart,
er könnte einfach $a=\log_g x$ berechnen.
Ebenso kann auch $b$ als $b=\log_g y$ erhalten werden, die beiden
privaten Schlüssel wären also nicht mehr privat.
Statt der Potenzfunktion in $\mathbb{R}$ muss also eine Funktion
verwendet werden, die nicht so leicht umgekehrt werden kann.
Die Potenzfunktion in $\mathbb{F}_p$ erfüllt genau diese Eigenschaft.
Die Kommunikationspartner einigen sich also auch noch auf die (grosse)
Primzahl $p$ und übermitteln $x=g^a\in\mathbb{F}_p$ und
$y=g^b\in\mathbb{F}_p$.

\begin{figure}
\centering
\includegraphics{chapters/90-crypto/images/dh.pdf}
\caption{Schlüsselaustausch nach Diffie-Hellman.
Die Kommunikationspartner $A$ und $B$ einigen sich öffentlich auf
$p\in\mathbb{N}$ und $g\in\mathbb{F}_p$.
$A$ wählt dann einen privaten Schlüssel $a\in\mathbb{N}$ und
$B$ wählt $b\in\mathbb{N}$, sie tauschen dann $x=g^a$ und $y=g^b$
aus.
$A$ erhält den gemeinsamen Schlüssel aus $y^a$, $B$ erhält ihn
aus $x^b$.
\label{buch:crypto:fig:dh}}
\end{figure}

Aus $x$ und $y$ muss jetzt der gemeinsame Schlüssel abgeleitet werden.
$A$ kennt $y=g^b$ und $a$, $B$ kennt $x=g^a$ und $b$.
Beide können die Zahl $s=g^{ab}\in\mathbb{F}_p$ berechnen.
$A$ macht das, indem er $y^a=(g^b)^a = g^{ab}$ rechnet,
$B$ rechnet $x^b = (g^a)^b = g^{ab}$, beide natürlich in $\mathbb{F}_p$.
Der Lauscher kann aber $g^{ab}$ nicht ermitteln, dazu müsste er
$a$ oder $b$ ermitteln können.
Die Zahl $s=g^{ab}$ kann also als gemeinsamer Schlüssel verwendet
werden.



\subsection{Elliptische Kurven
\label{buch:subsection:elliptische-kurven}}
Das Diffie-Hellman-Verfahren basiert auf der Schwierigkeit, in einem 
Körper $\mathbb{F}_p$ die Gleichung $a^x=b$ nach $x$ aufzulösen.
Die Addition in $\mathbb{F}_p$ wird dazu nicht benötigt.
Es reicht, eine Menge mit einer Multiplikation zu haben, in der das
die Gleichung $a^x=b$ schwierig zu lösen ist.
Ein Gruppe wäre also durchaus ausreichend.

Ein Kandidat für eine solche Gruppe könnte der Einheitskreis
$S^1=\{z\in\mathbb{C}\;|\; |z|=1\}$ in der komplexen Ebene sein.
Wählt man eine Zahl $g=e^{i\alpha}$, wobei $\alpha$ ein irrationales
Vielfaches von $\pi$ ist, dann sind alle Potenzen $g^n$ für natürliche
Exponenten voneinander verschieden.
Wäre nämlich $g^{n_1}=g^{n_2}$, dann wäre $e^{i\alpha(n_1-n_2)}=1$ und
somit müsste $\alpha=2k\pi/(n_1-n_2)$ sein.
Damit wäre aber $\alpha$ ein rationales Vielfaches von $\pi$, im Widerspruch
zur Voraussetzung.
Die Abbildung $n\mapsto g^n\in S^1$ ist auf den ersten Blick etwa ähnlich
undurchschaubar wie die Abbildung $n\mapsto g^n\in\mathbb{F}_p$.
Es gibt zwar die komplexe Logarithmusfunktion, mit der man $n$ bestimmen
kann, dazu muss man aber den Wert von $g^n$ mit beliebiger Genauigkeit
kennen, denn die Werte von $g^n$ können beliebig nahe beieinander liegen.

Der Einheitskreis ist die Lösungsmenge der Gleichung $x^2+y^2=1$ für
reelle Koordinaten $x$ und $y$,
doch Rundungsunsicherheiten verunmöglichen den Einsatz in einem 
Verfahren ähnlich dem Diffie-Hellman-Verfahren.
Dieses Problem kann gelöst werden, indem für die Variablen Werte
aus einem endlichen Körper verwendet werden.
Gesucht ist also eine Gleichung in zwei Variablen, deren Lösungsmenge
in einem endlichen Körper eine Gruppenstruktur trägt.
Die Lösungsmenge ist eine ``Kurve'' von Punkten mit
Koordinaten in einem endlichen Körper.

In diesem Abschnitt wird gezeigt, dass sogenannte elliptische Kurven
über endlichen Körpern genau die verlangen Eigenschaften haben.

\subsubsection{Elliptische Kurven}
Elliptische Kurven sind Lösungen einer Gleichung der Form
\begin{equation}
Y^2+XY=X^3+aX+b
\label{buch:crypto:eqn:ellipticcurve}
\end{equation}
mit Werten von $X$ und $Y$ in einem geeigneten Körper.
Die Koeffizienten $a$ und $b$ müssen so gewählt werden, dass die
Gleichung~\eqref{buch:crypto:eqn:ellipticcurve} genügend viele
Lösungen hat.
Über den komplexen Zahlen hat die Gleichung natürlich für jede Wahl von
$X$ drei Lösungen.
Für einen endlichen Körper können wir dies im allgemeinen nicht erwarten,
aber wenn wir genügend viele Wurzeln zu $\mathbb{F}$ hinzufügen können wir
mindestens erreichen, dass die Lösungsmenge so viele Elemente hat, 
dass ein Versuch, die Gleichung $g^x=b$ mittels Durchprobierens zu
lösen, zum Scheitern verurteil ist.

\begin{definition}
\label{buch:crypto:def:ellipticcurve}
Die {\em elliptische Kurve} $E_{a,b}(\Bbbk)$ über dem Körper $\Bbbk$ ist 
die Menge
\[
E_{a,b}(\Bbbk)
=
\{(X,Y)\in\Bbbk^2\;|\;Y^2+XY=X^3+aX+b\},
\]
für $a,b\in\Bbbk$.
\end{definition}

Um die Anschauung zu vereinfachen, werden wir elliptische Kurven über
dem Körper $\mathbb{R}$ visualisieren.
Die daraus gewonnenen geometrischen Einsichten werden wir anschliessend
algebraisch umsetzen.
In den reellen Zahlen kann man die
Gleichung~\eqref{buch:crypto:eqn:ellipticcurve}
noch etwas vereinfachen.
Indem man in \eqref{buch:crypto:eqn:ellipticcurve} 
quadratisch ergänzt, bekommt man
\begin{align}
Y^2 + XY + \frac14X^2 &= X^3+\frac14 X^2 +aX+b
\notag
\\
\Rightarrow\qquad
v^2&=X^3+\frac14X^2+aX+b,
\label{buch:crypto:eqn:ell2}
\end{align}
indem man $v=Y+\frac12X$ setzt.
Man beachte, dass man diese Substition nur machen kann, wenn $\frac12$
definiert ist.
In $\mathbb{R}$ ist dies kein Problem, aber genau über den Körpern
mit Charakteristik $2$, die wir für die Computer-Implementation
bevorzugen, ist dies nicht möglich.
Es geht hier aber nur um die Visualisierung.

Auch die Form \eqref{buch:crypto:eqn:ell2} lässt sich noch etwas 
vereinfachen.
Setzt man $X=u-\frac1{12}$, dann verschwindet nach einiger Rechnung,
die wir hier nicht durchführen wollen, der quadratische Term
auf der rechten Seite.
Die interessierenden Punkte sind Lösungen der einfacheren Gleichung
\begin{equation}
v^2
=
u^3+\biggl(a-\frac{1}{48}\biggr)u + b-\frac{a}{12}+\frac{1}{864}
=
u^3+Au+B.
\label{buch:crypto:ellvereinfacht}
\end{equation}
In dieser Form ist mit $(u,v)$ immer auch $(u,-v)$ eine Lösung,
die Kurve ist symmetrisch bezüglich der $u$-Achse.
Ebenso kann man ablesen, dass nur diejenigen $u$-Werte möglich sind,
für die das kubische Polynom $u^3+Au+B$ auf der rechten Seite von
\eqref{buch:crypto:ellvereinfacht}
nicht negativ ist.

Sind $u_1$, $u_2$ und $u_3$ die Nullstellen des kubischen Polynoms
auf der rechten Seite von~\eqref{buch:crypto:ellvereinfacht}, folgt
\[
v^2
=
(u-u_1)(u-u_2)(u-u_3)
=
u^3
-(u_1+u_2+u_3)u^2
+(u_1u_2+u_1u_3+u_2u_3)u
-
u_1u_2u_3.
\]
Durch Koeffizientenvergleich sieht man, dass $u_1+u_2+u_3=0$ sein muss.
\begin{figure}
\centering
\includegraphics{chapters/90-crypto/images/elliptic.pdf}
\caption{Elliptische Kurve in $\mathbb{R}$ in der Form
$v^2=u^3+Au+B$ mit Nullstellen $u_1$, $u_2$ und $u_3$ des
kubischen Polynoms auf der rechten Seite.
Die blauen Punkte und Geraden illustrieren die Definition der
Gruppenoperation in der elliptischen Kurve.
\label{buch:crypto:fig:elliptischekurve}}
\end{figure}
Abbildung~\ref{buch:crypto:fig:elliptischekurve}
zeigt eine elliptische Kurve in der Ebene.

\subsubsection{Gruppenoperation}

\subsubsection{Beispiele}

