%
% arith.tex
%
% (c) 2021 Prof Dr Andreas Müller, Hochschule Rapperswil
%
\section{Arithmetik für die Kryptographie
\label{buch:section:arithmetik-fuer-kryptographie}}
\rhead{Arithmetik für die Kryptographie}

\subsection{Potenzieren
\label{buch:subsection:potenzieren}}
% XXX Divide-and-Conquer Algorithmus

\subsection{Rechenoperationen in $\mathbb{F}_p$
\label{buch:subsection:rechenoperationen-in-fp}}
% XXX Multiplikation: modulare Reduktion mit jedem Digit
% XXX Divide-and-Conquer

\subsection{Rechenoperationen in $\mathbb{F}_{2^l}$
\label{buch:subsection:rechenoperatione-in-f2l}}
% XXX Darstellung eines Körpers der Art F_{2^l}
% XXX Addition (XOR) und Multiplikation
% XXX Beispiel F_{2^8}
% XXX Beispiel F einer Oakley-Gruppe

