%
% normalformen.tex -- Normalformen einer Matrix
%
% (c) 2021 Prof Dr Andreas Müller, OST Ostschweizer Fachhochschule
%
\section{Normalformen
\label{buch:section:normalformen}}
\rhead{Normalformen}
In den Beispielen im vorangegangenen wurde wiederholt der Trick
verwendet, den Koeffizientenkörper so zu erweitern, dass das
charakteristische Polynom in Linearfaktoren zerfällt und 
für jeden Eigenwert Eigenvektoren gefunden werden können.
Diese Idee ermöglicht, eine Matrix in einer geeigneten Körpererweiterung
in eine besonders einfache Form zu bringen, das Problem dort zu lösen.
Anschliessend kann man sich darum kümmern in welchem Mass die gewonnenen
Resultate wieder in den ursprünglichen Körper transportiert werden können.

\subsection{Diagonalform}
Sei $A$ eine beliebige Matrix mit Koeffizienten in $\Bbbk$ und sei $\Bbbk'$
eine Körpererweiterung von $\Bbbk$ derart, dass das charakteristische
Polynom in Linearfaktoren
\[
\chi_A(x)
=
(x-\lambda_1)^{k_1}\cdot (x-\lambda_2)^{k_2}\cdot\dots\cdot (x-\lambda_m)^{k_m}
\]
mit Vielfachheiten $k_m$ zerfällt, $\lambda_i\in\Bbbk'$.
Zu jedem Eigenwert $\lambda_i$ gibt es sicher einen Eigenvektor, wir 
wollen aber in diesem Abschnitt zusätzlich annehmen, dass es eine Basis
aus Eigenvektoren gibt.
In dieser Basis bekommt die Matrix Diagonalform, wobei auf der 
Diagonalen nur Eigenwerte vorkommen können.
Man kann die Vektoren so anordnen, dass die Diagonalmatrix in Blöcke
der Form $\lambda_iE$ zerfällt
\[
\def\temp#1{\multicolumn{1}{|c}{\raisebox{0pt}[12pt][7pt]{\phantom{x}$#1$}\phantom{x}}}
A'
=\left(
\begin{array}{cccc}
\cline{1-1}
\temp{\lambda_1E} &\multicolumn{1}{|c}{}&        &           \\
\cline{1-2}
          &\temp{\lambda_2E}&\multicolumn{1}{|c}{}&           \\
\cline{2-3}
          &           &\temp{\ddots}&\multicolumn{1}{|c}{}\\
\cline{3-4}
          &           &        &\multicolumn{1}{|c|}{\raisebox{0pt}[12pt][7pt]{\phantom{x}$\lambda_mE$}\phantom{x}}\\
\cline{4-4}
\end{array}
\right)
\]
Über die Grösse eines solchen $\lambda_iE$-Blockes können wir zum jetzigen
Zeitpunkt noch keine Aussagen machen.

Die Matrizen $A-\lambda_kE$ enthalten jeweils einen Block aus lauter
Nullen.
Das Produkt all dieser Matrizen  ist daher
\[
(A-\lambda_1E)
(A-\lambda_2E)
\cdots
(A-\lambda_mE)
=
0.
\]
Über dem Körper $\Bbbk'$ gibt es also das Polynom
$m(x)=(x-\lambda_1)(x-\lambda_2)\cdots(x-\lambda_m)$ mit der Eigenschaft
$m(A)=0$.
Dies ist auch das Polynom von kleinstmöglichem Grad, denn für jeden
Eigenwert muss ein entsprechender Linearfaktor in so einem Polynom vorkommen.
Das Polynom $m(x)$ ist daher das Minimalpolynom der Matrix $A$.
Da jeder Faktor in $m(x)$ auch ein Faktor von $\chi_A(x)$ ist,
folgt wieder $\chi_A(A)=0$.
Ausserdem ist über dem Körper $\Bbbk'$ das Polynom $m(x)$ ein Teiler
des charakteristischen Polynoms $\chi_A(x)$.

\subsection{Jordan-Normalform}

\subsection{Reelle Normalform}

\subsection{Obere Hessenberg-Form}
