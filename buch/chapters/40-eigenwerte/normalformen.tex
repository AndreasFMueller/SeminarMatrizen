%
% normalformen.tex -- Normalformen einer Matrix
%
% (c) 2021 Prof Dr Andreas Müller, OST Ostschweizer Fachhochschule
%
\section{Normalformen
\label{buch:section:normalformen}}
\rhead{Normalformen}
In den Beispielen im vorangegangenen wurde wiederholt der Trick
verwendet, den Koeffizientenkörper so zu erweitern, dass das
charakteristische Polynom in Linearfaktoren zerfällt und 
für jeden Eigenwert Eigenvektoren gefunden werden können.
Diese Idee ermöglicht, eine Matrix in einer geeigneten Körpererweiterung
in eine besonders einfache Form zu bringen, das Problem dort zu lösen.
Anschliessend kann man sich darum kümmern in welchem Mass die gewonnenen
Resultate wieder in den ursprünglichen Körper transportiert werden können.

\subsection{Diagonalform}
Sei $A$ eine beliebige Matrix mit Koeffizienten in $\Bbbk$ und sei $\Bbbk'$
eine Körpererweiterung von $\Bbbk$ derart, dass das charakteristische
Polynom in Linearfaktoren
\[
\chi_A(x)
=
(x-\lambda_1)^{k_1}\cdot (x-\lambda_2)^{k_2}\cdot\dots\cdot (x-\lambda_m)^{k_m}
\]
mit Vielfachheiten $k_1$ bis $k_m$ zerfällt, $\lambda_i\in\Bbbk'$.
Zu jedem Eigenwert $\lambda_i$ gibt es sicher einen Eigenvektor, wir 
wollen aber in diesem Abschnitt zusätzlich annehmen, dass es eine Basis
aus Eigenvektoren gibt.
In dieser Basis bekommt die Matrix Diagonalform, wobei auf der 
Diagonalen nur Eigenwerte vorkommen können.
Man kann die Vektoren so anordnen, dass die Diagonalmatrix in Blöcke
der Form $\lambda_iE$ zerfällt
\[
\def\temp#1{\multicolumn{1}{|c}{\raisebox{0pt}[12pt][7pt]{\phantom{x}$#1$}\phantom{x}}}
A'
=\left(
\begin{array}{cccc}
\cline{1-1}
\temp{\lambda_1E} &\multicolumn{1}{|c}{}&        &           \\
\cline{1-2}
          &\temp{\lambda_2E}&\multicolumn{1}{|c}{}&           \\
\cline{2-3}
          &           &\temp{\ddots}&\multicolumn{1}{|c}{}\\
\cline{3-4}
          &           &        &\multicolumn{1}{|c|}{\raisebox{0pt}[12pt][7pt]{\phantom{x}$\lambda_mE$}\phantom{x}}\\
\cline{4-4}
\end{array}
\right)
\]
Über die Grösse eines solchen $\lambda_iE$-Blockes können wir zum jetzigen
Zeitpunkt noch keine Aussagen machen.

Die Matrizen $A-\lambda_kE$ enthalten jeweils einen Block aus lauter
Nullen.
Das Produkt all dieser Matrizen  ist daher
\[
(A-\lambda_1E)
(A-\lambda_2E)
\cdots
(A-\lambda_mE)
=
0.
\]
Über dem Körper $\Bbbk'$ gibt es also das Polynom
$m(x)=(x-\lambda_1)(x-\lambda_2)\cdots(x-\lambda_m)$ mit der Eigenschaft
$m(A)=0$.
Dies ist auch das Polynom von kleinstmöglichem Grad, denn für jeden
Eigenwert muss ein entsprechender Linearfaktor in so einem Polynom vorkommen.
Das Polynom $m(x)$ ist daher das Minimalpolynom der Matrix $A$.
Da jeder Faktor in $m(x)$ auch ein Faktor von $\chi_A(x)$ ist,
folgt wieder $\chi_A(A)=0$.
Ausserdem ist über dem Körper $\Bbbk'$ das Polynom $m(x)$ ein Teiler
des charakteristischen Polynoms $\chi_A(x)$.

\subsection{Jordan-Normalform
\label{buch:subsection:jordan-normalform}}
Die Eigenwerte einer Matrix $A$ können als Nullstellen des 
charakteristischen Polynoms gefunden werden.
Da der Körper $\Bbbk$ nicht unbedingt algebraische abgeschlossen ist,
zerfällt das charakteristische Polynom nicht unbedingt in Linearfaktoren,
die Nullstellen sind nicht unbedingt in $\Bbbk$.
Wir können aber immer zu einem grösseren Körper $\Bbbk'$ übergehen,
in dem das charakteristische Polynom in Linearfaktoren zerfällt.
Wir nehmen im Folgenden an, dass 
\[
\chi_A(x)
=
(x-\lambda_1)^{k_1}
\cdot
(x-\lambda_2)^{k_2}
\cdot
\dots
\cdot
(x-\lambda_l)^{k_l}
\]
ist mit $\lambda_i\in\Bbbk'$.

Nach Satz~\ref{buch:eigenwerte:satz:zerlegung-in-eigenraeume} liefern
die verallgemeinerten Eigenräume $V_i=\mathcal{E}_{\lambda_i}(A)$ eine
Zerlegung von $V$ in invariante Eigenräume
\[
V=V_1\oplus V_2\oplus \dots\oplus V_l,
\]
derart, dass $A-\lambda_iE$ auf $V_i$ nilpotent ist.
Wählt man in jedem der Unterräume $V_i$ eine Basis, dann zerfällt die
Matrix $A$ in Blockmatrizen
\begin{equation}
\def\temp#1{\multicolumn{1}{|c}{\raisebox{0pt}[17pt][12pt]{\phantom{x}$#1\mathstrut$}\phantom{x}}}
A'
=\left(
\begin{array}{cccc}
\cline{1-1}
\temp{A_{1}} &\multicolumn{1}{|c}{}&        &           \\
\cline{1-2}
          &\temp{A_{2}}&\multicolumn{1}{|c}{}&           \\
\cline{2-3}
          &           &\temp{\ddots}&\multicolumn{1}{|c}{}\\
\cline{3-4}
          &           &        &\multicolumn{1}{|c|}{\raisebox{0pt}[17pt][12pt]{\phantom{x}$A_{l}$}\phantom{x}}\\
\cline{4-4}
\end{array}
\right)
\label{buch:eigenwerte:eqn:allgjordan}
\end{equation}
wobei, $A_i$ Matrizen mit dem einzigen Eigenwert $\lambda_i$ sind.

Nach Satz~\ref{buch:eigenwerte:satz:allgnilpotent}
kann man in den Unterräume die Basis zusätzlich so wählen, dass 
die entstehenden Blöcke $A_i-\lambda_i E$ spezielle nilpotente Matrizen
aus lauter Null sind, die höchstens unmittelbar über der Diagonalen
Einträge $1$ haben kann.
Dies bedeutet, dass sich immer eine Basis so wählen lässt, dass die
Matrix $A_i$ zerfällt in sogenannte Jordan-Blöcke.

\begin{definition}
Ein $m$-dimensionaler {\em Jordan-Block} ist eine $m\times m$-Matrix
\index{Jordan-Block}%
der Form
\[
J_m(\lambda)
=
\begin{pmatrix}
\lambda &    1    &         &        &         &         \\
        & \lambda &    1    &        &         &         \\
        &         & \lambda &        &         &         \\
        &         &         & \ddots &         &         \\
        &         &         &        & \lambda &     1   \\
        &         &         &        &         & \lambda 
\end{pmatrix}.
\]
Eine {\em Jordan-Matrix} ist eine Blockmatrix Matrix
\[
J
=
\def\temp#1{\multicolumn{1}{|c}{\raisebox{0pt}[17pt][12pt]{\phantom{x}$#1\mathstrut$}\phantom{x}}}
\left(
\begin{array}{cccc}
\cline{1-1}
\temp{J_{m_1}(\lambda)} &\multicolumn{1}{|c}{}&        &           \\
\cline{1-2}
          &\temp{J_{m_2}(\lambda)}&\multicolumn{1}{|c}{}&           \\
\cline{2-3}
          &           &\temp{\ddots}&\multicolumn{1}{|c}{}\\
\cline{3-4}
          &           &        &\multicolumn{1}{|c|}{\raisebox{0pt}[17pt][12pt]{\phantom{x}$J_{m_p}(\lambda)$}\phantom{x}}\\
\cline{4-4}
\end{array}
\right)
\]
mit $m_1+m_2+\dots+m_p=m$.
\index{Jordan-Matrix}%
\end{definition}

Da Jordan-Blöcke obere Dreiecksmatrizen sind, ist
das charakteristische Polynom eines Jordan-Blocks oder einer Jordan-Matrix
besonders einfach zu berechnen.
Es gilt
\[
\chi_{J_m(\lambda)}(x)
=
\det (J_m(\lambda) - xE)
=
(\lambda-x)^m
\]
für einen Jordan-Block $J_m(\lambda)$.
Für eine $m\times m$-Jordan-Matrix $J$ mit Blöcken $J_{m_1}(\lambda)$
bis $J_{m_p}(\lambda)$ ist
\[
\chi_{J(\lambda)}(x)
=
\chi_{J_{m_1}(\lambda)}(x)
\chi_{J_{m_2}(\lambda)}(x)
\cdot
\dots
\cdot
\chi_{J_{m_p}(\lambda)}(x)
=
(\lambda-x)^{m_1}
(\lambda-x)^{m_2}
\cdot\dots\cdot
(\lambda-x)^{m_p}
=
(\lambda-x)^m.
\]

\begin{satz}
\label{buch:eigenwerte:satz:jordannormalform}
Über einem Körper $\Bbbk'\supset\Bbbk$, über dem das charakteristische
Polynom $\chi_A(x)$ in Linearfaktoren zerfällt, lässt sich immer
eine Basis finden derart, dass die Matrix $A$ zu einer Blockmatrix wird,
die aus lauter Jordan-Matrizen besteht.
Die Dimension der Jordan-Matrix zum Eigenwert $\lambda_i$ ist die
Vielfachheit des Eigenwerts im charakteristischen Polynom.
\end{satz}

\begin{proof}[Beweis]
Es ist nur noch die Aussage über die Dimension der Jordan-Blöcke zu
beweisen.
Die Jordan-Matrizen zum Eigenwert $\lambda_i$ werden mit $J_i$
bezeichnet und sollen $m_i\times m_i$-Matrizen sein.
Das charakteristische Polynom jedes Jordan-Blocks ist dann
$\chi_{J_i}(x)=(\lambda_i-x)^{m_i}$.
Das charakteristische Polynom der Blockmatrix mit diesen Jordan-Matrizen
als Blöcken ist das Produkt
\[
\chi_A(x)
=
(\lambda_1-x)^{m_1}
(\lambda_2-x)^{m_2}
\cdots
(\lambda_p-x)^{m_p}
\]
mit $m_1+m_2+\dots+m_p$.
Die Blockgrösse $m_i$ ist also auch die Vielfachheit von $\lambda_i$ im
charakteristischen Polynom $\chi_A(x)$.
\end{proof}



\begin{satz}[Cayley-Hamilton]
\label{buch:normalformen:satz:cayley-hamilton}
Ist $A$ eine $n\times n$-Matrix über dem Körper $\Bbbk$, dann gilt
$\chi_A(A)=0$.
\end{satz}

\begin{proof}[Beweis]
Zunächst gehen wir über zu einem Körper $\Bbbk'\supset\Bbbk$, indem
das charakteristische Polynom $\chi_A(x)$ in Linearfaktoren
$\chi_A(x)
=
(\lambda_1-x)^{m_1}
(\lambda_2-x)^{m_2}
\dots
(\lambda_p-x)^{m_p}$
zerfällt.
Im Vektorraum $\Bbbk'$ kann man eine Basis finden, in der die Matrix
$A$ in Jordan-Matrizen $J_1,\dots,J_p$ zerfällt, wobei $J_i$ eine
$m_i\times m_i$-Matrix ist.
Für den Block mit der Nummer $i$ erhalten wir
$(J_i - \lambda_i E)^{m_i} = 0$.
Setzt man also den Block $J_i$ in das charakteristische Polynom
$\chi_A(x)$ ein, erhält man
\[
\chi_A(J_i)
=
(\lambda_1E - J_1)^{m_1}
\cdot
\ldots
\cdot
\underbrace{
(\lambda_iE - J_i)^{m_i}
}_{\displaystyle=0}
\cdot
\ldots
\cdot
(\lambda_iE - J_p)^{m_p}
=
0.
\]
Jeder einzelne Block $J_i$ wird also zu $0$, wenn man ihn in das
charakteristische Polynome $\chi_A(x)$ einsetzt.
Folglich gilt auch $\chi_A(A)=0$.

Die Rechnung hat zwar im Körper $\Bbbk'$ stattgefunden, aber die Berechnung
$\chi_A(A)$ kann in $\Bbbk$ ausgeführt werden, also ist $\chi_A(A)=0$.
\end{proof}

Aus dem Beweis kann man auch noch eine strengere Bedingung ableiten.
Auf jedem verallgemeinerten Eigenraum $\mathcal{E}_{\lambda_i}(A)$
ist $A_i-\lambda_i$ nilpotent, es gibt also einen minimalen Exponenten
$q_i$ derart, dass $(A_i-\lambda_iE)^{q_i}=0$ ist.
Wählt man eine Basis in jedem verallgemeinerten Eigenraum derart,
dass $A_i$ eine Jordan-Matrix ist, kann man wieder zeigen, dass
für das Polynom
\[
m_A(x)
=
(x-\lambda_1x)^{q_1}
(x-\lambda_2x)^{q_2}
\cdot
\ldots
\cdot
(x-\lambda_px)^{q_p}
\]
gilt $m_A(A)=0$.
$m_A(x)$ ist das {\em Minimalpolynom} der Matrix $A$.
\index{Minimalpolynom einer Matrix}%

\begin{satz}[Minimalpolynom]
Über dem Körper $\Bbbk'\subset\Bbbk$, über dem das charakteristische
Polynom $\chi_A(x)$ in Linearfaktoren zerfällt, ist das Minimalpolynom
von $A$ das Polynom
\[
m(x)
=
(x-\lambda_1)^{q_1}
(x-\lambda_2)^{q_2}
\cdots
\ldots
\cdots
(x-\lambda_p)^{q_p}
\]
wobei $q_i$ der kleinste Index ist, für den die $q_i$-te Potenz
derEinschränkung von $A-\lambda_i E$ auf den verallgemeinerten Eigenraum
$\mathcal{E}_{\lambda_i}(A)$ verschwindet.
Es ist das Polynom geringsten Grades über $\Bbbk'$, welches $m(A)=0$ erfüllt.
\end{satz}


\subsection{Reelle Normalform
\label{buch:subsection:reelle-normalform}}
Wenn eine reelle Matrix $A$ komplexe Eigenwerte hat, ist die Jordansche
Normalform zwar möglich, aber die zugehörigen Basisvektoren werden ebenfalls
komplexe Komponenten haben.
Für eine rein reelle Rechnung ist dies nachteilig, da der Speicheraufwand
dadurch verdoppelt und der Rechenaufwand für Multiplikationen vervierfacht
wird.

Die nicht reellen Eigenwerte von $A$ treten in konjugiert komplexen Paaren
$\lambda_i$ und $\overline{\lambda}_i$ auf.
Wir betrachten im Folgenden nur ein einziges Paar $\lambda=a+ib$ und
$\overline{\lambda}=a-ib$ von konjugiert komplexen Eigenwerten mit
nur je einem einzigen $n\times n$-Jordan-Block $J$ und $\overline{J}$.
Ist $\mathcal{B}=\{b_1,\dots,b_n\}$ die Basis für den Jordan-Block $J$,
dann kann man die Vektoren
$\overline{\mathcal{B}}=\{\overline{b}_1,\dots,\overline{b}_n\}$ als Basis für
$\overline{J}$ verwenden.
Die vereinigte Basis
$\mathcal{C} = \mathcal{B}\cup\overline{\mathcal{B}}
= \{b_1,\dots,b_n,\overline{b}_1,\dots,\overline{b}_n\}$
erzeugen einen $2n$-dimensionalen Vektorraum,
der direkte Summe der beiden von $\mathcal{B}$ und $\overline{\mathcal{B}}$
erzeugen Vektorräume $V=\langle\mathcal{B}\rangle$ und
$\overline{V}=\langle\overline{\mathcal{B}}\rangle$ ist.
Es ist also
\[
U=\langle \mathcal{C}\rangle
=
V\oplus \overline{V}.
\]
Wir bezeichnen die lineare Abbildung mit den Jordan-Blöcken
$J$ und $\overline{J}$ wieder mit $A$.

Auf dem Vektorraum $U$ hat die lineare Abbildung in der Basis
$\mathcal{C}$ die Matrix
\[
A=
\begin{pmatrix}
J&0\\
0&\overline{J}
\end{pmatrix}
=
\begin{pmatrix}
\lambda&   1   &       &      &       &&&&&\\
       &\lambda&   1   &      &       &&&&&\\
       &       &\lambda&\ddots&       &&&&&\\
       &       &       &\ddots&   1   &&&&&\\
       &       &       &      &\lambda&&&&&\\
&&&& &\overline{\lambda}&1&&     & \\
&&&& &&\overline{\lambda}&1&     & \\
&&&& &&&\overline{\lambda} &\dots& \\
&&&& &&&                   &\dots&1\\
&&&& &&&                   &&\overline{\lambda}\\
\end{pmatrix}.
\]

Die Jordan-Normalform bedeutet, dass
\[
\begin{aligned}
Ab_1&=\lambda b_1           &
	A\overline{b}_1 &= \overline{\lambda} \overline{b}_1      \\
Ab_2&=\lambda b_2 + b_1     &
	A\overline{b}_2 &= \overline{\lambda} \overline{b}_2 +\overline{b_1}\\
Ab_3&=\lambda b_3 + b_2     &
	A\overline{b}_3 &= \overline{\lambda} \overline{b}_3 +\overline{b_2}\\
    &\;\vdots               &
	                 &\;\vdots \\
Ab_n&=\lambda b_n + b_{n-1} &
	A\overline{b}_n &= \overline{\lambda} \overline{b}_n +\overline{b_{n-1}}
\end{aligned}
\]
Für die Linearkombinationen
\begin{equation}
\begin{aligned}
c_i &= \frac{b_i+\overline{b}_i}{\sqrt{2}},
&
d_i &= \frac{b_i-\overline{b}_i}{i\sqrt{2}}
\end{aligned}
\label{buch:eigenwerte:eqn:reellenormalformumrechnung}
\end{equation}
folgt dann für $k>1$
\begin{align*}
Ac_k
&=
\frac{Ab_k+A\overline{b}_k}{2}
&
Ad_k
&=
\frac{Ab_k-A\overline{b}_k}{2i}
\\
&=
\frac1{\sqrt{2}}(\lambda b_k + b_{k-1}
+ \overline{\lambda}\overline{b}_k + \overline{b}_{k-1})
&
&=
\frac1{i\sqrt{2}}(\lambda b_k + b_{k-1}
- \overline{\lambda}\overline{b}_k - \overline{b}_{k-1})
\\
&=
\frac1{\sqrt{2}}(\alpha b_k + i\beta b_k + \alpha \overline{b}_k -i\beta \overline{b}_k)
+
c_{k-1}
&
&=
\frac1{i\sqrt{2}}(
\alpha b_k + i\beta b_k - \alpha \overline{b}_k +i\beta \overline{b}_k)
+
d_{k-1}
\\
&=
\alpha
\frac{b_k+\overline{b}_k}{\sqrt{2}}
+
i \beta \frac{b_k-\overline{b}_k}{\sqrt{2}}
+
c_{k-1}
&
&=
\alpha
\frac{b_k-\overline{b}_k}{i\sqrt{2}}
+
i \beta \frac{b_k+\overline{b}_k}{i\sqrt{2}}
+
d_{k-1}
\\
&= \alpha c_k -\beta d_k
+
c_{k-1}
&
&= \alpha d_k + \beta c_k
+
d_{k-1}.
\end{align*}
Für $k=1$ fallen die Terme $c_{k-1}$ und $d_{k-1}$ weg.
In der Basis $\mathcal{D}=\{c_1,d_1,\dots,c_n,d_n\}$ hat die Matrix
also die {\em reelle Normalform}
\begin{equation}
\def\temp#1{\multicolumn{1}{|c}{#1\mathstrut}}
\def\semp#1{\multicolumn{1}{c|}{#1\mathstrut}}
A_{\text{reell}}
=
\left(
\begin{array}{cccccccccccc}
\cline{1-4}
\temp{\alpha}& \beta&\temp{     1}&     0&\temp{}      &      &      &      &      &      &&\\
\temp{-\beta}&\alpha&\temp{     0}&     1&\temp{}      &      &      &      &      &      &&\\
\cline{1-6}
      &      &\temp{\alpha}& \beta&\temp{     1}&     0&\temp{}      &      &      &      &&\\
      &      &\temp{-\beta}&\alpha&\temp{     0}&     1&\temp{}      &      &      &      &&\\
\cline{3-6}
      &      &      &      &\temp{\alpha}& \beta&\temp{}      &      &      &      &&\\
      &      &      &      &\temp{-\beta}&\alpha&\temp{}      &      &      &      &&\\
\cline{5-8}
      &      &      &      &      &      &\temp{\phantom{0}}&\phantom{0}&\temp{      }&      &&\\
      &      &      &      &      &      &\temp{\phantom{0}}&\phantom{0}&\temp{      }&      &&\\
\cline{7-12}
      &      &      &      &      &      &      &      &\temp{\alpha}& \beta&\temp{     1}&\semp{     0}\\
      &      &      &      &      &      &      &      &\temp{-\beta}&\alpha&\temp{     0}&\semp{     1}\\
\cline{9-12}
      &      &      &      &      &      &      &      &      &      &\temp{\alpha}&\semp{ \beta}\\
      &      &      &      &      &      &      &      &      &      &\temp{-\beta}&\semp{\alpha}\\
\cline{11-12}
\end{array}\right).
\label{buch:eigenwerte:eqn:reellenormalform}
\end{equation}

Wir bestimmen noch die Transformationsmatrix, die $A$ in die reelle
Normalform bringt.
Dazu beachten wir, dass die Vektoren $c_k$ und $d_k$ in der Basis
$\mathcal{B}$ nur in den Komponenten $k$ und $n+k$ von $0$ verschiedene
Koordinaten haben, nämlich
\[
c_k
=
\frac1{\sqrt{2}}
\left(
\begin{array}{c}
\vdots\\ 1 \\ \vdots\\\hline \vdots\\ 1\\\vdots
\end{array}\right)
\qquad\text{und}\qquad
d_k
=
\frac1{i\sqrt{2}}
\left(\begin{array}{c}
\vdots\\ 1 \\ \vdots\\\hline\vdots\\-1\\\vdots
\end{array}\right)
=
\frac1{\sqrt{2}}
\left(\begin{array}{c}
\vdots\\-i \\ \vdots\\\hline \vdots\\ i\\\vdots
\end{array}\right)
\]
gemäss \eqref{buch:eigenwerte:eqn:reellenormalformumrechnung}.
Die Umrechnung der Koordinaten von der Basis $\mathcal{B}$ in die Basis
$\mathcal{D}$
wird daher durch die Matrix
\[
S
=
\frac{1}{\sqrt{2}}
\left(\begin{array}{cccccccccc}
1&-i& &  & &  &     &     & &  \\
 &  &1&-i& &  &     &     & &  \\
 &  & &  &1&-i&     &     & &  \\
 &  & &  & &  &\dots&\dots& &  \\
 &  & &  & &  &     &     &1&-i\\
\hline
1& i& &  & &  &     &     & &  \\
 &  &1& i& &  &     &     & &  \\
 &  & &  &1& i&     &     & &  \\
 &  & &  & &  &\dots&\dots& &  \\
 &  & &  & &  &     &     &1& i\\
\end{array}\right)
\]
vermittelt.
Der Nenner $\sqrt{2}$ wurde so gewählt, dass die
Zeilenvektoren der Matrix $S$ als komplexe Vektoren orthonormiert sind,
die Matrix $S$ ist daher unitär und hat die Inverse
\[
S^{-1}
=
S^*
=
\frac{1}{\sqrt{2}}
\left(\begin{array}{ccccc|ccccc}
 1&  &  &     &  & 1&  &  &     &  \\
 i&  &  &     &  &-i&  &  &     &  \\
  & 1&  &     &  &  & 1&  &     &  \\
  & i&  &     &  &  &-i&  &     &  \\
  &  & 1&     &  &  &  & 1&     &  \\
  &  & i&     &  &  &  &-i&     &  \\
  &  &  &\dots&  &  &  &  &\dots&  \\
  &  &  &\dots&  &  &  &  &\dots&  \\
  &  &  &     & 1&  &  &  &     & 1\\
  &  &  &     & i&  &  &  &     &-i\\
\end{array}\right).
\]
Insbesondere folgt jetzt
\[
A
=
S^{-1}A_{\text{reell}}S
=
S^*A_{\text{reell}}S
\qquad\text{und}\qquad
A_{\text{reell}}
=
SAS^{-1}
=
SAS^*.
\]

%\subsection{Obere Hessenberg-Form
%\label{buch:subsection:obere-hessenberg-form}}



