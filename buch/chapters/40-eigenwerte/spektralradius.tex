%
% spektralradius.tex
%
% (c) 2020 Prof Dr Andreas Müller, Hochschule Rapperswi
%
\section{Funktionen einer Matrix
\label{buch:section:funktionen-einer-matrix}}
\rhead{Funktionen einer Matrix}
Eine zentrale Motivation in der Entwicklung der Eigenwerttheorie
war das Bestreben, Potenzen $A^k$ auch für grosse $k$ effizient
zu berechnen.
Mit der Jordan-Normalform ist dies auch gelungen, wenigstens über
einem Körper, in dem das charakteristische Polynom in Linearfaktoren
zerfällt.
Die Berechnung von Potenzen war aber nur der erste Schritt, das Ziel
in diesem Abschnitt ist, $f(A)$ für eine genügend grosse Klasse von
Funktionen $f$ berechnen zu können.

%
% Polynom-Funktionen von Matrizen
%
\subsection{Polynom-Funktionen
\label{buch:subsection:polynom-funktionen}}
In diesem Abschnitt ist $B\in M_n(\Bbbk)$ und $\Bbbk'\supset\Bbbk$ ein
Körper, über dem das charakteristische Polynome $\chi_A(x)$ in
Linearfaktoren
\[
\chi_A(x)
=
(\lambda_1-x)^{m_1}(\lambda_2-x)^{m_2}\cdot\ldots\cdot(\lambda_p-x)^{m_p}
\]
zerfällt.

Für jedes beliebige Polynome $p(X)=\Bbbk[X]$ der Form
\[
p(X) = a_nX^n + a_{n-1}X^{n-1} + \dots a_1x + a_0
\]
kann man auch
\[
p(A) = a_nA^n + a_{n-1}A^{n-1} + \dots a_1A + a_0E
\]
berechnen.
In der Jordan-Normalform können die Potenzen $A^k$ leicht zusammengstellt
werden, sobald man die Potenzen von Jordan-Blöcken berechnet hat.

\begin{satz}
Die $k$-te Potenz von $J_n(\lambda)$ ist die Matrix mit
\begin{equation}
J_n(\lambda)^k
=
\begin{pmatrix}
\lambda^k
	& \binom{k}{1}\lambda^{k-1}
		& \binom{k}{2} \lambda^{k-2}
			& \binom{k}{3} \lambda^{k-3}
				& \dots
					&\binom{k}{n-1}\lambda^{k-n+1}
\\
0
	& \lambda^k
		& \binom{k}{1}\lambda^{k-1}
			& \binom{k}{2} \lambda^{k-2}
				& \dots
					&\binom{k}{n-2}\lambda^{k-n+2}
\\
0
	& 0
		& \lambda^k
			& \binom{k}{1}\lambda^{k-1}
				& \dots
					&\binom{k}{n-3}\lambda^{k-n+3}
\\
\vdots  &\vdots &\vdots &\vdots &\ddots & \vdots
\\
0	& 0	& 0	& 0	& \dots	& \lambda^k
\end{pmatrix}
\label{buch:eigenwerte:eqn:Jnkpotenz}
\end{equation}
mit den Matrixelementen
\[
(J_n(\lambda)^k)_{ij}
=
\binom{k}{j-i}\lambda^{k-j+i}
\]
Die Binomialkoeffizienten verschwinden für $j<i$ und $j>i+k$.
\end{satz}

\begin{proof}[Beweis]
Die Herkunft der Binomialkoeffizienten wird klar, wenn man
\[
J_n(\lambda) = \lambda E + N_n
\]
schreibt, wobei $N_n$ die Matrix \eqref{buch:eigenwerte:eqn:nnilpotent} ist.
Die Potenzen von $N_n$ haben die Matrix-Elemente
\[
(N_n^k)_{ij}
=
\delta_{i,j-k}
=
\begin{cases}
1&\qquad j-i=k\\
0&\qquad\text{sonst,}
\end{cases}
\]
sie haben also Einsen genau dort, wo in der
\label{buch:eigenwerte:eqn:Jnkpotenz} die Potenz $\lambda^{k}$ steht.
Die $kt$-te Potenz von $J_n(\lambda)$ kann dann mit dem binomischen
Satz berechnet werden:
\[
J_n(\lambda)^k
=
\sum_{l=0}^k \binom{k}{l}\lambda^l N_n^{k-l},
\]
dies ist genau die Form \eqref{buch:eigenwerte:eqn:Jnkpotenz}.
\end{proof}

Wir haben bereits gesehen, dass $\chi_A(A)=0$, ersetzt man also das
Polynom $p(X)$ durch $p(X)+\chi_A(X)$, dann ändert sich am Wert 
\[
(p+\chi_A)(A)
=
p(A) + \chi_A(A)
=
p(A)
\]
nichts.
Man kann also nicht erwarten, dass verschiedene Polynome 
$p(X)$ zu verschiedenen Matrizen $p(A)$ führen.
Doch welche Unterschiede zwischen Polynomen wirken sich genau aus?

\begin{satz}
Für zwei Polynome $p(X)$ und $q(X)$ ist genau dann $p(A)=q(A)$, wenn
das Minimalpolynom von $A$ die Differenz $p-q$ teilt.
\end{satz}

\begin{proof}[Beweis]
Wenn $p(A)=q(A)$, dann ist $h(X)=p(X)-q(X)$ ein Polynom mit $h(A)=0$,
daher muss $h(X)$ vom Minimalpolynom geteilt werden.
Ist andererseits $p(X)-q(X)=m(X)t(X)$, dann ist
$p(A)-q(A)=m(A)t(A)=0\cdot t(A) = 0$, also $p(A)=q(A)$.
\end{proof}

Über einem Körper $\Bbbk'\supset\Bbbk$, über dem das charakteristische
Polynom in Linearfaktoren zerfällt, kann man das Minimalpolynom aus
der Jordanschen Normalform ableiten.
Es ist
\[
m(X)
=
(\lambda_1-X)^{q_1}
(\lambda_2-X)^{q_2}
\cdot\ldots
\cdot
(\lambda_p-X)^{q_p},
\]
wobei $q_i$ die Dimension des grössten Jordan-Blocks ist, der in der
Jordan-Normalform vorkommt.
Zwei Polynome $p_1(X)$ und $p_2(X)$ haben genau dann den gleichen Wert,
wenn die Differenz $p_1(X)-p_2(X)$ genau die Nullstellen
$\lambda_1,\dots,\lambda_p$ mit Vielfachheiten $q_1,\dots,q_p$ hat.

\begin{beispiel}
Wir betrachten die Matrix
\[
A
=
\begin{pmatrix}
   1&  9& -4\\
  -1&  3&  0\\
  -2&  0&  3
\end{pmatrix}
\]
mit dem charakteristischen Polynom
\[
\chi_A(x)
=
-x^3+7x^2-16 x+12
=
-(x-3)(x-2)^2.
\]
Daraus kann man bereits ablesen, dass das Minimalpolynom $m(X)$ von $A$ 
entweder $(X-2)(X-3)$ oder $(X-2)^2(X-3)$ ist.
Es genügt also nachzuprüfen, ob $p(A)=0$ für das Polynom
$p(X)=(X-2)(X-3) = X^2-5X+6$ ist.
Tatsächlich sind die Potenzen von $A$:
\[
A^2=
\begin{pmatrix}
  0&  36& -16 \\
 -4&   0&   4 \\
 -8& -18&  17 
\end{pmatrix}
,\qquad
A^3=
\begin{pmatrix}
 -4& 108& -48\\
-12& -36&  28\\
-24&-126&  83
\end{pmatrix}
\]
und daraus kann man jetzt $P(A)$ berechnen:
\begin{equation}
p(A)
=
\begin{pmatrix}
  0&  36& -16 \\
 -4&   0&   4 \\
 -8& -18&  17 
\end{pmatrix}
-5
\begin{pmatrix}
   1&  9& -4\\
  -1&  3&  0\\
  -2&  0&  3
\end{pmatrix}
+
6
\begin{pmatrix}
1&0&0\\
0&1&0\\
0&0&1
\end{pmatrix}
=
\begin{pmatrix}
   1& -9&  4\\
   1& -9&  4\\
   2&-18&  8
\end{pmatrix}
=
\begin{pmatrix}1\\1\\2\end{pmatrix}
\begin{pmatrix}1&-9&4\end{pmatrix}
\label{buch:eigenwerte:eqn:nichtminimalpolynom}
\end{equation}
Also ist tatsächlich $(X-2)^2(X-3)$ das Minimalpolynom.

Das Quadrat des Polynoms $p(X)$ ist $p(X)^2 = (X-2)^2(X-3)^2$, es hat
das Minimalpolynom als Teiler, also muss $p(A)^2=0$ sein.
Die Gleichung \eqref{buch:eigenwerte:eqn:nichtminimalpolynom} ermöglicht,
das Quaddrat $p(A)^2$ leichter zu berechnen:
\[
p(A)^2
=
\begin{pmatrix}1\\1\\2\end{pmatrix}
\underbrace{
\begin{pmatrix}1&-9&4\end{pmatrix}
\begin{pmatrix}1\\1\\2\end{pmatrix}
}_{\displaystyle = 0}
\begin{pmatrix}1&-9&4\end{pmatrix}
=
0
,
\]
wie zu erwarten war.

Wenn sich zwei Polynome nur um das charakteristische Polynom unterscheiden,
dann haben sie den gleichen Wert auf $A$.
Das Polynom $p_1(X)=X^3$ unterschiedet sich vom Polynom $p_2(X)=7X^2-16X+12$ 
um das charakteristische Polynom, welches wir bereits als das Minimalpolynom
von $A$ erkannt haben.
Die dritte Potenz $A^3$ von $A$ muss sich daher auch mit $p_2(X)$ berechnen
lassen:
\[
7
\begin{pmatrix}
  0&  36& -16 \\
 -4&   0&   4 \\
 -8& -18&  17 
\end{pmatrix}
-16
\begin{pmatrix}
   1&  9& -4\\
  -1&  3&  0\\
  -2&  0&  3
\end{pmatrix}
+12
\begin{pmatrix}
1&0&0\\
0&1&0\\
0&0&1
\end{pmatrix}
=
\begin{pmatrix}
 -4& 108&  -48\\
-12& -36&   28\\
-24&-126&   83
\end{pmatrix}
=
A^3.
\qedhere
\]
\end{beispiel}

\begin{satz}
Wenn $A$ diagonalisierbar ist über einem geeignet erweiterten Körper $\Bbbk'$,
dann haben zwei Polynome $p(X)$ und $q(X)$ in $\Bbbk[X]$ genau dann
den gleichen Wert auf $A$, also $p(A)=q(A)$, wenn $p(\lambda) = q(\lambda)$
für alle Eigenwerte $\lambda$ von $A$.
\end{satz}

Über dem Körper der komplexen Zahlen ist die Bedingung, dass die Differenz
$d(X)=p_1(X)-p_2(X)$ vom Minimalpolynom geteilt werden muss, gleichbedeutend
damit, dass $p_1(X)$ und $p_2(X)$ den gleichen Wert und gleiche Ableitungen
bis zur Ordnung $q_i-1$ haben in allen Eigenwerten $\lambda_i$, wobei 
$q_i$ der Exponent von $\lambda_i-X$ im Minimalpolynom von $A$ ist.

Das Beispiel illustriert auch noch ein weiteres wichtiges Prinzip.
Schreiben wir das Minimalpolynom von $A$ in der Form
\[
m(X)
=
X^k + a_{k-1}X^{k-1} + \dots + a_1X + a_0,
\]
dann kann man wegen $m(A)=0$ die Potenzen $A^i$ mit $i\ge k$ mit der
Rekursionsformel
\[
A^i
=
A^{i-k}A^k
=
A^{i-k}(-a_{k-1}A^{k-1}+ \dots + a_1 A + a_0E)
\]
in einer Linearkombination kleinerer Potenzen reduzieren.
Jedes Polynom vom Grad $\ge k$ kann also reduizert werden in
ein Polynom vom Grad $<k$ mit dem gleichen Wert auf $A$.

\begin{satz}
\label{buch:eigenwerte:satz:reduktion}
Sei $A$ eine Matrix über $\Bbbk$ mit Minimalpolynom $m(X)$.
Zu jedem $p(X)\in\Bbbk[X]$ gibt es ein Polynom $q(X)\in\Bbbk[X]$
vom Grad $\deg q<\deg m$ mit $p(A)=q(A)$.
\end{satz}

%
% Approximationen für Funktionswerte f(A)
%
\subsection{Approximation von $f(A)$
\label{buch:subsection:approximation}}
Die Quadratwurzelfunktion $x\mapsto\sqrt{x}$ lässt sich nicht durch ein
Polynom darstellen, es gibt also keine direkte Möglichkeit, $\sqrt{A}$
für eine beliebige Matrix zu definieren.
Wir können versuchen, die Funktion durch ein Polynom zu approximieren.
Damit dies geht, müssen wir folgende zwei Fragen klären:
\begin{enumerate}
\item
Wie misst man, ob ein Polynom eine Funktion gut approximiert?
\item
Was bedeutet es genau, dass zwei Matrizen ``nahe beeinander'' sind?
\item
In welchem Sinne müssen Polynome ``nahe'' beeinander sein, damit
auch die Werte auf $A$ nahe beeinander sind.
\end{enumerate}

Wir wissen bereits, dass nur die Werte und gewisse Ableitungen des
Polynoms $p(X)$ in den Eigenwerten einen Einfluss auf $p(A)$ haben.
Es genügt also, Approximationspolynome zu verwenden, welche in der Nähe
der Eigenwerte ``gut genug'' approximieren.
Solche Polynome gibt es dank dem Satz von Stone-Weierstrass immer:

\begin{satz}[Stone-Weierstrass]
Ist $I\subset\mathbb{R}$ kompakt, dann lässt sich jede stetige Funktion
durch eine Folge $p_n(x)$ beliebig genau approximieren.
\end{satz}

Wir haben schon gezeigt, dass es dabei auf die höheren Potenzen gar nicht
ankommt, nach Satz~\ref{buch:eigenwerte:satz:reduktion} kann man ein
approximierendes Polynom immer durch ein Polynom von kleinerem Grad
als das Minimalpolynom ersetzen.

\begin{definition}
\index{Norm}%
Die {\em Norm} einer Matrix $M$ ist
\[
\|M\|
=
\max\{|Mx|\,|\, x\in\mathbb R^n\wedge |x|=1\}.
\]
Für einen Vektor $x\in\mathbb R^n$ gilt $|Mx| \le \|M\|\cdot |x|$.
\end{definition}

\begin{beispiel}
Die Matrix
\[
M=\begin{pmatrix}
0&2\\
\frac13&0
\end{pmatrix}
\]
hat Norm
\[
\|M\|
=
\max_{|x|=1} |Mx| 
=
\max_{t\in\mathbb R} \sqrt{2^2\cos^2 t +\frac1{3^2}\sin^2t} \ge 2.
\]
Da aber
\[
M^2 = \begin{pmatrix}
\frac{2}{3}&0\\
0&\frac{2}{3}
\end{pmatrix}
\qquad\Rightarrow\qquad \|M^2\|=\frac23
\]
ist, wird eine Iteration mit Ableitungsmatrix $M$ trotzdem
konvergieren, weil der Fehler nach jedem zweiten Schritt um den
Faktor $\frac23$ kleiner geworden ist.
\end{beispiel}

\begin{beispiel}
Wir berechnen die Norm eines Jordan-Blocks.

\end{beispiel}

%
% Potenzreihen für Funktionen $f(z)$
%
\subsection{Potenzreihen
\label{buch:subsection:potenzreihen}}



Dies führt uns auf die Grösse
\begin{equation}
\pi(M)
=
\limsup_{n\to\infty} \|M^n\|^\frac1n.
\label{buch:eqn:gelfand-grenzwert}
\end{equation}
Ist $\pi(M) > 1$, dann gibt es Anfangsvektoren $v$ für die Iteration,
für die $M^kv$ über alle Grenzen wächst.
Ist $\pi(M) < 1$, dann wird jeder Anfangsvektor $v$ zu einer Iterationsfolge
$M^kv$ führen, die gegen $0$ konvergiert.
Die Kennzahl $\pi(M)$ erlaubt also zu entscheiden, ob ein
Iterationsverfahren konvergent ist.
\index{Konvergenzbedingung}%

Die Berechnung von $\pi(M)$ als Grenzwert ist sehr unhandlich.
Viel einfacher ist der Begriff des Spektralradius.
\index{Spektralradius}%

\begin{definition}
\label{buch:definition:spektralradius}
Der {\em Spektralradius} der Matrix $M$ ist der Betrag des betragsgrössten
Eigenwertes.
\end{definition}

%
% Gelfand-Radius und Eigenwerte
%
\subsection{Gelfand-Radius und Eigenwerte
\label{buch:subsection:spektralradius}}
In Abschnitt~\ref{buch:subsection:konvergenzbedingung}
ist der Gelfand-Radius mit Hilfe eines Grenzwertes definiert worden.
\index{Gelfand-Radius}%
Nur dieser Grenzwert ist in der Lage, über die Konvergenz eines 
Iterationsverfahrens Auskunft zu geben.
Der Grenzwert ist aber sehr mühsam zu berechnen.
\index{Grenzwert}%
Es wurde angedeutet, dass der Gelfand-Radius mit dem Spektralradius
übereinstimmt, dem Betrag des des betragsgrössten Eigenwertes.
Dies hat uns ein vergleichsweise einfach auszuwertendes Konvergenzkriterium
geliefert.
\index{Konvergenzkriterium}%
In diesem Abschnitt soll diese Identität zunächst an Spezialfällen
und später ganz allgemein gezeigt werden.

\subsubsection{Spezialfall: Diagonalisierbare Matrizen}
Ist eine Matrix $A$ diagonalisierbar, dann kann Sie durch eine Wahl
einer geeigneten Basis in Diagonalform
\index{diagonalisierbar}%
\index{Diagonalform}%
\[
A'
=
\begin{pmatrix}
\lambda_1&        0&\dots &0\\
0        &\lambda_2&\dots &0\\
\vdots   &         &\ddots&\vdots\\
0        &        0&\dots &\lambda_n
\end{pmatrix}
\]
gebracht werden, wobei die Eigenwerte $\lambda_i$  möglicherweise auch
komplex sein können.
\index{komplex}%
Die Bezeichnungen sollen so gewählt sein, dass $\lambda_1$ der
betragsgrösste Eigenwert ist, dass also
\[
|\lambda_1| \ge |\lambda_2| \ge \dots \ge |\lambda_n|.
\]
Wir nehmen für die folgende, einführende Diskussion ausserdem an, dass
sogar $|\lambda_1|>|\lambda_2|$ gilt.

Unter den genannten Voraussetzungen kann man jetzt den Gelfand-Radius
von $A$ berechnen.
Dazu muss man $|A^nv|$ für einen beliebigen Vektor $v$ und für
beliebiges $n$ berechnen.
Der Vektor $v$ lässt sich in der Eigenbasis von $A$ zerlegen, also
als Summe
\index{Eigenbasis}%
\[
v = v_1+v_2+\dots+v_n
\]
schreiben, wobei $v_i$ Eigenvektoren zum Eigenwert $\lambda_i$ sind oder
Nullvektoren.
Die Anwendung von $A^k$ ergibt dann
\[
A^k v
=
A^k v_1 + A^k v_2 + \dots + A^k v_n
=
\lambda_1^k v_1 + \lambda_2^k v_2 + \dots + \lambda_n^k v_n.
\]
Für den Grenzwert braucht man die Norm von $A^kv$, also
\begin{align}
|A^kv|
&= |\lambda_1^k v_1 + \lambda_2^k v_2 + \dots + \lambda_3 v_3|
\notag
\\
\Rightarrow\qquad
\frac{|A^kv|}{\lambda_1^k}
&=
\biggl|
v_1 +
\biggl(\frac{\lambda_2}{\lambda_1}\biggr)^k v_2
+
\dots
+
\biggl(\frac{\lambda_n}{\lambda_1}\biggr)^k v_n
\biggr|.
\label{buch:spektralradius:eqn:eigenwerte}
\end{align}
Da alle Quotienten $|\lambda_i/\lambda_1|<1$ sind für $i\ge 2$,
konvergieren alle Terme auf der rechten Seite von
\eqref{buch:spektralradius:eqn:eigenwerte}
ausser dem ersten gegen $0$.
Folglich ist
\[
\lim_{k\to\infty} \frac{|A^kv|}{|\lambda_1|^k}
=
|v_1|
\qquad\Rightarrow\qquad
\lim_{k\to\infty} \frac{|A^kv|^\frac1k}{|\lambda_1|}
=
\lim_{k\to\infty}|v_1|^{\frac1k}
=
1.
\]
Dies gilt für alle Vektoren $v$, für die $v_1\ne 0$ ist.
Der maximale Wert dafür wird erreicht, wenn man für 
$v$ einen Eigenvektor der Länge $1$ zum Eigenwert $\lambda_1$ einsetzt,
dann ist $v=v_1$.
Es folgt dann
\[
\pi(A)
=
\lim_{k\to\infty} \| A^k\|^\frac1k
=
\lim_{k\to\infty} |A^kv|^\frac1k
=
|\lambda_1|
=
\varrho(A).
\]
Damit ist gezeigt, dass im Spezialfall einer diagonalisierbaren Matrix der
Gelfand-Radius tatsächlich der Betrag des betragsgrössten Eigenwertes ist.
\index{Gelfand-Radius}%

\subsubsection{Blockmatrizen}
Wir betrachten jetzt eine $(n+m)\times(n+m)$-Blockmatrix der Form
\begin{equation}
A = \begin{pmatrix} B & 0 \\ 0 & C\end{pmatrix}
\label{buch:spektralradius:eqn:blockmatrix}
\end{equation}
mit einer $n\times n$-Matrix $B$ und einer $m\times m$-Matrix $C$.
Ihre Potenzen haben ebenfalls Blockform:
\[
A^k = \begin{pmatrix} B^k & 0 \\ 0 & C^k\end{pmatrix}.
\]
Ein Vektor $v$ kann in die zwei Summanden $v_1$ bestehen aus den
ersten $n$ Komponenten und $v_2$ bestehen aus den letzten $m$ 
Komponenten zerlegen.
Dann ist
\[
A^kv = B^kv_1 + C^kv_2.
\qquad\Rightarrow\qquad
|A^kv|
\le
|B^kv_1| + |C^kv_2|
\le 
\pi(B)^k |v_1| + \pi(C)^k |v_2|.
\]
Insbesondere haben wir das folgende Lemma gezeigt:

\begin{lemma}
\label{buch:spektralradius:lemma:diagonalbloecke}
Eine diagonale Blockmatrix $A$ \eqref{buch:spektralradius:eqn:blockmatrix}
Blöcken $B$ und $C$  hat Gelfand-Radius
\[
\pi(A) = \max ( \pi(B), \pi(C) )
\]
\end{lemma}

Selbstverständlich lässt sich das Lemma auf Blockmatrizen mit beliebig
vielen diagonalen Blöcken verallgemeinern.
\index{Blockmatrix}%

Für Diagonalmatrizen der genannten Art sind aber auch die 
Eigenwerte leicht zu bestimmen.
\index{Diagonalmatrix}%
Hat $B$ die Eigenwerte $\lambda_i^{(B)}$ mit $1\le i\le n$ und $C$ die
Eigenwerte $\lambda_j^{(C)}$ mit $1\le j\le m$, dann ist das charakteristische
Polynom der Blockmatrix $A$ natürlich
\index{charakteristisches Polynom}%
\index{Polynom!charakteristisch}%
\[
\chi_A(\lambda) = \chi_B(\lambda)\chi_C(\lambda),
\]
woraus folgt, dass die Eigenwerte von $A$ die Vereinigung der Eigenwerte
von $B$ und $C$ sind.
Daher gilt auch für die Spektralradius die Formel
\[
\varrho(A) = \max(\varrho(B) , \varrho(C)).
\]

\subsubsection{Jordan-Blöcke}
\index{Jordan-Block}%
Nicht jede Matrix ist diagonalisierbar, die bekanntesten Beispiele sind
die Matrizen
\begin{equation}
J_n(\lambda)
=
\begin{pmatrix}
\lambda &      1&       &       &       &       \\
        &\lambda&      1&       &       &       \\[-5pt]
        &       &\lambda&\ddots &       &       \\[-5pt]
        &       &       &\ddots &      1&       \\
        &       &       &       &\lambda&      1\\
        &       &       &       &       &\lambda
\end{pmatrix},
\label{buch:spektralradius:eqn:jordan}
\end{equation}
wobei $\lambda\in\mathbb C$ eine beliebige komplexe Zahl ist.
Wir nennen diese Matrizen {\em Jordan-Matrizen}.
Es ist klar, dass $J_n(\lambda)$ nur den $n$-fachen Eigenwert
$\lambda$ hat und dass der erste Standardbasisvektor ein
Eigenvektor zu diesem Eigenwert ist.

In der linearen Algebra lernt man, dass jede Matrix durch Wahl
\index{lineare!Algebra}%
einer geeigneten Basis als Blockmatrix der Form
\[
A
=
\begin{pmatrix}
J_{n_1}(\lambda_1) &        0         & \dots & 0 \\
       0         & J_{n_2}(\lambda_2) & \dots & 0 \\[-4pt]
\vdots           &\vdots            &\ddots &\vdots \\
       0         &        0         & \dots &J_{n_l}(\lambda_l)
\end{pmatrix}
\]
geschrieben werden kann\footnote{Sofern die Matrix komplexe Eigenwerte
hat muss man auch komplexe Basisvektoren zulassen.}.
Die früheren Beobachtungen über den Spektralradius und den
Gelfand-Radius von Blockmatrizen zeigen uns daher, dass
nur gezeigt werden muss, dass nur die Gleichheit des Gelfand-Radius
und des Spektral-Radius von Jordan-Blöcken gezeigt werden muss.

\subsubsection{Iterationsfolgen}
\begin{satz}
\label{buch:spektralradius:satz:grenzwert}
Sei $A$ eine $n\times n$-Matrix mit Spektralradius $\varrho(A)$.
Dann ist $\varrho(A)<1$ genau dann, wenn
\[
\lim_{k\to\infty} A^k = 0.
\]
Ist andererseits $\varrho(A) > 1$, dann ist
\[
\lim_{k\to\infty} \|A^k\|=\infty.
\]
\end{satz}

\begin{proof}[Beweis]
Wie bereits angedeutet reicht es, diese Aussagen für einen einzelnen
Jordan-Block mit Eigenwert $\lambda$ zu beweisen.
Die $k$-te Potenz von $J_n(\lambda)$ ist
\[
J_n(\lambda)^k
=
\renewcommand\arraystretch{1.35}
\begin{pmatrix}
\lambda^k    & \binom{k}{1} \lambda^{k-1} & \binom{k}{2}\lambda^{k-2}&\dots&
\binom{k}{n-1}\lambda^{k-n+1}\\
      0      &\lambda^k & \binom{k}{1} \lambda^{k-1} & \dots &\binom{k}{n-2}\lambda^{k-n+2}\\
      0     &      0    & \lambda^k & \dots &\binom{k}{n-k+3}\lambda^{k-n+3}\\
\vdots      & \vdots    &               &\ddots & \vdots\\
     0      &      0    &      0        &\dots  &\lambda^k
\end{pmatrix}.
\]
Falls $|\lambda| < 1$ ist, gehen alle Potenzen von $\lambda$ exponentiell
schnell gegen $0$, während die Binomialkoeffizienten nur polynomiell
schnell anwachsen. 
\index{Binomialkoeffizient}%
In diesem Fall folgt also $J_n(\lambda)\to 0$.

Falls $|\lambda| >1$ divergieren bereits die Elemente auf der Diagonalen,
also ist $\|J_n(\lambda)^k\|\to\infty$ mit welcher Norm auch immer man
man die Matrix misst.
\end{proof}

Aus dem Beweis kann man noch mehr ablesen.
Für $\varrho(A)< 1$ ist die Norm $ \|A^k\| \le M \varrho(A)^k$ für eine
geeignete Konstante $M$,
für $\varrho(A) > 1$ gibt es eine Konstante $m$ mit
$\|A^k\| \ge m\varrho(A)^k$.

\subsubsection{Der Satz von Gelfand}
Der Satz von Gelfand ergibt sich jetzt als direkte Folge aus dem
Satz~\ref{buch:spektralradius:satz:grenzwert}.

\begin{satz}[Gelfand]
\index{Satz von Gelfand}%
\index{Gelfand!Satz von}%
\label{buch:satz:gelfand}
Für jede komplexe $n\times n$-Matrix $A$ gilt
\[
\pi(A)
=
\lim_{k\to\infty}\|A^k\|^\frac1k
=
\varrho(A).
\]
\end{satz}

\begin{proof}[Beweis]
Der Satz~\ref{buch:spektralradius:satz:grenzwert} zeigt, dass der
Spektralradius ein scharfes Kriterium dafür ist, ob $\|A^k\|$ 
gegen 0 oder $\infty$ konvergiert.
Andererseits ändert ein Faktor $t$ in der Matrix $A$ den Spektralradius
ebenfalls um den gleichen Faktor, also $\varrho(tA)=t\varrho(A)$.
Natürlich gilt auch
\[
\pi(tA)
=
\lim_{k\to\infty} \|t^kA^k\|^\frac1k
=
\lim_{k\to\infty} t\|A^k\|^\frac1k
=
t\lim_{k\to\infty} \|A^k\|^\frac1k
=
t\pi(A).
\]

Wir betrachten jetzt die Matrix
\[
A(\varepsilon) = \frac{A}{\varrho(A) + \varepsilon}.
\]
Der Spektralradius von $A(\varepsilon)$ ist
\[
\varrho(A(\varepsilon)) = \frac{\varrho(A)}{\varrho(A)+\varepsilon},
\]
er ist also $>1$ für negatives $\varepsilon$ und $<1$ für positives
$\varepsilon$.
Aus dem Satz~\ref{buch:spektralradius:satz:grenzwert} liest man daher ab,
dass $\|A(\varepsilon)^k\|$ genau dann gegen $0$ konvergiert, wenn
$\varepsilon > 0$ ist und divergiert genau dann, wenn $\varepsilon< 0$ ist.

Aus der Bemerkung nach dem Beweis von
Satz~\ref{buch:spektralradius:satz:grenzwert} schliesst man daher, dass 
es im Fall $\varepsilon > 0$ eine Konstante $M$ gibt mit
\begin{align*}
\|A(\varepsilon) ^k\|\le M\varrho(A(\varepsilon))^k
\quad&\Rightarrow\quad
\|A(\varepsilon) ^k\|^\frac1k\le M^\frac1k\varrho(A(\varepsilon))
\\
&\Rightarrow\quad
\pi(A) \le  \varrho(A(\varepsilon))
\underbrace{\lim_{k\to\infty} M^\frac1k}_{\displaystyle=1}
=
\varrho(A(\varepsilon))
=
\varrho(A)+\varepsilon.
\end{align*}
Dies gilt für beliebige $\varepsilon >0$, es folgt daher
$\pi(A) \le \varrho(A)$.

Andererseits gibt es für $\varepsilon <0$ eine Konstante $m$ mit
\begin{align*}
\|A(\varepsilon) ^k\|\ge m\varrho(A(\varepsilon))^k
\quad&\Rightarrow\quad
\|A(\varepsilon) ^k\|^\frac1k\ge m^\frac1k\varrho(A(\varepsilon))
\\
&\Rightarrow\quad
\pi(A) \ge  \varrho(A(\varepsilon))
\underbrace{\lim_{k\to\infty} m^\frac1k}_{\displaystyle=1}
=
\varrho(A(\varepsilon))
=
\varrho(A)+\varepsilon.
\end{align*}
Dies gilt für beliebige $\varepsilon> 0$, es folgt daher
$\pi(A) \ge \varrho(A)$.
Zusammen mit $\pi(A) \le \varrho(A)$ folgt $\pi(A)=\varrho(A)$.
\end{proof}

