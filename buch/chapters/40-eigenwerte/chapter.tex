%
% chapter.tex -- Kapitel über Eigenwerte und Eigenvektoren
%
% (c) 2021 Prof Dr Andreas Müller, OST Ostschweizer Fachhochschule
%
\chapter{Eigenwerte und Eigenvektoren
\label{buch:chapter:eigenwerte-und-eigenvektoren}}
\lhead{Eigenwerte und Eigenvektoren}
\rhead{}
Die algebraischen Eigenschaften einer Matrix $A$ sind eng mit der
Frage nach linearen Beziehungen unter den Potenzen von $A^k$ verbunden.
Im Allgemeinen ist die Berechnung dieser Potenzen eher unübersichtlich,
es sei denn, die Matrix hat eine spezielle Form.
Die Potenzen einer Diagonalmatrix erhält man, indem man die Diagonalelemente
potenziert.
Auch für Dreiecksmatrizen ist mindestens die Berechnung der Diagonalelemente
von $A^k$ einfach.
Die Theorie der Eigenwerte und Eigenvektoren ermöglicht, Matrizen in
eine solche besonders einfache Form zu bringen.

In Abschnitt~\ref{buch:section:grundlagen} werden die grundlegenden
Definitionen der Eigenwerttheorie in Erinnerung gerufen.
Damit kann dann in Abschnitt~\ref{buch:section:normalformen}
gezeigt werden, wie Matrizen in besonders einfache Form gebracht
werden können.
Die Eigenwerte bestimmen auch die Eigenschaften von numerischen
Algorithmen, wie in den Abschnitten~\ref{buch:section:spektralradius}
und \ref{buch:section:numerisch} dargestellt wird.
Für viele Funktionen kann man auch den Wert $f(A)$ berechnen, unter
geeigneten Voraussetzungen an den Spektralradius.
Dies wird in Abschnitt~\ref{buch:section:spektraltheorie} beschrieben.


%
% grundlagen.tex -- Grundlagen
%
% (c) 2021 Prof Dr Andreas Müller, OST Ostschweizer Fachhochschule
%
\section{Grundlagen
\label{buch:section:grundlagen}}
\rhead{Grundlagen}
Die Potenzen $A^k$ sind besonders einfach zu berechnen, wenn die Matrix
Diagonalform hat, wenn also $A=\operatorname{diag}(\lambda_1,\dots,\lambda_n)$
ist.
In diesem Fall ist $Ae_k=\lambda_k e_k$ für jeden Standardbasisvektor $e_k$.
Statt sich auf Diagonalmatrizen zu beschränken könnten man also auch
Vektoren $v$ suchen, für die gilt $Av=\lambda v$, die also von $A$ nur
gestreckt werden.
Gelingt es, eine Basis aus solchen sogenanten {\em Eigenvektoren} zu finden,
dann kann man die Matrix $A$ durch Basiswechsel in diese Form bringen.

\begin{figure}
\centering
\includegraphics[width=\textwidth]{chapters/40-eigenwerte/images/kernbild.pdf}
\caption{Iterierte Kerne und Bilder einer $3\times 3$-Matrix mit Rang~2.
Die abnehmend geschachtelten iterierten Bilder
$\mathcal{J}^1(A) \subset \mathcal{J}^2(A)$
sind links dargestellt, die zunehmen geschachtelten iterierten Kerne
$\mathcal{K}^1(A) \subset \mathcal{K}^2(A)$ rechts.
\label{buch:eigenwerte:img:kernbild}}
\end{figure}

\begin{figure}
\centering
\includegraphics[width=\textwidth]{chapters/40-eigenwerte/images/kombiniert.pdf}
\caption{Iterierte Kerne und Bilder einer $3\times 3$-Matrix mit Rang~2.
Da $\dim\mathcal{J}^2(A)=1$ und $\dim\mathcal{J}^1(A)=2$ ist, muss es
einen Vektor in $\mathcal{J}^1(A)$ geben, der von $A$ auf $0$ abgebildet
wird, der also auch im Kern $\mathcal{K}^1(A)$ liegt.
Daher ist $\mathcal{K}^1(A)$ die Schnittgerade von $\mathcal{J}^1(A)$ und
$\mathcal{K}^2(A)$.
Man kann auch gut erkennen, dass
$\mathbb{R}^3
=
\mathcal{K}^1(A)\oplus \mathcal{J}^1(A)
=
\mathcal{K}^2(A) \oplus \mathcal{J}^2(A)$
ist.
\label{buch:eigenwerte:img:kombiniert}}
\end{figure}

%
% Kern und Bild von Matrixpotenzen
%
\subsection{Kern und Bild von Matrixpotenzen
\label{buch:subsection:kern-und-bild}}
In diesem Abschnitt ist $A\in M_n(\Bbbk)$, $A$ beschreibt eine lineare
Abbildung $f\colon\Bbbk^n\to \Bbbk^n$.
In diesem Abschnitt sollen Kern und Bild der Potenzen $A^k$ untersucht
werden.
\begin{definition}
Wir bezeichnen Kern und Bild der iterierten Abbildung $A^k$ mit
\[
\mathcal{K}^k(A)
=
\ker A^k
\qquad\text{und}\qquad
\mathcal{J}^k(A)
=
\operatorname{im} A^k.
\]
\end{definition}

Durch Iteration wird das Bild immer kleiner.
Wegen
\[
\mathcal{J}^k (A)
=
\operatorname{im} A^k
=
\operatorname{im} A^{k-1} A
=
\{ A^{k-1} Av\;|\; v \in \Bbbk^n\}
\subset
\{ A^{k-1} v\;|\; v \in \Bbbk^n\}
=
\mathcal{J}^{k-1}(A)
\]
folgt
\begin{equation}
\Bbbk^n
=
\operatorname{im}E
=
\operatorname{im}A^0
=
\mathcal{J}^0(A)
\supset
\mathcal{J}^1(A)
=
\operatorname{im}A
\supset
\mathcal{J}^2(A)
\supset\dots\supset
\mathcal{J}^k(A)
\supset
\mathcal{J}^{k+1}(A)
\supset \dots \supset
\{0\}.
\label{buch:eigenwerte:eqn:Jkchain}
\end{equation}
Für die Kerne gilt etwas Ähnliches.
Ein Vektor $x\in \mathcal{K}^k(A)$ erfüllt $A^kx=0$.
Dann erfüllt er aber erst recht auch
\[
A^{k+1}x=A\underbrace{A^kx}_{\displaystyle=0}=0,
\]
also ist $x\in\mathcal{K}^k(A)$.
Es folgt
\begin{equation}
\{0\}
=
\mathcal{K}^0(A) = \ker A^0 = \ker E
\subset
\mathcal{K}^1(A) = \ker A
\subset
\dots
\subset
\mathcal{K}^k(A)
\subset
\mathcal{K}^{k+1}(A)
\subset
\dots
\subset
\Bbbk^n.
\label{buch:eigenwerte:eqn:Kkchain}
\end{equation}
Neben diesen offensichtlichen Resultaten kann man aber noch mehr
sagen.
Es ist klar, dass in beiden Ketten
\label{buch:eigenwerte:eqn:Jkchain}
und
\label{buch:eigenwerte:eqn:Kkchain}
nur in höchstens $n$ Schritten eine wirkliche Änderung stattfinden 
kann.
Man kann aber sogar genau sagen, wo Änderungen stattfinden:

\begin{satz}
\label{buch:eigenwerte:satz:ketten}
Ist $A\in M_n(\Bbbk)$ eine $n\times n$-Matrix, dann gibt es eine Zahl $k$
so, dass
\[
\begin{array}{rcccccccccccl}
0=\mathcal{K}^0(A)
&\subsetneq& \mathcal{K}^1(A) &\subsetneq& \mathcal{K}^2(A)
&\subsetneq&\dots&\subsetneq&
\mathcal{K}^k(A) &=& \mathcal{K}^{k+1}(A) &=& \dots
\\
\Bbbk^n= \mathcal{J}^0(A)
&\supsetneq& \mathcal{J}^1(A) &\supsetneq& \mathcal{J}^2(A)
&\supsetneq&\dots&\supsetneq&
\mathcal{J}^k(A) &=& \mathcal{J}^{k+1}(A) &=& \dots
\end{array}
\]
ist.
\end{satz}

\begin{proof}[Beweis]
Es sind zwei Aussagen zu beweisen.
Erstens müssen wir zeigen, dass die Dimension von $\mathcal{K}^i(A)$ 
nicht mehr grösser werden kann, wenn sie zweimal hintereinander gleich war.
Nehmen wir daher an, dass $\mathcal{K}^i(A) = \mathcal{K}^{i+1}(A)$.
Wir müssen $\mathcal{K}^{i+2}(A)$ bestimmen.
$\mathcal{K}^{i+2}(A)$ besteht aus allen Vektoren $x\in\Bbbk^n$ derart,
dass $Ax\in \mathcal{K}^{i+1}(A)=\mathcal{K}^i(A)$ ist.
Daraus ergibt sich, dass $AA^ix=0$, also ist $x\in\mathcal{K}^{i+1}(A)$.
Wir erhalten also
$\mathcal{K}^{i+2}(A)\subset\mathcal{K}^{i+1}\subset\mathcal{K}^{i+2}(A)$,
dies ist nur möglich, wenn beide gleich sind.

Analog kann man für die Bilder vorgehen.
Wir nehmen an, dass $\mathcal{J}^i(A) = \mathcal{J}^{i+1}(A)$ und
bestimmten $\mathcal{J}^{i+2}(A)$.
$\mathcal{J}^{i+2}(A)$ besteht aus all jenen Vektoren, die als
$Ax$ mit $x\in\mathcal{J}^{i+1}(A)=\mathcal{J}^i(A)$ erhalten
werden können.
Es gibt also insbesondere ein $y\in\Bbbk^i$ mit $x=A^iy$.
Dann ist $Ax=A^{i+1}y\in\mathcal{J}^{i+1}(A)$.
Insbesondere besteht $\mathcal{J}^{i+2}(A)$ genau aus den Vektoren
von $\mathcal{J}^{i+1}(A)$.

Zweitens müssen wir zeigen, dass die beiden Ketten bei der gleichen
Potenz von $A$ konstant werden.
Dies folgt jedoch daraus, dass $\dim\mathcal{J}^i(A) = \operatorname{Rang} A^i
= n - \dim\ker A^i = n -\dim\mathcal{K}^i(A)$.
Der Raum $\mathcal{J}^k(A)$ hört also beim gleichen $i$ auf, kleiner
zu werden, bei dem auch $\mathcal{K}^i(A)$ aufhört, grösser zu werden.
\end{proof}

\begin{satz}
Die Zahl $k$ in Satz~\ref{buch:eigenwerte:satz:ketten}
ist nicht grösser als $n$, also
\[
\mathcal{K}^n(A) = \mathcal{K}^l(A)
\qquad\text{und}\qquad
\mathcal{J}^n(A) = \mathcal{J}^l(A)
\]
für $l\ge n$.
\end{satz}

\begin{proof}[Beweis]
Nach Satz~\ref{buch:eigenwerte:satz:ketten} muss die
Dimension von $\mathcal{K}^i(A)$ in jedem Schritt um mindestens
$1$ zunehmen, das ist nur möglich, bis zur Dimension $n$.
Somit können sich $\mathcal{K}^i(A)$ und $\mathcal{J}^i(A)$ für $i>n$
nicht mehr ändern.
\end{proof}

\begin{definition}
\label{buch:eigenwerte:def:KundJ}
Die gemäss Satz~\ref{buch:eigenwerte:satz:ketten} identischen Unterräume
$\mathcal{K}^i(A)$ für $i\ge k$ und die identischen Unterräume
$\mathcal{J}^i(A)$ für $i\ge k$ werden mit
\[
\begin{aligned}
\mathcal{K} &= \mathcal{K}^i(A)&&\forall i\ge k \qquad\text{und}
\\
\mathcal{J} &= \mathcal{J}^i(A)&&\forall i\ge k
\end{aligned}
\]
bezeichnet.
\end{definition}

%
% Inveriante Unterräume
%
\subsection{Invariante Unterräume
\label{buch:subsection:invariante-unterraeume}}
Kern und Bild sind der erste Schritt zu einem besseren Verständnis 
einer linearen Abbildung oder ihrer Matrix.
Invariante Räume dienen dazu, eine lineare Abbildung in einfachere
Abbildungen zwischen ``kleineren'' Räumen zu zerlegen, wo sie leichter
analysiert werden können.

\begin{definition}
Sei $f\colon V\to V$ eine lineare Abbildung eines Vektorraums in sich
selbst.
Ein Unterraum $U\subset V$ heisst {\em invarianter Unterraum},
wenn
\[
f(U) = \{ f(x)\;|\; x\in U\} \subset U
\]
gilt.
\end{definition}

Der Kern $\ker A$  einer linearen Abbildung ist trivialerweise ein
invarianter Unterraum, da alle Vektoren in $\ker A$ auf $0\in\ker A$
abgebildet werden.
Ebenso ist natürlich $\operatorname{im}A$ ein invarianter Unterraum,
denn jeder Vektor wird in $\operatorname{im}A$ abgebildet, insbesondere
auch jeder Vektor in $\operatorname{im}A$.

\begin{satz}
\label{buch:eigenwerte:satz:KJinvariant}
Sei $f\colon V\to V$ eine lineare Abbildung mit Matrix $A$.
Jeder der Unterräume $\mathcal{J}^i(A)$ und $\mathcal{K}^i(A)$ 
ist ein invarianter Unterraum.
\end{satz}

\begin{proof}[Beweis]
Sei $x\in\mathcal{K}^i(A)$, es gilt also $A^ix=0$.
Wir müssen überprüfen, dass $Ax\in\mathcal{K}^i(A)$.
Wir berechnen daher $A^i\cdot Ax=A^{i+1}x=A\cdot A^ix = A\cdot 0=0$,
was zeigt, dass $Ax\in\mathcal{K}^i(A)$.

Sei jetzt $x\in\mathcal{J}^i(A)$, es gibt also ein $y\in V$ derart, dass
$A^iy=x$.
Wir müssen überprüfen, dass $Ax\in\mathcal{J}^i(A)$.
Dazu berechnen wir $Ax=AA^iy=A^iAy\in\mathcal{J}^i(A)$, $Ax$ ist also das
Bild von $Ay$ unter $A^i$.
\end{proof}

\begin{korollar}
Die Unterräume $\mathcal{K}(A)\subset V$ und $\mathcal{J}(A)\subset V$
sind invariante Unterräume.
\end{korollar}

Die beiden Unterräume $\mathcal{K}(A)$ und $\mathcal{J}(A)$ sind besonders
interessant, da wir aus der Einschränkung der Abbildung $f$ auf diese
Unterräume mehr über $f$ lernen können.

\begin{satz}
\label{buch:eigenwerte:satz:fJinj}
Die Einschränkung von $f$ auf $\mathcal{J}(A)$ ist injektiv.
\end{satz}

\begin{proof}[Beweis]
Die Einschränkung von $f$ auf $\mathcal{J}^k(A)$ ist
$\mathcal{J}^k(A) \to \mathcal{J}^{k+1}(A)$, nach Definition von
$\mathcal{J}^{k+1}(A)$ ist diese Abbildung surjektiv.
Da aber $\mathcal{J}^k(A)=\mathcal{J}^{k+1}(A)$ ist, ist
$f\colon \mathcal{J}^k(A)\to\mathcal{J}^k(A)$ surjektiv,
also ist $f$ auf $\mathcal{J}^k(A)$ auch injektiv.
\end{proof}

Die beiden Unterräume $\mathcal{J}(A)$ und $\mathcal{K}(A)$
sind Bild und Kern der iterierten Abbildung mit Matrix $A^k$.
Das bedeutet, dass $\dim\mathcal{J}(A)+\mathcal{K}(A)=n$.
Da $\mathcal{K}(A)=\ker A^k$ und andererseits $A$ injektiv ist auf
$\mathcal{J}(A)$, muss $\mathcal{J}(A)\cap\mathcal{K}(A)=0$.
Es folgt, dass $V=\mathcal{J}(A) + \mathcal{K}(A)$.

In $\mathcal{K}(A)$ und $\mathcal{J}(A)$ kann man unabhängig voneinander
jeweils eine Basis wählen.
Die Basen von $\mathcal{K}(A)$ und $\mathcal{J}(A)$ zusammen ergeben
eine Basis von $V$.
Die Matrix $A'$ in dieser Basis wird die Blockform
\[
A'
=
\left(
\begin{array}{ccc|ccc}
&&&&&\\
&A_{\mathcal{K}'}&&&&\\
&&&&&\\
\hline
&&&&&\\
&&&&A_{\mathcal{J}'}&\\
&&&&&\\
\end{array}
\right)
\]
haben, wobei die Matrix $A_\mathcal{J}'$ invertierbar ist.
Die Zerlegung in invariante Unterräume ergibt also eine natürlich
Aufteilung der Matrix $A$ in kleiner Matrizen mit zum Teil bekannten
Eigenschaften.

%
% Spezialfall, nilpotente Matrizen
%
\subsection{Nilpotente Matrizen
\label{buch:subsection:nilpotente-matrizen}}
Die Zerlegung von $V$ in die beiden invarianten Unterräume $\mathcal{J}(A)$
und $\mathcal{K}(A)$ reduziert die lineare Abbildung auf zwei Abbildungen
mit speziellen Eigenschaften.
Es wurde bereits in Satz~\label{buch:eigenwerte:satz:fJinj} gezeigt,
dass die Einschränkung auf $\mathcal{J}(A)$ injektiv ist.
Die Einschränkung auf $\mathcal{K}(A)$ bildet nach Definition alle
Vektoren nach $k$-facher Iteration auf $0$ ab, $A^k\mathcal{K}(A)=0$.
Solche Abbildungen haben eine speziellen Namen.

\begin{definition}
\label{buch:eigenwerte:def:nilpotent}
Eine Matrix $A$ heisst nilpotent, wenn es eine Zahl $k$ gibt, so dass
$A^k=0$.
\end{definition}

\begin{beispiel}
Obere (oder untere) Dreiecksmatrizen mit Nullen auf der Diagonalen
sind nilpotent.
Wir rechnen dies wie folgt nach.
Die Matrix $A$ mit Einträgen $a_{ij}$
\[
A=\begin{pmatrix}
  0   &a_{12}&a_{13}&\dots &a_{1,n-1}&a_{1n}   \\
  0   &  0   &a_{23}&\dots &a_{1,n-1}&a_{2n}   \\
  0   &  0   &  0   &\dots &a_{1,n-1}&a_{3n}   \\
\vdots&\vdots&\vdots&\ddots&\vdots   &\vdots   \\
  0   &  0   &  0   &\dots &  0      &a_{n-1,n}\\
  0   &  0   &  0   &\dots &  0      &  0
\end{pmatrix}
\]
erfüllt $a_{ij}=0$ für $i\ge j$.
Wir zeigen jetzt, dass sich bei der Multiplikation die nicht
verschwinden Elemente bei der Multiplikation noch rechts oben
verschieben.
Dazu multiplizieren wir zwei Matrizen $B$ und $C$ mit
$b_{ij}=0$ für $i+k>j$ und $c_{ij}=0$ für $i+l>j$.
In der folgenden graphischen Darstellung der Matrizen sind die
Bereiche, wo die Matrixelemente verschwinden, weiss.
\begin{center}
\includegraphics{chapters/40-eigenwerte/images/nilpotent.pdf}
\end{center}
Bei der Berechnung des Elementes $d_{ij}$ wird die Zeile $i$ von $B$
mit der Spalte $j$ von $C$ multipliziert.
Die blau eingefärbten Elemente in dieser Zeile und Spalte sind $0$.
Aus der Darstellung ist abzulesen, dass das Produkt verschwindet, 
die roten, von $0$ verschiedenen Elemente von den blauen Elementen
annihiliert werden.
Dies passiert immer, wenn $i+k>j-l$ ist, oder $i+(k+l)> j$.

Wir wenden diese Beobachtung jetzt auf die Potenzen $A^s$ an.
Für die Matrixelemente von $A^s$ schreiben wir $a^s_{ij}$.
Wir behaupten, dass die Matrixelemente $A^s$ die Bedingung
$a_{ij}^s=0$ für $i+s>j$ erfüllen.
Dies ist für $s=1$ nach Voraussetzung richtig, dies ist die
Induktionsvoraussetzung.
Nehmen wir jetzt an, dass $a_{ij}^s=0$ für $i+s>j$, dann folgt
aus obiger Rechnung, dass $a_{ij}^{s+1}=0$ für $i+s+1>j$, so
dass die Bedingung auch für $A^s$ gilt (Induktionsschritt).
Mit vollständiger Induktion folgt, dass $a_{ij}^s=0$ für $i+s>j$.
Insbesondere ist $A^n=0$, die Matrix $A$ ist nilpotent.
\end{beispiel}

Man kann die Konstruktion der Unterräume $\mathcal{K}^i(A)$ weiter
dazu verwenden, eine Basis zu finden, in der eine nilpotente Matrix
eine besonders einfach Form erhält.

\begin{satz}
\label{buch:eigenwerte:satz:nnilpotent}
Sei $A$ eine nilpotente $n\times n$-Matrix mit der Eigenschaft, dass
$A^{n-1}\ne 0$.
Dann gibt es eine Basis so, dass $A$ die Form
\begin{equation}
A'
=
\begin{pmatrix}
0&1& &      & & \\
 &0&1&      & & \\
 & &0&      & & \\
 & & &\ddots&1& \\
 & & &      &0&1\\
 & & &      & &0\\
\end{pmatrix}
\label{buch:eigenwerte:eqn:nnilpotent}
\end{equation}
bekommt.
\end{satz}

\begin{proof}[Beweis]
Da $A^{n-1}\ne 0$ ist, gibt es einen Vektor $b_n$ derart, dass $A^{n-1}b_n\ne0$.
Wir konstruieren die Vektoren
\[
b_n,\;
b_{n-1}=Ab_n,\;
b_{n-2}=Ab_{n-1},\;
\dots,\;
b_2=Ab_3,\;
b_1=Ab_2.
\]
Aus der Konstruktion folgt $b_1=A^{n-1}b_n\ne 0$, aber $Ab_1=A^nb_n=0$.
Aus der Konstruktion der iterierten Kerne $\mathcal{K}^i(A)$ folgt jetzt,
dass die Vektoren $b_1,\dots,b_n$ eine Basis bilden.
In dieser Basis hat die Matrix die Form~\ref{buch:eigenwerte:eqn:nnilpotent}.
\end{proof}

\begin{definition}
Wir bezeichnen mit $N_n$ eine Matrix der Form
\eqref{buch:eigenwerte:eqn:nnilpotent}.
\end{definition}

Mit etwas mehr Sorgfalt kann man auch die Bedingung, dass $A^{n-1}\ne 0$
sein muss, im Satz~\ref{buch:eigenwerte:satz:nnilpotent} loswerden.

\begin{satz}
\label{buch:eigenwerte:satz:allgnilpotent}
Sei $A$ ein nilpotente Matrix, dann gibt es eine Basis, in der die Matrix
aus lauter Nullen besteht ausser in den Einträgen unmittelbar oberhalb der 
Hauptdiagonalen, wo die Einträge $0$ oder $1$ sind.
Insbesondere zerfällt eine solche Matrix in Blöcke der Form $N_{k_i}$,
$i=1,\dots,l$,
wobei $k_1+\dots+k_l=n$ sein muss:
\begin{equation}
\def\temp#1{\multicolumn{1}{|c}{\raisebox{0pt}[17pt][12pt]{\phantom{x}$#1\mathstrut$}\phantom{x}}}
A'
=\left(
\begin{array}{cccc}
\cline{1-1}
\temp{N_{k_1}} &\multicolumn{1}{|c}{}&        &           \\
\cline{1-2}
          &\temp{N_{k_2}}&\multicolumn{1}{|c}{}&           \\
\cline{2-3}
          &           &\temp{\ddots}&\multicolumn{1}{|c}{}\\
\cline{3-4}
          &           &        &\multicolumn{1}{|c|}{\raisebox{0pt}[17pt][12pt]{\phantom{x}$N_{k_l}$}\phantom{x}}\\
\cline{4-4}
\end{array}
\right)
\label{buch:eigenwerte:eqn:allgnilpotent}
\end{equation}
\end{satz}

Die Einschränkung von $f$ auf den invarianten Unterraum $\mathcal{K}(A)$
ist nilpotent.
Die Zerlegung $V=\mathcal{J}(A)\oplus \mathcal{K}(A)$ führt also zu einer
Zerlegung der Abbildung $f$ in eine invertierbare Abbildung
$\mathcal{J}(A)\to\mathcal{J}(A)$ und eine
nilpotente Abbildung $\mathcal{K}(A)\to\mathcal{K}(A)$.
Nach Satz~\ref{buch:eigenwerte:satz:allgnilpotent} kann man in
$\mathcal{K}(A)$ eine Basis so wählen, dass die Matrix die Blockform
\eqref{buch:eigenwerte:eqn:allgnilpotent} erhält.

%
% Begriff des Eigenwertes und Eigenvektors
%
\subsection{Eigenwerte und Eigenvektoren
\label{buch:subsection:eigenwerte-und-eigenvektoren}}
In diesem Abschnitt betrachten wir Vektorräume $V=\Bbbk^n$ über einem
beliebigen Körper $\Bbbk$ und quadratische Matrizen
$A\in M_n(\Bbbk)$.
In den meisten Anwendungen wird $\Bbbk=\mathbb{R}$ sein.
Da aber in $\mathbb{R}$ nicht alle algebraischen Gleichungen lösbar sind,
ist es manchmal notwendig, den Vektorraum zu erweitern um zum Beispiel
Eigenschaften der Matrix $A$ abzuleiten.

\begin{definition}
Ein Vektor $v\in V$ heisst {\em Eigenvektor} von $A$ zum Eigenwert
$\lambda\in\Bbbk$, wenn $v\ne 0$ und $Av=\lambda v$ gilt.
\end{definition}

Die Bedingung $v\ne 0$ dient dazu, pathologische Situationen auszuschliessen.
Für den Nullvektor gilt $A0=\lambda 0$ für jeden beliebigen Wert von
$\lambda\in\Bbbk$.
Würde man $v=0$ zulassen, wäre jede Zahl in $\Bbbk$ ein Eigenwert,
ein Eigenwert von $A$ wäre nichts besonderes.
Ausserdem wäre $0$ ein Eigenvektor zu jedem beliebigen Eigenwert.

Eigenvektoren sind nicht eindeutig bestimmt, jedes von $0$ verschiedene
Vielfache von $v$ ist ebenfalls ein Eigenvektor.
Zu einem Eigenwert kann man also einen Eigenvektor jeweils mit 
geeigneten Eigenschaften finden, zum Beispiel kann man für $\Bbbk = \mathbb{R}$
Eigenvektoren auf Länge $1$ normieren.
Im Folgenden werden wir oft die abkürzend linear unabhängige Eigenvektoren
einfach als ``verschiedene'' Eigenvektoren bezeichnen.

Wenn $v$ ein Eigenvektor von $A$ zum Eigenwert $\lambda$ ist, dann kann
man ihn mit zusätzlichen Vektoren $v_2,\dots,v_n$ zu einer Basis
$\mathcal{B}=\{v,v_2,\dots,v_n\}$
von $V$ ergänzen.
Die Vektoren $v_k$ mit $k=2,\dots,n$ werden von $A$ natürlich auch
in den Vektorraum $V$ abgebildet, können also als Linearkombinationen
\[
Av = a_{1k}v + a_{2k}v_2 + a_{3k}v_3 + \dots a_{nk}v_n
\]
dargestellt werden.
In der Basis $\mathcal{B}$ bekommt die Matrix $A$ daher die Form
\[
A'
=
\begin{pmatrix}
\lambda&a_{12}&a_{13}&\dots &a_{1n}\\
    0  &a_{22}&a_{23}&\dots &a_{2n}\\
    0  &a_{32}&a_{33}&\dots &a_{3n}\\
\vdots &\vdots&\vdots&\ddots&\vdots\\
    0  &a_{n2}&a_{n3}&\dots &a_{nn}
\end{pmatrix}.
\]
Bereits ein einzelner Eigenwert und ein zugehöriger Eigenvektor
ermöglichen uns also, die Matrix in eine etwas einfachere Form
zu bringen.

\begin{definition}
Für $\lambda\in\Bbbk$ heisst
\[
E_\lambda
=
\{ v\;|\; Av=\lambda v\}
\]
der {\em Eigenraum} zum Eigenwert $\lambda$.
\index{Eigenraum}%
\end{definition}

Der Eigenraum $E_\lambda$ ist ein Unterraum von $V$, denn wenn
$u,v\in E_\lambda$, dann ist
\[
A(su+tv)
=
sAu+tAv
=
s\lambda u + t\lambda v
=
\lambda(su+tv),
\]
also ist auch $su+tv\in E_\lambda$.
Der Fall $E_\lambda = \{0\}=0$ bedeutet natürlich, dass $\lambda$ gar kein
Eigenwert ist.

\begin{satz}
Wenn $\dim E_\lambda=n$, dann ist $A=\lambda E$.
\end{satz}

\begin{proof}[Beweis]
Da $V$ ein $n$-dimensionaler Vektoraum ist, ist $E_\lambda=V$.
Jeder Vektor $v\in V$ erfüllt also die Bedingung $Av=\lambda v$,
oder $A=\lambda E$.
\end{proof}

Wenn man die Eigenräume von $A$ kennt, dann kann man auch die Eigenräume
von $A+\mu E$ berechnen.
Ein Vektor $v\in E_\lambda$ erfüllt
\[
Av=\lambda v
\qquad\Rightarrow\qquad
(A+\mu)v = \lambda v + \mu v
=
(\lambda+\mu)v,
\]
somit ist $v$ ein Eigenvektor von $A+\mu E$ zum Eigenwert $\lambda+\mu$.
Insbesondere können wir statt die Eigenvektoren von $A$ zum Eigenwert $\lambda$
zu studieren, auch die Eigenvektoren zum Eigenwert $0$ von $A-\lambda E$
untersuchen.

%
% Invariante Räume
%
\subsection{Verallgemeinerte Eigenräume
\label{buch:subsection:verallgemeinerte-eigenraeume}}
Wenn $\lambda$ ein Eigenwert der Matrix $A$ ist, dann ist
ist $A-\lambda E$ injektiv und $\ker(A-\lambda E)\ne 0$.
Man kann daher die invarianten Unterräume $\mathcal{K}(A-\lambda E)$
und $\mathcal{J}(A-\lambda E)$.

\begin{beispiel}
Wir untersuchen die Matrix
\[
A
=
\begin{pmatrix}
1&1&-1&0\\
0&3&-1&1\\
0&2& 0&1\\
0&0& 0&2
\end{pmatrix}
\]
Man kann zeigen, dass $\lambda=1$ ein Eigenwert ist.
Wir suchen die Zerlegung des Vektorraums $\mathbb{R}^4$ in invariante
Unterräume $\mathcal{K}(A-E)$ und $\mathcal{J}(A-E)$.
Die Matrix $B=A-E$ ist
\[
B
=
\begin{pmatrix}
0&1&-1&0\\
0&2&-1&1\\
0&2&-1&1\\
0&0& 0&2
\end{pmatrix}
\]
und wir berechnen davon die Potenz
\[
D=B^4=(A-E)^4
=
\begin{pmatrix}
0&0& 0&0\\
0&2&-1&4\\
0&2&-1&4\\
0&0& 0&1
\end{pmatrix}.
\]
Daraus kann man ablesen, dass das Bild $\operatorname{im}D$
von $D$ die Basis
\[
b_1
=
\begin{pmatrix}
0\\0\\0\\1
\end{pmatrix}
, \qquad
b_2
=
\begin{pmatrix}
0\\1\\1\\0
\end{pmatrix}
\]
hat.
Für den Kern von $D$ können wir zum Beispiel die Basisvektoren
\[
b_3
=
\begin{pmatrix}
0\\1\\2\\0
\end{pmatrix}
,\qquad
b_4
=
\begin{pmatrix}
1\\0\\0\\0
\end{pmatrix}
\]
verwenden.

Als erstes überprüfen wir, ob diese Basisvektoren tatsächlich invariante
Unterräume sind.
Für $\mathcal{J}(A-E) = \langle b_1,b_2\rangle$
berechnen wir
\begin{align*}
(A-E)b_1
&=
\begin{pmatrix} 0\\4\\4\\1 \end{pmatrix}
=
4b_2+b_1,
\\
(A-E)b_2
&=
\begin{pmatrix} 0\\1\\1\\0 \end{pmatrix}
=
b_2.
\end{align*}
Dies beweist, dass $\mathcal{J}(A-E)$ invariant ist.
In dieser Basis hat die von $A-E$ beschriebene lineare Abbildung
auf $\mathcal{J}(A-E)$ die Matrix
\[
A_{\mathcal{J}(A-E)}
=
\begin{pmatrix}
1&4\\
0&1
\end{pmatrix}.
\]

Für den Kern $\mathcal{K}(A-E)$ findet man analog
\[
\left.
\begin{aligned}
Ab_3
&=
-b_4
\\
Ab_4
&=0
\end{aligned}
\quad\right\}
\qquad\Rightarrow\qquad
A_{\mathcal{K}(A-E)}
=
\begin{pmatrix}
0&-1\\
0& 0
\end{pmatrix}.
\]
In der Basis $\mathcal{B}=\{b_1,b_2,b_3,b_4\}$ hat $A$ die Matrix
in Blockform
\[
A'
=
\left(
\begin{array}{cc|cr}
2&4& & \\
0&2& & \\
\hline
 & &1&-1\\
 & &0& 1
\end{array}\right),
\]
die Blöcke gehören zu den invarianten Unterräumen $\mathcal{K}(A-E)$
und $\mathcal{K}(A-E)$.
Die aus $A-E$ gewonnen invarianten Unterräume sind offenbar auch invariante
Unterräume für $A$.
\end{beispiel}

\begin{definition}
Ist $A$ eine Matrix mit Eigenwert $\lambda$, dann heisst der invariante
Unterraum
\[
\mathcal{E}_{\lambda}(A)
=
\mathcal{K}(A-\lambda E)
\]
der verallgemeinerte Eigenraum von $A$.
\end{definition}

Es ist klar, dass
$E_\lambda(A)=\ker (A-\lambda E)\subset\mathcal{E}_{\lambda}(A)$.

\subsection{Zerlegung in invariante Unterräume
\label{buch:subsection:zerlegung-in-invariante-unterraeume}}
Wenn $\lambda$ kein Eigenwert von $A$ ist, dann ist $A-\lambda E$
injektiv und damit $\ker(A-\lambda E)=0$.
Es folgt, dass $\mathcal{K}^i(A-\lambda E)=0$ und daher auch
$\mathcal{J}^i(A-\lambda E)=V$.
Die Zerlegung in invariante Unterräume $\mathcal{J}(A-\lambda E)$ und
$\mathcal{K}(A-\lambda E)$ liefert in diesem Falle also nichts Neues.

Für einen Eigenwert $\lambda_1$ von $A$ dagegen, erhalten wir die Zerlegung
\[
V
=
\mathcal{E}_{\lambda_1}(A)
\oplus
\underbrace{\mathcal{J}(A-\lambda_1 E)}_{\displaystyle =V_2},
\]
wobei $\mathcal{E}_{\lambda_1}(A)\ne 0$ ist.
Die Matrix $A-\lambda_1 E$ ist eingeschränkt auf $\mathcal{E}_{\lambda_1}(A)$
nilpotent.
Die Zerlegung in invariante Unterräume ist zwar mit Hilfe von $A-\lambda_1E$
gewonnen worden, ist aber natürlich auch eine Zerlegung in invariante 
Unterräume für $A$.
Wir können daher das Problem auf $V_2$ einschränken und nach einem weiteren
Eigenwert $\lambda_2$ von $A$ in $V_2$ suchen, was wieder eine Zerlegung
in invariante Unterräume liefert.
Indem wir so weiterarbeiten, bis wir den ganzen Raum ausgeschöpft haben,
können wir eine Zerlegung des ganzen Raumes $V$ finden, so dass $A$ auf
jedem einzelnen Summanden eine sehr einfach Form hat:

\begin{satz}
\label{buch:eigenwerte:satz:zerlegung-in-eigenraeume}
Sei $V$ ein $\Bbbk$-Vektorraum und $f$ eine lineare Abbildung mit Matrix
$A$ derart, dass alle Eigenwerte $\lambda_1,\dots,\lambda_l$ von $A$
in $\Bbbk$ sind.
Dann gibt es eine Zerlegung von $V$ in verallgemeinerte Eigenräume
\[
V
=
\mathcal{E}_{\lambda_1}(A)
\oplus
\mathcal{E}_{\lambda_2}(A)
\oplus
\dots
\oplus
\mathcal{E}_{\lambda_l}(A).
\]
Die Einschränkung von $A-\lambda_{i}E$ auf den Eigenraum 
$\mathcal{E}_{\lambda_i}(A)$ ist nilpotent.
\end{satz}

\subsection{Das charakteristische Polynom
\label{buch:subsection:das-charakteristische-polynom}}
Ein Eigenvektor von $A$ erfüllt $Av=\lambda v$ oder gleichbedeutend
$(A-\lambda E)v=0$, er ist also eine nichttriviale Lösung des homogenen
Gleichungssystems mit Koeffizientenmatrix $A-\lambda E$. 
Ein Eigenwert ist also ein Skalar derart, dass $A-\lambda E$
singulär ist.
Ob eine Matrix singulär ist, kann mit der Determinante festgestellt
werden.
Die Eigenwerte einer Matrix $A$ sind daher die Nullstellen
von $\det(A-\lambda E)$.

\begin{definition}
Das {\em charakteristische Polynom}
\[
\chi_A(x)
=
\det (A-x E)
=
\left|
\begin{matrix}
a_{11}-x & a_{12}   & \dots  & a_{1n} \\
a_{21}   & a_{22}-x & \dots  & a_{2n} \\
\vdots   &\vdots    &\ddots  & \vdots \\
a_{n1}   & a_{n2}   &\dots   & a_{nn}-x
\end{matrix}
\right|.
\]
der Matrix $A$ ist ein Polynom vom Grad $n$ mit Koeffizienten in $\Bbbk$.
\end{definition}

Findet man eine Nullstelle $\lambda\in\Bbbk$ von $\chi_A(x)$,
dann ist die Matrix $A-\lambda E\in M_n(\Bbbk)$ und mit dem Gauss-Algorithmus
kann man auch mindestens einen Vektor $v\in \Bbbk^n$ finden,
der $Av=\lambda v$ erfüllt.
Eine Matrix der Form  wie in Satz~\ref{buch:eigenwerte:satz:jordanblock}
hat
\[
\chi_A(x)
=
\left|
\begin{matrix}
\lambda-x &     1     &           &      &         &         \\
          & \lambda-x &     1     &      &         &         \\
          &           & \lambda-x &      &         &         \\
          &           &           &\ddots&         &         \\
          &           &           &      &\lambda-x&     1   \\
          &           &           &      &         &\lambda-x
\end{matrix}
\right|
=
(\lambda-x)^n
=
(-1)^n (x-\lambda)^n
\]
als charakteristisches Polynom, welches $\lambda$ als einzige
Nullstelle hat.
Der Eigenraum der Matrix ist aber nur eindimensional, man kann also
im Allgemeinen für jede Nullstelle des charakteristischen Polynoms
nicht mehr als einen Eigenvektor (d.~h.~einen eindimensionalen Eigenraum)
erwarten.

Wenn das charakteristische Polynom von $A$ keine Nullstellen in $\Bbbk$ hat,
dann kann es auch keine Eigenvektoren in $\Bbbk^n$ geben.
Gäbe es nämlich einen solchen Vektor, dann müsste eine der Komponenten
des Vektors von $0$ verschieden sein, wir nehmen an, dass es die Komponente
in Zeile $k$ ist.
Die Komponente $v_k$ kann man auf zwei Arten berechnen, einmal als
die $k$-Komponenten von $Av$ und einmal als $k$-Komponente von $\lambda v$:
\[
a_{k1}v_1+\dots+a_{kn}v_n = \lambda v_k.
\]
Da $v_k\ne 0$ kann man nach $\lambda$ auflösen und erhält
\[
\lambda = \frac{a_{k1}v_1+\dots + a_{kn}v_n}{v_k}.
\]
Alle Terme auf der rechten Seite sind in $\Bbbk$ und werden nur mit
Körperoperationen in $\Bbbk$ verknüpft, also muss auch $\lambda\in\Bbbk$
sein, im Widerspruch zur Annahme.

Durch Hinzufügen von geeigneten Elementen können wir immer zu einem 
Körper $\Bbbk'$ übergehen, in dem das charakteristische Polynom
in Linearfaktoren zerfällt.
In diesem Körper kann man jetzt das homogene lineare Gleichungssystem
mit Koeffizientenmatrix $A-\lambda E$ lösen und damit mindestens 
einen Eigenvektor $v$ für jeden Eigenwert finden.
Die Komponenten von $v$ liegen in $\Bbbk'$, und mindestens eine davon kann
nicht in $\Bbbk$ liegen.
Das bedeutet aber nicht, dass man diese Vektoren nicht für theoretische
Überlegungen über von $\Bbbk'$ unabhängige Eigenschaften der Matrix $A$ machen.
Das folgende Beispiel soll diese Idee illustrieren.

\begin{beispiel}
Wir arbeiten in diesem Beispiel über dem Körper $\Bbbk=\mathbb{Q}$.
Die Matrix
\[
A=\begin{pmatrix}
-4&7\\
-2&4
\end{pmatrix}
\in
M_2(\mathbb{Q})
\]
hat das charakteristische Polynom
\[
\chi_A(x)
=
\left|
\begin{matrix}
-4-x&7\\-2&4-x
\end{matrix}
\right|
=
(-4-x)(4-x)-7\cdot(-2)
=
-16+x^2+14
=
x^2-2.
\]
Die Nullstellen sind $\pm\sqrt{2}$ und damit nicht in $\mathbb{Q}$.
Wir gehen daher über zum Körper $\mathbb{Q}(\!\sqrt{2})$, in dem
sich zwei Nullstellen $\lambda=\pm\sqrt{2}$ finden lassen.
Zu jedem Eigenwert lässt sich auch ein Eigenvektor
$v_{\pm\sqrt{2}}\in \mathbb{Q}(\!\sqrt{2})^2$, und unter Verwendung dieser
Basis bekommt die Matrix $A'=TAT^{-1}$ Diagonalform.
Die Transformationsmatrix $T$ enthält Matrixelemente aus
$\mathbb{Q}(\!\sqrt{2})$, die nicht in $\mathbb{Q}$ liegen.
Die Matrix $A$ lässt sich also über dem Körper $\mathbb{Q}(\!\sqrt{2})$
diagonalisieren, nicht aber über dem Körper $\mathbb{Q}$.

Da $A'$ Diagonalform hat mit $\pm\sqrt{2}$ auf der Diagonalen, folgt
$A^{\prime 2} = 2E$, die Matrix $A'$ erfüllt also die Gleichung
\begin{equation}
A^{\prime 2}-E= \chi_{A}(A) = 0.
\label{buch:grundlagen:eqn:cayley-hamilton-beispiel}
\end{equation}
Dies is ein Spezialfall des Satzes von Cayley-Hamilton~\ref{XXX}
welcher besagt, dass jede Matrix $A$ eine Nullstelle ihres 
charakteristischen Polynoms ist: $\chi_A(A)=0$.
Die Gleichung~\ref{buch:grundlagen:eqn:cayley-hamilton-beispiel}
wurde zwar in $\mathbb{Q}(\!\sqrt{2})$ hergeleitet, aber in ihr kommen
keine Koeffizienten aus $\mathbb{Q}(\!\sqrt{2})$ vor, die man nicht auch
in $\mathbb{Q}$ berechnen könnte.
Sie gilt daher ganz allgemein.
\end{beispiel}

\begin{beispiel}
Die Matrix
\[
A=\begin{pmatrix}
32&-41\\
24&-32
\end{pmatrix}
\in
M_2(\mathbb{R})
\]
über dem Körper $\Bbbk = \mathbb{R}$
hat das charakteristische Polynom
\[
\det(A-xE)
=
\left|
\begin{matrix}
32-x&-41  \\
25  &-32-x
\end{matrix}
\right|
=
(32-x)(-32-x)-25\cdot(-41)
=
x^2-32^2 + 1025
=
x^2+1.
\]
Die charakteristische Gleichung $\chi_A(x)=0$ hat in $\mathbb{R}$
keine Lösungen, daher gehen wir zum Körper $\Bbbk'=\mathbb{C}$ über,
in dem dank dem Fundamentalsatz der Algebra alle Nullstellen zu finden
sind, sie sind $\pm i$.
In $\mathbb C$ lassen sich dann auch Eigenvektoren finden, man muss dazu die
folgenden linearen Gleichungssyteme lösen:
\begin{align*}
\begin{tabular}{|>{$}c<{$}>{$}c<{$}|}
32-i&-41\\
25  &-32-i
\end{tabular}
&
\rightarrow
\begin{tabular}{|>{$}c<{$}>{$}c<{$}|}
1 & t\\
0 &  0 
\end{tabular}
&
\begin{tabular}{|>{$}c<{$}>{$}c<{$}|}
32+i&-41\\
25  &-32+i
\end{tabular}
&
\rightarrow
\begin{tabular}{|>{$}c<{$}>{$}c<{$}|}
1 & \overline{t}\\
0 &  0 
\end{tabular},
\intertext{wobei wir $t=-41/(32-i) =-41(32+i)/1025= -1.28 -0.04i = (64-1)/50$
abgekürzt haben.
Die zugehörigen Eigenvektoren sind}
v_i&=\begin{pmatrix}t\\i\end{pmatrix}
&
v_{-i}&=\begin{pmatrix}\overline{t}\\i\end{pmatrix}
\end{align*}
Mit den Vektoren $v_i$ und $v_{-i}$ als Basis kann die Matrix $A$ als
komplexe Matrix, also mit komplexem $T$ in die komplexe Diagonalmatrix 
$A'=\operatorname{diag}(i,-i)$ transformiert werden.
Wieder kann man sofort ablesen, dass $A^{\prime2}+E=0$, und wieder kann
man schliessen, dass für die relle Matrix $A$ ebenfalls $\chi_A(A)=0$
gelten muss.
\end{beispiel}





%
% normalformen.tex -- Normalformen einer Matrix
%
% (c) 2021 Prof Dr Andreas Müller, OST Ostschweizer Fachhochschule
%
\section{Normalformen
\label{buch:section:normalformen}}
\rhead{Normalformen}
In den Beispielen im vorangegangenen wurde wiederholt der Trick
verwendet, den Koeffizientenkörper so zu erweitern, dass das
charakteristische Polynom in Linearfaktoren zerfällt und 
für jeden Eigenwert Eigenvektoren gefunden werden können.
Diese Idee ermöglicht, eine Matrix in einer geeigneten Körpererweiterung
in eine besonders einfache Form zu bringen, das Problem dort zu lösen.
Anschliessend kann man sich darum kümmern in welchem Mass die gewonnenen
Resultate wieder in den ursprünglichen Körper transportiert werden können.

\subsection{Diagonalform}
Sei $A$ eine beliebige Matrix mit Koeffizienten in $\Bbbk$ und sei $\Bbbk'$
eine Körpererweiterung von $\Bbbk$ derart, dass das charakteristische
Polynom in Linearfaktoren
\[
\chi_A(x)
=
(x-\lambda_1)^{k_1}\cdot (x-\lambda_2)^{k_2}\cdot\dots\cdot (x-\lambda_m)^{k_m}
\]
mit Vielfachheiten $k_1$ bis $k_m$ zerfällt, $\lambda_i\in\Bbbk'$.
Zu jedem Eigenwert $\lambda_i$ gibt es sicher einen Eigenvektor, wir 
wollen aber in diesem Abschnitt zusätzlich annehmen, dass es eine Basis
aus Eigenvektoren gibt.
In dieser Basis bekommt die Matrix Diagonalform, wobei auf der 
Diagonalen nur Eigenwerte vorkommen können.
Man kann die Vektoren so anordnen, dass die Diagonalmatrix in Blöcke
der Form $\lambda_iE$ zerfällt
\[
\def\temp#1{\multicolumn{1}{|c}{\raisebox{0pt}[12pt][7pt]{\phantom{x}$#1$}\phantom{x}}}
A'
=\left(
\begin{array}{cccc}
\cline{1-1}
\temp{\lambda_1E} &\multicolumn{1}{|c}{}&        &           \\
\cline{1-2}
          &\temp{\lambda_2E}&\multicolumn{1}{|c}{}&           \\
\cline{2-3}
          &           &\temp{\ddots}&\multicolumn{1}{|c}{}\\
\cline{3-4}
          &           &        &\multicolumn{1}{|c|}{\raisebox{0pt}[12pt][7pt]{\phantom{x}$\lambda_mE$}\phantom{x}}\\
\cline{4-4}
\end{array}
\right)
\]
Über die Grösse eines solchen $\lambda_iE$-Blockes können wir zum jetzigen
Zeitpunkt noch keine Aussagen machen.

Die Matrizen $A-\lambda_kE$ enthalten jeweils einen Block aus lauter
Nullen.
Das Produkt all dieser Matrizen  ist daher
\[
(A-\lambda_1E)
(A-\lambda_2E)
\cdots
(A-\lambda_mE)
=
0.
\]
Über dem Körper $\Bbbk'$ gibt es also das Polynom
$m(x)=(x-\lambda_1)(x-\lambda_2)\cdots(x-\lambda_m)$ mit der Eigenschaft
$m(A)=0$.
Dies ist auch das Polynom von kleinstmöglichem Grad, denn für jeden
Eigenwert muss ein entsprechender Linearfaktor in so einem Polynom vorkommen.
Das Polynom $m(x)$ ist daher das Minimalpolynom der Matrix $A$.
Da jeder Faktor in $m(x)$ auch ein Faktor von $\chi_A(x)$ ist,
folgt wieder $\chi_A(A)=0$.
Ausserdem ist über dem Körper $\Bbbk'$ das Polynom $m(x)$ ein Teiler
des charakteristischen Polynoms $\chi_A(x)$.

\subsection{Jordan-Normalform
\label{buch:subsection:jordan-normalform}}
Die Eigenwerte einer Matrix $A$ können als Nullstellen des 
charakteristischen Polynoms gefunden werden.
Da der Körper $\Bbbk$ nicht unbedingt algebraische abgeschlossen ist,
zerfällt das charakteristische Polynom nicht unbedingt in Linearfaktoren,
die Nullstellen sind nicht unbedingt in $\Bbbk$.
Wir können aber immer zu einem grösseren Körper $\Bbbk'$ übergehen,
in dem das charakteristische Polynom in Linearfaktoren zerfällt.
Wir nehmen im Folgenden an, dass 
\[
\chi_A(x)
=
(x-\lambda_1)^{k_1}
\cdot
(x-\lambda_2)^{k_2}
\cdot
\dots
\cdot
(x-\lambda_l)^{k_l}
\]
ist mit $\lambda_i\in\Bbbk'$.

Nach Satz~\ref{buch:eigenwerte:satz:zerlegung-in-eigenraeume} liefern
die verallgemeinerten Eigenräume $V_i=\mathcal{E}_{\lambda_i}(A)$ eine
Zerlegung von $V$ in invariante Eigenräume
\[
V=V_1\oplus V_2\oplus \dots\oplus V_l,
\]
derart, dass $A-\lambda_iE$ auf $V_i$ nilpotent ist.
Wählt man in jedem der Unterräume $V_i$ eine Basis, dann zerfällt die
Matrix $A$ in Blockmatrizen
\begin{equation}
\def\temp#1{\multicolumn{1}{|c}{\raisebox{0pt}[17pt][12pt]{\phantom{x}$#1\mathstrut$}\phantom{x}}}
A'
=\left(
\begin{array}{cccc}
\cline{1-1}
\temp{A_{1}} &\multicolumn{1}{|c}{}&        &           \\
\cline{1-2}
          &\temp{A_{2}}&\multicolumn{1}{|c}{}&           \\
\cline{2-3}
          &           &\temp{\ddots}&\multicolumn{1}{|c}{}\\
\cline{3-4}
          &           &        &\multicolumn{1}{|c|}{\raisebox{0pt}[17pt][12pt]{\phantom{x}$A_{l}$}\phantom{x}}\\
\cline{4-4}
\end{array}
\right)
\label{buch:eigenwerte:eqn:allgjordan}
\end{equation}
wobei, $A_i$ Matrizen mit dem einzigen Eigenwert $\lambda_i$ sind.

Nach Satz~\ref{buch:eigenwerte:satz:allgnilpotent}
kann man in den Unterräume die Basis zusätzlich so wählen, dass 
die entstehenden Blöcke $A_i-\lambda_i E$ spezielle nilpotente Matrizen
aus lauter Null sind, die höchstens unmittelbar über der Diagonalen
Einträge $1$ haben kann.
Dies bedeutet, dass sich immer eine Basis so wählen lässt, dass die
Matrix $A_i$ zerfällt in sogenannte Jordan-Blöcke.

\begin{definition}
Ein $m$-dimensionaler {\em Jordan-Block} ist eine $m\times m$-Matrix
\index{Jordan-Block}%
der Form
\[
J_m(\lambda)
=
\begin{pmatrix}
\lambda &    1    &         &        &         &         \\
        & \lambda &    1    &        &         &         \\
        &         & \lambda &        &         &         \\
        &         &         & \ddots &         &         \\
        &         &         &        & \lambda &     1   \\
        &         &         &        &         & \lambda 
\end{pmatrix}.
\]
Eine {\em Jordan-Matrix} ist eine Blockmatrix Matrix
\[
J
=
\def\temp#1{\multicolumn{1}{|c}{\raisebox{0pt}[17pt][12pt]{\phantom{x}$#1\mathstrut$}\phantom{x}}}
\left(
\begin{array}{cccc}
\cline{1-1}
\temp{J_{m_1}(\lambda)} &\multicolumn{1}{|c}{}&        &           \\
\cline{1-2}
          &\temp{J_{m_2}(\lambda)}&\multicolumn{1}{|c}{}&           \\
\cline{2-3}
          &           &\temp{\ddots}&\multicolumn{1}{|c}{}\\
\cline{3-4}
          &           &        &\multicolumn{1}{|c|}{\raisebox{0pt}[17pt][12pt]{\phantom{x}$J_{m_p}(\lambda)$}\phantom{x}}\\
\cline{4-4}
\end{array}
\right)
\]
mit $m_1+m_2+\dots+m_p=m$.
\index{Jordan-Matrix}%
\end{definition}

Da Jordan-Blöcke obere Dreiecksmatrizen sind, ist
das charakteristische Polynom eines Jordan-Blocks oder einer Jordan-Matrix
besonders einfach zu berechnen.
Es gilt
\[
\chi_{J_m(\lambda)}(x)
=
\det (J_m(\lambda) - xE)
=
(\lambda-x)^m
\]
für einen Jordan-Block $J_m(\lambda)$.
Für eine $m\times m$-Jordan-Matrix $J$ mit Blöcken $J_{m_1}(\lambda)$
bis $J_{m_p}(\lambda)$ ist
\[
\chi_{J(\lambda)}(x)
=
\chi_{J_{m_1}(\lambda)}(x)
\chi_{J_{m_2}(\lambda)}(x)
\cdot
\dots
\cdot
\chi_{J_{m_p}(\lambda)}(x)
=
(\lambda-x)^{m_1}
(\lambda-x)^{m_2}
\cdot\dots\cdot
(\lambda-x)^{m_p}
=
(\lambda-x)^m.
\]

\begin{satz}
\label{buch:eigenwerte:satz:jordannormalform}
Über einem Körper $\Bbbk'\supset\Bbbk$, über dem das charakteristische
Polynom $\chi_A(x)$ in Linearfaktoren zerfällt, lässt sich immer
eine Basis finden derart, dass die Matrix $A$ zu einer Blockmatrix wird,
die aus lauter Jordan-Matrizen besteht.
Die Dimension der Jordan-Matrix zum Eigenwert $\lambda_i$ ist die
Vielfachheit des Eigenwerts im charakteristischen Polynom.
\end{satz}

\begin{proof}[Beweis]
Es ist nur noch die Aussage über die Dimension der Jordan-Blöcke zu
beweisen.
Die Jordan-Matrizen zum Eigenwert $\lambda_i$ werden mit $J_i$
bezeichnet und sollen $m_i\times m_i$-Matrizen sein.
Das charakteristische Polynom jedes Jordan-Blocks ist dann
$\chi_{J_i}(x)=(\lambda_i-x)^{m_i}$.
Das charakteristische Polynom der Blockmatrix mit diesen Jordan-Matrizen
als Blöcken ist das Produkt
\[
\chi_A(x)
=
(\lambda_1-x)^{m_1}
(\lambda_2-x)^{m_2}
\cdots
(\lambda_p-x)^{m_p}
\]
mit $m_1+m_2+\dots+m_p$.
Die Blockgrösse $m_i$ ist also auch die Vielfachheit von $\lambda_i$ im
charakteristischen Polynom $\chi_A(x)$.
\end{proof}



\begin{satz}[Cayley-Hamilton]
\label{buch:normalformen:satz:cayley-hamilton}
Ist $A$ eine $n\times n$-Matrix über dem Körper $\Bbbk$, dann gilt
$\chi_A(A)=0$.
\end{satz}

\begin{proof}[Beweis]
Zunächst gehen wir über zu einem Körper $\Bbbk'\supset\Bbbk$, indem
das charakteristische Polynom $\chi_A(x)$ in Linearfaktoren
$\chi_A(x)
=
(\lambda_1-x)^{m_1}
(\lambda_2-x)^{m_2}
\dots
(\lambda_p-x)^{m_p}$
zerfällt.
Im Vektorraum $\Bbbk'$ kann man eine Basis finden, in der die Matrix
$A$ in Jordan-Matrizen $J_1,\dots,J_p$ zerfällt, wobei $J_i$ eine
$m_i\times m_i$-Matrix ist.
Für den Block mit der Nummer $i$ erhalten wir
$(J_i - \lambda_i E)^{m_i} = 0$.
Setzt man also den Block $J_i$ in das charakteristische Polynom
$\chi_A(x)$ ein, erhält man
\[
\chi_A(J_i)
=
(\lambda_1E - J_1)^{m_1}
\cdot
\ldots
\cdot
\underbrace{
(\lambda_iE - J_i)^{m_i}
}_{\displaystyle=0}
\cdot
\ldots
\cdot
(\lambda_iE - J_p)^{m_p}
=
0.
\]
Jeder einzelne Block $J_i$ wird also zu $0$, wenn man ihn in das
charakteristische Polynome $\chi_A(x)$ einsetzt.
Folglich gilt auch $\chi_A(A)=0$.

Die Rechnung hat zwar im Körper $\Bbbk'$ stattgefunden, aber die Berechnung
$\chi_A(A)$ kann in $\Bbbk$ ausgeführt werden, also ist $\chi_A(A)=0$.
\end{proof}

Aus dem Beweis kann man auch noch eine strengere Bedingung ableiten.
Auf jedem verallgemeinerten Eigenraum $\mathcal{E}_{\lambda_i}(A)$
ist $A_i-\lambda_i$ nilpotent, es gibt also einen minimalen Exponenten
$q_i$ derart, dass $(A_i-\lambda_iE)^{q_i}=0$ ist.
Wählt man eine Basis in jedem verallgemeinerten Eigenraum derart,
dass $A_i$ eine Jordan-Matrix ist, kann man wieder zeigen, dass
für das Polynom
\[
m_A(x)
=
(x-\lambda_1x)^{q_1}
(x-\lambda_2x)^{q_2}
\cdot
\ldots
\cdot
(x-\lambda_px)^{q_p}
\]
gilt $m_A(A)=0$.
$m_A(x)$ ist das {\em Minimalpolynom} der Matrix $A$.
\index{Minimalpolynom einer Matrix}%

\begin{satz}[Minimalpolynom]
Über dem Körper $\Bbbk'\subset\Bbbk$, über dem das charakteristische
Polynom $\chi_A(x)$ in Linearfaktoren zerfällt, ist das Minimalpolynom
von $A$ das Polynom
\[
m(x)
=
(x-\lambda_1)^{q_1}
(x-\lambda_2)^{q_2}
\cdots
\ldots
\cdots
(x-\lambda_p)^{q_p}
\]
wobei $q_i$ der kleinste Index ist, für den die $q_i$-te Potenz
derEinschränkung von $A-\lambda_i E$ auf den verallgemeinerten Eigenraum
$\mathcal{E}_{\lambda_i}(A)$ verschwindet.
Es ist das Polynom geringsten Grades über $\Bbbk'$, welches $m(A)=0$ erfüllt.
\end{satz}


\subsection{Reelle Normalform
\label{buch:subsection:reelle-normalform}}
Wenn eine reelle Matrix $A$ komplexe Eigenwerte hat, ist die Jordansche
Normalform zwar möglich, aber die zugehörigen Basisvektoren werden ebenfalls
komplexe Komponenten haben.
Für eine rein reelle Rechnung ist dies nachteilig, da der Speicheraufwand
dadurch verdoppelt und der Rechenaufwand für Multiplikationen vervierfacht
wird.

Die nicht reellen Eigenwerte von $A$ treten in konjugiert komplexen Paaren
$\lambda_i$ und $\overline{\lambda}_i$ auf.
Wir betrachten im Folgenden nur ein einziges Paar $\lambda=a+ib$ und
$\overline{\lambda}=a-ib$ von konjugiert komplexen Eigenwerten mit
nur je einem einzigen $n\times n$-Jordan-Block $J$ und $\overline{J}$.
Ist $\mathcal{B}=\{b_1,\dots,b_n\}$ die Basis für den Jordan-Block $J$,
dann kann man die Vektoren
$\overline{\mathcal{B}}=\{\overline{b}_1,\dots,\overline{b}_n\}$ als Basis für
$\overline{J}$ verwenden.
Die vereinigte Basis
$\mathcal{C} = \mathcal{B}\cup\overline{\mathcal{B}}
= \{b_1,\dots,b_n,\overline{b}_1,\dots,\overline{b}_n\}$
erzeugen einen $2n$-dimensionalen Vektorraum,
der direkte Summe der beiden von $\mathcal{B}$ und $\overline{\mathcal{B}}$
erzeugen Vektorräume $V=\langle\mathcal{B}\rangle$ und
$\overline{V}=\langle\overline{\mathcal{B}}\rangle$ ist.
Es ist also
\[
U=\langle \mathcal{C}\rangle
=
V\oplus \overline{V}.
\]
Wir bezeichnen die lineare Abbildung mit den Jordan-Blöcken
$J$ und $\overline{J}$ wieder mit $A$.

Auf dem Vektorraum $U$ hat die lineare Abbildung in der Basis
$\mathcal{C}$ die Matrix
\[
A=
\begin{pmatrix}
J&0\\
0&\overline{J}
\end{pmatrix}
=
\begin{pmatrix}
\lambda&   1   &       &      &       &&&&&\\
       &\lambda&   1   &      &       &&&&&\\
       &       &\lambda&\ddots&       &&&&&\\
       &       &       &\ddots&   1   &&&&&\\
       &       &       &      &\lambda&&&&&\\
&&&& &\overline{\lambda}&1&&     & \\
&&&& &&\overline{\lambda}&1&     & \\
&&&& &&&\overline{\lambda} &\dots& \\
&&&& &&&                   &\dots&1\\
&&&& &&&                   &&\overline{\lambda}\\
\end{pmatrix}.
\]

Die Jordan-Normalform bedeutet, dass
\[
\begin{aligned}
Ab_1&=\lambda b_1           &
	A\overline{b}_1 &= \overline{\lambda} \overline{b}_1      \\
Ab_2&=\lambda b_2 + b_1     &
	A\overline{b}_2 &= \overline{\lambda} \overline{b}_2 +\overline{b_1}\\
Ab_3&=\lambda b_3 + b_2     &
	A\overline{b}_3 &= \overline{\lambda} \overline{b}_3 +\overline{b_2}\\
    &\;\vdots               &
	                 &\;\vdots \\
Ab_n&=\lambda b_n + b_{n-1} &
	A\overline{b}_n &= \overline{\lambda} \overline{b}_n +\overline{b_{n-1}}
\end{aligned}
\]
Für die Linearkombinationen
\begin{equation}
\begin{aligned}
c_i &= \frac{b_i+\overline{b}_i}{\sqrt{2}},
&
d_i &= \frac{b_i-\overline{b}_i}{i\sqrt{2}}
\end{aligned}
\label{buch:eigenwerte:eqn:reellenormalformumrechnung}
\end{equation}
folgt dann für $k>1$
\begin{align*}
Ac_k
&=
\frac{Ab_k+A\overline{b}_k}{2}
&
Ad_k
&=
\frac{Ab_k-A\overline{b}_k}{2i}
\\
&=
\frac1{\sqrt{2}}(\lambda b_k + b_{k-1}
+ \overline{\lambda}\overline{b}_k + \overline{b}_{k-1})
&
&=
\frac1{i\sqrt{2}}(\lambda b_k + b_{k-1}
- \overline{\lambda}\overline{b}_k - \overline{b}_{k-1})
\\
&=
\frac1{\sqrt{2}}(\alpha b_k + i\beta b_k + \alpha \overline{b}_k -i\beta \overline{b}_k)
+
c_{k-1}
&
&=
\frac1{i\sqrt{2}}(
\alpha b_k + i\beta b_k - \alpha \overline{b}_k +i\beta \overline{b}_k)
+
d_{k-1}
\\
&=
\alpha
\frac{b_k+\overline{b}_k}{\sqrt{2}}
+
i \beta \frac{b_k-\overline{b}_k}{\sqrt{2}}
+
c_{k-1}
&
&=
\alpha
\frac{b_k-\overline{b}_k}{i\sqrt{2}}
+
i \beta \frac{b_k+\overline{b}_k}{i\sqrt{2}}
+
d_{k-1}
\\
&= \alpha c_k -\beta d_k
+
c_{k-1}
&
&= \alpha d_k + \beta c_k
+
d_{k-1}.
\end{align*}
Für $k=1$ fallen die Terme $c_{k-1}$ und $d_{k-1}$ weg.
In der Basis $\mathcal{D}=\{c_1,d_1,\dots,c_n,d_n\}$ hat die Matrix
also die {\em reelle Normalform}
\begin{equation}
\def\temp#1{\multicolumn{1}{|c}{#1\mathstrut}}
\def\semp#1{\multicolumn{1}{c|}{#1\mathstrut}}
A_{\text{reell}}
=
\left(
\begin{array}{cccccccccccc}
\cline{1-4}
\temp{\alpha}& \beta&\temp{     1}&     0&\temp{}      &      &      &      &      &      &&\\
\temp{-\beta}&\alpha&\temp{     0}&     1&\temp{}      &      &      &      &      &      &&\\
\cline{1-6}
      &      &\temp{\alpha}& \beta&\temp{     1}&     0&\temp{}      &      &      &      &&\\
      &      &\temp{-\beta}&\alpha&\temp{     0}&     1&\temp{}      &      &      &      &&\\
\cline{3-6}
      &      &      &      &\temp{\alpha}& \beta&\temp{}      &      &      &      &&\\
      &      &      &      &\temp{-\beta}&\alpha&\temp{}      &      &      &      &&\\
\cline{5-8}
      &      &      &      &      &      &\temp{\phantom{0}}&\phantom{0}&\temp{      }&      &&\\
      &      &      &      &      &      &\temp{\phantom{0}}&\phantom{0}&\temp{      }&      &&\\
\cline{7-12}
      &      &      &      &      &      &      &      &\temp{\alpha}& \beta&\temp{     1}&\semp{     0}\\
      &      &      &      &      &      &      &      &\temp{-\beta}&\alpha&\temp{     0}&\semp{     1}\\
\cline{9-12}
      &      &      &      &      &      &      &      &      &      &\temp{\alpha}&\semp{ \beta}\\
      &      &      &      &      &      &      &      &      &      &\temp{-\beta}&\semp{\alpha}\\
\cline{11-12}
\end{array}\right).
\label{buch:eigenwerte:eqn:reellenormalform}
\end{equation}

Wir bestimmen noch die Transformationsmatrix, die $A$ in die reelle
Normalform bringt.
Dazu beachten wir, dass die Vektoren $c_k$ und $d_k$ in der Basis
$\mathcal{B}$ nur in den Komponenten $k$ und $n+k$ von $0$ verschiedene
Koordinaten haben, nämlich
\[
c_k
=
\frac1{\sqrt{2}}
\left(
\begin{array}{c}
\vdots\\ 1 \\ \vdots\\\hline \vdots\\ 1\\\vdots
\end{array}\right)
\qquad\text{und}\qquad
d_k
=
\frac1{i\sqrt{2}}
\left(\begin{array}{c}
\vdots\\ 1 \\ \vdots\\\hline\vdots\\-1\\\vdots
\end{array}\right)
=
\frac1{\sqrt{2}}
\left(\begin{array}{c}
\vdots\\-i \\ \vdots\\\hline \vdots\\ i\\\vdots
\end{array}\right)
\]
gemäss \eqref{buch:eigenwerte:eqn:reellenormalformumrechnung}.
Die Umrechnung der Koordinaten von der Basis $\mathcal{B}$ in die Basis
$\mathcal{D}$
wird daher durch die Matrix
\[
S
=
\frac{1}{\sqrt{2}}
\left(\begin{array}{cccccccccc}
1&-i& &  & &  &     &     & &  \\
 &  &1&-i& &  &     &     & &  \\
 &  & &  &1&-i&     &     & &  \\
 &  & &  & &  &\dots&\dots& &  \\
 &  & &  & &  &     &     &1&-i\\
\hline
1& i& &  & &  &     &     & &  \\
 &  &1& i& &  &     &     & &  \\
 &  & &  &1& i&     &     & &  \\
 &  & &  & &  &\dots&\dots& &  \\
 &  & &  & &  &     &     &1& i\\
\end{array}\right)
\]
vermittelt.
Der Nenner $\sqrt{2}$ wurde so gewählt, dass die
Zeilenvektoren der Matrix $S$ als komplexe Vektoren orthonormiert sind,
die Matrix $S$ ist daher unitär und hat die Inverse
\[
S^{-1}
=
S^*
=
\frac{1}{\sqrt{2}}
\left(\begin{array}{ccccc|ccccc}
 1&  &  &     &  & 1&  &  &     &  \\
 i&  &  &     &  &-i&  &  &     &  \\
  & 1&  &     &  &  & 1&  &     &  \\
  & i&  &     &  &  &-i&  &     &  \\
  &  & 1&     &  &  &  & 1&     &  \\
  &  & i&     &  &  &  &-i&     &  \\
  &  &  &\dots&  &  &  &  &\dots&  \\
  &  &  &\dots&  &  &  &  &\dots&  \\
  &  &  &     & 1&  &  &  &     & 1\\
  &  &  &     & i&  &  &  &     &-i\\
\end{array}\right).
\]
Insbesondere folgt jetzt
\[
A
=
S^{-1}A_{\text{reell}}S
=
S^*A_{\text{reell}}S
\qquad\text{und}\qquad
A_{\text{reell}}
=
SAS^{-1}
=
SAS^*.
\]

%\subsection{Obere Hessenberg-Form
%\label{buch:subsection:obere-hessenberg-form}}




%
% spektralradius.tex
%
% (c) 2020 Prof Dr Andreas Müller, Hochschule Rapperswi
%
\section{Analytische Funktionen einer Matrix
\label{buch:section:analytische-funktionen-einer-matrix}}
\rhead{Analytische Funktionen einer Matrix}
Eine zentrale Motivation in der Entwicklung der Eigenwerttheorie
war das Bestreben, Potenzen $A^k$ auch für grosse $k$ effizient
zu berechnen.
Mit der Jordan-Normalform ist dies auch gelungen, wenigstens über
einem Körper, in dem das charakteristische Polynom in Linearfaktoren
zerfällt.
Die Berechnung von Potenzen war aber nur der erste Schritt, das Ziel
in diesem Abschnitt ist, $f(A)$ für eine genügend grosse Klasse von
Funktionen $f$ berechnen zu können.

%
% Polynom-Funktionen von Matrizen
%
\subsection{Polynom-Funktionen
\label{buch:subsection:polynom-funktionen}}
Die einfachsten Funktionen $f(x)$, für die der Wert $f(A)$ 
auf offensichtliche Weise berechnet werden kann, sind Polynome.
Die Jordan-Normalform kann dabei helfen, die Potenzen von $A$
zu berechnen.

In diesem Abschnitt ist $B\in M_n(\Bbbk)$ und $\Bbbk'\supset\Bbbk$ ein
Körper, über dem das charakteristische Polynome $\chi_A(X)$ in
Linearfaktoren
\[
\chi_A(X)
=
(\lambda_1-X)^{m_1}(\lambda_2-X)^{m_2}\cdots(\lambda_p-X)^{m_p}
\]
zerfällt.

Für jedes beliebige Polynom $p(X)\in\Bbbk[X]$ der Form
\[
p(X) = a_nX^n + a_{n-1}X^{n-1} + \dots + a_1x + a_0
\]
kann man auch
\[
p(A) = a_nA^n + a_{n-1}A^{n-1} + \dots + a_1A + a_0I
\]
berechnen.
In der Jordan-Normalform können die Potenzen $A^k$ leicht zusammengstellt
werden, sobald man die Potenzen von Jordan-Blöcken berechnet hat.

\begin{satz}
Die $k$-te Potenz von $J_n(\lambda)$ ist die Matrix mit
\begin{equation}
J_n(\lambda)^k
=
\renewcommand{\arraystretch}{1.4}
\begin{pmatrix}
\lambda^k
	& \binom{k}{1}\lambda^{k-1}
		& \binom{k}{2} \lambda^{k-2}
			& \binom{k}{3} \lambda^{k-3}
				& \dots
					&\binom{k}{n-1}\lambda^{k-n+1}
\\
0
	& \lambda^k
		& \binom{k}{1}\lambda^{k-1}
			& \binom{k}{2} \lambda^{k-2}
				& \dots
					&\binom{k}{n-2}\lambda^{k-n+2}
\\
0
	& 0
		& \lambda^k
			& \binom{k}{1}\lambda^{k-1}
				& \dots
					&\binom{k}{n-3}\lambda^{k-n+3}
\\
0
	& 0
		& 0
			& \lambda^k
				& \dots
					&\binom{k}{n-4}\lambda^{k-n+4}
\\
\vdots  &\vdots &\vdots &\vdots &\ddots & \vdots
\\
0	& 0	& 0	& 0	& \dots	& \lambda^k
\end{pmatrix}
\label{buch:eigenwerte:eqn:Jnkpotenz}
\end{equation}
mit den Matrixelementen
\[
(J_n(\lambda)^k)_{i\!j}
=
\binom{k}{j-i}\lambda^{k-j+i}.
\]
Die Binomialkoeffizienten verschwinden für $j<i$ und $j>i+k$.
\end{satz}

\begin{proof}[Beweis]
Die Herkunft der Binomialkoeffizienten wird klar, wenn man
\[
J_n(\lambda) = \lambda I + N_n
\]
schreibt, wobei $N_n$ die Matrix \eqref{buch:eigenwerte:eqn:nnilpotent} ist.
Die Potenzen von $N_n$ haben die Matrix-Elemente
\[
(N_n^k)_{i\!j}
=
\delta_{i,j-k}
=
\begin{cases}
1&\qquad j-i=k\\
0&\qquad\text{sonst,}
\end{cases}
\]
sie haben also Einsen genau dort, wo in der
\label{buch:eigenwerte:eqn:Jnkpotenz} die Potenz $\lambda^{k}$ steht.
Die $k$-te Potenz von $J_n(\lambda)$ kann dann mit dem binomischen
Satz berechnet werden:
\[
J_n(\lambda)^k
=
\sum_{l=0}^k \binom{k}{l}\lambda^l N_n^{k-l},
\]
dies ist genau die Form \eqref{buch:eigenwerte:eqn:Jnkpotenz}.
\end{proof}

Wir haben bereits gesehen, dass $\chi_A(A)=0$.
Ersetzt man also das
Polynom $p(X)$ durch $p(X)+\chi_A(X)$, dann ändert sich am Wert 
\[
(p+\chi_A)(A)
=
p(A) + \chi_A(A)
=
p(A)
\]
nichts.
Man kann also nicht erwarten, dass verschiedene Polynome 
$p(X)$ zu verschiedenen Matrizen $p(A)$ führen.
Doch genau welche Unterschiede zwischen Polynomen wirken sich
auf den Wert $p(A)$ aus?

\begin{satz}
Für zwei Polynome $p(X)$ und $q(X)$ ist genau dann $p(A)=q(A)$, wenn
das Minimalpolynom von $A$ die Differenz $p-q$ teilt.
\end{satz}

\begin{proof}[Beweis]
Wenn $p(A)=q(A)$, dann ist $h(X)=p(X)-q(X)$ ein Polynom mit $h(A)=0$,
daher muss $h(X)$ vom Minimalpolynom geteilt werden.
Ist andererseits $p(X)-q(X)=m(X)t(X)$, dann ist
$p(A)-q(A)=m(A)t(A)=0\cdot t(A) = 0$, also $p(A)=q(A)$.
\end{proof}

Über einem Körper $\Bbbk'\supset\Bbbk$, über dem das charakteristische
Polynom in Linearfaktoren zerfällt, kann man das Minimalpolynom aus
der Jordanschen Normalform ableiten.
Es ist
\[
m(X)
=
(\lambda_1-X)^{q_1}
(\lambda_2-X)^{q_2}
\cdots
(\lambda_p-X)^{q_p},
\]
wobei $q_i$ die Dimension des grössten Jordan-Blocks ist, der in der
Jordan-Normalform vorkommt.
Zwei Polynome $p_1(X)$ und $p_2(X)$ ergeben genau dann den gleichen Wert
auf $A$,
wenn die Differenz $p_1(X)-p_2(X)$ genau die Nullstellen
$\lambda_1,\dots,\lambda_p$ mit Vielfachheiten $q_1,\dots,q_p$ hat.

\begin{beispiel}
Wir betrachten die Matrix
\[
A
=
\begin{pmatrix}
   1&  9& -4\\
  -1&  3&  0\\
  -2&  0&  3
\end{pmatrix}
\]
mit dem charakteristischen Polynom
\[
\chi_A(X)
=
-X^3+7X^2-16 X+12
=
-(X-3)(X-2)^2.
\]
Daraus kann man bereits ablesen, dass das Minimalpolynom $m(X)$ von $A$ 
entweder $(X-2)(X-3)$ oder $(X-2)^2(X-3)$ ist.
Es genügt also nachzuprüfen, ob $p(A)=0$ für das Polynom
$p(X)=(X-2)(X-3) = X^2-5X+6$ ist.
Tatsächlich sind die Potenzen von $A$:
\begin{equation}
A^2=
\begin{pmatrix}
  0&  36& -16 \\
 -4&   0&   4 \\
 -8& -18&  17 
\end{pmatrix}
,\qquad
A^3=
\begin{pmatrix}
 -4& 108& -48\\
-12& -36&  28\\
-24&-126&  83
\end{pmatrix}
\label{buch:eigenwerte:eqn:A2A3}
\end{equation}
und daraus kann man jetzt $p(A)$ berechnen:
\begin{equation}
p(A)
=
\begin{pmatrix}
  0&  36& -16 \\
 -4&   0&   4 \\
 -8& -18&  17 
\end{pmatrix}
-5
\begin{pmatrix}
   1&  9& -4\\
  -1&  3&  0\\
  -2&  0&  3
\end{pmatrix}
+
6
\begin{pmatrix}
1&0&0\\
0&1&0\\
0&0&1
\end{pmatrix}
=
\begin{pmatrix}
   1& -9&  4\\
   1& -9&  4\\
   2&-18&  8
\end{pmatrix}
=
\begin{pmatrix}1\\1\\2\end{pmatrix}
\begin{pmatrix}1&-9&4\end{pmatrix}
\ne 0
\label{buch:eigenwerte:eqn:nichtminimalpolynom}
\end{equation}
Daher kann $p(X)$ nicht das Minimalpolynom $A$
sein, daher muss $(X-2)^2(X-3)$ das Minimalpolynom sein.

Das Quadrat des Polynoms $p(X)$ ist $p(X)^2 = (X-2)^2(X-3)^2$, es hat
das Minimalpolynom als Teiler, also muss $p(A)^2=0$ sein.
Die Gleichung \eqref{buch:eigenwerte:eqn:nichtminimalpolynom} ermöglicht,
das Quaddrat $p(A)^2$ leichter zu berechnen:
\[
p(A)^2
=
\begin{pmatrix}1\\1\\2\end{pmatrix}
\underbrace{
\begin{pmatrix}1&-9&4\end{pmatrix}
\begin{pmatrix}1\\1\\2\end{pmatrix}
}_{\displaystyle = 0}
\begin{pmatrix}1&-9&4\end{pmatrix}
=
0
,
\]
wie zu erwarten war.

Wenn sich zwei Polynome nur um das charakteristische Polynom unterscheiden,
dann haben sie den gleichen Wert auf $A$.
Das Polynom $p_1(X)=X^3$ unterschiedet sich vom Polynom
$p_2(X)=7X^2-16X+12=\chi_A(X)+X^3=p_1(X)+\chi_A(X)$ 
um das charakteristische Polynom, welches wir bereits als das Minimalpolynom
von $A$ erkannt haben.
Die dritte Potenz $A^3=p_1(A)$ von $A$ muss sich daher auch als $p_2(A)$
berechnen lassen:
\[
7
\begin{pmatrix}
  0&  36& -16 \\
 -4&   0&   4 \\
 -8& -18&  17 
\end{pmatrix}
-16
\begin{pmatrix}
   1&  9& -4\\
  -1&  3&  0\\
  -2&  0&  3
\end{pmatrix}
+12
\begin{pmatrix}
1&0&0\\
0&1&0\\
0&0&1
\end{pmatrix}
=
\begin{pmatrix}
 -4& 108&  -48\\
-12& -36&   28\\
-24&-126&   83
\end{pmatrix}
=
A^3,
\]
wie in \eqref{buch:eigenwerte:eqn:A2A3} vorsorglich berechnet worden ist.
\end{beispiel}

\begin{satz}
Wenn $A$ diagonalisierbar ist über einem geeignet erweiterten Körper $\Bbbk'$,
dann haben zwei Polynome $p(X)$ und $q(X)$ in $\Bbbk[X]$ genau dann
den gleichen Wert auf $A$, also $p(A)=q(A)$, wenn $p(\lambda) = q(\lambda)$
für alle Eigenwerte $\lambda$ von $A$.
\end{satz}

Über dem Körper der komplexen Zahlen ist die Bedingung, dass die Differenz
$d(X)=p_1(X)-p_2(X)$ vom Minimalpolynom geteilt werden muss, gleichbedeutend
damit, dass $p_1(X)$ und $p_2(X)$ die gleichen Nullstellen mit den gleichen
Vielfachheiten haben.
Eine andere Art, dies auszudrücken, ist, dass $p_1(x)$ und $p_2(X)$
die gleichen Werte und Ableitungen bis zur Ordnung $q_i-1$ haben, wenn
$q_i$ der Exponente von $\lambda_I-X$ im Minimalpolynom von $A$ ist.

Das Beispiel illustriert auch noch ein weiteres wichtiges Prinzip.
Schreiben wir das Minimalpolynom von $A$ in der Form
\[
m(X)
=
X^k + a_{k-1}X^{k-1} + \dots + a_1X + a_0,
\]
dann kann man wegen $m(A)=0$ die Potenzen $A^i$ mit $i\ge k$ mit der
Rekursionsformel
\[
A^i
=
A^{i-k}A^k
=
A^{i-k}(-a_{k-1}A^{k-1}+ \dots + a_1 A + a_0I)
\]
in einer Linearkombination kleinerer Potenzen reduzieren.
Jedes Polynom vom Grad $\ge k$ kann also reduziert werden in
ein Polynom vom Grad $<k$ mit dem gleichen Wert auf $A$.

\begin{satz}
\label{buch:eigenwerte:satz:reduktion}
Sei $A$ eine Matrix über $\Bbbk$ mit Minimalpolynom $m(X)$.
Zu jedem $p(X)\in\Bbbk[X]$ gibt es ein Polynom $q(X)\in\Bbbk[X]$
vom Grad $\deg q<\deg m$ mit $p(A)=q(A)$.
\end{satz}

%
% Approximationen für Funktionswerte f(A)
%
\subsection{Approximation von $f(A)$
\label{buch:subsection:approximation}}
Die Quadratwurzelfunktion $x\mapsto\sqrt{x}$ lässt sich nicht durch ein
\index{Quadratwurzelfunktion}%
Polynom darstellen, es gibt also keine direkte Möglichkeit, $\sqrt{A}$
für eine beliebige Matrix zu definieren.
Wir können versuchen, die Funktion durch ein Polynom zu approximieren.
Damit dies geht, müssen wir folgende zwei Fragen klären:
\begin{enumerate}
\item
Wie misst man, ob ein Polynom eine Funktion gut approximiert?
\item
Was bedeutet es genau, dass zwei Matrizen ``nahe beeinander'' sind?
\item
In welchem Sinne müssen Polynome ``nahe'' beeinander sein, damit
auch die Werte auf $A$ nahe beeinander sind.
\end{enumerate}

Wir wissen bereits, dass nur die Werte und gewisse Ableitungen des
Polynoms $p(X)$ in den Eigenwerten einen Einfluss auf $p(A)$ haben.
Es genügt also, Approximationspolynome zu verwenden, welche in der Nähe
der Eigenwerte ``gut genug'' approximieren.
Solche Polynome gibt es dank dem Satz von Stone-Weierstrass immer:

\begin{satz}[Stone-Weierstrass]
Ist $I\subset\mathbb{R}$ kompakt, dann lässt sich jede stetige Funktion
$f(x)$
durch eine Folge $p_n(x)$ beliebig genau approximieren.
\end{satz}

Die Hoffnung ist, $f(A)$ als Grenzwert der Approximationen $p_n(A)$
zu definieren.
Dazu muss sichergestellt sein, dass verschiedene Approximationen
der Funktion $f$ den gleichen Grenzwert $\lim_{n\to\infty}p_n(A)$ 
ergeben.
Im Folgenden soll genauer untersucht werden, ob sich von der
Konvergenz einer Folge $p_n(x)$ auf die Konvergenz von $p_n(A)$
geschlossen werden kann.

Wir haben schon gezeigt, dass es dabei auf die höheren Potenzen gar nicht
ankommt, nach Satz~\ref{buch:eigenwerte:satz:reduktion} kann man ein
approximierendes Polynom immer durch ein Polynom von kleinerem Grad
als das Minimalpolynom ersetzen.

\begin{definition}
\index{Norm}%
Die {\em Norm} einer Matrix $M$ ist
\[
\|M\|
=
\max\{|Mx|\,|\, x\in\mathbb R^n\wedge |x|=1\}.
\]
Für einen Vektor $x\in\mathbb R^n$ gilt $|Mx| \le \|M\|\cdot |x|$.
\end{definition}

\begin{beispiel}
Die Matrix
\[
M=\begin{pmatrix}
0&2\\
\frac13&0
\end{pmatrix}
\]
hat Norm
\[
\|M\|
=
\max_{|x|=1} |Mx| 
=
\max_{t\in\mathbb R} \sqrt{2^2\cos^2 t +\frac1{3^2}\sin^2t} = 2.
\]
Da aber
\[
M^2 = \begin{pmatrix}
\frac{2}{3}&0\\
0&\frac{2}{3}
\end{pmatrix}
\qquad\Rightarrow\qquad \|M^2\|=\frac23
\]
ist, wird eine Iteration mit Ableitungsmatrix $M$ trotzdem
konvergieren, weil der Fehler nach jedem zweiten Schritt um den
Faktor $\frac23$ kleiner geworden ist.
\end{beispiel}

\begin{beispiel}
Wir berechnen die Norm eines $2\times2$-Jordan-Blocks.
Ein $2$-dimensionaler Einheitsvektor kann als
\[
v\colon
t\mapsto v(t)=
\begin{pmatrix}\cos t\\\sin t\end{pmatrix}
\]
parametrisiert werden.
Für die Zahl $\lambda=a+bi$ bildet der
Jordanblock $J_2(\lambda)$ den Vektor $v(t)$ auf den Vektor
\[
J_2(\lambda)v(t)
=
\begin{pmatrix}
\lambda&1\\
0&\lambda
\end{pmatrix}
\begin{pmatrix}\cos t\\\sin t\end{pmatrix}
=
\begin{pmatrix}
\lambda\cos t + \sin t\\
\lambda\sin t
\end{pmatrix}
\]
ab
mit der Länge
\begin{align*}
|J_2(\lambda)v(t)|^2
&=
|\lambda\cos t + \sin t|^2 + |\lambda\sin t|^2
=
(\Re\lambda \cos t + \sin t)^2
+
(\Im\lambda \cos t)^2
+
|\lambda|^2 \sin^2t
\\
&=
a^2\cos^2 t
+
2a\cos t\sin t + \sin^2 t + b^2\cos^2t + (a^2+b^2) \sin^2 t
\\
&=
(a^2+b^2)(\cos^2t + \sin^2t) + \sin^2t + 2a\cos t\sin t
=
|\lambda|^2+2a\cos t\sin t + \sin^2 t
\\
&=
|\lambda|^2 + a\sin 2t + \frac12(1-\cos 2t).
\end{align*}
Um den maximalen Wert zu finden, leiten wir nach $t$ ab und finden
\begin{align*}
\frac{d}{dt}
|J_2(\lambda)v(t)|^2
&=
2a\cos 2t
+
\sin 2t
=
0.
\end{align*}
Dividieren wir durch $\cos t$, ergibt sich die Gleichung
\[
\tan 2t = -2a
\quad\Rightarrow\quad
2t
=
\arctan(-2a)
\quad\Rightarrow\quad
\left\{
\renewcommand{\arraystretch}{2.1}
\setlength\arraycolsep{1pt}
\begin{array}{ccc}
\cos 2t &=& \displaystyle\frac{1}{\sqrt{1+4a^2}}\phantom{.}\\
\sin 2t &=& \displaystyle\frac{-2a}{\sqrt{1+4a^2}}.
\end{array}
\right.
\]
Setzt man dies in die ursprüngliche Formel für die Länge des
Bildvektors ein, erhält man
\begin{align*}
\|J_2\|^2
=
|J_2(\lambda)v(t)|^2
&=
|\lambda|^2 + \frac{-2a}{\sqrt{1+4a^2}} + \frac12\biggl(1-\frac{1}{\sqrt{1+4a^2}}\biggr)
\\
&=
|\lambda|^2
+ \frac12
-\frac{1+4a}{2\sqrt{1+4a^2}}.
\end{align*}
Für $a\to\infty$ wächst dies asymptotisch wie $a^2-1$.
\end{beispiel}

%
% Potenzreihen für Funktionen $f(z)$
%
\subsection{Potenzreihen
\label{buch:subsection:potenzreihen}}
Funktionen, die eine konvergente Potenzreihenentwicklung
\begin{equation}
f(z)
=
\sum_{k=0}^\infty a_kz^k
\label{buch:eigenwerte:eqn:potenzreihe}
\end{equation}
\index{Potenzreihe}
haben, wie
zum Beispiel $e^x$, $\sin x$ oder $\cos x$, haben eine in der Folge
der Partialsummen
\[
p_n(z) = \sum_{k=0}^n a_kz^k
\]
eine Approximation mit Polynomen.
Nach dem {\em Wurzelkriterium} ist die
Reihe~\eqref{buch:eigenwerte:eqn:potenzreihe}
konvergent, wenn 
\[
\limsup_{k\to\infty} \sqrt[n]{|a_kz^k|} < 1
\]
ist.
\index{Wurzelkriterium}%
Dies führt auf die Formel $1/\varrho = \limsup_{k\to\infty}|a_k|^{\frac1k}$ 
für den Konvergenzradius der Potenzreihe.

Setzt man die Matrix $M\in M_r(\Bbbk)$ in die Potenzreihe ein,
folgt, dass
\[
\limsup_{n\to\infty} \sqrt[n]{\|a_kM^n\|}
\le
\limsup_{n\to\infty} \sqrt[n]{|a_n|}
\cdot
\limsup_{n\to\infty} \|M^k\|^{\frac1k}
=
\frac{1}{\varrho}
\limsup_{n\to\infty} \|M^k\|^{\frac1k}
\]
sein muss.
Dies führt uns auf die Grösse
\begin{equation}
\pi(M)
=
\limsup_{n\to\infty} \|M^n\|^\frac1n,
\label{buch:eqn:gelfand-grenzwert}
\end{equation}
\index{pi(M)@$\pi(M)$}%
die
darüber entscheidet, ob die Potenzreihe $f(A)$ konvergiert.

Die Zahl $\pi(M)$ erlaubt zunächst einmal zu bestimmen, wie
sich die Potenzen $M^k$ entwickeln.
Für Zahlen ist diese Frage sehr einfach zu entscheiden: wenn $q>1$ ist,
dann geht $q^n\to\infty$, wenn $|q|<1$ ist, dann geht $q^n\to 0$.
Für Matrizen ist die Frage etwas schieriger.
Man kann sich vorstellen, dass eine Streckung in einer Richtung
von einer Stauchung in eine andere Richtung kompensiert wird, wenn
dazwischen eine Drehung stattfindet.
Es ist also durchaus möglich, dass $\|M\|>1$ ist, die
Iterierten $M^k$ aber trotzdem gegen $0$ gehen.

Ist $\pi(M) > 1$, dann gibt es Anfangsvektoren $v$ für die Iteration,
für die $M^kv$ über alle Grenzen wächst.
Ist $\pi(M) < 1$, dann wird jeder Anfangsvektor $v$ zu einer Iterationsfolge
$M^kv$ führen, die gegen $0$ konvergiert.
Die Kennzahl $\pi(M)$ erlaubt also zu entscheiden, ob die
Iteration konvergent ist.
\index{Konvergenzbedingung}%

\begin{definition}
\label{buch:eigenwerte:def:gelfand-radius}
Der Grenzwert
\[
\pi(M)
=
\limsup_{n\to\infty} \|M^k\|^{\frac1k}
\]
heisst {\em Gelfand-Radius} der Matrix $M$.
\index{Gelfand-Radius}%
\end{definition}


%
% Gelfand-Radius und Eigenwerte
%
\subsection{Gelfand-Radius und Eigenwerte
\label{buch:subsection:potenzreihen}}
Die Berechnung des Gelfand-Radius als Grenzwert ist sehr unhandlich.
Viel einfacher ist der Begriff des Spektralradius.
\index{Spektralradius}%

\begin{definition}
\label{buch:definition:spektralradius}
Der {\em Spektralradius} $\varrho(M)$ der Matrix $M$ ist der Betrag des
betragsgrössten
\index{Spektralradius}%
Eigenwertes.
\index{rho(M)@$\varrho(M)$}%
\end{definition}

Wir wollen in diesem Abschnitt zeigen, dass der Gelfand-Radius mit
dem Spektralradius übereinstimmt.
Dies liefert uns ein vergleichsweise einfach auszuwertendes Konvergenzkriterium.
\index{Konvergenzkriterium}%

\subsubsection{Spezialfall: Diagonalisierbare Matrizen}
Ist eine Matrix $A$ diagonalisierbar, dann kann Sie durch eine Wahl
einer geeigneten Basis in die Diagonalform
\index{diagonalisierbar}%
\index{Diagonalform}%
\[
A'
=
\begin{pmatrix}
\lambda_1&        0&\dots &0\\
0        &\lambda_2&\dots &0\\
\vdots   &         &\ddots&\vdots\\
0        &        0&\dots &\lambda_n
\end{pmatrix}
\]
gebracht werden, wobei die Eigenwerte $\lambda_i$  möglicherweise auch
komplex sein können.
\index{komplex}%
Die Bezeichnungen sollen so gewählt sein, dass $\lambda_1$ der
betragsgrösste Eigenwert ist, dass also
\[
|\lambda_1| \ge |\lambda_2| \ge \dots \ge |\lambda_n|.
\]
Wir nehmen für die folgende, einführende Diskussion ausserdem an, dass
sogar $|\lambda_1|>|\lambda_2|$ gilt.

Unter den genannten Voraussetzungen kann man jetzt den Gelfand-Radius
von $A$ berechnen.
Dazu muss man $|A^nv|$ für einen beliebigen Vektor $v$ und für
beliebiges $n$ berechnen.
Der Vektor $v$ lässt sich in der Eigenbasis von $A$ zerlegen, also
als Summe
\index{Eigenbasis}%
\[
v = v_1+v_2+\dots+v_n
\]
schreiben, wobei $v_i$ Eigenvektoren zum Eigenwert $\lambda_i$ sind oder
Nullvektoren.
Die Anwendung von $A^k$ ergibt dann
\[
A^k v
=
A^k v_1 + A^k v_2 + \dots + A^k v_n
=
\lambda_1^k v_1 + \lambda_2^k v_2 + \dots + \lambda_n^k v_n.
\]
Für den Grenzwert braucht man die Norm von $A^kv$, also
\begin{align}
|A^kv|
&= |\lambda_1^k v_1 + \lambda_2^k v_2 + \dots + \lambda_3 v_3|
\notag
\\
\Rightarrow\qquad
\frac{|A^kv|}{\lambda_1^k}
&=
\biggl|
v_1 +
\biggl(\frac{\lambda_2}{\lambda_1}\biggr)^k v_2
+
\dots
+
\biggl(\frac{\lambda_n}{\lambda_1}\biggr)^k v_n
\biggr|.
\label{buch:spektralradius:eqn:eigenwerte}
\end{align}
Da alle Quotienten $|\lambda_i/\lambda_1|<1$ sind für $i\ge 2$,
konvergieren alle Terme auf der rechten Seite von
\eqref{buch:spektralradius:eqn:eigenwerte}
ausser dem ersten gegen $0$.
Folglich ist
\[
\lim_{k\to\infty} \frac{|A^kv|}{|\lambda_1|^k}
=
|v_1|
\qquad\Rightarrow\qquad
\lim_{k\to\infty} \frac{|A^kv|^\frac1k}{|\lambda_1|}
=
\lim_{k\to\infty}|v_1|^{\frac1k}
=
1.
\]
Dies gilt für alle Vektoren $v$, für die $v_1\ne 0$ ist.
Der maximale Wert dafür wird erreicht, wenn man für 
$v$ einen Eigenvektor der Länge $1$ zum Eigenwert $\lambda_1$ einsetzt,
dann ist $v=v_1$.
Es folgt dann
\[
\pi(A)
=
\lim_{k\to\infty} \| A^k\|^\frac1k
=
\lim_{k\to\infty} |A^kv|^\frac1k
=
|\lambda_1|
=
\varrho(A).
\]
Damit ist gezeigt, dass im Spezialfall einer diagonalisierbaren Matrix der
Gelfand-Radius tatsächlich der Betrag des betragsgrössten Eigenwertes ist.
\index{Gelfand-Radius}%

\subsubsection{Blockmatrizen}
Wir betrachten jetzt eine $(n+m)\times(n+m)$-Blockmatrix der Form
\begin{equation}
A = \begin{pmatrix} B & 0 \\ 0 & C\end{pmatrix}
\label{buch:spektralradius:eqn:blockmatrix}
\end{equation}
mit einer $n\times n$-Matrix $B$ und einer $m\times m$-Matrix $C$.
Ihre Potenzen haben ebenfalls Blockform:
\[
A^k = \begin{pmatrix} B^k & 0 \\ 0 & C^k\end{pmatrix}.
\]
Ein Vektor $v$ kann in die zwei Summanden $v_1$ bestehend aus den
ersten $n$ Komponenten und $v_2$ bestehend aus den letzten $m$ 
Komponenten zerlegen.
Dann ist
\[
A^kv = B^kv_1 + C^kv_2.
\qquad\Rightarrow\qquad
|A^kv|
\le
|B^kv_1| + |C^kv_2|
\le 
\pi(B)^k |v_1| + \pi(C)^k |v_2|.
\]
Insbesondere haben wir das folgende Lemma gezeigt:

\begin{lemma}
\label{buch:spektralradius:lemma:diagonalbloecke}
Eine diagonale Blockmatrix $A$ \eqref{buch:spektralradius:eqn:blockmatrix}
Blöcken $B$ und $C$  hat Gelfand-Radius
\[
\pi(A) = \max ( \pi(B), \pi(C) )
\]
\end{lemma}

Selbstverständlich lässt sich das Lemma auf Blockmatrizen mit beliebig
vielen diagonalen Blöcken verallgemeinern.
\index{Blockmatrix}%

Für Diagonalmatrizen der genannten Art sind aber auch die 
Eigenwerte leicht zu bestimmen.
\index{Diagonalmatrix}%
Hat $B$ die Eigenwerte $\lambda_i^{(B)}$ mit $1\le i\le n$ und $C$ die
Eigenwerte $\lambda_j^{(C)}$ mit $1\le j\le m$, dann ist das charakteristische
Polynom der Blockmatrix $A$ natürlich
\index{charakteristisches Polynom}%
\index{Polynom!charakteristisch}%
\[
\chi_A(\lambda) = \chi_B(\lambda)\chi_C(\lambda).
\]
Es folgt, dass die Eigenwerte von $A$ die Vereinigung der Eigenwerte
von $B$ und $C$ sind.
Daher gilt auch für den Spektralradius die Formel
\[
\varrho(A) = \max(\varrho(B) , \varrho(C)).
\]

\subsubsection{Jordan-Blöcke}
\index{Jordan-Block}%
Nicht jede Matrix ist diagonalisierbar, die bekanntesten Beispiele sind
die Jordan-Blöcke
\begin{equation}
J_n(\lambda)
=
\begin{pmatrix}
\lambda &      1&       &       &       &       \\
        &\lambda&      1&       &       &       \\[-5pt]
        &       &\lambda&\ddots &       &       \\[-5pt]
        &       &       &\ddots &      1&       \\
        &       &       &       &\lambda&      1\\
        &       &       &       &       &\lambda
\end{pmatrix},
\label{buch:spektralradius:eqn:jordan}
\end{equation}
wobei $\lambda\in\mathbb C$ eine beliebige komplexe Zahl ist.
Es ist klar, dass $J_n(\lambda)$ nur den $n$-fachen Eigenwert
$\lambda$ hat und dass der erste Standardbasisvektor ein
Eigenvektor zu diesem Eigenwert ist.

In Abschnitt~\ref{buch:subsection:jordan-normalform}
haben wir gesehen, dass jede Matrix durch die Wahl
\index{lineare!Algebra}%
einer geeigneten Basis als Blockmatrix der Form
\[
A
=
\begin{pmatrix}
J_{n_1}(\lambda_1) &        0         & \dots & 0 \\
       0         & J_{n_2}(\lambda_2) & \dots & 0 \\[-4pt]
\vdots           &\vdots            &\ddots &\vdots \\
       0         &        0         & \dots &J_{n_l}(\lambda_l)
\end{pmatrix}
\]
geschrieben werden kann.
Die früheren Beobachtungen über den Spektralradius und den
Gelfand-Radius von Blockmatrizen führen uns dazu, dass
nur gezeigt werden muss, dass nur die Gleichheit des Gelfand-Radius
und des Spektral-Radius von Jordan-Blöcken gezeigt werden muss.

\subsubsection{Potenzen von Jordan-Blöcken}
\begin{satz}
\label{buch:spektralradius:satz:grenzwert}
Sei $A$ eine $n\times n$-Matrix mit Spektralradius $\varrho(A)$.
Dann ist $\varrho(A)<1$ genau dann, wenn
\[
\lim_{k\to\infty} A^k = 0.
\]
Ist andererseits $\varrho(A) > 1$, dann ist
\[
\lim_{k\to\infty} \|A^k\|=\infty.
\]
\end{satz}

\begin{proof}[Beweis]
Wie bereits angedeutet reicht es, diese Aussagen für einen einzelnen
Jordan-Block mit Eigenwert $\lambda$ zu beweisen.
Die $k$-te Potenz von $J_n(\lambda)$ ist
\[
J_n(\lambda)^k
=
\renewcommand\arraystretch{1.35}
\begin{pmatrix}
\lambda^k    & \binom{k}{1} \lambda^{k-1} & \binom{k}{2}\lambda^{k-2}&\dots&
\binom{k}{n-1}\lambda^{k-n+1}\\
      0      &\lambda^k & \binom{k}{1} \lambda^{k-1} & \dots &\binom{k}{n-2}\lambda^{k-n+2}\\
      0     &      0    & \lambda^k & \dots &\binom{k}{n-k+3}\lambda^{k-n+3}\\
\vdots      & \vdots    &               &\ddots & \vdots\\
     0      &      0    &      0        &\dots  &\lambda^k
\end{pmatrix}.
\]
Falls $|\lambda| < 1$ ist, gehen alle Potenzen von $\lambda$ exponentiell
schnell gegen $0$, während die Binomialkoeffizienten nur polynomiell
schnell anwachsen. 
\index{Binomialkoeffizient}%
In diesem Fall folgt also $J_n(\lambda)\to 0$.

Falls $|\lambda| >1$ divergieren bereits die Elemente auf der Diagonalen,
also ist $\|J_n(\lambda)^k\|\to\infty$ mit welcher Norm auch immer man
man die Matrix misst.
\end{proof}

Aus dem Beweis kann man noch mehr ablesen.
Für $\varrho(A)< 1$ ist die Norm $ \|A^k\| \le M \varrho(A)^k$ für eine
geeignete Konstante $M$,
für $\varrho(A) > 1$ gibt es eine Konstante $m$ mit
$\|A^k\| \ge m\varrho(A)^k$.

\subsubsection{Der Satz von Gelfand}
Der Satz von Gelfand ergibt sich jetzt als direkte Folge aus dem
Satz~\ref{buch:spektralradius:satz:grenzwert}.

\begin{satz}[Gelfand]
\index{Satz von Gelfand}%
\index{Gelfand!Satz von}%
\label{buch:satz:gelfand}
Für jede komplexe $n\times n$-Matrix $A$ gilt
\[
\pi(A)
=
\lim_{k\to\infty}\|A^k\|^\frac1k
=
\varrho(A).
\]
\end{satz}

\begin{proof}[Beweis]
Der Satz~\ref{buch:spektralradius:satz:grenzwert} zeigt, dass der
Spektralradius ein scharfes Kriterium dafür ist, ob $\|A^k\|$ 
gegen 0 oder $\infty$ konvergiert.
Andererseits ändert ein Faktor $t$ in der Matrix $A$ den Spektralradius
ebenfalls um den gleichen Faktor, also $\varrho(tA)=t\varrho(A)$.
Natürlich gilt auch
\[
\pi(tA)
=
\lim_{k\to\infty} \|t^kA^k\|^\frac1k
=
\lim_{k\to\infty} t\|A^k\|^\frac1k
=
t\lim_{k\to\infty} \|A^k\|^\frac1k
=
t\pi(A).
\]

Wir betrachten jetzt die Matrix
\[
A(\varepsilon) = \frac{A}{\varrho(A) + \varepsilon}.
\]
Der Spektralradius von $A(\varepsilon)$ ist
\[
\varrho(A(\varepsilon)) = \frac{\varrho(A)}{\varrho(A)+\varepsilon},
\]
er ist also $>1$ für negatives $\varepsilon$ und $<1$ für positives
$\varepsilon$.
Aus dem Satz~\ref{buch:spektralradius:satz:grenzwert} liest man daher ab,
dass $\|A(\varepsilon)^k\|$ genau dann gegen $0$ konvergiert, wenn
$\varepsilon > 0$ ist und divergiert genau dann, wenn $\varepsilon< 0$ ist.

Aus der Bemerkung nach dem Beweis von
Satz~\ref{buch:spektralradius:satz:grenzwert} schliesst man daher, dass 
es im Fall $\varepsilon > 0$ eine Konstante $M$ gibt mit
\begin{align*}
\|A(\varepsilon) ^k\|\le M\varrho(A(\varepsilon))^k
\quad&\Rightarrow\quad
\|A(\varepsilon) ^k\|^\frac1k\le M^\frac1k\varrho(A(\varepsilon))
\\
&\Rightarrow\quad
\pi(A(\varepsilon)) \le  \varrho(A(\varepsilon))
\underbrace{\lim_{k\to\infty} M^\frac1k}_{\displaystyle=1}
=
\varrho(A(\varepsilon))
=
\varrho(A)+\varepsilon.
\end{align*}
Dies gilt für beliebige $\varepsilon >0$, es folgt daher
$\pi(A) \le \varrho(A)$.

Andererseits gibt es für $\varepsilon <0$ eine Konstante $m$ mit
\begin{align*}
\|A(\varepsilon) ^k\|\ge m\varrho(A(\varepsilon))^k
\quad&\Rightarrow\quad
\|A(\varepsilon) ^k\|^\frac1k\ge m^\frac1k\varrho(A(\varepsilon))
\\
&\Rightarrow\quad
\pi(A(\varepsilon)) \ge  \varrho(A(\varepsilon))
\underbrace{\lim_{k\to\infty} m^\frac1k}_{\displaystyle=1}
=
\varrho(A(\varepsilon))
=
\varrho(A)+\varepsilon.
\end{align*}
Dies gilt für beliebige $\varepsilon> 0$, es folgt daher
$\pi(A) \ge \varrho(A)$.
Zusammen mit $\pi(A) \le \varrho(A)$ folgt $\pi(A)=\varrho(A)$.
\end{proof}


%
% numerisch.tex
%
% (c) 2020 Prof Dr Andreas Müller, Hochschule Rapeprswil
%
\section{Numerische Verfahren zur Eigenwertbestimmung
\label{buch:section:numerische-verfahren-eigenwerte}}
% Jacobi-Algorithmus
% Potenzverfahren
% Francis-Algorithmus


%
% spektraltheorie.tex
%
% (c) 2020 Prof Dr Andreas Müller, Hochschule Rapperswil
% 
\section{Spektraltheorie
\label{buch:section:spektraltheorie}}
% Matrix-Exponentialfunktion
% Wurzel einer Matrix
% Beliebige Funktion f(A) für normale Matrizen



\section*{Übungsaufgaben}
\rhead{Übungsaufgaben}
\aufgabetoplevel{chapters/40-eigenwerte/uebungsaufgaben}
\begin{uebungsaufgaben}
\uebungsaufgabe{4001}
\uebungsaufgabe{4002}
\uebungsaufgabe{4003}
\end{uebungsaufgaben}

