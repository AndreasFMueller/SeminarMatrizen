%
% grundlagen.tex -- Grundlagen
%
% (c) 2021 Prof Dr Andreas Müller, OST Ostschweizer Fachhochschule
%
\section{Grundlagen
\label{buch:section:grundlagen}}
\rhead{Grundlagen}
Die Potenzen $A^k$ sind besonders einfach zu berechnen, wenn die Matrix
Diagonalform hat, wenn also $A=\operatorname{diag}(\lambda_1,\dots,\lambda_n)$
ist.
In diesem Fall ist $Ae_k=\lambda_k e_k$ für jeden Standardbasisvektor $e_k$.
Statt sich auf Diagonalmatrizen zu beschränken könnten man also auch
Vektoren $v$ suchen, für die gilt $Av=\lambda v$, die also von $A$ nur
gestreckt werden.
Gelingt es, eine Basis aus solchen sogenanten {\em Eigenvektoren} zu finden,
dann kann man die Matrix $A$ durch Basiswechsel in diese Form bringen.

%
%
%
\subsection{Kern und Bild
\label{buch:subsection:kern-und-bild}}

%
% Begriff des Eigenwertes und Eigenvektors
%
\subsection{Eigenwerte und Eigenvektoren
\label{buch:subsection:eigenwerte-und-eigenvektoren}}
In diesem Abschnitt betrachten wir Vektorräume $V=\Bbbk^n$ über einem
beliebigen Körper $\Bbbk$ und quadratische Matrizen
$A\in M_n(\Bbbk)$.
In den meisten Anwendungen wird $\Bbbk=\mathbb{R}$ sein.
Da aber in $\mathbb{R}$ nicht alle algebraischen Gleichungen lösbar sind,
ist es manchmal notwendig, den Vektorraum zu erweitern um zum Beispiel
Eigenschaften der Matrix $A$ abzuleiten.

\begin{definition}
Ein Vektor $v\in V$ heisst {\em Eigenvektor} von $A$ zum Eigenwert
$\lambda\in\Bbbk$, wenn $v\ne 0$ und $Av=\lambda v$ gilt.
\end{definition}

Die Bedingung $v\ne 0$ dient dazu, pathologische Situationen auszuschliessen.
Für den Nullvektor gilt $A0=\lambda 0$ für jeden beliebigen Wert von
$\lambda\in\Bbbk$.
Würde man $v=0$ zulassen, wäre jede Zahl in $\Bbbk$ ein Eigenwert,
ein Eigenwert von $A$ wäre nichts besonderes.
Ausserdem wäre $0$ ein Eigenvektor zu jedem beliebigen Eigenwert.

Eigenvektoren sind nicht eindeutig bestimmt, jedes von $0$ verschiedene
Vielfache von $v$ ist ebenfalls ein Eigenvektor.
Zu einem Eigenwert kann man also einen Eigenvektor jeweils mit 
geeigneten Eigenschaften finden, zum Beispiel kann man für $\Bbbk = \mathbb{R}$
Eigenvektoren auf Länge $1$ normieren.
Im Folgenden werden wir oft die abkürzend linear unabhängige Eigenvektoren
einfach als ``verschiedene'' Eigenvektoren bezeichnen.

Wenn $v$ ein Eigenvektor von $A$ zum Eigenwert $\lambda$ ist, dann kann
man ihn mit zusätzlichen Vektoren $v_2,\dots,v_n$ zu einer Basis
$\mathcal{B}=\{v,v_2,\dots,v_n\}$
von $V$ ergänzen.
Die Vektoren $v_k$ mit $k=2,\dots,n$ werden von $A$ natürlich auch
in den Vektorraum $V$ abgebildet, können also als Linearkombinationen
\[
Av = a_{1k}v + a_{2k}v_2 + a_{3k}v_3 + \dots a_{nk}v_n
\]
dargestellt werden.
In der Basis $\mathcal{B}$ bekommt die Matrix $A$ daher die Form
\[
A'
=
\begin{pmatrix}
\lambda&a_{12}&a_{13}&\dots &a_{1n}\\
    0  &a_{22}&a_{23}&\dots &a_{2n}\\
    0  &a_{32}&a_{33}&\dots &a_{3n}\\
\vdots &\vdots&\vdots&\ddots&\vdots\\
    0  &a_{n2}&a_{n3}&\dots &a_{nn}
\end{pmatrix}.
\]
Bereits ein einzelner Eigenwert und ein zugehöriger Eigenvektor
ermöglichen uns also, die Matrix in eine etwas einfachere Form
zu bringen.

\begin{definition}
Für $\lambda\in\Bbbk$ heisst
\[
E_\lambda
=
\{ v\;|\; Av=\lambda v\}
\]
der {\em Eigenraum} zum Eigenwert $\lambda$.
\index{Eigenraum}%
\end{definition}

Der Eigenraum $E_\lambda$ ist ein Unterraum von $V$, denn wenn
$u,v\in E_\lambda$, dann ist
\[
A(su+tv)
=
sAu+tAv
=
s\lambda u + t\lambda v
=
\lambda(su+tv),
\]
also ist auch $su+tv\in E_\lambda$.
Der Fall $E_\lambda = \{0\}=0$ bedeutet natürlich, dass $\lambda$ gar kein
Eigenwert ist.

\begin{satz}
Wenn $\dim E_\lambda=n$, dann ist $A=\lambda E$.
\end{satz}

\begin{proof}[Beweis]
Da $V$ ein $n$-dimensionaler Vektoraum ist, ist $E_\lambda=V$.
Jeder Vektor $v\in V$ erfüllt also die Bedingung $Av=\lambda v$,
oder $A=\lambda E$.
\end{proof}

Wenn man die Eigenräume von $A$ kennt, dann kann man auch die Eigenräume
von $A+\mu E$ berechnen.
Ein Vektor $v\in E_\lambda$ erfüllt
\[
Av=\lambda v
\qquad\Rightarrow\qquad
(A+\mu)v = \lambda v + \mu v
=
(\lambda+\mu)v,
\]
somit ist $v$ ein Eigenvektor von $A+\mu E$ zum Eigenwert $\lambda+\mu$.
Insbesondere können wir statt die Eigenvektoren von $A$ zum Eigenwert $\lambda$
zu studieren, auch die Eigenvektoren zum Eigenwert $0$ von $A-\lambda E$
untersuchen.

\begin{satz}
\label{buch:eigenwerte:satz:jordanblock}
Wenn $\dim E_\lambda=1$ ist, dann gibt es eine Basis von $V$ derart, dass
$A$ in dieser Matrix die Form
\begin{equation}
A'
=
\begin{pmatrix}
 \lambda &    1    &         &        &         &         \\
         & \lambda &    1    &        &         &         \\
         &         & \lambda &        &         &         \\
         &         &         & \ddots &         &         \\
         &         &         &        & \lambda &    1    \\
         &         &         &        &         & \lambda
\end{pmatrix}
\label{buch:eigenwerte:eqn:jordanblock}
\end{equation}
hat.
\end{satz}

\begin{proof}[Beweis]
Entsprechend der Bemerkung vor dem Satz können wir uns auf die Betrachtung
der Matrix $B=A-\lambda E$ konzentrieren, deren Eigenraum zum Eigenwert $0$
$1$-dimensional ist.
Es gibt also einen Vektor $v_1\ne 0$ mit $Bv_1=0$.
Der Vektor $v_1$ spannt den Eigenraum auf: $E_0 = \langle v_1\rangle$.

Wir konstruieren jetzt rekursiv eine Folge $v_2,\dots,v_n$ von Vektoren
mit folgenden Eigenschaften.
Zunächst soll $v_k=Bv_{k+1}$ für $k=1,\dots,n-1$ sein.
Ausserdem soll $v_{k+1}$ in jedem Schritt linear unabhängig von den
Vektoren $v_1,\dots,v_{k-1}$ gewählt werden.
Wenn diese Konstruktion gelingt, dann ist $\mathcal{B}=\{v_1,\dots,v_n\}$
eine Basis von $V$ und die Matrix von $B$ in dieser Basis ist
$A'$ wie in \eqref{buch:eigenwerte:eqn:jordanblock}.
\end{proof}

\subsection{Das charakteristische Polynom
\label{buch:subsection:das-charakteristische-polynom}}
Ein Eigenvektor von $A$ erfüllt $Av=\lambda v$ oder gleichbedeutend
$(A-\lambda E)v=0$, er ist also eine nichttriviale Lösung des homogenen
Gleichungssystems mit Koeffizientenmatrix $A-\lambda E$. 
Ein Eigenwert ist also ein Skalar derart, dass $A-\lambda E$
singulär ist.
Ob eine Matrix singulär ist, kann mit der Determinante festgestellt
werden.
Die Eigenwerte einer Matrix $A$ sind daher die Nullstellen
von $\det(A-\lambda E)$.

\begin{definition}
Das {\em charakteristische Polynom}
\[
\chi_A(x)
=
\det (A-x E)
=
\left|
\begin{matrix}
a_{11}-x & a_{12}   & \dots  & a_{1n} \\
a_{21}   & a_{22}-x & \dots  & a_{2n} \\
\vdots   &\vdots    &\ddots  & \vdots \\
a_{n1}   & a_{n2}   &\dots   & a_{nn}-x
\end{matrix}
\right|.
\]
der Matrix $A$ ist ein Polynom vom Grad $n$ mit Koeffizienten in $\Bbbk$.
\end{definition}

Findet man eine Nullstelle $\lambda\in\Bbbk$ von $\chi_A(x)$,
dann ist die Matrix $A-\lambda E\in M_n(\Bbbk)$ und mit dem Gauss-Algorithmus
kann man auch mindestens einen Vektor $v\in \Bbbk^n$ finden,
der $Av=\lambda v$ erfüllt.
Eine Matrix der Form  wie in Satz~\ref{buch:eigenwerte:satz:jordanblock}
hat
\[
\chi_A(x)
=
\left|
\begin{matrix}
\lambda-x &     1     &           &      &         &         \\
          & \lambda-x &     1     &      &         &         \\
          &           & \lambda-x &      &         &         \\
          &           &           &\ddots&         &         \\
          &           &           &      &\lambda-x&     1   \\
          &           &           &      &         &\lambda-x
\end{matrix}
\right|
=
(\lambda-x)^n
=
(-1)^n (x-\lambda)^n
\]
als charakteristisches Polynom, welches $\lambda$ als einzige
Nullstelle hat.
Der Eigenraum der Matrix ist aber nur eindimensional, man kann also
im Allgemeinen für jede Nullstelle des charakteristischen Polynoms
nicht mehr als einen Eigenvektor (d.~h.~einen eindimensionalen Eigenraum)
erwarten.

Wenn das charakteristische Polynom von $A$ keine Nullstellen in $\Bbbk$ hat,
dann kann es auch keine Eigenvektoren in $Bbbk^n$ geben.
Gäbe es nämlich einen solchen Vektor, dann müsste eine der Komponenten
des Vektors von $0$ verschieden sein, wir nehmen an, dass es die Komponente
in Zeile $k$ ist.
Die Komponente $v_k$ kann man auf zwei Arten berechnen, einmal als
die $k$-Komponenten von $Av$ und einmal als $k$-Komponente von $\lambda v$:
\[
a_{k1}v_1+\dots+a_{kn}v_n = \lambda v_k.
\]
Da $v_k\ne 0$ kann man nach $\lambda$ auflösen und erhält
\[
\lambda = \frac{a_{k1}v_1+\dots + a_{kn}v_n}{v_k}.
\]
Alle Terme auf der rechten Seite sind in $\Bbbk$ und werden nur mit
Körperoperationen in $\Bbbk$ verknüpft, also muss auch $\lambda\in\Bbbk$
sein, im Widerspruch zur Annahme.

Durch hinzufügen von geeigneten Elementen können wir immer zu einem 
Körper $\Bbbk'$ übergehen, in dem das charakteristische Polynom
in Linearfaktoren zerfällt.
In diesem Körper kann man jetzt das homogene lineare Gleichungssystem
mit Koeffizientenmatrix $A-\lambda E$ lösen und damit mindestens 
einen Eigenvektor $v$ für jeden Eigenwert finden.
Die Komponenten von $v$ liegen in $\Bbbk'$, und mindestens eine davon kann
nicht in $\Bbbk$ liegen.
Das bedeutet aber nicht, dass man diese Vektoren nicht für theoretische
Überlegungen über von $\Bbbk'$ unabhängige Eigenschaften der Matrix $A$ machen.
Das folgende Beispiel soll diese Idee illustrieren.

\begin{beispiel}
Wir arbeiten in diesem Beispiel über dem Körper $\Bbbk=\mathbb{Q}$.
Die Matrix
\[
A=\begin{pmatrix}
-4&7\\
-2&4
\end{pmatrix}
\in
M_2(\mathbb{Q})
\]
hat das charakteristische Polynom
\[
\chi_A(x)
=
\left|
\begin{matrix}
-4-x&7\\-2&4-x
\end{matrix}
\right|
=
(-4-x)(4-x)-7\cdot(-2)
=
-16+x^2+14
=
x^2-2.
\]
Die Nullstellen sind $\pm\sqrt{2}$ und damit nicht in $\mathbb{Q}$.
Wir gehen daher über zum Körper $\mathbb{Q}(\sqrt{2})$, in dem 
sich zwei Nullstellen $\lambda=\pm\sqrt{2}$ finden lassen.
Zu jedem Eigenwert lässt sich auch ein Eigenvektor
$v_{\pm\sqrt{2}}\in \mathbb{Q}(\sqrt{2})^2$, und unter Verwendung dieser
Basis ist bekommt die Matrix $A'=TAT^{-1}$ Diagonalform.
Die Transformationsmatrix $T$ enthält Matrixelemente aus
$\mathbb{Q}(\sqrt{2})$, die nicht in $\mathbb{Q}$ liegen.
Die Matrix $A$ lässt sich also über dem Körper $\mathbb{Q}(\sqrt{2})$
diagonalisieren, nicht aber über dem Körper $\mathbb{Q}$.

Da $A'$ Diagonalform hat mit $\pm\sqrt{2}$ auf der Diagonalen, folgt
$A^{\prime 2} = 2E$, die Matrix $A'$ erfüllt also die Gleichung
\begin{equation}
A^{\prime 2}-E= \chi_{A}(A) = 0.
\label{buch:grundlagen:eqn:cayley-hamilton-beispiel}
\end{equation}
Dies is ein Spezialfall des Satzes von Cayley-Hamilton~\ref{XXX}
welcher besagt, dass jede Matrix $A$ eine Nullstelle ihres 
charakteristischen Polynoms ist: $\chi_A(A)=0$.
Da in Gleichung~\ref{buch:grundlagen:eqn:cayley-hamilton-beispiel}
wurde zwar in $\mathbb{Q}(\sqrt{2})$ hergeleitet, aber in ihr kommen
keine Koeffizienten aus $\mathbb{Q}(\sqrt{2})$ vor, die man nicht auch
in $\mathbb{Q}$ berechnen könnte.
Sie gilt daher ganz allgemein.
\end{beispiel}

\begin{beispiel}
Die Matrix
\[
A=\begin{pmatrix}
32&-41\\
24&-32
\end{pmatrix}
\in
M_2(\mathbb{R})
\]
über dem Körper $\Bbbk = \mathbb{R}$
hat das charakteristische Polynom
\[
\det(A-xE)
=
\left|
\begin{matrix}
32-x&-41  \\
25  &-32-x
\end{matrix}
\right|
=
(32-x)(-32-x)-25\cdot(-41)
=
x^2-32^2 + 1025
=
x^2+1.
\]
Die charakteristische Gleichung $\chi_A(x)=0$ hat in $\mathbb{R}$
keine Lösungen, daher gehen wir zum Körper $\Bbbk'=\mathbb{C}$ über,
in dem dank dem Fundamentalsatz der Algebra alle Nullstellen zu finden
sind, sie sind $\pm i$.
In $C$ lassen sich dann auch Eigenvektoren finden, man muss dazu die
folgenden linearen Gleichungssyteme lösen:
\begin{align*}
\begin{tabular}{|>{$}c<{$}>{$}c<{$}|}
32-i&-41\\
25  &-32-i
\end{tabular}
&
\rightarrow
\begin{tabular}{|>{$}c<{$}>{$}c<{$}|}
1 & t\\
0 &  0 
\end{tabular}
&
\begin{tabular}{|>{$}c<{$}>{$}c<{$}|}
32+i&-41\\
25  &-32+i
\end{tabular}
&
\rightarrow
\begin{tabular}{|>{$}c<{$}>{$}c<{$}|}
1 & \overline{t}\\
0 &  0 
\end{tabular},
\intertext{wobei wir $t=-41/(32-i) =-41(32+i)/1025= -1.28 -0.04i = (64-1)/50$
abgekürzt haben.
Die zugehörigen Eigenvektoren sind}
v_i&=\begin{pmatrix}t\\i\end{pmatrix}
&
v_{-i}&=\begin{pmatrix}\overline{t}\\i\end{pmatrix}
\end{align*}
Mit den Vektoren $v_i$ und $v_{-i}$ als Basis kann die Matrix $A$ als
komplexe Matrix, also mit komplexem $T$ in die komplexe Diagonalmatrix 
$A'=\operatorname{diag}(i,-i)$ transformiert werden.
Wieder kann man sofort ablesen, dass $A^{\prime2}+E=0$, und wieder kann
man schliessen, dass für die relle Matrix $A$ ebenfalls $\chi_A(A)=0$
gelten muss.
\end{beispiel}




