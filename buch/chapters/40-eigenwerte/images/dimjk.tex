%
% dimjk.tex -- dimensionen von K^l und J^l
%
% (c) 2021 Prof Dr Andreas Müller, OST Ostschweizer Fachhochschule
%
\documentclass[tikz]{standalone}
\usepackage{amsmath}
\usepackage{times}
\usepackage{txfonts}
\usepackage{pgfplots}
\usepackage{csvsimple}
\usetikzlibrary{arrows,intersections,math}
\begin{document}
\def\skala{1.2}
\begin{tikzpicture}[>=latex,thick,scale=\skala]

\definecolor{darkgreen}{rgb}{0,0.6,0}

\def\pfad{
        ({0*\sx},{6-6}) --
        ({1*\sx},{6-4.5}) --
        ({2*\sx},{6-3.5}) --
        ({3*\sx},{6-2.9}) --
        ({4*\sx},{6-2.6}) --
        ({5*\sx},{6-2.4}) --
        ({8*\sx},{6-2.4})
}
\def\sx{1.2}

\fill[color=orange!20] \pfad -- ({6*\sx},6) -- (0,6) -- cycle;
\fill[color=darkgreen!20] \pfad -- ({6*\sx},0) -- cycle;

\fill[color=orange!40] ({5*\sx},6) rectangle ({8*\sx},{6-2.4});
\fill[color=darkgreen!40] ({5*\sx},0) rectangle ({8*\sx},{6-2.4});

\draw[color=darkgreen,line width=2pt] ({3*\sx},{6-6}) -- ({3*\sx},{6-2.9});
\node[color=darkgreen] at ({3*\sx},{6-4.45}) [rotate=90,above] {$\dim\mathcal{K}^k(A)$};
\draw[color=orange,line width=2pt] ({3*\sx},{6-0}) -- ({3*\sx},{6-2.9});
\node[color=orange] at ({3*\sx},{6-1.45}) [rotate=90,above] {$\dim\mathcal{J}^k(A)$};

\node[color=orange] at ({6.5*\sx},{6-1.2}) {bijektiv};
\node[color=darkgreen] at ({6.5*\sx},{6-4.2}) {konstant};

\fill ({0*\sx},{6-6}) circle[radius=0.08];
\fill ({1*\sx},{6-4.5}) circle[radius=0.08];
\fill ({2*\sx},{6-3.5}) circle[radius=0.08];
\fill ({3*\sx},{6-2.9}) circle[radius=0.08];
\fill ({4*\sx},{6-2.6}) circle[radius=0.08];
\fill ({5*\sx},{6-2.4}) circle[radius=0.08];
\fill ({6*\sx},{6-2.4}) circle[radius=0.08];
\fill ({7*\sx},{6-2.4}) circle[radius=0.08];
\fill ({8*\sx},{6-2.4}) circle[radius=0.08];

\draw \pfad;

\draw[->] (-0.5,0) -- ({8*\sx+0.5},0) coordinate[label={$k$}];
\draw[->] (-0.5,6) -- ({8*\sx+0.5},6);

\foreach \x in {0,...,8}{
        \draw ({\x*\sx},-0.05) -- ({\x*\sx},0.05);
}
\foreach \x in {0,...,3}{
        \node at ({\x*\sx},-0.05) [below] {$\x$};
}
\node at ({4*\sx},-0.05) [below] {$\dots\mathstrut$};
\node at ({5*\sx},-0.05) [below] {$l$};
\node at ({6*\sx},-0.05) [below] {$l+1$};
\node at ({7*\sx},-0.05) [below] {$l+2$};
\node at ({8*\sx},-0.05) [below] {$l+3$};

\node[color=orange] at ({1.2*\sx},5.6)
	{$\mathcal{J}^k(A)\supset\mathcal{J}^{k+1}(A)$};
\node[color=darkgreen] at ({1.2*\sx},0.4)
	{$\mathcal{K}^k(A)\subset\mathcal{K}^{k+1}(A)$};

\end{tikzpicture}
\end{document}

