Finden Sie eine Basis von $\mathbb{Q}^4$ derart, dass die Matrix $A$
\[
A
=
\begin{pmatrix}
-13&  5& -29& 29\\
-27& 11& -51& 51\\
 -3&  1&  -2&  5\\
 -6&  2& -10& 13
\end{pmatrix}
\]
Jordansche Normalform hat.

\begin{loesung}
Zunächst muss man die Eigenwerte finden.
Dazu kann man das charakteristische Polynom berechnen, man findet nach
einiger Rechnung oder mit Hilfe einer Software für symbolische Rechnung:
\[
\chi_A(\lambda)
=
x^4-9x^3+30x^2-44x+24
=
(x-3)^3(x-2),
\]
Eigenwerte sind also $\lambda=3$ und $\lambda=2$.

Der Eigenwert $\lambda=2$ ist ein einfacher Eigenwert, der zugehörige
Eigenraum ist daher eindimensional.
Ein Eigenvektor kann mit Hilfe des linearen Gleichungssystems
\begin{align*}
\begin{tabular}{|>{$}c<{$}>{$}c<{$}>{$}c<{$}>{$}c<{$}|}
\hline
-13-\lambda& 5        &-29        &29        \\
-27        &11-\lambda&-51        &51        \\
 -3        & 1        & -2-\lambda& 5        \\
 -6        & 2        &-10        &13-\lambda\\
\hline
\end{tabular}
&\rightarrow
\begin{tabular}{|>{$}c<{$}>{$}c<{$}>{$}c<{$}>{$}c<{$}|}
\hline
  -16&   5& -29&  29\\
  -27&   8& -51&  51\\
   -3&   1&  -5&   5\\
   -6&   2& -10&  10\\
\hline
\end{tabular}
\to
\begin{tabular}{|>{$}c<{$}>{$}c<{$}>{$}c<{$}>{$}c<{$}|}
\hline
1&0&0& 0\\
0&1&0& 0\\
0&0&1&-1\\
0&0&0& 0\\
\hline
\end{tabular}
\end{align*}
gefunden werden.
Daraus liest man den Eigenvektor
\[
b_1
=
\begin{pmatrix} 0\\0\\1\\1\end{pmatrix},
\qquad
Ab_1 = 
\begin{pmatrix}
-13&  5& -29& 29\\
-27& 11& -51& 51\\
 -3&  1&  -2&  5\\
 -6&  2& -10& 13
\end{pmatrix}
\begin{pmatrix} 0\\0\\1\\1\end{pmatrix}
=
\begin{pmatrix}
0\\0\\3\\3
\end{pmatrix}
=
3b_1
\]
ab.
Diesen  Vektor können wir auch finden, indem wir $\mathcal{J}(A-2I)$
bestimmen.
Die vierte Potenz von $A-2I$ ist
\begin{equation}
(A-2I)^4
=
\begin{pmatrix}
   0&  0&  0&  0\\
   0&  0&  0&  0\\
   0&  0&  2& -1\\
   0&  0&  2& -1
\end{pmatrix},
\label{4003:potenz}
\end{equation}
der zugehörige Bildraum ist wieder aufgespannt von $b_1$.

Aus \eqref{4003:potenz} kann man aber auch eine Basis
\[
b_2
=
\begin{pmatrix}1\\0\\0\\0\end{pmatrix}
,\qquad
b_3
=
\begin{pmatrix}0\\1\\0\\0\end{pmatrix}
,\qquad
b_4
=
\begin{pmatrix}0\\0\\1\\2\end{pmatrix}
\]
für den Kern $\mathcal{K}(A-2I)$ ablesen.
Da $\lambda=2$ der einzige andere Eigenwert ist, muss $\mathcal{K}(A-2I)
= \mathcal{J}(A-3I)$ sein.
Dies lässt sich überprüfen, indem wir die vierte Potenz von $A-2I$
berechnen, sie ist
\[
(A-2I)^4
=
\begin{pmatrix}
    79&  -26&  152& -152\\
   162&  -53&  312& -312\\
    12&   -4&   23&  -23\\
    24&   -8&   46&  -46\\
\end{pmatrix}.
\]
Die Spaltenvektoren lassen sich alle durch die Vektoren $b_2$, $b_3$
und $b_4$ ausdrücken, also ist $\mathcal{J}(A-2I)=\langle b_2,b_3,b_4\rangle$.

Indem die Vektoren $b_i$ als Spalten in eine Matrix $T$ schreibt, kann man
jetzt berechnen, wie die Matrix der linearen Abbildung in dieser neuen
Basis aussieht, es ist
\[
A'=T^{-1}AT
\left(
\begin{array}{r|rrr}
    3&   0&   0&   0\\
\hline
    0& -13&   5&  29\\
    0& -27&  11&  51\\
    0&  -3&   1&   8
\end{array}
\right),
\]
wir haben also tatsächlich die versprochene Blockstruktur.

Der $3\times 3$-Block
\[
A_1
=
\begin{pmatrix}
 -13&   5&  29\\
 -27&  11&  51\\
  -3&   1&   8
\end{pmatrix}
\]
in der rechten unteren Ecke hat den dreifachen Eigenwert $2$, 
und die Potenzen von $A_1-2I$ sind
\[
A_1-2I
\begin{pmatrix}
  -15 &  5&  29\\
  -27 &  9&  51\\
   -3 &  1&   6
\end{pmatrix}
,\qquad
(A_1-2I)^2
=
\begin{pmatrix}
    3 & -1 & -6\\
    9 & -3 &-18\\
    0 &  0 &  0\\
\end{pmatrix}
,\qquad
(A_1-2I)^3=0.
\]
Für die Jordan-Normalform brauchen wir einen von $0$ verschiedenen
Vektor im Kern von $(A_1-2I)^2$, zum Beispiel den Vektor mit den
Komponenten $1,3,1$.
Man beachte aber, dass diese Komponenten jetzt in der neuen Basis
$b_2,\dots,b_4$ zu verstehen sind, d.~h.~der Vektor, den wir suchen, ist
\[
c_3
=
b_1+ 3b_2+b_3
=
\begin{pmatrix}1\\3\\1\\2\end{pmatrix}.
\]
Jetzt berechnen wir die Bilder von $c_3$ unter $A-2I$:
\[
c_2
=
\begin{pmatrix}
29\\51Ò\\6\\12
\end{pmatrix}
,\qquad
c_1
=
\begin{pmatrix}
-6\\-18\\0\\0
\end{pmatrix}.
\]
Die Basis $b_1,c_1,c_2,c_3$ ist also eine Basis, in der die Matrix $A$
Jordansche Normalform annimmt.

Die Umrechnung der Matrix $A$ in die Basis $\{b_1,c_1,c_2,c_3\}$ kann 
mit der Matrix
\[
T_1
=
\begin{pmatrix}
    0&  -6&  29&   1\\
    0& -18&  51&   3\\
    1&   0&   6&   1\\
    1&   0&  12&   2\\
\end{pmatrix},
\qquad
T_1^{-1}
=
\frac{1}{216}
\begin{pmatrix}
     0&    0&  432& -216\\
    33&  -23&  -36&   36\\
    18&   -6&    0&    0\\
  -108&   36& -216&  216
\end{pmatrix}
\]
erfolgen und ergibt die Jordansche Normalform
\[
A'
=
\begin{pmatrix}
3&0&0&0\\
0&2&1&0\\
0&0&2&1\\
0&0&0&2
\end{pmatrix}
\]
wie erwartet.
\end{loesung}


