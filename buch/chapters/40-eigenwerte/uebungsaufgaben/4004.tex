Berechnen Sie $\sin At$ für die Matrix
\[
A=\begin{pmatrix}
\omega& 1 \\
 0 &\omega
\end{pmatrix}.
\]
Kontrollieren Sie Ihr Resultat, indem Sie den Fall $\omega = 0$ gesondert
ausrechnen.
\begin{hinweis}
Schreiben Sie $A=\omega I + N$ mit einer nilpotenten Matrix.
\end{hinweis}

\begin{loesung}
Man muss $At$ in die Potenzreihe
\[
\sin z = z - \frac{z^3}{3!} + \frac{z^5}{5!} - \frac{z^7}{7!} + \dots
\]
für die Sinus-Funktion einsetzen.
Mit der Schreibweise $A=\omega I + N$, wobei $N^2=0$ können die Potenzen etwas
leichter berechnet werden:
\begin{align*}
A^0 &= I
\\
A^1 &= \omega I + N
\\
A^2 &= \omega^2 I + 2\omega N
\\
A^3 &= \omega^3 I + 3\omega^2 N
\\
A^4 &= \omega^4 I + 4\omega^3 N
\\
&\phantom{a}\vdots
\\
A^k &= \omega^k I + k\omega^{k-1} N
\end{align*}
Damit kann man jetzt $\sin At$ berechnen:
\begin{align}
\sin At
&=
At - \frac{A^3t^3}{3!}  + \frac{A^5t^5}{5!} - \frac{A^7t^7}{7!}
\dots
\notag
\\
&=
\biggl(
\omega t  - \frac{\omega^3t^3}{3!} + \frac{\omega^5t^5}{5!} - \frac{\omega^7t^7}{7!}
+\dots
\biggr)I
+
\biggl(
t -\frac{3\omega^2t^3}{3!} + \frac{5\omega^4t^5}{5!} - \frac{7\omega^6t^7}{7!}+\dots
\biggr)N
\notag
\\
&=
I\sin\omega t
+tN\biggl(1-\frac{\omega^2t^2}{2!} +\frac{\omega^4t^4}{4!}
- \frac{\omega^6t^6}{6!}
+\dots\biggr)
\notag
\\
&=I\sin\omega t + tN\cos\omega t.
\label{4004:resultat}
\end{align}
Im Fall $\omega=0$ ist $A=N$ und $A^2=0$, so dass
\[
\sin At = tN,
\]
dies stimmt mit \eqref{4004:resultat} für $\omega=0$ überein, da
$\cos\omega t = \cos 0=1$ in diesem Fall.
\end{loesung}
