Man findet eine Basis, in der die Matrix
\[
A=\begin{pmatrix}
 -5&  2&  6&   0\\
-11& 12& -3& -15\\
 -7&  0&  9&   4\\
  0&  5& -7&  -8
\end{pmatrix}
\]
die relle Normalform bekommt.

\begin{loesung}
Das charakteristische Polynom der Matrix ist 
\[
\chi_{A}(\lambda)
=
\lambda^4-8\lambda^3+42\lambda^2-104\lambda+169
=
(\lambda^2-4\lambda+13)^2.
\]
Es hat die doppelten Nullstellen
\[
\lambda_\pm
=
2\pm \sqrt{4-13}
=
2\pm \sqrt{-9}
=
2\pm 3i.
\]
Zur Bestimmung der Basis muss man jetzt zunächst den Kern von 
$A_+=A-\lambda_+I$ bestimmen, zum Beispiel mit Hilfe des Gauss-Algorithmus,
man findet
\[
b_1
=
\begin{pmatrix}
1+i\\
2+2i\\
i\\
1
\end{pmatrix}
\]
Als nächstes braucht man einen Vektor $b_1\in \ker A_+^2$, der 
$b_1$ auf $b_1+\lambda_+b_2$ abbildet.
Durch Lösen des Gleichungssystems $Ab_2-\lambda b_2=b_1$ findet man
\[
b_2
=
\begin{pmatrix}
2-i\\3\\2\\0
\end{pmatrix}
\qquad\text{und damit weiter}\qquad
\overline{b}_1
=
\begin{pmatrix}
1-i\\
2-2i\\
-i\\
1
\end{pmatrix},\quad
\overline{b}_2
=
\begin{pmatrix}
2+i\\3\\2\\0
\end{pmatrix}.
\]
Als Basis für die reelle Normalform von $A$ kann man jetzt die Vektoren
\begin{align*}
c_1
&=
b_1+\overline{b}_1 = \begin{pmatrix}2\\4\\0\\2\end{pmatrix},&
d_1
&=
\frac{1}{i}(b_1-\overline{b}_1) = \begin{pmatrix}2\\4\\2\\0\end{pmatrix},&
c_2
&=
b_2+\overline{b}_2 = \begin{pmatrix}4\\6\\4\\0\end{pmatrix},&
d_2
&=
\frac{1}{i}(b_2-\overline{b}_2) = \begin{pmatrix}-2\\0\\0\\0\end{pmatrix}
\end{align*}
verwenden.
In dieser Basis hat $A$ die Matrix
\[
A'
=
\begin{pmatrix}
 2& 3& 1& 0\\
-3& 2& 0& 1\\
 0& 0& 2& 3\\
 0& 0&-3& 2
\end{pmatrix},
\]
wie man einfach nachrechnen kann.
\end{loesung}

