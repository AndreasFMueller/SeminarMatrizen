%
% errata.tex -- errors found in the book
%
% (c) 2021 Prof Dr Andreas Müller, OST Ostschweizer Fachhochschule
%

\documentclass{article}
\usepackage{geometry}
\geometry{papersize={170mm,240mm},total={140mm,200mm},top=21mm}
\usepackage[english,ngerman]{babel}
\usepackage[utf8]{inputenc}
\usepackage[T1]{fontenc}
\usepackage{times}
\usepackage{amsmath,amscd}
\usepackage{amssymb}
\usepackage{amsfonts}
\usepackage{amsthm}
\usepackage{txfonts}
\usepackage{verbatim}
\begin{document}

\def\fehler#1#2{
\bgroup\parindent0pt
\rule[0mm]{\textwidth}{0.5pt}
\hbox to\hsize{%
#1
\hfill
#2%
}
\rule[2mm]{\textwidth}{0.5pt}
\egroup
}

\def\korrektur#1{
\vspace{5mm}
\parindent0pt
Korrektur-Diff:\\
\small
\verbatiminput{diffs/#1.diff}
}

\section*{Mathematisches Seminar Matrizen --- Errata}

Die folgenden Fehler im Buch {\em Mathematisches Seminar Matrizen}
sind seit der Drucklegung im November 2021 bekannt geworden.

\vspace*{1cm}

\fehler{Seite 242, Zeile 6}{(7.~Dezember 2021)}

\[
A^{\odot0}=U
\qquad\text{und}\qquad
A^{\odot k}=A\odot A^{\odot(k-1)}
\]

\korrektur{1}

\fehler{Seite 265, Zeile 2}{(19.~Dezember 2021)}

{\Huge Anwendungen in der Kryptographie}

\korrektur{2}

\end{document}
