\documentclass[12pt, xcolor, aspectratio=169]{beamer}

% language
\usepackage{polyglossia}
\setmainlanguage{german}

% Theme
\beamertemplatenavigationsymbolsempty

% set look
\usetheme{default} 
\usecolortheme{fly}
\usefonttheme{serif}

%% Set font
\usepackage[p,osf]{scholax}
\usepackage{amsmath}
\usepackage[scaled=1.075,ncf,vvarbb]{newtxmath}

% set colors
\definecolor{background}{HTML}{202020}

\setbeamercolor{normal text}{fg=white, bg=background}
\setbeamercolor{structure}{fg=white}

\setbeamercolor{item projected}{use=item,fg=background,bg=item.fg!35}

\setbeamercolor*{palette primary}{use=structure,fg=white,bg=structure.fg}
\setbeamercolor*{palette secondary}{use=structure,fg=white,bg=structure.fg!75}
\setbeamercolor*{palette tertiary}{use=structure,fg=white,bg=structure.fg!50}
\setbeamercolor*{palette quaternary}{fg=white,bg=background}

\setbeamercolor*{block title}{parent=structure}
\setbeamercolor*{block body}{fg=background, bg=}

\setbeamercolor*{framesubtitle}{fg=white}

\setbeamertemplate{section page}
{
  \begin{center}
    \Huge
    \insertsection
  \end{center}
}
\AtBeginSection{\frame{\sectionpage}}

% Metadata
\title{\LARGE \scshape Punktgruppen und Kristalle}
\author[N. Pross, T. T\"onz]{Naoki Pross, Tim T\"onz}
\institute{Hochschule f\"ur Technik OST, Rapperswil}
\date{10. Mai 2021}

% Slides
\begin{document}
\frame{\titlepage}
\frame{\tableofcontents}

\section{Einleitung}
\frame{
  \[
    \psi
  \]
}

\section{Geometrische Symmetrien}
%% Made in video

\section{Algebraische Symmetrien}
\frame{
  \begin{columns}
    \begin{column}{.3\textwidth}
      Produkt mit \(i\)
      \begin{align*}
        1 \cdot i &= i \\
        i \cdot i &= -1 \\
        -1 \cdot i &= -i \\
        -i \cdot i &= 1
      \end{align*}
      \pause
      %
      Gruppe
      \begin{align*}
        G &= \left\{
          1, i, -1, -i
        \right\} \\
        &= \left\{
          1, i, i^2, i^3
        \right\} \\
        Z_4 &= \left\{
          \mathbb{1}, r, r^2, r^3
        \right\}
      \end{align*}
      \pause
      %
    \end{column}
    \begin{column}{.5\textwidth}
      %
      Darstellung
      \[
        \phi : Z_4 \to G
      \]
      \begin{align*}
        \phi(\mathbb{1}) &= 1 & \phi(r^2) &= i^2 \\
        \phi(r) &= i & \phi(r^3) &= i^3
      \end{align*}
      \pause
      %
      Homomorphismus
      \begin{align*}
        \phi(r \circ \mathbb{1}) &= \phi(r) \cdot \phi(\mathbb{1}) \\
        &= i \cdot 1
      \end{align*}
      \pause
      %
      \(\phi\) ist bijektiv \(\implies Z_4 \cong G\)
    \end{column}
  \end{columns}
}

\section{Kristalle}

\section{Anwendungen}

\end{document}
