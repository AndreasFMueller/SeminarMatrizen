\documentclass[12pt, xcolor, aspectratio=169]{beamer}

% language
\usepackage{polyglossia}
\setmainlanguage{german}

% pretty drawings
\usepackage{tikz}
\usetikzlibrary{positioning}
\usetikzlibrary{arrows.meta}
\usetikzlibrary{calc}

% Theme
\beamertemplatenavigationsymbolsempty

% set look
\usetheme{default} 
\usecolortheme{fly}
\usefonttheme{serif}

%% Set font
\usepackage[p,osf]{scholax}
\usepackage{amsmath}
\usepackage[scaled=1.075,ncf,vvarbb]{newtxmath}

% set colors
\definecolor{background}{HTML}{202020}

\setbeamercolor{normal text}{fg=white, bg=background}
\setbeamercolor{structure}{fg=white}

\setbeamercolor{item projected}{use=item,fg=background,bg=item.fg!35}

\setbeamercolor*{palette primary}{use=structure,fg=white,bg=structure.fg}
\setbeamercolor*{palette secondary}{use=structure,fg=white,bg=structure.fg!75}
\setbeamercolor*{palette tertiary}{use=structure,fg=white,bg=structure.fg!50}
\setbeamercolor*{palette quaternary}{fg=white,bg=background}

\setbeamercolor*{block title}{parent=structure}
\setbeamercolor*{block body}{fg=background, bg=}

\setbeamercolor*{framesubtitle}{fg=white}

\setbeamertemplate{section page}
{
  \begin{center}
    \Huge
    \insertsection
  \end{center}
}
\AtBeginSection{\frame{\sectionpage}}

% Macros
\newcommand{\ten}[1]{#1}

% Metadata
\title{\LARGE \scshape Punktgruppen und Kristalle}
\author[N. Pross, T. T\"onz]{Naoki Pross, Tim T\"onz}
\institute{Hochschule f\"ur Technik OST, Rapperswil}
\date{10. Mai 2021}

% Slides
\begin{document}
\frame{\titlepage}
\frame{\tableofcontents}

\frame{
  \begin{itemize}
    \item Was heisst \emph{Symmetrie} in der Mathematik? \pause
    \item Wie kann ein Kristall modelliert werden? \pause
    \item Aus der Physik: Piezoelektrizit\"at \pause
  \end{itemize}
  \begin{center}
    \begin{tikzpicture}
      \begin{scope}[
          node distance = 0cm
        ]
        \node[
          rectangle, fill = gray!40!background,
          minimum width = 3cm, minimum height = 2cm,
        ] (body) {\(\vec{E}_p = \vec{0}\)};

        \node[
          draw, rectangle, thick, white, fill = red!50,
          minimum width = 3cm, minimum height = 1mm,
          above = of body
        ] (pos) {};

        \node[
          draw, rectangle, thick, white, fill = blue!50,
          minimum width = 3cm, minimum height = 1mm,
          below = of body
        ] (neg) {};

        \draw[white, very thick, -Circle] (pos.east) to ++ (1,0) node (p) {};
        \draw[white, very thick, -Circle] (neg.east) to ++ (1,0) node (n) {};

        \draw[white, thick, ->] (p) to[out = -70, in = 70] node[midway, right] {\(U = 0\)} (n);
      \end{scope}
      \begin{scope}[
          node distance = 0cm,
          xshift = 7cm
        ]
        \node[
          rectangle, fill = gray!40!background,
          minimum width = 3cm, minimum height = 1.5cm,
        ] (body) {\(\vec{E}_p = \vec{0}\)};

        \node[
          draw, rectangle, thick, white, fill = red!50,
          minimum width = 3cm, minimum height = 1mm,
          above = of body
        ] (pos) {};

        \node[
          draw, rectangle, thick, white, fill = blue!50,
          minimum width = 3cm, minimum height = 1mm,
          below = of body
        ] (neg) {};

        \draw[orange, very thick, <-] (pos.north) to node[near end, right] {\(\vec{F}\)} ++(0,1);
        \draw[orange, very thick, <-] (neg.south) to node[near end, right] {\(\vec{F}\)} ++(0,-1);

        \draw[white, very thick, -Circle] (pos.east) to ++ (1,0) node (p) {};
        \draw[white, very thick, -Circle] (neg.east) to ++ (1,0) node (n) {};

        \draw[white, thick, ->] (p) to[out = -70, in = 70] node[midway, right] {\(U > 0\)} (n);
      \end{scope}
    \end{tikzpicture}
  \end{center}
}

\section{2D Symmetrien}
%% Made in video

\section{Algebraische Symmetrien}
%% Made in video
\frame{
  \begin{columns}[T]
    \begin{column}{.5\textwidth}
      Produkt mit \(i\)
      \begin{align*}
        1 \cdot i &= i \\
        i \cdot i &= -1 \\
        -1 \cdot i &= -i \\
        -i \cdot i &= 1
      \end{align*}
      \pause
      %
      Gruppe
      \begin{align*}
        G &= \left\{
          1, i, -1, -i
        \right\} \\
        &= \left\{
          1, i, i^2, i^3
        \right\} \\
        C_4 &= \left\{
          \mathbb{1}, r, r^2, r^3
        \right\}
      \end{align*}
      \pause
    \end{column}
    \begin{column}{.5\textwidth}
      Darstellung \(\phi : C_4 \to G\)
      \begin{align*}
        \phi(\mathbb{1}) &= 1 & \phi(r^2) &= i^2 \\
        \phi(r) &= i & \phi(r^3) &= i^3
      \end{align*}
      \pause
      %
      Homomorphismus
      \begin{align*}
        \phi(r \circ \mathbb{1}) &= \phi(r) \cdot \phi(\mathbb{1}) \\
        &= i \cdot 1
      \end{align*}
      \pause
      %
      \(\phi\) ist bijektiv \(\implies C_4 \cong G\)
			\pause
			%
			\begin{align*}
				\psi : C_4 &\to (\mathbb{Z}/4\mathbb{Z}, +) \\
				\psi(\mathbb{1}\circ r^2) &= 0 + 2 \pmod{4}
			\end{align*}
    \end{column}
  \end{columns}
}

\section{3D Symmetrien}
%% Made in video

\section{Matrizen}
\frame{
  \begin{columns}[T]
    \begin{column}{.5\textwidth}
      Symmetriegruppe
      \[
        G = \left\{\mathbb{1}, r, \sigma, \dots \right\}
      \]
      \pause
      Matrixdarstellung
      \begin{align*}
        \Phi : G &\to O(3) \\
        g &\mapsto \Phi_g
      \end{align*}
      \pause
      Orthogonale Gruppe
      \[
        O(n) = \left\{ Q : QQ^t = Q^tQ = I \right\}
      \]
    \end{column}
    \pause
    \begin{column}{.5\textwidth}
      \begin{align*}
        \Phi_\mathbb{1} &= \begin{pmatrix}
          1 & 0 & 0 \\
          0 & 1 & 0 \\
          0 & 0 & 1
        \end{pmatrix} = I \\[1em]
        \Phi_\sigma &= \begin{pmatrix}
          1 & 0 & 0 \\
          0 & -1 & 0 \\
          0 & 0 & 1
        \end{pmatrix} \\[1em]
        \Phi_r &= \begin{pmatrix}
          \cos \alpha & -\sin \alpha & 0 \\
          \sin \alpha & \cos \alpha & 0 \\
          0 & 0 & 1
        \end{pmatrix}
      \end{align*}
    \end{column}
  \end{columns}
}

\section{Kristalle}
\begin{frame}[fragile]{M\"ogliche Kristallstrukturen}
  \begin{center}
    \begin{tikzpicture}[]
      \node[circle, dashed, draw = gray, 
        thick, fill = background,
        minimum size = 4cm] {};
      \node[gray] at (.9,-1.2) {674};

      \node[circle, draw = white, thick,
        fill = orange!40!background,
        xshift = -3mm, yshift = 2mm,
        minimum size = 2.75cm] (A) {};
      \node[white, yshift = 2mm] at (A) {230};

      \node[circle, draw = white, thick, 
        fill = red!20!background,
        xshift = -5mm, yshift = -5mm,
        minimum size = 1cm] {32};
    \end{tikzpicture}
  \end{center}
\end{frame}

\begin{frame}[fragile]{}
  \begin{columns}
    \begin{column}{.5\textwidth}
      \begin{center}
          \begin{tikzpicture}[
              dot/.style = {
                draw, circle, thick, white, fill = gray!40!background,
                minimum size = 2mm,
                inner sep = 0pt,
                outer sep = 1mm,
              },
            ]

            \begin{scope}
              \clip (-2,-2) rectangle (3,4);
              \foreach \y in {-7,-6,...,7} {
                \foreach \x in {-7,-6,...,7} {
                  \node[dot, xshift=3mm*\y] (N\x\y) at (\x, \y) {};
                }
              }
            \end{scope}
            \draw[white, thick] (-2, -2) rectangle (3,4);

            \draw[red!80!background, thick, ->] 
              (N00) to node[midway, below] {\(\vec{a}_1\)} (N10);
            \draw[cyan!80!background, thick, ->]
              (N00) to node[midway, left] {\(\vec{a}_2\)} (N01);
          \end{tikzpicture}
      \end{center}
    \end{column}
    \pause
    \begin{column}{.5\textwidth}
      Kristallgitter:
      \(n_i \in \mathbb{Z}\),
      \(\vec{a}_i \in \mathbb{R}^3\)
      \[
        \vec{r} = n_1 \vec{a}_1 + n_2 \vec{a}_2 + n_3 \vec{a}_3
      \]
      \vspace{1cm}
      \pause
      
      Invariant unter Translation
      \[
        Q_i(\vec{r}) = \vec{r} + \vec{a}_i
      \]
    \end{column}
  \end{columns}
\end{frame}

\begin{frame}[fragile]{}
  \begin{columns}[T]
    \begin{column}{.5\textwidth}
      \onslide<1->{
        Wie kombiniert sich \(Q_i\) mit der anderen Symmetrien?
      }
      \begin{center}
        \begin{tikzpicture}[
            dot/.style = {
              draw, circle, thick, white, fill = gray!40!background,
              minimum size = 2mm,
              inner sep = 0pt,
              outer sep = 1mm,
            },
          ]

          \onslide<2->{
            \node[dot] (A1) at (0,0) {};
            \node[below left] at (A1) {\(A\)};
          }

          \onslide<3->{
            \node[dot] (A2) at (2.5,0) {};
            \node[below right] at (A2) {\(A'\)};

            \draw[red!80!background, thick, ->] 
              (A1) to node[midway, below] {\(\vec{Q}\)} (A2);
          }

          \onslide<4->{
            \node[dot] (B1) at (120:2.5) {};
            \node[above left] at (B1) {\(B\)};

            \draw[green!70!background, thick, ->]
              (A1) ++(.5,0) arc (0:120:.5)
              node[midway, above, xshift=1mm] {\(C_n\)};
            \draw[red!80!background, dashed, thick, ->] (A1) to (B1);
          }

          \onslide<5->{
            \node[dot] (B2) at ($(A2)+(60:2.5)$) {};
            \node[above right] at (B2) {\(B'\)};

            \draw[green!70!background, thick, dashed, ->]
              (A2) ++(-.5,0) arc (180:60:.5);
            \draw[red!80!background, dashed, thick, ->] (A2) to (B2);
          }

          \onslide<6->{
            \draw[yellow!80!background, thick, ->]
              (B1) to node[above, midway] {\(\vec{Q}'\)} (B2);
          }

          \onslide<7->{
          \draw[gray, dashed, thick] (A1) to (A1 |- B1) node (X) {};
          \draw[gray, dashed, thick] (A2) to (A2 |- B2);
          }

          \onslide<8->{
            \node[above left, xshift=-2mm] at (X) {\(x\)};
          }
        \end{tikzpicture}
      \end{center}
    \end{column}
    \begin{column}{.5\textwidth}
      \onslide<9->{
        Sei \(q = |\vec{Q}|\), \(\alpha = 2\pi/n\) und \(n \in \mathbb{N}\)
      }
      \begin{align*}
        \onslide<10->{q' = n q &= q + 2x \\}
        \onslide<11->{nq &= q + 2q\sin(\alpha - \pi/2) \\}
        \onslide<12->{n &= 1 - 2\cos\alpha}
      \end{align*}
      \onslide<13->{
        Somit muss
        \begin{align*}
          \alpha &= \cos^{-1}\left(\frac{1-n}{2}\right) \\[1em]
          \alpha &\in \left\{ 0, 60^\circ, 90^\circ, 120^\circ, 180^\circ \right\}
        \end{align*}
      }
    \end{column}
  \end{columns}
\end{frame}

\section{Anwendungen}
\begin{frame}[fragile]{}
  \centering
  \begin{tikzpicture}[
      box/.style = {
        rectangle, thick, draw = white, fill = darkgray!50!background,
        minimum height = 1cm, outer sep = 2mm,
      },
    ]

    \matrix [nodes = {box, align = center}, column sep = 1cm, row sep = 1.5cm] {
      & \node (A) {32 Punktgruppen}; \\
      \node (B) {11 Mit\\ Inversionszentrum}; & \node (C) {21 Ohne\\ Inversionszentrum}; \\
      & \node[fill=red!20!background] (D) {20 Piezoelektrisch}; & \node (E) {1 Nicht\\ piezoelektrisch}; \\
    };

    \draw[thick, ->] (A.west) to[out=180, in=90] (B.north);
    \draw[thick, ->] (A.south) to (C);
    \draw[thick, ->] (C.south) to (D.north);
    \draw[thick, ->] (C.east) to[out=0, in=90] (E.north);
  \end{tikzpicture}
\end{frame}

\begin{frame}[fragile]{}
  \begin{tikzpicture}[
      overlay, xshift = 1.5cm, yshift = 1.5cm,
      node distance = 2mm,
      charge/.style = {
        circle, draw = white, thick,
        minimum size = 5mm
      },
      positive/.style = { fill = red!50 },
      negative/.style = { fill = blue!50 },
    ]

    \node[font = {\large\bfseries}, align = center] (title) at (5.5,0) {Mit und Ohne\\ Symmetriezentrum};
    \node[below = of title] {Polarisation Feld \(\vec{E}_p\)};
    \pause

    \begin{scope}
      \matrix[nodes = { charge }, row sep = 8mm, column sep = 8mm] {
        \node[positive] {}; & \node[negative] (N) {}; & \node [positive] {}; \\
        \node[negative] (W) {}; & \node[positive] {}; & \node [negative] (E) {}; \\
        \node[positive] {}; & \node[negative] (S) {}; & \node [positive] {}; \\
      };
      \draw[gray, dashed] (W) to (N) to (E) to (S) to (W);
    \end{scope}
    \pause

    \begin{scope}[yshift=-4.5cm]
      \matrix[nodes = { charge }, row sep = 5mm, column sep = 1cm] {
        \node[positive] (NW) {}; & \node[negative] (N) {}; & \node [positive] (NE) {}; \\
        \node[negative] (W) {}; & \node[positive] {}; & \node [negative] (E) {}; \\
        \node[positive] (SW) {}; & \node[negative] (S) {}; & \node [positive] (SE) {}; \\
      };

      \foreach \d in {NW, N, NE} {
        \draw[orange, very thick, <-] (\d) to ++(0,.7);
      }

      \foreach \d in {SW, S, SE} {
        \draw[orange, very thick, <-] (\d) to ++(0,-.7);
      }

      \draw[gray, dashed] (W) to (N) to (E) to (S) to (W);
    \end{scope}
    \pause

    \begin{scope}[xshift=11cm]
      \foreach \x/\t [count=\i] in {60/positive, 120/negative, 180/positive, 240/negative, 300/positive, 360/negative} {
        \node[charge, \t] (C\i) at (\x:1.5cm) {};
      }

      \draw[white] (C1) to (C2) to (C3) to (C4) to (C5) to (C6) to (C1);
      % \draw[gray, dashed] (C2) to (C4) to (C6) to (C2);
    \end{scope}
    \pause

    \begin{scope}[xshift=11cm, yshift=-4.5cm]
      \node[charge, positive, yshift= 2.5mm] (C1) at ( 60:1.5cm) {};
      \node[charge, negative, yshift= 2.5mm] (C2) at (120:1.5cm) {};
      \node[charge, positive, xshift= 2.5mm] (C3) at (180:1.5cm) {};
      \node[charge, negative, yshift=-2.5mm] (C4) at (240:1.5cm) {};
      \node[charge, positive, yshift=-2.5mm] (C5) at (300:1.5cm) {};
      \node[charge, negative, xshift=-2.5mm] (C6) at (360:1.5cm) {};

      \draw[white] (C1) to (C2) to (C3) to (C4) to (C5) to (C6) to (C1);
      % \draw[gray, dashed] (C2) to (C4) to (C6) to (C2);

      \draw[orange, very thick, <-] (C6) to ++(.7,0);
      \draw[orange, very thick, <-] (C3) to ++(-.7,0);

      \node[white] (E) {\(\vec{E}_p\)};
      \begin{scope}[node distance = .5mm]
        \node[red!50, right = of E] {\(+\)};
        \node[blue!50, left = of E] {\(-\)};
      \end{scope}
      \draw[gray, thick, dotted] (E) to ++(0,2);
      \draw[gray, thick, dotted] (E) to ++(0,-2);
    \end{scope}
    \pause

    \begin{scope}[xshift=5.5cm, yshift=-4.5cm]
      \node[charge, positive, yshift=-2.5mm] (C1) at ( 60:1.5cm) {};
      \node[charge, negative, yshift=-2.5mm] (C2) at (120:1.5cm) {};
      \node[charge, positive, xshift=-2.5mm] (C3) at (180:1.5cm) {};
      \node[charge, negative, yshift= 2.5mm] (C4) at (240:1.5cm) {};
      \node[charge, positive, yshift= 2.5mm] (C5) at (300:1.5cm) {};
      \node[charge, negative, xshift= 2.5mm] (C6) at (360:1.5cm) {};

      \draw[white] (C1) to (C2) to (C3) to (C4) to (C5) to (C6) to (C1);
      % \draw[gray, dashed] (C2) to (C4) to (C6) to (C2);

      \foreach \d in {C1, C2} {
        \draw[orange, very thick, <-] (\d) to ++(0,.7);
      }

      \foreach \d in {C4, C5} {
        \draw[orange, very thick, <-] (\d) to ++(0,-.7);
      }

      \node[white] (E) {\(\vec{E}_p\)};
      \begin{scope}[node distance = .5mm]
        \node[red!50, right = of E] {\(+\)};
        \node[blue!50, left = of E] {\(-\)};
      \end{scope}
      \draw[gray, thick, dotted] (E) to ++(0,2);
      \draw[gray, thick, dotted] (E) to ++(0,-2);
    \end{scope}
  \end{tikzpicture}
\end{frame}

\frame{
  \frametitle{Licht in Kristallen}
  \begin{columns}[T]
    \begin{column}{.5\textwidth}
      Symmetriegruppe und Darstellung
      \begin{align*}
        G &= \left\{\mathbb{1}, r, \sigma, \dots \right\} \\
        &\Phi : G \to O(n)
      \end{align*}
      \begin{align*}
        U_\lambda &= \left\{ v : \Phi v = \lambda v \right\} \\
        &= \mathrm{null}\left(\Phi - \lambda I\right)
      \end{align*}
      Helmholtz Wellengleichung
      \[
        \nabla^2 \vec{E} = \ten{\varepsilon}\mu
        \frac{\partial^2}{\partial t^2} \vec{E}
      \]
    \end{column}
    \begin{column}{.5\textwidth}
      Ebene Welle
      \[
        \vec{E} = \vec{E}_0 \exp\left[i
          \left(\vec{k}\cdot\vec{r} - \omega t \right)\right]
      \]
      Anisotropisch Dielektrikum
      \[
        (\ten{K}\ten{\varepsilon})\vec{E} = \frac{k^2}{\mu \omega^2} \vec{E}
      \]
      \[
        \vec{E} \in U_\lambda \implies (\ten{K}\ten{\varepsilon}) \vec{E} = \lambda \vec{E}
      \]
    \end{column}
  \end{columns}
}

\end{document}
