\documentclass[a4paper]{article}

\usepackage[cm]{manuscript}
\usepackage{xcolor}

\newcommand{\scene}[1]{\noindent[ #1 ]\par}
\newenvironment{totranslate}{\color{red!60!black}}{}

\begin{document}

\section{Intro}

\section{Geometrie}
\scene{Intro}
\scene{Zyklische Gruppe}

\begin{totranslate}
  Let's now focus our attention on the simplest class of symmetries: those
  generated by a single rotation. We describe the symmetries with a group
  \(G\), and denote that it is generated by a rotation \(r\) with these angle
  brackets.
  
  Take this shape as an example. By applying the rotation \emph{action} 5
  times, it looks as if we had not done anything, furthermore, if we \emph{act}
  with higher ``powers'' \(r\), they will have the same effect as one of the
  previous action.  Thus the group only contain the identity and the powers of
  \(r\) up to 4.
  
  In general, groups with this structure are known as the
  ``Cyclic Groups'' of order \(n\), where the action \(r\) can be applied
  \(n-1\) times before wrapping around.
\end{totranslate}

\scene{Diedergruppe}

\begin{totranslate}
  Okay that was not difficult, now let's spice this up a bit. 
\end{totranslate}

\scene{Symmetrische Gruppe}
\scene{Alternierende Gruppe}

\section{Algebra}

\section{Krystalle}

\end{document}
