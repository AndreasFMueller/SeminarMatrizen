\documentclass[a4paper]{article}

\usepackage{amsmath}
\usepackage{amssymb}

\usepackage[cm]{manuscript}
\usepackage{xcolor}

\newcommand{\scene}[1]{\par\noindent[ #1 ]\par}
\newenvironment{totranslate}{\color{blue!70!black}}{}

\begin{document}
\section{Das sind wir}
\scene{Tim}
Willkommen zu unserer Präsentation über Punktgruppen und deren Anwendung in der
Kristallographie.  Ich bin Tim Tönz habe vor dem Studium die Lehre als
Elektroinstallateur abgeschlossen und studiere jetzt Elektrotechnik im Vierten
Semester mit Herrn Naoki Pross. 
\scene{Naoki}
  Das bin ich  \ldots  Nun zum Inhalt 

\section{Ablauf}
Wir möchten Euch zeigen, was eine Punktgruppe ausmacht, Konkret an Bespielen in 2D zeigen mit Gemainsamkeiten zu Algebraischen Symmetrien. 
Da wir Menschen jedoch 3 Räumliche Dimensionen Wahrnehmen möchten wir euch die 3D Symetrien natürlcih nicht vorenthalten.
Um dem Thema des Mathematikseminars gerecht zu werden, Werden wir die einfache Verbindung zwischen Matrizen und Punktsymetrien zeigen.
Dammit die Praxis nicht ganz vergessen geht, Kristalle Mathematisch beschreiben und dessen Limitationen in hinsicht Symmetrien. 
Als Abschluss Zeigen wir euch einen zusammenhan zwischen Piezoelektrizität und Symmetrien.

\section{intro}
Ich hoffe wir konnten schon mit der Einleitung ein wenig Neugirde wecken.
fals dies noch nicht der Fall ist, sind hier noch die wichtigsten fragen, welche wir euch beantworten wollen, oder zumindest überzeugen, wieso dies spannende Fragen sind.  
Als erstes, was eine Symetrie ist oder in unserem Fall eine Punktsymetrie.
Was macht ein Kristall aus, also wie kann man seine Wichtigsten eigenschaften mathematisch beschreiben.
Als letztes noch zu der Piezoelektrizität, welche ein Effekt beschreibt, dass bestimmte Krisstalle eine elektrische Spannung erzeugen, wenn sie unter mechanischen Druck gesetzt werden. 
welche kristalle diese fähigkeit haben, hat ganz konkret mit ihrer Symmetrie zu tun.

\section{Geometrie}
\begin{totranslate}
We'll start with geometric symmetries as they are the simplest to grasp.

\scene{Intro}
  To mathematically formulate the concept, we will think of symmetries as
  actions to perform on an object, like this square. The simplest action, is to
  take this square, do nothing and put it back down. Another action could be to
  flip it along an axis, or to rotate it around its center by 90 degrees.

\scene{Cyclic Groups}
  Let's focus our attention on the simplest class of symmetries: those
  generated by a single rotation. We will gather the symmetries in a group
  \(G\), and denote that it is generated by a rotation \(r\) with these angle
  brackets.
  
  Take this pentagon as an example. By applying the rotation \emph{action} 5
  times, it is the same as if we had not done anything, furthermore, if we
  \emph{act} a sixth time with \(r\), it will be the same as if we had just
  acted with \(r\) once.  Thus the group only contain the identity and the
  powers of \(r\) up to 4.
  
  In general, groups with this structure are known as the ``Cyclic Groups'' of
  order \(n\), where the action \(r\) can be applied \(n-1\) times before
  wrapping around. 

  % You can think of them as the rotational symmetries of an \(n\)-gon.

\scene{Dihedral Groups}
  Okay that was not difficult, now let's spice this up a bit. Consider this
  group for a square, generated by two actions: a rotation \(r\) and a
  reflection \(\sigma\). Because we have two actions we have to write in the
  generator how they relate to each other.

  Let's analyze this expression. Two reflections are the same as the identity.
  Four rotations are the same as the identity, and a rotation followed by a
  reflection, twice, is the same as the identity.

  This forms a group with 8 possible unique actions. This too can be generalized
  to an \(n\)-gon, and is known as the ``Dihedral Group'' of order \(n\).
\end{totranslate}

\section{Algebra}
\scene{Produkt mit \(i\)}
\"Uberlegen wir uns eine spezielle algebraische Operation: Multiplikation mit
der imagin\"aren Einheit. \(1\) mal \(i\) ist gleich \(i\). Wieder mal \(i\)
ist \(-1\), dann \(-i\) und schliesslich kommen wir z\"uruck auf \(1\).  Diese
fassen wir in eine Gruppe \(G\) zusammen. Oder sch\"oner geschrieben:. Sieht das
bekannt aus?

\scene{Morphismen}
Das Gefühl, dass es sich um dasselbe handelt, kann wie folgt formalisiert
werden.  Sei \(\phi\) eine Funktion von \(C_4\) zu \(G\). Ordnen wir zu jeder
Symmetrieoperation ein Element aus \(G\). Wenn man die Zuordnung richtig
definiert, dann sieht man die folgende Eigenschaft: Eine Operation nach eine
andere zu nutzen, und dann die Funktion des Resultats zu nehmen, ist gleich wie
die Funktion der einzelnen Operazionen zu nehmen und das Resultat zu
multiplizieren. Dieses Ergebnis ist so bemerkenswert, dass es in der Mathematik
einen Namen bekommen hat: Homorphismus, von griechisch "homos" dasselbe und
"morphe" Form.  Manchmal wird es auch so geschrieben. Ausserdem, wenn \(\phi\)
eins zu eins ist, heisst es \emph{Iso}morphismus: "iso" gleiche Form. Was
man typischerweise mit diesem Symbol schreibt.

\scene{Animation}
Sie haben wahrscheinlich schon gesehen, worauf das hinausläuft. Dass die
zyklische Gruppe \(C_4\) und \(G\) die gleiche Form haben, ist im wahrste Sinne
des Wortes. %% Ask Tim: literally true

\scene{Modulo}
Der Beispiel mit der komplexen Einheit, war wahrscheinlich nicht so
\"uberraschend. Aber was merkw\"urdig ist, ist das diese geometrische Struktur,
kann man auch in anderen Sachen finden, die erst nicht geometrisch aussehen.
Ein Beispiel für Neugierige: Summe in der Modulo-Arithmetik. Um die Geometrie
zu finden denken Sie an einer Uhr.

\section{Matrizen}
\scene{Titelseite}
Nun gehen wir kurz auf den Thema unseres Seminars ein: Matrizen.  Das man mit
Matrizen Dinge darstellen kann, ist keine Neuigkeit mehr, nach einem
Semester MatheSeminar.  Also überrascht es wohl auch keinen, das man alle
punktsymmetrischen Operationen auch mit Matrizen Formulieren kann.

\scene{Matrizen}

Sei dann \(G\) unsere Symmetrie Gruppe, die unsere abstrakte Drehungen und
Spiegelungen enth\"ahlt. Die Matrix Darstellung dieser Gruppe, ist eine
Funktion gross \(\Phi\), von \(G\) zur orthogonalen Gruppe \(O(3)\), die zu
jeder Symmetrie Operation klein \(g\) eine Matrix gross \(\Phi_g\) zuordnet.

Zur Erinnerung, die Orthogonale Gruppe ist definiert als die Matrizen, deren
transponierte auch die inverse ist. Da diese Volumen und Distanzen erhalten,
natuerlich nur bis zu einer Vorzeichenumkehrung, macht es Sinn, dass diese
Punksymmetrien genau beschreiben.

Nehmen wir die folgende Operationen als Beispiele. Die Matrix der trivialen
Operation, dass heisst nichts zu machen, ist die Einheitsmatrix. Eine
Spiegelung ist dasselbe aber mit einem Minus, und Drehungen sind uns schon
dank Herrn M\"uller bekannt.

% (Beispiel zu Rotation mit video) Für die Spiegelung wie auch eine Punkt
% inversion habt ihr dank dem matheseminar bestmmt schon eine Idee wie diese
% Operationen als Matrizen aussehen.  Ich weis nicht obe der Tipp etwas nützt,
% aber ih müsst nur in der Gruppe O(3) suchen.  Was auch sinn macht, denn die
% Gruppe O(3) zeichnet sich aus weil ihre Matrizen distanzen konstant hallten
% wie auch einen fixpunkt haben was sehr erwünscht ist, wenn man
% Punktsymmetrien beschreiben will.

\section{Krystalle}
  Jenen welchen die Kristalle bis jetzt ein wenig zu kurz gekommen sind, Freuen sich hoffentlich zurecht an dieser Folie.
  Es geht ab jetzt nähmlich um Kristalle. 
  Bevor wir mit ihnen arbeiten könne sollten wir jedoch klähren, was ein Kristall ist. 
  Per definition aus eienm Anerkanten Theoriebuch von XXXXXXXXXX Zitat:"YYYYYYYYYYYYYYY"
  Was so viel  heist wie, ein Idealer Kristall ist der schlimmste Ort um sich zu verlaufen.
  Macht man nähmlich einen Schritt in genau in das nächste lattice feld hat siet der kristall wieser genau gleich aus. 
  Als Orentierungshilfe ist diese eigenschaft ein grosser Nachteil nicht jedoch wenn man versucht alle möglichen Symmetrien in einem Kristall zu finden.
  Denn die Lattice Strucktur schränkt die unendlichen möglichen Punktsymmetrien im 3D Raum beträchtlich ein. 
  Was im Englischen bekannt is unter dem Crystallographic Restrictiontheorem.  
  
  \scene{Crystallographic restriction Theorem}
    Die Punktsymmetrien von Kristallen sind auf grund verschiedensten geometrischen überlegungen eingeschränkt.
    Wir zeigen euch hier nur den beweis wieso die in einem Kristall nur Rotations symetrien um 360,180,120,90 und 60 grad haben kann.
    Für den Beweis beginnen wir mit einem Punkt A in dem Gitter wir wssen das in nach einer translation um  eine gitterbasis wieder ein Punkt A' existieren muss.
    Wir suchen Rotationssymmetrien also drehen wir um den winkel \( \alpha \) und müssen dank der drehsymmetrie \(\alpha\) wieder einen punkt im Gitter finden hier B.
    Das selbe oder hier genau die die inverse drehung um \(\alpha\) von A' aus muss uns daher den Punkt B' liefern.
    Zwischen zwei punkten im Gitter muss aber die Opertation Q angewendet werden können.
    Das heisst der Abstand zwischen B und B' mmuss ein ganzes vielfachen von dem Abstand B zu B' sein. 
    
  \scene{Restriktion in Algebra}
    Ausgeschrieben setzen wir klein auf die Länge der Translation, \(\alpha\) auf  \(2\pi / n\) und \(n\) auf \(\mathbb{N}\).

\end{document}
% vim:et ts=2 sw=2:
