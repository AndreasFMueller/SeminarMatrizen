\documentclass[a4paper]{article}

\usepackage[cm]{manuscript}
\usepackage{xcolor}

\newcommand{\scene}[1]{\par\noindent[ #1 ]\par}
\newenvironment{totranslate}{\color{blue!70!black}}{}

\begin{document}
\section{hello}
(TT) Willkommen zu unserer Präsentation über Punktgruppen und deren Anwendung in der Kristallographie. 
Ich bin Tim Tönz habe vor dem Studium die Lehre als Elektroinstallateur abgeschlossen und studiere jetzt Elektrotechnik im Vierten Semester mit Herrn Naoki Pross. 
(NP)Das bin ich\ldots   \ldots  Nun zum Inhalt 

\section{Introtim}
Wir möchten Euch zeigen, was eine Punktgruppe ausmacht, an bespielen zeigen, wie sie im 2D und 3D Raum aussehen kann und Zusammenhänge zu Algebraischen Symmetrien erläutern. 
Mit dem Wissen über Punktgruppen können wir uns versuchen der Praxis anzunähern, in unserem Fall dem Kristall und seiner Strukturellen Eigenschaften.  
Als Abschluss Zeigen wir euch konkret wieso ein inversionszentrum ein Piezoelektrisches verhalten in einem Kristall ausschliesst.

\section{intro}


\section{Geometrie}
\begin{totranslate}
We'll start with geometric symmetries as they are the simplest to grasp.

\scene{Intro}
  To mathematically formulate the concept, we will think of symmetries as
  actions to perform on an object, like this square. The simplest action, is to
  take this square, do nothing and put it back down. Another action could be to
  flip it along an axis, or to rotate it around its center by 90 degrees.

\scene{Cyclic Groups}
  Let's focus our attention on the simplest class of symmetries: those
  generated by a single rotation. We will gather the symmetries in a group
  \(G\), and denote that it is generated by a rotation \(r\) with these angle
  brackets.
  
  Take this pentagon as an example. By applying the rotation \emph{action} 5
  times, it is the same as if we had not done anything, furthermore, if we
  \emph{act} a sixth time with \(r\), it will be the same as if we had just
  acted with \(r\) once.  Thus the group only contain the identity and the
  powers of \(r\) up to 4.
  
  In general, groups with this structure are known as the ``Cyclic Groups'' of
  order \(n\), where the action \(r\) can be applied \(n-1\) times before
  wrapping around. 

  % You can think of them as the rotational symmetries of an \(n\)-gon.

\scene{Dihedral Groups}
  Okay that was not difficult, now let's spice this up a bit. Consider this
  group for a square, generated by two actions: a rotation \(r\) and a
  reflection \(\sigma\). Because we have two actions we have to write in the
  generator how they relate to each other.

  Let's analyze this expression. Two reflections are the same as the identity.
  Four rotations are the same as the identity, and a rotation followed by a
  reflection, twice, is the same as the identity.

  This forms a group with 8 possible unique actions. This too can be generalized
  to an \(n\)-gon, and is known as the ``Dihedral Group'' of order \(n\).
\end{totranslate}

\scene{Symmetrische Gruppe}
\scene{Alternierende Gruppe}

\section{Algebra}
\begin{totranslate}
Let's now move into something seemingly unrelated: \emph{algebra}.
\scene{Complex numbers and cyclic groups}
\end{totranslate}

\section{Krystalle}

\end{document}
% vim:et ts=2 sw=2:
