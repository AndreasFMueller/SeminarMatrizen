%
% slides.tex -- XXX
%
% (c) 2017 Prof Dr Andreas Müller, Hochschule Rapperswil
%
\section{Quotient}
\folie{2/quotient.tex}
\folie{2/quotientv.tex}

\section{Polynome}
\folie{3/motivation.tex}
\folie{3/polynome.tex}
\folie{3/operatoren.tex}
\folie{3/division.tex}
\folie{3/division2.tex}
\folie{3/teilbarkeit.tex}
\folie{3/ideal.tex}
\folie{3/idealverband.tex}
\folie{3/faktorisierung.tex}
\folie{3/faktorzerlegung.tex}
\folie{3/einsetzen.tex}
\folie{3/minimalpolynom.tex}

\section{Adjunktion}
\folie{3/quotientenring.tex}
\folie{3/maximalideal.tex}
\folie{3/adjunktion.tex}
\folie{3/adjalgebra.tex}
\folie{3/wurzel2.tex}
% XXX Beispiel: Adjunktion von \varphi
\folie{3/fibonacci.tex}

