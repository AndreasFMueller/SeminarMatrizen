%
% slides.tex -- XXX
%
% (c) 2017 Prof Dr Andreas Müller, Hochschule Rapperswil
%

\section{Matrixnormen}
%\folie{2/norm.tex}
%\folie{2/skalarprodukt.tex}
%\folie{2/cauchyschwarz.tex}
%\folie{2/polarformel.tex}
%\folie{2/operatornorm.tex}
%\folie{2/frobeniusnorm.tex}

\section{Approximation mit Polynomen}
% XXX Stone-Weierstrass
% XXX \folie{5/stoneweierstrass.tex}
% XXX Spektrum einer Matrix
% XXX \folie{5/spektrum.tex}
% XXX Approximation einer Funktion auf dem Spektrum
% XXX \folie{5/spektrumapproximation.tex}
% XXX Approximation einer Matrix in der erzeugten Algebra
% XXX \folie{5/matrixapproximation.tex}
% XXX Gelfand-Transformation
% XXX \folie{5/gelfandtransformation.tex}

\section{Potenzreihen}
% XXX Konvergenzradius
% XXX \folie{5/konvergenzradius.tex}
% XXX Gelfand-Radius
% XXX \folie{5/gelfandradius.tex}
% XXX Gleichheit von Konvergenz-Radius und Gelfand-Radius (braucht JNF)
% XXX \folie{5/satzvongelfand.tex}

\section{Differentialgleichungen}
% XXX Potenzreihenmethode zur Lösung von Differentialgleichungen
% XXX \folie{5/potenzreihenmethode.tex}
% XXX Exponentialfunktion
% XXX \folie{5/exponentialfunktion.tex}
% XXX Exponentialreihe
% XXX \folie{5/exponentialreihe.tex}
% XXX Logarithmus
% XXX \folie{5/logarithmusreihe.tex}
% XXX Sinus und Cosinus, Eulerscher Satz
% XXX \folie{5/sinuscosinus.tex}

