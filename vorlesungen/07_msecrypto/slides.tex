%
% slides.tex -- Präsentation zur Kryptographie
%
% (c) 2017 Prof Dr Andreas Müller, Hochschule Rapperswil
%
\section{Diffie-Hellmann}
\folie{4/dh.tex}

\section{Divide and Conquer}
% Potenzieren
\folie{a/dc/prinzip.tex}
% effiziente Durchführung
\folie{a/dc/effizient.tex}
% Beispieldurchführung
\folie{a/dc/beispiel.tex}

\section{Elliptische Kurven}
% Idee
\folie{a/ecc/gruppendh.tex}
% Was ist eine elliptische Kurve (char 0 Bild)
\folie{a/ecc/kurve.tex}
% Involution/Inverse
\folie{a/ecc/inverse.tex}
% Verknüpfung
\folie{a/ecc/operation.tex}
% Quadrieren
\folie{a/ecc/quadrieren.tex}
% Oakley Gruppe
\ifthenelse{\boolean{presentation}}{
\folie{a/ecc/oakley.tex}
}{}

\section{AES}
% Byte-Operationen
\folie{a/aes/bytes.tex}
\folie{a/aes/sinverse.tex}
% Block-Operationen
\folie{a/aes/blocks.tex}
% Key-Schedule
\folie{a/aes/keys.tex}
% Zusammensetzung
\folie{a/aes/runden.tex}

