%
% slides.tex -- Präsentation zur Kryptographie
%
% (c) 2017 Prof Dr Andreas Müller, Hochschule Rapperswil
%
\section{Diffie-Hellmann}
\folie{4/dh.tex}

\section{Divide and Conquer}
% XXX Potenzieren
%\folie{a/dc/prinzip.tex}
% XXX effiziente Durchführung
%\folie{a/dc/effizient.tex}
% XXX Beispieldurchführung
%\folie{a/dc/beispiel.tex}

\section{Elliptische Kurven}
% XXX Idee
%\folie{a/ecc/gruppendh.tex}
% XXX Was ist eine elliptische Kurve (char 0 Bild)
%\folie{a/ecc/kurve.tex}
% XXX Verknüpfung
%\follie{a/ecc/operation.tex}
% XXX Quadrieren
%\folie{a/ecc/quadrieren.tex}

\section{AES}
% XXX Byte-Operationen
%\folie{a/aes/bytes.tex}
% XXX Block-Operationen
%\folie{a/aes/blocks.tex}
% XXX Key-Schedule
%\folie{a/aes/keys.tex}
% XXX Zusammensetzung
%\folie{a/aes/runden.tex}

