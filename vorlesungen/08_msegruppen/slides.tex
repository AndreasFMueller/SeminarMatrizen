%
% slides.tex -- XXX
%
% (c) 2017 Prof Dr Andreas Müller, Hochschule Rapperswil
%

\section{Punktgruppen}
% Punktgruppen in der Ebene
\folie{6/punktgruppen/ebene.tex}
% semidirektes Produkt
\folie{6/punktgruppen/semidirekt.tex}
% Zyklische Gruppen/Drehgruppen um endliche Winkel
\folie{6/punktgruppen/c.tex}
% Diedergruppen
\folie{6/punktgruppen/d.tex}
% Tetraeder, Oktaeder, Ikosaeder
\folie{6/punktgruppen/p.tex}
% Anwendung Schrödingergleichung
\folie{6/punktgruppen/chemie.tex}
\folie{6/punktgruppen/aufspaltung.tex}

\section{Produkte}
% direktes Produkt
\folie{6/produkte/direkt.tex}

\section{Normalteiler}
% freie Gruppen
\folie{6/produkte/frei.tex}
% Normalteiler
\folie{6/normalteiler/normal.tex}
% Konjugationsklassen
\folie{6/normalteiler/konjugation.tex}

\section{Permutationsgruppen}
% Permutationen, Transpositionen, Signum
% Alternierende Gruppe
% Darstellung als Matrizen
\folie{6/permutationen/matrizen.tex}

\section{Darstellungen}
% Was ist eine Darstellung?
\folie{6/darstellungen/definition.tex}
% Charakter
\folie{6/darstellungen/charakter.tex}
% Summe
\folie{6/darstellungen/summe.tex}
% Irreduzible Darstellung
\folie{6/darstellungen/irreduzibel.tex}
% Folgerungen aus Schurs Lemma
\folie{6/darstellungen/schur.tex}
% Skalarprodukt
\folie{6/darstellungen/skalarprodukt.tex}
% Beispiel zyklische Gruppen
\folie{6/darstellungen/zyklisch.tex}
