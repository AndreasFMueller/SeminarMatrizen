%
% slides.tex -- XXX
%
% (c) 2017 Prof Dr Andreas Müller, Hochschule Rapperswil
%

\section{Punktgruppen}
% XXX Punktgruppen in der Ebene
%\folie{6/punktgruppen/ebene.tex}
% XXX semidirektes Produkt
%\folie{6/punktgruppen/semidirekt.tex}
% XXX Zyklische Gruppen/Drehgruppen um endliche Winkel
%\folie{6/punktgruppen/c.tex}
% XXX Diedergruppen
%\folie{6/punktgruppen/d.tex}
% XXX Tetraeder, Oktaeder, Ikosaeder
%\folie{6/punktgruppen/p.tex}
% XXX Anwendung Schrödingergleichung
\folie{6/punktgruppen/chemie.tex}
\folie{6/punktgruppen/aufspaltung.tex}

\section{Permutationsgruppen}
% XXX Permutationen, Transpositionen, Signum
% XXX Alternierende Gruppe
% Darstellung als Matrizen
%\folie{6/permutationen/matrizen.tex}

\section{Normalteiler}
% XXX Faktor
% XXX Konjugationsklassen

\section{Produkte}
% XXX direktes Produkt
% XXX freie Gruppen

\section{Darstellungen}
% Was ist eine Darstellung?
%\folie{6/darstellungen/definition.tex}
% Charakter
%\folie{6/darstellungen/charakter.tex}
% XXX Summe
%\folie{6/darstellungen/summe.tex}
% XXX Irreduzible Darstellung
%\folie{6/darstellungen/irreduzibel.tex}
% XXX Folgerungen aus Schurs Lemma
%\folie{6/darstellungen/schur.tex}
% XXX Skalarprodukt
%\folie{6/darstellungen/skalarprodukt.tex}
% XXX Beispiel zyklische Gruppen
%\folie{6/darstellungen/zyklisch.tex}
