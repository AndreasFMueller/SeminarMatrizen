%
% slides.tex -- XXX
%
% (c) 2017 Prof Dr Andreas Müller, Hochschule Rapperswil
%

\section{Hilbertraum}
% XXX Definition
\folie{2/hilbertraum/definition.tex}
% XXX Norm und Konvergenz
% XXX \folie{2/hilbertraum/norm.tex}
% XXX Hilbert-Basis
\folie{2/hilbertraum/l2beispiel.tex}
\folie{2/hilbertraum/basis.tex}
\folie{2/hilbertraum/plancherel.tex}

\section{Beispiele}
% XXX Endlichdimensionale euklidische Räume
% XXX \folie{2/hilbertraum/endlichdimensional.tex}
% XXX Fourier-Theorie und L^2
\folie{2/hilbertraum/l2.tex}

\section{Riesz-Darstellungssatz}
% XXX Was sagt der Satz
\folie{2/hilbertraum/riesz.tex}
\folie{2/hilbertraum/rieszbeispiel.tex}
% XXX Beweisidee
% XXX \folie{2/hilbertraum/rieszbeweis.tex}

\section{$A^*$}
% XXX Definition als Anwendung des Satzes von Riesz
\folie{2/hilbertraum/adjungiert.tex}
% XXX Spektraltheorie
\folie{2/hilbertraum/spektral.tex}

\section{PDE und Hilbertraum}
% XXX Der Operator D^2 + p(x) auf [0,1]
\folie{2/hilbertraum/sturm.tex}
% XXX Laplace-Operator und L^2
\folie{2/hilbertraum/laplace.tex}
\folie{2/hilbertraum/qm.tex}
\folie{2/hilbertraum/energie.tex}
\folie{2/hilbertraum/sobolev.tex}


