%
% normal.tex -- slide template
%
% (c) 2021 Prof Dr Andreas Müller, OST Ostschweizer Fachhochschule
%
\bgroup
\begin{frame}[t]
\setlength{\abovedisplayskip}{5pt}
\setlength{\belowdisplayskip}{5pt}
\frametitle{Normalteiler}
\vspace{-20pt}
\begin{columns}[t,onlytextwidth]
\begin{column}{0.48\textwidth}
\begin{block}{Gegeben}
Eine Gruppe $G$ mit Untergruppe $N\subset G$
\end{block}
\uncover<2->{%
\begin{block}{Bedingung}
Welche Eigenschaft muss $N$ zusätzlich haben,
damit 
\[
G/N
=
\{ gN \;|\; g\in G\}
\]
eine Gruppe wird.

\uncover<3->{Wähle Repräsentaten $g_1N=g_2N$}
\uncover<4->{%
\begin{align*}
g_1g_2N
&\uncover<5->{=
g_1g_2NN}
\uncover<6->{=
g_1g_2Ng_2^{-1}g_2N}
\\
&\uncover<7->{=
g_1(g_2Ng_2^{-1})g_2N}
\\
&\uncover<8->{\stackrel{?}{=} g_1Ng_2N}
\end{align*}}
\uncover<9->{Funktioniert nur wenn $g_2Ng_2^{-1}=N$ ist}
\end{block}}
\end{column}
\begin{column}{0.48\textwidth}
\uncover<10->{%
\begin{block}{Universelle Eigenschaft}
Ist $\varphi\colon G\to G'$ ein Homomorphismus mit $\varphi(N)=\{e\}$%
\uncover<11->{, dann gibt es einen Homomorphismus $G/N\to G'$:}
\begin{center}
\begin{tikzpicture}[>=latex,thick]
\coordinate (N) at (-2.5,0);
\coordinate (G) at (0,0);
\coordinate (quotient) at (2.5,0);
\coordinate (Gprime) at (0,-2.5);
\coordinate (e) at (-2.5,-2.5);
\node at (N) {$N$};
\node at (e) {$\{e\}$};
\node at (G) {$G$};
\node at (Gprime) {$G'$};
\node at (quotient) {$G/N$};
\draw[->,shorten >= 0.3cm,shorten <= 0.4cm] (N) -- (G);
\draw[->,shorten >= 0.3cm,shorten <= 0.4cm] (N) -- (e);
\draw[->,shorten >= 0.3cm,shorten <= 0.4cm] (e) -- (Gprime);
\draw[->,shorten >= 0.3cm,shorten <= 0.4cm] (G) -- (Gprime);
\draw[->,shorten >= 0.4cm,shorten <= 0.4cm] (G) -- (quotient);
\uncover<11->{
\draw[->,shorten >= 0.3cm,shorten <= 0.4cm,color=red] (quotient) -- (Gprime);
\node[color=red] at ($0.5*(quotient)+0.5*(Gprime)$) [below right] {$\exists!$};
}
\node at ($0.5*(quotient)$) [above] {$\pi$};
\node at ($0.5*(Gprime)$) [left] {$\varphi$};
\end{tikzpicture}
\end{center}
\end{block}}
\end{column}
\end{columns}
\end{frame}
\egroup
