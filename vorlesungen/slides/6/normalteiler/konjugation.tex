%
% konjugation.tex -- slide template
%
% (c) 2021 Prof Dr Andreas Müller, OST Ostschweizer Fachhochschule
%
\bgroup
\begin{frame}[t]
\setlength{\abovedisplayskip}{5pt}
\setlength{\belowdisplayskip}{5pt}
\frametitle{Konjugation}
\vspace{-20pt}
\begin{columns}[t,onlytextwidth]
\begin{column}{0.48\textwidth}
\begin{block}{``Basiswechsel''}
In der Gruppe $\operatorname{GL}_n(\Bbbk)$
\[
A' = TAT^{-1}
\]
$T\in\operatorname{GL}_n(\Bbbk)$
\\
$A$ und $A'$ sind ``gleichwertig''
\end{block}
\uncover<2->{%
\begin{block}{Definition}
$g_1,g_2\in G$ sind {\em konjugiert}, wenn es
$h\in G$ gibt mit
\[
g_1 = hg_2h^{-1}
\]
\end{block}}
\uncover<3->{%
\begin{block}{Beispiel}
Konjugierte Elemente in $\operatorname{GL}_n(\Bbbk)$ haben die
gleiche Spur und Determinante
\end{block}}
\end{column}
\begin{column}{0.48\textwidth}
\uncover<4->{%
\begin{block}{Konjugationsklasse}
Die Konjugationsklasse von $g$ ist
\[
\llbracket g\rrbracket
=
\{h\in G\;|\; \text{$h$ konjugiert zu $g$}\}
\]
\end{block}}
\vspace{-7pt}
\uncover<5->{%
\begin{block}{Klassenzerlegung}
\begin{align*}
G
&=
\{e\}
\cup
\llbracket g_1\rrbracket
\cup
\llbracket g_2\rrbracket
\cup
\dots
\\
&\uncover<6->{=
C_e\cup C_1 \cup C_2\cup\dots}
\end{align*}
\end{block}}
\vspace{-7pt}
\uncover<7->{%
\begin{block}{Klassenfunktionen}
Funktionen, die auf Konjugationsklassen konstant sind
\end{block}}
\uncover<8->{%
\begin{block}{Beispiele}
Spur, Determinante
\end{block}}
\end{column}
\end{columns}
\end{frame}
\egroup
