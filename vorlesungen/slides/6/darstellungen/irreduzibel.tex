%
% irreduzibel.tex -- slide template
%
% (c) 2021 Prof Dr Andreas Müller, OST Ostschweizer Fachhochschule
%
\bgroup
\begin{frame}[t]
\setlength{\abovedisplayskip}{5pt}
\setlength{\belowdisplayskip}{5pt}
\frametitle{Irreduzible Darstellungen}
\vspace{-20pt}
\begin{columns}[t,onlytextwidth]
\begin{column}{0.48\textwidth}
\begin{block}{Definition}
Eine Darstellung $\varrho\colon G\to\operatorname{GL}(V)$ heisst
irreduzibel, wenn es keine Zerlegung von $\varrho$ in zwei
Darstellungen $\varrho_i\colon G\to\operatorname{GL}(U_i)$ ($i=1,2$)
gibt derart, dass $\varrho = \varrho_1\oplus\varrho_2$
\end{block}
\uncover<2->{%
\begin{block}{Isomorphe Darstellungen}
$\varrho_i$ sind {\em isomorphe} Darstellungen in $V_i$ wenn es
$f\colon V_1\overset{\cong}{\to} V_2$ gibt mit
\begin{align*}
f \circ \varrho_i(g)\circ f^{-1} &= \varrho_2(g)
\\
\uncover<3->{%
f \circ \varrho_i(g)\phantom{\mathstrut\circ f^{-1}}&= \varrho_2(g)\circ f 
}
\end{align*}
\end{block}}
\end{column}
\begin{column}{0.48\textwidth}
\uncover<4->{%
\begin{block}{Lemma von Schur}
$\varrho_i$ zwei irreduzible Darstellungen und $f$ so, dass
$f\circ \varrho_1(g)=\varrho_2(g)\circ f$ für alle $g$.
Dann gilt
\begin{enumerate}
\item<5-> $\varrho_i$ nicht isomorph $\Rightarrow$ $f=0$
\item<6-> $V_1=V_2,\varrho_1=\varrho_2$ $\Rightarrow$ $f=\lambda I$
\end{enumerate}
\end{block}}
\end{column}
\end{columns}
\end{frame}
\egroup
