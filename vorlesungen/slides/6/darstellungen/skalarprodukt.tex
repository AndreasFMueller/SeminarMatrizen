%
% skalarprodukt.tex -- slide template
%
% (c) 2021 Prof Dr Andreas Müller, OST Ostschweizer Fachhochschule
%
\bgroup
\begin{frame}[t]
\setlength{\abovedisplayskip}{5pt}
\setlength{\belowdisplayskip}{5pt}
\frametitle{Skalarprodukt}
\vspace{-20pt}
\begin{columns}[t,onlytextwidth]
\begin{column}{0.48\textwidth}
\begin{block}{Definition des Skalarproduktes}
$\varphi$, $\psi$ komplexe Funktionen auf $G$:
\[
\langle \varphi,\psi\rangle
=
\frac{1}{|G|} \sum_{g\in G} \overline{\varphi(g)} \psi(g)
\]
\end{block}
\end{column}
\begin{column}{0.48\textwidth}
\uncover<2->{%
\begin{block}{Satz}
\begin{enumerate}
\item
$\chi$ der Charakter einer irrediziblen Darstellung
$\Rightarrow$ $\langle \chi,\chi\rangle=1$.
\item<3->
$\chi$ und $\chi'$ Charaktere nichtisomorpher Darstellungen
$\Rightarrow$
$\langle \chi,\chi'\rangle=0$
\end{enumerate}
\uncover<4->{%
D.~h.~Charaktere irreduzibler Darstellungen sind orthonormiert
}
\end{block}}
\end{column}
\end{columns}
\end{frame}
\egroup
