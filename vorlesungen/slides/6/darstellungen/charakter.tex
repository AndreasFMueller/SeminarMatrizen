%
% chrakter.tex -- Charakter einer Darstellung
%
% (c) 2021 Prof Dr Andreas Müller, OST Ostschweizer Fachhochschule
%
\bgroup
\begin{frame}[t]
\setlength{\abovedisplayskip}{5pt}
\setlength{\belowdisplayskip}{5pt}
\frametitle{Charakter einer Darstellung}
\vspace{-20pt}
\begin{columns}[t,onlytextwidth]
\begin{column}{0.44\textwidth}
\begin{block}{Definition}
$\varrho\colon G\to\operatorname{GL}_n(\mathbb{C})$ eine Darstellung.
\\
Der {\em Charakter} von $\varrho$ ist die Abbildung
\[
\chi_{\varrho}
\colon
G\to \mathbb{C}^n
:
g\mapsto \chi_{\varrho}(g)=\operatorname{Spur}\varrho(g)
\]
\end{block}
\uncover<2->{%
\begin{block}{Eigenschaften}
\begin{enumerate}
\item
$\chi_{\varrho}(e) = n$
\item<6->
$\chi_{\varrho}(g^{-1}) = \overline{\chi_{\varrho}(g)}$
\item<15->
$\chi_{\varrho}(hgh^{-1}) = \chi_{\varrho}(g)$
\end{enumerate}
\uncover<21->{%
Aus 3. folgt, dass Charaktere {\em Klassenfunktionen} sind}
\end{block}}
\end{column}
\begin{column}{0.52\textwidth}
\uncover<2->{%
\begin{block}{Begründung}
\begin{enumerate}
\item<3->
$\chi_{\varrho}(e)
=
\operatorname{Spur}\varrho(e)
\uncover<4->{=
\operatorname{Spur}I_n}
\uncover<5->{=
n}
$
\item<6->
$g$ hat endliche Ordnung, d.~h.~$g^k=e$
\\
\uncover<7->{%
$\lambda_i$ in der Jordan-NF erfüllen $\lambda_i^k=1$}
\\
$\uncover<8->{\Rightarrow|\lambda_i|=1}
\uncover<9->{\Rightarrow \lambda_i^{-1} = \overline{\lambda_i}}$
\begin{align*}
\uncover<10->{
\llap{$\chi_{\varrho}(g^{-1})$}
&=
\operatorname{Spur}(\varrho(g^{-1}))}
\uncover<11->{=
\sum_{i} n_i\overline{\lambda_i}}
\\[-4pt]
&\uncover<12->{=
\overline{
\sum_{i} n_i\lambda_i
}}
\uncover<13->{=
\operatorname{Spur}\varrho(g)}
\uncover<14->{=
\chi_{\varrho}(g)}
\end{align*}
\item<16->
Durch Nachrechnen:
\begin{align*}
\chi_{\varrho}(hgh^{-1})
&\uncover<17->{=
\operatorname{Spur}
(
\varrho(h)
\varrho(g)
\varrho(h^{-1})
)}
\\
&\uncover<18->{=
\operatorname{Spur}
(
\varrho(h^{-1})
\varrho(h)
\varrho(g)
)}
\\
&\uncover<19->{=
\operatorname{Spur}\varrho(g)}
\uncover<20->{=
\chi_{\varrho}(g)}
\end{align*}
\end{enumerate}
\end{block}}
\end{column}
\end{columns}
\end{frame}
\egroup
