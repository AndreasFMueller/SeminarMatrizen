%
% Summe.tex -- slide template
%
% (c) 2021 Prof Dr Andreas Müller, OST Ostschweizer Fachhochschule
%
\bgroup
\begin{frame}[t]
\setlength{\abovedisplayskip}{5pt}
\setlength{\belowdisplayskip}{5pt}
\frametitle{Direkte Summe}
\vspace{-20pt}
\begin{columns}[t,onlytextwidth]
\begin{column}{0.48\textwidth}
\begin{block}{Gegeben}
Gegeben zwei Darstellungen
\begin{align*}
\varrho_1&\colon G \to \mathbb{C}^{n_1}
\\
\varrho_2&\colon G \to \mathbb{C}^{n_2}
\end{align*}
\end{block}
\vspace{-12pt}
\uncover<2->{%
\begin{block}{Direkte Summe der Darstellungen}
%\vspace{-12pt}
\begin{align*}
\varrho_1\oplus\varrho_2
&\colon
G\to \mathbb{C}^{n_1+n_2}
\only<3|handout:0>{
= \mathbb{C}^{n_1}\times\mathbb{C}^{n_2}}
\uncover<4->{=:
\mathbb{C}^{n_1}\oplus\mathbb{C}^{n_2}}
\hspace*{5cm}
\\
&\colon g\mapsto (\varrho_1(g),\varrho_2(g))
\end{align*}
\end{block}}
\vspace{-12pt}
\uncover<5->{%
\begin{block}{Charakter}
%\vspace{-12pt}
\begin{align*}
\chi_{\varrho_1\oplus\varrho_2}(g)
&=
\operatorname{Spur}(\varrho_1\oplus\varrho_2)(g)
\\
&\uncover<6->{=
\operatorname{Spur}{\varrho_1(g)}
+
\operatorname{Spur}{\varrho_1(g)}}
\\
&\uncover<7->{=
\chi_{\varrho_1}(g)
+
\chi_{\varrho_2}(g)}
\end{align*}
\end{block}}
\end{column}
\begin{column}{0.48\textwidth}
\uncover<8->{%
\begin{block}{Tensorprodukt}
$n_1\times n_2$-dimensionale
Darstellung $\varrho_1\otimes\varrho_2$ mit Matrix
\[
\begin{pmatrix}
\varrho_1(g)_{11} \varrho_2(g)
	&\dots
		&\varrho_1(g)_{1n_1} \varrho_2(g)\\
\vdots&\ddots&\vdots\\
\varrho_1(g)_{n_11} \varrho_2(g)
	&\dots
		&\varrho_1(g)_{n_1n_1} \varrho_2(g)
\end{pmatrix}
\]
\uncover<9->{Die ``Einträge'' sind $n_2\times n_2$-Blöcke}
\end{block}}
\uncover<10->{%
\begin{block}{Darstellungsring}
Die Menge der Darstellungen $R(G)$  einer Gruppe hat
einer Ringstruktur mit $\oplus$ und $\otimes$
\\
\uncover<11->{$\Rightarrow$
Algebra zum Studium der möglichen Darstellungen von $G$ verwenden}
\end{block}}
\end{column}
\end{columns}
\end{frame}
\egroup
