%
% zyklisch.tex -- slide template
%
% (c) 2021 Prof Dr Andreas Müller, OST Ostschweizer Fachhochschule
%
\bgroup
\begin{frame}[t]
\setlength{\abovedisplayskip}{5pt}
\setlength{\belowdisplayskip}{5pt}
\frametitle{Beispiel: Zyklische Gruppen}
\vspace{-20pt}
\begin{columns}[t,onlytextwidth]
\begin{column}{0.48\textwidth}
\begin{block}{Gruppe}
\(
C_n = \mathbb{Z}/n\mathbb{Z}
\)
\end{block}
\uncover<2->{%
\begin{block}{Darstellungen von $C_n$}
Gegeben durch $\varrho_k(1)=e^{2\pi i k/n}$,
\[
\varrho_k(l) = e^{2\pi ikl/n}
\]
\end{block}}
\vspace{-10pt}
\uncover<3->{
\begin{block}{Charaktere}
%\vspace{-10pt}
\[
\chi_k(l) = e^{2\pi ikl/n}
\]
haben Skalarprodukte
\[
\langle \chi_k,\chi_{k'}\rangle
=
\begin{cases}
1&\quad k= k'\\
0&\quad\text{sonst}
\end{cases}
\]
Die Darstellungen $\chi_k$ sind nicht isomorph
\end{block}}
\end{column}
\begin{column}{0.48\textwidth}
\uncover<5->{%
\begin{block}{Orthonormalbasis}
Die Funktionen $\chi_k$ bilden eine Orthonormalbasis von $L^2(C_n)$
\end{block}}
\vspace{-4pt}
\uncover<6->{%
\begin{block}{Analyse einer Darstellung}
$\varrho\colon C_n\to \mathbb{C}^n$ eine Darstellung,
$\chi_\varrho$ der Charakter lässt zerlegen:
\begin{align*}
c_k
&=
\langle \chi_k, \chi\rangle = \frac{1}{n} \sum_{l} \chi_k(l) e^{-2\pi ilk/n}
\\
\uncover<7->{
\chi(l)
&=
\sum_{k} c_k \chi_k
=
\sum_{k} c_k e^{2\pi ikl/n}
}
\end{align*}
\end{block}}
\vspace{-13pt}
\uncover<8->{%
\begin{block}{Fourier-Theorie}
\vspace{-3pt}
\begin{center}
\begin{tabular}{>{$}l<{$}l}
\uncover<9->{C_n&Diskrete Fourier-Theorie}\\
\uncover<10->{U(1)&Fourier-Reihen}\\
\uncover<11->{\mathbb{R}&Fourier-Integral}
\end{tabular}
\end{center}
\end{block}}
\end{column}
\end{columns}
\end{frame}
\egroup
