%
% definition.tex -- Definition einer Darstellung
%
% (c) 2021 Prof Dr Andreas Müller, OST Ostschweizer Fachhochschule
%
\bgroup
\begin{frame}[t]
\setlength{\abovedisplayskip}{5pt}
\setlength{\belowdisplayskip}{5pt}
\frametitle{Darstellung}
\vspace{-20pt}
\begin{columns}[t,onlytextwidth]
\begin{column}{0.48\textwidth}
\begin{block}{Definition}
$G$ eine Gruppe, $V$ ein $\Bbbk$-Vektorraum.
\\
\uncover<2->{%
Ein Homomorphismus
\[
\varrho
\colon
G\to \operatorname{GL}(V)
\]
heisst {\em $n$-dimensionale Darstellung} der Gruppe $G$.}
\end{block}
\uncover<3->{%
\begin{block}{Idee}
Algebra und Analysis in $\operatorname{GL}_n(\Bbbk)$ nutzen, um
mehr über $G$ herauszufinden
\end{block}}
\end{column}
\begin{column}{0.48\textwidth}
\uncover<4->{%
\begin{block}{Beispiel $S_n$}
$S_n$ die symmetrische Gruppe,
$\sigma\mapsto A_{\tilde{f}}$ die
Abbildung auf die zugehörige Permutationsmatrix
ist eine $n$-dimensionale Darstellung von $S_n$
\end{block}}
\uncover<5->{%
\begin{block}{Beispiel Matrizengruppe}
Eine Matrizengruppe $G$ ist eine Teilmenge von $M_n(\Bbbk)$.
\\
\uncover<6->{%
$g\in G \Rightarrow g^{-1}\in G$, daher $G\subset\operatorname{GL}_n(\Bbbk)$}
\\
\uncover<7->{%
Die Einbettung
\[
G\to\operatorname{GL}_n(\Bbbk)
:
g \mapsto g
\]
ist eine Darstellung}\uncover<8->{, die sog.~{\em reguläre Darstellung}}
\end{block}}
\end{column}
\end{columns}
\end{frame}
\egroup
