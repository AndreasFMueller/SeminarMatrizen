%
% direkt.tex -- slide template
%
% (c) 2021 Prof Dr Andreas Müller, OST Ostschweizer Fachhochschule
%
\bgroup
\begin{frame}[t]
\setlength{\abovedisplayskip}{5pt}
\setlength{\belowdisplayskip}{5pt}
\frametitle{Direktes Produkt}
\vspace{-20pt}
\begin{columns}[t,onlytextwidth]
\begin{column}{0.48\textwidth}
\begin{block}{Definition}
Zwei Gruppen $H_1$ und $H_2$
\\
Gruppe $G=H_1\times H_2$ mit
\begin{itemize}
\item<2-> Elemente  $(h_1,h_2)\in H_1\times H_2$
\item<3-> Neutrales Element $(e_1,e_2)$
\item<4-> Inverses Elemente $(h_1,h_2)^{-1}=(h_1^{-1},h_2^{-1})$
\end{itemize}
heisst {\em direktes Produkt}
\end{block}
\uncover<5->{%
\begin{block}{Vertauschbarkeit}
Das direkte Produkt ist ein Produkt, in dem Elemente von $H_1$ und
$H_2$ vollständig vertauschbar sind
\end{block}}
\end{column}
\begin{column}{0.48\textwidth}
\uncover<6->{%
\begin{block}{Universelle Eigenschaft}
\begin{center}
\begin{tikzpicture}[>=latex,thick]
\coordinate (S) at (0,2.5);
\coordinate (H1) at (-2.5,0);
\coordinate (H2) at (2.5,0);

\node at (H1) {$H_1$};
\node at (H2) {$H_2$};
\node at (0,0) {$H_1\times H_2$};
\node at (S) {$S$};

\draw[->,shorten >= 0.25cm,shorten <= 0.8cm] (0,0) -- (H1);
\draw[->,shorten >= 0.25cm,shorten <= 0.8cm] (0,0) -- (H2);

\draw[->,shorten >= 0.25cm,shorten <= 0.25cm] (S) -- (H1);
\draw[->,shorten >= 0.25cm,shorten <= 0.25cm] (S) -- (H2);

\node at ($0.5*(S)+0.5*(H1)$) [above left] {$f_1$};
\node at ($0.5*(S)+0.5*(H2)$) [above right] {$f_2$};

\uncover<7->{
\draw[->,shorten >= 0.25cm,shorten <= 0.25cm,color=red] (S) -- (0,0);
\node[color=red] at ($0.36*(S)$) [left] {$f_1\times f_2$};
\node[color=red] at ($0.36*(S)$) [right] {$\exists!$};
}

\end{tikzpicture}
\end{center}
\end{block}}
\end{column}
\end{columns}
\end{frame}
\egroup
