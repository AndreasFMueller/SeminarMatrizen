%
% template.tex -- slide template
%
% (c) 2021 Prof Dr Andreas Müller, OST Ostschweizer Fachhochschule
%
\bgroup
\begin{frame}[t]
\setlength{\abovedisplayskip}{5pt}
\setlength{\belowdisplayskip}{5pt}
\frametitle{Freie Gruppen}
\vspace{-20pt}
\begin{columns}[t,onlytextwidth]
\begin{column}{0.48\textwidth}
\begin{block}{Gruppe aus Symbolen}
Erzeugende Elemente $\{a,b,c,\dots\}$
\\
\uncover<2->{%
Wörter = 
Folgen von Symbolen $a$, $a^{-1}$, $b$, $b^{-1}$}
\\
\uncover<3->{
{\em freie Gruppe}:
\begin{align*}
F&=\langle a,b,c,\dots\rangle
\\
&=
\{\text{Wörter}\}
/\text{Kürzungsregel}
\end{align*}}
\vspace{-10pt}
\begin{itemize}
\item<4-> neutrales Element: $e = \text{leere Symbolfolge}$
\item<5-> Gruppenoperation: Verkettung
\item<6-> Kürzungsregel:
\begin{align*}
xx^{-1}&\to e,
&
x^{-1}x&\to e
\end{align*}
\end{itemize}
\end{block}
\end{column}
\begin{column}{0.48\textwidth}
\uncover<7->{%
\begin{block}{Universelle Eigenschaft}
$g_i\in G$, dann gibt es genau einen Homomorphismus
\[
\varphi
\colon
\langle g_i| 1\le i\le k\rangle
\to
G
\]
\end{block}}
\vspace{-10pt}
\uncover<8->{%
\begin{block}{Quotient einer freien Gruppe}
Jede endliche Gruppe ist Quotient einer freien Gruppe
\[
N
\xhookrightarrow{}
\langle g_i\rangle
\twoheadrightarrow
G
\]
oder
\[
G = \langle g_i\rangle / N
\]
\end{block}}
\vspace{-10pt}
\uncover<11->{%
\begin{block}{Maximal nichtkommutativ}
Die freie Gruppe ist die ``maximal nichtkommutative'' Gruppe
\end{block}}
\end{column}
\end{columns}
\end{frame}
\egroup
