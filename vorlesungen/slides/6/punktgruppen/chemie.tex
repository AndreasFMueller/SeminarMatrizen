%
% chemie.tex -- Anwendung
%
% (c) 2021 Prof Dr Andreas Müller, OST Ostschweizer Fachhochschule
%
\bgroup
\begin{frame}[t]
\setlength{\abovedisplayskip}{5pt}
\setlength{\belowdisplayskip}{5pt}
\frametitle{Anwendung: Energieniveaus eines Atoms}
\vspace{-20pt}
\begin{columns}[t,onlytextwidth]
\begin{column}{0.48\textwidth}
\begin{block}{Schrödingergleichung}
Partielle Differentialgleichung für die Wellenfunktion
eines Teilchens im Potential $V(x)$
\[
-\frac{\hbar^2}{2m}\Delta \Psi
+
V(x)\Psi
=
E\Psi
\]
$V(x)$ = Potential der Atomkerne eines Molekuls
\end{block}
\uncover<2->{%
\begin{block}{Symmetrien}
$g\in\operatorname{O}(3)$  wirkt auf $V$ und $\Psi$
\begin{align*}
(g\cdot V)(x) &= V(g\cdot x)
\\
(g\cdot \Psi)(x) &= \Psi(g\cdot x)
\end{align*}
Symmetrie von $V$: $g\cdot V=V$
\end{block}}
\end{column}
\begin{column}{0.48\textwidth}
\uncover<3->{%
\begin{block}{Lösungen}
Eigenfunktionen $\Psi$ zum Eigenwert $E$
\[
g\cdot V=V
\Rightarrow
g\cdot \Psi
\text{ Lösung}
\]
mit gleichem Eigenwert!
\end{block}}
\uncover<4->{%
\begin{block}{Eigenräume}
Die Symmetriegruppe $G\subset \operatorname{O}(3)$ eines Moleküls
operiert auf dem Eigenraum
\end{block}}
\uncover<5->{%
\begin{block}{Externe Felder}
Externe Felder zerstören die Symmetrie
$\Rightarrow$
die Energieniveaus/Spektrallinien spalten sich auf
\end{block}}
\end{column}
\end{columns}
\end{frame}
\egroup
