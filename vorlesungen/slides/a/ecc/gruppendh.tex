%
% template.tex -- slide template
%
% (c) 2021 Prof Dr Andreas Müller, OST Ostschweizer Fachhochschule
%
\bgroup
\begin{frame}[t]
\setlength{\abovedisplayskip}{5pt}
\setlength{\belowdisplayskip}{5pt}
\frametitle{Diffie-Hellmann verallgemeinern}
\vspace{-20pt}
\begin{columns}[t,onlytextwidth]
\begin{column}{0.48\textwidth}
\begin{block}{Diffie-Hellman in $\mathbb{F}_p$\strut}
\begin{enumerate}
\item<2-> Parteien einigen sich auf $g\in \mathbb{F}_p$, $g\ne 0$, $g\ne 1$
\item<3-> $A$ und $B$ wählen Exponenten $a,b\in \mathbb{N}$
\item<4-> Parteien tauschen $u=g^a$ und $v=g^b$ aus
\item<5-> Parteien berechnen $v^a$ und $u^b$
\[
v^a = (g^b)^a = g^{ab} =(g^a)^b = u^b
\]
gemeinsamer privater Schlüssel
\end{enumerate}
\end{block}
\uncover<11->{%
{\usebeamercolor[fg]{title}Spezialfall:} $G=\mathbb{F}_p^*$
}
\end{column}
\begin{column}{0.48\textwidth}
\uncover<6->{%
\begin{block}{Diffie-Hellmann in $G$\strut}
\begin{enumerate}
\item<7-> Parteien einigen sich auf $g\in G$, $g\ne e$
\item<8-> $A$ und $B$ wählen Exponenten $a,b\in \mathbb{N}$
\item<9-> Parteien tauschen $u=g^a$ und $v=g^b$ aus
\item<10-> Parteien berechnen $v^a$ und $u^b$
\[
v^a = (g^b)^a = g^{ab} =(g^a)^b = u^b
\]
gemeinsamer privater Schlüssel
\end{enumerate}
\end{block}}
\uncover<12->{%
{\usebeamercolor[fg]{title}Idee:} Wähle effizient zu berechnende, ``grosse''
Gruppen, mit ``komplizierter'' Multiplikation
}
\end{column}
\end{columns}
\end{frame}
\egroup
