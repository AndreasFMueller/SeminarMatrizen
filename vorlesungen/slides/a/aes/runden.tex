%
% runden.tex -- slide template
%
% (c) 2021 Prof Dr Andreas Müller, OST Ostschweizer Fachhochschule
%
\bgroup
\begin{frame}[t]
\setlength{\abovedisplayskip}{5pt}
\setlength{\belowdisplayskip}{5pt}
\frametitle{$n$ Runden}
\vspace{-23pt}
\begin{columns}[t,onlytextwidth]
\begin{column}{0.48\textwidth}
\begin{block}{Verschlüsselung}
In Runde $i=0,\dots,n-1$
\begin{enumerate}
\item<2-> Wende die $S$-Box auf alle Bytes des Blocks an
\item<3-> Führe den Zeilenschift durch
\item<4-> Mische die Spalten
\item<5-> Berechne den Schlüsselblock $i$  ($i=0$: ursprünglicher Schlüssel)
\item<6-> Addiere (XOR) den Rundenschlüssel
\end{enumerate}
\end{block}
\end{column}
\begin{column}{0.48\textwidth}
\uncover<7->{%
\begin{block}{Entschlüsselung}
In Runde $i=0,\dots,n-1$
\begin{enumerate}
\item<8-> Addiere den Rundenschlüssel $n-1-i$
\item<9-> Invertiere Spaltenmischung (mit $C^{-1}$)
\item<10-> Invertiere den Zeilenshift
\item<11-> Wende $S^{-1}$ an auf jedes Byte
\end{enumerate}
\end{block}}
\end{column}
\end{columns}
\uncover<12->{%
\begin{block}{Charakteristika}
\begin{itemize}
\item<13-> Invertierbar
\item<14-> Skalierbar: beliebig grosse Blöcke (Vielfache von 32\,bit)
\item<15-> Keine ``magischen'' Schritte
\end{itemize}
\end{block}}
\end{frame}
\egroup
