%
% beispiel.tex -- slide template
%
% (c) 2021 Prof Dr Andreas Müller, OST Ostschweizer Fachhochschule
%
\bgroup
\def\u#1#2{\uncover<#1->{#2}}
\begin{frame}[t]
\setlength{\abovedisplayskip}{5pt}
\setlength{\belowdisplayskip}{5pt}
\frametitle{Beispiel}
\vspace{-20pt}
\begin{columns}[t,onlytextwidth]
\begin{column}{0.48\textwidth}
\begin{block}{Aufgabe}
Berechne $1291^{17}\in\mathbb{F}_{2027}$
\end{block}
\uncover<2->{%
\begin{block}{Exponent}
\vspace{-10pt}
\[
17 = 2^4 + 1
=
\texttt{10001}_2
=
\texttt{0x11}
\]
\end{block}}
\end{column}
\begin{column}{0.48\textwidth}
\uncover<3->{%
\begin{block}{Divide-and-Conquor}
\begin{center}
\begin{tabular}{|>{$}r<{$}>{$}r<{$}|>{$}r<{$}|>{$}r<{$}|>{$}r<{$}|>{$}r<{$}|}
\hline
i&2^i&    a^{2^i} &       n &     n_i &                      m \\
\hline
0&  1&       1291 &      17 & \u{4}{1}&\u{5}{             1291}\\
1&  2& \u{6}{ 487}& \u{7}{8}& \u{8}{0}& \u{9}{\color{gray}1291}\\
2&  4&\u{10}{  10}&\u{11}{4}&\u{12}{0}&\u{13}{\color{gray}1291}\\
3&  8&\u{14}{ 100}&\u{15}{2}&\u{16}{0}&\u{17}{\color{gray}1291}\\
4& 16&\u{18}{1892}&\u{19}{1}&\u{20}{1}&\u{21}{              37}\\
\hline
\end{tabular}
\end{center}
\end{block}}
\uncover<22->{%
\begin{block}{Resultat}
\(1291^{17} \equiv 37\mod 2027\)
\end{block}}
\end{column}
\end{columns}
\end{frame}
\egroup
