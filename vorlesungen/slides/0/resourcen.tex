%
% resourcen.tex
%
% (c) 2020 Prof Dr Andreas Müller, Hochschule Rapperswil
%
\begin{frame}[t]
\frametitle{Resourcen}
\begin{columns}[t,onlytextwidth]
\begin{column}{0.48\textwidth}
\begin{block}{Moodle Modul MathSem}
\begin{enumerate}
\item<2-> Skript
\begin{itemize}
\item<3-> Aktuellste Version in Github
\item<4-> regelmässige Updates in Moodle: \texttt{buch.pdf}
\end{itemize}
\item<5-> Informationen zur Planung: Kurztests, Vorträge
\item<6-> Anleitung für die Seminararbeit
\item<7-> Aufgabenstellungen
\end{enumerate}
\end{block}
\end{column}
\begin{column}{0.48\textwidth}
\uncover<8->{%
\begin{block}{Weitere Quellen}
\begin{enumerate}
\item<9-> Zusätzliche Literaturhinweise in der Aufgabenbeschreibung im Moodle
\item<10-> Bibliothek
\item<11-> Google
\item<12-> Google Scholar
\item<13-> Paper ist nicht öffentlich zugänglich? $\rightarrow$ kann via
Bibliothek organisiert werden
\end{enumerate}
\end{block}}
\end{column}
\end{columns}
\end{frame}
