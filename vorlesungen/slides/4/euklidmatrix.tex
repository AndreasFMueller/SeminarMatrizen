%
% euklidmatrix.tex
%
% (c) 2021 Prof Dr Andreas Müller, OST Ostscheizer Fachhochschule
%
\begin{frame}[t]
\frametitle{Matrixform des euklidischen Algorithmus}
\setlength{\abovedisplayskip}{5pt}
\setlength{\belowdisplayskip}{5pt}
\vspace{-20pt}
\begin{columns}[t,onlytextwidth]
\begin{column}{0.52\textwidth}
\begin{block}{Einzelschritt}
\vspace{-10pt}
\[
a_k = b_kq_k + r_k
\uncover<2->{
\;\Rightarrow\;
\left\{
\begin{aligned}
a_{k+1} &= b_k = \phantom{a_k-q_k}b_k \\
b_{k+1} &= \phantom{b_k}\llap{$r_k$} = a_k - q_kb_k 
\end{aligned}
\right.}
\]
\end{block}
\end{column}
\begin{column}{0.44\textwidth}
\uncover<3->{%
\begin{block}{Matrixschreibweise}
\vspace{-10pt}
\begin{align*}
\begin{pmatrix}
a_{k+1}\\
b_{k+1}
\end{pmatrix}
&=
\begin{pmatrix}
b_k\\r_k
\end{pmatrix}
=
\uncover<4->{
\underbrace{\begin{pmatrix}
\uncover<5->{0&1}\\
\uncover<6->{1&-q_k}
\end{pmatrix}}_{\uncover<7->{\displaystyle =Q(q_k)}}
}
\begin{pmatrix}
a_k\\b_k
\end{pmatrix}
\end{align*}
\end{block}}
\end{column}
\end{columns}
\vspace{-10pt}
\uncover<8->{%
\begin{block}{Ende des Algorithmus}
\vspace{-10pt}
\begin{align*}
\uncover<9->{
\begin{pmatrix}
a_{n+1}\\
b_{n+1}\\
\end{pmatrix}
&=}
\begin{pmatrix}
r_{n-1}\\
r_{n}
\end{pmatrix}
=
\begin{pmatrix}
\operatorname{ggT}(a,b) \\
0
\end{pmatrix}
\uncover<11->{
=
\underbrace{\uncover<15->{Q(q_n)}
\uncover<14->{\dots}
\uncover<13->{Q(q_1)}
\uncover<12->{Q(q_0)}}_{\displaystyle =Q}}
\uncover<10->{
\begin{pmatrix} a_0\\ b_0\end{pmatrix}
\uncover<6->{
=
Q\begin{pmatrix}a\\b\end{pmatrix}
}
}
\end{align*}
\end{block}}
\uncover<16->{%
\begin{block}{Konsequenzen}
\[
Q=\begin{pmatrix}
q_{11}&q_{12}\\
q_{21}&q_{22}
\end{pmatrix}
\quad\Rightarrow\quad
\left\{
\quad
\begin{aligned}
\operatorname{ggT}(a,b) &= q_{11}a + q_{12}b = {\color{red}s}a+{\color{red}t}b\\
                      0 &= q_{21}a + q_{22}b
\end{aligned}
\right.
\]
\end{block}}

\end{frame}
