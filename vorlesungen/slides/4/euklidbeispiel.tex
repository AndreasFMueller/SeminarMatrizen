%
% euklidmatrix.tex
%
% (c) 2021 Prof Dr Andreas Müller, OST Ostscheizer Fachhochschule
%
\bgroup
\definecolor{darkgreen}{rgb}{0,0.6,0}
\begin{frame}[t]
\frametitle{Beispiel}
\setlength{\abovedisplayskip}{0pt}
\setlength{\belowdisplayskip}{0pt}
\vspace{-0pt}
\begin{block}{Finde $\operatorname{ggT}(25,15)$}
\vspace{-12pt}
\begin{align*}
a_0&=25 & b_0 &= 15 &25&=15 \cdot {\color{red}      1} + 10 &q_0 &= {\color{red}1} & r_0 &= 10\\
a_1&=15 & b_1 &= 10 &15&=10 \cdot {\color{darkgreen}1} + \phantom{0}5  &q_1 &= {\color{darkgreen}1} & r_1 &= \phantom{0}5 \\
a_2&=10 & b_2 &= \phantom{0}5 &10&=\phantom{0}5  \cdot {\color{blue}     2} + \phantom{0}0  &q_2 &= {\color{blue}2} & r_2 &= \phantom{0}0 
\end{align*}
\end{block}
\vspace{-5pt}
\begin{block}{Matrix-Operationen}
\begin{align*}
Q
&=
Q({\color{blue}2}) Q({\color{darkgreen}1}) Q({\color{red}1})
=
\begin{pmatrix}0&1\\1&-{\color{blue}2}\end{pmatrix}
\begin{pmatrix}0&1\\1&-{\color{darkgreen}1}\end{pmatrix}
\begin{pmatrix}0&1\\1&-{\color{red}1}\end{pmatrix}
=\begin{pmatrix}
-1&2\\3&-5
\end{pmatrix}
\end{align*}
\end{block}
\vspace{-5pt}
\begin{block}{Relationen ablesen}
\begin{align*}
\operatorname{ggT}({\usebeamercolor[fg]{title}25},{\usebeamercolor[fg]{title}15}) &= 5 = -1\cdot {\usebeamercolor[fg]{title}25} + 2\cdot {\usebeamercolor[fg]{title}15} \\
 0                        &= \phantom{5=-}3\cdot {\usebeamercolor[fg]{title}25} -5\cdot {\usebeamercolor[fg]{title}15}
\end{align*}
\end{block}

\end{frame}
