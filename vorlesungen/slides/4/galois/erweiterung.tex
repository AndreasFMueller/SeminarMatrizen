%
% erweiterung.tex
%
% (c) 2021 Prof Dr Andreas Müller, OST Ostschweizer Fachhochschule
%
\begin{frame}[t]
\setlength{\abovedisplayskip}{5pt}
\setlength{\belowdisplayskip}{5pt}
\frametitle{Körpererweiterungen}
\vspace{-20pt}
\begin{columns}[t,onlytextwidth]
\begin{column}{0.48\textwidth}
\begin{block}{Körpererweiterung}
$E,F$ Körper: $E\subset F$
\end{block}
\uncover<6->{%
\begin{block}{Vektorraum}
$F$ ist ein Vektorraum über $E$
\end{block}}
\uncover<7->{%
\begin{block}{Endliche Körpererweiterung}
$\dim_E F < \infty$
\end{block}}
\uncover<8->{%
\begin{block}{Adjunktion eines $\alpha$}
$\Bbbk(\alpha)$ kleinster Körper, der $\Bbbk$ und
$\alpha$ enthält.
\end{block}}
\uncover<9->{%
\begin{block}{Algebraische Erweiterung}
$\alpha$ algebraisch über $\Bbbk$, i.~e.~Nullstelle von
$m(X)\in\Bbbk[X]$
\end{block}}
\end{column}
\begin{column}{0.48\textwidth}
\uncover<2->{%
\begin{block}{Beispiele}
\begin{enumerate}
\item<3->
$\mathbb{R} \subset \mathbb{R}(i) = \mathbb{C}$
\item<4->
$\mathbb{Q}\subset \mathbb{Q}(\sqrt{2})$
\item<5->
$\mathbb{Q} \subset \mathbb{Q}(\sqrt{2}) \subset \mathbb{Q}(\sqrt[4]{2})$
\end{enumerate}
\end{block}}
\uncover<7->{%
\begin{block}{Grad}
$E\subset F$ heisst Körpererweiterung vom Grad $n$, falls
\[
\dim_E F  = n =: [F:E]
\]
\uncover<8->{%
Gleichbedeutend: $\deg m(X) = n$}
\uncover<10->{%
\[
E\subset F\subset G
\Rightarrow
[G:E] = [G:F]\cdot [F:E]
\]
(in unseren Fällen)}
\end{block}}
\end{column}
\end{columns}
\end{frame}
