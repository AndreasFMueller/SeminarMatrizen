%
% aufloesbarkeit.tex
%
% (c) 2021 Prof Dr Andreas Müller, OST Ostschweizer Fachhochschule
%
\begin{frame}[t]
\setlength{\abovedisplayskip}{5pt}
\setlength{\belowdisplayskip}{5pt}
\frametitle{Auflösbarkeit}
\vspace{-20pt}
\begin{columns}[t,onlytextwidth]
\begin{column}{0.48\textwidth}
\uncover<2->{%
\begin{block}{Radikalerweiterung}
Automorphismen $f\in \operatorname{Gal}(\Bbbk(\alpha)/\Bbbk)$
einer Radikalerweiterung
\[
\Bbbk \subset \Bbbk(\alpha)
\]
sind festgelegt durch Wahl von $f(\alpha)$.

\begin{itemize}
\item<3-> Warum: Alle $f(\alpha^k)$ sind auch festgelegt
\item<4-> $f(\alpha)$ muss eine andere Nullstelle des Minimalpolynoms sein
\end{itemize}

\end{block}}
\uncover<8->{%
\begin{block}{Irreduzibles Polynom $m(X)\in\mathbb{Q}[X]$}
$\mathbb{Q}\subset \Bbbk$,
$n$ verschiedene Nullstellen $\mathbb{C}$:
\[
\uncover<9->{
\operatorname{Gal}(\Bbbk/\mathbb{Q})
\cong
S_n}
\uncover<10->{
\quad
\text{auflösbar?}}
\]
\end{block}}
\end{column}
\begin{column}{0.48\textwidth}
\begin{block}{\uncover<5->{Galois-Gruppen}}
\begin{center}
\begin{tikzpicture}[>=latex,thick]
\def\s{1.2}

\uncover<2->{
\fill[color=blue!20] (-1.1,-0.3)  rectangle (0.3,{5*\s+0.3});
\node[color=blue] at (-0.7,{2.5*\s}) [rotate=90] {Radikalerweiterungen};
}

\node at (0,0) {$\mathbb{Q}$};
\node at (0,{1*\s}) {$E_1$};
\node at (0,{2*\s}) {$E_2$};
\node at (0,{3*\s}) {$E_3$};
\node at (0,{4*\s}) {$\vdots\mathstrut$};
\node at (0,{5*\s}) {$\Bbbk$};
\draw[shorten >= 0.3cm,shorten <= 0.3cm] (0,{0*\s}) -- (0,{1*\s});
\draw[shorten >= 0.3cm,shorten <= 0.3cm] (0,{1*\s}) -- (0,{2*\s});
\draw[shorten >= 0.3cm,shorten <= 0.3cm] (0,{2*\s}) -- (0,{3*\s});
\draw[shorten >= 0.3cm,shorten <= 0.3cm] (0,{3*\s}) -- (0,{4*\s});
\draw[shorten >= 0.3cm,shorten <= 0.3cm] (0,{4*\s}) -- (0,{5*\s});

\begin{scope}[xshift=0.5cm]
\uncover<7->{
\fill[color=red!20] (0,{0*\s-0.3}) rectangle (4.8,{5*\s+0.3});
\node[color=red] at (4.5,{2.5*\s}) [rotate=90] {Auflösung der Galois-Gruppe};
}
\uncover<5->{
\node at (0,{0*\s}) [right] {$\operatorname{Gal}(\Bbbk/\mathbb{Q})$};
\node at (0,{1*\s}) [right] {$\operatorname{Gal}(\Bbbk/E_1)$};
\node at (0,{2*\s}) [right] {$\operatorname{Gal}(\Bbbk/E_2)$};
\node at (0,{3*\s}) [right] {$\operatorname{Gal}(\Bbbk/E_3)$};
\node at (1,{4*\s}) {$\vdots\mathstrut$};
\node at (0,{5*\s}) [right] {$\operatorname{Gal}(\Bbbk/\Bbbk)$};
\node at (1,{0.5*\s}) {$\cap\mathstrut$};
\node at (1,{1.5*\s}) {$\cap\mathstrut$};
\node at (1,{2.5*\s}) {$\cap\mathstrut$};
\node at (1,{3.5*\s}) {$\cap\mathstrut$};
\node at (1,{4.5*\s}) {$\cap\mathstrut$};
}

\uncover<6->{
\begin{scope}[xshift=2.5cm]
\node at (0,{0*\s}) {$G_n$};
\node at (0,{1*\s}) {$G_{n-1}$};
\node at (0,{2*\s}) {$G_{n-2}$};
\node at (0,{3*\s}) {$G_{n-3}$};
\node at (0,{5*\s}) {$G_0=\{e\}$};
\node at (0,{0.5*\s}) {$\cap\mathstrut$};
\node at (0,{1.5*\s}) {$\cap\mathstrut$};
\node at (0,{2.5*\s}) {$\cap\mathstrut$};
\node at (0,{3.5*\s}) {$\cap\mathstrut$};
\node at (0,{4.5*\s}) {$\cap\mathstrut$};
}

\uncover<7->{
\node[color=red] at (0.2,{0.5*\s+0.1}) [right] {\tiny $G_n/G_{n-1}$};
\node[color=red] at (0.2,{0.5*\s-0.1}) [right] {\tiny abelsch};

\node[color=red] at (0.2,{1.5*\s+0.1}) [right] {\tiny $G_{n-1}/G_{n-2}$};
\node[color=red] at (0.2,{1.5*\s-0.1}) [right] {\tiny abelsch};

\node[color=red] at (0.2,{2.5*\s+0.1}) [right] {\tiny $G_{n-2}/G_{n-3}$};
\node[color=red] at (0.2,{2.5*\s-0.1}) [right] {\tiny abelsch};
}

\end{scope}
\end{scope}



\end{tikzpicture}
\end{center}
\end{block}
\end{column}
\end{columns}
\end{frame}
