%
% radikale.tex
%
% (c) 2021 Prof Dr Andreas Müller, OST Ostschweizer Fachhochschule
%
\begin{frame}[t]
\setlength{\abovedisplayskip}{5pt}
\setlength{\belowdisplayskip}{5pt}
\frametitle{Lösung durch Radikale}
\vspace{-20pt}
\begin{columns}[t,onlytextwidth]
\begin{column}{0.48\textwidth}
\begin{block}{Problemstellung}
Finde Nullstellen eines Polynomes
\[
p(X)
=
a_nX^n + a_{n-1}X^{n-1}
+\dots+
a_1X+a_0
\]
$p\in\mathbb{Q}[X]$
\end{block}
\uncover<2->{%
\begin{block}{Radikale}
Geschachtelte Wurzelausdrücke
\[
\sqrt[3]{
-\frac{q}2 +\sqrt{\frac{q^2}{4}+\frac{p^3}{27}}
}
+
\sqrt[3]{
-\frac{q}2 -\sqrt{\frac{q^2}{4}+\frac{p^3}{27}}
}
\]
\uncover<3->{(Lösung von $x^3+px+q=0$)}
\end{block}}
\uncover<4->{%
\begin{block}{Lösbar durch Radikale}
Nullstelle von $p(X)$ ist ein Radikal
\end{block}}
\end{column}
\begin{column}{0.48\textwidth}
\uncover<5->{%
\begin{block}{Algebraische Formulierung}
Gegeben ein irreduzibles Polynom $p\in\mathbb{Q}[X]$,
finde eine Körpererweiterung $\mathbb{Q}\subset\Bbbk$, derart,
dass $p$ in $\Bbbk$ eine Nullstelle hat\uncover<6->{:
$\Bbbk = \mathbb{Q}[X]/(p)$}
\end{block}}
\uncover<7->{%
\begin{block}{Radikalerweiterung}
Körpererweiterung $\Bbbk\subset\Bbbk'$ um $\alpha$ mit einer der Eigenschaften
\begin{itemize}
\item<8-> $\alpha$ ist eine Einheitswurzel
\item<9-> $\alpha^k\in\Bbbk$
\end{itemize}
\end{block}}
\vspace{-5pt}
\uncover<10->{%
\begin{block}{Lösbar durch Radikale}
Radikalerweiterungen
\[
\mathbb{Q} \subset \Bbbk \subset \Bbbk' \subset \dots \subset \Bbbk'' \ni \alpha
\]
\end{block}}
\end{column}
\end{columns}
\end{frame}
