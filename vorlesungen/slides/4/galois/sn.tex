%
% sn.tex
%
% (c) 2021 Prof Dr Andreas Müller, OST Ostschweizer Fachhochschule
%
\begin{frame}[t]
\setlength{\abovedisplayskip}{5pt}
\setlength{\belowdisplayskip}{5pt}
\frametitle{Nichtauflösbarkeit von $S_n$}
\vspace{-20pt}
\begin{columns}[t,onlytextwidth]
\begin{column}{0.48\textwidth}
\begin{block}{Die symmetrische Gruppe $S_n$}
Permutationen auf $n$ Elementen
\[
\sigma
=
\begin{pmatrix}
1&2&3&\dots&n\\
\sigma(1)&\sigma(2)&\sigma(3)&\dots&\sigma(n)
\end{pmatrix}
\]
\end{block}
\vspace{-10pt}
\uncover<2->{%
\begin{block}{Signum}
$t(\sigma)=\mathstrut$ Anzahl Transpositionen
\[
\operatorname{sgn}(\sigma)
=
(-1)^{t(\sigma)}
=
\begin{cases}
\phantom{-}1&\text{$t(\sigma)$ gerade}
\\
-1&\text{$t(\sigma)$ ungerade}
\end{cases}
\]
Homomorphismus!
\end{block}}
\uncover<3->{%
\begin{block}{Die alternierende Gruppe $A_n$}
\vspace{-12pt}
\[
A_n = \ker \operatorname{sgn}
=
\{\sigma\in  S_n\;|\;\operatorname{sgn}(\sigma)=1\}
\]
\end{block}}
\end{column}
\begin{column}{0.48\textwidth}
\uncover<4->{%
\begin{block}{Normale Untergruppe}
\begin{itemize}
\item
$H\triangleleft G$ wenn $gHg^{-1}\subset G\;\forall g\in G$
\item
$G/N$ ist wohldefiniert
\end{itemize}
\end{block}}
\vspace{-10pt}
\uncover<5->{%
\begin{block}{Einfache Gruppe}
$G$ einfach $\Leftrightarrow$
\[
H\triangleleft G
\;
\Rightarrow
\;
\text{$H=\{e\}$ oder $H=G$}
\]
\end{block}}
\vspace{-10pt}
\uncover<6->{%
\begin{block}{$n\ge 5 \Rightarrow A_n \text{ einfach}$}
\begin{enumerate}
\item<7-> Zeigen, dass $A_5$ einfach ist
\item<8-> Vollständige Induktion: $A_n$ einfach $\Rightarrow A_{n+1}$ einfach
\end{enumerate}
\uncover<9->{%
$\Rightarrow$ i.~A.~keine Lösung der 
einer Polynomgleichung vom Grad $\ge 5$ durch Radikale
}
\end{block}}
\end{column}
\end{columns}
\end{frame}
