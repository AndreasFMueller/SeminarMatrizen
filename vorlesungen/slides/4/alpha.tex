%
% alpha.tex
%
% (c) 2021 Prof Dr Andreas Müller, OST Ostschweizer Fachhochschule
%
\begin{frame}[t]
\setlength{\abovedisplayskip}{5pt}
\frametitle{Was ist $\alpha$?}
$m(X)$ ein irreduzibles Polynome in $\mathbb{F}_2[X]$

Beispiel: $m(X) = X^8{\color{red}\mathstrut+X^4+X^3+X^2+1}\in\mathbb{F}_2[X]$
\begin{columns}[t]
\begin{column}{0.40\textwidth}
\uncover<2->{%
\begin{block}{Abstrakt}
$\alpha$ ist ein ``imaginäres''
Objekt mit der Rechenregel $m(\alpha)=0$
\begin{align*}
\alpha^8 &= {\color{red}\alpha^4+\alpha^3+\alpha^2+1}\\
\uncover<3->{
\alpha^9 &= \alpha^5+\alpha^4+\alpha^3+\alpha}\\
\uncover<4->{
\alpha^{10}&= \alpha^6+\alpha^5+\alpha^4+\alpha^2}\\
\uncover<5->{
\alpha^{11}&= \alpha^7+\alpha^6+\alpha^5+\alpha^3}\\
\uncover<6->{
\alpha &= \alpha^7+\alpha^3+\alpha^2+\alpha}
\\
\end{align*}
\end{block}}
\end{column}
\begin{column}{0.54\textwidth}
\uncover<7->{%
\begin{block}{Matrix}
Eine konkretes Element in $M_n(\mathbb{F}_2)$
\[
\alpha
=
\begin{pmatrix}
0& 0& 0& 0& 0& 0& 0& {\color{red}1}\\
1& 0& 0& 0& 0& 0& 0& {\color{red}0}\\
0& 1& 0& 0& 0& 0& 0& {\color{red}1}\\
0& 0& 1& 0& 0& 0& 0& {\color{red}1}\\
0& 0& 0& 1& 0& 0& 0& {\color{red}1}\\
0& 0& 0& 0& 1& 0& 0& {\color{red}0}\\
0& 0& 0& 0& 0& 1& 0& {\color{red}0}\\
0& 0& 0& 0& 0& 0& 1& {\color{red}0}
\end{pmatrix}
\]
\end{block}}
\end{column}
\end{columns}

\end{frame}
