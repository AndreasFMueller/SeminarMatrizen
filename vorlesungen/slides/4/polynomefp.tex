%
% polynomefp.tex
%
% (c) 2021 Prof Dr Andreas Müller, OST Ostschweizer Fachhochschule
%
\begin{frame}[t]
\frametitle{Polynome über $\mathbb{F}_p[X]$}
\setlength{\abovedisplayskip}{5pt}
\setlength{\belowdisplayskip}{5pt}
\vspace{-15pt}
\begin{columns}[t,onlytextwidth]
\begin{column}{0.48\textwidth}
\begin{block}{Polynomring}
$\mathbb{F}_p[X]$ sind Polynome
\[
p(X)
=
a_0+a_1X+\dots+a_nX^n
\]
mit $a_i\in\mathbb{F}_p$.
\uncover<2->{ObdA: $a_n=1$}%

\end{block}
\uncover<3->{%
\begin{block}{Irreduzible Polynome}
$m(X)$ ist irreduzibel, wenn es keine Faktorisierung
$m(X)=p(X)q(X)$ mit $p,q\in\mathbb{F}_p[X]$ gibt
\end{block}}
\uncover<4->{%
\begin{block}{Rest modulo $m(X)$}
$X^{n+k}$ kann immer reduziert werden:
\[
X^{n+k} = -(a_0+a_1X+\dots+a_{n-1}X^{n-1})X^k
\]
\end{block}}
\end{column}
\begin{column}{0.48\textwidth}
\uncover<5->{%
\begin{block}{Körper $\mathbb{F}_p/(m(X))$}
Wenn $m(X)$ irreduzibel ist, dann ist
$\mathbb{F}_p[X]$ nullteilerfrei.
\medskip

\uncover<6->{$a\in \mathbb{F}_p[X]$ mit $\deg a < \deg m$, dann ist}
\begin{enumerate}
\item<7->
$\operatorname{ggT}(a,m) = 1$
\item<8->
Es gibt $s,t\in\mathbb{F}_p[X]$ mit
\[
s(X)m(X)+t(X)a(X) = 1
\]
(aus dem euklidischen Algorithmus)
\item<9->
$a^{-1} = t(X)$
\end{enumerate}
\uncover<9->{$\Rightarrow$ $\mathbb{F}_p[X]/(m(X))$ ist ein Körper
mit genau $p^n$ Elementen}
\end{block}}
\end{column}
\end{columns}
\end{frame}
