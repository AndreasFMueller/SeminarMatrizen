%
% liealgbeispiel.tex -- slide template
%
% (c) 2021 Prof Dr Andreas Müller, OST Ostschweizer Fachhochschule
%
\bgroup
\begin{frame}[t]
\setlength{\abovedisplayskip}{5pt}
\setlength{\belowdisplayskip}{5pt}
\frametitle{Lie-Algebra Beispiele}
\vspace{-20pt}
\begin{columns}[t,onlytextwidth]
\begin{column}{0.48\textwidth}
\begin{block}{$\operatorname{sl}_2(\mathbb{R})$}
Spurlose Matrizen:
\[
\operatorname{sl}_2(\mathbb{R})
=
\{A\in M_n(\mathbb{R})\;|\; \operatorname{Spur}A=0\}
\]
\end{block}
\begin{block}{Lie-Algebra?}
Nachrechnen: $[A,B]\in \operatorname{sl}_2(\mathbb{R})$:
\begin{align*}
\operatorname{Spur}([A,B])
&=
\operatorname{Spur}(AB-BA)
\\
&=
\operatorname{Spur}(AB)-\operatorname{Spur}(BA)
\\
&=
\operatorname{Spur}(AB)-\operatorname{Spur}(AB)
\\
&=0
\end{align*}
$\Rightarrow$ $\operatorname{sl}_2(\mathbb{R})$ ist eine Lie-Algebra
\end{block}
\end{column}
\begin{column}{0.48\textwidth}
\begin{block}{$\operatorname{so}(n)$}
Antisymmetrische Matrizen:
\[
\operatorname{so}(n)
=
\{A\in M_n(\mathbb{R})
\;|\;
A=-A^t
\}
\]
\end{block}
\begin{block}{Lie-Algebra?}
Nachrechnen: $A,B\in \operatorname{so}(n)$
\begin{align*}
[A,B]^t
&=
(AB-BA)^t
\\
&=
B^tA^t - A^tB^t
\\
&=
(-B)(-A)-(-A)(-B)
\\
&=
BA-AB
=
-(AB-BA)
\\
&=
-[A,B]
\end{align*}
$\Rightarrow$ $\operatorname{so}(n)$ ist eine Lie-Algebra
\end{block}
\end{column}
\end{columns}
\end{frame}
\egroup
