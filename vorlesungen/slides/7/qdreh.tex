%
% template.tex -- slide template
%
% (c) 2021 Prof Dr Andreas Müller, OST Ostschweizer Fachhochschule
%
\bgroup
\begin{frame}[t]
\setlength{\abovedisplayskip}{5pt}
\setlength{\belowdisplayskip}{5pt}
\frametitle{Drehungen mit Quaternionen}
\vspace{-20pt}
\begin{columns}[t,onlytextwidth]
\begin{column}{0.48\textwidth}
\begin{block}{Drehung?}
Abbildung von $\vec{x}$ mit $\operatorname{Re}\vec{x}=0$:
\[
\varrho_{q}
\colon
\vec{x}\mapsto  q\vec{x}q^{-1} = q\vec{x}\overline{q}
\]
\end{block}
\uncover<2->{%
\begin{block}{Achse}
\begin{align*}
\varrho_q(q)
&=
qq\overline{q}
\uncover<3->{=
q(qq^{-1})}
\uncover<4->{=
q}
\end{align*}
\end{block}}
\uncover<4->{%
\begin{block}{Norm}
\begin{align*}
|\varrho_q(\vec{x})|^2
&=
q\vec{x}\overline{q}\overline{(q\vec{x}\overline{q})}
\uncover<5->{=
q\vec{x}\overline{q}\overline{\overline{q}}\overline{\vec{x}}\overline{q}
}
\\
&\uncover<6->{=
q\vec{x}(\overline{q}q)\overline{\vec{x}}\overline{q}}
\uncover<7->{=
q(\vec{x}\overline{\vec{x}})\overline{q}}
\uncover<8->{=
q\overline{q}|\vec{x}|^2}
\\
&\uncover<9->{=
|\vec{x}|^2}
\end{align*}
\uncover<10->{%
$\Rightarrow$ $\varrho_q\in\operatorname{O}(3)$}
\end{block}}
\end{column}
\begin{column}{0.48\textwidth}
\uncover<11->{%
\begin{block}{Drehung!}
$\vec{a},\vec{b},\vec{n}$ bilden ein on.~Rechtssystem
\begin{align*}
\uncover<12->{
qa
&=
c\vec{a}+s\vec{n}\times \vec{a}}
\uncover<13->{=
c\vec{a} + s\vec{b}}
\\
\uncover<14->{
q\vec{a}\overline{q}
&=
(c\vec{a}+s\vec{b}) c
-(c\vec{a}+s\vec{b})\times s\vec{n}}
\\
&\uncover<15->{=
c^2 \vec{a}+ sc\vec{b}
+sc\vec{b} - s^2 \vec{a}}
\\
&\uncover<16->{=
\vec{a} \cos\alpha  +\vec{b} \sin\alpha }
\end{align*}
\vspace{-5pt}
\uncover<17->{wegen
%\vspace{-5pt}
\[
\begin{aligned}
\cos\alpha &= \cos^2\frac{\alpha}2 - \sin^2\frac{\alpha}2 &&=c^2-s^2
\\
\sin\alpha &= 2\cos\frac{\alpha}2\sin\frac{\alpha}2&&=2cs
\end{aligned}\]}
\end{block}}
\vspace{-18pt}
\uncover<18->{%
\begin{block}{Matrix}
\[
D
=
\tiny
\begin{pmatrix}
1-2(q_2^2+q_3^2)&-2q_0q_3+2q_1q_2&-2q_0q_2+2q_1q_3\\
 2q_0q_3+2q_1q_2&1-2(q_1^2+q_3^2)&-2q_0q_1+2q_2q_3\\
-2q_0q_2+2q_1q_3& 2q_0q_1+2q_2q_3&1-2(q_1^2+q_2^2)
\end{pmatrix}
\]
\end{block}}
\end{column}
\end{columns}
\end{frame}
\egroup
