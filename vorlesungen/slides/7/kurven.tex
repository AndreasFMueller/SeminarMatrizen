%
% kurven.tex -- slide template
%
% (c) 2021 Prof Dr Andreas Müller, OST Ostschweizer Fachhochschule
%
\bgroup
\begin{frame}[t]
\setlength{\abovedisplayskip}{5pt}
\setlength{\belowdisplayskip}{5pt}
\frametitle{Kurven und Tangenten}
\vspace{-20pt}
\begin{columns}[t,onlytextwidth]
\begin{column}{0.48\textwidth}
\begin{block}{Kurven}
Kurve in $\mathbb{R}^n$:
\vspace{-12pt}
\[
\gamma
\colon
I=[a,b] \to \mathbb{R}^n
:
t\mapsto \gamma(t)
=
\begin{pmatrix}
x_1(t)\\
x_2(t)\\
\vdots\\
x_n(t)
\end{pmatrix}
\]
\vspace{-15pt}
\begin{center}
\begin{tikzpicture}[>=latex,thick]
\coordinate (A) at (1,0.5);
\coordinate (B) at (4,0.5);
\coordinate (C) at (2,2.2);
\coordinate (D) at (5,2);
\coordinate (E) at ($(C)+(80:2)$);

\draw[color=red,line width=1.4pt]
	(A) to[in=-160] (B) to[out=20,in=-100] (C) to[out=80] (D);
\fill[color=red] (C) circle[radius=0.06];
\node[color=red] at (C) [left] {$\gamma(t)$};

\draw[->,color=blue,line width=1.4pt,shorten <= 0.06cm] (C) -- (E);
\node[color=blue] at (E) [right] {$\dot{\gamma}(t)$};

\draw[->] (-0.1,0) -- (5.9,0) coordinate[label={$x_1$}];
\draw[->] (0,-0.1) -- (0,4.3) coordinate[label={right:$x_2$}];
\end{tikzpicture}
\end{center}
\end{block}
\end{column}
\begin{column}{0.48\textwidth}
\begin{block}{Tangenten}
Ableitung
\[
\frac{d}{dt}\gamma(t)
=
\dot{\gamma}(t)
=
\begin{pmatrix}
\dot{x}_1(t)\\
\dot{x}_2(t)\\
\vdots\\
\dot{x}_n(t)
\end{pmatrix}
\]
Lineare Approximation:
\[
\gamma(t+h)
=
\gamma(t)
+
\dot{\gamma}(t) \cdot h
+
o(h)
\]
Sinnvoll, weil sowohl $\gamma(t)$ und $\dot{\gamma}(t)$
in $\mathbb{R}^n$ liegen
\end{block}
\end{column}
\end{columns}
\end{frame}
\egroup
