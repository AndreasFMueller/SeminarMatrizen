%
% drehung.tex -- Drehung aus streckungen
%
% (c) 2021 Prof Dr Andreas Müller, OST Ostschweizer Fachhochschule
%
\bgroup
\definecolor{darkgreen}{rgb}{0,0.6,0}
\begin{frame}[t]
\setlength{\abovedisplayskip}{5pt}
\setlength{\belowdisplayskip}{5pt}
\frametitle{Drehung aus Streckungen und Scherungen}
\vspace{-20pt}
\begin{columns}[t,onlytextwidth]
\begin{column}{0.38\textwidth}
\begin{block}{Drehung}
{\color{blue}Längen}, {\color<2->{blue}Winkel},
{\color<2->{darkgreen}Orientierung}
erhalten
\uncover<2->{
\[
\operatorname{SO}(2)
=
{\color{blue}\operatorname{O}(2)}
\cap
{\color{darkgreen}\operatorname{SL}_2(\mathbb{R})}
\]}
\vspace{-20pt}
\end{block}
\uncover<3->{%
\begin{block}{Zusammensetzung}
Eine Drehung muss als Zusammensetzung geschrieben werden können:
\[
D_{\alpha} 
=
\begin{pmatrix}
\cos\alpha & -\sin\alpha\\
\sin\alpha &\phantom{-}\cos\alpha
\end{pmatrix}
=
DST
\]
\end{block}}
\vspace{-10pt}
\uncover<12->{%
\begin{block}{Beispiel}
\vspace{-12pt}
\[
D_{60^\circ}
=
{\tiny
\begin{pmatrix}2&0\\0&\frac12\end{pmatrix}
\begin{pmatrix}1&-\frac{\sqrt{3}}4\\0&1\end{pmatrix}
\begin{pmatrix}1&0\\\sqrt{3}&1\end{pmatrix}
}
\]
\end{block}}
\end{column}
\begin{column}{0.58\textwidth}
\uncover<4->{%
\begin{block}{Ansatz}
%\vspace{-12pt}
\begin{align*}
DST
&=
\begin{pmatrix}
c^{-1}&0\\
  0   &c
\end{pmatrix}
\begin{pmatrix}
1&-s\\
0&1
\end{pmatrix}
\begin{pmatrix}
1&0\\
t&1
\end{pmatrix}
\\
&\uncover<5->{=
\begin{pmatrix}
c^{-1}&0\\
  0   &c
\end{pmatrix}
\begin{pmatrix}
1-st&-s\\
   t& 1
\end{pmatrix}
}
\\
&\uncover<6->{=
\begin{pmatrix}
{\color<11->{orange}(1-st)c^{-1}}&{\color<10->{darkgreen}sc^{-1}}\\
{\color<9->{blue}ct}&{\color<8->{red}c}
\end{pmatrix}}
\uncover<7->{=
\begin{pmatrix}
{\color<11->{orange}\cos\alpha} & {\color<10->{darkgreen}- \sin\alpha} \\
{\color<9->{blue}\sin\alpha} & \phantom{-}  {\color<8->{red}\cos\alpha}
\end{pmatrix}}
\end{align*}
\end{block}}
\vspace{-10pt}
\uncover<7->{%
\begin{block}{Koeffizientenvergleich}
%\vspace{-15pt}
\begin{align*}
\uncover<8->{
{\color{red} c}
&=
{\color{red}\cos\alpha }}
&&
&
\uncover<9->{
{\color{blue}
t}&=\rlap{$\displaystyle\frac{\sin\alpha}{c} = \tan\alpha$}}\\
\uncover<10->{
{\color{darkgreen}sc^{-1}}&={\color{darkgreen}-\sin\alpha}
&
&\Rightarrow&
{\color{darkgreen}s}&={\color{darkgreen}-\sin\alpha}\cos\alpha
}
\\
\uncover<11->{
{\color{orange} (1-st)c^{-t}}
&=
\rlap{$\displaystyle\frac{(1-\sin^2\alpha)}{\cos\alpha} = \cos\alpha $}
}
\end{align*}
\end{block}}
\end{column}
\end{columns}
\end{frame}
\egroup
