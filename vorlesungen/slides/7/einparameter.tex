%
% einparameter.tex -- Einparameter Untergruppen
%
% (c) 2021 Prof Dr Andreas Müller, OST Ostschweizer Fachhochschule
%
\bgroup
\begin{frame}[t]
\setlength{\abovedisplayskip}{5pt}
\setlength{\belowdisplayskip}{5pt}
\frametitle{Einparameter-Untergruppen}
\vspace{-20pt}
\begin{columns}[t,onlytextwidth]
\begin{column}{0.48\textwidth}
\begin{block}{Definition}
Eine Kurve $\gamma\colon \mathbb{R}\to G\subset\operatorname{GL}_n(\mathbb{R})$,
die {\color<2->{red}gleichzeitig eine Untergruppe von $G$} ist \uncover<3->{mit}
\[
\uncover<3->{
\gamma(t+s) = \gamma(t)\gamma(s)\quad\forall t,s\in\mathbb{R}
}
\]
\end{block}
\uncover<4->{%
\begin{block}{Drehungen}
Drehmatrizen bilden Einparameter- Untergruppen
\begin{align*}
t \mapsto D_{x,t}
&=
\begin{pmatrix}
1&0&0\\
0&\cos t&-\sin t\\
0&\sin t& \cos t
\end{pmatrix}
\\
D_{x,t}D_{x,s}
&=
D_{x,t+s}
\end{align*}
\end{block}}
\end{column}
\begin{column}{0.48\textwidth}
\uncover<5->{%
\begin{block}{Scherungen in $\operatorname{SL}_2(\mathbb{R})$}
%\vspace{-12pt}
\[
\begin{pmatrix}
1&s\\
0&1
\end{pmatrix}
\begin{pmatrix}
1&t\\
0&1
\end{pmatrix}
=
\begin{pmatrix}
1&s+t\\
0&1
\end{pmatrix}
\]
\end{block}}
\vspace{-12pt}
\uncover<6->{%
\begin{block}{Skalierungen in $\operatorname{SL}_2(\mathbb{R})$}
%\vspace{-12pt}
\[
\begin{pmatrix}
e^s&0\\0&e^{-s}
\end{pmatrix}
\begin{pmatrix}
e^t&0\\0&e^{-t}
\end{pmatrix}
=
\begin{pmatrix}
e^{t+s}&0\\0&e^{-(t+s)}
\end{pmatrix}
\]
\end{block}}
\vspace{-12pt}
\uncover<7->{%
\begin{block}{Gemischt}
%\vspace{-12pt}
\begin{gather*}
A_t = I \cosh t + \begin{pmatrix}1&a\\0&-1\end{pmatrix}\sinh t
\\
\text{dank}\quad
\begin{pmatrix}1&s\\0&-1\end{pmatrix}^2
=I
\end{gather*}
\end{block}}
\end{column}
\end{columns}
\end{frame}
\egroup
