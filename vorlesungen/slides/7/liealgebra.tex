%
% liealgebra.tex -- Lie-Algebra
%
% (c) 2021 Prof Dr Andreas Müller, OST Ostschweizer Fachhochschule
%
\bgroup
\begin{frame}[t]
\setlength{\abovedisplayskip}{5pt}
\setlength{\belowdisplayskip}{5pt}
\frametitle{Lie-Algebra}
\ifthenelse{\boolean{presentation}}{\vspace{-15pt}}{\vspace{-8pt}}
\begin{block}{Vektorraum}
Tangentialvektoren im Punkt $I$:
\begin{center}
\begin{tabular}{>{$}c<{$}|p{6cm}|>{$}c<{$}}
\text{Lie-Gruppe $G$}&Tangentialvektoren&\text{Lie-Algebra $LG$} \\
\hline
\uncover<2->{
\operatorname{GL}_n(\mathbb{R})
& beliebige Matrizen
& M_n(\mathbb{R})
}
\\
\uncover<3->{
\operatorname{O(n)}
& antisymmetrische Matrizen
& \operatorname{o}(n)
}
\\
\uncover<4->{
\operatorname{SL}_n(\mathbb{R})
& spurlose Matrizen
& \operatorname{sl}_2(\mathbb{R})
}
\\
\uncover<5->{
\operatorname{U(n)}
& antihermitesche Matrizen
& \operatorname{u}(n)
}
\\
\uncover<6->{
\operatorname{SU(n)}
& spurlose, antihermitesche Matrizen
& \operatorname{su}(n)
}
\end{tabular}
\end{center}
\end{block}
\vspace{-20pt}
\begin{columns}[t,onlytextwidth]
\begin{column}{0.40\textwidth}
\uncover<7->{%
\begin{block}{Lie-Klammer}
Kommutator: $[A,B] = AB-BA$
\end{block}}
\uncover<8->{%
\begin{block}{Nachprüfen}
$[A,B]\in LG$
für $A,B\in LG$
\end{block}}
\end{column}
\begin{column}{0.56\textwidth}
\uncover<9->{%
\begin{block}{Algebraische Eigenschaften}
\begin{itemize}
\item<10-> antisymmetrisch: $[A,B]=-[B,A]$
\item<11-> Jacobi-Identität
\[
[A,[B,C]]+
[B,[C,A]]+
[C,[A,B]]
= 0
\]
\end{itemize}
\vspace{-13pt}
\uncover<12->{%
{\usebeamercolor[fg]{title}
Beispiel:} $\mathbb{R}^3$ mit Vektorprodukt $\mathstrut = \operatorname{so}(3)$
}
\end{block}}
\end{column}
\end{columns}
\end{frame}
\egroup
