%
% dg.tex -- Differentialgleichung für die Exponentialabbildung
%
% (c) 2021 Prof Dr Andreas Müller, OST Ostschweizer Fachhochschule
%
\bgroup
\begin{frame}[t]
\setlength{\abovedisplayskip}{5pt}
\setlength{\belowdisplayskip}{5pt}
\frametitle{Zurück zur Lie-Gruppe}
\vspace{-20pt}
\begin{columns}[t,onlytextwidth]
\begin{column}{0.48\textwidth}
\begin{block}{Tangentialvektor im Punkt $\gamma(t)$}
Ableitung von $\gamma(t)$ an der Stelle $t$:
\begin{align*}
\dot{\gamma}(t)
&=
\frac{d}{d\tau}\gamma(\tau)\bigg|_{\tau=t}
\\
&=
\frac{d}{ds}
\gamma(t+s)
\bigg|_{s=0}
\\
&=
\frac{d}{ds}
\gamma(t)\gamma(s)
\bigg|_{s=0}
\\
&=
\gamma(t)
\frac{d}{ds}
\gamma(s)
\bigg|_{s=0}
=
\gamma(t) \dot{\gamma}(0)
\end{align*}
\end{block}
\vspace{-10pt}
\begin{block}{Differentialgleichung}
\vspace{-10pt}
\[
\dot{\gamma}(t) = \gamma(t) A
\quad
\text{mit}
\quad
A=\dot{\gamma}(0)\in LG
\]
\end{block}
\end{column}
\begin{column}{0.50\textwidth}
\begin{block}{Lösung}
Exponentialfunktion
\[
\exp\colon LG\to G : A \mapsto \exp(At) = \sum_{k=0}^\infty \frac{t^k}{k!}A^k
\]
\end{block}
\vspace{-5pt}
\begin{block}{Kontrolle: Tangentialvektor berechnen}
\vspace{-10pt}
\begin{align*}
\frac{d}{dt}e^{At}
&=
\sum_{k=1}^\infty A^k \frac{d}{dt} t^{k}{k!}
\\
&=
\sum_{k=1}^\infty A^{k-1}\frac{t^{k-1}}{(k-1)!} A
\\
&=
\sum_{k=0} A^k\frac{t^k}{k!}
A
=
e^{At} A
\end{align*}
\end{block}
\end{column}
\end{columns}
\end{frame}
\egroup
