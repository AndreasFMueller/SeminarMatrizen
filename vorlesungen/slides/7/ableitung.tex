%
% ableitung.tex -- Ableitung in der Lie-Gruppe
%
% (c) 2021 Prof Dr Andreas Müller, OST Ostschweizer Fachhochschule
%
\bgroup
\begin{frame}[t]
\setlength{\abovedisplayskip}{5pt}
\setlength{\belowdisplayskip}{5pt}
\frametitle{Ableitung in der Matrix-Gruppe}
\vspace{-20pt}
\begin{columns}[t,onlytextwidth]
\begin{column}{0.48\textwidth}
\begin{block}{Ableitung in $\operatorname{O}(n)$}
$s \mapsto A(s)\in\operatorname{O}(n)$
\begin{align*}
I
&=
A(s)^tA(s)
\\
0
=
\frac{d}{ds} I
&=
\frac{d}{ds} (A(s)^t A(s))
\\
&=
\dot{A}(s)^tA(s) + A(s)^t \dot{A}(s)
\intertext{An der Stelle $s=0$, d.~h.~$A(0)=I$}
0
&=
\dot{A}(0)^t
+
\dot{A}(0)
\\
\Leftrightarrow
\qquad
\dot{A}(0)^t &= -\dot{A}(0)
\end{align*}
``Tangentialvektoren'' sind antisymmetrische Matrizen
\end{block}
\end{column}
\begin{column}{0.48\textwidth}
\begin{block}{Ableitung in $\operatorname{SL}_2(\mathbb{R})$}
$s\mapsto A(s)\in\operatorname{SL}_n(\mathbb{R})$
\begin{align*}
1 &= \det A(t)
\\
0
=
\frac{d}{dt}1
&=
\frac{d}{dt} \det A(t)
\intertext{mit dem Entwicklungssatz kann man nachrechnen:}
0&=\operatorname{Spur}\dot{A}(0)
\end{align*}
``Tangentialvektoren'' sind spurlose Matrizen
\end{block}
\end{column}
\end{columns}
\end{frame}
\egroup
