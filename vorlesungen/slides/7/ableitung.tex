%
% ableitung.tex -- Ableitung in der Lie-Gruppe
%
% (c) 2021 Prof Dr Andreas Müller, OST Ostschweizer Fachhochschule
%
\bgroup
\begin{frame}[t]
\setlength{\abovedisplayskip}{5pt}
\setlength{\belowdisplayskip}{5pt}
\frametitle{Ableitung in der Matrix-Gruppe}
\vspace{-20pt}
\begin{columns}[t,onlytextwidth]
\begin{column}{0.48\textwidth}
\begin{block}{Ableitung in $\operatorname{O}(n)$}
\uncover<2->{%
$s \mapsto A(s)\in\operatorname{O}(n)$
}
\begin{align*}
\uncover<3->{I
&=
A(s)^tA(s)}
\\
\uncover<4->{0
=
\frac{d}{ds} I
&=
\frac{d}{ds} (A(s)^t A(s))}
\\
&\uncover<5->{=
\dot{A}(s)^tA(s) + A(s)^t \dot{A}(s)}
\intertext{\uncover<6->{An der Stelle $s=0$, d.~h.~$A(0)=I$}}
\uncover<7->{0
&=
\dot{A}(0)^t
+
\dot{A}(0)}
\\
\uncover<8->{\Leftrightarrow
\qquad
\dot{A}(0)^t &= -\dot{A}(0)}
\end{align*}
\uncover<9->{%
``Tangentialvektoren'' sind antisymmetrische Matrizen
}
\end{block}
\end{column}
\begin{column}{0.48\textwidth}
\begin{block}{Ableitung in $\operatorname{SL}_2(\mathbb{R})$}
\uncover<2->{%
$s\mapsto A(s)\in\operatorname{SL}_n(\mathbb{R})$
}
\begin{align*}
\uncover<3->{1 &= \det A(t)}
\\
\uncover<10->{0
=
\frac{d}{dt}1
&=
\frac{d}{dt} \det A(t)}
\intertext{\uncover<11->{mit dem Entwicklungssatz kann man nachrechnen:}}
\uncover<12->{0&=\operatorname{Spur}\dot{A}(0)}
\end{align*}
\uncover<13->{``Tangentialvektoren'' sind spurlose Matrizen}
\end{block}
\end{column}
\end{columns}
\end{frame}
\egroup
