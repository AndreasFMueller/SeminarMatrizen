%
% integration.tex -- slide template
%
% (c) 2021 Prof Dr Andreas Müller, OST Ostschweizer Fachhochschule
%
\bgroup
\begin{frame}[t]
\setlength{\abovedisplayskip}{5pt}
\setlength{\belowdisplayskip}{5pt}
\frametitle{Invariante Integration}
\vspace{-20pt}
\begin{columns}[t,onlytextwidth]
\begin{column}{0.48\textwidth}
\begin{block}{Koordinatenwechsel}
Die Koordinatentransformation
$f\colon\mathbb{R}^n\to\mathbb{R}^n:x\to y$
hat die Ableitungsmatrix
\[
t_{ij}
=
\frac{\partial y_i}{\partial x_j}
\]
\uncover<2->{%
$n$-faches Integral
\begin{gather*}
\int\dots\int
h(f(x))
\det
\biggl(
\frac{\partial y_i}{\partial x_j}
\biggr)
\,dx_1\,\dots dx_n
\\
=
\int\dots\int
h(y)
\,dy_1\,\dots dy_n
\end{gather*}}
\end{block}
\end{column}
\begin{column}{0.48\textwidth}
\uncover<3->{%
\begin{block}{auf einer Lie-Gruppe}
Koordinatenwechsel sind Multiplikationen mit einer
Matrix $g\in G$
\end{block}}
\uncover<4->{%
\begin{block}{Volumenelement in $I$}
Man muss nur das Volumenelement in $I$ in einem beliebigen
Koordinatensystem definieren:
\[
dV = dy_1\,\dots\,dy_n
\]
\end{block}}
\uncover<5->{%
\begin{block}{Volumenelement in $g$}
\[
\text{``\strut}g\cdot dV\text{\strut''}
=
\det(g) \, dy_1\,\dots\,dy_n
\]
\end{block}}
\end{column}
\end{columns}
\end{frame}
\egroup
