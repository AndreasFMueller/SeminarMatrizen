%
% haar.tex -- slide template
%
% (c) 2021 Prof Dr Andreas Müller, OST Ostschweizer Fachhochschule
%
\bgroup
\begin{frame}[t]
\setlength{\abovedisplayskip}{5pt}
\setlength{\belowdisplayskip}{5pt}
\frametitle{Haar-Mass}
\vspace{-20pt}
\begin{columns}[t,onlytextwidth]
\begin{column}{0.48\textwidth}
\begin{block}{Invariantes Mass}
Auf jeder lokalkompakten Gruppe $G$ gibt es ein \only<2->{invariantes }%
Integral
\begin{align*}
\uncover<2->{\text{rechts:}}&&
\int_G f(g)\,d\mu(g)
&\uncover<2->{=
\int_G f(gh)\,d\mu(g)}
\\
\uncover<3->{
\text{links:}&&
\int_G f(g)\,d\mu(g)
&=
\int_G f(hg)\,d\mu(g)}
\end{align*}

\end{block}
\uncover<7->{%
\begin{block}{Modulus-Funktion}
$\mu$ linksinvariant, dann ist die Rechtsverschiebung ebenfalls
linksinvariant
\[
\int_G f(gh) \, d\mu(g)
\uncover<8->{
=
\int_G f(g) \Delta(h)\, d\mu(g)
}
\]
\uncover<9->{$\Delta(h)$ heisst Modulus-Funktion}
\end{block}}
\end{column}
\begin{column}{0.48\textwidth}
\uncover<4->{%
\begin{block}{Beispiel: $G=\mathbb{R}$}
\[
\int_Gf(g)\,d\mu(g)
=
\int_{-\infty}^{\infty} f(x)\,dx
\]
\end{block}}
\vspace{-10pt}
\uncover<5->{%
\begin{block}{Beispiel: $\operatorname{SO}(2)$}
\[
\int_{\operatorname{SO}(2)}
f(g)\,d\mu(g)
=
\frac{1}{2\pi}
\int_{0}^{2\pi} f(D_{\alpha})\,d\alpha
\]
\end{block}}
\vspace{-10pt}
\uncover<6->{%
\begin{block}{Beispiel: $G$ endlich}
\[
\int_G f(g)\,d\mu(g) = \frac{1}{|G|}\sum_{g\in G}f(g)
\]
\end{block}}
\vspace{-10pt}
\uncover<10->{%
\begin{block}{Unimodular}
$\Delta(h)=1$ heisst rechtsinvariant = linksinvariant
\\
\uncover<11->{%
$G$ kompakt $\Rightarrow$ unimodular
}
\end{block}}
\end{column}
\end{columns}
\end{frame}
\egroup
