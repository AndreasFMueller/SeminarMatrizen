%
% semi.tex -- Beispiele: semidirekte Produkte
%
% (c) 2021 Prof Dr Andreas Müller, OST Ostschweizer Fachhochschule
%
\bgroup
\begin{frame}[t]
\setlength{\abovedisplayskip}{5pt}
\setlength{\belowdisplayskip}{5pt}
\frametitle{Drehung/Skalierung und Verschiebung}
\vspace{-20pt}
\begin{columns}[t,onlytextwidth]
\begin{column}{0.48\textwidth}
\begin{block}{Skalierung und Verschiebung}
Gruppe $G=\{(e^s,t)\;|\;s,t\in\mathbb{R}\}$
\\
Wirkung auf $\mathbb{R}$:
\[
x\mapsto \underbrace{e^s\cdot x}_{\text{Skalierung}} \mathstrut+ t
\]
\end{block}
\end{column}
\begin{column}{0.48\textwidth}
\uncover<2->{%
\begin{block}{Drehung und Verschiebung}
Gruppe
$G=
\{ (\alpha,\vec{t})
\;|\;
\alpha\in\mathbb{R},\vec{t}\in\mathbb{R}^2
\}$
Wirkung auf $\mathbb{R}^2$:
\[
\vec{x}\mapsto \underbrace{D_\alpha \vec{x}}_{\text{Drehung}} \mathstrut+ \vec{t}
\]
\end{block}}
\end{column}
\end{columns}
\vspace{-15pt}
\begin{columns}[t,onlytextwidth]
\begin{column}{0.48\textwidth}
\uncover<3->{%
\begin{block}{Verknüpfung}
\vspace{-15pt}
\begin{align*}
(e^{s_1},t_1)(e^{s_2},t_2)x
&\uncover<4->{=
(e^{s_1},t_1)(e^{s_2}x+t_2)}
\\
&\uncover<5->{=
e^{s_1+s_2}x + e^{s_1}t_2+t_1}
\\
\uncover<6->{
(e^{s_1},t_1)(e^{s_2},t_2)
&=
(e^{s_1}e^{s_2},t_1+e^{s_1}t_2)}
\end{align*}
\end{block}}
\end{column}
\begin{column}{0.48\textwidth}
\uncover<7->{%
\begin{block}{Verknüpfung}
\vspace{-15pt}
\begin{align*}
(\alpha_1,\vec{t}_1)
(\alpha_2,\vec{t}_2)
\vec{x}
&\uncover<8->{=
(\alpha_1,\vec{t}_1)(D_{\alpha_2}\vec{x}+\vec{t}_2)}
\\
&\uncover<9->{=D_{\alpha_1+\alpha_2}\vec{x} + D_{\alpha_1}\vec{t}_2+\vec{t}_1}
\\
\uncover<10->{
(\alpha_1,\vec{t}_1)
(\alpha_2,\vec{t}_2)
&=
(\alpha_1+\alpha_2, D_{\alpha_1}\vec{t}_2+\vec{t}_1)
}
\end{align*}
\end{block}}
\end{column}
\end{columns}
\vspace{-10pt}
\begin{columns}[t,onlytextwidth]
\begin{column}{0.48\textwidth}
\uncover<11->{%
\begin{block}{Matrixschreibweise}
\vspace{-12pt}
\[
g=(e^s,t) =
\begin{pmatrix}
e^s&t\\
0&1
\end{pmatrix}
\quad\text{auf}\quad
\begin{pmatrix}x\\1\end{pmatrix}
\]
\end{block}}
\end{column}
\begin{column}{0.48\textwidth}
\uncover<12->{%
\begin{block}{Matrixschreibweise}
\vspace{-12pt}
\[
g=(\alpha,\vec{t}) =
\begin{pmatrix}
D_{\alpha}&\vec{t}\\
0&1
\end{pmatrix}
\quad\text{auf}\quad
\begin{pmatrix}\vec{x}\\1\end{pmatrix}
\]
\end{block}}
\end{column}
\end{columns}
\end{frame}
\egroup
