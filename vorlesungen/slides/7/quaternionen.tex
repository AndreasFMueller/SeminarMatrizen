%
% quaternionen.tex -- slide template
%
% (c) 2021 Prof Dr Andreas Müller, OST Ostschweizer Fachhochschule
%
\bgroup
\begin{frame}[t]
\setlength{\abovedisplayskip}{5pt}
\setlength{\belowdisplayskip}{5pt}
\frametitle{Quaternionen}
\vspace{-20pt}
\begin{columns}[t,onlytextwidth]
\begin{column}{0.48\textwidth}
\begin{block}{Quaternionen}
$4$-dimensionaler $\mathbb{R}$-Vektorraum
\[
\mathbb{H}
=
\langle 1,i,j,k\rangle_{\mathbb{R}}
\]
mit Rechenregeln
\[
i^2=j^2=k^2=ijk=-1
\]
$x=x_0+x_1i+x_2j+x_3k\in\mathbb{H}$
\begin{itemize}
\item<2-> Realteil: $\operatorname{Re}x=x_0$
\item<3-> Vektorteil: $\operatorname{Im}x=x_1i+x_2j+x_3k$
\item<4-> Konjugation: $\overline{x}=\operatorname{Re}x-\operatorname{Im}x$
\item<5-> Norm: $|x|^2 = x\overline{x} = x_0^2+x_1^2+x_2^2+x_3^2$
\item<6-> Inverse: $x^{1}= \overline{x}/x\overline{x}$
\end{itemize}
\end{block}
\end{column}
\begin{column}{0.50\textwidth}
\uncover<7->{%
\begin{block}{Skalarprodukt und Vektorprodukt}
\begin{align*}
pq
&=
\operatorname{Re}p \operatorname{Re}q
-
\operatorname{Im}p\cdot \operatorname{Im}q
\\
&\phantom{=}
+
\operatorname{Re}p\operatorname{Im}q
+
\operatorname{Im}p\operatorname{Re}q
+
\operatorname{Im}p\times\operatorname{Im}q
\end{align*}
\end{block}}
\uncover<8->{%
\begin{block}{Einheitsquaternionen}
$q\in \mathbb{H}$, $|q|=1, q^{-1}=\overline{q}$
\end{block}}
\uncover<9->{%
\begin{block}{Polardarstellung}
\[
q = \cos\frac{\alpha}2 + \vec{n} \sin\frac{\alpha}2
\]
\vspace{-8pt}
\begin{itemize}
\item<10->
Drehmatrix: 9 Parameter, 6 Bedingungen
\item<11->
Quaternionen: 4 Parameter, 1 Bedingung
\end{itemize}
\end{block}}
\end{column}
\end{columns}
\end{frame}
\egroup
