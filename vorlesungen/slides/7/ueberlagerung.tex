%
% ueberlagerung.tex -- slide template
%
% (c) 2021 Prof Dr Andreas Müller, OST Ostschweizer Fachhochschule
%
\bgroup
\begin{frame}[t]
\setlength{\abovedisplayskip}{5pt}
\setlength{\belowdisplayskip}{5pt}
\frametitle{$S^3$, $\operatorname{SU}(2)$ und $\operatorname{SO}(3)$}
\vspace{-20pt}
\begin{columns}[t,onlytextwidth]
\begin{column}{0.38\textwidth}
\uncover<6->{%
\begin{block}{Überlagerung}
\begin{center}
\begin{tikzpicture}[>=latex,thick]
\coordinate (A) at (0,0);
\coordinate (B) at (2,0);
\coordinate (C) at (2,-2);
\coordinate (D) at (0,-2);

\uncover<7->{
\node at (A) {$\{\pm 1\}\mathstrut$};
}
\uncover<6->{
\node at (B) {$S^3\mathstrut$};
\node at ($(B)+(0.1,0)$) [right] {$=\operatorname{SU}(2)\mathstrut$};
}
\uncover<7->{
\node at (C) {$\operatorname{SO}(3)\mathstrut$};
\node at (D) {$\{I\}\mathstrut$};
}

\uncover<7->{
\draw[->,shorten >= 0.3cm,shorten <= 0.5cm] (A) -- (B);
\draw[->,shorten >= 0.3cm,shorten <= 0.3cm] (A) -- (D);
\draw[->,shorten >= 0.3cm,shorten <= 0.3cm] (B) -- (C);
\draw[->,shorten >= 0.6cm,shorten <= 0.3cm] (D) -- (C);
}

\end{tikzpicture}
\end{center}
\begin{itemize}
\item<7->
$\pm q\in S^3$ $\Rightarrow$ $\varrho_{q}=\varrho_{-q}$
\item<8->
In der Nähe von $I$ sehen die Gruppen
$\operatorname{SO}(3)$
und
$\operatorname{SU}(2)$ 
``gleich'' aus
\item<9->
$\operatorname{SU}(2)$ ist geometrisch ``einfacher''
\end{itemize}
\end{block}}
\end{column}
\begin{column}{0.58\textwidth}
\begin{block}{Pauli-Matrizen}
Quaternionen als $2\times 2$-Matrizen schreiben
\begin{align*}
1&=\begin{pmatrix}1&0\\0&1\end{pmatrix}=\sigma_0,
&
i&=\begin{pmatrix}0&i\\i&0\end{pmatrix}=-i\sigma_1
\\
j&=\begin{pmatrix}0&-1\\1&0\end{pmatrix}=-i\sigma_2,
&
k&=\begin{pmatrix}i&0\\0&-i\end{pmatrix}=-i\sigma_3
\end{align*}
\uncover<2->{%
erfüllen $i^2=j^2=k^2=ijk=-1$.}
\end{block}
\uncover<3->{%
\begin{block}{$S^3 = \operatorname{SU}(2)$}
\[
a+bi+cj+dk
=
\begin{pmatrix}
a+id&-c+bi\\
c+ib&a-id
\end{pmatrix}
=
A
\]
\begin{align*}
\uncover<4->{
\det A &= a^2 + b^2 + c^2 + d^2 = 1
}
\\
\uncover<5->{
A^* &= a - ib - jc - kd
}
\end{align*}
\end{block}}
\end{column}
\end{columns}
\end{frame}
\egroup
