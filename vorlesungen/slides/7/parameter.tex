%
% parameter.tex -- Parametrisierung der Matrizen
%
% (c) 2021 Prof Dr Andreas Müller, OST Ostschweizer Fachhochschule
%
\bgroup
\begin{frame}[t]
\setlength{\abovedisplayskip}{5pt}
\setlength{\belowdisplayskip}{5pt}
\frametitle{Drehungen Parametrisieren}
\vspace{-20pt}
\begin{columns}[t,onlytextwidth]
\begin{column}{0.4\textwidth}
\begin{block}{Drehung um Achsen}
\begin{align*}
D_{x,\alpha}
&=
\begin{pmatrix}
1&0&0\\0&\cos\alpha&-\sin\alpha\\0&\sin\alpha&\cos\alpha
\end{pmatrix}
\\
D_{y,\beta}
&=
\begin{pmatrix}
\cos\beta&0&-\sin\beta\\0&1&0\\\sin\beta&0&\cos\beta
\end{pmatrix}
\\
D_{z,\gamma}
&=
\begin{pmatrix}
\cos\gamma&-\sin\gamma&0\\\sin\gamma&\cos\gamma&0\\0&0&1
\end{pmatrix}
\end{align*}
\end{block}
\end{column}
\begin{column}{0.56\textwidth}
\begin{block}{Drehung um $\vec{\omega}$}
$\omega=|\vec{\omega}|=\mathstrut$Drehwinkel
\\
$\vec{k}=\vec{\omega}^0=\mathstrut$Drehachse
\[
\vec{x}
\mapsto
\cos\omega
\vec{x}
+
(\vec{k}\times\vec{x})\sin\omega
+
\vec{k}(\vec{k}\cdot\vec{x}) (1-\cos\omega)
\]
XXX TODO: Bild für Rodriguez Formel
\end{block}
\end{column}
\end{columns}
{\usebeamercolor[fg]{title}Dimension:} $\operatorname{SO}(3)$ ist eine
dreidimensionale Gruppe
\end{frame}
\egroup
