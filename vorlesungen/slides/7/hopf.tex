%
% hopf.tex -- slide template
%
% (c) 2021 Prof Dr Andreas Müller, OST Ostschweizer Fachhochschule
%
\bgroup
\begin{frame}[t]
\setlength{\abovedisplayskip}{5pt}
\setlength{\belowdisplayskip}{5pt}
\frametitle{Orbit-Räume}
\vspace{-20pt}
\begin{columns}[t,onlytextwidth]
\begin{column}{0.48\textwidth}
\begin{block}{Aktion von $\operatorname{SO}(3)$ auf $S^2$}
\begin{align*}
S^2 &= \{x\in\mathbb{R}^3\;|\; |x|=1\}
\\
\operatorname{SO}(3) \times S^2 &\to S^2: (g, x) \mapsto gx
\end{align*}
\uncover<2->{%
Allgemein: Aktion von $G$ auf $X$
\begin{align*}
\text{links:}&&
G\times X \to X &: (g,x) \mapsto gx
\\
\text{rechts:}&&
X\times G \to X &: (x,g) \mapsto xg
\end{align*}}
\end{block}
\vspace{-10pt}
\uncover<3->{%
\begin{block}{Stabilisator}
Zu $x\in X$ gibt es eine Untergruppe
\begin{align*}
G_x = \{g\in G\;|\; gx=x\},
\end{align*}
der {\em Stabilisator} von $x$.

\uncover<4->{%
Der Stabilisator von $v\in S^2$ ist die Gruppe der Drehungen um
die Achse $v$}
\end{block}}
\end{column}
\begin{column}{0.48\textwidth}
\uncover<5->{%
\begin{block}{Quotient}
$G$ operiert von rechts auf $X$
\[
X/G = \{ xG \;|\; x\in X\}
\]
heisst Quotient
\end{block}}
\uncover<6->{
\begin{block}{$\operatorname{SO}(3)/\operatorname{SO}(2)$}
Wähle $\operatorname{SO}(2)$ als Drehungen um die $z$-Achse:
\[
\operatorname{SO}(3) \to S^2
:
g \mapsto \text{letzte Spalte von $g$}
\]
\uncover<7->{Daher
\[
S^2 \cong \operatorname{SO}(3) / \operatorname{SO}(2)
\]}
\end{block}}
\end{column}
\end{columns}
\end{frame}
\egroup
