%
% test.tex collection of all slides
%
% (c) 2019 Prof Dr Andreas Müller, Hochschule Rapperswil
%

%\folie{3/motivation.tex}
%\folie{3/inverse.tex}
%\folie{3/polynome.tex}
%\folie{3/division.tex}
%\folie{3/division2.tex}
%\folie{3/ringstruktur.tex}
%\folie{3/teilbarkeit.tex}
%\folie{3/faktorisierung.tex}
%\folie{3/faktorzerlegung.tex}
%\folie{3/einsetzen.tex}
%\folie{3/maximalergrad.tex}
%\folie{3/minimalbeispiel.tex}
%\folie{3/minimalpolynom.tex}
%\folie{3/drehmatrix.tex}
%\folie{3/drehfaktorisierung.tex}
%\folie{3/fibonacci.tex}
%\folie{3/operatoren.tex}
%\folie{3/adjunktion.tex}
%\folie{3/adjalgebra.tex}

%\folie{4/ggt.tex}
%\folie{4/euklidmatrix.tex}
%\folie{4/euklidbeispiel.tex}
%\folie{4/euklidtabelle.tex}
%\folie{4/fp.tex}
%\folie{4/division.tex}
%\folie{4/gauss.tex}
% \folie{4/dh.tex}
% XXX \folie{4/frobenius.tex}

%\folie{4/divisionpoly.tex}
%\folie{4/euklidpoly.tex}
%\folie{4/polynomefp.tex}
%\folie{4/alpha.tex}

% XXX \folie{4/f2.tex}
%\folie{4/schieberegister.tex}

% XXX Idee der elliptischen Kurve
% XXX \folie{4/ecidee.tex}
                                                              

\section{Eigenwertproblem}
% XXX Motivation: beliebige Funktionen f(A) berechnen
%\folie{5/motivation.tex}
%\folie{5/charpoly.tex}

\section{Invariante Unterräume}
%\folie{5/kernbild.tex}
%\folie{5/ketten.tex}
%\folie{5/dimension.tex}
%\folie{5/folgerungen.tex}
%\folie{5/injektiv.tex}
%\folie{5/nilpotent.tex}
%\folie{5/eigenraeume.tex}
%\folie{5/zerlegung.tex}
%\folie{5/normalnilp.tex}
%\folie{5/bloecke.tex}

% Jordan Normalform
\section{Jordan-Normalform}
%\folie{5/jordanblock.tex}
%\folie{5/jordan.tex}
% XXX Diagonalform
% XXX \folie{5/diagonalform.tex}
%\folie{5/reellenormalform.tex}
% XXX \folie{5/hessenberg.tex}

\section{Satz von Cayley-Hamilton}
%\folie{5/cayleyhamilton.tex}

\section{Matrixnormen}
% XXX Vektornormen
%\folie{2/norm.tex}
% XXX Skalarprodukt und L^2-Norm
%\folie{2/skalarprodukt.tex}
% XXX Cauchy-Schwarz-Ungleichung
%\folie{2/cauchyschwarz.tex}
\folie{2/funktionenraum.tex}
% XXX Polarformel
%\folie{2/polarformel.tex}
% XXX Normen, die sich aus der Vektornorm ableiten lassen
%\folie{2/operatornorm.tex}
\folie{2/funktionenalgebra.tex}
\folie{2/linearformnormen.tex}
% XXX Frobenius-Norm 
%\folie{2/frobeniusnorm.tex}
%\folie{2/frobeniusanwendung.tex}

\section{Approximation mit Polynomen}
% XXX Stone-Weierstrass
% XXX \folie{5/stoneweierstrass.tex}
% XXX Spektrum einer Matrix
% XXX \folie{5/spektrum.tex}
% XXX Approximation einer Funktion auf dem Spektrum
% XXX \folie{5/spektrumapproximation.tex}
% XXX Approximation einer Matrix in der erzeugten Algebra
% XXX \folie{5/matrixapproximation.tex}
% XXX Gelfand-Transformation
% XXX \folie{5/gelfandtransformation.tex}

\section{Potenzreihen}
% XXX Konvergenzradius
% XXX \folie{5/konvergenzradius.tex}
% XXX Gelfand-Radius
% XXX \folie{5/gelfandradius.tex}
% XXX Gleichheit von Konvergenz-Radius und Gelfand-Radius (braucht JNF)
% XXX \folie{5/satzvongelfand.tex}

\section{Differentialgleichungen}
% XXX Potenzreihenmethode zur Lösung von Differentialgleichungen
% XXX \folie{5/potenzreihenmethode.tex}
% XXX Exponentialfunktion
% XXX \folie{5/exponentialfunktion.tex}
% XXX Exponentialreihe
% XXX \folie{5/exponentialreihe.tex}
% XXX Logarithmus
% XXX \folie{5/logarithmusreihe.tex}
% XXX Sinus und Cosinus, Eulerscher Satz
% XXX \folie{5/sinuscosinus.tex}

