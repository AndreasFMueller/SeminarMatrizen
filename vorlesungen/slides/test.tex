%
% test.tex collection of all slides
%
% (c) 2019 Prof Dr Andreas Müller, Hochschule Rapperswil
%

\folie{3/motivation.tex}
\folie{3/inverse.tex}
\folie{3/polynome.tex}
\folie{3/division.tex}
\folie{3/division2.tex}
\folie{3/ringstruktur.tex}
\folie{3/teilbarkeit.tex}
\folie{3/faktorisierung.tex}
\folie{3/faktorzerlegung.tex}
\folie{3/einsetzen.tex}
\folie{3/maximalergrad.tex}
\folie{3/minimalbeispiel.tex}
\folie{3/minimalpolynom.tex}
\folie{3/drehmatrix.tex}
\folie{3/drehfaktorisierung.tex}
\folie{3/operatoren.tex}
\folie{3/adjunktion.tex}
\folie{3/adjalgebra.tex}
% XXX \folie{3/adjunktioni.tex}
% XXX \folie{3/adjunktionsqrt2.tex}
% XXX \folie{3/adjunktionphi.tex}
% Adjunktion von \cos(\pi/1291) und \cos(\pi/1291)
% XXX \folie{3/adj1291.tex}

