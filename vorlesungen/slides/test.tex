%
% test.tex collection of all slides
%
% (c) 2019 Prof Dr Andreas Müller, Hochschule Rapperswil
%

%\folie{3/motivation.tex}
%\folie{3/inverse.tex}
%\folie{3/polynome.tex}
%\folie{3/division.tex}
%\folie{3/division2.tex}
%\folie{3/ringstruktur.tex}
%\folie{3/teilbarkeit.tex}
%\folie{3/faktorisierung.tex}
%\folie{3/faktorzerlegung.tex}
%\folie{3/einsetzen.tex}
%\folie{3/maximalergrad.tex}
%\folie{3/minimalbeispiel.tex}
%\folie{3/minimalpolynom.tex}
%\folie{3/drehmatrix.tex}
%\folie{3/drehfaktorisierung.tex}
%\folie{3/fibonacci.tex}
%\folie{3/operatoren.tex}
%\folie{3/adjunktion.tex}
%\folie{3/adjalgebra.tex}

%\folie{4/ggt.tex}
%\folie{4/euklidmatrix.tex}
%\folie{4/euklidbeispiel.tex}
%\folie{4/euklidtabelle.tex}
%\folie{4/fp.tex}
%\folie{4/division.tex}
% XXX \folie{4/gauss.tex}
% XXX \folie{4/dh.tex}
% XXX ? \folie{4/polynomefp.tex}
% XXX \folie{4/frobenius.tex}

% XXX \folie{4/ggtpoly.tex}
% XXX \folie{4/divisionpoly.tex}
% XXX \folie{4/euklidpoly.tex}

% XXX \folie{4/f2.tex}
% XXX \folie{4/schieberegister.tex}

% XXX Idee der elliptischen Kurve
% XXX \folie{4/ecidee.tex}
                                                              

\section{Eigenwertproblem}
% XXX Motivation: beliebige Funktionen f(A) berechnen
%\folie{5/motivation.tex}
%\folie{5/charpoly.tex}

\section{Invariante Unterräume}
%\folie{5/kernbild.tex}
%\folie{5/ketten.tex}
%\folie{5/dimension.tex}
%\folie{5/folgerungen.tex}
%\folie{5/injektiv.tex}
%\folie{5/nilpotent.tex}
%\folie{5/eigenraeume.tex}
%\folie{5/zerlegung.tex}
%\folie{5/normalnilp.tex}
%\folie{5/bloecke.tex}

% Jordan Normalform
\section{Jordan-Normalform}
%\folie{5/jordanblock.tex}
%\folie{5/jordan.tex}
% XXX Diagonalform
% XXX \folie{5/diagonalform.tex}
\folie{5/reellenormalform.tex}
% XXX \folie{5/hessenberg.tex}

\section{Satz von Cayley-Hamilton}
%\folie{5/cayleyhamilton.tex}


