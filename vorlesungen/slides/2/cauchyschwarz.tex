%
% cauchyschwarz.tex
%
% (c) 2021 Prof Dr Andreas Müller, OST Ostschweizer Fachhochschule
%
\bgroup
\definecolor{darkgreen}{rgb}{0,0.5,0}
\begin{frame}[t]
\setlength{\abovedisplayskip}{5pt}
\setlength{\belowdisplayskip}{5pt}
\frametitle{Cauchy-Schwarz-Ungleichung}
\vspace{-15pt}
\begin{columns}[t,onlytextwidth]
\begin{column}{0.48\textwidth}
\begin{block}{Satz (Cauchy-Schwarz)}
$\langle\;,\;\rangle$ eine positiv definite, hermitesche Sesquilinearform
\[
{\color{darkgreen}
|\operatorname{Re}\langle u,v\rangle|
\le
|\langle u,v\rangle|
\le
\|u\|_2\cdot \|v\|_2
}
\]
Gleichheit genau dann, wenn $u$ und $v$ linear abhängig sind
\end{block}
\begin{block}{Dreiecksungleichung}
\vspace{-12pt}
\begin{align*}
\|u+v\|_2^2
&=
\|u\|_2^2 + 2\operatorname{Re}\langle u,v\rangle + \|v\|_2^2
\\
&\le
\|u\|_2^2 + 2{\color{darkgreen}|\langle u,v\rangle|} + \|v\|_2^2
\\
&\le
\|u\|_2^2 + 2{\color{darkgreen}\|u\|_2\cdot \|v\|_2} + \|v\|_2^2
\\
&=(\|u\|_2 + \|v\|_2)^2
\end{align*}
\end{block}
\end{column}
\begin{column}{0.48\textwidth}
\uncover<2->{%
\begin{proof}[Beweis]
Die quadratische Funktion
\begin{align*}
Q(t)
&=
\langle u+tv,u+tv\rangle \ge 0
\\
\uncover<3->{
Q(t)
&=
\|u\|_2^2 + 2t\operatorname{Re}\langle u,v\rangle + t^2\|v\|_2^2}
\end{align*}
\uncover<4->{hat ihr Minimum bei}%
\begin{align*}
\uncover<5->{
t&=
-\operatorname{Re}\langle u,v\rangle/\|v\|_2^2}
\intertext{\uncover<6->{mit Wert}}
\uncover<7->{
Q(t)
&=
\|u\|_2^2
-2\operatorname{Re}\langle u,v\rangle^2/\|v\|_2^2}
\\
\uncover<7->{
&\qquad + \operatorname{Re}\langle u,v\rangle^2/\|v\|_2^2}
\\
\uncover<8->{
0
&\le
\|u\|_2^2-\operatorname{Re}\langle u,v\rangle^2/\|v\|_2^2}
\\
\uncover<9->{
\operatorname{Re}\langle u,v\rangle^2
&\le
\|u\|_2^2\cdot\|v\|_2^2}
\\
\uncover<10->{
\operatorname{Re}\langle u,v\rangle
&\le
\|u\|_2\cdot\|v\|_2}
\qedhere
\end{align*}
\end{proof}}
\end{column}
\end{columns}
\end{frame}
\egroup
