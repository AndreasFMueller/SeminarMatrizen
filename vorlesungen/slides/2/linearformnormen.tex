%
% linearformnormen.tex
%
% (c) 2021 Prof Dr Andreas Müller, OST Ostschweizer Fachhochschule
%
\begin{frame}[t]
\setlength{\abovedisplayskip}{5pt}
\setlength{\belowdisplayskip}{5pt}
\frametitle{Linearformen}
\vspace{-15pt}
\begin{columns}[t,onlytextwidth]
\begin{column}{0.48\textwidth}
\begin{block}{Linearformen $\varphi\colon L^1\to\mathbb{R}$}
Beispiel: $g\in C([a,b])$
\[
\varphi(f)
=
\int_a^b g(x)f(x)\,dx
\]
\uncover<2->{%
erfüllt
\begin{align*}
|\varphi(f)|
&=
\biggl|\int_a^b g(x)f(x)\,dx\biggr|
\\
\uncover<3->{
&\le \|g\|_\infty\cdot \|f\|_1
}
\end{align*}}
\uncover<4->{%
und hat daher die Operatornorm
\[
\|\varphi\|_{C([a,b])^*}
=
\|g\|_\infty
\]}
\end{block}
\end{column}
\begin{column}{0.48\textwidth}
\begin{block}{Linearformen $\varphi\colon L^2\to\mathbb{R}$}
\uncover<5->{%
Darstellungssatz von Riesz: $\exists g\in L^2$
\[
\varphi(f) = \langle g,f\rangle
\]}
\uncover<6->{%
erfüllt Cauchy-Schwarz}
\begin{align*}
\uncover<7->{
|\varphi(f)|
&=
|\langle g,f\rangle|}
\\
\uncover<8->{
&\le
\|g\|_2 \cdot \|f\|_2
}
\end{align*}
\uncover<9->{%
und hat daher die Operatornorm
\[
\|\varphi\|_{L^2([a,b])^*}
= \|g\|_2
\]}
\end{block}
\end{column}
\end{columns}

\vspace{8pt}
{\usebeamercolor[fg]{title}
\uncover<10->{%
$\Rightarrow$
Operatornorm hängt von den Vektorraumnormen ab}
}
\end{frame}
