%
% riesz.tex -- slide template
%
% (c) 2021 Prof Dr Andreas Müller, OST Ostschweizer Fachhochschule
%
\bgroup
\begin{frame}[t]
\setlength{\abovedisplayskip}{5pt}
\setlength{\belowdisplayskip}{5pt}
\frametitle{Darstellungssatz von Riesz}
\vspace{-20pt}
\begin{columns}[t,onlytextwidth]
\begin{column}{0.48\textwidth}
\begin{block}{Dualraum}
$V$ ein Vektorraum, $V^*$ der Raum aller Linearformen
\[
f\colon V\to \mathbb{C}
\]
\end{block}
\begin{block}{Beispiel: $l^\infty$}
$l^\infty=\text{beschränkte Folgen in $\mathbb{C}$}$,
Linearformen:
\begin{align*}
f(x)
&=
\sum_{i=0}^\infty f_ix_i
\\
\|f\|
&=
\sup_{\|x\|_{\infty}\le 1}
|f(x)|
=
\sum_{k\in\mathbb{N}} |f_k|
\\
\Rightarrow
l^{\infty*}
&=
l^1
\qquad(\ne l^2)
\\
&=\{\text{summierbare Folgen in $\mathbb{C}$}\}
\end{align*}

\end{block}
\end{column}
\begin{column}{0.48\textwidth}
\begin{block}{Beispiel: $\mathbb{C}^n$}
${\mathbb{C}^n}^* = \mathbb{C}^n$
\end{block}
\begin{theorem}[Riesz]
Zu einer stetigen Linearform $f\colon H\to\mathbb{C}$ gibt es $v\in H$ mit
\[
f(x) = \langle v,x\rangle
\quad\forall x\in H
\]
und $\|f\| = \|v\|$
\end{theorem}
\begin{block}{Dualraum von $H$}
$H^*=H$
\end{block}
Der Hilbertraum ist die ``intuitiv richtige, unendlichdimensionale''
Verallgemeinerung von $\mathbb{C}^n$
\end{column}
\end{columns}
\end{frame}
\egroup
