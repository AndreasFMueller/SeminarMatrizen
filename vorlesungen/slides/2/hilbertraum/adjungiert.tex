%
% adjungiert.tex -- slide template
%
% (c) 2021 Prof Dr Andreas Müller, OST Ostschweizer Fachhochschule
%
\bgroup
\begin{frame}[t]
\setlength{\abovedisplayskip}{5pt}
\setlength{\belowdisplayskip}{5pt}
\frametitle{Adjungierter Operator}
\vspace{-20pt}
\begin{columns}[t,onlytextwidth]
\begin{column}{0.48\textwidth}
\begin{block}{Definition}
\begin{itemize}
\item
$A\colon H\to L$ lineare Abbildung zwischen Hilberträumen, $y\in L$
\item
\[
H\to\mathbb{C}
:
x\mapsto \langle y, Ax\rangle_L
\]
ist eine lineare Abbildung $H\to\mathbb{C}$
\item
Nach dem Darstellungssatz gibt es $v\in H$ mit
\[
\langle y,Ax\rangle_L = \langle v,x\rangle_H
\quad 
\forall x\in H
\]
\end{itemize}
Die Abbildung 
\[
L\to H
:
y\mapsto v =: A^*y
\]
heisst {\em adjungierte Abbildung}
\end{block}
\end{column}
\begin{column}{0.48\textwidth}
\begin{block}{Endlichdimensional (Matrizen)}
\[
A^* = \overline{A}^t
\]
\end{block}
\vspace{-8pt}
\begin{block}{Selbstabbildungen}
Für Operatoren $A\colon H\to H$ ist $A^*\colon H\to H$
\[
\langle x,Ay\rangle
=
\langle A^*x, y\rangle
\quad
\forall x,y\in H
\]
\end{block}
\vspace{-8pt}
\begin{block}{Selbstadjungierte Operatoren}
\[
A=A^*
\;\Leftrightarrow\;
\langle x,Ay \rangle
=
\langle A^*x,y \rangle
=
\langle Ax,y \rangle
\]
Matrizen:
\begin{itemize}
\item hermitesch
\item für reelle Hilberträume: symmetrisch
\end{itemize}
\end{block}
\end{column}
\end{columns}
\end{frame}
\egroup
