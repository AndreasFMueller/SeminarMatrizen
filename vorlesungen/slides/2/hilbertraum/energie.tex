%
% energie.tex -- slide template
%
% (c) 2021 Prof Dr Andreas Müller, OST Ostschweizer Fachhochschule
%
\bgroup
\begin{frame}[t]
\setlength{\abovedisplayskip}{5pt}
\setlength{\belowdisplayskip}{5pt}
\frametitle{Energie --- Zeitentwicklung --- Schrödinger}
\vspace{-20pt}
\begin{columns}[t,onlytextwidth]
\begin{column}{0.30\textwidth}
\begin{block}{Totale Energie}
Hamilton-Funktion
\begin{align*}
H
&=
\frac12mv^2 + V(x)
\\
&=
\frac{p^2}{2m} + V(x)
\end{align*}
\end{block}
\begin{block}{Quantisierungsregel}
\begin{align*}
\text{Variable}&\to \text{Operator}
\\
x_k & \to x_k
\\
p_k & \to \frac{\hbar}{i} \frac{\partial}{\partial x_k}
\end{align*}
\end{block}
\end{column}
\begin{column}{0.66\textwidth}
\begin{block}{Energie-Operator}
\[
H
=
-\frac{\hbar^2}{2m}\Delta + V(x)
\]
\end{block}
\begin{block}{Eigenwertgleichung}
\[
-\frac{\hbar^2}{2m}\Delta\psi(x,t) + V(x)\psi(x,t) = E\psi(x,t)
\]
Zeitunabhängige Schrödingergleichung
\end{block}
\begin{block}{Zeitabhängigkeit = Schrödingergleichung}
\[
-\frac{\hbar}{i}
\frac{\partial}{\partial t}
\psi(x,t)
=
-\frac{\hbar^2}{2m}\Delta\psi(x,t) + V(x)\psi(x,t)
\]
Eigenwertgleichung durch Separation von $t$
\end{block}
\end{column}
\end{columns}
\end{frame}
\egroup
