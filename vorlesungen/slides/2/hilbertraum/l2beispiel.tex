%
% l2beispiel.tex -- slide template
%
% (c) 2021 Prof Dr Andreas Müller, OST Ostschweizer Fachhochschule
%
\bgroup
\begin{frame}[t]
\setlength{\abovedisplayskip}{5pt}
\setlength{\belowdisplayskip}{5pt}
\frametitle{Beispiele: $\mathbb{R},\mathbb{R}^2,\dots,\mathbb{R}^n,\dots,l^2$}
\vspace{-20pt}
\begin{columns}[t,onlytextwidth]
\begin{column}{0.48\textwidth}
\begin{block}{Definition}
\begin{itemize}
\item<2-> Quadratsummierbare Folgen von komplexen Zahlen
\[
l^2
=
\biggl\{
(x_k)_{k\in\mathbb{N}}\,\bigg|\, \sum_{k=0}^\infty |x_k|^2 < \infty
\biggr\}
\]
\item<3-> Skalarprodukt:
\begin{align*}
\langle x,y\rangle
&=
\sum_{k=0}^\infty \overline{x}_ky_k,
&
\uncover<4->{\|x\|^2 = \sum_{k=0}^\infty |x_k|^2}
\end{align*}
\item<5-> Vollständigkeit,
Konvergenz: Cauchy-Schwarz-Ungleichung
\[
\biggl|
\sum_{k=0}^\infty \overline{x}_ky_k
\biggr|
\le
\sum_{k=0}^\infty |x_k|^2
\sum_{l=0}^\infty |y_l|^2
\]
\end{itemize}
\end{block}
\end{column}
\begin{column}{0.48\textwidth}
\uncover<6->{%
\begin{block}{Standardbasisvektoren}
\begin{align*}
e_i
&=
(0,\dots,0,\underset{\underset{\textstyle i}{\textstyle\uparrow}}{1},0,\dots)
\\
\uncover<7->{(e_i)_k &= \delta_{ik}}
\end{align*}
\uncover<8->{sind orthonormiert:
\begin{align*}
\langle e_i,e_j\rangle
&=
\sum_k \overline{\delta}_{ik}\delta_{jk}
\uncover<9->{=
\delta_{ij}}
\end{align*}}
\end{block}}
\vspace{-16pt}
\uncover<10->{%
\begin{block}{Analyse}
$x_k$ kann mit Skalarprodukten gefunden werden:
\begin{align*}
\hat{x}_i
=
\langle e_i,x\rangle
&\uncover<11->{=
\sum_{k=0}^\infty \overline{\delta}_{ik}  x_k}
\uncover<12->{=
x_i}
\end{align*}
\uncover<13->{(Fourier-Koeffizienten)}
\end{block}}
\end{column}
\end{columns}
\end{frame}
\egroup
