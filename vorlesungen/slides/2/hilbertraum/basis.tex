%
% basis.tex -- slide template
%
% (c) 2021 Prof Dr Andreas Müller, OST Ostschweizer Fachhochschule
%
\bgroup
\begin{frame}[t]
\setlength{\abovedisplayskip}{5pt}
\setlength{\belowdisplayskip}{5pt}
\frametitle{Hilbert-Basis}
\vspace{-20pt}
\begin{columns}[t,onlytextwidth]
\begin{column}{0.48\textwidth}
\begin{block}{Definition}
Eine Menge $\mathcal{B}=\{b_k|k>0\}$ ist eine Hilbertbasis, wenn
\begin{itemize}
\item<2-> $\mathcal{B}$ ist orthonormiert: $\langle b_k,b_l\rangle=\delta_{kl}$
\item<3-> Der Unterraum $\langle b_k|k>0\rangle\subset H$ ist
dicht:
Jeder Vektor von $H$ kann beliebig genau durch Linearkombinationen von $b_k$
approximiert werden.
\end{itemize}
\uncover<4->{%
Ein Hilbertraum mit einer Hilbertbasis heisst {\em separabel}}
\end{block}
\uncover<5->{%
\begin{block}{Endlichdimensional}
Der Algorithmus bricht nach endlich vielen Schritten ab.
\end{block}}
\end{column}
\begin{column}{0.48\textwidth}
\uncover<6->{%
\begin{block}{Konstruktion}
Iterativ: $\mathcal{B}_0=\emptyset$
\begin{enumerate}
\item<7-> $V_k = \langle \mathcal{B}_k \rangle$
\item<8-> Wenn $V_k\ne H$, wähle einen Vektor
\begin{align*}
x\in V_k^{\perp}
&=
\{
x\in H\;|\; x\perp V_k
\}
\\
&=
\{x\in H\;|\;
x\perp y\;\forall y\in V_k
\}
\end{align*}
\item<9-> $b_{k+1} = x/\|x\|$
\[
\mathcal{B}_{k+1} = \mathcal{B}_k\cup \{b_{k+1}\}
\]
\end{enumerate}
\uncover<10->{%
Wenn $H$ separabel ist, dann ist
\[
\mathcal{B} = \bigcup_{k} \mathcal{B}_k
\]
eine Hilbertbasis für $H$}
\end{block}}
\end{column}
\end{columns}
\end{frame}
\egroup
