%
% definition.tex -- slide template
%
% (c) 2021 Prof Dr Andreas Müller, OST Ostschweizer Fachhochschule
%
\bgroup
\begin{frame}[t]
\setlength{\abovedisplayskip}{5pt}
\setlength{\belowdisplayskip}{5pt}
\frametitle{Hilbertraum --- Definition}
\vspace{-20pt}
\begin{columns}[t,onlytextwidth]
\begin{column}{0.48\textwidth}
\begin{block}{$\mathbb{C}$-Hilbertraum $H$}
\begin{enumerate}
\item<2-> $\mathbb{C}$-Vektorraum, muss nicht endlichdimensional sein
\item<3-> Sesquilineares Skalarprodukt
\[
\langle \cdot,\cdot\rangle
\colon H \to \mathbb{C}: (x,y) \mapsto \langle x,y\rangle
\]
Dazugehörige Norm:
\[
\|x\| = \sqrt{\langle x,x\rangle}
\]
\item<4-> Vollständigkeit: jede Cauchy-Folge konvergiert
\end{enumerate}
\uncover<5->{%
Ohne Vollständigkeit: {\em Prähilbertraum}}
\end{block}
\uncover<6->{%
\begin{block}{$\mathbb{R}$-Hilbertraum}
Vollständiger $\mathbb{R}$-Vektorraum mit bilinearem Skalarprodukt
\end{block}}
\end{column}
\begin{column}{0.48\textwidth}
\uncover<7->{%
\begin{block}{Vollständigkeit}
\begin{itemize}
\item<8-> $(x_n)_{n\in\mathbb{N}}$ ist eine Cauchy-Folge:
Für alle $\varepsilon>0$ gibt es $N>0$ derart, dass
\[
\| x_n-x_m\| < \varepsilon\quad\forall n,m>N
\]
\item<9-> Grenzwert existiert: $\exists x\in H$ derart, dass es für alle
$\varepsilon >0$ ein $N>0$ gibt derart, dass
\[
\|x_n-x\|<\varepsilon\quad\forall n>N
\]
\end{itemize}
\end{block}}
\uncover<10->{%
\begin{block}{Cauchy-Schwarz-Ungleichung}
\[
|\langle x,y\rangle|
\le  \|x\| \cdot \|y\|
\]
Gleichheit für linear abhängige $x$ und $y$
\end{block}}
\end{column}
\end{columns}
\end{frame}
\egroup
