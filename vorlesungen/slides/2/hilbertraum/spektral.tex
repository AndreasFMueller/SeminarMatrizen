%
% spektral.tex -- slide template
%
% (c) 2021 Prof Dr Andreas Müller, OST Ostschweizer Fachhochschule
%
\bgroup
\begin{frame}[t]
\setlength{\abovedisplayskip}{5pt}
\setlength{\belowdisplayskip}{5pt}
\frametitle{Spektraltheorie für selbstadjungierte Operatoren}
\vspace{-20pt}
\begin{columns}[t,onlytextwidth]
\begin{column}{0.48\textwidth}
\begin{block}{Voraussetzungen}
\begin{itemize}
\item
Hilbertraum $H$
\item
$A\colon H\to H$ linear
\end{itemize}
\end{block}
\uncover<2->{%
\begin{block}{Eigenwerte}
$x\in H$ ein EV von $A$ zum EW $\lambda\ne 0$
\begin{align*}
\uncover<3->{\langle x,x\rangle
&=
\frac1{\lambda}
\langle x,\lambda x\rangle}
\uncover<3->{=
\frac1{\lambda}
\langle x,Ax\rangle}
\\
&\uncover<4->{=
\frac1{\lambda}
\langle Ax,x\rangle}
\uncover<5->{=
\frac{\overline{\lambda}}{\lambda}
\langle x,x\rangle}
\\
\uncover<6->{\frac{\overline{\lambda}}{\lambda}&=1
\quad\Rightarrow\quad
\overline{\lambda} = \lambda}
\uncover<7->{\quad\Rightarrow\quad
\lambda\in\mathbb{R}}
\end{align*}
\end{block}}
\end{column}
\begin{column}{0.48\textwidth}
\uncover<8->{%
\begin{block}{Orthogonalität}
$u,v$ EV zu EW $\mu,\lambda\in \mathbb{R}\setminus\{0\}$, $\overline{\mu}=\mu\ne\lambda$
\begin{align*}
\uncover<9->{
\langle u,v\rangle
&=
\frac{1}{\mu}
\langle \mu u,v\rangle}
\uncover<10->{=
\frac{1}{\mu}
\langle Au,v\rangle}
\\
&\uncover<11->{=
\frac{1}{\mu}
\langle u,Av\rangle}
\uncover<12->{=
\frac{1}{\mu}
\langle u,\lambda v\rangle}
\uncover<13->{=
\frac{\lambda}{\mu}
\langle u,v\rangle}
\\
\uncover<14->{\Rightarrow
\;
0
&=
\underbrace{\biggl(\frac{\lambda}{\mu}-1\biggr)}_{\displaystyle \ne 0}
\langle u,v\rangle}
\uncover<15->{\;\Rightarrow\;
\langle u,v\rangle = 0}
\end{align*}
\uncover<16->{EV zu verschiedenen EW sind orthogonal}
\end{block}}
\end{column}
\end{columns}
\uncover<17->{%
\begin{block}{Spektralsatz}
Es gibt eine Hilbertbasis von $H$ aus Eigenvektoren von $A$
\end{block}}
\end{frame}
\egroup
