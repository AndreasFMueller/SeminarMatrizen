%
% l2.tex -- slide template
%
% (c) 2021 Prof Dr Andreas Müller, OST Ostschweizer Fachhochschule
%
\bgroup
\begin{frame}[t]
\setlength{\abovedisplayskip}{5pt}
\setlength{\belowdisplayskip}{5pt}
\frametitle{$L^2$-Hilbertraum}
\vspace{-20pt}
\begin{columns}[t,onlytextwidth]
\begin{column}{0.48\textwidth}
\begin{block}{Definition}
\begin{itemize}
\item<2->
Vektorraum: Funktionen
\[
f\colon [a,b] \to \mathbb{C}
\]
\item<3->
Sesquilineares Skalarprodukt
\[
\langle f,g\rangle
=
\int_a^b \overline{f(x)}\, g(x) \,dx
\]
\item<4->
Norm:
\[
\|f\|^2 = \int_a^b |f(x)|^2\,dx
\]
\item<5->
Vollständigkeit?
\uncover<6->{$\rightarrow$
Lebesgue Konvergenz-Satz}
\end{itemize}
\end{block}
\end{column}
\begin{column}{0.48\textwidth}
\uncover<7->{%
\begin{block}{Vollständigkeit}
\begin{itemize}
\item
Funktioniert nicht für Riemann-Integral
\item<8->
Erweiterung des Integrals auf das sogenannte Lebesgue-Integral (nach
Henri Lebesgue)
\item<9->
Abzählbare Mengen spielen keine Rolle $\rightarrow$ Nullmengen
\item<10->
Funktionen $\rightarrow$ Klassen von Funktionen, die sich auf einer Nullmenge
unterscheiden
\item<11->
Konvergenz-Satz von Lebesgue $\rightarrow$ es funktioniert
\end{itemize}
\end{block}}
\end{column}
\end{columns}
\end{frame}
\egroup
