%
% laplace.tex -- slide template
%
% (c) 2021 Prof Dr Andreas Müller, OST Ostschweizer Fachhochschule
%
\bgroup
\begin{frame}[t]
\setlength{\abovedisplayskip}{5pt}
\setlength{\belowdisplayskip}{5pt}
\frametitle{Höhere Dimension}
\vspace{-20pt}
\begin{columns}[t,onlytextwidth]
\begin{column}{0.44\textwidth}
\begin{block}{Problem}
Gegeben: $\Omega\subset\mathbb{R}^n$ ein Gebiet
\\
Gesucht: Lösungen von $\Delta u=0$ mit $u_{|\partial\Omega}=0$
\end{block}
\begin{block}{Funktionen}
Hilbertraum $H$ der Funktionen $f:\overline{\Omega}\to\mathbb{C}$
mit $f_{|\partial\Omega}=0$
\end{block}
\begin{block}{Skalarprodukt}
\[
\langle f,g\rangle
=
\int_{\Omega} \overline{f}(x) g(x)\,d\mu(x)
\]
\end{block}
\begin{block}{Laplace-Operator}
\[
\Delta \psi = \operatorname{div}\operatorname{grad}\psi
\]
\end{block}
\end{column}
\begin{column}{0.52\textwidth}
\begin{block}{Selbstadjungiert}
\begin{align*}
\langle f,\Delta g\rangle
&=
\int_{\Omega} \overline{f}(x)\operatorname{div}\operatorname{grad}g(x)\,d\mu(x)
\\
&=
\int_{\partial\Omega}
\underbrace{\overline{f}(x)}_{\displaystyle=0}\operatorname{grad}g(x)\,d\nu(x)
\\
&\qquad
-
\int_{\Omega}
\operatorname{grad}\overline{f}(x)\cdot \operatorname{grad}g(x)
\,d\mu(x)
\\
&=\int_{\Omega}\operatorname{div}\operatorname{grad}\overline{f}(x)g(x)\,d\mu(x)
\\
&=
\langle \Delta f,g\rangle
\end{align*}
\end{block}
\end{column}
\end{columns}
\end{frame}
\egroup
