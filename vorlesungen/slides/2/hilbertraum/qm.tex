%
% qm.tex -- slide template
%
% (c) 2021 Prof Dr Andreas Müller, OST Ostschweizer Fachhochschule
%
\bgroup
\begin{frame}[t]
\setlength{\abovedisplayskip}{5pt}
\setlength{\belowdisplayskip}{5pt}
\frametitle{Anwendung: Quantenmechanik}
\vspace{-20pt}
\begin{columns}[t,onlytextwidth]
\begin{column}{0.48\textwidth}
\begin{block}{Zustände (Wellenfunktion)}
$L^2$-Funktionen auf $\mathbb{R}^3$
\[
\psi\colon\mathbb{R}^3\to\mathbb{C}
\]
\end{block}
\vspace{-6pt}
\begin{block}{Wahrscheinlichkeitsinterpretation}
\[
|\psi(x)|^2 = \left\{
\begin{minipage}{4.6cm}\raggedright
Wahrscheinlichkeitsdichte für Position $x$ des Teilchens
\end{minipage}\right.
\]
\end{block}
\vspace{-6pt}
\begin{block}{Skalarprodukt}
\[
\langle\psi,\psi\rangle
=
\int_{\mathbb{R}^3} |\psi(x)|^2\,dx = 1
\]
\end{block}
\vspace{-6pt}
\begin{block}{Messgrösse $A$}
Selbstadjungierter Operator $A$
\\
$\rightarrow$
Hilbertbasis $|i\rangle$ von EV von $A$
\end{block}
\end{column}
\begin{column}{0.48\textwidth}
\begin{block}{Überlagerung}
\begin{align*}
|\psi\rangle
&=
\sum_i
w_i|i\rangle
\\
\langle \psi|\psi\rangle
&=
\sum_i |w_i|^2 \qquad\text{(Plancherel)}
\end{align*}
$|w_i|^2=|\langle \psi|i\rangle|^2$ Wahrscheinlichkeit für Zustand $|i\rangle$
\end{block}
\begin{block}{Erwartungswert}
\begin{align*}
E(A)
&=
\sum_i |w_i|^2 \alpha_i
=
\sum_i \overline{w}_i\alpha_i w_i 
\\
&=
\sum_{i,j} \overline{w}_j\alpha_i w_i \langle j|i\rangle
=
\sum_{i} \overline{w}_j\langle j| \sum_i \alpha_i w_i |i\rangle
\\
&=
\sum_{i,j} \overline{w}_j w_i \langle j|
A|i\rangle
=
\langle \psi| A |\psi\rangle
\end{align*}
\end{block}
\end{column}
\end{columns}
\end{frame}
\egroup
