%
% plancherel.tex -- slide template
%
% (c) 2021 Prof Dr Andreas Müller, OST Ostschweizer Fachhochschule
%
\bgroup
\begin{frame}[t]
\setlength{\abovedisplayskip}{5pt}
\setlength{\belowdisplayskip}{5pt}
\frametitle{Plancherel-Gleichung}
\vspace{-20pt}
\begin{columns}[t,onlytextwidth]
\begin{column}{0.48\textwidth}
\begin{block}{Hilbertraum mit Hilbert-Basis}
$H$ Hilbertraum mit Hilbert-Basis
$\mathcal{B}=\{b_k\;|\; k>0\}$, $x\in H$
\end{block}
\uncover<2->{%
\begin{block}{Analyse: Fourier-Koeffizienten}
\begin{align*}
a_k = \hat{x}_k &=\langle b_k, x\rangle
\\
\uncover<3->{\hat{x}&=\mathcal{F}x}
\end{align*}
\end{block}}
\vspace{-10pt}
\uncover<4->{%
\begin{block}{Synthese: Fourier-Reihe}
\begin{align*}
\tilde{x}
&=
\sum_k a_k b_k
\uncover<5->{=
\sum_k \langle x,b_k\rangle b_k}
\end{align*}
\end{block}}
\vspace{-6pt}
\uncover<6->{%
\begin{block}{Analyse von $\tilde{x}$}
\begin{align*}
\langle b_l,\tilde{x}\rangle
&=
\biggl\langle
b_l,\sum_{k}\langle b_k,x\rangle b_k
\biggr\rangle
\uncover<7->{=
\sum_k \langle b_k,x\rangle\langle b_l,b_k\rangle}
\uncover<8->{=
\sum_k \langle b_k,x\rangle\delta_{kl}}
\uncover<9->{=
\langle b_l,x\rangle}
\uncover<10->{=
\hat{x}_l}
\end{align*}
\end{block}}
\end{column}
\begin{column}{0.48\textwidth}
\uncover<11->{%
\begin{block}{Plancherel-Gleichung}
\begin{align*}
\|\tilde{x}\|^2
&=
\langle \tilde{x},\tilde{x}\rangle
=
\biggl\langle
\sum_k \hat{x}_kb_k,
\sum_l \hat{x}_lb_l
\biggr\rangle
\\
&\uncover<12->{=
\sum_{k,l} \overline{\hat{x}}_k\hat{x}_l\langle b_k,b_l\rangle}
\uncover<13->{=
\sum_{k,l} \overline{\hat{x}}_k\hat{x}_l\delta_{kl}}
\\
\uncover<14->{
\|\tilde{x}\|^2
&=
\sum_k |\hat{x}_k|^2}
\uncover<15->{=
\|\hat{x}\|_{l^2}^2}
\uncover<16->{=
\|\mathcal{F}x\|_{l^2}^2}
\end{align*}
\end{block}}
\vspace{-12pt}
\uncover<17->{%
\begin{block}{Isometrie}
\begin{align*}
\mathcal{F}
\colon
H \to l^2
\colon
x\mapsto \hat{x}
\end{align*}
\uncover<18->{Alle separablen Hilberträume sind isometrisch zu $l^2$ via
%Fourier-Transformation
$\mathcal{F}$}
\end{block}}
\end{column}
\end{columns}
\end{frame}
\egroup
