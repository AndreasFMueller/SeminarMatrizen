%
% sturm.tex -- slide template
%
% (c) 2021 Prof Dr Andreas Müller, OST Ostschweizer Fachhochschule
%
\bgroup
\begin{frame}[t]
\setlength{\abovedisplayskip}{5pt}
\setlength{\belowdisplayskip}{5pt}
\frametitle{Sturm-Liouville-Problem}
\vspace{-20pt}
\begin{columns}[t,onlytextwidth]
\begin{column}{0.48\textwidth}
\begin{block}{Wellengleichung}
Saite mit variabler Massedichte führt auf die DGL
\[
-y''(t) + q(t) y(t) = \lambda y(t),
\quad
q(t) > 0
\]
mit Randbedingungen $y(0)=y(1)=0$
\end{block}
\end{column}
\begin{column}{0.48\textwidth}
\uncover<2->{%
\begin{block}{Sturm-Liouville-Operator}
\[
A=-\frac{d^2}{dt^2} + q(t) = -D^2 + p
\]
auf differenzierbaren Funktionen $\Omega=[0,1]\to\mathbb{C}$ mit Randwerten
\[
f(0)=f(1)=0
\]
\end{block}}
\end{column}
\end{columns}
\uncover<3->{%
\begin{block}{Selbstadjungiert}
\begin{align*}
\langle f,Ag \rangle
&\uncover<4->{=
\langle f,-D^2 g\rangle + \langle f,qg\rangle
=
-
\int_0^1 \overline{f}(t) \frac{d^2}{dt^2}g(t)\,dt
+\langle f,qg\rangle}
\\
&\uncover<5->{=-\underbrace{[\overline{f}(t)g'(t)]_0^1}_{\displaystyle=0}
+\int_0^1 \overline{f}'(t)g'(t)\,dt
+\langle f,qg\rangle}
\uncover<6->{=-\int_0^1 \overline{f}''(t)g(t)\,dt
+\langle qf,g\rangle}
\\
&\uncover<7->{=\langle Af,g\rangle}
\end{align*}
\end{block}}
\end{frame}
\egroup
