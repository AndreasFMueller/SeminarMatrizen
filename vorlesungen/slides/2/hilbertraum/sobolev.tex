%
% sobolev.tex -- slide template
%
% (c) 2021 Prof Dr Andreas Müller, OST Ostschweizer Fachhochschule
%
\bgroup
\begin{frame}[t]
\setlength{\abovedisplayskip}{5pt}
\setlength{\belowdisplayskip}{5pt}
\frametitle{Sobolev-Raum}
\vspace{-20pt}
\begin{columns}[t,onlytextwidth]
\begin{column}{0.48\textwidth}
\begin{block}{Vektorrraum $W$}
Funktionen $f\colon \Omega\to\mathbb{C}$
\begin{itemize}
\item
$f\in L^2(\Omega)$
\item
$\nabla f\in L^2(\Omega)$
\item
homogene Randbedingungen:
$f_{|\partial \Omega}=0$
\end{itemize}
\end{block}
\begin{block}{Skalarprodukt}
\begin{align*}
\langle f,g\rangle_W
&=
\int_\Omega \overline{\nabla f}(x)\cdot\nabla g(x)\,d\mu(x)
\\
&\qquad + \int_{\Omega} \overline{f}(x)\,g(x)\,d\mu(x)
\\
&=\langle f,-\Delta g + g\rangle_{L^2(\Omega)}
\end{align*}
\end{block}
\end{column}
\begin{column}{0.48\textwidth}
\begin{block}{Vollständigkeit}
\dots
\end{block}
\begin{block}{Anwendung}
``Ein Hilbertraum für jedes partielle Differentialgleichungsproblem''
\end{block}
\end{column}
\end{columns}
\end{frame}
\egroup
