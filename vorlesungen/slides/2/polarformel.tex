%
% polarformel.tex
%
% (c) 2021 Prod Dr Andreas Müller, OST Ostschweizer Fachhochschule
%
\bgroup
\definecolor{darkcolor}{rgb}{0,0.6,0}
\def\yone{-2.1}
\def\ytwo{-3.55}
\def\ythree{-5.0}
\begin{frame}[t]
\setlength{\abovedisplayskip}{5pt}
\setlength{\belowdisplayskip}{5pt}
\frametitle{Polarformel}
\vspace{-5pt}
\begin{block}{Aufgabe}
$\langle x,y\rangle$ aus Werten von $\|\cdot\|_2$ rekonstruieren:

\end{block}
\begin{center}
\begin{tikzpicture}[>=latex,thick]

\node at (0,0) {$
\begin{aligned}
\uncover<2->{
\|x+ty\|_2^2
&=
\|x\|_2^2
+t\langle x,y\rangle
+\overline{t}\langle y,x\rangle
+ \|y\|_2^2}
\\
\uncover<3->{
&=
\|x\|_2^2
+t\langle x,y\rangle
+\overline{t\langle x,y\rangle}
+ \|y\|_2^2}
\\
\uncover<4->{
&=
\|x\|_2^2
+2\operatorname{Re}(t\langle x,y\rangle)
+ \|y\|_2^2}
\end{aligned}$};

\uncover<5->{
	\draw[->] (-1,-0.9) -- (-3.3,{\yone+0.25});
	\node at (-3.5,\yone) {$
	\|x\pm y\|_2^2
	=
	\|x\|_2^2
	\pm2\operatorname{Re}\langle x,y\rangle
	+
	\|y\|_2^2
	$};
}

\uncover<8->{
	\draw[->] (1,-0.9) -- (3.3,{\yone+0.25});
	\node at (3.5,\yone) {$
	\|x\pm iy\|_2^2
	=
	\|x\|_2^2
	\pm2i\operatorname{Im}\langle x,y\rangle
	+
	\|y\|_2^2
	$};
}

\uncover<6->{
	\draw[->] (-3.5,{\yone-0.2}) -- (-3.5,{\ytwo+0.2});
	\node at (-3.5,\ytwo) {$\operatorname{Re}\langle x,y\rangle
	=
	\frac12\bigl(
	\|x+y\|_2^2-\|x-y\|_2^2
	\bigr)
	$};
}

\uncover<9->{
	\draw[->] (3.5,{\yone-0.2}) -- (3.5,{\ytwo+0.2});
	\node at (3.5,\ytwo) {$
	\operatorname{Im}\langle x,y\rangle
	=
	\frac1{2i}\bigl(
	\|x+iy\|_2^2-\|x-iy\|_2^2
	\bigr)
	$};
}

\uncover<7->{
	\draw[->] (-3.3,{\ytwo-0.25}) -- (-1.5,{\ythree+0.25});
	\node at (0,\ythree) {$
	\langle x,y\rangle
	=
	\frac12\bigl(
	\|x+y\|_2^2-\|x-y\|_2^2
	\uncover<10->{
		+
		\|x+iy\|_2^2-\|x-iy\|_2^2
	}
	\bigr)$};
}

\uncover<10->{
	\draw[->] (3.3,{\ytwo-0.25}) -- (1.5,{\ythree+0.25});
}

\end{tikzpicture}
\end{center}
\end{frame}
\egroup
