%
% frobeniusanwendung.tex
%
% (c) 2021 Prof Dr Andreas Müller, OST Ostschweizer Fachhochschule
%
\begin{frame}[t]
\setlength{\abovedisplayskip}{5pt}
\setlength{\belowdisplayskip}{5pt}
\frametitle{Anwendung der Frobenius-Norm}
\vspace{-15pt}
\begin{columns}[t,onlytextwidth]
\begin{column}{0.48\textwidth}
\begin{block}{Ableitung nach $X\in M_{m\times n}(\mathbb{R})$}
Die Ableitung $Df=\partial f/\partial X$ der Funktion
$f\colon M_{m\times n}(\mathbb{R})\to \mathbb{R}$ ist die Matrix
mit Einträgen
\begin{align*}
\biggl(
\frac{\partial f}{\partial X}
\biggr)_{ij}
&=
\frac{\partial f}{\partial x_{ij}}
=
D_{ij}f
\end{align*}
\end{block}
\uncover<2->{%
\begin{block}{Richtungsableitung}
\uncover<5->{Die Matrix $Df$ ist ein Gradient:}
\begin{align*}
\frac{\partial}{\partial t}f(X+tY)\bigg|_{t=0}
&=\uncover<3->{
\sum_{i,j}
D_{ij} f(X) \cdot y_{ij}}
\\
&\uncover<4->{=
\langle D_{ij}f(X), Y\rangle_F}
\end{align*}
\end{block}}
\end{column}
\begin{column}{0.48\textwidth}
\uncover<6->{%
\begin{block}{Quadratische Minimalprobleme}
$A=A^t,B,X\in M_n(\mathbb{R})$, Minimum von
\begin{align*}
f(X)&=\langle X,AX\rangle_F + \langle B,X\rangle_F
\intertext{\uncover<7->{Folgerungen:}}
\uncover<8->{
\langle X,AY\rangle_F&=\langle AX,Y\rangle_F
}
\\
\uncover<9->{
D\langle B,\mathstrut\cdot\mathstrut\rangle_F
&=
B
}
\\
\uncover<10->{
D_X\langle X, AY\rangle_F
&=AY
}
\\
\uncover<11->{
D_Y\langle X, AY\rangle_F
&=AX
}
\\
\uncover<12->{
Df &= 2AX + B
}
\intertext{\uncover<13->{Minimum:}}
\uncover<14->{
X&=-\frac12 A^{-1}B
}
\end{align*}
\end{block}}
\end{column}
\end{columns}
\end{frame}
