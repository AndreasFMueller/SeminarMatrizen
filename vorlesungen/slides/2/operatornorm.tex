%
% operatorname.tex
%
% (c) 2021 Prof Dr Andreas Müller, OST Ostschweizer Fachhochschule
%
\begin{frame}[t]
\setlength{\abovedisplayskip}{5pt}
\setlength{\belowdisplayskip}{5pt}
\frametitle{Operatornorm}
\vspace{-15pt}
\begin{columns}[t,onlytextwidth]
\begin{column}{0.48\textwidth}
\uncover<2->{%
\begin{block}{Lineare Operatoren}
$A\colon U\to V$ lineare Abbildung mit $U$, $V$ normiert
\end{block}}
\uncover<3->{%
\begin{block}{Operatornorm}
eines linearen Operators $A$:
\[
\|A\| 
=
\sup_{\|x\|_U\le 1} \|Ax\|_V
\]
\uncover<4->{$\Rightarrow \|Ax\| \le \| A \|\cdot \|x\|$}
\end{block}}
\uncover<5->{%
\begin{block}{Stetigkeit}
Wenn $\|A\|<\infty$, dann ist $A$ stetig, d.~h.
\[
\lim_{n\to\infty} Ax_n
=
A\lim_{n\to\infty} x_n
\]
\end{block}}
\end{column}
\begin{column}{0.48\textwidth}
\uncover<6->{%
\begin{block}{Algebranorm}
$A$ ein normierter Raum, der auch ein Algebra ist.
Dann heisst $A$ eine normierte Algebra, wenn
\[
\| ab\| \le \| a\|\cdot \|b\|
\quad\forall a,b\in A
\]
\end{block}}
\vspace{-10pt}
\uncover<7->{%
\begin{block}{Operatoralgebra}
$U$ ein normierter Raum, dann ist die Algebra der linearen Operatoren
$A\colon U\to U$ mit der Operatornorm eine normierte Algebra
\end{block}}
\uncover<8->{%
\begin{block}{Banach-Algebra}
Ein Banach-Raum, der auch eine normierte Algebra ist
\end{block}}
\end{column}
\end{columns}
\end{frame}
