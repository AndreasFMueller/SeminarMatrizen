%
% norm.tex
%
% (c) 2021 Prof Dr Andreas Müller, OST Ostschweizer Fachhochschule
%
\begin{frame}[t]
\setlength{\abovedisplayskip}{5pt}
\setlength{\belowdisplayskip}{5pt}
\frametitle{Norm}
\vspace{-20pt}
\begin{columns}[t,onlytextwidth]
\begin{column}{0.48\textwidth}
\begin{block}{Wozu}
Ziel: Konvergenz von Folgen, Grenzwert in einem Vektorraum
\end{block}
\uncover<7->{%
\begin{block}{Cauchy-Folge}
Eine Folge $(x_n)_{n\in\mathbb{N}}$ von Vektoren in $V$ heisst
{\em Cauchy-Folge},
wenn es für alle $\varepsilon >0$ ein $N$ gibt mit
\[
\|x_n-x_m\| < \varepsilon\; \forall n,m>N
\]
\end{block}}
\vspace{-8pt}
\uncover<8->{%
\begin{block}{Grenzwert}
$x\in V$ heisst Grenzwert der Folge $x_n$, wenn es für alle $\varepsilon>0$
ein $N$ gibt mit
\[
\| x-x_n\| < \varepsilon \;\forall n>N
\]
\end{block}}
\end{column}
\begin{column}{0.48\textwidth}
\uncover<2->{%
\begin{block}{Definition}
$V$ ein $\mathbb{R}$-Vektorraum.
Eine Funktion
\[
\|\cdot\| \colon V \to \mathbb{R}_{\ge 0} : v \mapsto \|v\|
\]
heisst eine {\em Norm}, wenn
\begin{itemize}
\item<3-> $\| v \|>0$ für $v\ne 0$
\item<4-> $\|\lambda v\| = |\lambda|\cdot\|v\|$
\item<5-> $\| u + v \| \le \|u\| + \|v\|$ (Dreiecksungleichung)
\end{itemize}
\uncover<6->{%
Ein Vektorraum mit einer Norm heisst {\em normierter Raum}}
\end{block}}
\uncover<9->{%
\begin{block}{Banach-Raum}
Normierter Raum, in dem jede Cauchy-Folge einen Grenwzert hat
\end{block}}
\end{column}
\end{columns}
\end{frame}
