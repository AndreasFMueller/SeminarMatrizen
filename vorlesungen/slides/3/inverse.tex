%
% inverse.tex
%
% (c) 2021 Prof Dr Andreas Müller, OST Ostschweizer Fachhochschule
%
\begin{frame}[t]
\frametitle{Inverse Matrix}
\vspace{-20pt}
\begin{columns}[t,onlytextwidth]
\begin{column}{0.24\textwidth}
\begin{block}{Imaginäre Einheit}
\vspace{-15pt}
\begin{align*}
J &= \begin{pmatrix} 0&-1\\1&0\end{pmatrix}
\\
0&=
J^2 + I
\\
0&=
J+J^{-1}
\\
J^{-1}&=-J
\end{align*}
\end{block}
\end{column}
\begin{column}{0.25\textwidth}
\uncover<2->{%
\begin{block}{Wurzel $\sqrt{2}$}
\vspace{-15pt}
\begin{align*}
W&=\begin{pmatrix}0&2\\1&0\end{pmatrix}
\\
0 &= X^2-2
\\
0 &= W-2W^{-1}
\\
W^{-1}&=\frac12 W
\end{align*}
\end{block}}
\end{column}
\begin{column}{0.41\textwidth}
\uncover<3->{%
\begin{block}{Drehmatrix}
\vspace{-15pt}
\begin{align*}
D&=\begin{pmatrix}
\cos \frac{\pi}{1291} & -\sin\frac{\pi}{1291}\\
\sin \frac{\pi}{1291} &  \cos\frac{\pi}{1291}
\end{pmatrix}
\\
0 &= \ifthenelse{\boolean{presentation}}{\only<-3>{D^{1291}+I\phantom{+\frac{\mathstrut}{\mathstrut}}}}{}
\only<4->{D^2-2D\cos\frac{\pi\mathstrut}{1291\mathstrut}+I}
\\
0 &= \ifthenelse{\boolean{presentation}}{\only<-3>{D^{1290}+D^{-1}\phantom{+\frac{\mathstrut}{\mathstrut}}}}{}
\only<4->{D-2\cos\frac{\pi\mathstrut}{1291\mathstrut}+D^{-1}}
\\
D^{-1}
&= \only<-3>{-D^{1290}\phantom{+\frac{\mathstrut}{\mathstrut}}}%
\only<4->{-D+2I\cos\frac{\pi\mathstrut}{1291\mathstrut}}
\end{align*}
\end{block}}
\end{column}
\end{columns}
\vspace{-25pt}
\uncover<5->{
\begin{block}{3D-Beispiel}
$p(x) = -x^3-5x^2+5x+1$
\[
A=
\begin{pmatrix*}[r]
-5&-1&1\\
-5&-2&3\\
-1&-1&2
\end{pmatrix*}
\quad\Rightarrow\quad
A^{-1}
=
A^2+5A-5I
=
\begin{pmatrix*}[r]
-1& 1&-1\\
 7&-9&10\\
 3&-4& 5
\end{pmatrix*}
\]
\end{block}}
\vspace{-10pt}

\end{frame}
