%
% faktorisierung.tex
%
% (c) 2021 Prof Dr Andreas Müller, OST Ostschweizer Fachhochschule
%
\begin{frame}[t]
\frametitle{Faktorisierung}
\begin{columns}[t,onlytextwidth]
\begin{column}{0.48\textwidth}
\begin{block}{Primzahlen\strut}
Eine Zahl $p\in\mathbb{Z}$, $p>1$ heisst Primzahl, wenn sie nicht als Produkt
$p=ab$ mit $a,b\in\mathbb{Z},a>1, b>1$ geschrieben werden kann.
\begin{align*}
\uncover<2->{p&=7}
\\
\uncover<3->{2021 &= 43 \cdot 47}
\\
\uncover<4->{2048 &= 2^{11}}
\\
\uncover<5->{4095667&=2021\cdot 2027}
\\
\uncover<6->{p&=43, 47, 1291, 2017, 2027}
\end{align*}
\end{block}
\end{column}
\begin{column}{0.48\textwidth}
\uncover<7->{%
\begin{block}{Irreduzible Polynome in $\mathbb{Q}[X]$}
Ein Polynome $p\in\mathbb{Q}[X]$, $\deg p>0$ wenn es nicht als Produkt
$p=ab$ mit $a,b\in\mathbb{Q}[X]$, $\deg a>0$, $\deg b>0$ geschrieben
werden kann.
\begin{align*}
\uncover<8->{p&=X-9}
\\
\uncover<9->{X^2-1&= (X+1)(X-1)}
\\
\uncover<10->{X^2-2&\text{\; irreduzibel}}
\\
\uncover<11->{X^2-2&=(X-\sqrt{2})(X+\sqrt{2})}
\end{align*}
\uncover<12->{%
aber: $X\pm\sqrt{2}\not\in\mathbb{Q}[X]$
}
\end{block}}
\end{column}
\end{columns}
\end{frame}
