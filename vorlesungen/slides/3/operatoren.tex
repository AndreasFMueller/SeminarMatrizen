%
% operatoren.tex
%
% (c) 2021 Prof Dr Andreas Müller, OST Ostschweizer Fachhochschule
%
\begin{frame}[t]
\frametitle{$X$ als Operator}
\begin{columns}[t,onlytextwidth]
\begin{column}{0.38\textwidth}
\begin{block}{Polynome}
$a(X)=a_0+a_1X+\dots+a_nX^n$
\[
a(X)
=
\begin{pmatrix}
a_0\\a_1\\a_2\\a_3\\\vdots\\a_nX^n
\end{pmatrix}
\]
\end{block}
\end{column}
\begin{column}{0.58\textwidth}
\begin{block}{Multiplikation mit $X$}
\strut
\[
\begin{pmatrix}
1\\0\\0\\0\\\vdots\\0
\end{pmatrix}
\mapsto
\begin{pmatrix}
0\\1\\0\\0\\\vdots\\0
\end{pmatrix}
\mapsto
\begin{pmatrix}
0\\0\\1\\0\\\vdots\\0
\end{pmatrix}
\mapsto
\begin{pmatrix}
0\\0\\0\\1\\\vdots\\0
\end{pmatrix}
\mapsto\dots\mapsto
\begin{pmatrix}
0\\0\\0\\0\\\vdots\\1
\end{pmatrix}
\]
\end{block}
\end{column}
\end{columns}
\end{frame}
