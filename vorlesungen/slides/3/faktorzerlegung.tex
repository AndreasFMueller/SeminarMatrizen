%
% faktorzerlegung.tex
%
% (c) 2021 Prof Dr Andreas Müller, OST Ostschweizer Fachhochschule
%
\begin{frame}[t]
\frametitle{Faktorzerlegung}
\setlength{\abovedisplayskip}{5pt}
\setlength{\belowdisplayskip}{5pt}
\begin{columns}[t,onlytextwidth]
\begin{column}{0.48\textwidth}
\begin{block}{in $\mathbb{Z}$}
Jede Zahl kann eindeutig in Primfaktoren zerlegt werden:
\[
z = p_1^{n_1}\cdot p_2^{n_2} \cdot\dots\cdot p_k^{n_k}
\]
\end{block}
\end{column}
\begin{column}{0.48\textwidth}
\uncover<2->{%
\begin{block}{in $\mathbb{Q}[X]$}
Jedes Polynom $p\in\mathbb{Q}[X]$
kann eindeutig faktorisiert werden in irreduzible, normierte Polynome
\[
p
=
a_n
p_1^{n_1}
\cdot
p_2^{n_2}
\cdot
\dots
\cdot
p_k^{n_k}
\]
\end{block}}
\end{column}
\end{columns}
\uncover<3->{%
\begin{block}{Polynomfaktorisierung hängt vom Koeffizientenring ab}
Ist $X^2-2$ irreduzibel?
\vspace{-5pt}
\begin{columns}[t,onlytextwidth]
\begin{column}{0.48\textwidth}
\uncover<4->{%
\begin{block}{in $\mathbb{Q}[X]$}
\[
X^2-2\quad\text{ist irreduzibel}
\]
\end{block}}
\end{column}
\begin{column}{0.48\textwidth}
\uncover<5->{%
\begin{block}{in $\mathbb{R}[X]$}
\[
X^2-2 = (X-\sqrt{2})(X+\sqrt{2})
\]
\end{block}}
\end{column}
\end{columns}
\end{block}}
\end{frame}
