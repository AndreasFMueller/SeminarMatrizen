%
% adjalgebra.tex
%
% (c) 2021 Prof Dr Andreas Müller, OST Ostschweizer Fachhochschule
%
\begin{frame}[t]
\frametitle{Adjunktion einer Nullstelle, abstrakt}
\setlength{\abovedisplayskip}{5pt}
\setlength{\belowdisplayskip}{5pt}
Sei $m(X)=m_0+m_1X+\dots + X^n\in \Bbbk[X]$ ein irreduzibles Polynom.

\uncover<2->{%
\begin{block}{Existenz}
Es gibt ein ``Objekt'' $\alpha$ mit
\(
m(\alpha) = 0
\)
\end{block}}

\uncover<3->{%
\begin{block}{Körpererweiterung}
Der kleinste Körper, der $\Bbbk$ und $\alpha$ enthält ist
\[
\Bbbk(\alpha)
=
\left
\{ p(\alpha)
\;\left|\;
\begin{minipage}{8cm}\raggedright
$p\in\Bbbk[X]$ ein Polynom vom Grad
$\deg p<\deg m$
\end{minipage}
\right.
\right\}
\]
\uncover<4->{Das Polynom $m$ definiert, wie mit $\alpha$ gerechnet werden
muss:
\[
\alpha^n = -m_0-m_1\alpha-m_2\alpha^2 - \dots - m_{n-1}\alpha^{n-1}
\]}
\end{block}}

\end{frame}
