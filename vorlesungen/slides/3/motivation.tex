%
% motivation.tex
%
% (c) 2021 Prof Dr Andreas Müller, OST Ostschweizer Fachhochschule
%
\begin{frame}[t]
\frametitle{Motivation}
\vspace{-20pt}
\begin{columns}[t,onlytextwidth]
\begin{column}{0.24\textwidth}
\begin{block}{Imaginäre Einheit}
\vspace{-15pt}
\begin{align*}
J &= \begin{pmatrix} 0&-1\\1&0\end{pmatrix}
\\
p(X) &= X^2 + 1
\\
p(J) &= J^2 + I = 0
\end{align*}
\end{block}
\end{column}
\begin{column}{0.25\textwidth}
\uncover<2->{%
\begin{block}{Wurzel $\sqrt{2}$}
\vspace{-15pt}
\begin{align*}
W&=\begin{pmatrix}0&2\\1&0\end{pmatrix}
\\
p(X) &= X^2-2
\\
p(W) &= W^2-2I=0
\end{align*}
\end{block}}
\end{column}
\begin{column}{0.41\textwidth}
\uncover<3->{%
\begin{block}{Drehmatrix}
\vspace{-15pt}
\begin{align*}
D&=\begin{pmatrix}
\cos \frac{\pi}{1291} & -\sin\frac{\pi}{1291}\\
\sin \frac{\pi}{1291} &  \cos\frac{\pi}{1291}
\end{pmatrix}
\\
p(X)&=
\ifthenelse{\boolean{presentation}}{\only<-3>{X^{1291}+1\phantom{+\frac{\mathstrut}{\mathstrut}}}}{}
\only<4->{X^2-2X\cos\frac{\pi\mathstrut}{1291\mathstrut}+I}
\\
p(D) &= \ifthenelse{\boolean{presentation}}{\only<-3>{D^{1291}+I\phantom{+\frac{\mathstrut}{\mathstrut}}}}{}
\only<4->{D^2-2D\cos\frac{\pi\mathstrut}{1291\mathstrut}+I}
\end{align*}
\end{block}}
\end{column}
\end{columns}
\vspace{-20pt}
\uncover<5->{
\begin{block}{3D-Beispiel}
$p(x) = -x^3-5x^2+5x+1$
\[
\ifthenelse{\boolean{presentation}}{
\only<5-8>{
A=
\begin{pmatrix*}[r]
-5&-1&1\\
-5&-2&3\\
-1&-1&2
\end{pmatrix*}}
\only<6-8>{
\quad\Rightarrow\quad}}{}
\uncover<6->{
-
\only<-9>{A^3}\only<10->{
\begin{pmatrix*}[r]
-169&-35&35\\
-185&-39&40\\
 -45&-10&11
\end{pmatrix*}}
-5
\only<-8>{A^2}\only<9->{
\begin{pmatrix*}[r]
29&6&-6\\
32&6&-5\\
 8&1& 0
\end{pmatrix*}}
+5
\only<-7>{A}\only<8->{
\begin{pmatrix*}[r]
-5&-1&1\\
-5&-2&3\\
-1&-1&2
\end{pmatrix*}}
+
\only<-6>{I}\only<7->{
\begin{pmatrix*}[r]
1&0&0\\
0&1&0\\
0&0&1
\end{pmatrix*}}
}
\uncover<11->{=0}
\]
\end{block}}
\vspace{-10pt}
\uncover<12->{%
{\usebeamercolor[fg]{title}$\Rightarrow$
Rechenregeln von Matrizen können durch Polynome ausgedrückt werden}
}
\end{frame}
