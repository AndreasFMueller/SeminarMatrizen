%
% teilbarkeit.tex
%
% (c) 2021 Prof Dr Andreas Müller, OST Ostschweizer Fachhochschule
%
\begin{frame}[t]
\frametitle{Teilen}
\vspace{-15pt}
\begin{columns}[t,onlytextwidth]
\begin{column}{0.48\textwidth}
\begin{block}{Teilen in $\mathbb{Z}$}
Zu zwei Zahlen $a,b\in \mathbb{Z}$, \only<3->{{\color<3-4>{red}$a>b$}, }gibt es
immer \only<3->{{\color<3-4>{red}genau}} ein Paar $q,r\in\mathbb{Z}$ derart, dass
\begin{align*}
a&=bq+r
\\
\uncover<3->{{\color<3-4>{red}r}&{\color<3-4>{red}< b}}
\end{align*}
\end{block}
\end{column}
\begin{column}{0.48\textwidth}
\uncover<2->{%
\begin{block}{Teilen in $\mathbb{Q}[X]$}
Zu zwei Polynomen $a,b\in\mathbb{Q}[X]$, \only<4->{{\color<4>{red}$\deg a > \deg b$},}
gibt es 
immer \only<4->{{\color<4>{red}bis auf eine Einheit genau }}%
ein Paar $q,r\in\mathbb{Q}[X]$ derart, dass
\begin{align*}
a&=bq+r
\\
\uncover<4->{{\color<4>{red}\deg r}&{\color<4>{red}< \deg b}}
\end{align*}
\end{block}}
\end{column}
\end{columns}
\uncover<5->{%
\begin{block}{Allgemein: euklidischer Ring}
Nullteilerfreier Ring $R$ mit einer Funktion
$d\colon R\setminus{0}\to\mathbb{N}$ mit
\begin{itemize}
\item Für $x,y\in R$ gilt $d(xy) \ge d(x)$.
\item Für $x,y\in R$ gibt es $q,r\in R$ derart
$x=qy +r$ mit $d(y)>d(r)$
\end{itemize}
Euklidische Ringe haben ähnliche Eigenschaften wie Polynomringe
\end{block}}
\end{frame}
