%
% minimalbeispiel.tex
%
% (c) 2021 Prof Dr Andreas Müller, OST Ostschweizer Fachhochschule
%
\begin{frame}[t]
\frametitle{Beispiel für $p(A)=0$}
\begin{block}{Potenzen einer $2\times 2$-Matrix $A$}
\setlength{\abovedisplayskip}{5pt}
\setlength{\belowdisplayskip}{5pt}
\vspace{-10pt}
\[
I  ={\tiny\begin{pmatrix}  1 &  0 \\  0 &  1 \end{pmatrix}},\quad
A  ={\tiny\begin{pmatrix}  3 &  2 \\ -1 & -2 \end{pmatrix}},\quad
\uncover<2->{A^2={\tiny\begin{pmatrix}  7 &  2 \\ -1 &  2 \end{pmatrix}}}
\uncover<3->{,\quad A^3={\tiny\begin{pmatrix} 19 & 10 \\ -5 & -6 \end{pmatrix}}}
\uncover<4->{,\quad A^4={\tiny\begin{pmatrix} 47 & 18 \\ -9 &  2 \end{pmatrix}}}
\]
\end{block}
\vspace{-5pt}
\uncover<5->{%
\begin{block}{linear abhängig}
Bereits die ersten $3$ sind linear abhängig:
\[
-4I - A + A^2
=
-4\begin{pmatrix}  1 &  0 \\  0 &  1 \end{pmatrix}
-\begin{pmatrix}  3 &  2 \\ -1 & -2 \end{pmatrix}
+\begin{pmatrix}  7 &  2 \\ -1 &  2 \end{pmatrix}
= 
\begin{pmatrix} 0 & 0 \\ 0 & 0 \end{pmatrix}
\]
\uncover<6->{$p(X) = X^2 - X - 4 \in \mathbb{Q}[X]$ hat die Eigenschaft
$p(A)=0$}
\end{block}}
\end{frame}
