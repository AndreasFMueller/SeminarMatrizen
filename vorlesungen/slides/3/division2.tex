%
% division2.tex
%
% (c) 2021 Prof Dr Andreas Müller, OST Ostschweizer Fachhochschule
%
\begin{frame}[t]
\frametitle{Division in $\Bbbk[X]$}
\vspace{-5pt}
\begin{block}{Aufgabe}
Finde Quotienten und Rest der Polynome
$a(X) = X^4-X^3-7X^2+X+6$
und
$b(X) = 2X^2+X+1$
\end{block}
\begin{block}{Lösung}
\[
\arraycolsep=1.4pt
\renewcommand{\arraystretch}{1.2}
\begin{array}{rcrcrcrcrcrcrcrcrcrcr}
X^4&-&       X^3&-&         7X^2&+&          X&+&           6&:&2X^2&+&X&+&1&=&\frac12X^2&-&\frac34X&-\frac{27}{8} = q\\
\llap{$-($}X^4&+&\frac12X^3&+&   \frac12X^2\rlap{$)$}& &           & &            & &    & & & & & &          & &        &             \\ \cline{1-5}
   &-&\frac32X^3&-&\frac{15}2X^2&+&          X& &            & &    & & & & & &          & &        &             \\
   &\llap{$-($}-&\frac32X^3&-&\frac{ 3}4X^2&-&\frac{ 3}4X\rlap{$)$}& &            & &    & & & & & &          & &        &             \\\cline{2-7}
   & &          &-&\frac{27}4X^2&+&\frac{ 7}4X&+&           6& &    & & & & & &          & &        &             \\
   & &          &\llap{$-($}-&\frac{27}4X^2&-&\frac{27}8X&-&\frac{27}{8}\rlap{$)$}& &    & & & & & &          & &        &             \\\cline{4-9}
   & &          & &             & &\frac{41}8X&+&\frac{75}{8}\rlap{$\mathstrut=r$}& &    & & & & & &          & &        &             \\
\end{array}
\]
Funktioniert, weil man in $\Bbbk[X]$ immer normieren kann
\end{block}

\end{frame}
