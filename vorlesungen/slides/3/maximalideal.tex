%
% maximalideal.tex
%
% (c) 2021 Prof Dr Andreas Müller, OST Ostschweizer Fachhochschule
%
\begin{frame}[t]
\frametitle{Maximale Ideale}
\vspace{-20pt}
\begin{columns}[t,onlytextwidth]
\begin{column}{0.48\textwidth}
\begin{block}{Teilbarkeit}
$a|b$
\uncover<2->{$\Rightarrow$
$b\in aR$}
\uncover<3->{$\Rightarrow$
$bR\subset aR$}
\end{block}
\uncover<4->{%
\begin{block}{Nicht mehr teilbar}
$a\in R$ nicht faktorisierbar
\\
\uncover<5->{$\Rightarrow$
\\
es gibt kein Ideal zwischen $aR$ und $R$}
\\
\uncover<6->{$\Leftrightarrow$
\\
$J$ ein Ideal
$aR \subset J \subset R$, dann ist
$J=aR$ oder $J=R$}
\end{block}}
\uncover<7->{
\begin{block}{maximales Ideal}
$I\subset R$ heisst maximal, wenn für jedes Ideal $J$
mit $I\subset J\subset R$ gilt
$I=J$ oder $J=R$
\end{block}}
\end{column}
\begin{column}{0.48\textwidth}
\uncover<8->{
\begin{block}{Beispiele}
\begin{itemize}
\item Primzahlen $p$ erzeugen maximale Ideale in $\mathbb{Z}$
\item<9-> Irreduzible Polynome erzeugen maximale Ideale in $\Bbbk[X]$
\end{itemize}
\end{block}}
\uncover<10->{%
\begin{block}{Körper}
$M\subset R$ ein maximales Ideal, dann ist
$R/M$ ein Körper
\end{block}}
\uncover<11->{%
\begin{block}{Beispiel}
\begin{itemize}
\item
$\mathbb{F}_p = \mathbb{Z}/p\mathbb{Z}$
\item<12->
$m$ ein irreduzibles Polynom:
$\Bbbk[X]/ (m)$ ist ein Körper
\end{itemize}
\end{block}}
\end{column}
\end{columns}
\end{frame}
