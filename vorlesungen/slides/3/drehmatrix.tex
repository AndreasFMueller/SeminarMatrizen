%
% drehmatrix.tex
%
% (c) 2021 Prof Dr Andreas Müller, OST Ostschweizer Fachhochschule
%
\begin{frame}[t]
\frametitle{Analyse einer Drehung um $120^\circ$}
$D$ eine Drehung des $\mathbb{R}^3$ um $120^\circ$
\begin{enumerate}
\item<2->
Drehwinkel = $120^\circ\quad\Rightarrow\quad D^3 = I$
\uncover<3->{
$\quad\Rightarrow\quad \chi_D(X)=X^3-1$
}
\item<4->
$m_D(X)=X^3-1$
\item<5->
$m_D$ ist nicht irreduzibel, weil $m_D(1)=0$:
$
m_D(X) = (X-1)(X^2+X+1)
$
\item<6->
Welche Matrix hat $X^2+X+1$ als Minimalpolynom?
\uncover<7->{%
\[
\arraycolsep=1.4pt
W
=
\biggl(\begin{array}{cc}
-\frac12          & -\frac{\sqrt{3}}2 \\
 \frac{\sqrt{3}}2 & -\frac12
\end{array}\biggr)
\quad\Rightarrow\quad
W^2+W+I
=
\biggl(\begin{array}{cc}
-\frac12          & -\frac{\sqrt{3}}2 \\
 \frac{\sqrt{3}}2 & -\frac12
\end{array}\biggr)
+
\biggl(\begin{array}{cc}
-\frac12          & \frac{\sqrt{3}}2 \\
 -\frac{\sqrt{3}}2 & -\frac12
\end{array}\biggr)
+
\biggl(\begin{array}{cc}
1&0\\0&1
\end{array}\biggr)
=0
\]}
\item<8-> In einer geeigneten Basis hat $D$ die Form
\[
D=\begin{pmatrix}
1&0&0\\
0&-\frac12 & -\frac{\sqrt{3}}2 \\
0&\frac{\sqrt{3}}2 & -\frac12
\end{pmatrix}
\uncover<9->{=
\begin{pmatrix}
1&0&0\\
0&\cos 120^\circ & -\sin 120^\circ\\
0&\sin 120^\circ & \cos 120^\circ
\end{pmatrix}}
\]
\end{enumerate}
\end{frame}
