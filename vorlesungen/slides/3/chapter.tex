%
% chapter.tex
%
% (c) 2021 Prof Dr Andreas Müller, Hochschule Rapperswi
%
\folie{3/motivation.tex}
\folie{3/inverse.tex}
\folie{3/polynome.tex}
\folie{3/division.tex}
\folie{3/division2.tex}
\folie{3/ringstruktur.tex}
\folie{3/teilbarkeit.tex}
\folie{3/ideal.tex}
\folie{3/nichthauptideal.tex}
\folie{3/nichthauptideal2.tex}
\folie{3/maximalideal.tex}
\folie{3/idealverband.tex}
\folie{3/quotientenring.tex}
\folie{3/faktorisierung.tex}
\folie{3/faktorzerlegung.tex}
\folie{3/einsetzen.tex}
\folie{3/maximalergrad.tex}
\folie{3/minimalbeispiel.tex}
\folie{3/fibonacci.tex}
\folie{3/minimalpolynom.tex}
\folie{3/drehmatrix.tex}
\folie{3/drehfaktorisierung.tex}
\folie{3/operatoren.tex}
\folie{3/adjunktion.tex}
\folie{3/adjalgebra.tex}
\folie{3/wurzel2.tex}
\folie{3/phi.tex}
\folie{3/multiplikation.tex}
