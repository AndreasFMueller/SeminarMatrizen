%
% fibonacci.tex
%
% (c) 2021 Prof Dr Andreas Müller, OST Ostschweizer Fachhochschule
%

\begin{frame}[t]
\frametitle{Fibonacci}
\setlength{\abovedisplayskip}{5pt}
\setlength{\belowdisplayskip}{5pt}
\begin{block}{Fibonacci-Rekursion}
$x_i$ Fibonacci-Zahlen\uncover<2->{, d.~h.~$x_{n+1\mathstrut}=x_{n\mathstrut}+x_{n-1\mathstrut}$}
\[
\uncover<3->{
v_n
=
\begin{pmatrix}
x_{n+1}\\
x_n
\end{pmatrix}}
\uncover<4->{
\quad\Rightarrow\quad
v_n = 
\underbrace{
\begin{pmatrix}
1&1\\
1&0
\end{pmatrix}
}_{\displaystyle=\Phi}
v_{n-1}}
\uncover<5->{
\quad\Rightarrow\quad
v_n
=
\Phi^n
v_0}\uncover<6->{,
\;
v_0 = \begin{pmatrix} 1\\0\end{pmatrix}}
\]
\end{block}
\vspace{-5pt}
\uncover<7->{%
\begin{block}{Rekursionsformel für $\Phi$}
\vspace{-12pt}
\begin{align*}
v_{n}&=v_{n-1}+v_{n-2}
&&\uncover<8->{\Rightarrow&
\Phi^n v_0 &= \Phi^{n-1} v_0 + \Phi^{n-2}v_0}
&&\uncover<9->{\Rightarrow&
\Phi^{n-2}(\Phi^2-\Phi-I)v_0&=0}
\\
\end{align*}
\vspace{-22pt}%

\uncover<10->{$\Phi$ ist $\chi_\Phi(X)=m_\Phi(X) = X^2-X-1$, irreduzibel}
\end{block}}

\uncover<11->{%
\begin{block}{Faktorisierung}
\vspace{-12pt}
\[
(X-\Phi)(X-(I-\Phi))
\uncover<12->{=
X^2-X +\Phi(I-\Phi)}
\uncover<13->{=
X^2-X -(\underbrace{\Phi^2-\Phi}_{\displaystyle=I})
}
\]
\end{block}}

\end{frame}
