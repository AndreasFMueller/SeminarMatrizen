%
% drehfaktorisierung.tex
%
% (c) 2021 Prof Dr Andreas Müller, OST Ostschweizer Fachhochschule
%
\begin{frame}[t]
\setlength{\abovedisplayskip}{4pt}
\setlength{\belowdisplayskip}{4pt}
\frametitle{Faktorisierung von $X^2+X+1$}
\vspace{-3pt}
$X^2+X+1$ kann faktorisiert werden, wenn man $i\sqrt{3}$
hinzufügt:
\uncover<2->{%
\[
\biggl(X+\frac12+\frac{i\sqrt{3}}2\biggr)
\biggl(X+\frac12-\frac{i\sqrt{3}}2\biggr)
=
X^2+X+\frac14
+
\frac34
\uncover<3->{=
X^2+X+1}
\]}
\vspace{-10pt}
\uncover<4->{%
\begin{block}{Was ist $i\sqrt{3}$?}
Matrix mit Minimalpolynom $X^2+3$:
\[
W=\begin{pmatrix}0&-3\\1&0\end{pmatrix}
\uncover<5->{%
\qquad\Rightarrow\qquad
W^2=\begin{pmatrix}3&0\\0&3\end{pmatrix} = -3I}
\uncover<6->{%
\qquad\Rightarrow\qquad
W^2+3I=0}
\]
\end{block}}
\vspace{-10pt}
\uncover<7->{%
\begin{block}{Faktorisierung von $X^2+X+1$}
\vspace{-10pt}
\begin{align*}
\uncover<8->{B_\pm
&=
-\frac12I\pm\frac12W}
&
&\uncover<10->{\Rightarrow
&
(X+B_+)(X+B_-)}
&\uncover<11->{=
(X+\frac12I+\frac12W)
(X+\frac12I-\frac12W)}
\\
&\uncover<9->{=
\smash{
{\textstyle\begin{pmatrix}-\frac12&-\frac32\\\frac12&-\frac12\end{pmatrix}}
}}
&
&
&
&\uncover<12->{=
X^2+X + \frac14I - \frac14W^2}
\\
&
&
&%\Rightarrow
&
&\uncover<13->{=
X^2+X + \frac14I + \frac34I}
\uncover<14->{=
X^2+X+I}
\end{align*}
\end{block}}

\end{frame}
