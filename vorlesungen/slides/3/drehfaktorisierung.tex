%
% drehfaktorisierung.tex
%
% (c) 2021 Prof Dr Andreas Müller, OST Ostschweizer Fachhochschule
%
\begin{frame}[t]
\setlength{\abovedisplayskip}{4pt}
\setlength{\belowdisplayskip}{4pt}
\frametitle{Faktorisierung von $X^2+X+1$}
\vspace{-3pt}
$X^2+X+1$ kann faktorisiert werden, wenn man $i\sqrt{3}$
hinzufügt
\[
\biggl(X+\frac12+\frac{i\sqrt{3}}2\biggr)
\biggl(X+\frac12-\frac{i\sqrt{3}}2\biggr)
=
X^2+X+\frac14
+
\frac34
=
X^2+X+1
\]
\vspace{-10pt}
\begin{block}{Was ist $i\sqrt{3}$?}
Matrix mit Minimalpolynom $X^2+3$:
\[
W=\begin{pmatrix}0&-3\\1&0\end{pmatrix}
\qquad\Rightarrow\qquad
W^2=\begin{pmatrix}3&0\\0&3\end{pmatrix} = -3I
\qquad\Rightarrow\qquad
W^2+3I=0
\]
\end{block}
\vspace{-10pt}
\begin{block}{Faktorisierung von $X^2+X+1$}
\vspace{-10pt}
\begin{align*}
B_\pm
&=
-\frac12I\pm\frac12W
&
&\Rightarrow
&
(X+B_+)(X+B-)
&=
(X+\frac12I+\frac12W)
(X+\frac12I-\frac12W)
\\
&=
\smash{
{\textstyle\begin{pmatrix}-\frac12&-\frac32\\\frac12&-\frac12\end{pmatrix}}
}
&
&
&
&=
X^2+X + \frac14I - \frac14W^2
\\
&
&
&%\Rightarrow
&
&=
X^2+X + \frac14I + \frac34I
=
X^2+X+I
\end{align*}
\end{block}

\end{frame}
