%
% maximalergrad.tex
%
% (c) 2021 Prof Dr Andreas Müller, OST Ostschweizer Fachhochschule
%
\begin{frame}[t]
\frametitle{Jede Matrix hat eine Polynomrelation}
\setlength{\abovedisplayskip}{5pt}
\setlength{\belowdisplayskip}{5pt}
\vspace{-5pt}
\begin{block}{Dimension des Matrizenrings}
Der Ring $M_{n}(\Bbbk)$ ist ein $n^2$-dimensionaler Vektorraum mit
Basis
{\tiny
\begin{align*}
&\uncover<2->{\begin{pmatrix}
1&0&\dots&0\\
0&0&\dots&0\\
\vdots&\vdots&\ddots&\vdots\\
\end{pmatrix}}
&
&\uncover<3->{\begin{pmatrix}
0&1&\dots&0\\
0&0&\dots&0\\
\vdots&\vdots&\ddots&\vdots\\
\end{pmatrix}}
&
&\uncover<4->{\dots}
&
&\uncover<5->{\begin{pmatrix}
0&0&\dots&1\\
0&0&\dots&0\\
\vdots&\vdots&\ddots&\vdots\\
\end{pmatrix}}
\\
&\uncover<6->{\begin{pmatrix}
0&0&\dots&0\\
1&0&\dots&0\\
\vdots&\vdots&\ddots&\vdots\\
\end{pmatrix}}
&
&\uncover<7->{\begin{pmatrix}
0&0&\dots&0\\
0&1&\dots&0\\
\vdots&\vdots&\ddots&\vdots\\
\end{pmatrix}}
&
&\uncover<8->{\dots}
&
&\uncover<9->{\begin{pmatrix}
0&0&\dots&0\\
0&0&\dots&1\\
\vdots&\vdots&\ddots&\vdots\\
\end{pmatrix}}
\end{align*}}
\end{block}
\vspace{-10pt}
\uncover<10->{%
\begin{block}{Potenzen von $A$}
Die $n^2+1$ Matrizen $I,A,A^2,\dots,A^{n^2-1},A^{n^2}$ müssen linear abhängig
sein:
\[
\uncover<11->{
a_0I+a_1A+a_2A^2+\dots+a_{n^2-1}A^{n^2-1}+a_{n^2}A^{n^2} = 0
}
\]
\uncover<12->{d.~h.~$p(X) = a_0+a_1X+a_2X^2+\dots +a_{n^2-1}X^{n^2-1}+a_{n^2}A^{n^2}\in\Bbbk[X]$ ist ein Polynom mit $p(A)=0$.}
\end{block}}
\uncover<13->{%
$\Rightarrow$ $A$ über die Eigenschaften (Faktorisierung) von $p$ studieren
}
\end{frame}
