%
% adjunktion.tex
%
% (c) 2021 Prof Dr Andreas Müller, Hochschule Rapperswil
%
\begin{frame}[t]
\frametitle{Adjunktion einer Nullstelle von $m(X)$}
\setlength{\abovedisplayskip}{5pt}
\setlength{\belowdisplayskip}{5pt}
Sei $m(X)=m_0+m_1X+\dots + X^n\in \Bbbk[X]$ ein irreduzibles Polynom.
\uncover<2->{%
\[
X^n = -m_{n-1}X^{n-1} - \dots - m_1X - m_0
\]
}%
\uncover<3->{%
Nullstelle $W$ als Operator betrachten:
\[
W = \begin{pmatrix}
     0&     0&     0&\dots &     0&   -m_0\\
     1&     0&     0&\dots &     0&   -m_1\\
     0&     1&     0&\dots &     0&   -m_2\\
     0&     0&     1&\dots &     0&   -m_3\\
\vdots&\vdots&\vdots&\ddots&\vdots& \vdots\\
     0&     0&     0&\dots &     1&-m_{n-1}
\end{pmatrix}
\]}
\uncover<4->{%
Man kann nachrechnen, dass immer $m(W)=0$.
}
\medskip

\uncover<5->{$\Rightarrow \Bbbk(W) = \{p(W)\;|\;p\in\Bbbk[X], \deg p<\deg m\}$
ist ein Körper, in dem $m(X)$ faktorisiert werden kann $m(X) = (X-W)q(X)$.}
\end{frame}
