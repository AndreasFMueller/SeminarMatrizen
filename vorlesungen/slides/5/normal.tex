%
% normal.tex
%
% (c) 2021 Prof Dr Andreas Müller, OST Ostschweizer Fachhochschule
%
\begin{frame}[t]
\setlength{\abovedisplayskip}{5pt}
\setlength{\belowdisplayskip}{5pt}
\frametitle{Normale Operatoren}
\vspace{-20pt}
\begin{columns}[t,onlytextwidth]
\begin{column}{0.48\textwidth}
\begin{block}{Frage}
$f,g\colon \mathbb{C}\to\mathbb{C}$.
\\
In welchen Punkten müssen $f$ und $g$ übereinstimmen, damit
$f(A)=g(A)$?
\end{block}
\uncover<2->{%
\begin{block}{Definition $f(A)$}
$f$ durch eine Folge von Polynomen
appoximieren: $p_n\to f$
\[
f(A) = \lim_{n\to\infty}p_n(A)
\]
\end{block}}
\vspace{-15pt}
\uncover<3->{%
\begin{block}{Vermutung}
Falls $f(z)=g(z)$ für $z\in\operatorname{Sp}(A)$,
dann ist $f(A)=g(A)$

\smallskip
\uncover<4->{%
{\usebeamercolor[fg]{title}Stimmt für: } $A$ diagonalisierbar
}
\end{block}}
\end{column}
\begin{column}{0.48\textwidth}
\uncover<5->{%
\begin{block}{Beispiel}
\[
A=\begin{pmatrix}2&1\\0&2\end{pmatrix}, \quad
\operatorname{Sp}(A)=\{2\}
\]
\uncover<6->{%
\begin{align*}
f(z)&=(z-2)^2 &g(z)&=z-2
\\
\uncover<7->{
f(A)&=0&g(A)&=\begin{pmatrix}0&1\\0&0\end{pmatrix}
}
\end{align*}}
\end{block}}
\vspace{-18pt}
\uncover<8->{%
\begin{block}{Normal}
$A$ heisst {\em normal}, wenn $AA^*=A^*A$
\begin{itemize}
\item<9->
symmetrische Matrizen: $A=A^*$
\item<10->
unitäre Matrizen: $A^*=A^{-1}\Rightarrow
AA^*=AA^{-1}=A^{-1}A=A^*A$
\end{itemize}
\end{block}}
\end{column}
\end{columns}
\end{frame}
