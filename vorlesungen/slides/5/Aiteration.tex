%
% Aiteration.tex
%
% (c) 2021 Prof Dr Andreas Müller, Hochschule Rapperswil
%
\begin{frame}[t]
\setlength{\abovedisplayskip}{5pt}
\setlength{\belowdisplayskip}{5pt}
\frametitle{Iteration von $A$}
\vspace{-15pt}
\begin{columns}[t,onlytextwidth]
\begin{column}{0.34\textwidth}
\begin{block}{$\varrho(A) > 1\uncover<4->{\Rightarrow \|A^k\|\to\infty}$}
\uncover<2->{%
Eigenvektor $v$, $\|v\|=1$, zum Eigenwert $\lambda$ mit $|\lambda| > 1$}
\uncover<3->{%
\[
\|A^kv\| = |\lambda|^k\to \infty
\]}
\uncover<4->{$\Rightarrow \|A\|^k\to\infty$}

\end{block}
\end{column}
\begin{column}{0.63\textwidth}
\begin{block}{$\varrho(A) < 1\uncover<12->{\Rightarrow \|A\|^k\to 0}$}
\uncover<5->{%
$A$ setzt sich zusammen aus Jordanblöcken:
\[
J(\lambda)^k
=
\renewcommand{\arraystretch}{1.2}
\begin{pmatrix}
\lambda^k&\binom{k}{1}\lambda^{k-1}&\binom{k}{2}\lambda^{k-2}
	&\dots&\binom{k}{n-1}\lambda^{k-n+1}\\
    0    &\lambda^k&\binom{k}{1}\lambda^{k-1}
	&\dots&\binom{k}{n-2}\lambda^{k-n+2}\\
    0    &    0    &\lambda^k&\dots  &\binom{k}{n-3}\lambda^{k-n+3}\\
 \vdots  & \vdots  & \vdots  &\ddots &\vdots\\
    0    &    0    &    0    &\dots  &\lambda^k
\end{pmatrix}
\]}
\uncover<6->{Alle Matrixelemente konvergieren gegen $0$:}
\[
\uncover<7->{\binom{k}{s} \le k^s}
\uncover<8->{\Rightarrow
\underbrace{\binom{k}{s}}_{\text{\uncover<9->{polynomiell $\to \infty$}}}
\underbrace{\lambda^{k-s}}_{\text{\uncover<10->{exponentiell $\to 0$}}}
}
\uncover<11->{\to 0}
\]
\end{block}
\end{column}
\end{columns}
\uncover<13->{%
{\usebeamercolor[fg]{title}Folgerung:}
Es gibt $m,M$ derart, dass
$m\varrho(A)^k \le \|A^k\|  \le M \varrho(A)^k$
}
\end{frame}
