%
% exponentialfunktion.tex
%
% (c) 2021 Prof Dr Andreas Müller, OST Ostschweizer Fachhochschule
%
\begin{frame}[t]
\setlength{\abovedisplayskip}{5pt}
\setlength{\belowdisplayskip}{5pt}
\frametitle{Exponentialfunktion}
\vspace{-15pt}
\begin{columns}[t,onlytextwidth]
\only<1-6>{%
\ifthenelse{\boolean{presentation}}{
\begin{column}{0.48\textwidth}
\begin{block}{$x(t) \in\mathbb{R}$}
\vspace{-10pt}
\begin{align*}
\frac{d}{dt}x(t) &= ax(t) &a&\in\mathbb{R}
\\
x(0) &= c&&\in\mathbb{R}
\intertext{\uncover<2->{Lösung:}}
\uncover<2->{x(t) &= ce^{at}}
\end{align*}
\end{block}
\end{column}}{}}
\begin{column}{0.48\textwidth}
\uncover<3->{%
\begin{block}{$X(t) \in M_n(\mathbb{R})$}
\vspace{-10pt}
\begin{align*}
\frac{d}{dt}X(t)
&=
A
X(t)&A&\in M_n(\mathbb{R})
\\
X(0)&=C&&\in M_n(\mathbb{R})
\intertext{\uncover<4->{gekoppelte Differentialgleichung für
vier Funktionen $x_{ij}(t)$}}
\uncover<5->{\dot{x}_{11} &= \rlap{$a_{11} x_{11}(t) + a_{12} x_{21}(t)$}}\\
\uncover<5->{\dot{x}_{12} &= \rlap{$a_{11} x_{12}(t) + a_{12} x_{22}(t)$}}\\
\uncover<5->{\dot{x}_{21} &= \rlap{$a_{21} x_{11}(t) + a_{22} x_{21}(t)$}}\\
\uncover<5->{\dot{x}_{22} &= \rlap{$a_{21} x_{12}(t) + a_{22} x_{22}(t)$}}\\
\intertext{\uncover<6->{Lösung:}}
\uncover<6->{X(t) &= \exp(At) C}
\end{align*}
\end{block}}
\end{column}
\only<7-9>{%
\ifthenelse{\boolean{presentation}}{
\begin{column}{0.48\textwidth}
\begin{block}{Beispiel: Diagonalmatrix}
%$D=\operatorname{diag}(\lambda_1,\dots,\lambda_n)$
\begin{align*}
\frac{d}{dt}X&=DX &&\uncover<8->{\Rightarrow &\dot{x}_{ij}(t) &= \lambda_i x_{ij}(t)}
\\
X(0)&=C
&&\uncover<8->{\Rightarrow&x_{ij}(t)&=c_{ij}}
\end{align*}
\uncover<9->{%
Lösung:
\[
x_{ij}(t) =c_{ij}e^{\lambda_i t}
\]}
\end{block}
\end{column}}{}}
\uncover<10->{%
\begin{column}{0.48\textwidth}
\begin{block}{Beispiel: Jordan-Block}
\vspace{-10pt}
\begin{align*}
A&=\begin{pmatrix}\lambda&1\\0&\lambda\end{pmatrix}
\rlap{$\displaystyle,\;
X(t)
=
\ifthenelse{\boolean{presentation}}{
\only<22>{
	e^{\lambda t}
	\begin{pmatrix} 1&t/\lambda\\ 0&1 \end{pmatrix}
}}{}
\only<23>{
	\frac{e^{\lambda t}}{\lambda}
	\begin{pmatrix} \lambda&t\\ 0&\lambda \end{pmatrix}
}
C
$}
\\
\uncover<11->{
\dot{x}_{1i}(t)&=\lambda x_{1i}(t) + \phantom{\lambda}x_{2i}(t),&&x_{1i}(0)&=c_{1i}
}
\\
\uncover<12->{
\dot{x}_{2i}(t)&=\phantom{\lambda x_{1i}(t)+\mathstrut}\lambda x_{2i}(t),&&x_{2i}(0)&=c_{2i}
}
\end{align*}
\uncover<13->{%
Lösung:}
\begin{align*}
\uncover<14->{
x_{2i}(t)&=c_{2i}e^{\lambda t}
}
\\
\uncover<15->{
\dot{x}_{1i}(t)&=\lambda x_{1i}(t) + c_{2i}e^{\lambda t}
}
\\
\ifthenelse{\boolean{presentation}}{
\only<16-17>{x_{1i\only<16>{,h}}(t)}}{}
\only<18->{\dot{x}_{1i}(t)}
&
\ifthenelse{\boolean{presentation}}{
\only<16-17>{=c\only<17>{(t)}\lambda e^{\lambda t}}
\only<18>{=\dot{c}(t)\lambda e^{\lambda t}
+
c(t)\lambda^2 e^{\lambda t}}
}{}
\only<19->{=\lambda x_{1i}(t) + \dot{c}(t)\lambda e^{\lambda t}}
\\
\uncover<20->{\Rightarrow
\dot{c}(t)&= c_{2i}/\lambda
\Rightarrow
c(t) = c_{2i}(0) +tc_{2i}/\lambda
}
\\
\uncover<21->{
x_{1i}(t) & =c_{1i}e^{\lambda t} + t(c_{2i}/\lambda)e^{\lambda t}
}
\end{align*}
\end{block}
\end{column}}
\end{columns}
\end{frame}
