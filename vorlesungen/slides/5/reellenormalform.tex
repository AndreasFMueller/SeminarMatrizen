%
% reellenormalform.tex
%
% (c) 2021 Prof Dr Andreas Müller, OST Ostschweizer Fachhochschule
%
\begin{frame}[t]
\frametitle{Reelle Normalform}
$A\in M_n(\mathbb{R})\subset M_n(\mathbb{C})$ hat reelle und Paare von
konjugiert komplexen Eigenwerten
\medskip

$\Rightarrow$ Konjugiert komplexe Eigenvektoren $v$ und $\overline{v}$,
$x=\operatorname{Re}v$ und $y=\operatorname{Im}v$
\begin{align*}
\only<-2>{
\begin{pmatrix}
Av\\
A\overline v
\end{pmatrix}
=
\begin{pmatrix}
Ax+Ay J \\
Ax-Ay J
\end{pmatrix}
&=
\begin{pmatrix}
\lambda v\\
\overline{\lambda}\overline{v}
\end{pmatrix}
=
\begin{pmatrix}
a+bJ & 0 \\
  0  & a-bJ
\end{pmatrix}
\begin{pmatrix}
x+ yJ\\
x- yJ
\end{pmatrix}
\\
}
\only<2-3>{
\begin{pmatrix}
Ax&-Ay\\
Ay& Ax\\
Ax& Ay\\
-Ay&Ax
\end{pmatrix}
&=
\begin{pmatrix}
a&-b& 0& 0\\
b& a& 0& 0\\
0& 0& a& b\\
0& 0&-b& a
\end{pmatrix}
\begin{pmatrix}
x&-y\\
y& x\\
x& y\\
-y&x
\end{pmatrix}
\\
}
\only<3-4>{
\ifthenelse{\boolean{presentation}}{
\begin{pmatrix}
Ax&-Ay\\
Ax& Ay\\
Ay& Ax\\
-Ay&Ax
\end{pmatrix}
&
=
\begin{pmatrix}
a& 0&-b& 0\\
0& a& 0& b\\
b& 0& a& 0\\
0&-b& 0& a
\end{pmatrix}
\begin{pmatrix}
x&-y\\
x& y\\
y& x\\
-y&x
\end{pmatrix}
\Rightarrow
\\
}{}
}
\only<4->{
Ax &= ax -by \\
Ay &= bx +ay
}
\end{align*}
\uncover<5->{%
D.h. in Basis $x=\operatorname{Re}v,y=\operatorname{Im}v$ hat $A$ die Matrix
$\begin{pmatrix}a&-b\\b&a\end{pmatrix}$}
\uncover<6->{%
\[
\text{
Reeller
Jordan-Block:
}
\qquad
J_{\lambda,\overline{\lambda}}
=
\begin{pmatrix}
a&-b&1& 0&0& 0\\
b& a&0& 1&0& 0\\
 &  &a&-b&1& 0\\
 &  &b& a&0& 1\\
 &  & &  &a&-b\\
 &  & &  &b& a
\end{pmatrix}
\]}
\end{frame}
