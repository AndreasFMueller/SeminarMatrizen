%
% stoneweierstrass.tex
%
% (c) 2021 Prof Dr Andreas Müller, Hochschule Rapperswil
%
\bgroup
\definecolor{darkgreen}{rgb}{0,0.6,0}
\begin{frame}[t]
\frametitle{Allgemeiner Approximationssatz}
\vspace{-20pt}
\begin{columns}[t]
\begin{column}{0.5\textwidth}
\begin{theorem}[Stone-Weierstrass, $\mathbb{R}$]
$A$ eine {\color{darkgreen}$\mathbb{R}$}-Algebra
von stetigen Funktionen auf einem
%abgeschlossenen und beschränkten
kompakten
Definitionsgebiet $D\subset {\color{darkgreen}\mathbb{R}}$, 
\begin{itemize}
\item<2-> konstante Funktion $c\in A$,
\item<3-> für $d_1,d_2\in D$ gibt es ein $s\in A$ mit
$s(d_1)\ne s(d_2)$.
\end{itemize}
\uncover<4->{%
Dann lässt sich jede stetige Funktion durch Funktionen aus $A$
approximieren}
\end{theorem}
\uncover<5->{
\begin{block}{Anwendung}
\uncover<6->{$A={\color{darkgreen}\mathbb{R}}[X]$}\uncover<7->{,
$s(X)=X$}\uncover<8->{,
jede stetige Funktion kann durch
Polynome in $X$ approximiert werden}
\end{block}}
\end{column}
\begin{column}{0.5\textwidth}
\uncover<9->{%
\begin{theorem}[Stone-Weierstrass, $\mathbb{C}$]
$A$ eine {\color<10->{red}$\mathbb{C}$}-Algebra von stetigen Funktionen
auf einem
%abgeschlossenen und beschränkten
kompakten
Definitionsgebiet $D\subset {\color<10->{red}\mathbb{C}}$, 
\begin{itemize}
\item konstante Funktion $c\in A$,
\item für $d_1,d_2\in D$ gibt es ein $s\in A$ mit
$s(d_1)\ne s(d_2)$.
\only<11->{
\item {\color{red}$f\in A\Rightarrow \overline{f}\in A$}
}
\end{itemize}
Dann lässt sich jede stetige Funktion durch Funktionen aus $A$
approximieren
\end{theorem}}
\vspace{-5pt}
\uncover<12->{%
\begin{block}{Anwendung}
$A={\color{red}\mathbb{C}}[Z,\overline{Z}]$\uncover<13->{,
$s(Z{\color{red},\overline{Z}})=Z$}\uncover<14->{,
jede stetige Funktion
lässt sich durch Polynome in $Z{\color{red},\overline{Z}}$ approximieren}
\end{block}}
\end{column}
\end{columns}
\end{frame}
\egroup
