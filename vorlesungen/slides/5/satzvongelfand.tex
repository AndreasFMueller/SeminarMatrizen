%
% satzvongelfand.tex
%
% (c) 2021 Prof Dr Andreas Müller, OST Ostschweizer Fachhochschule
%
\bgroup
\begin{frame}[t]
\setlength{\abovedisplayskip}{0pt}
\setlength{\belowdisplayskip}{0pt}
\setbeamercolor{block body}{bg=blue!20}
\setbeamercolor{block title}{bg=blue!20}
\frametitle{Satz von Gelfand}
{\usebeamercolor[fg]{title}Behauptung:} $\varrho(A)=\pi(A)$\uncover<2->{,
$A(\varepsilon) = \displaystyle\frac{A}{\varrho(A)+\varepsilon}$}\uncover<3->{,
$\varrho(A(\varepsilon))=\displaystyle\frac{\varrho(A)}{\varrho(A)+\varepsilon}
\uncover<4->{=\frac{1}{1+\varepsilon/\varrho(A)}}$}

\uncover<5->{%
%{\usebeamercolor[fg]{title}Beweisidee:}
%$\displaystyle\pi\biggl(\frac{A}{\varrho(A)+\epsilon}\biggr)
%=
%\frac{\pi(A)}{\varrho(A)+\epsilon}$ berechnen
\vspace{-5pt}
\begin{columns}[t,onlytextwidth]
\begin{column}{0.48\textwidth}
\begin{block}{$\varepsilon < 0$}
\vspace{-10pt}
\begin{align*}
\uncover<6->{
\varrho(A(\varepsilon))&>1}\uncover<7->{\quad\Rightarrow\quad \|A(\varepsilon)^k\|\to \infty}
\\
\uncover<8->{\|A(\varepsilon)^k\| &\ge m\varrho(A(\varepsilon))^k}
\\
\uncover<9->{\|A(\varepsilon)^k\|^{\frac1k} &\ge m^{\frac1k} \varrho(A(\varepsilon))}
\\
\uncover<10->{\pi(A) &\ge \lim_{k\to\infty}m^{\frac1k}\varrho(A(\varepsilon))}
\\
&\uncover<11->{= \varrho(A(\varepsilon))}\uncover<12->{ > 1}
\\
\uncover<13->{\frac{ \pi(A(\varepsilon))}{\varrho(A)+\varepsilon} &> 1}
\\
\uncover<14->{
\pi(A) &> \varrho(A)+\varepsilon
}
\end{align*}
\end{block}
\end{column}
\begin{column}{0.48\textwidth}
\begin{block}{$\varepsilon > 0$}
\vspace{-10pt}
\begin{align*}
\uncover<16->{
\varrho(A(\varepsilon)) &<1}
\uncover<17->{\quad\Rightarrow\quad \|A(\varepsilon)^k\| \to 0}
\\
\uncover<18->{\|A(\varepsilon)^k\|
&\le M\varrho(A(\varepsilon))^k}
\\
\uncover<19->{
\|A(\varepsilon)^k\|^{\frac1k}
&\le M^{\frac1k}\varrho(A(\varepsilon))
}
\\
\uncover<20->{
\pi(A(\varepsilon))
&\le
\varrho(A(\varepsilon)) \lim_{k\to\infty} M^{\frac1k}
}
\\
&\uncover<21->{= \varrho(A(\varepsilon))}
\uncover<22->{ < 1}
\\
\uncover<23->{\frac{\pi(A)}{\varrho(A)+\varepsilon}&< 1}
\\
\uncover<24->{\pi(A)&< \varrho(A) + \varepsilon}
\end{align*}
\end{block}
\end{column}
\end{columns}}
\uncover<15->{%
\vspace{2pt}
{\usebeamercolor[fg]{title}Folgerung:}
$\varrho(A)-\varepsilon < \pi(A) \uncover<25->{< \varrho(A)+\varepsilon}\quad\forall\varepsilon>0
\uncover<26->{
\qquad\Rightarrow\qquad
\varrho(A)=\pi(A)}$
}
\end{frame}
\egroup
