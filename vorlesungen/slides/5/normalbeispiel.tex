%
% normalbeispiel.tex
%
% (c) 2021 Prof Dr Andreas Müller, OST Ostschweizer Fachhochschule
%
\bgroup
\definecolor{darkred}{rgb}{0.8,0,0}
\definecolor{darkgreen}{rgb}{0,0.6,0}
\begin{frame}[t]
\setlength{\abovedisplayskip}{5pt}
\setlength{\belowdisplayskip}{5pt}
\frametitle{Beispiele für normale Matrizen}
\vspace{-15pt}
\begin{columns}[t,onlytextwidth]
\begin{column}{0.49\textwidth}
\uncover<3->{%
\begin{block}{Symmetrisch und Antisymmetrisch}
$A\in M_n(\mathbb{C})$
\begin{align*}
A&=\pm A^t &&\Rightarrow &AA^* &=A\overline{A^t} =\pm A\overline{A}
\\
 &      &&            &     &=\pm\overline{A}A =\overline{A^t}A
\\
 &      &&            &     &=A^*A
\end{align*}
\end{block}}
\end{column}
\begin{column}{0.49\textwidth}
\uncover<4->{%
\begin{block}{Orthogonal}
$A\in M_n(\mathbb{R})\;\Rightarrow\; A^*=A^t$
\begin{align*}
AA^t&=I &&\Rightarrow& AA^*&=AA^t=I\\
    &   &&           &     &=A^tA=A^*A
\end{align*}
\end{block}}
\end{column}
\end{columns}
\vspace{-15pt}
\begin{columns}[t,onlytextwidth]
\begin{column}{0.49\textwidth}
\uncover<1->{%
\begin{block}{Hermitesch und Antihermitesch}
$A\in M_n(\mathbb{C})$
\begin{align*}
A&=\pm A^* &&\Rightarrow &AA^* &=\pm A^2=A^*A
\end{align*}
\end{block}}
\end{column}
\begin{column}{0.49\textwidth}
\uncover<2->{%
\begin{block}{Unitär}
$A\in M_n(\mathbb{C})$
\begin{align*}
AA^*&=I &&\Rightarrow& AA^*=I=A^*A
\end{align*}
\end{block}}
\end{column}
\end{columns}
%\uncover<5->{%
%\begin{block}{Weitere}
%$N\in M_n(\mathbb{C})$ nilpotent, $N^k=0$\uncover<11->{
%$\Rightarrow$
%normal für $l=k-l\Rightarrow l=\frac{k}{2}$}
%\uncover<6->{%
%\[
%\left.
%\begin{aligned}
%A  &=N^l+(N^t)^{k-l}
%\\
%A^t&=(N^t)^l+N^{k-1}
%\end{aligned}
%\right\}
%\uncover<7->{%
%\Rightarrow
%\left\{
%\begin{aligned}
%\mathstrut
%A^t A
%&\only<8>{=
%((N^t)^l+N^{k-l}) (N^l+(N^t)^{k-l})}
%\uncover<9->{=
%{\color<10>{darkgreen}(N^t)^lN^l}
%\only<9>{+
%{\color{orange}(N^t)^k}}
%+
%{\color<10>{darkred}N^{k-l}(N^t)^{k-l}}
%\only<9>{+
%{\color{orange}N^k}}}
%\\
%\mathstrut
%A A^t
%&\only<8>{= 
%(N^l+(N^t)^{k-l})((N^t)^l+N^{k-l})}
%\uncover<9->{=
%{\color<10>{darkred}N^l(N^t)^l}
%+
%\only<9>{{\color{orange}N^k}
%+
%{\color{orange}(N^t)^k}
%+}
%{\color<10>{darkgreen}(N^t)^{k-l}N^{k-l}}}
%\end{aligned}
%\right.}
%\hspace{20cm}
%\]}
%\end{block}}
\end{frame}
