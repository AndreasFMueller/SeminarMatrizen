%
% spektrum.tex
%
% (c) 2021 Prof Dr Andreas Müller, OST Ostschweizer Fachhochschule
%
\begin{frame}[t]
\setlength{\abovedisplayskip}{5pt}
\setlength{\belowdisplayskip}{5pt}
\frametitle{Spektrum}
\vspace{-15pt}
\begin{columns}[t,onlytextwidth]
\begin{column}{0.48\textwidth}
\begin{block}{Definition}
$A\colon V\to V$ beschränkter Operator zwischen Banach-Räumen
\[
\operatorname{Sp}A
=
\left\{
\lambda\in\mathbb{C}
\;\left|\;
\begin{minipage}{2cm}\raggedright
$A-\lambda I$ nicht invertierbar
\end{minipage}
\right.
\right\}
\]
\end{block}
\uncover<2->{%
\begin{block}{Endlichdimensionale Räume}
\vspace{-15pt}
\begin{align*}
&\lambda\in\operatorname{Sp}A
\\
\uncover<3->{
\Leftrightarrow\quad&\text{$(A-\lambda I)$ nicht invertierbar}
}
\\
\uncover<4->{
\Leftrightarrow\quad&\text{$(A-\lambda I)$ singulär}
}
\\
\uncover<5->{
\Leftrightarrow\quad&\ker(A-\lambda I)\ne 0
}
\\
\uncover<6->{
\Leftrightarrow\quad&\exists v\in V, v\ne 0, Av=\lambda v
}
\end{align*}
\uncover<7->{%
$\Rightarrow$ $\operatorname{Sp}A$ ist die Menge der Eigenwerte
}
\end{block}}
\end{column}
\begin{column}{0.48\textwidth}
\uncover<8->{%
\begin{block}{Unendlichdimensional}
Es gibt eine Folge $x_n\in V$ von Einheitsvektoren
$\|x_n\|=1$
mit
\begin{align*}
\lim_{n\to\infty} (A - \lambda)x_n &= 0
\end{align*}
\end{block}}
\uncover<9->{%
\begin{block}{Spektrum und Norm}
\[
\operatorname{Sp}(A)
\subset
\{\lambda\in\mathbb{C}\;|\;
|\lambda|\le \|A\|\}
\]
\end{block}}
\end{column}
\end{columns}
\end{frame}
