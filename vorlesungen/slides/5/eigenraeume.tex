%
% eigenraeume.tex
%
% (c) 2021 Prof Dr Andreas Müller, OST Ostschweizer Fachhochschule
%
\begin{frame}[t]
\frametitle{Eigenräume}
\vspace{-15pt}
\begin{columns}[t,onlytextwidth]
\begin{column}{0.48\textwidth}
\begin{block}{Eigenraum}
Für $\lambda\in\Bbbk$ heisst
\begin{align*}
E_\lambda(f)
&=
\ker (f-\lambda)
\\
\uncover<2->{
&=
\{v\in V\;|\; f(v) = \lambda v\}
}
\end{align*}
\uncover<3->{%
{\em Eigenraum} von $f$ zum Eigenwert $\lambda$.}
\end{block}
\uncover<4->{%
$E_\lambda(f)\subset V$ ist ein Unterraum}

\uncover<5->{%
\begin{block}{Eigenwert}
Falls $\dim E_\lambda(f)>0$ heisst $\lambda$ Eigenwert von $f$.
\end{block}}

\end{column}
\begin{column}{0.48\textwidth}
\uncover<6->{%
\begin{block}{verallgemeinerter Eigenraum}
Für $\lambda\in \Bbbk$ heisst
\[
\mathcal{E}_\lambda(f)
=
\mathcal{K}(f-\lambda)
\]
verallgemeinerter Eigenraum
\end{block}}
\end{column}
\end{columns}
\end{frame}
