%
% planbeispiele.tex
%
% (c) 2021 Prof Dr Andreas Müller, OST Ostschweizer Fachhochschule
%
\bgroup
\definecolor{darkgreen}{rgb}{0,0.6,0}
\definecolor{darkred}{rgb}{0.8,0,0}
\setlength{\abovedisplayskip}{5pt}
\setlength{\belowdisplayskip}{5pt}
\begin{frame}[t]
\frametitle{Beispiele}
\vspace{-15pt}
\begin{columns}[t]
\begin{column}{0.33\textwidth}
\setbeamercolor{block body}{bg=blue!20}
\setbeamercolor{block title}{bg=blue!20}
\uncover<2->{%
\begin{block}{$A$ diagonal, $\operatorname{Sp}(A)\subset\mathbb{R}$\strut}
Beispiele:
\begin{align*}
f(x)
&=
x^k,
\\
f(x)&=
\sqrt{x},
\sqrt[k]{x}
\\
f(x)&=|x|
\end{align*}
\vspace{43pt}
\end{block}}
\end{column}
\begin{column}{0.33\textwidth}
\setbeamercolor{block body}{bg=darkgreen!20}
\setbeamercolor{block title}{bg=darkgreen!20}
\uncover<1->{%
\begin{block}{$f(z)$ analytisch\strut}
Beispiele:
\begin{align*}
e^z
&=
\sum_{k=0}^\infty \frac{z^k}{k!}
\\
\cos z
&=
\sum_{k=0}^\infty (-1)^k\frac{z^{2k}}{2k!}
\\
\sin z
&=
\sum_{k=0}^\infty (-1)^k\frac{z^{2k+1}}{(2k+1)!}
\end{align*}
\end{block}}
\end{column}
\begin{column}{0.33\textwidth}
\setbeamercolor{block body}{bg=darkred!20}
\setbeamercolor{block title}{bg=darkred!20}
\uncover<3->{%
\begin{block}{$A$ normal, $AA^*=A^*A$\strut}
Beispiele:
\begin{align*}
f(z)&=\sqrt{z\overline{z}}=|z|
\end{align*}
\vspace{76pt}
\end{block}}
\end{column}
\end{columns}
\vspace{-10pt}
\begin{columns}[t]
\begin{column}{0.33\textwidth}
\setbeamercolor{block body}{bg=blue!20}
\setbeamercolor{block title}{bg=blue!20}
\uncover<5->{%
\begin{block}{}
\vspace{-6pt}
$f(A)$ wohldefiniert für {\color{blue}diagonalisierbare}
Matrizen $A\in M_n(\mathbb{R})$
\end{block}}
\end{column}
\begin{column}{0.33\textwidth}
\setbeamercolor{block body}{bg=darkgreen!20}
\setbeamercolor{block title}{bg=darkgreen!20}
\uncover<4->{%
\begin{block}{}
\vspace{-6pt}
$f(A)$ wohldefiniert für {\color{darkgreen}jedes} $A\in M_n(\mathbb{C})$
\vspace{14pt}
\end{block}}
\end{column}
\begin{column}{0.33\textwidth}
\setbeamercolor{block body}{bg=darkred!20}
\setbeamercolor{block title}{bg=darkred!20}
\uncover<6->{%
\begin{block}{}
\vspace{-6pt}
$f(A)$ wohldefiniert für {\color{darkred}normale}
Matrizen $A\in M_n(\mathbb{C})$
\end{block}}
\end{column}
\end{columns}
\end{frame}
\egroup
