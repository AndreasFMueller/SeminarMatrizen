%
% unitaer.tex
%
% (c) 2021 Prof Dr Andreas Müller, OST Ostschweizer Fachhochschule
%
\begin{frame}[t]
\setlength{\abovedisplayskip}{5pt}
\setlength{\belowdisplayskip}{5pt}
\frametitle{Unitäre Matrizen}
\vspace{-15pt}
\begin{columns}[t,onlytextwidth]
\begin{column}{0.48\textwidth}
\begin{block}{Eigenwerte}
$U$ unitär lässt das Skalarprodukt invariant
\[
\langle Ux,Uy\rangle
=
\langle x,y\rangle
\]
\uncover<2->{%
$\lambda$ ein Eigenwert mit Eigenvektor $v$:
\begin{align*}
\langle v,v\rangle
&=
\langle Uu,Uv\rangle
\uncover<3->{= \langle \lambda v,\lambda v\rangle}
\uncover<4->{= |\lambda|^2 \langle v,v\rangle}
\\
\uncover<5->{\Rightarrow\;|\lambda|&=1}
\end{align*}}
\end{block}
\uncover<6->{%
\begin{block}{Diagonalisierbar}
Unitäre Matrizen sind über $\mathbb{C}$ diagonalisierbar
\end{block}
\end{column}
\begin{column}{0.48\textwidth}
\begin{block}{Grosse Jordan-Blöcke?}
Falls es Vektoren $v,w$ gibt mit
\begin{align*}
\uncover<7->{ Uv&=\lambda v}
\\
\uncover<8->{ Uw&=\lambda w + v}
\intertext{\uncover<9->{Skalarprodukt:}}
\uncover<10->{
\langle v,w\rangle
&=
\langle Uv,Uw\rangle}
\\
\uncover<11->{
&=
\langle \lambda v,\lambda w\rangle
+
\langle\lambda v,v\rangle}
\\
\uncover<12->{
&=
|\lambda|^2 \langle v,w\rangle
+
\langle\lambda v,v\rangle}
\\
\uncover<13->{
&=
\langle v,w\rangle
+
\lambda \| v\|^2}
\\
\uncover<14->{
\Rightarrow\quad
0&=\|v\|^2\quad\Rightarrow\quad \|v\|=0}
\end{align*}
\end{block}}
\end{column}
\end{columns}
\end{frame}
