%
% plan.tex
%
% (c) 2021 Prof Dr Andreas Müller, OST Ostschweizer Fachhochschule
%
\bgroup
\definecolor{darkgreen}{rgb}{0,0.5,0}
\definecolor{darkred}{rgb}{0.8,0.0,0}
\begin{frame}[t]
\frametitle{Was ist $f(A)$?}
\vspace{-5pt}
\begin{center}
\begin{tikzpicture}[>=latex,thick]

\uncover<7->{
	\fill[color=blue!20] (-1.5,0.7) rectangle (11.5,3.8);
}

\uncover<4->{
	\fill[color=darkgreen!20] (-1.5,-0.7) rectangle (11.5,0.7);
}

\uncover<12->{
	\fill[color=darkred!20] (-1.5,-0.7) rectangle (11.5,-3.8);
}

\begin{scope}[xshift=-1cm]
\node at (0,0) [left] {$A$};
\end{scope}

%\foreach \x in {1,...,20}{
%	\only<\x>{ \node at (-1,3) {\x}};
%}

%
% Blauer Ast
%

\uncover<2->{
	\draw[->,color=blue,shorten <= 0.3cm, shorten >= 0.0cm]
		(-1.2,0) -- (0,1.3);

	\begin{scope}[xshift=0cm,yshift=1.5cm]
	\fill[color=white,opacity=0.7] (0,-0.6) rectangle (3.4,0.6);
	\draw[color=blue] (0,-0.6) rectangle (3.4,0.6);
	\node at (0,0) [right] {$\begin{aligned}
	f&=p\in\mathbb{R}[X]\\
	f(A)&=p(A)
	\end{aligned}
	$};
	\end{scope}
}

\uncover<7->{
	\draw[->,color=blue] (1.8,2.1) -- (3.6,3);

	\begin{scope}[xshift=3.6cm,yshift=3cm]
	\fill[color=white,opacity=0.7] (0,-0.6) rectangle (3.7,0.6);
	\draw[color=blue] (0,-0.6) rectangle (3.7,0.6);
	\node at (0,0) [right] {\begin{minipage}{3cm}\raggedright
	$f$ durch $p_n\in\mathbb{R}[X]$\\
	approximieren
	\end{minipage}};
	\end{scope}
}

\uncover<8->{
	\draw[->,color=blue] (7.3,3) -- (9.5,1.9);

	\begin{scope}[xshift=7.6cm,yshift=1.5cm]
	\fill[color=white,opacity=0.7] (0,-0.35) rectangle (3.8,0.4);
	\draw[color=blue] (0,-0.35) rectangle (3.8,0.4);
	\node at (0,0) [right] {$\displaystyle f(A) = \lim_{n\to\infty}p_n(A)$};
	\end{scope}
}

\uncover<9->{
	\node[color=blue] at (3.6,1.6) [right] {\begin{minipage}{4cm}
	\raggedright
	Konvergenz $p_n\to f$\\
	auf Spektrum $\operatorname{Sp}(A)\subset\mathbb{R}$
	\end{minipage}};
}

\uncover<11->{
	\node[color=blue] at (-1.5,3.8) [below right]
		{$A$ symmetrisch: $A=A^*$};
}
\uncover<10->{
	\node[color=blue] at (11.5,3.8) [below left] {$A$ diagonalisierbar};
}

%
% Roter Ast
%

\uncover<12->{
	\draw[->,color=darkred,shorten <= 0.3cm, shorten >= 0.0cm] (-1.2,0) -- (0,-1.3);

	\begin{scope}[xshift=0cm,yshift=-1.5cm]
	\fill[color=white,opacity=0.7] (0,-0.6) rectangle (3.4,0.6);
	\draw[color=darkred] (0,-0.6) rectangle (3.4,0.6);
	\node at (0,0) [right] {$\begin{aligned}
	f&=p\in\mathbb{C}[Z,\overline{Z}]\\
	f(A)&=p(A,A^*)
	\end{aligned}$};
	\end{scope}
}

\uncover<13->{
	\node[color=darkred] at (1.7,-2.1) [below left]
		{Für $|Z|^2 = Z\overline{Z}$};
}

\uncover<14->{
	\draw[->,color=darkred] (1.8,-2.1) -- (3.6,-3);

	\begin{scope}[xshift=3.6cm,yshift=-3cm]
	\fill[color=white,opacity=0.7] (0,-0.6) rectangle (3.7,0.6);
	\draw[color=darkred] (0,-0.6) rectangle (3.7,0.6);
	\node at (0,0) [right] {\begin{minipage}{3.5cm}\raggedright
	$f$ durch $q_n\in\mathbb{C}[Z,\overline{Z}]$\\
	approximieren
	\end{minipage}};
	\end{scope}
}

\uncover<15->{
	\draw[->,color=darkred] (7.3,-3) -- (9.5,-1.85);

	\begin{scope}[xshift=7.6cm,yshift=-1.5cm]
	\fill[color=white,opacity=0.7] (0,-0.35) rectangle (3.8,0.4);
	\draw[color=darkred] (0,-0.35) rectangle (3.8,0.4);
	\node at (0,0) [right]
		{$\displaystyle f(A) = \lim_{n\to\infty}q_n(A,A^*)$};
	\end{scope}
}

\uncover<16->{
	\node[color=darkred] at (3.6,-1.8) [right] {\begin{minipage}{4cm}
	\raggedright
	Konvergenz $p_n\to f$\\
	auf $\operatorname{Sp}(A)\cup\operatorname{Sp}(A^*)$
	\end{minipage}};
}

\uncover<17->{
	\node[color=darkred] at (11.5,-3.8) [above left] {%
	\begin{minipage}{3.5cm}\raggedleft
	nur sinnvoll definiert wenn 
	$AA^*=A^*A$
	\end{minipage}};
}

\uncover<18->{
	\node[color=darkred] at (-1.5,-3.8) [above right]
		{$A$ normal: $AA^*=A^*A$};
}

%
% Grüner Ast
%

\uncover<3->{
	\draw[->,color=darkgreen,shorten <= 0.0cm, shorten >= 0.0cm]
		(-1,0) -- (0,0);

	\begin{scope}[xshift=0cm,yshift=0cm]
	\fill[color=white,opacity=0.7] (0,-0.6) rectangle (2.9,0.6);
	\draw[color=darkgreen] (0,-0.6) rectangle (2.9,0.6);
	\node at (0,0) [right] {$\displaystyle
	f(z)=\sum_{k=0}^\infty a_kz^k$};
	\end{scope}
}

\uncover<5->{
	\node[color=darkgreen] at (5.9,0) [above] {$f(z)$ analytisch!};
}
\uncover<6->{
	\node[color=darkgreen] at (5.9,0) [below]
		{$\varrho(A)<\text{Konvergenzradius}$};
}

\uncover<4->{
	\draw[->,color=darkgreen] (2.9,0) -- (8.5,0);

	\begin{scope}[xshift=8.5cm]
	\fill[color=white,opacity=0.7] (0,-0.6) rectangle (2.9,0.6);
	\draw[color=darkgreen] (0,-0.6) rectangle (2.9,0.6);
	\node at (0,0) [right] {$\displaystyle
	 f(A)=\sum_{k=0}^\infty a_kA^k$};
	\end{scope}
}

\end{tikzpicture}
\end{center}
\end{frame}
\egroup
