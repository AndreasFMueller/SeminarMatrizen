%
% ketten.tex
%
% (c) 2021 Prof Dr Andreas Müller, OST Ostschweizer Fachhochschule
%
\begin{frame}[t]
\setlength{\abovedisplayskip}{5pt}
\setlength{\belowdisplayskip}{5pt}
\frametitle{Ketten von Unterräumen}
\begin{block}{Schachtelung}
Die Unterräume $\mathcal{J}^k(f)$ und $\mathcal{K}^k(f)$ sind geschachtelt:
\[
\arraycolsep=1.4pt
\begin{array}{rcrcrcrcrcrcrcccc}
0       &=&\mathcal{K}^0(f)
	&\subset&\mathcal{K}^1(f)
	&\subset&\dots
	&\subset&\mathcal{K}^k(f)
	&\subset&\mathcal{K}^{k+1}(f)
	&\subset&\dots
	&\subset&\displaystyle\bigcup_{k=0}^\infty \mathcal{K}^k(f)
	&=:&\mathcal{K}(f)
\\[14pt]
\Bbbk^n &=&\mathcal{J}^0(f)
	&\supset&\mathcal{J}^1(f)
	&\supset&\dots
	&\supset&\mathcal{J}^{k}(f)
	&\supset&\mathcal{J}^{k+1}(f)
	&\supset&\dots
	&\supset&\displaystyle\bigcap_{k=0}^\infty \mathcal{J}^k(f)
	&=:&\mathcal{J}(f)
\end{array}
\]
\end{block}
\vspace{-20pt}
\begin{columns}[t,onlytextwidth]
\begin{column}{0.48\textwidth}
\begin{block}{Abildung der Kerne}
\vspace{-10pt}
\begin{align*}
f \mathcal{K}^k(f)
&=
\{f(v)\;|\; f^k(v) = 0\}
\\
&\subset
\{ v\;|\; f^{k+1}(v)=0\}
\\
&=
\mathcal{K}^{k+1}(f)
\\
\Rightarrow
f\mathcal{K}(f)&= f\mathcal{K}(f)
\quad\text{invariant}
\end{align*}
\end{block}
\end{column}
\begin{column}{0.48\textwidth}
\begin{block}{Abbildung der Bild}
\vspace{-10pt}
\begin{align*}
f\mathcal{J}^k(f)
&=
\{f(f^{k}(v))\;|\; v\in V\}
\\
&=
\{f^{k+1}(v)\;|\; v\in V\}
\\
&=
\mathcal{J}^{k+1}(f)
\\
\Rightarrow
f\mathcal{J}(f)&= \mathcal{J}(f)
\quad\text{invariant}
\end{align*}
\end{block}
\end{column}
\end{columns}
\end{frame}
