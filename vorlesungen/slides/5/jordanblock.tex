%
% jordanblock.tex
%
% (c) 2021 Prof Dr Andreas Müller, OST Ostschweizer Fachhochschule
%
\bgroup

\def\NL{
\ifthenelse{\boolean{presentation}}{
\only<-8>{\phantom{\lambda}\llap{$0$}}\only<9->{\lambda}
}{
\lambda
}
}

\begin{frame}[t]
\frametitle{Jordan-Block}
\vspace{-15pt}
\begin{columns}[t,onlytextwidth]
\begin{column}{0.48\textwidth}
\begin{block}{Gegeben}
Matrix $A\in M_n(\Bbbk)$ derart, dass
\begin{itemize}
\item<2->
$A-\lambda I$ nilpotent
\item<5->
$A^{n-1}\ne 0$
\end{itemize}
\end{block}
\vspace{-5pt}
\uncover<3->{
\begin{block}{Folgerungen}
Es gibt eine Basis derart, dass
\begin{enumerate}
\item<4->
$A-\lambda I$ hat Normalform einer nilpotenten Matrix
\item<6->
Es gibt nur einen Block, da $\dim\ker(A-\lambda I)=1$
\end{enumerate}
\end{block}}
\end{column}
\begin{column}{0.48\textwidth}
\uncover<4->{%
\begin{block}{\ifthenelse{\boolean{presentation}}{\only<-8>{Normalform einer nilpotenten Matrix\strut}}{}\only<9->{Normalform: genau ein Eigenwert\strut}}
\[
A\uncover<-8>{-\lambda I}=\begin{pmatrix}
\NL &1& & & & & & & \\
 &\NL &1& & & & & & \\
 & &\NL &\uncover<7->{{\color<7>{red}1}}& & & & & \\
 & & &\NL &1& & & & \\
 & & & &\NL &1& & & \\
 & & & & &\NL &1& & \\
 & & & & & &\NL &\uncover<7->{{\color<7>{red}1}}& \\
 & & & & & & &\NL &\uncover<7->{{\color<7>{red}1}}\\
 & & & & & & & &\NL
\end{pmatrix}
\]
\end{block}}
\end{column}
\end{columns}
\vspace{-5pt}
\uncover<8->{%
\begin{block}{Jordan-Normalform}
In dieser Basis hat $A$ Jordan-Normalform
\end{block}}
\end{frame}

\egroup
