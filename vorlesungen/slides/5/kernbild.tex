%
% kernbild.tex
%
% (c) 2021 Prof Dr Andreas Müller, OST Ostschweizer Fachhochschule
%
\begin{frame}[t]
\setlength{\abovedisplayskip}{5pt}
\setlength{\belowdisplayskip}{5pt}
\frametitle{Kern und Bild}
\vspace{-15pt}
\begin{columns}[t,onlytextwidth]
\begin{column}{0.48\textwidth}
\begin{block}{Kern}
Lineare Abbildung $f\colon V\to V$
\[
\ker f = \mathcal{K}(F) = \{v\in V\;|\; f(v)=0\}
\]
\end{block}
\begin{block}{Kern von $A^k$}
\[
\mathcal{K}^k(f) = \operatorname{ker} f^k
\]
\begin{align*}
\mathcal{K}^k(f)
&=
\{v\in V\;|\; f^{k}(v)=0\}
\\
&\subset
\{v\in V\;|\; f^{k+1}(v)=0\}
\\
&=\mathcal{K}^{k+1}(f)
\end{align*}
\end{block}
\end{column}
\begin{column}{0.48\textwidth}
\begin{block}{Bild}
Lineare Abbildung $f\colon V\to V$
\[
\operatorname{im}f
=
\mathcal{J}(f)
=
\{f(v)\;|\; v\in V\}
\]
\end{block}
\begin{block}{Bild von $A^k$}
\[
\mathcal{J}^k(f) = \operatorname{im}f^k
\]
\begin{align*}
\mathcal{J}^k(f)
&=
\operatorname{im}f^k
=
\operatorname{im}(f^{k}\circ f)
\\
&=
\{f^{k-1} w\;|\; w = f(v)\}
\\
&\subset
\{f^{k-1} w\;|\; w \in V\}
\\
&\mathcal{J}^{k-1}(f)
\end{align*}
\end{block}
\end{column}
\end{columns}
\end{frame}
