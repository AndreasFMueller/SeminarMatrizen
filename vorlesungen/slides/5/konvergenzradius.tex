%
% konvergenzradius.tex
%
% (c) 2021 Prof Dr Andreas Müller, OST Ostschweizer Fachhochschule
%
\bgroup
\setbeamercolor{column}{bg=blue!20}
\def\punkt#1{
	\fill[color=blue!30] #1 circle[radius=0.05];
	\draw[color=blue] #1 circle[radius=0.05];
}
\definecolor{darkgreen}{rgb}{0,0.6,0}
\begin{frame}
\setlength{\abovedisplayskip}{5pt}
\setlength{\belowdisplayskip}{5pt}
\frametitle{Konvergenzradius}
\vspace{-15pt}
\begin{columns}[t,onlytextwidth]
\begin{column}{0.48\textwidth}
\begin{block}{Potenzreihen}
$f\colon\mathbb{C}\to\mathbb{C}$ (komplex differenzierbar)
\begin{equation}
f(z) = \sum_{k=0}^\infty a_kz^k
\label{reihe}
\end{equation}
\end{block}
\vspace{-8pt}
\uncover<2->{%
\begin{block}{Konvergenz}
\eqref{reihe} konvergiert für $|z| < {\color{darkgreen}R}$,
\[
\frac{1}{{\color{darkgreen}R}}
=
\limsup_{k\to\infty} |a_k|^{\frac1k}
\]
\end{block}}
\uncover<3->{%
\begin{block}{Polstellen}
{\color{darkgreen}$R$} ist der Radius des grössten Kreises um $O$,
auf dessen Rand eine
{\color{blue}Polstelle} der Funktion $f(z)$ liegt
\end{block}}
\end{column}
\begin{column}{0.48\textwidth}
\begin{center}
\begin{tikzpicture}[>=latex,thick]
\def\r{2.5}
\uncover<2->{
	\fill[color=red!20] (0,0) circle[radius=\r];
	\draw[color=red] (0,0) circle[radius=\r];
}
\draw[->] (-2.6,0) -- (2.9,0) coordinate[label={$\operatorname{Re}z$}];
\draw[->] (0,-2.6) -- (0,2.9) coordinate[label={$\operatorname{Im}z$}];

\uncover<2->{
	\draw[->,color=darkgreen,shorten >= 0.05cm] (0,0) -- (100:\r);
	\draw[->,color=darkgreen,shorten >= 0.05cm] (0,0) -- (220:\r);
	\node[color=darkgreen] at ($0.5*(100:\r)$) [left] {$R$};
	\node[color=darkgreen] at ($0.5*(220:\r)+(-0.1,0.1)$)
		[below right] {$R$};

	\fill[color=white] (0,0) circle[radius=0.05];
	\draw (0,0) circle[radius=0.05];
}

\node at (2.8,2.8) {$\mathbb{C}$};

\uncover<3->{
	\punkt{(100:\r)}
	\punkt{(220:\r)}

	\begin{scope}
		\clip (-2.6,-2.6) rectangle (2.9,2.9);

		\punkt{(144.2527:2.7232)}
		%\punkt{(226.1822:2.5164)}
		\punkt{(173.7501:3.4140)}
		\punkt{(267.4103,2.7668)}
		\punkt{(137.7328:3.1683)}
		%\punkt{(30.1155:3.3629)}
		%\punkt{(139.1036:2.5366)}
		\punkt{(167.4964:3.0503)}
		\punkt{(289.2650:3.4324)}
		\punkt{(120.1911:3.2966)}
		%\punkt{(292.3422:2.7550)}
		\punkt{(141.4877:2.6494)}
		\punkt{(70.8326:2.9005)}
		\punkt{(56.0758:3.2098)}
		\punkt{(99.0585:3.2340)}
		\punkt{(299.7242:2.5990)}
		\punkt{(158.8802:2.6539)}
		\punkt{(235.2721:2.9476)}
		\punkt{(108.0584:2.8344)}
		\punkt{(220.0117:2.7679)}

	\end{scope}

	\begin{scope}[yshift=-3.2cm,xshift=-1.0cm]
		\punkt{(0,-0.05)}
		\node at (0,0) [right] {$=$ Polstelle};
	\end{scope}
}

\end{tikzpicture}
\end{center}
\end{column}
\end{columns}
\end{frame}
\egroup
