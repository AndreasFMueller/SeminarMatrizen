%
% cayleyhamilton.tex
%
% (c) 2021 Prof Dr Andreas Müller, OST Ostschweizer Fachhochschule
%
\begin{frame}[t]
\setlength{\abovedisplayskip}{5pt}
\setlength{\belowdisplayskip}{5pt}
\frametitle{Satz von Cayley-Hamilton}
\vspace{-15pt}
\begin{columns}[t,onlytextwidth]
\begin{column}{0.48\textwidth}
\begin{block}{Ein Eigenwert $\lambda$\strut}
$A$ besteht aus
$b$ Blöcken $J_\lambda$ mit maximaler Dimension $l$:
\phantom{blubb\strut}
\begin{align*}
\uncover<2->{
\chi_{A}(X)
&=
\det (A-XI) = (\lambda-X)^n
}
\\
\uncover<3->{
m_{A}(X)
&=
(\lambda-X)^l
}
\\
\uncover<4->{
b&= \ker A
}
\end{align*}
\uncover<5->{%
Wegen $l \le n$ folgt
\[
m_A(X) | \chi_A(X)
\uncover<6->{\quad\Rightarrow\quad
\chi_A(A) = 0}
\]}
\end{block}
\end{column}
\begin{column}{0.48\textwidth}
\uncover<7->{%
\begin{block}{$A=A_1\oplus\dots\oplus A_k$}
\uncover<8->{%
$A_i\in M_{n_i}(\Bbbk)$ mit EW $\lambda_i$,
$A_i$ besteht aus
$b_i$ Blöcken $J_{\lambda_i}$ mit max.~Dimension $l_i$\strut:}
\begin{align*}
\uncover<9->{
\chi_A(X)
&=
(\lambda_1-X)^{n_1}
\dots
(\lambda_k-X)^{n_k}
}
\\
\uncover<10->{
m_A(X)
&=
(\lambda_1-X)^{l_1}
\dots
(\lambda_k-X)^{l_k}
}
\\
\uncover<11->{
b_i &= \ker (A-\lambda_iI)
}
\end{align*}
\uncover<12->{%
$A=A_1\oplus\dots\oplus A_k$}
\begin{align*}
\uncover<13->{
\chi_{A_i}(A_i)&=0\;\forall i
}
\\
\uncover<14->{%
\chi_A(A) &=
\chi_{A_1}(A)\dots\chi_{A_k}(A)
 = 0}
\end{align*}
\end{block}}
\end{column}
\end{columns}
\uncover<15->{%
\begin{block}{Satz}
Für jede Matrix $A\in M_n(\Bbbk)$ gilt
$m_A(X) | \chi_A(X)$ oder $\chi_A(A)=0$
\end{block}}
\end{frame}
