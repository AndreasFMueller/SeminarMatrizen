%
% movitation.tex
%
% (c) 2021 Prof Dr Andreas Müller, OST Ostschweizer Fachhochschule
%
\begin{frame}[t]
\setlength{\abovedisplayskip}{5pt}
\setlength{\belowdisplayskip}{5pt}
\frametitle{Motivation}
\begin{columns}[t,onlytextwidth]
\begin{column}{0.48\textwidth}
\begin{block}{Matrix $A$ analysieren}
Matrix $A$ mit Minimalpolynom $m_A(X)$ vom
Grad $s$
\end{block}
\begin{block}{Faktorisieren}
Minimalpolynom faktorisieren
\[
m_A(X)
=
(X-\mu_1)(X-\mu_2)\dots(X-\mu_s)
\]
\end{block}
\begin{block}{Vertauschen}
$\sigma\in S_s$ eine Permutation von $1,\dots,s$
ist
\begin{align*}
m_A(X)
&=
(X-\mu_{\sigma(1)})
%(X-\mu_{\sigma(2)})
\dots
(X-\mu_{\sigma(s)})
\\
0
&=
(A-\mu_{\sigma(1)})
%(A-\mu_{\sigma(2)})
\dots
(A-\mu_{\sigma(s)})
\end{align*}
\end{block}
\end{column}
\begin{column}{0.48\textwidth}
\begin{block}{Bedingung für $\mu_k$}
Permutation wählen so dass $\mu_k$ an erster Stelle steht:
\[
0=(A-\mu_k) \prod_{i\ne k}(A-\mu_i) v
\]
für alle $v\in\Bbbk^n$.
\end{block}
\begin{block}{Eigenwerte}
Nur diejenigen ${\color{red}\mu}$ sind möglich, für die es $v\in\Bbbk^n$
gibt mit
\[
(A-\mu)v = 0
\Rightarrow Av = {\color{red}\mu} v
\]
Eigenwertbedingung
\end{block}
\end{column}
\end{columns}
\end{frame}
