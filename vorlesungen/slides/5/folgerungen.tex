%
% folgerungen.tex
%
% (c) 2021 Prof Dr Andreas Müller, OST Ostschweizer Fachhochschule
%
\begin{frame}[t]
\frametitle{Folgerungen}
\begin{columns}[t]
\begin{column}{0.48\textwidth}
\begin{block}{Zunahme}
Für alle $k<l$ gilt
\begin{align*}
\mathcal{J}^k(f) &\supsetneq \mathcal{J}^{k+1}(f)
\\
\mathcal{K}^k(f) &\subsetneq \mathcal{K}^{k+1}(f)
\end{align*}
Für $k\ge l$ gilt
\begin{align*}
\mathcal{J}^k(f) &= \mathcal{J}^{k+1}(f)
\\
\mathcal{K}^k(f) &= \mathcal{K}^{k+1}(f)
\end{align*}
Ausserdem ist $l\le n$
\end{block}
\end{column}
\begin{column}{0.48\textwidth}
\begin{proof}[Beweis]
XXX
\end{proof}
\end{column}
\end{columns}
\end{frame}
