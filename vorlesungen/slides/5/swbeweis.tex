%
% swbeweis.tex
%
% (c) 2021 Prof Dr Andreas Müller, OST Ostschweizer Fachhochschule
%
\bgroup
\definecolor{darkgreen}{rgb}{0,0.6,0}
\begin{frame}[t]
\setlength{\abovedisplayskip}{5pt}
\setlength{\belowdisplayskip}{5pt}
\frametitle{Beweisidee Stone-Weierstrass}
\vspace{-15pt}
\begin{columns}[t]
\begin{column}{0.5\textwidth}
\begin{enumerate}
\item<1->
$\exists$ eine monoton wachsende Folge von Polynomen $u_n(t)\to \sqrt{t}$
gleichmässig auf $[0,1]\subset{\color{darkgreen}\mathbb{R}}$
\item<2->
$f\in A$, dann kann man $|f| = \sqrt{f^2}$ beliebig genau approximieren
durch Funktionen
in $A$
\item<3->
$f,g\in A$, dann kann
\begin{align*}
\max(a,b)&={\textstyle\frac12}(f+g+|f-g|)\\
\min(a,b)&={\textstyle\frac12}(f+g-|f-g|)
\end{align*}
in $A$ beliebig genau approximiert werden.
\end{enumerate}
\end{column}
\begin{column}{0.5\textwidth}
\begin{enumerate}
\setcounter{enumi}{3}
\item<4->
Für $x,y\in D$ und $\alpha,\beta\in\mathbb{R}$ gibt es $f\in A$ mit
$f(x)=\alpha$ und $f(y)=\beta$
\item<5->
Zu
$f\colon D\to\mathbb{R}$ stetig und $x\in D$ gibt es $g\in A$ mit $g(x)=f(x)$
und $g(y) \le f(y)+\varepsilon$ für $y\ne x$
\item<6->
Für $f$ gibt es endlich viele Approximationen $g_i$ mit Punkten $x_i$
wie in Schritt~4.
Dann ist $\max_i g_i$ eine Approximation von $f$, die beliebig genau in
$A$ approximiert werden kann.
\end{enumerate}
\end{column}
\end{columns}

\vspace{10pt}
\uncover<7->{%
Schritt~2 braucht in {\color{red}$\mathbb{C}$} die komplex Konjugierte:
$|f|^2=f\overline{f}$}
\end{frame}
\egroup
