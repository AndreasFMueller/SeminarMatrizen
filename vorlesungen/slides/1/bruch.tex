%
% bruch.tex
%
% (c) 2021 Prof Dr Andreas Müller, OST Ostschweizer Fachhochschule
%
\begin{frame}[t]
\frametitle{Brüche}
\setlength{\abovedisplayskip}{5pt}
\setlength{\belowdisplayskip}{5pt}
\vspace{-8pt}
\begin{columns}[t,onlytextwidth]
\begin{column}{0.48\textwidth}
\begin{block}{Division}
Nicht für alle $a,b\in\mathbb{Z}$ hat die Gleichung
\[
ax=b
\uncover<2->{
\;\Rightarrow\;
x=\frac{b}{a}}
\]
eine Lösung in $\mathbb{Z}$\uncover<2->{, nämlich wenn $b\nmid a$}
\end{block}
\uncover<3->{%
\begin{block}{Brüche}
Idee: $\displaystyle\frac{b}{a} = (b,a)$
\begin{enumerate}
\item<4-> $(b,a)\in\mathbb{Z}\times\mathbb{Z}$
\item<5-> Äquivalenzrelation
\[
(b,a)\sim (d,c)
\only<5>{
\Leftrightarrow
\text{``
$\displaystyle
\frac{b}{a}=\frac{d}{c}
$
''}
}
\only<6->{
\Leftrightarrow
bc=ad
}
\]
\end{enumerate}
\vspace{-15pt}
\uncover<7->{%
$\Rightarrow$ alle Quotienten
}
\end{block}}
\end{column}
\begin{column}{0.48\textwidth}
\uncover<9->{%
\begin{block}{Gruppe}
$\mathbb{Q}^* = \mathbb{Q}\setminus\{0\}$ ist eine multiplikative Gruppe:
\begin{enumerate}
\item<10-> Neutrales Element: $1\in \mathbb{Q}^*$
\item<11-> Inverses Element $q=\frac{b}{a}\in\mathbb{Q}
\Rightarrow
q^{-1}=\frac{a}{b}\in\mathbb{Q}$
\end{enumerate}
\end{block}
}
\uncover<8->{%
\begin{block}{Rationale Zahlen}
Alle Brüche, gleiche Werte zusammengefasst:
\[
\mathbb{Q} = \mathbb{Z}\times\mathbb{Z}/\sim
\]
\end{block}}
\end{column}
\end{columns}
\end{frame}
