%
% zahlensysteme.tex
%
% (c) 2021 Prof Dr Andreas Müller, Hochschule Rapperswil
%
\begin{frame}[t]
\frametitle{Zahlensysteme}
\begin{center}
\begin{tabular}{|>{$}c<{$}|p{7cm}|p{3cm}|}
\hline
\text{Zahlenmenge}&\text{Eigenschaften}&\text{Struktur}
\\
\hline
\mathbb{N}
&\phantom{}\raggedright\uncover<2->{Addition, neutrales Element $0$}
&\phantom{}\uncover<2->{Monoid}
\\
\mathbb{Z}
&\phantom{}\raggedright\uncover<3->{Addition, neutrales Element $0$,
inverses Element der Addition}
&\phantom{}\uncover<3->{Gruppe}
\\
\mathbb{Z}
&\phantom{}\raggedright\uncover<4->{zusätzlich: Multiplikation, neutrales Element $1$}
&\phantom{}\uncover<4->{Ring}
\\
\mathbb{Q}
&\phantom{}\raggedright\uncover<5->{Addition und Multiplikation mit Inversen}
&\phantom{}\uncover<5->{Körper}
\\
\mathbb{R}
&\phantom{}\raggedright\uncover<6->{zusätzlich: Ordnungsrelation, Vollständigkeit}
&\phantom{}\uncover<6->{Körper mit Ordnung}
\\
\mathbb{C}
&\phantom{}\raggedright\uncover<7->{zusätzlich: Alle Wurzeln}
&\phantom{}\uncover<7->{algebraisch abgeschlossener Körper}
\\
\uncover<8->{\mathbb{H}}
&\phantom{}\raggedright\uncover<8->{höhere Dimension, nichtkommutativ}
&\phantom{}\uncover<8->{Schiefkörper}
\\
\hline
\end{tabular}
\end{center}
\end{frame}
