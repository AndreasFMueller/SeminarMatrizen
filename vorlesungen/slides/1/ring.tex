%
% ring.tex
%
% (c) 2021 Prof Dr Andreas Müller, Hochschule Rapperswil
%
\begin{frame}[t]
\frametitle{Ring\only<15->{/Körper}}
\vspace{-10pt}
\setlength{\abovedisplayskip}{5pt}
\setlength{\belowdisplayskip}{5pt}
\begin{columns}[t,onlytextwidth]
\begin{column}{0.48\textwidth}
\begin{block}{Addition und Multiplikation}
$\mathbb{Z}$ und $\mathbb{Q}$
haben zwei Verknüpfungen:
\begin{enumerate}
\item<2-> Addition
\[
a,b\in R\Rightarrow a+b\in R
\]
\item<3-> Multiplikation
\[
a,b\in R\Rightarrow a\cdot b=ab\in R
\]
\end{enumerate}
\vspace{-5pt}
\uncover<4->{%
Gilt auch für
\begin{itemize}
\item<5-> Polynome
\item<6-> $M_{n}(\mathbb{R})$
\item<7-> $\mathbb{R}^3$ mit Vektorprodukt
\end{itemize}}
\end{block}
\end{column}
\begin{column}{0.48\textwidth}
\uncover<8->{%
\begin{block}{Definition}
Ein Ring\only<15->{/{\color{red}Körper}} ist eine Menge $R$ mit zwei
Verknüpfungen $+$ und $\cdot$:
\begin{enumerate}
\item<9->
$R$ mit $+$ ist eine abelsche Gruppe
\item<10->
$R$ mit $\cdot$ ist ein Monoid\only<15->{/{\color{red}eine Gruppe}}
\item<11->
Verträglichkeit: Distributivgesetz
\begin{align*}
\uncover<12->{a(b+c)&=ab+bc}
\\
\uncover<13->{(a+b)c&=ac+bc}
\end{align*}
\uncover<14->{(Ausmultiplizieren)}
\end{enumerate}
\end{block}}
\end{column}
\end{columns}
\end{frame}
