%
% dreieck.tex
%
% (c) 2021 Prof Dr Andreas Müller, OST Ostschweizer Fachhochschule
%
\begin{frame}[t]
\frametitle{Dreiecksmatrizen}
\begin{columns}[t,onlytextwidth]
\begin{column}{0.31\textwidth}
\begin{block}{Dreiecksmatrix}
\begin{align*}
R&=
\begin{pmatrix}
*&*&*&\dots&*\\
0&*&*&\dots&*\\
0&0&*&\dots&*\\
\vdots&\vdots&\vdots&\ddots&\vdots\\
0&0&0&\dots&*
\end{pmatrix}
\\
U&=
\begin{pmatrix}
1&*&*&\dots&*\\
0&1&*&\dots&*\\
0&0&1&\dots&*\\
\vdots&\vdots&\vdots&\ddots&\vdots\\
0&0&0&\dots&1
\end{pmatrix}
\end{align*}
\end{block}
\end{column}
\begin{column}{0.31\textwidth}
\begin{block}{Nilpotente Matrix}
\[
N=
\begin{pmatrix}
*&*&*&\dots&*\\
0&*&*&\dots&*\\
0&0&*&\dots&*\\
\vdots&\vdots&\vdots&\ddots&\vdots\\
0&0&0&\dots&*
\end{pmatrix}
\]
$\Rightarrow N^n=0$
\end{block}
\end{column}
\begin{column}{0.31\textwidth}
\begin{block}{Jordan-Matrix}
\[
J_\lambda=\begin{pmatrix}
\lambda&1&0&\dots&0\\
0&\lambda&1&\dots&0\\
0&0&\lambda&\dots&0\\
\vdots&\vdots&\vdots&\ddots&\vdots\\
0&0&0&\dots&\lambda
\end{pmatrix}
\]
$\Rightarrow J_\lambda -\lambda I$ ist nilpotent
\end{block}
\end{column}
\end{columns}
\end{frame}
