%
% peano.tex
%
% (c) 2021 Prof Dr Andreas Müller, OST Ostschweizer Fachhochschule
%
\begin{frame}[t]
\frametitle{Natürliche Zahlen\uncover<2->{: Peano}}
\vspace{-20pt}
\begin{columns}[t,onlytextwidth]
\begin{column}{0.48\textwidth}
\begin{block}{Zählen}
Mit den natürlichen Zahlen zählt man: 
\[
\mathbb{N}
=
\left\{
\begin{minipage}{5cm}
\raggedright
Äquivalenzklassen von gleich mächtigen
endlichen Mengen
\end{minipage}
\right\}
\]
\end{block}
\vspace{-10pt}
\uncover<2->{%
\begin{block}{Peano-Axiome}
\begin{enumerate}
\item<3-> $0\in\mathbb{N}$
\item<4-> $n\in\mathbb{N}\Rightarrow \text{Nachfolger }n'\in\mathbb{N}$
\item<5-> $0$ ist nicht Nachfolger
\item<6-> $n,m\in\mathbb{N}\wedge n'=m'\Rightarrow n=m$
\item<7-> $X\subset \mathbb{N}\wedge 0\in X\wedge \forall n\in X(n'\in X)
\Rightarrow
\mathbb{N}=X
$
\end{enumerate}
\end{block}}
\end{column}
\begin{column}{0.48\textwidth}
\uncover<8->{%
\begin{block}{Monoid}
\setlength{\abovedisplayskip}{5pt}
\setlength{\belowdisplayskip}{5pt}
Menge $\only<8-10>{\mathbb{N}}\only<11->{M}$ mit einer
zweistelligen Verknüpfung $a\only<8-10>{+}\only<11->{*}b$
\begin{enumerate}
\item<9-> Assoziativ: $a,b,c\in M$
\[
(a\only<8-10>{+}\only<11->{*}b)\only<8-10>{+}\only<11->{*}c=a\only<8-10>{+}\only<11->{*}(b\only<8-10>{+}\only<11->{*}c)
\]
\item<10-> Neutrales Element: $\only<8-10>{0}\only<11->{e}\in M$
\[
\only<8-10>{0+}\only<11->{e*} a
=
a \only<8-10>{+0}\only<11->{*e}
\]
\end{enumerate}
\end{block}}%
\vspace{-15pt}
\uncover<12->{%
\begin{block}{Axiom 5 = Vollständige Induktion}
$X=\{n\in\mathbb{N}\;|\; \text{$P(n)$ ist wahr}\}$
\begin{enumerate}
\item<13-> Verankerung: $0\in X$
\item<14-> Induktionsannahme: $n\in X$
\item<15-> Induktionsschritt: $n'\in X$
\end{enumerate}
\end{block}}
\end{column}
\end{columns}
\end{frame}
