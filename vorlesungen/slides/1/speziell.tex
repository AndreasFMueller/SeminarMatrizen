%
% speziell.tex
%
% (c) 2021 Prof Dr Andreas Müller, OST Ostschweizer Fachhochschule
%
\begin{frame}[t]
\begin{columns}[t,onlytextwidth]
\begin{column}{0.38\textwidth}
\frametitle{Diagonalmatrizen}
\begin{block}{Einheitsmatrix}
\[
I=\begin{pmatrix}
1&0&\dots&0\\
0&1&\dots&0\\
\vdots&\vdots&\ddots&\vdots\\
0&0&\dots&1
\end{pmatrix}
\]
Neutrales Element der Matrixmultiplikation:
\[
AI=IA=A
\]
\end{block}
\end{column}
\begin{column}{0.58\textwidth}
\uncover<2->{%
\begin{block}{Diagonalmatrix}
\[
\operatorname{diag}(\lambda_1,\lambda_2,\dots,\lambda_n)
=
\begin{pmatrix}
\lambda_1&0&\dots&0\\
0&\lambda_2&\dots&0\\
\vdots&\vdots&\ddots&\vdots\\
0&0&\dots&\lambda_n
\end{pmatrix}
\]
\end{block}}
\uncover<3->{%
\begin{block}{Hadamard-Algebra}
Die Algebra der Diagonalmatrizen ist die Hadamard-Algebra
(siehe später)
\end{block}}
\end{column}
\end{columns}
\end{frame}
