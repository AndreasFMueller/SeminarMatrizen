%
% so2.tex -- Illustration of so(2) -> SO(2)
%
% (c) 2021 Prof Dr Andreas Müller, OST Ostschweizer Fachhochschule
% Erstellt durch Roy Seitz
%
% !TeX spellcheck = de_CH
\bgroup

\begin{frame}[t]
  \setlength{\abovedisplayskip}{5pt}
  \setlength{\belowdisplayskip}{5pt}
  \frametitle{Von der Lie-Gruppe zur -Algebra}
  \vspace{-20pt}
  \begin{columns}[t,onlytextwidth]
    \begin{column}{0.48\textwidth}
      \uncover<1->{
        \begin{block}{Lie-Gruppe}
          Darstellung von \gSO2:
          \begin{align*}
            \mathbb R 
            &\to 
            \gSO2
            \\
            t
            &\mapsto
            \begin{pmatrix}
              \cos t &         -\sin t \\ 
              \sin t & \phantom-\cos t
            \end{pmatrix}
          \end{align*}
        \end{block}
      }
      \uncover<2->{
        \begin{block}{Ableitung am neutralen Element}
          \begin{align*}
            \frac{d}{d t}
            &
            \left.
            \begin{pmatrix}
              \cos t &         -\sin t \\ 
              \sin t & \phantom-\cos t
            \end{pmatrix}
            \right|_{ t = 0}
            \\
            =
            & 
            \begin{pmatrix} -\sin0 & -\cos0 \\ \phantom-\cos0 & -\sin0 \end{pmatrix}
            = 
            \begin{pmatrix} 0 & -1 \\ 1 &  \phantom-0 \end{pmatrix}
          \end{align*}
        \end{block}
      }
    \end{column}
    \begin{column}{0.48\textwidth}
      \uncover<3->{
        \begin{block}{Lie-Algebra}
          Darstellung von \aso2:
          \begin{align*}
            \mathbb R 
            &\to 
            \aso2
            \\
            t
            &\mapsto
            \begin{pmatrix}
              0 &         -t \\ 
              t & \phantom-0
            \end{pmatrix}
          \end{align*}
        \end{block}
      }
    \end{column}
  \end{columns}
\end{frame}


\begin{frame}[t]
  \setlength{\abovedisplayskip}{5pt}
  \setlength{\belowdisplayskip}{5pt}
  \frametitle{Von der Lie-Algebra zur -Gruppe}
  \vspace{-20pt}
  \begin{columns}[t,onlytextwidth]
    \begin{column}{0.48\textwidth}
      \uncover<1->{
      \begin{block}{Differentialgleichung}
        Gegeben:
        \[
        J
        =
        \dot\gamma(0) = \begin{pmatrix} 0 & -1 \\ 1 & \phantom-0 \end{pmatrix}
        \]
        Gesucht:
        \[ \dot \gamma (t) = J \gamma(t) \qquad \gamma \in \gSO2 \]
        \[ \Rightarrow \gamma(t) = \exp(Jt) \gamma(0) = \exp(Jt) \]
      \end{block}
    }
    \end{column}
    \begin{column}{0.48\textwidth}
      \uncover<2->{
      \begin{block}{Lie-Algebra}
        Potenzen von $J$:
        \begin{align*}
          J^2 &= -I &
          J^3 &= -J &
          J^4 &=  I &
          \ldots
        \end{align*}
      \end{block}
    }
    \end{column}
  \end{columns}
\uncover<3->{
  Folglich:
  \begin{align*}
    \exp(Jt)
    &= I + Jt 
    + J^2\frac{t^2}{2!} 
    + J^3\frac{t^3}{3!}
    + J^4\frac{t^4}{4!}
    + J^5\frac{t^5}{5!}
    + \ldots \\
    &= \begin{pmatrix}
      \vspace*{3pt}
      1 - \frac{t^2}{2} + \frac{t^4}{4!} - \ldots
      &
      -t + \frac{t^3}{3!} - \frac{t^5}{5!} + \ldots
      \\ 
      t - \frac{t^3}{3!} + \frac{t^5}{5!} - \ldots
      &
      1 - \frac{t^2}{2!} + \frac{t^4}{4!} - \ldots
    \end{pmatrix}
    =
    \begin{pmatrix}
      \cos t &         -\sin t \\ 
      \sin t & \phantom-\cos t
    \end{pmatrix}
  \end{align*}
  }
\end{frame}
\egroup
