%
% ableitung-exp.tex -- Ableitung von exp(x)
%
% (c) 2021 Prof Dr Andreas Müller, OST Ostschweizer Fachhochschule
% Erstellt durch Roy Seitz
%
% !TeX spellcheck = de_CH
\bgroup
\begin{frame}[t]
  \setlength{\abovedisplayskip}{5pt}
  \setlength{\belowdisplayskip}{5pt}
  %\frametitle{Ableitung von $\exp(x)$}
  %\vspace{-20pt}
  \begin{columns}[t,onlytextwidth]
    \begin{column}{0.48\textwidth}
      \begin{block}{Ableitung von $\exp(at)$}
        \begin{align*}
          \frac{d}{dt} \exp(at)
          &=
          \frac{d}{dt} \sum_{k=0}^{\infty} a^k \frac{t^k}{k!}
          \\
          &\uncover<2->{
            = \sum_{k=0}^{\infty} a^k\frac{kt^{k-1}}{k(k-1)!}
          }
          \\
          &\uncover<3->{
            = a \sum_{k=1}^{\infty}
            a^{k-1}\frac{t^{k-1}}{(k-1)!}
          }
          \\
          &\uncover<4->{
            = a \exp(at)
          }
        \end{align*}
      \end{block}
    \end{column}
    \begin{column}{0.48\textwidth}
      \uncover<5->{
        \begin{block}{Ableitung von $\exp(At)$}
          \begin{align*}
            \frac{d}{dt} \exp(At)
            &=
            \frac{d}{dt} \sum_{k=0}^{\infty} A^k \frac{t^k}{k!}
            \\
            &=
            \sum_{k=0}^{\infty} A^k\frac{kt^{k-1}}{k(k-1)!}
            \\
            &=
            A \sum_{k=1}^{\infty} A^{k-1}\frac{t^{k-1}}{(k-1)!}
            \\
            &=
            A \exp(At)
          \end{align*}
        \end{block}
      }
    \end{column}
  \end{columns}  
\end{frame}

\egroup
