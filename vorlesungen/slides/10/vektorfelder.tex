%
% iterativ.tex -- Iterative Approximation in \dot x = J x
%
% (c) 2021 Prof Dr Andreas Müller, OST Ostschweizer Fachhochschule
% Erstellt durch Roy Seitz
%
% !TeX spellcheck = de_CH
\bgroup
\begin{frame}[t]
  \setlength{\abovedisplayskip}{5pt}
  \setlength{\belowdisplayskip}{5pt}
  \frametitle{Als Strömungsfeld}
  \vspace{-20pt}
  \begin{columns}[t,onlytextwidth]
    \begin{column}{0.48\textwidth}
      \vfil
      \only<1|handout:0>{
        \includegraphics[width=\linewidth,keepaspectratio]
        {../slides/10/vektorfelder-1.pdf}
      }
      \only<2|handout:0>{
        \includegraphics[width=\linewidth,keepaspectratio]
        {../slides/10/vektorfelder-2.pdf}
      }
      \only<3>{
        \includegraphics[width=\linewidth,keepaspectratio]
        {../slides/10/vektorfelder-3.pdf}
      }
      \only<4|handout:0>{
        \includegraphics[width=\linewidth,keepaspectratio]
        {../slides/10/vektorfelder-4.pdf}
      }
      \only<5|handout:0>{
        \includegraphics[width=\linewidth,keepaspectratio]
        {../slides/10/vektorfelder-5.pdf}
      }
      \only<6-|handout:0>{
        \includegraphics[width=\linewidth,keepaspectratio]
        {../slides/10/vektorfelder-6.pdf}
      }
      \vfil
    \end{column}
    \begin{column}{0.48\textwidth}
      \begin{block}{Differentialgleichung}
        \[ 
        \dot x(t) = J x(t)
        \quad
        J = \begin{pmatrix} 0 & -1 \\ 1 & \phantom-0 \end{pmatrix}
        \quad
        x_0 = \begin{pmatrix} 1 \\ 0 \end{pmatrix}
        \]
      \end{block}
    
      \only<2|handout:0>{
        Nach einem Schritt der Länge $t$:
        \[
        x(t) = x_0 + \dot x t = x_0 + Jx_0t = (1 + Jt)x_0
        \]
      }

      \only<3|handout:0>{
        Nach zwei Schritten der Länge $t/2$:
        \[
        x(t) = \left(1 + \frac{Jt}{2}\right)^2x_0
        \]
      }

      \only<4->{
        Nach n Schritten der Länge $t/n$:
        \[
        x(t) = \left(1 + \frac{Jt}{n}\right)^nx_0
        \]
      }
      \only<6->{
      \[
      \lim_{n\to\infty}\left(1 + \frac{At}{n}\right)^n = \exp(At)
      \]
      }
    \end{column}
  \end{columns}
\end{frame}
\egroup
