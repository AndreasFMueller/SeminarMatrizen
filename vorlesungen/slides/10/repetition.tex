%
% intro.tex -- Repetition Lie-Gruppen und -Algebren
%
% (c) 2021 Prof Dr Andreas Müller, OST Ostschweizer Fachhochschule
% Erstellt durch Roy Seitz
%
% !TeX spellcheck = de_CH
\bgroup

\begin{frame}[t]
  \setlength{\abovedisplayskip}{5pt}
  \setlength{\belowdisplayskip}{5pt}
  \frametitle{Repetition}
  \vspace{-20pt}
  \begin{columns}[t,onlytextwidth]
    \begin{column}{0.48\textwidth}
      \uncover<1->{
        \begin{block}{Lie-Gruppe}
          Kontinuierliche Matrix-Gruppe $G$ mit bestimmter Eigenschaft
        \end{block}
      }
      \uncover<3->{
        \begin{block}{Ein-Parameter-Untergruppe}
          Darstellung der Lie-Gruppe $G$:
          \[
          \gamma \colon \mathbb R \to G
          : \quad
          t \mapsto \gamma(t),
          \]
          so dass
          \[ \gamma(s + t) = \gamma(t) \gamma(s). \]
        \end{block}
      }
    \end{column}
    \begin{column}{0.48\textwidth}
      \uncover<2->{
        \begin{block}{Beispiel}
          Volumen-erhaltende Abbildungen:
          \[ \gSL2R= \{A \in M_2 \,|\, \det(A) = 1\} .\]
          \begin{align*}
            \uncover<4->{ \gamma_x(t) }
            &
            \uncover<4->{= \begin{pmatrix} 1 & t \\ 0 & 1 \end{pmatrix} }
            \\
            \uncover<5->{ \gamma_y(t) }
            &
            \uncover<5->{= \begin{pmatrix} 1 & 0 \\ t & 1 \end{pmatrix} }
            \\
            \uncover<6->{ \gamma_h(t)}
            &
            \uncover<6->{= \begin{pmatrix} e^t & 0 \\ 0 & e^{-t} \end{pmatrix} }
          \end{align*}
        \end{block}
      }
    \end{column}
  \end{columns}
\end{frame}


\begin{frame}[t]
  \setlength{\abovedisplayskip}{5pt}
  \setlength{\belowdisplayskip}{5pt}
  \frametitle{Repetition}
  \vspace{-20pt}
  \begin{columns}[t,onlytextwidth]
    \begin{column}{0.48\textwidth}
      \uncover<1->{
        \begin{block}{Lie-Algebra aus Lie-Gruppe}
          Ableitungen der Ein-Parameter-Untergruppen:
          \begin{align*}
            G       &\to      \mathcal A    \\
            \gamma  &\mapsto  \dot\gamma(0)
          \end{align*}
          \uncover<3->{
            Lie-Klammer als Produkt: 
            \[ [A, B] = AB - BA \in \mathcal A \]
          }
        \end{block}
      }
      \uncover<7->{\vspace*{-4ex}
        \begin{block}{Lie-Gruppe aus Lie-Algebra}
          Lösung der Differentialgleichung:
          \[
          \dot\gamma(t) = A\gamma(t)
          \quad \text{mit} \quad
          A = \dot\gamma(0)
          \]
          \[
          \Rightarrow \gamma(t) = \exp(At)
          \]
        \end{block}
      }
    \end{column}
    \begin{column}{0.48\textwidth}
      \uncover<2->{
        \begin{block}{Beispiel}
          Lie-Algebra von \gSL2R:
          \[ \asl2R = \{ A \in M_2 \,|\, \Spur(A) = 0 \} \]
        \end{block}
        }
        \begin{align*}
          \uncover<4->{ X(t) }
          &
          \uncover<4->{= \begin{pmatrix} 0 & t \\ 0 & 0 \end{pmatrix} }
          \\
          \uncover<5->{ Y(t) }
          &
          \uncover<5->{= \begin{pmatrix} 0 & 0 \\ t & 0 \end{pmatrix} }
          \\
          \uncover<6->{ H(t) }
          &
          \uncover<6->{= \begin{pmatrix} t & 0 \\ 0 & -t \end{pmatrix} }
        \end{align*}

    \end{column}
  \end{columns}
\end{frame}

\begin{frame}[t]
  \setlength{\abovedisplayskip}{5pt}
  \setlength{\belowdisplayskip}{5pt}
  \frametitle{Repetition}
  \vspace{-20pt}
  \begin{block}{Offene Fragen}
    \begin{itemize}[<+->]
      \item Woher kommt die Exponentialfunktion?
      \begin{fleqn} 
        \[
        \exp(At)
        =
        1 
        + At 
        + A^2\frac{t^2}{2}
        + A^3\frac{t^3}{3!} 
        + \ldots
        \]
      \end{fleqn}
      \item Wie löst man eine Matrix-DGL?
      \begin{fleqn} 
        \[ 
        \dot\gamma(t) = A\gamma(t),
        \qquad
        \gamma(t) \in G \subset M_n
        \]
      \end{fleqn}
      \item Was bedeutet $\exp(At)$?
    \end{itemize}
  \end{block}
\end{frame}

\egroup
