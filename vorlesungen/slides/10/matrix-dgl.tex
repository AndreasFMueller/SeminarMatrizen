%
% matrix-dgl.tex -- Matrix-Differentialgleichungen
%
% (c) 2021 Prof Dr Andreas Müller, OST Ostschweizer Fachhochschule
% Erstellt durch Roy Seitz
%
% !TeX spellcheck = de_CH
\bgroup

\begin{frame}[t]
  \setlength{\abovedisplayskip}{5pt}
  \setlength{\belowdisplayskip}{5pt}
  \frametitle{1.~Ordnung mit Skalaren}
  \vspace{-20pt}
  \begin{columns}[t,onlytextwidth]
  \begin{column}{0.48\textwidth}
    \begin{block}{Aufgabe}
      Sei $a, x(t), x_0 \in \mathbb R$,
      \[
      \dot x(t) = ax(t),
      \quad
      x(0) = x_0
      \]
    \end{block}
    \begin{block}{Potenzreihen-Ansatz}
      Sei $a_k \in \mathbb R$,
      \[
      x(t) = a_0 + a_1t + a_2t^2 + a_3t^3 \ldots
      \]
    \end{block}
  \end{column}
  \begin{column}{0.48\textwidth}
    \begin{block}{Lösung}
      Einsetzen in DGL, Koeffizientenvergleich liefert
      \[ x(t) = \exp(at) \, x_0, \]
      wobei
      \begin{align*}
      \exp(at)
      &= 1 + at + \frac{a^2t^2}{2} + \frac{a^3t^3}{3!} + \ldots \\
      &{\color{gray}(= e^{at}.)}
      \end{align*}
    \end{block}
  \end{column}
  \end{columns}
\end{frame}

\begin{frame}[t]
  \setlength{\abovedisplayskip}{5pt}
  \setlength{\belowdisplayskip}{5pt}
  \frametitle{1.~Ordnung mit Matrizen}
  \vspace{-20pt}
  \begin{columns}[t,onlytextwidth]
    \begin{column}{0.48\textwidth}
      \begin{block}{Aufgabe}
        Sei $A \in M_n$, $x(t), x_0 \in \mathbb R^n$,
        \[
        \dot x(t) = Ax(t),
        \quad
        x(0) = x_0
        \]
      \end{block}
      \begin{block}{Potenzreihen-Ansatz}
        Sei $A_k \in \mathbb M_n$,
        \[
        x(t) = A_0 + A_1t + A_2t^2 + A_3t^3 \ldots
        \]
      \end{block}
    \end{column}
    \begin{column}{0.48\textwidth}
      \begin{block}{Lösung}
        Einsetzen in DGL, Koeffizientenvergleich liefert
        \[ x(t) = \exp(At) \, x_0, \]
        wobei
        \[
        \exp(At)
        = 1 + At + \frac{A^2t^2}{2} + \frac{A^3t^3}{3!} + \ldots
        \]
      \end{block}
    \end{column}
  \end{columns}
\end{frame}

\egroup
