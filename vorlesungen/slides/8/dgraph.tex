%
% dgraph.tex
%
% (c) 2021 Prof Dr Andreas Müller, OST Ostschweizer Fachhochschule
%
\bgroup
\definecolor{darkgreen}{rgb}{0,0.6,0}
\begin{frame}
\frametitle{Gerichteter Graph}
\vspace{-20pt}
\begin{columns}[t,onlytextwidth]
\begin{column}{0.44\textwidth}
\begin{center}
\begin{tikzpicture}[>=latex,thick]
\def\r{2.4}

\coordinate (A) at ({\r*cos(0*72)},{\r*sin(0*72)});
\coordinate (B) at ({\r*cos(1*72)},{\r*sin(1*72)});
\coordinate (C) at ({\r*cos(2*72)},{\r*sin(2*72)});
\coordinate (D) at ({\r*cos(3*72)},{\r*sin(3*72)});
\coordinate (E) at ({\r*cos(4*72)},{\r*sin(4*72)});

\uncover<3->{
	\draw[->,shorten >= 0.2cm,shorten <= 0.2cm] (A) -- (C);
	\draw[color=white,line width=5pt] (B) -- (D);
	\draw[->,shorten >= 0.2cm,shorten <= 0.2cm] (B) -- (D);

	\draw[->,shorten >= 0.2cm,shorten <= 0.2cm] (A) -- (B);
	\draw[->,shorten >= 0.2cm,shorten <= 0.2cm] (B) -- (C);
	\draw[->,shorten >= 0.2cm,shorten <= 0.2cm] (C) -- (D);
	\draw[->,shorten >= 0.2cm,shorten <= 0.2cm] (D) -- (E);
	\draw[->,shorten >= 0.2cm,shorten <= 0.2cm] (E) -- (A);
}

\uncover<2->{
	\draw (A) circle[radius=0.2];
	\draw (B) circle[radius=0.2];
	\draw (C) circle[radius=0.2];
	\draw (D) circle[radius=0.2];
	\draw (E) circle[radius=0.2];

	\node at (A) {$1$};
	\node at (B) {$2$};
	\node at (C) {$3$};
	\node at (D) {$4$};
	\node at (E) {$5$};
}
\node at (0,0) {$G$};

\uncover<3->{
	\node at ($0.5*(A)+0.5*(B)-(0.1,0.1)$) [above right] {$\scriptstyle 1$};
	\node at ($0.5*(B)+0.5*(C)+(0.05,-0.07)$) [above left] {$\scriptstyle 2$};
	\node at ($0.5*(C)+0.5*(D)+(0.05,0)$) [left] {$\scriptstyle 3$};
	\node at ($0.5*(D)+0.5*(E)$) [below] {$\scriptstyle 4$};
	\node at ($0.5*(E)+0.5*(A)+(-0.1,0.1)$) [below right] {$\scriptstyle 5$};
	\node at ($0.6*(A)+0.4*(C)$) [above] {$\scriptstyle 6$};
	\node at ($0.4*(B)+0.6*(D)$) [left] {$\scriptstyle 7$};
}

\uncover<7->{
	\draw[->,shorten >= 0.2cm,shorten <= 0.2cm,color=red]
                (E) to[out=-18,in=-126,distance=2cm] (E);
}

\uncover<9->{
	\draw[->,shorten >= 0.2cm,shorten <= 0.2cm,color=darkgreen]
		(D) to[out=120,in=-120] (C);
}

\end{tikzpicture}
\end{center}
\end{column}
\begin{column}{0.52\textwidth}
\begin{block}{Definition}
Ein gerichteter Graph $G=(V,E)$ ist
\begin{enumerate}
\item<2-> Eine Menge $V$ von Knoten (Vertizes)
$V=\{v_1,v_2,\dots\}$
\item<3->
Eine Menge $E$ von gerichteten Kanten
(Edges)
\[
E\subset \{ (v_1,v_2)\;|\; v_i\in V\}
\]
\end{enumerate}
\end{block}
\vspace{-30pt}
\uncover<6->{%
\begin{block}{Achtung}
\begin{itemize}
\item<6-> Kanten sind {\em geordnete} Paare
\uncover<7->{$\Rightarrow$ {\color{red}Schleifen} sind möglich}
\item<8-> Kanten sind immer ``Einbahnstrassen''
\item<9-> {\color{darkgreen}Gegenrichtung explizit angeben}
\end{itemize}
\end{block}}
\end{column}
\end{columns}
\end{frame}
\egroup
