%
% fourier.tex
%
% (c) 2021 Prof Dr Andreas Müller, OST Ostschweizer Fachhochschule
%
\begin{frame}[t]
\setlength{\abovedisplayskip}{5pt}
\setlength{\belowdisplayskip}{5pt}
\frametitle{Fourier-Transformation}
\vspace{-20pt}
\begin{columns}[t,onlytextwidth]
\begin{column}{0.48\textwidth}
\begin{block}{Algebra}
Die Laplace-Matrix eines Graphen ist symmetrisch
\uncover<2->{%

$\Rightarrow$
Es gibt eine Basis aus Eigenvektoren $g_i\in\mathbb{R}^n$ von $L(G)$:
\begin{align*}
L(G)g_i&=\lambda_i g_i
\end{align*}}
\end{block}
\uncover<12->{%
\vspace{-20pt}
\begin{block}{Fourier-Transformation}
Jedes $f\in\mathbb{R}^n$ kann durch die $g_i$ ausgedrückt werden
\begin{align*}
\uncover<13->{
f&= a_1 g_1 + \dots + a_n g_n
}
\\
\uncover<14->{
&= \hat{f}_1 g_1 + \dots + \hat{f}_ng_n = \sum_{k=1}^n \hat{f}_kg_k
}
\end{align*}
\uncover<15->{%
Zerlegung nach Zeitkonstante $\lambda_i$
}
\end{block}}
\end{column}
\begin{column}{0.48\textwidth}
\uncover<3->{%
\begin{block}{Anwendung}
Wärmeleitungsgleichung
\begin{align*}
\uncover<4->{
\frac{d}{dt}f &= L(G) f
}
\intertext{\uncover<5->{{\usebeamercolor[fg]{title}Ansatz:}}}
\uncover<6->{
f&=a_1g_1T_1(t)+\dots + a_ng_nT_n(t)
}
\\
\uncover<7->{
\frac{d}{dt}f
&=
a_1g_1\dot{T}_1(t) + \dots + a_1g_1 \dot{T}_n(t)
}
\\
\uncover<8->{
&=
a_1Lg_1 + \dots + a_nLg_n
}
\\
\uncover<9->{
&=
a_1\lambda_1 g_1 + \dots + a_n\lambda_n g_n
}
\\
\uncover<10->{
\dot{T}_i(t) &= \lambda_i T_i(t)
}
\uncover<11->{
\quad
\Rightarrow
\quad
T_i(t) = e^{\lambda_it} \uncover<-9>{T_i(0)}
}
\end{align*}
\end{block}}
\end{column}
\end{columns}
\end{frame}
