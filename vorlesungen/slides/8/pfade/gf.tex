%
% gf.tex
%
% (c) 2019 Prof Dr Andreas Müller, Hochschule Rapperswil
%
\begin{frame}
\definecolor{darkred}{rgb}{0.8,0,0}
\frametitle{Erzeugende Funktion}
Alle Weglängen zusammen:
\[
\uncover<7->{f({\color{darkred}t})=}
\uncover<4->{E+}
A
\uncover<3->{{\color{darkred}t}}
\uncover<2->{+}
\uncover<5->{\frac{1}{2!}}
A^2
\uncover<3->{{\color{darkred}t^2}}
\uncover<2->{+}
\uncover<5->{\frac{1}{3!}}
A^3
\uncover<3->{{\color{darkred}t^3}}
\uncover<2->{+}
\uncover<5->{\frac{1}{4!}}
A^4
\uncover<3->{{\color{darkred}t^4}}
\uncover<2->{+}
\uncover<5->{\frac{1}{5!}}
A^5
\uncover<3->{{\color{darkred}t^5}}
\uncover<2->{+}
\uncover<5->{\frac{1}{6!}}
A^6
\uncover<3->{{\color{darkred}t^6}}
\uncover<2->{+}
\uncover<5->{\frac{1}{7!}}
A^7
\uncover<3->{{\color{darkred}t^7}}
\dots
\uncover<6->{= e^{A{\color{darkred}t}}}
\]
\uncover<4->{%
heisst {\em\usebeamercolor[fg]{title} \only<5->{exponentiell} erzeugende Funktion}
der Wege-Anzahlen}

\begin{itemize}
\item<8->
Begriff der Entropie auf einem Graphen
\item<9->
Wahrscheinlichkeit, dass ein Zufallsspaziergänger auf einem Graphen an
einem bestimmten Knoten vorbeikommt
\end{itemize}

\end{frame}
