%
% subgraph.tex -- slide template
%
% (c) 2021 Prof Dr Andreas Müller, OST Ostschweizer Fachhochschule
%
\bgroup
\begin{frame}[t]
\setlength{\abovedisplayskip}{5pt}
\setlength{\belowdisplayskip}{5pt}
\frametitle{$\alpha_{\text{max}}$ eines Untergraphen}
\vspace{-20pt}
\begin{columns}[t,onlytextwidth]
\begin{column}{0.48\textwidth}
\begin{block}{Satz}
$X'$ ein echter Untergraph von $X$ mit Adjazenzmatrix $A'$ und grösstem
Eigenwert $\alpha_{\text{max}}'$
\[
\alpha_{\text{max}}' \le \alpha_{\text{max}}
\]
\end{block}
\uncover<2->{$V'$ die Knoten von $X'$}
\end{column}
\begin{column}{0.48\textwidth}
\uncover<3->{%
\begin{proof}[Beweis]
\begin{itemize}
\item<4->
$f'$ der positive Eigenvektor von $A'$
\item<5->
Definiere 
\[
g(v)
=
\begin{cases}
f'(v) &\qquad v\in V'\\
0     &\qquad \text{sonst}
\end{cases}
\]
\item<6-> Skalarprodukte:
\begin{align*}
\uncover<7->{\langle f',f'\rangle &= \langle g,g\rangle}
\\
\uncover<8->{\langle A'f',f'\rangle &\le \langle Ag,g\rangle}
\end{align*}
\item<9-> Vergleich
\[
\alpha_{\text{max}}'
=
\frac{\langle A'f',f'\rangle}{\langle f',f'\rangle}
\uncover<10->{\le 
\frac{\langle Ag,g\rangle}{\langle g,g\rangle}}
\uncover<11->{\le
\alpha_{\text{max}}}
\]
\end{itemize}
\end{proof}}
\end{column}
\end{columns}
\end{frame}
\egroup
