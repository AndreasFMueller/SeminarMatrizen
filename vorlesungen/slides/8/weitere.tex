%
% weitere.tex -- slide template
%
% (c) 2021 Prof Dr Andreas Müller, OST Ostschweizer Fachhochschule
%
\bgroup
\begin{frame}[t]
\setlength{\abovedisplayskip}{5pt}
\setlength{\belowdisplayskip}{5pt}
\frametitle{Weitere Resultate der spektralen Graphentheorie}
\vspace{-20pt}
\begin{columns}[t,onlytextwidth]
\begin{column}{0.48\textwidth}
\begin{block}{Satz (Hoffmann)}
\[
\operatorname{chr} X \ge 1 + \frac{\alpha_{\text{max}}}{-\alpha_{\text{min}}}
\]
\end{block}
\uncover<2->{%
\begin{block}{Satz (Hoffmann)}
\[
\operatorname{ind} X \le n \biggl(1-\frac{d_{\text{min}}}{\lambda_{\text{max}}}\biggr)
\]
\end{block}}
\end{column}
\begin{column}{0.48\textwidth}
\uncover<3->{%
\begin{block}{Korollar}
Für einen regulären Graphen mit $n$ Knoten gilt
\begin{align*}
\operatorname{ind} X
&\le
\frac{n}{\displaystyle 1-\frac{d}{\alpha_{\text{min}}}}
\\
\operatorname{chr} X
&\ge
1-\frac{d}{\alpha_{\text{min}}}
\end{align*}
\end{block}}
\end{column}
\end{columns}
\end{frame}
\egroup
