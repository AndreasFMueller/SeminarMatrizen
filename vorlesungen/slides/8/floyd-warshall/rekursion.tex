%
% rekursion.tex
%
% (c) 2019 Prof Dr Andreas Müller, Hochschule Rapperswil
%
\begin{frame}[fragile]
\frametitle{Rekursion}
\vspace{-20pt}
\begin{center}
\begin{tikzpicture}[>=latex]

\def\blob#1#2#3{
        \fill[color=#3] #1 circle[radius=0.3];
        \draw[line width=0.7pt] #1 circle[radius=0.3];
        \node at #1 {{#2}};
}

\def\kante#1#2{
        \draw[line width=0.7pt,shorten >= 0.3,shorten >= 0.3] #1 -- #2 ;
}

\coordinate (A) at (0,0);
\coordinate (B1) at (2,2);
\coordinate (B2) at (2,1);
\coordinate (B3) at (2,0);
\coordinate (B4) at (2,-1);
\coordinate (B5) at (2,-2);

\draw[line width=1.9pt,color=gray] (A)--(B1);
\draw[line width=1.9pt,color=gray] (A)--(B2);
\draw[line width=1.9pt,color=gray] (A)--(B3);
\draw[line width=1.9pt,color=gray] (A)--(B4);
\draw[line width=1.9pt,color=gray] (A)--(B5);

\coordinate (Z) at (10,0);

\begin{scope}
\clip (2,-2.3) rectangle (10,2.3);
\foreach \y in{-10,...,10}{
	\draw[line width=1.9pt,color=gray]
		(2,\y)--(10,{\y-8});
	\draw[line width=1.9pt,color=gray]
		(2,\y)--(10,{\y+8});
}
\end{scope}

\uncover<2>{
\draw[line width=4pt,color=red] (A)--(B1)--(5,-1)--(8,2)--(Z);
}

\uncover<3>{
\draw[line width=4pt,color=red] (A)--(B2)--(3,0)--(4,1)--(5,0)--(6,1)--(8.5,-1.5)--(Z);
}

\uncover<4>{
\draw[line width=4pt,color=red] (A)--(B3)--(2.5,0.5)--(3.5,-0.5)--(5,1.0)--(7,-1)--(9,1)--(Z);
}

\uncover<5>{
\draw[line width=4pt,color=red] (A)--(B4)--(3,0)--(4,1)--(5,0)--(6,1)--(7,0)
	--(6.0,-1.0)--(7,-2)--(7.5,-1.5)--(7,-1)--(7.5,-0.5)
	--(8.5,-1.5)--(Z);
}

\uncover<6->{
	\draw[line width=4pt,color=red] (A)--(B5)--(6,2);
}
\uncover<7->{
	\draw[line width=4pt,color=red] (6,2)--(7,1)--(5,-1);
}
\uncover<8->{
	\draw[line width=4pt,color=red] (5,-1)--(6,-2)--(8,0)--(9,-1);
}
\uncover<9->{
	\draw[line width=4pt,color=red] (9,-1)--(Z);
}

\blob{(A)}{$A$}{red!20}
\blob{(B1)}{$B_1$}{white}
\blob{(B2)}{$B_2$}{white}
\blob{(B3)}{$B_3$}{white}
\blob{(B4)}{$B_4$}{white}
\blob{(B5)}{$B_5$}{white}

\blob{(Z)}{$Z$}{red!20}

\uncover<6->{
	\node at (6,2) {\includegraphics[width=1.5cm]{../slides/8/floyd-warshall/macdonalds.png}};
}

\uncover<7->{
	\node at (5,-1) {\includegraphics[width=1.5cm]{../slides/8/floyd-warshall/starbucks.png}};
}

\uncover<8->{
	\node at (9,-1) {\includegraphics[width=2cm]{../slides/8/floyd-warshall/burgerking.png}};
}

\end{tikzpicture}
\end{center}

\begin{block}{Abstieg}
Für den kürzesten Weg von $A$ nach $Z$ suche denjenigen Nachbarn $B_i$
von $A$, der den kürzesten Weg von $B_i$ nach $Z$ hat.
\uncover<7->{$\Rightarrow$ wir brauchen {\color{red}alle} kürzesten Wege!}
\end{block}

\end{frame}
