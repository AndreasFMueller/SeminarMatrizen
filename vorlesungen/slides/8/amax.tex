%
% amax.tex -- slide template
%
% (c) 2021 Prof Dr Andreas Müller, OST Ostschweizer Fachhochschule
%
\bgroup
\begin{frame}[t]
\setlength{\abovedisplayskip}{5pt}
\setlength{\belowdisplayskip}{5pt}
\frametitle{$\alpha_{\text{max}}$ und $d$}
\vspace{-20pt}
\begin{columns}[t,onlytextwidth]
\begin{column}{0.44\textwidth}
\begin{block}{Definition}
$\alpha_{\text{max}}$ ist der grösste Eigenwert der Adjazenzmatrix
\end{block}
\uncover<2->{
\begin{block}{Fakten}
\begin{itemize}
\item<3->
Der Eigenwert $\alpha_{\text{max}}$ ist einfach
\item<4->
Es gibt einen positiven Eigenvektor $f$ zum Eigenwert $\alpha_{\text{max}}$
\item<5->
$f$ maximiert
\[
\frac{\langle Af,f\rangle}{\langle f,f\rangle}
=
\alpha_{\text{max}}
\]
\end{itemize}
Herkunft: Perron-Frobenius-Theorie positiver Matrizen (nächste Woche)
\end{block}}
\end{column}
\begin{column}{0.52\textwidth}
\uncover<6->{%
\begin{block}{Mittlerer Grad}
\[
\overline{d}
=
\frac1{n} \sum_{v} \operatorname{deg}(v)
\le
\alpha_{\text{max}}
\le
d
\]
\end{block}}
\vspace{-10pt}
\uncover<7->{%
\begin{proof}[Beweis]
\begin{itemize}
\item Konstante Funktion $1$ anstelle von $f$:
\[
\frac{\langle A1,1\rangle}{\langle 1,1\rangle}
\uncover<8->{=
\frac{\sum_v \operatorname{deg}(v)}{n}}
\uncover<9->{=
\overline{d}}
\uncover<10->{\le
\alpha_{\text{max}}}
\]
\item<11-> Komponenten von $Af$ summieren:
\begin{align*}
\uncover<12->{
\alpha_{\text{max}}
f(v) &= (Af)(v)}\uncover<13->{ = \sum_{u\sim v} f(u)}
\\
\uncover<14->{\alpha_{\text{max}}
\sum_{v}f(v)
&=
\sum_v
\operatorname{deg}(v) f(v)}
\\
&\uncover<15->{\le
d\sum_v f(v)}
\;
\uncover<16->{\Rightarrow
\;
\alpha_{\text{max}} \le d}
\end{align*}
\end{itemize}
\end{proof}}
\end{column}
\end{columns}
\end{frame}
\egroup
