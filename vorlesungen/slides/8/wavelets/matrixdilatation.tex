%
% matrixdilatation.tex -- slide template
%
% (c) 2021 Prof Dr Andreas Müller, OST Ostschweizer Fachhochschule
%
\bgroup
\begin{frame}[t]
\setlength{\abovedisplayskip}{5pt}
\setlength{\belowdisplayskip}{5pt}
\frametitle{Dilatation in Matrixform}
Dilatationsfaktor $a$, skaliertes Wavelet beim Knoten $k$ mit Spektrum
$\tilde{D}_{1/a}g$
\begin{align*}
D_{g,a}e_k
&=
\chi
\begin{pmatrix}
g(a\lambda_1)&      0      & \dots  &      0      \\
      0      &g(a\lambda_2)& \dots  &      0      \\
   \vdots    &    \vdots   & \ddots &    \vdots   \\
      0      &      0      & \dots  &g(a\lambda_n)
\end{pmatrix}
\chi^t
e_k
\intertext{\uncover<2->{``verschmierter'' Standardbasisvektor am Knoten $k$}}
\uncover<2->{D_he_k
&=
\chi
\begin{pmatrix}
h(\lambda_1)&     0      & \dots  &     0      \\
     0      &h(\lambda_2)& \dots  &     0      \\
  \vdots    &   \vdots   & \ddots &   \vdots   \\
     0      &     0      & \dots  &h(\lambda_n)
\end{pmatrix}
\chi^t
e_k}
\end{align*}
\end{frame}
\egroup
