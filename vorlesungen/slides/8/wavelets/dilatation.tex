%
% template.tex -- slide template
%
% (c) 2021 Prof Dr Andreas Müller, OST Ostschweizer Fachhochschule
%
\bgroup
\begin{frame}[t]
\setlength{\abovedisplayskip}{5pt}
\setlength{\belowdisplayskip}{5pt}
\frametitle{Dilatation}
\vspace{-20pt}
\begin{columns}[t,onlytextwidth]
\begin{column}{0.48\textwidth}
\begin{block}{Dilatation in $\mathbb{R}$}
$f\colon \mathbb{R}\to\mathbb{R}$
Definition im Ortsraum:
\[
(D_af)(x)
=
\frac{1}{\sqrt{|a|}}
f\biggl(\frac{x}{a}\biggr)
\]
\uncover<2->{%
Dilatation im Frequenzraum:
\[
\widehat{D_af}(\omega)
=
D_{1/a}\hat{f}(\omega)
\]}
\uncover<3->{%
Spektrum wird mit $1/a$ skaliert!}
\end{block}
\end{column}
\begin{column}{0.48\textwidth}
\uncover<4->{%
\begin{block}{``Dilatation'' auf einem Graphen}
\begin{itemize}
\item<5-> Dilatation auf dem Graphen gibt es nicht
\item<6-> Dilatation im Spektrum $\{\lambda_1,\dots,\lambda_n\}$ gibt es nicht
\item<7-> ``Spektrale Dilatation'' verwenden
\begin{enumerate}
\item<8-> Start: $e_k$
\item<9-> Fourier-Transformation: $\chi^te_k$
\item<10-> Spektrum skalieren: mit
$D_{1/a}g$ filtern
\item<11-> Rücktransformation
\[
D_{g,a}e_k
=
\chi
\uncover<12->{\operatorname{diag}(\tilde{D}_{1/a}g(\lambda_*))
\chi^t e_k}
\]
\end{enumerate}
\end{itemize}


\end{block}}
\end{column}
\end{columns}
\end{frame}
\egroup
