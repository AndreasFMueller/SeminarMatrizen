%
% beispiel.tex -- slide template
%
% (c) 2021 Prof Dr Andreas Müller, OST Ostschweizer Fachhochschule
%
\bgroup
\def\bild#1#2{
\node at (0,0) [rotate=-90]
{\includegraphics[width=#1\textwidth]{../../../SeminarWavelets/buch/papers/sgwt/images/#2}};
}
\begin{frame}[t]
\setlength{\abovedisplayskip}{5pt}
\setlength{\belowdisplayskip}{5pt}
\frametitle{Wavelets einer Strecke}
\vspace{-10pt}
\begin{center}
\begin{tikzpicture}[>=latex,thick]

\only<1>{ \bild{0.6}{wavelets-psi-line-5-10.pdf} }
\only<2>{ \bild{0.6}{wavelets-psi-line-4-10.pdf} }
\only<3>{ \bild{0.6}{wavelets-psi-line-3-10.pdf} }
\only<4>{ \bild{0.6}{wavelets-psi-line-2-10.pdf} }
\only<5>{ \bild{0.6}{wavelets-psi-line-1-10.pdf} }

\only<6>{ \bild{0.6}{wavelets-phi-line-10.pdf} }

\only<1>{ \node at (-7.6,2.8) [right] {Bandpass mit $g_1$}; }
\only<2>{ \node at (-7.6,2.8) [right] {Bandpass mit $g_2$}; }
\only<3>{ \node at (-7.6,2.8) [right] {Bandpass mit $g_3$}; }
\only<4>{ \node at (-7.6,2.8) [right] {Bandpass mit $g_4$}; }
\only<5>{ \node at (-7.6,2.8) [right] {Bandpass mit $g_5$}; }
\only<6>{ \node at (-7.6,2.8) [right] {Tiefpass mit $h$}; }


\only<1>{ \node at (-7.6,2) [right] {$D_{g,1/a_1}\chi_*$}; }
\only<2>{ \node at (-7.6,2) [right] {$D_{g,1/a_2}\chi_*$}; }
\only<3>{ \node at (-7.6,2) [right] {$D_{g,1/a_3}\chi_*$}; }
\only<4>{ \node at (-7.6,2) [right] {$D_{g,1/a_4}\chi_*$}; }
\only<5>{ \node at (-7.6,2) [right] {$D_{g,1/a_5}\chi_*$}; }

\only<6>{ \node at (-7.6,2) [right] {$D_{h}\chi_*$}; }

\end{tikzpicture}
\end{center}
\end{frame}
\egroup
