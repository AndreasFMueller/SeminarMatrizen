%
% template.tex -- slide template
%
% (c) 2021 Prof Dr Andreas Müller, OST Ostschweizer Fachhochschule
%
\bgroup
\begin{frame}[t]
\setlength{\abovedisplayskip}{5pt}
\setlength{\belowdisplayskip}{5pt}
\frametitle{Graph Wavelet Frame}
\vspace{-20pt}
\begin{columns}[t,onlytextwidth]
\begin{column}{0.48\textwidth}
\begin{block}{Frame-Vektoren}
Zu Dilatationsfaktoren $A=\{a_i\,|\,i=1,\dots,N\}$
konstruiere das Frame
\begin{align*}
F=
\{&D_he_1,\dots,D_he_n,\\
  &Dg_1e_1,\dots,Dg_1e_n,\\
  &Dg_2e_1,\dots,Dg_2e_n,\\
  &\dots\\
  &Dg_Ne_1,\dots,Dg_Ne_n\}
\end{align*}
\uncover<2->{Notation:
\begin{align*}
v_{0,k}
&=
D_he_k
\\
v_{i,k}
&=
Dg_ie_k
\end{align*}}
\end{block}
\end{column}
\begin{column}{0.48\textwidth}
\uncover<3->{%
\begin{block}{Frameoperator}
\begin{align*}
\mathcal{T}\colon \mathbb{R}^n\to\mathbb{R}^{nN}
:
v
&\mapsto
\begin{pmatrix}
\uncover<4->{\langle D_he_1,v\rangle}\\
\uncover<4->{\vdots}\\
\uncover<4->{\langle D_he_n,v\rangle}\\
\hline
\uncover<5->{\langle D_{g_1}e_1,v\rangle}\\
\uncover<5->{\vdots}\\
\uncover<5->{\langle D_{g_1}e_n,v\rangle}\\
\hline
\uncover<6->{\vdots}\\
\uncover<6->{\vdots}\\
\hline
\uncover<7->{\langle D_{g_N}e_1,v\rangle}\\
\uncover<7->{\vdots}\\
\uncover<7->{\langle D_{g_N}e_n,v\rangle}
\end{pmatrix}
\end{align*}
\end{block}}
\end{column}
\end{columns}
\end{frame}
\egroup
