%
% template.tex -- slide template
%
% (c) 2021 Prof Dr Andreas Müller, OST Ostschweizer Fachhochschule
%
\bgroup
\begin{frame}[t]
%\setlength{\abovedisplayskip}{5pt}
%\setlength{\belowdisplayskip}{5pt}
\frametitle{Framekonstanten}
\vspace{-20pt}
\begin{columns}[t,onlytextwidth]
\begin{column}{0.48\textwidth}
\begin{block}{Definition}
Eine Menge $\mathcal{F}$ von Vektoren heisst ein Frame,
falls es Konstanten $A$ und $B$ gibt derart, dass
\[
A\|v\|^2
\le
\|\mathcal{T}v\|^2
\sum_{b\in\mathcal{F}} |\langle b,v\rangle|^2
\le
B\|v\|^2
\]
\uncover<2->{$A>0$ garantiert Invertierbarkeit}
\end{block}
\uncover<3->{%
\begin{block}{$\|\mathcal{T}v\|$ für Graph-Wavelets}
\begin{align*}
\|\mathcal{T}v\|^2
&=
\sum_k |\langle D_he_k,v\rangle|^2
+
\sum_{i,k} |\langle D_{g_i}e_k, v\rangle|^2
\\
&\uncover<4->{=
\sum_k |h(\lambda_k) \hat{v}(k)|^2
+
\sum_{k,i} |g_i(\lambda_k) \hat{v}(k)|^2}
\end{align*}
\end{block}}
\end{column}
\begin{column}{0.48\textwidth}
\uncover<5->{%
\begin{block}{$A$ und $B$}
Frame-Norm-Funktion
\begin{align*}
f(\lambda)
&=
h(\lambda)
+
\sum_i g_i(\lambda)
\\
&\uncover<6->{=
h(\lambda)
+
\sum_i g(a_i\lambda)}
\end{align*}
\uncover<7->{Abschätzung für Frame-Konstanten
\begin{align*}
A&\uncover<8->{=
\min_{i} f(\lambda_i)}
\\
B&\uncover<9->{=
\max_{i} f(\lambda_i)}
\end{align*}}
\end{block}}
\end{column}
\end{columns}
\end{frame}
\egroup
