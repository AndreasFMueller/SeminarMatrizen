%
% lokalisierungsvergleich.tex -- slide template
%
% (c) 2021 Prof Dr Andreas Müller, OST Ostschweizer Fachhochschule
%
\bgroup
\begin{frame}[t]
\setlength{\abovedisplayskip}{5pt}
\setlength{\belowdisplayskip}{5pt}
\frametitle{Lokalisierung}
\vspace{-20pt}
\begin{columns}[t,onlytextwidth]
\begin{column}{0.48\textwidth}
\begin{block}{Ortsraum}
Ortsraum$\mathstrut=V$
\begin{itemize}
\item<3-> Standardbasis
\item<5-> lokalisiert in den Knoten
\item<7-> die meisten $\hat{f}(k)$ gross
\item<9-> vollständig delokalisiert im Frequenzraum
\end{itemize}
\end{block}
\end{column}
\begin{column}{0.48\textwidth}
\begin{block}{Frequenzraum}
\uncover<2->{Frequenzraum $\mathstrut=\{\lambda_1,\lambda_2,\dots,\lambda_n\}$}
\begin{itemize}
\item<4-> Laplace-Basis
\item<6-> lokalisiert in den Eigenwerten
\item<8-> die meisten Komponenten gross
\item<10-> vollständig delokalisiert im Ortsraum
\end{itemize}
\end{block}
\end{column}
\end{columns}
\uncover<11->{%
\begin{block}{Plan}
Gesucht sind Funktionen auf dem Graphen derart, die
\begin{enumerate}
\item<12-> in der Nähe einzelner Knoten konzentriert/lokalisiert sind und
\item<13-> deren Fourier-Transformation in der Nähe einzelner Eigenwerte 
konzentriert/lokalisiert ist
\end{enumerate}
\end{block}}
\end{frame}
\egroup
