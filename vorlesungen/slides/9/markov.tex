%
% markov.tex
%
% (c) 2021 Prof Dr Andreas Müller, OST Ostschweizer Fachhochschule
%
\bgroup
\setlength{\abovedisplayskip}{5pt}
\setlength{\belowdisplayskip}{5pt}
\begin{frame}[t]
\frametitle{Markovketten}
\vspace{-20pt}
\begin{columns}[t,onlytextwidth]
\begin{column}{0.48\textwidth}
\begin{center}
\begin{tikzpicture}[>=latex,thick]

\def\r{2.2}

\coordinate (A) at ({\r*cos(0*72)},{\r*sin(0*72)});
\coordinate (B) at ({\r*cos(1*72)},{\r*sin(1*72)});
\coordinate (C) at ({\r*cos(2*72)},{\r*sin(2*72)});
\coordinate (D) at ({\r*cos(3*72)},{\r*sin(3*72)});
\coordinate (E) at ({\r*cos(4*72)},{\r*sin(4*72)});

\draw[->,shorten >= 0.1cm,shorten <= 0.1cm,line width=4pt,color=black!40]
	(A) -- (C);
\draw[color=white,line width=8pt] (B) -- (D);
\draw[->,shorten >= 0.1cm,shorten <= 0.1cm,line width=4pt,color=black!80]
	(B) -- (D);

\draw[->,shorten >= 0.1cm,shorten <= 0.1cm,line width=4pt,color=black!60]
	(A) -- (B);
\draw[->,shorten >= 0.1cm,shorten <= 0.1cm,line width=4pt,color=black!20]
	(B) -- (C);
\draw[->,shorten >= 0.1cm,shorten <= 0.1cm,line width=4pt,color=black]
	(C) -- (D);
\draw[->,shorten >= 0.1cm,shorten <= 0.1cm,line width=4pt,color=black]
	(D) -- (E);
\draw[->,shorten >= 0.1cm,shorten <= 0.1cm,line width=4pt,color=black]
	(E) -- (A);

\fill[color=white] (A) circle[radius=0.2];
\fill[color=white] (B) circle[radius=0.2];
\fill[color=white] (C) circle[radius=0.2];
\fill[color=white] (D) circle[radius=0.2];
\fill[color=white] (E) circle[radius=0.2];

\draw (A) circle[radius=0.2];
\draw (B) circle[radius=0.2];
\draw (C) circle[radius=0.2];
\draw (D) circle[radius=0.2];
\draw (E) circle[radius=0.2];

\node at (A) {$1$};
\node at (B) {$2$};
\node at (C) {$3$};
\node at (D) {$4$};
\node at (E) {$5$};

\node at ($0.5*(A)+0.5*(B)-(0.1,0.1)$) [above right] {$\scriptstyle 0.6$};
\node at ($0.5*(B)+0.5*(C)+(0.05,-0.07)$) [above left] {$\scriptstyle 0.2$};
\node at ($0.5*(C)+0.5*(D)+(0.05,0)$) [left] {$\scriptstyle 1$};
\node at ($0.5*(D)+0.5*(E)$) [below] {$\scriptstyle 1$};
\node at ($0.5*(E)+0.5*(A)+(-0.1,0.1)$) [below right] {$\scriptstyle 1$};
\node at ($0.6*(A)+0.4*(C)$) [above] {$\scriptstyle 0.4$};
\node at ($0.4*(B)+0.6*(D)$) [left] {$\scriptstyle 0.8$};

\end{tikzpicture}
\end{center}
\vspace{-10pt}
\uncover<7->{%
\begin{block}{Verteilung}
\begin{itemize}
\item<8->
Welche stationäre Verteilung auf den Knoten stellt sich ein?
\item<9->
$P(i)=?$
\end{itemize}
\end{block}}
\end{column}
\begin{column}{0.48\textwidth}
\uncover<2->{%
\begin{block}{\strut\mbox{Übergang\only<3->{s-/Wahrscheinlichkeit}smatrix}}
$P_{ij} = P(i | j)$, Wahrscheinlichkeit, in den Zustand $i$ überzugehen,
\begin{align*}
P
&=
\begin{pmatrix}
   &   & & &1\phantom{.0}\\
0.6&   & & & \\
0.4&0.2& & & \\
   &0.8&1\phantom{.0}& & \\
   &   & &1\phantom{.0}& 
\end{pmatrix}
\end{align*}
\end{block}}
\vspace{-10pt}
\uncover<4->{%
\begin{block}{Eigenschaften}
\begin{itemize}
\item<5-> $P_{ij}\ge 0\;\forall i,j$
\item<6-> Spaltensumme:
\(
\displaystyle
\sum_{i=1}^n P_{ij} = 1\;\forall j
\)
\end{itemize}
\end{block}}
\end{column}
\end{columns}
\end{frame}
