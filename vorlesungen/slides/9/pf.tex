%
% pf.tex
%
% (c) 2021 Prof Dr Andreas Müller, OST Ostschweizer Fachhochschule
%
\begin{frame}[t]
\frametitle{Perron-Frobenius-Theorie}
\vspace{-20pt}
\begin{columns}[t,onlytextwidth]
\begin{column}{0.48\textwidth}
\begin{block}{Positive Matrizen und Vektoren}
$P\in M_{m\times n}(\mathbb{R})$ 
\begin{itemize}
\item<2->
$P$ heisst positiv, $P>0$, wenn $p_{ij}>0\;\forall i,j$
\item<3->
$P\ge 0$, wenn $p_{ij}\ge 0\;\forall i,j$
\end{itemize}
\end{block}
\uncover<4->{%
\begin{block}{Beispiele}
\begin{itemize}
\item<5->
Adjazenzmatrix $A(G)$
\item<6->
Gradmatrix $D(G)$
\item<7->
Wahrscheinlichkeitsmatrizen
\end{itemize}
\end{block}}
\end{column}
\begin{column}{0.48\textwidth}
\uncover<8->{%
\begin{block}{Satz}
Es gibt einen positiven Eigenvektor $p$ von $P$ zum Eigenwert $1$
\end{block}}
\uncover<9->{%
\begin{block}{Satz}
$P$ irreduzible Matrix, $P\ge 0$, hat einen Eigenvektor $p$, $p\ge 0$,
zum Eigenwert $1$
\end{block}}
\uncover<10->{%
\begin{block}{Potenzmethode}
Falls $P\ge 0$ einen eindeutigen Eigenvektor $p$ hat\uncover<11->{,
dann konveriert die rekursiv definierte Folge
\[
p_{n+1}=\frac{Pp_n}{\|Pp_n\|}, p_0 \ge 0, p_0\ne 0
\]}%
\uncover<12->{$\displaystyle\lim_{n\to\infty} p_n = p$}
\end{block}}
\end{column}
\end{columns}
\end{frame}
