%
% deformation.tex -- slide template
%
% (c) 2021 Prof Dr Andreas Müller, OST Ostschweizer Fachhochschule
%
\bgroup
\begin{frame}[t]
\setlength{\abovedisplayskip}{5pt}
\setlength{\belowdisplayskip}{5pt}
\frametitle{Deformation}
\vspace{-20pt}
\begin{columns}[t,onlytextwidth]
\begin{column}{0.48\textwidth}
\begin{block}{Verlustspiele}
Durch Deformation (Parameter $e$ und $\varepsilon$) kann man
aus $A_e$ und $B_\varepsilon$ Spiele mit negativer Gewinnerwartung machen
\uncover<2->{%
\begin{align*}
E(X)&=0&&\rightarrow&E(X_e)&<0\\
E(Y)&=0&&\rightarrow&E(Y_\varepsilon)&<0\\
\end{align*}}
\end{block}
\end{column}
\begin{column}{0.48\textwidth}
\begin{block}{Kombiniertes Spiel}
\uncover<3->{%
Die Deformation für das Spiel $C$ startet mit Erwartungswert $\frac{18}{709}$}%
\begin{align*}
\uncover<4->{E(Z)&=\frac{18}{709}>0}
&&\uncover<5->{\rightarrow&
E(Z_*)&>0}
\end{align*}
\uncover<6->{Wegen Stetigkeit!}
\\
\uncover<5->{Die Deformation ist immer noch ein Gewinnspiel (für Parameter klein genug)}
\end{block}
\uncover<7->{%
\begin{block}{Parrondo-Paradoxon}
Zufällig zwischen zwei Verlustspielen auswählen kann trotzdem ein
Gewinnspiel ergeben
\end{block}}
\end{column}
\end{columns}
\end{frame}
\egroup
