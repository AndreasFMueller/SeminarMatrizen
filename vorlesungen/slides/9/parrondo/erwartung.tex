%
% erwartung.tex -- slide template
%
% (c) 2021 Prof Dr Andreas Müller, OST Ostschweizer Fachhochschule
%
\bgroup
\begin{frame}[t]
\setlength{\abovedisplayskip}{5pt}
\setlength{\belowdisplayskip}{5pt}
\frametitle{Erwartung}
\vspace{-20pt}
\begin{columns}[t,onlytextwidth]
\begin{column}{0.48\textwidth}
\begin{block}{Zufallsvariable}
\begin{center}
\[
\begin{array}{c|c}
\text{Werte $X$}&\text{Wahrscheinlichkeit $p$}\\
\hline
x_1&p_1=P(X=x_1)\\
x_2&p_2=P(X=x_2)\\
\vdots&\vdots\\
x_n&p_n=P(X=x_n)
\end{array}
\]
\end{center}
\end{block}
\uncover<4->{%
\begin{block}{Einervektoren/-matrizen}
\[
U=\begin{pmatrix}
1&1&\dots&1\\
1&1&\dots&1\\
\vdots&\vdots&\ddots&\vdots\\
1&1&\dots&1
\end{pmatrix}
\in
M_{n\times m}(\Bbbk)
\]
\end{block}}
\end{column}
\begin{column}{0.48\textwidth}
\uncover<2->{%
\begin{block}{Erwartungswerte}
\begin{align*}
E(X)
&=
\sum_i x_ip_i
=
x^tp
\uncover<5->{=
U^t x\odot p}
\hspace*{3cm}
\\
\uncover<2->{E(X^2)
&=
\sum_i x_i^2p_i}
\ifthenelse{\boolean{presentation}}{
\only<6>{=
(x\odot x)^tp}}{}
\uncover<7->{=
U^t (x\odot x) \odot p}
\\
\uncover<3->{E(X^k)
&=
\sum_i x_i^kp_i}
\uncover<8->{=
U^t x^{\odot k}\odot p}
\end{align*}
\uncover<9->{%
Substitution:
\begin{align*}
\uncover<10->{\sum_i &\to U^t}\\
\uncover<11->{x_i^k &\to x^{\odot k}}
\end{align*}}%
\uncover<12->{Kann für Übergangsmatrizen von Markov-Ketten verallgemeinert werden}
\end{block}}
\end{column}
\end{columns}
\end{frame}
\egroup
