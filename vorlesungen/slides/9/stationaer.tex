%
% stationaer.tex
%
% (c) 2021 Prof Dr Andreas Müller, OST Ostschweizer Fachhochschule
%
\begin{frame}[t]
\frametitle{Stationäre Verteilung}
%\vspace{-15pt}
\begin{columns}[t,onlytextwidth]
\begin{column}{0.48\textwidth}
\begin{block}{Zeitentwicklung}
\begin{itemize}
\item<2->
$P$ eine Wahrscheinlichkeitsmatrix
\item<3->
$p_0\in\mathbb{R}^n$ Verteilung zur Zeit $t=0$ bekannt
\item<4->
$p_k\in\mathbb{R}^n$ Verteilung zur Zeit $t=k$
\end{itemize}
\uncover<5->{%
Entwicklungsgesetz
\begin{align*}
P(i,t=k)
&=
\sum_{j=1}^n P_{ij} P(j,t=k-1)
\\
\uncover<6->{
p_k &= Pp_{k-1}
}
\end{align*}}
\end{block}
\end{column}
\begin{column}{0.48\textwidth}
\uncover<7->{%
\begin{block}{Stationär}
Bedingung: $p_{k\mathstrut} = p_{k-1}$
\uncover<8->{
\begin{align*}
\Rightarrow
Pp &= p
\end{align*}}\uncover<9->{%
Eigenvektor zum Eigenwert $1$}
\end{block}}
\uncover<10->{%
\begin{block}{Fragen}
\begin{enumerate}
\item<11->
Gibt es eine stationäre Verteilung?
\item<12->
Gibt es einen Eigenvektor mit Einträgen $\ge 0$?
\item<13->
Gibt es mehr als eine Verteilung?
\end{enumerate}
\end{block}}
\end{column}
\end{columns}
\end{frame}
