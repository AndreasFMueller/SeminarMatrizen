%
% positiv.tex -- slide template
%
% (c) 2021 Prof Dr Andreas Müller, OST Ostschweizer Fachhochschule
%
\bgroup
\begin{frame}[t]
\setlength{\abovedisplayskip}{5pt}
\setlength{\belowdisplayskip}{5pt}
\frametitle{Positive und nichtnegative Matrizen}
\vspace{-20pt}
\begin{columns}[t,onlytextwidth]
\begin{column}{0.48\textwidth}
\begin{block}{Positive Matrix\strut}
Eine Matrix $A$ heisst positiv, wenn
\[
a_{ij} > 0\quad\forall i,j
\]
Man schreibt $A>0\mathstrut$
\end{block}
\begin{block}{Relation $>\mathstrut$}
Man schreibt $A>B$ wenn $A-B > 0\mathstrut$
\end{block}
\begin{block}{Wahrscheinlichkeitsmatrix}
\[
W=\begin{pmatrix}
0.7&0.2&0.1\\
0.2&0.6&0.1\\
0.1&0.2&0.8
\end{pmatrix}
\]
Spaltensumme$\mathstrut=1$, Zeilensumme$\mathstrut=?$
\end{block}
\end{column}
\begin{column}{0.48\textwidth}
\begin{block}{Nichtnegative Matrix\strut}
Eine Matrix $A$ heisst nichtnegativ, wenn
\[
a_{ij} \ge 0\quad\forall i,j
\]
Man schreibt $A\ge 0\mathstrut$
\end{block}
\begin{block}{Relation $\ge\mathstrut$}
Man schreibt $A\ge B$ wenn $A-B \ge 0\mathstrut$
\end{block}
\begin{block}{Permutationsmatrix}
\[
P=\begin{pmatrix}
0&0&1\\
1&0&0\\
0&1&0
\end{pmatrix}
\]
Genau eine $1$ in jeder Zeile/Spalte
\end{block}
\end{column}
\end{columns}
\end{frame}
\egroup
