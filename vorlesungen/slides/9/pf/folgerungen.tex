%
% template.tex -- slide template
%
% (c) 2021 Prof Dr Andreas Müller, OST Ostschweizer Fachhochschule
%
\bgroup
\begin{frame}[t]
\setlength{\abovedisplayskip}{5pt}
\setlength{\belowdisplayskip}{5pt}
\frametitle{Folgerungen für $A>0$}
\vspace{-20pt}
\begin{columns}[t,onlytextwidth]
\begin{column}{0.48\textwidth}
\begin{block}{Satz}
$u\ge 0$ ein EV zum EW $ \lambda\ne 0$,
dann ist $u>0$ und $\lambda >0$
\end{block}
\uncover<6->{%
\begin{block}{Satz}
$v$ ein EV zum EW $\lambda$ mit $|\lambda| = \varrho(A)$,
dann ist $u=|v|$ mit $u_i=|v_i|$ ein EV mit EW $\varrho(A)$
\end{block}}
\uncover<29->{%
\begin{block}{Satz}
$v$ ein EV zum EW $\lambda$ mit $|\lambda|=\varrho(A)$,
dann ist $\lambda=\varrho(A)$
\end{block}}
\uncover<46->{%
\begin{block}{Satz}
Der \only<57->{verallgemeinerte }Eigenraum zu EW $\varrho(A)$
ist eindimensional
\end{block}
}
\end{column}
\ifthenelse{\boolean{presentation}}{
\only<-6>{
\begin{column}{0.48\textwidth}
\begin{proof}[Beweis]
\begin{itemize}
\item<3->
Vergleich: $Au>0$
\item<4->
$Au=\lambda u > 0$
\item<5->
$\lambda >0$ und $u>0$
\end{itemize}
\end{proof}
\end{column}}
\only<7-20>{
\begin{column}{0.48\textwidth}
\begin{proof}[Beweis]
\begin{align*}
(Au)_i
&\only<-8>{=
\sum_j a_{ij}u_j}
\only<8-9>{=
\sum_j |a_{ij}v_j|}
\only<9->{\ge}
\only<9-10>{
\biggl|\sum_j a_{ij}v_j\biggr|}
\only<10>{=}
\only<10-11>{
|(Av)_i|}
\only<11>{=}
\only<11-12>{
|\lambda v_i|}
\only<12>{=}
\only<12-13>{
\varrho(A) |v_i|}
\only<13>{=}
\uncover<13->{
\varrho(A) u_i}
\hspace*{5cm}
\\
\uncover<14->{Au&\ge \varrho(A)u}
\intertext{\uncover<15->{Vergleich}}
\uncover<16->{A^2u&> \varrho(A)Au}
\intertext{\uncover<17->{Trennung: $\exists \vartheta >1$ mit}}
\uncover<18->{A^2u&\ge \vartheta \varrho(A) Au }\\
\uncover<19->{A^3u&\ge (\vartheta \varrho(A))^2 Au }\\
\uncover<20->{A^ku&\ge (\vartheta \varrho(A))^{k-1} Au }\\
\end{align*}
\end{proof}
\end{column}}
\only<21-29>{%
\begin{column}{0.48\textwidth}
\begin{proof}[Beweis, Fortsetzung]
Abschätzung der Operatornorm:
\begin{align*}
\|A^k\|\, |Au|
\ge
\|A^{k+1}u\|
\uncover<22->{
\ge
(\vartheta\varrho(A))^k |Au|}
\end{align*}
\uncover<23->{Abschätzung des Spektralradius}
\begin{align*}
\uncover<24->{\|A^k\| &\ge (\vartheta\varrho(A))^k}
\\
\uncover<25->{\|A^k\|^{\frac1k} &\ge \vartheta \varrho(A)}
\\
\uncover<26->{\lim_{k\to\infty}\|A^k\|^{\frac1k} &\ge \vartheta \varrho(A)}
\\
\uncover<27->{\varrho(A) &\ge \underbrace{\vartheta}_{>1} \varrho(A)}
\end{align*}
\uncover<28->{Widerspruch: $u=v$}
\end{proof}
\end{column}}
\only<30-46>{
\begin{column}{0.48\textwidth}
\begin{proof}[Beweis]
$u$ ist EV mit EW $\varrho(A)$:
\[
Au=\varrho(A)u
\uncover<31->{\Rightarrow
\sum_j a_{ij}|v_j| = {\color<38->{red}\varrho(A) |v_i|}}
\]
\uncover<33->{Andererseits: $Av=\lambda v$}
\[
\uncover<34->{\sum_{j}a_{ij}v_j=\lambda v_i}
\]
\uncover<35->{Betrag}
\begin{align*}
\uncover<36->{\biggl|\sum_j a_{ij}v_j\biggr|
&=
|\lambda v_i|}
\uncover<37->{=
{\color<38->{red}\varrho(A) |v_i|}}
\uncover<39->{=
\sum_j a_{ij}|v_j|}
\end{align*}
\uncover<40->{Dreiecksungleichung: $v_j=|v_j|c, c\in\mathbb{C}$}
\[
\uncover<41->{\lambda v = Av}
\uncover<42->{= Acu}
\uncover<43->{= c\varrho(A) u}
\uncover<44->{= \varrho(A)v}
\]
\uncover<45->{$\Rightarrow
\lambda=\varrho(A)
$}
\end{proof}
\end{column}}
\only<47-57>{
\begin{column}{0.48\textwidth}
\begin{proof}[Beweis]
\begin{itemize}
\item<48-> $u>0$ ein EV zum EW $\varrho(A)$
\item<49-> $v$ ein weiterer EV, man darf $v\in\mathbb{R}^n$ annehmen
\item<50-> Da $u>0$ gibt es $c>0$ mit $u\ge cv$ aber $u\not > cv$
\item<51-> $u-cv\ge 0$ aber $u-cv\not > 0$
\item<52-> $A$ anwenden:
\[
\begin{array}{ccc}
\uncover<53->{A(u-cv)}&\uncover<54->{>&0}
\\
\uncover<53->{\|}&&
\\
\uncover<53->{\varrho(A)(u-cv)}&\uncover<55->{\not>&0}
\end{array}
\]
\uncover<56->{Widerspruch: $v$ existiert nicht}
\end{itemize}
\end{proof}
\end{column}}
\only<58->{
\begin{column}{0.48\textwidth}
\begin{proof}[Beweis]
\begin{itemize}
\item<59-> $Au=\varrho(A)u$ und $A^tp^t=\varrho(A)p^t$
\item<60-> $u>0$ und $p>0$ $\Rightarrow$ $up>0$
\item<61-> $px=0$, dann ist
\[
\uncover<62->{pAx}
\only<62-63>{=
(A^tp^t)^t x}
\only<63-64>{=
\varrho(A) (p^t)^t x}
\uncover<64->{=
\varrho(A) px}
\uncover<65->{= 0}
\]
\uncover<66->{also ist $\{x\in\mathbb{R}^n\;|\; px=0\}$
invariant}
\item<67-> Annahme: $v\in \mathcal{E}_{\varrho(A)}$
\item<68-> Dann muss es einen EV zum EW $\varrho(A)$ in 
$\mathcal{E}_{\varrho(A)}$ geben
\item<69-> Widerspruch: der Eigenraum ist eindimensional
\end{itemize}
\end{proof}
\end{column}}
}{
\begin{column}{0.48\textwidth}
\begin{block}{}
\usebeamercolor[fg]{title}
Beweise: Buch Abschnitt 9.3 
\end{block}
\end{column}
}
\end{columns}
\end{frame}
\egroup
