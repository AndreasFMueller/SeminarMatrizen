%
% slides.tex -- XXX
%
% (c) 2017 Prof Dr Andreas Müller, Hochschule Rapperswil
%

% Wie findet man die Lösung von \dot x = Ax?
% Fall \dot x = ax
% Potenzreihenansatz -> exp(ax) x_0

%% Plan:
% 1. Tailor-Reihen p_n -> f
% 2. x' = ax => x = exp(ax) x_0 via Potenzreihe finden
% 3. n-Dim-skalar -> 1-Dim-Matrix
% 4. Analogie zur Vektor-Matrix-Form
% 5. exp(Ax) x_0 als Fluss
% 6. Strömungslinien = Pfade für Lie-Theorie, A lokal, exp(Ax) global
% 7. Beispiele so(2), Jordan-Block, vielleicht [0 1; 1 0]

\section{Einführung}
\folie{10/intro.tex}
\section{Woher kommt $\exp(At)$?}
\subsection{Taylor-Reihen}
\folie{10/taylor.tex}
\folie{10/potenzreihenmethode.tex}
\subsection{Ableitung von $\exp(At)$}
\folie{10/ableitung-exp.tex}
\section{Lösen einer Matrix-DGL}
\folie{10/n-zu-1.tex}
\folie{10/matrix-dgl.tex}
\section{Lie-Gruppen und -Algebren}
\folie{10/repetition.tex}
\folie{10/so2.tex}
\section{Was bedeutet $\exp(At)$?}
\folie{10/vektorfelder.tex}
