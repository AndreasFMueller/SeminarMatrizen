%
% slides.tex -- XXX
%
% (c) 2017 Prof Dr Andreas Müller, Hochschule Rapperswil
%
\section{Matrixrelationen}
\folie{3/maximalergrad.tex}
\folie{3/minimalbeispiel.tex}
\folie{3/minimalpolynom.tex}

\section{Eigenwertproblem}
% XXX Motivation: beliebige Funktionen f(A) berechnen
\folie{5/motivation.tex}
\folie{5/charpoly.tex}

\section{Invariante Unterräume}
\folie{5/kernbild.tex}
\folie{5/ketten.tex}
\folie{5/dimension.tex}
\folie{5/nilpotent.tex}
% XXX \folie{5/eigenraeume.tex}

% Jordan Normalform
\section{Jordan-Normalform}
% XXX Diagonalform
% XXX \folie{5/diagonalform.tex}
% XXX \folie{5/jordannormalform.tex}
% XXX \folie{5/minimalpolynom.tex}
% XXX \folie{5/reellenormalform.tex}
% XXX \folie{5/hessenberg.tex}

\section{Satz von Cayley-Hamilton}
% XXX \folie{5/cayleyhamilton.tex}

